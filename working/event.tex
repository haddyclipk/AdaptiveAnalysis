%
%
\subsection{Event}
%
% \todo{
% \[
% \begin{array}{llll}
% \mbox{Post-Processed Event} & \event & ::= & 
%       ({x}, l, n, v) | ({x}, l, n, \qval, v)  ~~~~~~~ \mbox{Post-Processed Assignment Event} \\
% &&& | (\bexpr, l, n, v)   ~~~~~~~~~~~~~~~~~~~~~~~~~~~~~~~~~ \mbox{Post-Processed Testing Event}
% \end{array}
% \]
% %
% Informally, $n$ marks the number of times this label appears in a trace including itself. Informally, the n-th appearance of this label.
% \\
% Below we use the same notation $\eventset$, $\eventset^{\asn}$ and $\eventset^{\test} $ represent the 
% set of post-processed events, assignment events and testing events.
% \[
% \begin{array}{lll}
% %
% \eventset  & : & \mbox{Set of Post-Processed Events}  
% \\
% %
% \eventset^{\asn}  & : & \mbox{Set of Post-Processed Assignment Events}  
% \\
% %
% \eventset^{\test}  & : & \mbox{Set of Post-Processed Testing Events}  
% \end{array}
% \]
% %
% Event Processor : $ \circ : \mbox{Event} \to \mbox{Post-Processed Events}$
% \[
% \begin{array}{lll}
% (x, l, v) \circ n \triangleq (x, l, n, v)
% &
% (x, l, \qval, v) \circ n \triangleq (x, l, n, \qval, v)
% &
% (b, l, v) \circ n  \triangleq (b, l, n, v)
% \end{array}
% \]
% }
% %
%
% Given a trace $\ostrace$ and $\osevent$, the corresponding well-formed event $\event$ 
% \[
%   \event \triangleq 
%   \left\{
%   \begin{array}{ll}
%   (x, l, \vcounter(\ostrace) l, v) & \osevent = (x, l, v)\\
%   (x, l, \vcounter(\ostrace) l, v, \qval) & \osevent = (x, l, v)\\
%   (x, l, \vcounter(\ostrace) l, v) & \osevent = (x, l, v)\\
%   \end{array}
%   \right.
% \]
%
% \\
% Event Value : $\pi_v : \eventset \to \mathcal{VAL}$
% \[
% \begin{array}{lll}
% \pi_v((x, l, n, v)) \triangleq v
% &
% \pi_v (x, l, n, \qval, v) \triangleq v
% &
% \pi_v (b, l, n, v)  \triangleq v
% \end{array}
% \]
% %
% %
% Event Distinguishable Value : $\pi_{q} : \eventset \to \mathcal{QVAL}$ 
% \[
% \begin{array}{lll}
% \pi_{q} (x, l, n, v) \triangleq v
% &
% \pi_{q} (x, l, n, \qval, v) \triangleq \qval
% &
% \pi_{q} (b, l, n, v)  \triangleq v
% \end{array}
% \]%
% 
Event projection operators $\pi_i$ projects the $i$th element from an event: 
\\
$\pi_i : 
\eventset \to \mathcal{VAR}\cup \mbox{Boolean Expression}  \cup \mathbb{N} \cup \mathcal{VAL} \cup \mathcal{QVAL} $ 
% \wqside{use b for Boolean expression?}
\\
% $\pi_{(i,j)} (\event) \triangleq (\pi_i(\event), \pi_j(\event)) $
% %
% \\
% Event Signature : $\pi_{\sig} : \eventset \to (\mathcal{VAR}\cup \mbox{Boolean Expression}) \times\mathbb{N}\times \mathbb{N}$
% \[
% \begin{array}{lll}
% \pi_{\sig} (x, l, n, v) \triangleq (x, l, n)
% &
% \pi_{\sig} (x, l, n, \qval, v) \triangleq (x, l, n, \query)
% &
% \pi_{\sig} (b, l, n, v)  \triangleq (b, l, n)
% \end{array}
% \]
%
%
Free Variables: $FV: \expr \to \mathcal{P}(\mathcal{VAR})$, the set of free variables in an expression.
\\
$FV(\qexpr)$ is the set of free variables in the query expression $\qexpr$.
\begin{defn}[Equivalence of Query Expression]
%
\label{def:query_equal}
% \mg{Two} \sout{2} 
Two query expressions $\qexpr_1$, $\qexpr_2$ are equivalent, denoted as $\qexpr_1 =_{q} \qexpr_2$, if and only if
% $$
%  \begin{array}{l} 
%   \exists \qval_1, \qval_2 \in \mathcal{QVAL} \st \forall \trace \in \mathcal{T} \st
%     (\config{\trace,  \qexpr_1} \qarrow \qval_1 \land \config{\trace,  \qexpr_2 } \qarrow \qval_2) 
%     \\
%     \quad \land (\forall D \in \dbdom, r \in D \st 
%     \exists v \in \mathcal{VAL} \st 
%           \config{\trace, \qval_1[r/\chi]} \aarrow v \land \config{\trace,  \qval_2[r/\chi] } \aarrow v)  
%   \end{array}.
% $$
$$
 \begin{array}{l} 
   \forall \trace \in \mathcal{T} \st \exists \qval_1, \qval_2 \in \mathcal{QVAL} \st
    (\config{\trace,  \qexpr_1} \qarrow \qval_1 \land \config{\trace,  \qexpr_2 } \qarrow \qval_2) 
    \\
    \quad \land (\forall D \in \dbdom, r \in D \st 
    \exists v \in \mathcal{VAL} \st 
          \config{\trace, \qval_1[r/\chi]} \aarrow v \land \config{\trace,  \qval_2[r/\chi] } \aarrow v)  
  \end{array}.
$$
% \mg{$$
%  \begin{array}{l} 
%    \forall \trace \in \mathcal{T} \st \exists \qval_1, \qval_2 \in \mathcal{QVAL} \st
%     (\config{\trace,  \qexpr_1} \qarrow \qval_1 \land \config{\trace,  \qexpr_2 } \qarrow \qval_2) 
%     \\
%     \quad \land (\forall D \in \dbdom, r \in D \st 
%     \exists v \in \mathcal{VAL} \st 
%           \config{\trace, \qval_1[r/\chi]} \aarrow v \land \config{\trace,  \qval_2[r/\chi] } \aarrow v)  
%   \end{array}.
% $$
% }
 %
 where $r \in D$ is a record in the database domain $D$. 
 As usual, we will denote by $\qexpr_1 \neq_{q} \qexpr_2$  the negation of the equivalence.
% \\ 
% where $r \in D$ is a record in the database domain $D$,
% \mg{is  $FV(\qexpr)$ being defined here? If yes, I suggest to put it in a different place, rather than in the middle of another definition.} 
% $FV(\qexpr)$ is the set of free variables in the query expression $\qexpr$.
% \sout{$\qexpr_1 \neq_{q}^{\trace} \qexpr_2$  is defined vice versa.}
% \mg{As usual, we will denote by $\qexpr_1 \neq_{q}^{\trace} \qexpr_2$  the negation of the equivalence.}
%
\end{defn}
%
% \mg{In the next definition you don’t need the subscript e, it is clear that it is equivalence of events by the fact that the elements on the two sides of = are events. That is also true for query expressions. Also, I am confused by this definition. What happen for two query events?}
% \\
% \jl{The last component of the event is equal based on Query equivalence, $\pi_{4}(\event_1) =_q \pi_{4}(\event_2)$.
% In the previous version, the query expression is in the third component and I defined $v \neq \qexpr$ for all $v$ that isn't a query value.}
% \begin{defn}[Event Equivalence $\eventeq$]
% Two events $\event_1, \event_2 \in \eventset$ \mg{are equivalent, \sout{is in \emph{Equivalence} relation,}} denoted as $\event_1 \eventeq \event_2$ if and only if:
% \[
% \pi_1(\event_1) = \pi_1(\event_2) 
% \land  
% \pi_2(\event_1) = \pi_2(\event_2) 
% \land
% \pi_{3}(\event_1) = \pi_{3}(\event_2)
% \land 
% \pi_{4}(\event_1) =_q \pi_{4}(\event_2)
% \]
% %
% % \sout{The $\event_1 \eventneq \event_2$ is defined as vice versa.}
% % \mg{As usual, we will denote by $\event_1 \eventneq \event_2$  the negation of the equivalence.}
% \end{defn}
\begin{defn}[Event Equivalence]
  Two events $\event_1, \event_2 \in \eventset$ are equivalent, 
  % denoted as $\event_1 \eventeq \event_2$ 
  denoted as $\event_1 = \event_2$ 
  if and only if:
  \[
  \pi_1(\event_1) = \pi_1(\event_2) 
  \land  
  \pi_2(\event_1) = \pi_2(\event_2) 
  \land
  \pi_{3}(\event_1) = \pi_{3}(\event_2)
  \land 
  \pi_{4}(\event_1) =_q \pi_{4}(\event_2)
  \]
  %
  As usual, we will denote by $\event_1 \neq \event_2$  the negation of the equivalence.
  % As usual, we will denote by $\event_1 \eventneq \event_2$  the negation of the equivalence.
  % When it is clear from the context, we omit the subscript $\kw{e}$ and use 
  % $\event_1 = \event_2$ (and $\event_1 \neq \event_2$) for event equivalent
\end{defn}
%
%
% \begin{defn}[Signature Equivalence of Events $\sigeq$]

% Two events $\event_1, \event_2 \in \eventset$ is in \emph{signature equivalence} relation, denoted as $\event_1 \sigeq \event_2$ if and only if:
% \[
% \forall i \in \{1, 2, 3\} \st \pi_{\sig}(\event_1) = \pi_{\sig}(\event_2) 
% \]
% The $\event_1 \signeq \event_2$ is defined as vice versa.
% \end{defn}
%
% \begin{defn}[Events Different up to Value ($\diff$)]
% Two events $\event_1, \event_2 \in \eventset$ \mg{are \sout{is}} \emph{Different up to Value}, 
% denoted as $\diff(\event_1, \event_2)$ if and only if:
% \[
% \pi_1(\event_1) = \pi_1(\event_2) 
% \land  
% \pi_2(\event_1) = \pi_2(\event_2) 
% \land  
% \pi_3(\event_1) \neq_q \pi_3(\event_2)
% \]
% \end{defn}
\begin{defn}[Events Different up to Value ($\diff$)]
  Two events $\event_1, \event_2 \in \eventset are $ \emph{Different up to Value}, 
  denoted as $\diff(\event_1, \event_2)$ if and only if:
  \[
    \begin{array}{l}
  \pi_1(\event_1) = \pi_1(\event_2) 
  \land  
  \pi_2(\event_1) = \pi_2(\event_2) \\
  \land  
  \big(
    (\pi_3(\event_1) \neq \pi_3(\event_2)
  \land 
  \pi_{4}(\event_1) = \pi_{4}(\event_2) = \bullet )
  % \qquad \qquad 
  \lor 
  (\pi_4(\event_1) \neq \bullet
  \land 
  \pi_4(\event_2) \neq \bullet
  \land 
  \pi_{4}(\event_1) \neq_q \pi_{4}(\event_2)) 
  \big)
  \end{array}
  \]
  \end{defn}
  %
% \begin{defn}[Order of Events].
% \label{def:query_dir}
% \\
% An event $\event_1 \in \eventset$, where $\pi_{\{1, 2, 3\}}(\event_1) = (x_1, l_1, n_1)$,
% has smaller order than another event $\event_2 \in \eventset$, 
% where 
% $ \pi_{\{1, 2, 3\}}(\event_2) = (x_2, l_2, n_2)$, defined as follows:
% %
% \[
% \event_1 \eventlt \event_2
% \triangleq 
% \left\{
% \begin{array}{ll}
%   l_1 < l_2 & n_1 = n_2
%   \\
%   n_1 < n_2  & o.w.
% \end{array}  
% \right.
% \]
% %
% $\event_1 \eventgeq \event_2$  is defined vice versa.
% \end{defn}
%
%
