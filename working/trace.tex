%
\subsection{Trace}
%
% An event $\event \in \eventset$ belongs to a trace $\trace$, i.e., $\event \eventin \trace$ are defined as follows:
% %
% \begin{equation}
%   \event \eventin \trace  
%   \triangleq \left\{
%   \begin{array}{ll} 
%     \etrue                  & \trace =  (\trace' \cdot \event') \land (\event \eventeq \event') \\
%     \event \eventin \trace' & \trace =  (\trace' \cdot \event') \land (\event \eventneq \event') \\ 
%     \efalse                 & o.w.
%   \end{array}
%   \right.
% \end{equation}
% %
% A well-formed event $\event \in \eventset$ belongs to a trace $\trace$ in signature, 
% i.e., $\event \sigin \trace$ are defined as follows:
%   %
% \begin{equation}
%   \event \sigin \trace  
%   \triangleq \left\{
%   \begin{array}{ll} 
%     \etrue                  & \trace =  (\trace' \cdot \event') \land (\event \sigeq \event') \\
%     \event \sigin \trace'   & \trace =  (\trace' \cdot \event') \land (\event \signeq \event') \\ 
%     \efalse                 & o.w.
%   \end{array}
%   \right.
% \end{equation}
%
%
%
%
% \todo{
% \[
% \mbox{Post-Processed Trace} \qquad \trace \qquad ::= \qquad \cdot | \trace \cdot \event
% % %
% \]
% Trace appending: $\cdot: \mathcal{T} \to \eventset \to \mathcal{T}$
% \[
%   \trace \cdot \event \triangleq
%   \left\{
%   \begin{array}{ll} 
%     {} \cdot \event           & \trace =  \cdot \\
%     \trace \cdot \event    & \trace =  \trace' \cdot \event\\ 
%   \end{array}
%   \right.
% \]
%
Trace concatenation: $\tracecat: \mathcal{T} \to \mathcal{T} \to \mathcal{T}$
\[
  \trace_1 \tracecat \trace_2 \triangleq
  \left\{
  \begin{array}{ll} 
    \trace_1                                      & \trace_2 =  \cdot \\
    (\trace_1 \cdot \event) \tracecat \trace_2'   & \trace_2 =  {} \cdot \event \cdot \trace_2'\\ 
  \end{array}
  \right.
\]
%
\todo{ need to consider the occurrence times }
Trace subtraction: $[ : ] : \mathcal{{T} \to \eventset \to \eventset \to \mathcal{T}}$ \wqside{Confusing, I can not understand the subtraction, it takes a trace, and two events, and this operator is used to subtract these two events?}
\[
  \trace[\event_1 : \event_2] \triangleq
  \left\{
  \begin{array}{ll} 
  \trace'[\event_1: \event_2]             & \trace = \event \tracecat \trace' \land \event \eventneq \event_1 \\
  \cdot {} \event_1 \tracecat \trace'[:\event_2]  & \trace = \event \tracecat \trace' \land \event \eventeq \event_1 \\
  \cdot {} & \trace = \cdot \\
  \end{array}
  \right.
\]
\[
  \trace[:\event] \triangleq
  \left\{
  \begin{array}{ll} 
  \cdot {} \event' \tracecat \trace'[: \event]             & \trace = \event' \tracecat \trace' \land \event' \eventneq \event \\
  \cdot {} \event'  & \trace = \event' \tracecat \trace' \land \event' \eventeq \event \\
  \cdot  & \trace = \cdot \\
  \end{array}
  \right.
\]
\[
  \trace[\event_1 : ] \triangleq
  \left\{
  \begin{array}{ll} 
  \trace'[\event_1: ] \cdot \event'     & \trace = \trace'\cdot \event' \land \event \eventneq \event' \\
  \cdot {}\event'  & \trace = \trace' \cdot \event' \land \event \eventeq \event' \\
  \cdot  & \trace = \cdot
  \end{array}
  \right.
\]
%
An event $\event \in \eventset$ belongs to a trace $\trace$, i.e., $\event \eventin \trace$ are defined as follows:
%
\begin{equation}
  \event \eventin \trace  
  \triangleq \left\{
  \begin{array}{ll} 
    \etrue                  & \trace =  (\trace' \cdot \event') \land (\event \eventeq \event')
                              \\
    \event \eventin \trace' & \trace =  (\trace' \cdot \event') \land (\event \eventneq \event') \\ 
    \efalse                 & o.w.
  \end{array}
  \right.
\end{equation}
%
% An event $\event \in \eventset$ belongs to a trace $\trace$ up to value, 
% i.e., $\event \sigin \trace$ are defined as follows:
%   %
% \begin{equation}
%   \event \sigin \trace  
%   \triangleq \left\{
%   \begin{array}{ll} 
%     \etrue                  & \trace =  (\trace' \cdot \event')                          \land \pi_1(\event_1) = \pi_1(\event_2) 
%                               \land  \pi_2(\event_1) = \pi_2(\event_2)  
%                               % \land \vcounter(\trace \event) = \vcounter()
%                               \\
%     \event \sigin \trace'   & \trace =  (\trace' \cdot \event') 
%                               \land 
%                               (\pi_1(\event_1) \neq \pi_1(\event_2) 
%                               \lor  \pi_2(\event_1) \neq \pi_2(\event_2)) 
%                               \\ 
%     \efalse                 & o.w.
%   \end{array}
%   \right.
% \end{equation}
%
Counter $\vcounter : \mathcal{T} \to \mathbb{N} \to \mathbb{N}$ \wq{The operator counter actually provides the number of times a specific label appears in a trace. Only a number, the position of label is ignored.}
\[
\begin{array}{lll}
\vcounter(\trace \cdot (x, l, v) ) l \triangleq \vcounter(\trace) l + 1
&
\vcounter(\trace \cdot (b, l, v)) l \triangleq \vcounter(\trace) l + 1
&
\vcounter(\trace \cdot (x, l, \qval, v)) l \triangleq \vcounter(\trace) l + 1
\\
\vcounter(\trace  \cdot (x, l', v)) l \triangleq \vcounter(\trace ) l
&
\vcounter(\trace \cdot (b, l', v)) l \triangleq \vcounter(\trace ) l
&
\vcounter(\trace \cdot (x, l', \qval, v)) l \triangleq \vcounter(\trace ) l
\\
\vcounter(\cdot) l \triangleq 0
&&
\end{array}
\]
%
Latest Label $\llabel : \mathcal{T} \to \mathcal{VAR} \to \mathbb{N}$ 
The label of the lastest assignment event which assigns value to variable $x$.
\[
  \begin{array}{lll}
\llabel(\trace \cdot (x, l, v) ) x \triangleq l
&
\llabel(\trace \cdot (b, l, v)) x \triangleq \llabel(\trace) x
&
\llabel(\trace \cdot (x, l, \qval, v)) x \triangleq l
\\
\llabel(\trace  \cdot (y, l, v)) x \triangleq \llabel(\trace ) x
&
\llabel(\trace \cdot (y, l, \qval, v)) x \triangleq \llabel(\trace ) x
\\
\llabel(\cdot) x \triangleq \bot
&&
\end{array}
\]
%
% Given a trace $\trace$, its -processed trace $\trace$ is computed by a function $p : \trace \to \trace$ as follows:
% \[
%   \trace \triangleq
%   \left\{
%   \begin{array}{ll} 
%   p(\trace' \cdot (x, l, v)) & = p(\trace') \cdot (x, l, \vcounter(\trace') l + 1, v) \\
%   p(\trace' \cdot (b, l, v)) & = p(\trace') \cdot (b, l, \vcounter(\trace') l + 1, v) \\
%   p(\trace' \cdot (x, l, \qval, v)) & = p(\trace') \cdot (x, l, \vcounter(\trace') l + 1, \qval, v) \\
%   p(\cdot) & = \cdot
%   \end{array}
%   \right.
% \]
%

%
% $\mathcal{T}$ : Set of Well-formed Traces (in Definition~\ref{def:wf_trace})
%
%
% \\
%
\todo{Need to Consider whether useful or not}
\begin{defn}[Well-formed Trace $\mathcal{T}$]
\label{def:wf_trace}
A -processed trace $\trace$ is well formed, i.e., $\trace \in \mathcal{T}$ if and only if it preserves the following two properties:
\begin{itemize}
\item{\emph{(Uniqueness)}} 
$\forall \event_1, \event_2 \eventin \trace \st \event_1 \signeq \event_2$
%
\item{\emph{(Ordering)}} $\forall \event_1, \event_2 \eventin \trace \st 
(\event_1 \eventlt \event_2) \Longleftrightarrow
\exists \trace_1, \trace_2, \trace_3 \in \mathbb{T},
 \event_1', \event_2' \in \eventset \st
(\event_1 \eventeq \event_1') \land (\event_2 \eventeq \event_2')
\land \trace_1 \cdot \event_1' \cdot \trace_2 \cdot \event_2' \cdot \trace_3 = \trace$
\end{itemize}
\end{defn}
%
%
\begin{thm}[Trace Generated from Operational Semantics after Post Processing is Well-formed $c \vDash \trace$].
\label{thm:os_wf_trace}
\\
\[
\forall \trace \in \mathcal{T}, c \st
\config{c, \vtrace} \to^{*} \config{\eskip, \trace \cdot \trace'} \land \trace' = p(\trace')
\implies
\vtrace' \in \mathcal{T}
\] \wqside{ is $p(\tau)$ defined some where? I guess it means post-processing.}
%
\end{thm}
\begin{proof}
Proof in File: {\tt ``thm\_os\_wf\_trace.tex''}.
% {
By induction on the program $c$, we have following cases:
%
\caseL{ $[\assign x \expr]^{l}$}
%
We assume a configuration $\config{m,[\assign x \expr]^{l} , t, w}$ s.t. .
\\
From the evaluation rule \rname{assn1} and rule \rname{assn2}, we know there exists an evaluation as follows:
%
\[
	\config{m, [\assign x \expr]^{l}, t, w} \to \config{m', \eskip, t, w}
\]
%
when we assume $\config{m,e} \to^{*} v$. 
%
\\
The resulting trace is $t$. 
\\
By unfolding the trace subtraction operation, we know $t - t = []$ is an empty list, which is well formed according to Definition~\ref{def:wf_trace}.
}
\caseL{$[\assign{x}{\query(\qexpr)}]^{l}$}
By repeatedly applying the operational semantics rule \rname{query-e}, we know there exist $\qval$ s.t.,:
%
\[
\config{m, [\assign{x}{\query(\qexpr)}]^{l}, t, w} \to^{*} \config{m', [\assign{x}{\query(\qval)}]^{l}, t, w}
\]
%
According to the evaluation rule \rname{query-v}, we have the following transition:
\[
	\config{m', [\assign{x}{\query(\qval)}]^{l}, t, w} \to \config{m', \eskip, t ++ [(\qval, l, w)], w}
\]
%
The resulting trace is $t ++ [(\qval, l, w)]$.
\\
By unfolding the trace subtraction operation, we know $(t ++ [(\qval, l, w)]) - t = [(\qval, l, w)]$,
which is well formed by Definition~\ref{def:wf_trace}.
%%
\caseL{$\eif([b]^l, c_1, c_2)$}
Let $\config{m', \eskip, t, w} ~ (\star)$ be the configuration 
s.t. $\config{m, \eif([b]^l, c_1, c_2), t, w} \to^{*} \config{m', \eskip, t', w'}$.
It is sufficient to show that $t' - t$ is well-formed.
\\
According to the evaluation rule \rname{if}, we know there exists an evaluation by repeatedly applying \rname{if} 
%
\[
\config{m, \eif([b]^l, c_1, c_2), t, w} \to^{*} \config{m, \eif([v]^l, c_1, c_2), t, w} ~ (1)
\]
%
, where $\config{m, b} \barrow v$ and $v$ is either $\etrue$ or $\efalse$.
\\
Considering 2 subcases where $v$ is either $\etrue$ or $\efalse$:
%
\subcaseL{$v = \etrue$}
%
By applying the evaluation rule \rname{if-t}, we have:
\[
\config{m, \eif([\etrue]^l, c_1, c_2), t, w} \to \config{m, c_1, t, w} ~ (t1)
\]
%
By induction hypothesis on $c_1$ with starting memory $m$, trace $t$ and while map $w$, 
let $\config{m_1, \eskip, t_1, w_1}$ be the configuration s.t.
\[
	\config{m, c_1, t, w} \to^{*} \config{m_1, \eskip, t_1, w_1} ~(t2)
\]
we know $(t_1 - t)$ is a well formed trace.
\\
According to the evaluation $(a)$, $(t1)$ and $(t2)$, we know $\config{m_1, \eskip, t_1, w_1}$ 
is the assumed configuration $\config{m', \eskip, t', w'}$ such that
\[
	\config{m, \eif([b]^l, c_1, c_2), t, w} \to^{*} \config{m_1, \eskip, t_1, w_1}
\]
%
Since $(t_1 - t)$ is well-formed and $t' - t = t_1 - t$, this subcase is proved.
%
\subcaseL{$v = \efalse$}
%
By applying the evaluation rule \rname{if-f}, we have:
\[
\config{m, \eif([\efalse]^l, c_1, c_2), t, w} \to \config{m, c_2, t, w} ~ (f1)
\]
%
By induction hypothesis on $c_1$ with starting memory $m$, trace $t$ and while map $w$, 
let $\config{m_2, \eskip, t_2, w_2}$ be the configuration s.t.
\[
	\config{m, c_2, t, w} \to^{*} \config{m_2, \eskip, t_2, w_2} ~(f2)
\]
we know $(t_2 - t)$ is a well formed trace.
\\
According to the evaluation $(a)$, $(f1)$ and $(f2)$, we know $\config{m_2, \eskip, t_2, w_2}$ 
is the assumed configuration $\config{m', \eskip, t', w'}$ in $(\star)$ such that
\[
	\config{m, \eif([b]^l, c_1, c_2), t, w} \to^{*} \config{m_2, \eskip, t_2, w_2}
\]
%
Since $(t_2 - t)$ is well-formed and $t' = t_2$, this subcase is proved.
%
\caseL{$c_1; c_2$}
%
According to the evaluation rule \rname{seq1}, we have the following evaluation by repeatedly applying \rname{seq1} 
%
\[
\config{m, c_1;c_2, t, w} \to^{*} \config{m_1, [\eskip]^l;c_2,  t_1, w_1} ~ (1)
\]
%
where 
%
\[
	\config{m, c_1, t, w} \to^{*} \config{m_1, [\eskip]^l, t_1, w_1]} ~ (2)
\]
%
%
By induction hypothesis on $c_1$ and $(2)$, we know $(t_1 - t)$ is well-formed.
%
According to the evaluation rule \rname{seq2}, we have:
%
\[
\config{m_1, [\eskip]^l;c_2,  t_1, w_1} \to \config{m_1, c_2, t_1, w_1} ~ (3)
\]
%   
%
By induction hypothesis on $c_2$ and let $\config{m_2, \eskip, t_2, w_2]}$ be the configuration s.t.,:
%
\[
	\config{m_1, c_2, t_1, w_1} \to^{*} \config{m_2, \eskip, t_2, w_2]} ~ (4)
\]
%
we know $(t_2 - t_1)$ is well-formed.
%
\\
According to evaluations $(1), (2), (3)$ and $(4)$, we have following evaluation:
%
\[
	\config{m, c_1;c_2, t, w} \to^{*} \config{m_2, \eskip, t_2, w_2]}
\]
%
It is sufficient to show $(t_2 - t)$ is well-formed.
%
Unfolding the definition~\ref{def:wf_trace}, it is sufficient to show that the trace $(t_2 - t)$ holds both the \emph{Uniqueness} and \emph{Ordering} property.
\\
Let $t_1' = t_1 - t$ and $t_2' = t_2 - t_1$, unfolding the subtraction operations, we have:
\[
	(t_2 - t) = t_1' ++ t_2'
\]
%
%
\caseL{Uniqueness $(\diamond)$} 
By the distinction properties of labels in program $c$, 
\[
	\forall \aq_1 \in t_1', \aq_2 \in t_2'. ~ \aq_1 \aqneq \aq_2
\] 
%
Since $t = t_1' ++ t_2'$ and both $t_1'$ and $t_2'$ are well-formed, we know:
\[
	\forall \aq_1, \aq_2 \in (t_1' ++ t_2'). ~ \aq_1 \aqneq \aq_2
\]
%
THe Uniqueness property for $(t_2 - t)$ is proved.
%
\caseL{Ordering $(\square)$}
%
By the definition of ordering property, we need to show $\forall \aq_1, \aq_2 \aqin (t_2 - t)$:
\[
	(\aq_1 <_{aq} \aq_2) \Longleftrightarrow
	\exists t^1, t^2, t^3, \aq_1', \aq_2'. ~ s.t.,~ 
	(\aq_1 \aqeq \aq_1') \land (\aq_2 \aqeq \aq_2') \land t^1 ++ [\aq_1'] ++ t^2 ++ [\aq_2'] ++ t^3 = t
\]
%
Since $(t_2 - t) = t_1' ++ t_2'$, there are 4 cases need to be proved:
\\
$(1) ~ \aq_1, \aq_2 \aqin t_1'$
\\
$(2) ~ \aq_2, \aq_2 \aqin t_2'$
\\
$(3) ~ \aq_2 \aqin t_1', \aq_1 \aqin t_2'$ 
\\
$(4) ~ \aq_1 \aqin t_1', \aq_2 \aqin t_2'$
\\
Given both $t_1'$ and $t_2'$ are well-formed, we know this property is true in case $(1)$ and $(2)$.
\\
By the ordering properties of labels in program $c_1; c_2$,
we know all the labels in $c_2$ are greater than the labels in $c_1$, i.e.,
\[
	\forall \aq_1 \aqin t_1', \aq_2 \aqin t_2'. ~ \aq_1 <_{aq} \aq_2 ~ (a)
\]
%
Then we have the case $(3)$ proved because the premise is $\efalse$ in this case.
%
To prove the ordering property in case $(4)$, it is sufficient to show the following 2 implications:
\[
	(\aq_1 <_{aq} \aq_2) 
	\Longrightarrow
	\exists t^1, t^2, t^3, \aq_1', \aq_2'. ~ s.t.,~ 
	(\aq_1 \aqeq \aq_1') \land (\aq_2 \aqeq \aq_2') \land t^1 ++ [\aq_1'] ++ t^2 ++ [\aq_2'] ++ t^3 = t
\]
%
\[
	% \forall \aq_1, \aq_2 \aqin t. ~
	\exists t^1, t^2, t^3, \aq_1', \aq_2'. ~ s.t.,~ 
	(\aq_1 \aqeq \aq_1') \land (\aq_2 \aqeq \aq_2') \land t^1 ++ [\aq_1'] ++ t^2 ++ [\aq_2'] ++ t^3 = t
	 \Longrightarrow
	(\aq_1 <_{aq} \aq_2)
\]
%
By Corollary~\ref{coro:aqintrace} and the hypothesis $(4)$, we know:
%
\[
	\exists t_{11}, t_{12}, \aq_1'. ~ (\aq_1 \aqeq \aq_1') \land t_{11} ++ [\aq_1'] ++ t_{12} = t_1'
\]
%
\[
	\exists t_{21}, t_{22}, \aq_2'. ~ (\aq_2 \aqeq \aq_2') \land t_{21} ++ [\aq_2'] ++ t_{22} = t_2'
\]
%
Then according to the list concatenation operations, we have:
%
\[
	\exists t_{11}, t_{12}, \aq_1', t_{21}, t_{22}, \aq_2'. ~ (\aq_1 \aqeq \aq_1') \land (\aq_2 \aqeq \aq_2') \land 
	t_{11} ++ [\aq_1'] ++ t_{12} ++ t_{21} ++ [\aq_2'] ++ t_{22} = t_1' ++ t_2'
\]
%
Let $t^1 = t_{11}$, $t^2 = t_{12} ++ t_{21}$ and $t^3 = t_{22}$, since $(t_2 - t) = t_1' ++ t_2'$, we have:
%
\[
	\exists t^1, t^2, t^3, \aq_1', \aq_2'. ~ s.t.,~ 
	(\aq_1 \aqeq \aq_1') \land (\aq_2 \aqeq \aq_2') \land t^1 ++ [\aq_1'] ++ t^2 ++ [\aq_2'] ++ t^3 = (t_2 - t)
\]
%
This case is proved.
%
\\
%
For the other directions, according to the hypothesis $(4)$ and $(a)$ proved above, we know:
\[
	(\aq_1 <_{aq} \aq_2)
\]
%
This case is proved.
%
\caseL{$\ewhile [b]^l \edo c'$}
%
According to the \rname{while-b} and \rname{ifw-b} rule, we have following evaluation:
\[
	\config{m, \ewhile [b]^l \edo c', t, w} \to^{*}
	\config{m, \eif_w([v]^l, c', \eskip), t, w} ~ (w1)
\]
%
where $\config{m, b} \barrow^{*} v$ and $v$ is either $\etrue$ or $\efalse$.
%
\subcaseL{$v = \etrue$ }
%
By evaluation rule \rname{ifw-true}, we have:
%
\[
	\config{m, \eif_w([\etrue]^l, c';\ewhile [b]^l, 2 \edo c', \eskip), t, w} 
	\to^{*} \config{m, c';\ewhile [b]^l \edo c', t, [l \mapsto 1]} ~ (w2)
\]
%
By rule \rname{seq1}, we have the following evaluation:
%
\[
	\config{m, c';\ewhile [b]^l, 2 \edo c', t, [l \mapsto 1]} \to^{*} \config{m', \eskip;\ewhile [b]^l, 2 \edo c', t', w'} ~ (w3)
\] 
%
where $\config{m, c', t, [l \mapsto 1]} \to^{*} \config{m', \eskip, t', w'} ~ (w4)$. 
%
By induction hypothesis on $c'$ and the evaluation $(w4)$, we know $t' - t$ is well-formed.
\\
By the rule \rname{seq2}, we have:
%
\[
	\config{m', \eskip;\ewhile [b]^l, 2 \edo c', t', w'} \to \config{m', \ewhile [b]^l, 2 \edo c', t', w'} ~ (w6)
\]
%
By the termination of the while loop, we know there exist an evaluation by repeating the evaluation in $(w1), (w2), (w3)$ and $(w6)$:
%
\[
	\config{m', \ewhile [b]^l, 2 \edo c', t', w'} \to^{*} \config{m^n, \eif_w([\efalse]^l, c';\ewhile [b]^l, n + 1, \edo c', \eskip), t^n, w^n}
\]
%
By the evaluation $(w5)$, we know for every resulting trace $t^i$ in the evaluation of each iteration $i = 2, \ldots, n$,
the following invariant holds:
\\
$t^i - t^{i - 1}$ is well formed.
\\
%
By evaluation rule \rname{ifw-false}, we have:
%
\[
	\config{m, \eif_w([\efalse]^l, c';\ewhile [b]^l, n + 1 \edo c', \eskip), t^n, w + l} 
	\to^{*} \config{m, \eskip, t^n, (w + l)/l}
\]
%
%
It is sufficient to show $(t^n - t)$ is well-formed.
%
Unfolding the definition~\ref{def:wf_trace}, it is sufficient to show that the trace $(t^n - t)$ holds both the \emph{Uniqueness} and \emph{Ordering} property.
\\
Let $t_1 = t' - t$ and $t_i = t^{i} - t^{i - 1}$ for $i = 2, \ldots, n$, unfolding the subtraction operations, we have:
\[
	(t^n - t) = t_1 ++ t_2 ++ \ldots ++ t_n
\]
\\
By repeating the proof $(\diamond)$ and $(\square)$ in the sequence case, we have this case proved. 
%
\subcaseL{$v = \efalse$}
%
By evaluation rule \rname{ifw-false}, we have:
%
\[
	\config{m, \eif_w([\efalse]^l, c';\ewhile [b]^l \edo c', \eskip), t, w} 
	\to^{*} \config{m, \eskip, t, w/l}
\]
%
%
Because $t - t = []$, this sub-case is proved.
%

\end{proof}
%
\begin{lem}
[Trace Non-Decreasing].
\\
$$
\forall \trace \in \mathcal{T}, c \st
\config{c, \trace} \rightarrow^{*} \config{\eskip, \trace'} 
\implies \exists \trace'' \in \mathcal{T} \st \trace \cdot \trace'' = \trace'
$$
\end{lem}
%
\begin{coro}
\label{coro:aqintrace}
\[
\event \eventin \trace \implies \exists \trace_1, \trace_2 \in \mathcal{T}, 
\event' \in \eventset \st (\event \eventeq \event') \land \trace_1 \cdot \event' \cdot \trace_2 = t  
\]
\end{coro}
\begin{subproof}
Proof in File: {\tt ``coro\_aqintrace.tex''}
% \begin{coro}
\label{coro:aqintrace}
\[
\aq \aqin t \implies \exists t_1, t_2, \aq'. ~ s.t., ~ (\aq \aqeq \aq') \land t_1 ++ [\aq'] ++ t_2 = t	
\]
\end{coro}
\begin{subproof}
Proof in File: {\tt ``coro\_aqintrace.tex''}
% \begin{coro}
\label{coro:aqintrace}
\[
\aq \aqin t \implies \exists t_1, t_2, \aq'. ~ s.t., ~ (\aq \aqeq \aq') \land t_1 ++ [\aq'] ++ t_2 = t	
\]
\end{coro}
\begin{subproof}
Proof in File: {\tt ``coro\_aqintrace.tex''}
% \begin{coro}
\label{coro:aqintrace}
\[
\aq \aqin t \implies \exists t_1, t_2, \aq'. ~ s.t., ~ (\aq \aqeq \aq') \land t_1 ++ [\aq'] ++ t_2 = t	
\]
\end{coro}
\begin{subproof}
Proof in File: {\tt ``coro\_aqintrace.tex''}
% \input{coro_aqintrace}
By unfolding the $\aq \aqin t$, we have the following cases:
%
\caseL{$t = []$} The hypothesis is $\efalse$, this case is proved.
%
\caseL{$t = \aq'::t' \land \aq' \aqeq \aq $}
%
Let $t_1 = []$ and $t_2 = t'$, by unfolding the list concatenation operation, we have:
%
\[
	t_1 ++ [\aq'] ++ t_2 = [] ++ [\aq'] ++ t' = t
\]
%
Since $\aq' \aqeq \aq$ by the hypothesis, this case is proved.
%
\caseL{$t = \aq'::t' \land \aq' \aqneq \aq $}
%
By induction hypothesis on $\aq \aqin t'$, we know:
%
\[
	\exists t_1', t_2', \aq''. ~ s.t., ~ (\aq \aqeq \aq'') \land t_1' ++ [\aq''] ++ t_2' = t'	
\]
%
Let $t_1 = \aq'::t_1'$ and $t_2 = t_2'$, by unfolding the list concatenation operation, we have:
%
\[
	t_1 ++ [\aq''] ++ t_2 = (\aq':: t_1') ++ [\aq''] ++ t_2' = \aq'::t' = t
\]
%
Since $\aq'' \aqeq \aq$ by the hypothesis, this case is proved.
%
\end{subproof}
By unfolding the $\aq \aqin t$, we have the following cases:
%
\caseL{$t = []$} The hypothesis is $\efalse$, this case is proved.
%
\caseL{$t = \aq'::t' \land \aq' \aqeq \aq $}
%
Let $t_1 = []$ and $t_2 = t'$, by unfolding the list concatenation operation, we have:
%
\[
	t_1 ++ [\aq'] ++ t_2 = [] ++ [\aq'] ++ t' = t
\]
%
Since $\aq' \aqeq \aq$ by the hypothesis, this case is proved.
%
\caseL{$t = \aq'::t' \land \aq' \aqneq \aq $}
%
By induction hypothesis on $\aq \aqin t'$, we know:
%
\[
	\exists t_1', t_2', \aq''. ~ s.t., ~ (\aq \aqeq \aq'') \land t_1' ++ [\aq''] ++ t_2' = t'	
\]
%
Let $t_1 = \aq'::t_1'$ and $t_2 = t_2'$, by unfolding the list concatenation operation, we have:
%
\[
	t_1 ++ [\aq''] ++ t_2 = (\aq':: t_1') ++ [\aq''] ++ t_2' = \aq'::t' = t
\]
%
Since $\aq'' \aqeq \aq$ by the hypothesis, this case is proved.
%
\end{subproof}
By unfolding the $\aq \aqin t$, we have the following cases:
%
\caseL{$t = []$} The hypothesis is $\efalse$, this case is proved.
%
\caseL{$t = \aq'::t' \land \aq' \aqeq \aq $}
%
Let $t_1 = []$ and $t_2 = t'$, by unfolding the list concatenation operation, we have:
%
\[
	t_1 ++ [\aq'] ++ t_2 = [] ++ [\aq'] ++ t' = t
\]
%
Since $\aq' \aqeq \aq$ by the hypothesis, this case is proved.
%
\caseL{$t = \aq'::t' \land \aq' \aqneq \aq $}
%
By induction hypothesis on $\aq \aqin t'$, we know:
%
\[
	\exists t_1', t_2', \aq''. ~ s.t., ~ (\aq \aqeq \aq'') \land t_1' ++ [\aq''] ++ t_2' = t'	
\]
%
Let $t_1 = \aq'::t_1'$ and $t_2 = t_2'$, by unfolding the list concatenation operation, we have:
%
\[
	t_1 ++ [\aq''] ++ t_2 = (\aq':: t_1') ++ [\aq''] ++ t_2' = \aq'::t' = t
\]
%
Since $\aq'' \aqeq \aq$ by the hypothesis, this case is proved.
%
\end{subproof}
%
\end{subproof}
% \\
%
Program entry label: $\entry_c : \mbox{Command} \to \mathbb{N}$ 
\[
  \entry_c \triangleq 
\left\{
  \begin{array}{ll} 
     l       
    & c = [\eskip]{}^l
    \\ 
    l    & c = [\assign{x}{\expr_1}]{}^l
    \\ 
    l      
    & c = [\assign{x}{\query(\qexpr_1)}]{}^l
    \\
   l
    & c_1 = \eif([b]{}^l, c_t, c_f)
    \\ 
    l         
    & c = \ewhile [b]^l \edo c'
    \\ 
    \entry_{c1}
    & c = c1;c2
  \end{array}
  \right.
\]
%
\begin{defn}[Equivalence of Program]
%
\label{def:aq_prog}
Given 2 programs $c_1$ and $c_2$:
\[
c_1 =_{c} c_2
\triangleq 
\left\{
  \begin{array}{ll} 
    \etrue        
    & c_1 = \eskip \land c_2 = \eskip
    \\ 
    \forall \trace \in \mathcal{T} \st \exists v \in \mathcal{VAL}
    \st \config{ \trace, \expr_1} \aarrow v \land \config{ \trace, \expr_1} \aarrow v     
    & c_1 = \assign{x}{\expr_1} \land c_2 = \assign{x}{\expr_2} 
    \\ 
    \qexpr_1 =_{q} \qexpr_2       
    & c_1 = \assign{x}{\query(\qexpr_1)} \land c_1 = \assign{x}{\query(\qexpr_2)} 
    \\
    c_1^f =_{c} c_2^f \land c_1^t =_{c} c_2^t
    & c_1 = \eif(b, c_1^t, c_1^f) \land c_2 = \eif(b, c_2^t, c_2^f)
    \\ 
    c_1' =_{c} c_2'         
    & c_1 = \ewhile b \edo c_1' \land c_2 = \ewhile b \edo c_2'
    \\ 
    c_1^h =_{c} c_2^h \land c_1^t =_{c} c_2^t
    & c_1 = c_1^h;c_1^t \land c_2 = c_2^h;c_2^t 
  \end{array}
  \right.
\]
%
$c_1 \neq_{c} c_2$  is defined vice versa.
%
\end{defn}
%
Given 2 programs $c$ and $c'$, $c'$ is a sub-program of$c$, i.e., $c' \in_{c} c$ is defined as:
\begin{equation}
c' \in_{c} c \triangleq \exists c_1, c_2, c''. ~ s.t.,~
c =_{c} c_1; c''; c_2 \land c' =_{c} c''
\end{equation} 
%
% \todo{
% \begin{lem}[While Map Remains Unchanged (Invariant)]
% \label{lem:wunchange}
% Given a program $c$ with a starting memory $m$, trace $t$ and while map $w$, s.t.,
% $\config{m, c, t, w} \to^{*} \config{m', \eskip, t', w'}$ and $Labels(c) \cap Keys(w) = \emptyset$, then 
% \[
%   w = w'
% \]
% \end{lem}
% \begin{subproof}[Proof of Lemma~\ref{lem:wunchange}]
% %
% Proof in File: {\tt ``lem\_wunchange.tex''}
% % 
%
\todo{
\begin{lem}[While Map Remains Unchanged (Invariant)]
\label{lem:wunchange}
Given a program $c$ with a starting memory $m$, trace $t$ and while map $w$, s.t.,
$\config{m, c, t, w} \to^{*} \config{m', \eskip, t', w'}$ and $Labels(c) \cap Keys(w) = \emptyset$, then 
\[
	w = w'
\]
\end{lem}
\begin{subproof}[Proof of Lemma~\ref{lem:wunchange}]
%
By induction on program $c$, we have following cases:
%
\caseL{$[\assign x \expr]^{l}$}
%
From the evaluation rule \rname{assn1} and rule \rname{assn2}, we know there exists an evaluation
$$
\config{m, [\assign x \expr]^{l}, t, w} \to^{*} \config{m[v/x], \eskip, t', w}
$$
%
, where $\config{m,e} \to^{*} v$.
%
The resulting while map is $w$, this case is proved because $w = w$.
%
\caseL{$[\assign{x}{\query(\qexpr)}]^{l}$}
%
By repeatedly applying the operational semantics rule \rname{query-e}, we know there exist $\qval$ s.t.,:
%
\[
\config{m, [\assign{x}{\query(\qexpr)}]^{l}, t, w} \to^{*} \config{m', [\assign{x}{\query(\qval)}]^{l}, t', w}
\]
%
According to the evaluation rule \rname{query-v}, we have the following transition:
\[
	\config{m', [\assign{x}{\query(\qval)}]^{l}, t', []} \to \config{m', \eskip, t' ++ [(\qval, l, w)], w}
\]
%
The resulting while map is $w$, this case is proved because $w = w$.
%%
\caseL{$\eif([b]^l, c_1, c_2)$}
%
According to the evaluation rule \rname{if}, we know there exists an evaluation by repeatedly applying the rule \rname{if}: 
%
\[
\config{m, \eif([b]^l, c_1, c_2), t, w} \to^{*} \config{m_b, \eif([v]^l, c_1, c_2), t', w} ~ (1)
\]
%
where $v$ is either $\etrue$ or $\efalse$.
\\
Considering 2 subcases where $v$ is either $\etrue$ or $\efalse$:
%
\subcaseL{$v = \etrue$}
%
By applying the evaluation rule \rname{if-t}, we have:
\[
\config{m_b, \eif([\etrue]^l, c_1, c_2), t', w} \to \config{m_b, c_1, t', w} ~ (t1)
\]
%
By induction hypothesis on $c_1$ with starting memory $m_b$, trace $t'$ and while map $w$, let $\config{m_1, \eskip, t_1, w_1}$ be the configuration s.t.
\[
	\config{m_b, c_1, t', w} \to^{*} \config{m_1, \eskip, t_1, w_1} ~(t2)
\]
we know $w_1 = w$.
\\
According to the evaluation $(a)$, $(t1)$ and $(t2)$, we have the following evaluation:
% 
\[
	\config{m, \eif([b]^l, c_1, c_2), t, w} \to^{*} \config{m_1, \eskip, t_1, w}
\]
%
, where the resulting while map is still $w$, this sub-case is proved.
%
\subcaseL{$v = \efalse$}
%
By applying the evaluation rule \rname{if-f}, we have:
\[
\config{m_b, \eif([\efalse]^l, c_1, c_2), t', w} \to \config{m', c_2, t', w} ~ (f1)
\]
%
By induction hypothesis on $c_1$ with starting memory $m_b$, trace $t'$ and while map $w$, 
let $\config{m_2, \eskip, t_2, w_2}$ be the configuration s.t.
\[
	\config{m_b, c_2, t', w} \to^{*} \config{m_2, \eskip, t_2, w_2} ~(f2)
\]
we know $w_2 = w$.
\\
According to the evaluation $(a)$, $(f1)$ and $(f2)$, we the following evaluation:
\[
	\config{m, \eif([b]^l, c_1, c_2), t, w} \to^{*} \config{m_2, \eskip, t_2, w}
\]
%
, where the resulting while map is still $w$, this sub-case is proved.
%
\caseL{$c_1; c_2$}
%
According to the evaluation rule \rname{seq1}, we have the following evaluation by repeatedly applying \rname{seq1} 
%
\[
\config{m, c_1;c_2, t, w} \to^{*} \config{m_1, [\eskip]^l;c_2,  t_1, w_1} ~ (1)
\]
%
where 
%
\[
	\config{m, c_1, t, w} \to^{*} \config{m_1, [\eskip]^l, t_1, w_1]} ~ (2)
\]
%
By induction hypothesis on $c_1$ and $(2)$, we know $w_1 = w$.
%
According to the evaluation rule \rname{seq2}, we have:
%
\[
\config{m_1, [\eskip]^l;c_2,  t_1, w} \to \config{m_1, c_2, t_1, w} ~ (3)
\]
%
let $\config{m_2, [\eskip]^l, t_2, w_2]}$ be the configuration s.t.,:
%
\[
	\config{m_1, c_2, t_1, w} \to^{*} \config{m_2, [\eskip]^l, t_2, w_2]}~ (4).
\]
%
%
By induction hypothesis on $c_2$, $m_1$, $t_1$, $w$ and $(4)$, we know $w_2 = w$.
%
\\
According to the evaluations $(1)$, $(3)$ and $(4)$, we have:
\[
	\config{m, c_1;c_2, t, w} \to^{*} \config{m', [\eskip]^l, t', w}
\]
%
This case is proved.
%
\caseL{$\ewhile [b]^l \edo c'$}
%
According to the \rname{while-b} and \rname{ifw-b} rule, we have following evaluation:
\[
	\config{m, \ewhile [b]^l \edo c', t, w} \to^{*}
	\config{m, \eif_w([v]^l, c', \eskip), t, w} ~ (w1)
\]
%
where $v$ is either $\etrue$ or $\efalse$.
%
\subcaseL{$v = \etrue$ }
%
By evaluation rule \rname{ifw-true}, we have:
%
\[
	\config{m, \eif_w([\etrue]^l, c';\ewhile [b]^l \edo c', \eskip), t, w} 
	\to^{*} \config{m, c';\ewhile [b]^l \edo c', t, w+l} ~ (w2)
\]
%
By rule \rname{seq1}, we have the following evaluation:
%
\[
	\config{m, c';\ewhile [b]^l \edo c', t, w+l} \to^{*} \config{m', \eskip;\ewhile [b]^l \edo c', t', w'} ~ (w3)
\] 
%
where $\config{m, c', t, w+l} \to^{*} \config{m', \eskip, t', w'} ~ (w4)$. 
%
By induction hypothesis on $c'$ and the evaluation $(w4)$, we know $w' = w + l$. Then we have $\config{m, c', t, w+l} \to^{*} \config{m', \eskip, t', w+l} ~ (w5)$.
%
By the rule \rname{seq2}, we have:
%
\[
	\config{m', \eskip;\ewhile [b]^l \edo c', t', w + l} \to \config{m', \ewhile [b]^l \edo c', t', w + l} ~ (w6)
\]
%
By the termination of the while loop, we know there exist an evaluation by repeating the evaluation in $(w1), (w2), (w3)$ and $(w6)$:
%
\[
	\config{m', \ewhile [b]^l \edo c', t', w + l} \to^{*} \config{m', \eif_w([\efalse]^l, c';\ewhile [b]^l \edo c', \eskip), t'', w''}
\]
%
By the evaluation $(w5)$, we know in the evaluation of each iterations, the resulting while map is always $w + l$. So we have $w'' = w + l$.
%
By evaluation rule \rname{ifw-false}, we have:
%
\[
	\config{m, \eif_w([\efalse]^l, c';\ewhile [b]^l \edo c', \eskip), t'', w + l} 
	\to^{*} \config{m, \eskip, t, (w + l)/l}
\]
%
By the hypothesis that $l \notin Keys(w)$, unfolding the $(w + l)/l$ operation, we have:
%
\[
	w = (w + l)/l.
\]
%
This sub-case is proved.
%
\subcaseL{$v = \efalse$}
%
By evaluation rule \rname{ifw-false}, we have:
%
\[
	\config{m, \eif_w([\efalse]^l, c';\ewhile [b]^l \edo c', \eskip), t, w} 
	\to^{*} \config{m, \eskip, t, w/l}
\]
%
By the hypothesis that $l \notin Keys(w)$, unfolding the $w/l$ operation, we have:
%
\[
	w = w/l.
\]
%
This sub-case is proved.
%
\end{subproof}
}



% %
% \end{subproof}
% }
%
% \todo{
% \begin{lem}[Trace is Written Only]
% \label{lem:twriteonly}
% Given a program $c$ with starting trace $t_1$ and $t_2$,
% for arbitrary starting memory $m$ and while map $w$,
% if there exist evaluations
% $$\config{m, c, t_1, w} \to^{*} \config{m_1', \eskip, t_1', w_1'}$$
% % 
% $$\config{m, c, t_2, w} \to^{*} \config{m_2', \eskip, t_2', w_2'}$$
% %
% then:
% %
% \[
%   m_1' = m_2' \land w_1' = w_2'
% \]
% \end{lem}
% %
% \begin{subproof}[Proof of Lemma~\ref{lem:twriteonly}]
% %
% Proof in File: {\tt ``lem\_twriteonly.tex''}
% % \todo{
\begin{lem}[Trace Uniqueness]
\label{lem:tunique}
Given a program $c$ with a starting memory $m$, 
for any starting trace $t_1$ and $t_2$, if there exist evaluations
$$\config{m, c, t_1, []} \to^{*} \config{m_1', \eskip, t_1', w_1'}$$
% 
$$\config{m, c, t_2, []} \to^{*} \config{m_2', \eskip, t_2', w_2'}$$
%
then:
%
\[
	t_1' - t_1 = t_2' - t_2
\]
\end{lem}
%
\begin{subproof}[Proof of Lemma~\ref{lem:tunique}]
%
By induction on program $c$, we have following cases:
%
\caseL{ $[\assign x \expr]^{l}$}
%
From the evaluation rule \rname{assn1} and rule \rname{assn2}, we know there exist evaluations
$$
\config{m, [\assign x \expr]^{l}, t_1, w} \to^{*} \config{m[v/x], \eskip, t_1, w}
$$
%
\[
\config{m, [\assign x \expr]^{l}, t_2, w} \to^{*} \config{m[v/x], \eskip, t_2, w}
\]
%
for starting trace $t_1$ and $t_2$ respectively, where $\config{m,e} \aarrow^{*} v$.
%
Since the resulting traces in the 2 evaluations are still $t_1$ and $t_2$, 
and $t_1 - t_1 = t_2 - t_2 = []$, this case is proved.
%
\caseL{$[\assign{x}{\query(\qexpr)}]^{l}$}
%
By repeatedly applying the operational semantics rule \rname{query-e}, we know there exist evaluations
%
\[
\config{m, [\assign{x}{\query(\qexpr)}]^{l}, t_1, w} \to^{*} \config{m, [\assign{x}{\query(\qval)}]^{l}, t_1, w}
\]
%
\[
\config{m, [\assign{x}{\query(\qexpr)}]^{l}, t_2, w} \to^{*} \config{m, [\assign{x}{\query(\qval)}]^{l}, t_2, w}
\]
%
for starting trace $t_1$ and $t_2$ respectively, where $\config{m, \qexpr} \qarrow^{*} \qval$.
%
\\
%
According to the evaluation rule \rname{query-v}, we then have the following 2 transitions with starting trace $t_1$ and $t_2$ respectively:
%
\[
\config{m, [\assign{x}{\query(\qval)}]^{l}, t_1, w} \to \config{m_1, \eskip, t_1 ++ [(\qval, l, w)], w}
\]
%
\[
\config{m, [\assign{x}{\query(\qval)}]^{l}, t_2, w} \to \config{m_2, \eskip, t_2 ++ [(\qval, l, w)], w}
\]
%
The resulting traces are $t_1 ++ [(\qval, l, w)]$, and $t_2 ++ [(\qval, l, w)]$. 
\\
By unfolding the trace subtraction operation, we have:
\[
(t_1 ++ [(\qval, l, w)]) - t_1 = [(\qval, l, w)]
%
\land
%
(t_2 ++ [(\qval, l, w)]) - t_2 = [(\qval, l, w)]
\]
%
This case is proved.
%%
\caseL{$\eif([b]^l, c_1, c_2)$}
%
According to the evaluation rule \rname{if}, we know there exist 2 evaluations for $t_1$ and $t_2$ respectively by repeatedly applying the rule \rname{if}: 
%
\[
\config{m, \eif([b]^l, c_1, c_2), t_1, w} \to^{*} \config{m, \eif([v]^l, c_1, c_2), t_1, w} ~ (1)
\]
%
\[
\config{m, \eif([b]^l, c_1, c_2), t_2, w} \to^{*} \config{m, \eif([v]^l, c_1, c_2), t_2, w} ~ (2)
\]
%
, where $\config{m, b} \barrow^{*} v$ and $v$ is either $\etrue$ or $\efalse$.
\\
Considering 2 subcases where $v$ is either $\etrue$ or $\efalse$:
%
\subcaseL{$v = \etrue$}
%
By applying the evaluation rule \rname{if-t}, we have:
\[
\config{m, \eif([\etrue]^l, c_1, c_2), t_1, w} \to \config{m, c_1, t_1, w} ~ (3)
\]
%
\[
\config{m, \eif([\etrue]^l, c_1, c_2), t_2, w} \to \config{m, c_1, t_2, w} ~ (4)
\]
%
By induction hypothesis on $c_1$ with $m$, $w$ $t_1$ and $t_2$, we know there exist evaluations:
\[
	\config{m, c_1, t_1, w} \to^{*} \config{m_1, \eskip, t_1', w_1} ~ (5)
\]
%
\[
	\config{m, c_1, t_2, w} \to^{*} \config{m_2, \eskip, t_2', w_2} ~ (6)
\]
%
, where $t_1' - t_1 = t_2' - t_2$.
\\
According to the evaluations $(1), (3), (5)$ and $(2), (4), (6)$, we have the following 2 evaluations:
% 
\[
	\config{m, \eif([b]^l, c_1, c_2), t_1, w} \to^{*} \config{m_1, \eskip, t_1', w_1}
\]
%
\[
	\config{m, \eif([b]^l, c_1, c_2), t_2, w} \to^{*} \config{m_2, \eskip, t_2', w_2}
\]
%
Since $t_1' - t_1 = t_2' - t_2$, this sub-case is proved.
%
\subcaseL{$v = \efalse$}
%
By applying the evaluation rule \rname{if-f}, we have:
%
\[
\config{m, \eif([\efalse]^l, c_1, c_2), t_1, w} \to \config{m, c_2, t_1, w} ~ (3)
\]
%
\[
\config{m, \eif([\efalse]^l, c_1, c_2), t_2, w} \to \config{m, c_2, t_2, w} ~ (4)
\]
%
By induction hypothesis on $c_2$ with $m$, $w$ $t_1$ and $t_2$, we know there exist evaluations:
\[
	\config{m, c_2, t_1, w} \to^{*} \config{m_1', \eskip, t_1'', w_1'} ~ (5)
\]
%
\[
	\config{m, c_2, t_2, w} \to^{*} \config{m_2', \eskip, t_2'', w_2'} ~ (6)
\]
%
, where $t_1'' - t_1 = t_2'' - t_2$.
\\
According to the evaluations $(1), (3), (5)$ and $(2), (4), (6)$, we have the following 2 evaluations:
% 
\[
	\config{m, \eif([b]^l, c_1, c_2), t_1, w} \to^{*} \config{m_1', \eskip, t_1'', w_1'}
\]
%
\[
	\config{m, \eif([b]^l, c_1, c_2), t_2, w} \to^{*} \config{m_2', \eskip, t_2'', w_2'}
\]
%
Since $t_1'' - t_1 = t_2'' - t_2$, this sub-case is proved.
%
\caseL{$c_1; c_2$}
%
According to the evaluation rule \rname{seq1}, we have the following 2 evaluations by repeatedly applying \rname{seq1} 
%
\[
\config{m, c_1;c_2, t_1, w} \to^{*} \config{m_1^1, [\eskip]^l;c_2,  t_1^1, w_1^1} ~ (1)
\]
%
%
\[
\config{m, c_1;c_2, t_2, w} \to^{*} \config{m_2^1, [\eskip]^l;c_2,  t_2^1, w_2^1} ~ (2)
\]
%
where 
%
\[
	\config{m, c_1, t_1, w} \to^{*} \config{m_1^1, [\eskip]^l, t_1^1, w_1^1]} ~ (3)
\]
%
%
\[
\config{m, c_1, t_2, w} \to^{*} \config{m_2^1, [\eskip]^l,  t_2^1, w_2^1} ~ (4)
\]
%
%
By induction hypothesis on $c_1$ and $(3), (4)$, we know $t_1^1 - t_1 = t_2^1 - t_2 ~ (a)$.
%
%
According to the evaluation rule \rname{seq2}, we have:
%
\[
\config{m_1^1, [\eskip]^l;c_2, t_1^1, w_1^1} \to \config{m_1^1, c_2, t_1^1, w_1^1}
\]
%
\[
\config{m_2^1, [\eskip]^l;c_2, t_2^1, w_2^1} \to \config{m_2^1, c_2, t_2^1, w_2^1}
\]
%
%
%
By induction hypothesis on $c_2$, $t_2^1$ and $t_1^1$ we have the following 2 evaluations 
%
\[
	\config{m_1^1, c_2, t_1^1, w} \to^{*} \config{m_1^2, \eskip, t_1^2, w_1^2]}
\]
%
\[
	\config{m_2^1, c_2, t_2^1, w} \to^{*} \config{m_2^2, \eskip, t_2^2, w_2^2]}
\]
%
and $t_1^2 - t_1^1 = t_2^2 - t_2^1 ~ (b)$.
%
\\
By $(a)$ and $(b)$, unfolding the trace subtraction operation, we know
%
\[
\exists t_1', t_1'', t_2', t_2''. ~ 
t_1 ++ t_1' = t_1^1 \land t_1^1 ++ t_1'' = t_1^2 
\land t_2 ++ t_2' = t_2^1 \land t_2^1 ++ t_2'' = t_2^2 
\]
%
By writing, we have:
%
\[
\exists t_1', t_1'', t_2', t_2''. ~ 
t_1 ++ t_1' ++ t_1'' = t_1^2 
\land t_2 ++ t_2' ++ t_2'' = t_2^2 
\]
%
Again, by the definition of trace subtraction, we have:
%
\[
	t_2^2 - t_2 = t_1^2 - t_1
\]
%
This case is proved.
%
\caseL{$\ewhile [b]^l \edo c'$}
%
According to the \rname{while-b} and \rname{ifw-b} rule, we have following 2 evaluations:
%
\[
	\config{m, \ewhile [b]^l \edo c', t_1, w} \to^{*}
	\config{m, \eif_w([v]^l, c', \eskip), t_1, w}
\]
%
\[
	\config{m, \ewhile [b]^l \edo c', t_2, w} \to^{*}
	\config{m, \eif_w([v]^l, c', \eskip), t_2, w}
\]
%
where $v$ is either $\etrue$ or $\efalse$.
%
\subcaseL{$v = \etrue$ }
%
By evaluation rule \rname{ifw-true}, we have:
%
\[
	\config{m, \eif_w([\etrue]^l, c';\ewhile [b]^l \edo c', \eskip), t_1, w} 
	\to^{*} \config{m, c';\ewhile [b]^l \edo c', t_1, w+l} ~ (w2)
\]
%
\[
	\config{m, \eif_w([\etrue]^l, c';\ewhile [b]^l \edo c', \eskip), t_2, w} 
	\to^{*} \config{m, c';\ewhile [b]^l \edo c', t_2, w+l} ~ (w2)
\]
%
%
By rule \rname{seq1}, we have the following evaluations:
%
\[
	\config{m, c';\ewhile [b]^l \edo c', t_1, w+l} \to^{*} \config{m', \eskip;\ewhile [b]^l \edo c', t_1', w'} ~ (w3)
\] 
%
%
\[
	\config{m, c';\ewhile [b]^l \edo c', t_2, w+l} \to^{*} \config{m', \eskip;\ewhile [b]^l \edo c', t_2', w'} ~ (w3)
\] 
%
, where 
$\config{m, c', t_1, w+l} \to^{*} \config{m_1', \eskip, t_1', w_1'}$
and
$\config{m, c', t_2, w+l} \to^{*} \config{m_2', \eskip, t_2', w_2'}$. 
%
\\
%
By induction hypothesis on $c'$, we know:
\[
t_1' - t_1 = t_2' - t_2.
\]
%
By the rule \rname{seq2}, we have:
%
\[
	\config{m_1', \eskip;\ewhile [b]^l \edo c', t_1', w + l} \to \config{m_1', \ewhile [b]^l \edo c', t_1', w_1'} ~ (w6)
\]
%
\[
	\config{m_2', \eskip;\ewhile [b]^l \edo c', t_2', w + l} \to \config{m_2', \ewhile [b]^l \edo c', t_2', w_2'} ~ (w6)
\]
%
By the termination of the while loop, we know there exist an evaluation by repeating the evaluation in $(w1), (w2), (w3)$ and $(w6)$:
%
\[
	\config{m_1', \ewhile [b]^l \edo c', t_1', w_1'} \to^{*} \config{m_1^n, \eif_w([\efalse]^l, c';\ewhile [b]^l \edo c', \eskip), t_1^n, w_1^n}
\]
%
\[
	\config{m_2', \ewhile [b]^l \edo c', t_2', w_2'} \to^{*} \config{m_2^n, \eif_w([\efalse]^l, c';\ewhile [b]^l \edo c', \eskip), t_2^n, w_2^n}
\]
%
By the evaluation $(w5)$ and induction hypothesis on each iteration, we know the resulting trace $t_1^i$ and $t_2^i$ in the evaluation of each iteration $i = 2, \ldots, k$ satisfies:
%
\[
t_1^i - t_1^{i - 1} = t_2^i - t_2^{i - 1}
\]
%
By unfolding the trace subtraction operation, we have:
%
\[
	t_1^n - t_1 = t_2^n - t_2
\]
%
By evaluation rule \rname{ifw-false}, we have:
%
\[
	\config{m_1^n, \eif_w([\efalse]^l, c';\ewhile [b]^l \edo c', \eskip), t_1^n, w_1^n} 
	\to^{*} \config{m_1^n, \eskip, t_1^n, w_1^n - l}
\]
%
\[
	\config{m_2^n, \eif_w([\efalse]^l, c';\ewhile [b]^l \edo c', \eskip), t_2^n, w_2^n} 
	\to^{*} \config{m_2^n, \eskip, t_2^n, w_2^n - l}
\]
%
Since $t_1^n - t_1 = t_2^n - t_2$, this sub-case is proved.
%
\subcaseL{$v = \efalse$}
%
By evaluation rule \rname{ifw-false}, we have:
%
\[
	\config{m, \eif_w([\efalse]^l, c';\ewhile [b]^l \edo c', \eskip), t_1, w} 
	\to^{*} \config{m, \eskip, t_1, w/l}
\]
%
%
\[
	\config{m, \eif_w([\efalse]^l, c';\ewhile [b]^l \edo c', \eskip), t_2, w} 
	\to^{*} \config{m, \eskip, t_2, w/l}
\]
%
Because $t_1 - t_1 = t_2 - t_2 = []$, this sub-case is proved.
%
\end{subproof}
}
% \end{subproof}
% }
%
% \todo{
% \begin{lem}[Trace Uniqueness]
% \label{lem:tunique}
% Given a program $c$ with a starting memory $m$, \wq{a while map w,}
% for any starting trace $t_1$ and $t_2$, if there exist evaluations
% $$\config{m, c, t_1, w} \to^{*} \config{m_1', \eskip, t_1', w_1'}$$
% % 
% $$\config{m, c, t_2, w} \to^{*} \config{m_2', \eskip, t_2', w_2'}$$
% %
% then:
% %
% \[
% t_1' - t_1 = t_2' - t_2
% \]
% \end{lem}
% %
% \begin{subproof}[Proof of Lemma~\ref{lem:tunique}]
% %
% Proof in File: {\tt ``lem\_tunique.tex''}
% % \todo{
\begin{lem}[Trace Uniqueness]
\label{lem:tunique}
Given a program $c$ with a starting memory $m$, 
for any starting trace $t_1$ and $t_2$, if there exist evaluations
$$\config{m, c, t_1, []} \to^{*} \config{m_1', \eskip, t_1', w_1'}$$
% 
$$\config{m, c, t_2, []} \to^{*} \config{m_2', \eskip, t_2', w_2'}$$
%
then:
%
\[
	t_1' - t_1 = t_2' - t_2
\]
\end{lem}
%
\begin{subproof}[Proof of Lemma~\ref{lem:tunique}]
%
By induction on program $c$, we have following cases:
%
\caseL{ $[\assign x \expr]^{l}$}
%
From the evaluation rule \rname{assn1} and rule \rname{assn2}, we know there exist evaluations
$$
\config{m, [\assign x \expr]^{l}, t_1, w} \to^{*} \config{m[v/x], \eskip, t_1, w}
$$
%
\[
\config{m, [\assign x \expr]^{l}, t_2, w} \to^{*} \config{m[v/x], \eskip, t_2, w}
\]
%
for starting trace $t_1$ and $t_2$ respectively, where $\config{m,e} \aarrow^{*} v$.
%
Since the resulting traces in the 2 evaluations are still $t_1$ and $t_2$, 
and $t_1 - t_1 = t_2 - t_2 = []$, this case is proved.
%
\caseL{$[\assign{x}{\query(\qexpr)}]^{l}$}
%
By repeatedly applying the operational semantics rule \rname{query-e}, we know there exist evaluations
%
\[
\config{m, [\assign{x}{\query(\qexpr)}]^{l}, t_1, w} \to^{*} \config{m, [\assign{x}{\query(\qval)}]^{l}, t_1, w}
\]
%
\[
\config{m, [\assign{x}{\query(\qexpr)}]^{l}, t_2, w} \to^{*} \config{m, [\assign{x}{\query(\qval)}]^{l}, t_2, w}
\]
%
for starting trace $t_1$ and $t_2$ respectively, where $\config{m, \qexpr} \qarrow^{*} \qval$.
%
\\
%
According to the evaluation rule \rname{query-v}, we then have the following 2 transitions with starting trace $t_1$ and $t_2$ respectively:
%
\[
\config{m, [\assign{x}{\query(\qval)}]^{l}, t_1, w} \to \config{m_1, \eskip, t_1 ++ [(\qval, l, w)], w}
\]
%
\[
\config{m, [\assign{x}{\query(\qval)}]^{l}, t_2, w} \to \config{m_2, \eskip, t_2 ++ [(\qval, l, w)], w}
\]
%
The resulting traces are $t_1 ++ [(\qval, l, w)]$, and $t_2 ++ [(\qval, l, w)]$. 
\\
By unfolding the trace subtraction operation, we have:
\[
(t_1 ++ [(\qval, l, w)]) - t_1 = [(\qval, l, w)]
%
\land
%
(t_2 ++ [(\qval, l, w)]) - t_2 = [(\qval, l, w)]
\]
%
This case is proved.
%%
\caseL{$\eif([b]^l, c_1, c_2)$}
%
According to the evaluation rule \rname{if}, we know there exist 2 evaluations for $t_1$ and $t_2$ respectively by repeatedly applying the rule \rname{if}: 
%
\[
\config{m, \eif([b]^l, c_1, c_2), t_1, w} \to^{*} \config{m, \eif([v]^l, c_1, c_2), t_1, w} ~ (1)
\]
%
\[
\config{m, \eif([b]^l, c_1, c_2), t_2, w} \to^{*} \config{m, \eif([v]^l, c_1, c_2), t_2, w} ~ (2)
\]
%
, where $\config{m, b} \barrow^{*} v$ and $v$ is either $\etrue$ or $\efalse$.
\\
Considering 2 subcases where $v$ is either $\etrue$ or $\efalse$:
%
\subcaseL{$v = \etrue$}
%
By applying the evaluation rule \rname{if-t}, we have:
\[
\config{m, \eif([\etrue]^l, c_1, c_2), t_1, w} \to \config{m, c_1, t_1, w} ~ (3)
\]
%
\[
\config{m, \eif([\etrue]^l, c_1, c_2), t_2, w} \to \config{m, c_1, t_2, w} ~ (4)
\]
%
By induction hypothesis on $c_1$ with $m$, $w$ $t_1$ and $t_2$, we know there exist evaluations:
\[
	\config{m, c_1, t_1, w} \to^{*} \config{m_1, \eskip, t_1', w_1} ~ (5)
\]
%
\[
	\config{m, c_1, t_2, w} \to^{*} \config{m_2, \eskip, t_2', w_2} ~ (6)
\]
%
, where $t_1' - t_1 = t_2' - t_2$.
\\
According to the evaluations $(1), (3), (5)$ and $(2), (4), (6)$, we have the following 2 evaluations:
% 
\[
	\config{m, \eif([b]^l, c_1, c_2), t_1, w} \to^{*} \config{m_1, \eskip, t_1', w_1}
\]
%
\[
	\config{m, \eif([b]^l, c_1, c_2), t_2, w} \to^{*} \config{m_2, \eskip, t_2', w_2}
\]
%
Since $t_1' - t_1 = t_2' - t_2$, this sub-case is proved.
%
\subcaseL{$v = \efalse$}
%
By applying the evaluation rule \rname{if-f}, we have:
%
\[
\config{m, \eif([\efalse]^l, c_1, c_2), t_1, w} \to \config{m, c_2, t_1, w} ~ (3)
\]
%
\[
\config{m, \eif([\efalse]^l, c_1, c_2), t_2, w} \to \config{m, c_2, t_2, w} ~ (4)
\]
%
By induction hypothesis on $c_2$ with $m$, $w$ $t_1$ and $t_2$, we know there exist evaluations:
\[
	\config{m, c_2, t_1, w} \to^{*} \config{m_1', \eskip, t_1'', w_1'} ~ (5)
\]
%
\[
	\config{m, c_2, t_2, w} \to^{*} \config{m_2', \eskip, t_2'', w_2'} ~ (6)
\]
%
, where $t_1'' - t_1 = t_2'' - t_2$.
\\
According to the evaluations $(1), (3), (5)$ and $(2), (4), (6)$, we have the following 2 evaluations:
% 
\[
	\config{m, \eif([b]^l, c_1, c_2), t_1, w} \to^{*} \config{m_1', \eskip, t_1'', w_1'}
\]
%
\[
	\config{m, \eif([b]^l, c_1, c_2), t_2, w} \to^{*} \config{m_2', \eskip, t_2'', w_2'}
\]
%
Since $t_1'' - t_1 = t_2'' - t_2$, this sub-case is proved.
%
\caseL{$c_1; c_2$}
%
According to the evaluation rule \rname{seq1}, we have the following 2 evaluations by repeatedly applying \rname{seq1} 
%
\[
\config{m, c_1;c_2, t_1, w} \to^{*} \config{m_1^1, [\eskip]^l;c_2,  t_1^1, w_1^1} ~ (1)
\]
%
%
\[
\config{m, c_1;c_2, t_2, w} \to^{*} \config{m_2^1, [\eskip]^l;c_2,  t_2^1, w_2^1} ~ (2)
\]
%
where 
%
\[
	\config{m, c_1, t_1, w} \to^{*} \config{m_1^1, [\eskip]^l, t_1^1, w_1^1]} ~ (3)
\]
%
%
\[
\config{m, c_1, t_2, w} \to^{*} \config{m_2^1, [\eskip]^l,  t_2^1, w_2^1} ~ (4)
\]
%
%
By induction hypothesis on $c_1$ and $(3), (4)$, we know $t_1^1 - t_1 = t_2^1 - t_2 ~ (a)$.
%
%
According to the evaluation rule \rname{seq2}, we have:
%
\[
\config{m_1^1, [\eskip]^l;c_2, t_1^1, w_1^1} \to \config{m_1^1, c_2, t_1^1, w_1^1}
\]
%
\[
\config{m_2^1, [\eskip]^l;c_2, t_2^1, w_2^1} \to \config{m_2^1, c_2, t_2^1, w_2^1}
\]
%
%
%
By induction hypothesis on $c_2$, $t_2^1$ and $t_1^1$ we have the following 2 evaluations 
%
\[
	\config{m_1^1, c_2, t_1^1, w} \to^{*} \config{m_1^2, \eskip, t_1^2, w_1^2]}
\]
%
\[
	\config{m_2^1, c_2, t_2^1, w} \to^{*} \config{m_2^2, \eskip, t_2^2, w_2^2]}
\]
%
and $t_1^2 - t_1^1 = t_2^2 - t_2^1 ~ (b)$.
%
\\
By $(a)$ and $(b)$, unfolding the trace subtraction operation, we know
%
\[
\exists t_1', t_1'', t_2', t_2''. ~ 
t_1 ++ t_1' = t_1^1 \land t_1^1 ++ t_1'' = t_1^2 
\land t_2 ++ t_2' = t_2^1 \land t_2^1 ++ t_2'' = t_2^2 
\]
%
By writing, we have:
%
\[
\exists t_1', t_1'', t_2', t_2''. ~ 
t_1 ++ t_1' ++ t_1'' = t_1^2 
\land t_2 ++ t_2' ++ t_2'' = t_2^2 
\]
%
Again, by the definition of trace subtraction, we have:
%
\[
	t_2^2 - t_2 = t_1^2 - t_1
\]
%
This case is proved.
%
\caseL{$\ewhile [b]^l \edo c'$}
%
According to the \rname{while-b} and \rname{ifw-b} rule, we have following 2 evaluations:
%
\[
	\config{m, \ewhile [b]^l \edo c', t_1, w} \to^{*}
	\config{m, \eif_w([v]^l, c', \eskip), t_1, w}
\]
%
\[
	\config{m, \ewhile [b]^l \edo c', t_2, w} \to^{*}
	\config{m, \eif_w([v]^l, c', \eskip), t_2, w}
\]
%
where $v$ is either $\etrue$ or $\efalse$.
%
\subcaseL{$v = \etrue$ }
%
By evaluation rule \rname{ifw-true}, we have:
%
\[
	\config{m, \eif_w([\etrue]^l, c';\ewhile [b]^l \edo c', \eskip), t_1, w} 
	\to^{*} \config{m, c';\ewhile [b]^l \edo c', t_1, w+l} ~ (w2)
\]
%
\[
	\config{m, \eif_w([\etrue]^l, c';\ewhile [b]^l \edo c', \eskip), t_2, w} 
	\to^{*} \config{m, c';\ewhile [b]^l \edo c', t_2, w+l} ~ (w2)
\]
%
%
By rule \rname{seq1}, we have the following evaluations:
%
\[
	\config{m, c';\ewhile [b]^l \edo c', t_1, w+l} \to^{*} \config{m', \eskip;\ewhile [b]^l \edo c', t_1', w'} ~ (w3)
\] 
%
%
\[
	\config{m, c';\ewhile [b]^l \edo c', t_2, w+l} \to^{*} \config{m', \eskip;\ewhile [b]^l \edo c', t_2', w'} ~ (w3)
\] 
%
, where 
$\config{m, c', t_1, w+l} \to^{*} \config{m_1', \eskip, t_1', w_1'}$
and
$\config{m, c', t_2, w+l} \to^{*} \config{m_2', \eskip, t_2', w_2'}$. 
%
\\
%
By induction hypothesis on $c'$, we know:
\[
t_1' - t_1 = t_2' - t_2.
\]
%
By the rule \rname{seq2}, we have:
%
\[
	\config{m_1', \eskip;\ewhile [b]^l \edo c', t_1', w + l} \to \config{m_1', \ewhile [b]^l \edo c', t_1', w_1'} ~ (w6)
\]
%
\[
	\config{m_2', \eskip;\ewhile [b]^l \edo c', t_2', w + l} \to \config{m_2', \ewhile [b]^l \edo c', t_2', w_2'} ~ (w6)
\]
%
By the termination of the while loop, we know there exist an evaluation by repeating the evaluation in $(w1), (w2), (w3)$ and $(w6)$:
%
\[
	\config{m_1', \ewhile [b]^l \edo c', t_1', w_1'} \to^{*} \config{m_1^n, \eif_w([\efalse]^l, c';\ewhile [b]^l \edo c', \eskip), t_1^n, w_1^n}
\]
%
\[
	\config{m_2', \ewhile [b]^l \edo c', t_2', w_2'} \to^{*} \config{m_2^n, \eif_w([\efalse]^l, c';\ewhile [b]^l \edo c', \eskip), t_2^n, w_2^n}
\]
%
By the evaluation $(w5)$ and induction hypothesis on each iteration, we know the resulting trace $t_1^i$ and $t_2^i$ in the evaluation of each iteration $i = 2, \ldots, k$ satisfies:
%
\[
t_1^i - t_1^{i - 1} = t_2^i - t_2^{i - 1}
\]
%
By unfolding the trace subtraction operation, we have:
%
\[
	t_1^n - t_1 = t_2^n - t_2
\]
%
By evaluation rule \rname{ifw-false}, we have:
%
\[
	\config{m_1^n, \eif_w([\efalse]^l, c';\ewhile [b]^l \edo c', \eskip), t_1^n, w_1^n} 
	\to^{*} \config{m_1^n, \eskip, t_1^n, w_1^n - l}
\]
%
\[
	\config{m_2^n, \eif_w([\efalse]^l, c';\ewhile [b]^l \edo c', \eskip), t_2^n, w_2^n} 
	\to^{*} \config{m_2^n, \eskip, t_2^n, w_2^n - l}
\]
%
Since $t_1^n - t_1 = t_2^n - t_2$, this sub-case is proved.
%
\subcaseL{$v = \efalse$}
%
By evaluation rule \rname{ifw-false}, we have:
%
\[
	\config{m, \eif_w([\efalse]^l, c';\ewhile [b]^l \edo c', \eskip), t_1, w} 
	\to^{*} \config{m, \eskip, t_1, w/l}
\]
%
%
\[
	\config{m, \eif_w([\efalse]^l, c';\ewhile [b]^l \edo c', \eskip), t_2, w} 
	\to^{*} \config{m, \eskip, t_2, w/l}
\]
%
Because $t_1 - t_1 = t_2 - t_2 = []$, this sub-case is proved.
%
\end{subproof}
}
% \end{subproof}
% }
%

%
%
%
%
%
%
%
% \subsection{SSA Transformation and Soundness of Transformation}
% in File {\tt ``ssa\_transform\_sound.tex''}
% %
\subsection{SSA Transformation}
We use a translation environment $\delta$, to map variables $x$ in the {\tt While} language to those variables $\ssa{x}$ in the SSA language.
We use a name environment denoted as $\Sigma$ as a set of ssa variables, to get a fresh variable by $fresh(\Sigma)$. 
We define $\delta_1 \bowtie \delta_2 $ in a similar way as
\cite{vekris2016refinement}.
%
\[ 
\delta_1 \bowtie \delta_2 = \{ ( x, {\ssa{x_1}, \ssa{x_2}} ) \in 
\mathcal{VAR} \times \mathcal{SVAR} \times \mathcal{SVAR} \mid x \mapsto {\ssa{x_1}} \in \delta_1 , x \mapsto {\ssa{x_2} } \in \delta_2, {\ssa{x_1} \not= {\ssa{x_2} }  }  \} 
\]
%
\[ 
\delta_1 \bowtie \delta_2 / \bar{x} = \{ ( x, {\ssa{x_1}, \ssa{x_2}} ) \in 
\mathcal{VAR} \times \mathcal{SVAR} \times \mathcal{SVAR}
 \mid x \not\in \bar{x} \land x \mapsto {\ssa{x_1}} \in \delta_1 , x \mapsto {\ssa{x_2} } \in \delta_2, {\ssa{x_1} \not= {\ssa{x_2} }   }  \} 
 \]
 %
We call a list of variables $\bar{x}$.
\[
 [\bar{x}, \bar{\ssa{x_1}}, \bar{\ssa{x_2}}] = \{ (x, x_1,x_2)  | \forall 0 \leq i < |\bar{x}|, x = \bar{x }[i] \land x_1 = \bar{x_1}[i] \land x_2 = \bar{x_2 }[i] \land |\bar{x}| = |\bar{x_1}| = |\bar{x_2}|   \}
\]
%
\begin{mathpar}
\boxed{ \delta ; e \hookrightarrow \ssa{e} }
\and
\inferrule{
}{
 \delta ; x \hookrightarrow \delta(x)
}~{\textbf{S-VAR}}
\and
\boxed{ \Sigma; \delta ; c  \hookrightarrow \ssa{c} ; \delta' ; \Sigma' }
\and
\inferrule{
  { \delta ; \bexpr \hookrightarrow \ssa{\bexpr} }
  \and
  { \Sigma; \delta ; c_1 \hookrightarrow \ssa{c_1} ; \delta_1;\Sigma_1 }
  \and
  {\Sigma_1; \delta ; c_2 \hookrightarrow \ssa{c_2} ; \delta_2 ; \Sigma_2 }
  \\
  {[\bar{x}, \ssa{\bar{{x_1}}, \bar{{x_2}}}] = \delta_1 \bowtie \delta_2  }
  \and
   {[\bar{y}, \ssa{\bar{{y_1}}, \bar{{y_2}}}] = \delta \bowtie \delta_1 / \bar{x} }
  \and
   {[\bar{z}, \ssa{\bar{{z_1}}, \bar{{z_2}}}] = \delta \bowtie \delta_2 / \bar{x} }
  \\
  { \delta' =\delta[\bar{x} \mapsto \ssa{\bar{{x}}'} ][\bar{y} \mapsto \ssa{\bar{{y}}'} ][\bar{z} \mapsto \ssa{\bar{{z}}'} ]}
  \and 
  {\ssa{\bar{{x}}', \bar{y}', \bar{z}'} \ fresh(\Sigma_2)
  }
  \and{\Sigma' = \Sigma_2 \cup \{ \ssa{ \bar{x}', \bar{y}', \bar{z}' } \} }
}{
 \Sigma; \delta ; [\eif(\bexpr, c_1, c_2)]^l  \hookrightarrow [\ssa{ \eif(\bexpr, [\bar{{x}}', \bar{{x_1}}, \bar{{x_2}}] ,[\bar{{y}}', \bar{{y_1}}, \bar{{y_2}}] ,[\bar{{z}}', \bar{{z_1}}, \bar{{z_2}}] , {c_1}, {c_2})}]^l; \delta';\Sigma'
}~{\textbf{S-IF}}
%
\and
%
\inferrule{
 {\delta ; \expr \hookrightarrow \ssa{\expr} }
 \and
 {\delta' = \delta[x \mapsto \ssa{{x}} ]}
 \and{ \ssa{x} \ fresh(\Sigma) }
 \and { \Sigma' = \Sigma \cup \{ \ssa{x} \} }
}{
 \Sigma;\delta ; [\assign x \expr]^{l} \hookrightarrow [\ssa{\assign {{x}}{ \expr}}]^{l} ; \delta'; \Sigma'
}~{\textbf{S-ASSN}}
%
\and
%
\inferrule{
 {\delta ; \query \hookrightarrow \ssa{\query}}
 \and
 {\delta ; \qexpr \hookrightarrow \ssa{\qexpr}}
 \and
 {\delta' = \delta[x \mapsto \ssa{x} ]}
 \and{ \ssa{x} \ fresh(\Sigma) }
  \and { \Sigma' = \Sigma \cup \{ \ssa{x} \} }
}{
 \Sigma;\delta ; [\assign{x}{\query(\qexpr)}]^{l} \hookrightarrow 
 [\assign {\ssa{x}}{ \ssa{\query(\qexpr)}}]^{l} ; \delta';\Sigma'
}~{\textbf{S-QUERY}}
%
%%
\and
%
%
\and
%
\inferrule{
    { \Sigma; \delta ; c \hookrightarrow \ssa{c_1} ; \delta_1; \Sigma_1 }
     \and
    { [ \bar{x}, \ssa{\bar{{x_1}}}, \ssa{\bar{{x_2}}} ] = \delta \bowtie \delta_1 }
    \\
     {
     \ssa{\bar{{x}}'} \ fresh(\Sigma_1 )}
    \and {\delta' = \delta[\bar{x} \mapsto \ssa{\bar{{x}}'}]}
    \and 
     {\delta' ; \bexpr \hookrightarrow \ssa{\bexpr} }
     \and
    {\ssa{c' = c_1[\bar{x}'/ \bar{x_1}]   } }
  }{ 
  \Sigma; \delta ;  \ewhile ~ [\bexpr]^{l} ~ \edo ~ c 
  \hookrightarrow 
  \ssa{\ewhile ~ [\bexpr]^{l}, 0, [\bar{{x}}', \bar{{x_1}}, \bar{{x_2}}] ~ \edo ~ {c} } ; \delta'; \Sigma_1 \cup \{\ssa{\bar{x}'}  \}
}~{\textbf{S-WHILE}
}
\and
%
\inferrule{
 {\Sigma;\delta ; c_1 \hookrightarrow \ssa{c_1} ; \delta_1; \Sigma_1} 
 \and
 {\Sigma_1; \delta_1 ; c_2 \hookrightarrow \ssa{c_2} ; \delta'; \Sigma'} 
}{
\Sigma;\delta ; c_1 ; c_2 \hookrightarrow \ssa{c_1} ; \ssa{c_2} \ ; \delta';\Sigma'
}~{\textbf{S-SEQ}}
\end{mathpar}

\paragraph{Concrete examples.}
\[
c_1 \triangleq
\begin{array}{l}
     \left[x \leftarrow \query(1) \right]^1; \\
     \eif \; (x ==0)^{2} \; \\
    \ethen \; \left[y \leftarrow \query(2) \right]^3\; \\
    \eelse \; \left[y \leftarrow 0 \right]^4 ; \\
    \eif \; (x == 1 )^5\; \\
    \ethen \; \left[ y \leftarrow 0 \right]^6\; \\
    \eelse \; \left[y \leftarrow \query(2) \right]^7\\
\end{array}
%
%
\hspace{20pt} \hookrightarrow  \hspace{20pt}
%
\begin{array}{l}
     \left[ \ssa{x_1} \leftarrow \query(1) \right]^1; \\
     \eif \; (\ssa{x_1 ==0})^{2}, [\ssa{ y_3, y_1,y_2  }],[],[]  \; \\
    \ethen \; \left[ \ssa{y_1} \leftarrow \query(2) \right]^3\; \\
    \eelse \; \left[\ssa{y_2 \leftarrow 0 } \right]^4 ; \\
    \eif \; (\ssa{x_1 == 1} )^{5} , [\ssa{ y_6, y_4, y_5 } ] \; \\
    \ethen \; \left[ \ssa{y_4 \leftarrow 0} \right]^6\; \\
    \eelse \; \left[\ssa{y_5} \leftarrow \query(2) \right]^7\\
\end{array}
\]
\[
c_2 \triangleq
\begin{array}{l}
   \left[ x \leftarrow \query(1) \right]^1; \\
   \left[y \leftarrow \query(2) \right]^2 ; \\
    \eif \;( x + y == 5 )^3\; \\
    \ethen \;\left[ z \leftarrow \query(3)\right]^4 \; \\
    \eelse \;\left[ \ssa{\eskip}\right]^5 ; \\
   \left[ w \leftarrow q_4 \right]^6; \\
\end{array}
\hspace{20pt} \hookrightarrow \hspace{20pt}
%
\begin{array}{l}
   \left[ \ssa{ x_1 } \leftarrow \query(1) \right]^1; \\
   \left[\ssa{ y_1} \leftarrow \query(2) \right]^2 ; \\
    \eif \;( \ssa{ x_1 + y_1 == 5} )^3, [ ],[] ,[ ]\; \\
    \ethen \;\left[ \ssa{ z_1 }
    \leftarrow \query(3)\right]^4 \; \\
    \eelse \;\left[ \eskip\right]^5 ; \\
   \left[ \ssa{ w_1} \leftarrow \query(4) \right]^6; \\
\end{array}
\]

{
\[
c_3 \triangleq
\begin{array}{l}
     \left[x \leftarrow \query(1) \right]^1 ; \\
     \left[i \leftarrow 0 \right]^2 ; \\
    \ewhile ~  [i < 100]^3 ~ \edo
    \\
    ~ \Big( 
    \left[z \leftarrow \query(3) \right]^4; \\
    \left[x \leftarrow z + x \right]^5; \\
    \left[i \leftarrow i + 1 \right]^6
    \Big) ;
\end{array}
%
\hspace{20pt} \hookrightarrow \hspace{20pt} 
%
\begin{array}{l}
     \left[\ssa{x_1} \leftarrow \query(1) \right]^1 ; \\
     \left[\ssa{i_1} \leftarrow 0 \right]^2 ; \\
    \ewhile
    ~ [\ssa{i_1} < 100]^3, 0,
    ~\ssa{[ x_3,x_1 ,x_2 ], [i_3, i_1, i_2] }~
    \edo \\
    ~ \Big( 
    \left[\ssa{z_1} \leftarrow \query(3) \right]^4; \\
    \left[ \ssa{x_2} \leftarrow \ssa{z_1 + x_3} \right]^5; \\
    \left[\ssa{i_2} \leftarrow \ssa{i_3} + 1 \right]^6
    \Big) ;
\end{array}
\]
}
%
\begin{figure}
   \[
 \begin{array}{lll}
    | \ewhile ~ [ \sbexpr ]^{l}, n, [\bar{\ssa{x}}, \bar{\ssa{x_1}}, \bar{\ssa{x_2}}] 
    ~ \edo ~  \ssa{c}|  
    &=& \ewhile ~ [|\sbexpr|]^{l},  ~ \edo ~ |\ssa{c}| 
	\\
    |\ssa{c_1 ; c_2}|  &=& |\ssa{c_1}| ; |\ssa{c_2}| 
    \\
    |[\eif(\sbexpr,
    [ \bar{\ssa{x}}, \bar{\ssa{x_1}}, \bar{\ssa{x_2}}] ,
    [ \bar{\ssa{y}}, \bar{\ssa{y_1}}, \bar{\ssa{y_2}}] , 
    [\bar{\ssa{z}}, \bar{\ssa{z_1}}, \bar{\ssa{z_2}}] , 
    \ssa{ c_1, c_2)}]^{l}|  
    &=&
    [\eif(|\sbexpr|, |\ssa{ c_1}|, |\ssa{c_2}|)]^{l}
    \\
    | [\assign {\ssa{x}}{\ssa{\expr}}]^{l}| & = & [\assign {|\ssa{x}|}{|\ssa{\expr}|} ]^{l}
    \\
    | [\assign {\ssa{x}}{\query(\ssa{\qexpr})} ]^{l} | & = & [\assign {|\ssa{x}|}{|\query(\ssa{\qexpr})|}]^{l}
    \\
    |\ssa{x}_i| & = & x 
    \\
    |n | & = & n 
    \\
    | \saexpr_1 \oplus_{a} \saexpr_2 | & = &  |\ssa{\aexpr_1}| \oplus_a |\ssa{\aexpr_2}| \\
    | \sbexpr_1 \oplus_{b} \sbexpr_2 | & = &  |\sbexpr_1| \oplus_b |\sbexpr_2|
 \end{array}
\]
    \caption{The Erasure of SSA}
    \label{fig:ssa_erasure-while}
\end{figure}
%
%
%
% 
%
\subsection{The Soundness of the Transformation}
In this subsection, we show our transformation from the {\tt While} language to its SSA form is sound with respect to the adaptivity. 
To be specific, a transformed program $\ssa{c}$ starting with appropriate configuration, generates the same trace as the program before the transformation $c$, in its corresponding configuration.
%
%
\begin{defn}[\todo{Written Variables}].
\\
We defined the assigned variables in the while language program $c$ as $\avars{c}$,the assigned variables in the ssa-form program $\ssa{c}$ as $\avarssa{\ssa{c}}$ defined as follows.
\[
\begin{array}{lll}
    \avars{\assign{x}{\expr}} & =& \{ x \} \\
    \avars{\assign{x}{\query(\qexpr)}} & =& \{ x \} \\
    \avars{c_1; c_2}  & = & \avars{c_1} \cup \avars{c_2} \\
    \avars{\ewhile ~ \bexpr ~ \edo ~ c} &= &  \avars{c} \\
    \avars{\eif(\bexpr, c_1, c_2)} & =&  \avars{c_1} \cup \avars{c_2}\\
\end{array} 
\]
%
\[
\begin{array}{lll}
    \avarssa{\ssa{\assign{x}{\expr}}} & =& \{ \ssa{x} \}
    \\
    \avarssa{\ssa{\assign{x}{\query(\ssa{\qexpr})}}} & =& \{ \ssa{x} \}
    \\
    \avarssa{\ssa{c_1; c_2 } }  & = & \avarssa{\ssa{c}_1} \cup \avarssa{\ssa{c}_2}
    \\
    \avarssa{\ewhile ~ \ssa{\bexpr, n, [\bar{x}, \bar{x_1}, \bar{x_2}] ~ \edo ~ \ssa{c}}}
    & = &  
    \{\ssa{\bar{x}}\} \cup \avarssa{\ssa{c}} 
    \\
    \avarssa{\eif(\ssa{\bexpr,[\bar{x}, \bar{x_1}, \bar{x_2}],[\bar{y}, \bar{y_1}, \bar{y_2}],[\bar{z}, \bar{z_1}, \bar{z_2}], c_1, c_2} )} 
    & =&  \{ \ssa{\bar{x}},\ssa{\bar{y}} , \ssa{\bar{z}} \} 
    \cup \avarssa{\ssa{c_1}} \cup \avarssa{\ssa{c_2}}\\
\end{array}
\]
\end{defn}
\begin{defn}[\todo{Read Variables}].
\\
{
The variables read in the while language programs $c$ as $\vars{c}$, variables used in ssa-form program $\ssa{c}$ : 
}
\[
\begin{array}{lll}
    \vars{\assign{x}{\expr}} & =& \vars{\expr}  \\
    \vars{\assign{x}{\query(\qexpr)}} & =&\{  \} \\
    \vars{ c_1; c_2  }  & = & \vars{c_1} \cup \vars{c_2} \\
    \vars{  \eloop ~ \aexpr ~ \edo ~ c  } &= &\vars{\aexpr} \cup \vars{c} \\
    \vars{\eif(\bexpr, c_1, c_2)} & =& \vars{\bexpr} \cup \vars{c_1} \cup \vars{c_2}\\
\end{array}
\]
\[
\begin{array}{lll}
    \varssa{\ssa{\assign{x}{\expr}}} & =& \varssa{\ssa{\expr}}  \\
    \varssa{\ssa{\assign{x}{\query(\qexpr)}}} & =& \{  \} \\
    \varssa{ \ssa{c_1; c_2}  }  & = & \varssa{\ssa{c}_1} \cup \varssa{\ssa{c}_2} \\
    % \varssa{  \eloop ~ \ssa{\aexpr, n, [\bar{x}, \bar{x_1}, \bar{x_2}] ~ \edo ~ c} } &= &\varssa{\ssa{\aexpr}} \cup \varssa{\ssa{c}}  \cup \{ \ssa{\bar{x_1}} \} \cup \{ \ssa{\bar{x_2}} \}\\
    {\varssa{  \ewhile ~ \ssa{\bexpr, n, [\bar{x}, \bar{x_1}, \bar{x_2}] ~ \edo ~ c} }} 
    &= &
    \varssa{\ssa{\bexpr}} \cup \varssa{\ssa{c}}  \cup \{ \ssa{\bar{x_1}} \} \cup \{ \ssa{\bar{x_2}} \}\\
    \varssa{\eif(\ssa{\bexpr,[\bar{x}, \bar{x_1}, \bar{x_2}], [\bar{y}, \bar{y_1}, \bar{y_2}],[\bar{z}, \bar{z_1}, \bar{z_2}], c_1, c_2} )} & =& \varssa{\ssa{\bexpr}} \cup \varssa{\ssa{c_1}} \cup \varssa{\ssa{c_2}} \cup \{\ssa{\bar{x_1}, \bar{x_2},\bar{y_1}, \bar{y_2},\bar{z_1}, \bar{z_2} }\}  \\
\end{array}
\]
\end{defn}
%
\begin{defn}[\todo{Necessary Variables}].
\\
{
We call the variables needed to be assigned before executing the program $c$ as necessary variables $\fv{c}$. Its ssa form is : $\fvssa{\ssa{c}}$.
}  
 \[
 \begin{array}{lll}
     \fvars{\assign{x}{\expr} }  & = & \vars{\expr}  \\
     \fvars{\assign{x}{\query(\qexpr)} }  & = & \{ \}  \\
     {\fvars{  \ewhile ~ \bexpr ~ \edo ~ c  } }&= & \vars{\bexpr} \cup \fvars{c} \\
     \fvars{\eif(\bexpr, c_1, c_2)} & =& \vars{\bexpr} \cup \fvars{c_1} \cup \fvars{c_2}  \\
      \fvars{c_1 ; c_2} & = & \fvars{c_1} \cup ( \fvars{c_2} - \avars{c_1})
 \end{array}
 \]
 \[
 \begin{array}{lll}
     \fvssa{\ssa{\assign{x}{\expr}} }  & = & \varssa{\ssa{\expr}}  \\
     \fvssa{ \ssa{ \assign{x}{\query(\qexpr)}} }  & = & \{ \}  \\
     {\fvssa{  \ewhile ~ \ssa{\bexpr, n, [\bar{x}, \bar{x_1}, \bar{x_2}] ~ \edo ~ c} } }
     &= & 
     \varssa{\ssa{\bexpr}} \cup \fvssa{\ssa{c}}[\ssa{ \bar{x_1}} / \ssa{\bar{x}}]\\
     \fvssa{\eif(\ssa{\bexpr,[\bar{x}, \bar{x_1}, \bar{x_2}],[\bar{y}, \bar{y_1}, \bar{y_2}],[\bar{z}, \bar{z_1}, \bar{z_2}], c_1, c_2} )} & =& \varssa{\ssa{\bexpr}} \cup \fvssa{\ssa{c_1}} \cup \fvssa{\ssa{c_2}}  \\
      \fvssa{\ssa{c_1 ; c_2}} & = & \fvssa{\ssa{c_1}} \cup ( \fvssa{\ssa{c_2}} - \avarssa{\ssa{c_1}})
 \end{array}
 \]
%
\end{defn}
%
The Lemma~\ref{lem:fv} and \ref{lem:same_value} proved the preserving properties for variables and values during the transformation.
%
\begin{lem}[Variable Preserving]
\label{lem:fv}
If $\Sigma;\delta ; c \hookrightarrow \ssa{c} ; \delta';\Sigma' $, $\fvssa{\ssa{c}} = \delta(\fvars{c})$. 
\end{lem}
\begin{proof}
 By induction on the transformation.
 \begin{itemize}
    \caseL{Case $\inferrule{
  { \delta ; \bexpr \hookrightarrow \ssa{\bexpr} }
  \and
  { \delta ; c_1 \hookrightarrow \ssa{c_1} ; \delta_1 }
  \and
  {\delta ; c_2 \hookrightarrow \ssa{c_2} ; \delta_2 }
  \\
  {[\bar{x}, \ssa{\bar{{x_1}}, \bar{{x_2}}}] = \delta_1 \bowtie \delta_2  }
  \and
   {[\bar{y}, \ssa{\bar{{y_1}}, \bar{{y_2}}}] = \delta \bowtie \delta_1 / \bar{x} }
  \and
   {[\bar{z}, \ssa{\bar{{z_1}}, \bar{{z_2}}}] = \delta \bowtie \delta_2 / \bar{x} }
  \\
  { \delta' =\delta[\bar{x} \mapsto \ssa{\bar{{x}}'} ]}
  \and 
  {\ssa{\bar{{x}}', \bar{y}', \bar{z}'} \ fresh }
}{
 \delta ; [\eif(\bexpr, c_1, c_2)]^l  \hookrightarrow [\ssa{ \eif(\bexpr, [\bar{{x}}', \bar{{x_1}}, \bar{{x_2}}] ,[\bar{{y}}', \bar{{y_1}}, \bar{{y_2}}] ,[\bar{{z}}', \bar{{z_1}}, \bar{{z_2}}] , {c_1}, {c_2})}]^l; \delta'
}~{\textbf{S-IF}} $}
From the definition of $\fvssa{[\eif(\sbexpr, [\bar{\ssa{x'}}, \bar{\ssa{x_1}}, \bar{\ssa{x_2}}] , \ssa{c_1}, \ssa{c_2})]^l} = \varssa{\ssa{\bexpr}} \cup \fvssa{\ssa{c_1}} \cup \fvssa{\ssa{c_2}}$. We want to show: \[\varssa{\ssa{\bexpr}}) \cup \fvssa{\ssa{c_1}} \cup \fvssa{\ssa{c_2}} = \delta( \vars{\bexpr}) \cup \delta(\fv{c_1}) \cup \delta(\fv{c_2}  )\]
By induction hypothosis on the second and third premise, we know that : $\fvssa{\ssa{c_1}} = \delta(\fv{c_1}) $ and $\fvssa{\ssa{c_2}} = \delta(\fv{c_2}) $.  We still need to show that: 
\[
  \varssa{\ssa{\bexpr}} = \delta(\vars{\bexpr})
\] 
From the first premise, we know that $\vars{b} \subseteq \dom(\delta)$. This is goal is proved by the rule $\textbf{S-VAR}$ on all the variables in $\bexpr$.\\
{\caseL{Case
$\inferrule{
    { \Sigma; \delta ; c \hookrightarrow \ssa{c_1} ; \delta_1; \Sigma_1 }
     \and
    { [ \bar{x}, \ssa{\bar{{x_1}}}, \ssa{\bar{{x_2}}} ] = \delta \bowtie \delta_1 }
    \\
     {\ssa{\bar{{x}}'} \ fresh(\Sigma_1 )}
    \and {\delta' = \delta[\bar{x} \mapsto \ssa{\bar{{x}}'}]}
    \and 
     {\delta' ; \bexpr \hookrightarrow \ssa{\bexpr} }
     \and
    {\ssa{c' = c_1[\bar{x}'/ \bar{x_1}]   } }
    % \and{ \delta' ; c \hookrightarrow \ssa{c'} ; \delta'' }
  }{ 
  \Sigma; \delta ;  \ewhile ~ [\bexpr]^{l} ~ \edo ~ c 
  \hookrightarrow 
  \ssa{\ewhile ~ [\bexpr]^{l}, 0, [\bar{{x}}', \bar{{x_1}}, \bar{{x_2}}] ~ \edo ~ {c} } ; \delta'; \Sigma_1 \cup \{\ssa{\bar{x}'}  \}
}~{\textbf{S-WHILE}}
$}
}
{
Unfolding the definition, we need to show:
\[\varssa{\ssa{\bexpr}} \cup \fvssa{\ssa{c'}}[\ssa{ \bar{x_1}} / \ssa{\bar{x}}] = \delta (\vars{\bexpr}) \cup \delta(\fv{c} ) \]
We can similarly show that $\varssa{\ssa{\bexpr}} = \delta(\vars{b})$ as in the if case. We still need to show that: 
\[
 \fvssa{\ssa{c_1[\bar{x}' / \bar{x_1}]}}[ \ssa{ \bar{x_1} } / \ssa{\bar{x}'}] =  \delta(\fv{c} )
\]
It is proved by induction hypothesis on $  { \Sigma; \delta ; c \hookrightarrow \ssa{c_1} ; \delta_1; \Sigma_1 }$.\\
}
%
\caseL{Case $\inferrule{
 {\Sigma;\delta ; c_1 \hookrightarrow \ssa{c_1} ; \delta_1; \Sigma_1} 
 \and
 {\Sigma_1; \delta_1 ; c_2 \hookrightarrow \ssa{c_2} ; \delta'; \Sigma'} 
}{
\Sigma;\delta ; c_1 ; c_2 \hookrightarrow \ssa{c_1} ; \ssa{c_2} \ ; \delta';\Sigma'
}~{\textbf{S-SEQ}}$}
To show:
  \[ \fvssa{\ssa{c_1}} \cup ( \fvssa{\ssa{c_2}} - \avarssa{\ssa{c_1}}) = \delta(\fv{c_1} )\cup \delta( \fv{c_2} - \avars{c_1}) \]
  By induction hypothesis on the first premise, we know that : $ \fvssa{\ssa{c_1}} = \delta(\fv{c_1} ) $, still to show: 
    \[ ( \fvssa{\ssa{c_2}} - \avarssa{\ssa{c_1}}) = \delta( \fv{c_2} - \avars{c_1})
    \]
    We know that $\delta_1 = \delta [\avars{c_1} \mapsto \avarssa{\ssa{c_1}} ]$, so by induction hypothesis, we know: $ \fvssa{\ssa{c_2}} = \delta[\avars{c_1} \mapsto \avarssa{\ssa{c_1}} ]( \fv{c_2})  = \delta(\fv{c_2}) \cup \avarssa{\ssa{c_1}} - \delta(\avars{c_1}) $.
    
    This case is proved.
 \end{itemize}
 
\end{proof}

{
We first define a good memory in the {\tt While} language $m$ or in the ssa language $\ssa{m}$ with respect to a translation environment $\delta$, denoted as $m \vDash \delta$ and $\ssa{m} \vDash \delta$ respectively. 
%
\begin{defn}[Well Defined Memory].
\begin{enumerate}
    % \item $m \vDash c \triangleq \forall x \in \fv{c}, \exists v, (x, v) \in m$.
    \item $ m \vDash \delta  \triangleq \forall x \in \dom(\delta), \exists v, (x,v) \in m$.
    % \item $\ssa{m} \vDash_{ssa} \ssa{c} \triangleq \forall \ssa{x} \in \fvssa{\ssa{c}}, \exists v, (\ssa{x}, v) \in \ssa{m}$.
    \item $ \ssa{m} \vDash_{ssa} \delta  \triangleq \forall \ssa{x} \in \codom(\delta), \exists v, (\ssa{x},v) \in \ssa{m}$.
\end{enumerate}
\end{defn}
%
The part declared in the translation environment $\delta$ in a ssa memory $\ssa{m}$ can be reverted to corresponding part of the memory $m$ with an inverse of $\delta$ as follows.
%
\begin{defn}[Inverse of Transformed memory]
 $m = \delta^{-1}(\ssa{m}) \triangleq \forall x \in \dom(\delta), (\delta(x), m(x)) \in \ssa{m} $.
\end{defn}
}
%
\begin{lem}[Value Preserving].
\label{lem:same_value}
{
Given $\delta; e \hookrightarrow \ssa{e}$,  $\forall m. m \vDash \delta. \forall \ssa{m}, \ssa{m} \vDash_{ssa} \delta \land m = \delta^{-1}(\ssa{m})$, then $\config{m, e} \to v $ and $\config{
\ssa{m}, \ssa{e}} \to {v}$.
}
\end{lem}

\begin{thm}[Soundness of transformation]
Given $\Sigma; \delta ; c \hookrightarrow \ssa{c} ; \delta';\Sigma' $, $\forall m. m \vDash \delta. \forall \ssa{m}, \ssa{m} \vDash_{ssa} \delta \land m = \delta^{-1}(\ssa{m})$, if there exist an execution of $c$ in the while language, starting with a trace $t$ and loop maps $w$, $\config{m, c, t, w} \to^{*} \config{m', \eskip, t', w' } $,  then there also exists a corresponding execution of $\ssa{c}$ in the ssa language so that 
  $\config{  {\ssa{m}}, \ssa{c}, t, w} \to^{*} \config{{  \ssa{m'}}, \eskip, t', w' } $ and $ m' = \delta'^{-1}(\ssa{m'}) $.
\end{thm}

\begin{proof}
 We assume that $q$ is the same when transformed to $\ssa{q}$, as the primitive in both languages. And the value remains the same during the transformation.  
 It is proved by induction on the transformation rules.
 \begin{itemize}
   \caseL{Case $\inferrule{
 {\Sigma;\delta ; c_1 \hookrightarrow \ssa{c_1} ; \delta_1;\Sigma_1} 
 \and
 {\Sigma_1; \delta_1 ; c_2 \hookrightarrow \ssa{c_2} ; \delta'; \Sigma'} 
}{
\Sigma;\delta ; c_1 ; c_2 \hookrightarrow \ssa{c_1} ; \ssa{c_2} \ ; \delta';\Sigma'
}~{\textbf{S-SEQ}}$}
We choose an arbitrary memory $m$ so that $m \vDash \delta$, we choose a trace $t$ and a loop maps $w$.
\[
\inferrule
{
{\config{m, c_1,  t,w} \xrightarrow{}^{*} \config{m_1, \eskip,  t_1,w_1}}
\and
{\config{m_1, c_2,  t_1,w_1} \xrightarrow{}^{*} \config{m', \eskip,  t',w'}}
}
{
\config{m, c_1; c_2,  t,w} \xrightarrow{}^{*} \config{m', \eskip, t',w'}
}
\]
 We choose the transformed memory ${\ssa{m}} $ so that  $ m =\delta^{-1}(\ssa{m})$.
 
 To show: $ \config{\ssa{ m, c_1;c_2 }, t, w } \xrightarrow{}^{*} \config{\ssa{m', \eskip}, t'. w' }$ and $ m' = \delta'^{-1} (\ssa{m'}) $.
 
 By induction hypothesis on the first premise, we have:
 \[ \config{\ssa{m, c_1}, t,w} \xrightarrow{}^{*} \config{\ssa{m_1, \eskip},t_1,w_1 } \land m_1 = \delta_1^{-1}(\ssa{m_1}) \]
  By induction hypothesis on the second premise, using the conclusion $ m_1 = \delta_1^{-1}(\ssa{m_1}) $.
  We have:
  \[
   \config{\ssa{m_1, c_2}, t_1,w_1} \xrightarrow{}^{*} \config{\ssa{m', \eskip},t',w' } \land m' = \delta'^{-1}(\ssa{m'})
  \]
  So we know that 
  \[
  \inferrule{
  { \config{\ssa{m, c_1}, t,w} \xrightarrow{}^{*} \config{\ssa{m_1, \eskip},t_1,w_1 }  }
  \and
  { \config{\ssa{m_1, c_2}, t_1,w_1} \xrightarrow{}^{*} \config{\ssa{m', \eskip},t',w' } }
  }{
  \config{\ssa{m, c_1;c_2 }, t,w} \xrightarrow{}^{*} \config{\ssa{m', \eskip}, t' , w' }
  }
  \]
 \caseL{Case $\inferrule{
 { \delta ; \expr \hookrightarrow \sexpr}
 \and
 {\delta' = \delta[x \mapsto \ssa{x} ]}
 \and{ \ssa{x} \ fresh(\Sigma) }
 \and {\Sigma' = \Sigma \cup \{x\} }
}{
 \Sigma;\delta ; [\assign x \expr]^{l} \hookrightarrow [\assign {\ssa{x}}{ \ssa{\expr}}]^{l} ; \delta';\Sigma'
}~{\textbf{S-ASSN}} $ }

 We choose an arbitrary memory $m$ so that $m \vDash \delta$, we choose a trace $t$ and a loop maps $w$, we know that the resulting trace is still $t$ from its evaluation rule $\textbf{assn}$ when we suppose $m, \expr \to v$.
 \[
 \inferrule
{
}
{
\config{m, [\assign x v]^{l},  t,w} \xrightarrow{} \config{m[v/x], [\eskip]^{l}, t,w}
}
~\textbf{assn}
 \]
 We choose the transformed memory ${\ssa{m}} $ so that  $ m =\delta^{-1}(\ssa{m})$.
 
 To show: $\config{\ssa{m}, [\assign {\ssa{x}}{ \ssa{\expr}}]^{l} , t, w} \to^{*} \config{\ssa{m'}, \eskip, t, w} $ and $ m' = \delta'^{-1}(\ssa{m'}) $.
 
 From the rule \textbf{ssa-assn}, we assume $\ssa{m}, \ssa{\expr} \to \ssa{v}$, we know that 
 \[
 \inferrule
{
}
{
\config{\ssa{ m, [\assign x v]^{l}},  t,w } \xrightarrow{} \config{\ssa{m[x \mapsto v], [\eskip]^{l}}, t,w }
}
~\textbf{ssa-assn}
 \]
 We also know that $\delta'= \delta[x \mapsto \ssa{x}]$ and $m = \delta^{-1}(\ssa{m})$, $m'= m[v/x]$. We can show that $ m[v/x] = \delta[x \mapsto \ssa{x}]^{-1}(\ssa{m}[\ssa{x} \mapsto v]) $.
 
\caseL{Case $\inferrule{
 {\delta ; q \hookrightarrow \ssa{q}}
 \and
 {\delta ; \expr \hookrightarrow \ssa{\expr}}
 \and
 {\delta' = \delta[x \mapsto \ssa{x} ]}
 \and{ \ssa{x} \ fresh(\Sigma) }
 \and{ \Sigma' = \Sigma \cup \{x\} }
}{
 \Sigma;\delta ; [\assign{x}{\query(\qexpr)}]^{l} \hookrightarrow [\assign {\ssa{x}}{ \ssa{\query(\qexpr)}}]^{l} ; \delta'
}~{\textbf{S-QUERY}}$} 
We choose an arbitrary memory $m$ so that $m \vDash \delta$, we choose a trace $t$ and a loop maps $w$, we know that when we suppose $\config{m, \expr} \to v$.
 \[
\inferrule
{
\query(v)(D) = \qval 
}
{
\config{m, [\assign{x}{\query(v)}]^l, t, w} \xrightarrow{} \config{m[ \qval/ x], \eskip,  t \mathrel{++} [\query(v),l,w )],w }
}
~\textbf{query}
 \]
 We choose the transformed memory ${\ssa{m}} $ so that  $ m =\delta^{-1}(\ssa{m})$.
 
 To show: $\config{\ssa{m}, [\assign {\ssa{x}}{ \ssa{\query(\qexpr)}}]^{l} , t, w} \to^{*} \config{\ssa{m'}, \eskip, t, w} $ and $ m' = \delta'^{-1}(\ssa{m'}) $.
 
 From the rule \textbf{ssa-query}, we know that 
 \[
 \inferrule
{
\ssa{\query(v)(D) = \qval} 
}
{
\config{ \ssa{ m, [\assign{\ssa{x}}{\ssa{\query(\qexpr)}}]^l}, t, w} \xrightarrow{} \config{\ssa{  m[  x \mapsto v], \eskip,}  t \mathrel{++} [(q^{(l,w )},v)],w }
}
~\textbf{ssa-query}
 \]
 We also know that $\delta'= \delta[x \mapsto \ssa{x}]$ and $m = \delta^{-1}(\ssa{m})$, $m'= m[v/x]$. We can show that $ m[v/x] = \delta[x \mapsto \ssa{x}]^{-1}(\ssa{m}[\ssa{x} \mapsto v]) $.

  \caseL{Case $\inferrule{
  { \delta ; \bexpr \hookrightarrow \ssa{\bexpr} }
  \and
  { \Sigma; \delta ; c_1 \hookrightarrow \ssa{c_1} ; \delta_1;\Sigma_1 }
  \and
  {\Sigma_1; \delta ; c_2 \hookrightarrow \ssa{c_2} ; \delta_2 ; \Sigma_2 }
  \\
  {[\bar{x}, \ssa{\bar{{x_1}}, \bar{{x_2}}}] = \delta_1 \bowtie \delta_2  }
  \and
   {[\bar{y}, \ssa{\bar{{y_1}}, \bar{{y_2}}}] = \delta \bowtie \delta_1 / \bar{x} }
  \and
   {[\bar{z}, \ssa{\bar{{z_1}}, \bar{{z_2}}}] = \delta \bowtie \delta_2 / \bar{x} }
  \\
  { \delta' =\delta[\bar{x} \mapsto \ssa{\bar{{x}}'} ][\bar{y} \mapsto \ssa{\bar{{y}}'} ][\bar{z} \mapsto \ssa{\bar{{z}}'} ]}
  \and 
  {\ssa{\bar{{x}}', \bar{y}', \bar{z}'} \ fresh(\Sigma_2)
  }
  \and{\Sigma' = \Sigma_2 \cup \{ \ssa{ \bar{x}', \bar{y}', \bar{z}' } \} }
}{
 \Sigma; \delta ; [\eif(\bexpr, c_1, c_2)]^l  \hookrightarrow [\ssa{ \eif(\bexpr, [\bar{{x}}', \bar{{x_1}}, \bar{{x_2}}] ,[\bar{{y}}', \bar{{y_1}}, \bar{{y_2}}] ,[\bar{{z}}', \bar{{z_1}}, \bar{{z_2}}] , {c_1}, {c_2})}]^l; \delta';\Sigma'
}~{\textbf{S-IF}}$}
We choose an arbitrary memory $m$ so that $m \vDash \delta$, we choose a trace $t$ and a loop maps $w$.
There are two possible evaluation rules depending on the the condition $b$, we choose the case when $b = \etrue$, we know there is an execution in ssa language so that $\ssa{\bexpr} = \etrue$, we use the rule $\textbf{if-t}$.  
 \[\inferrule
{
}
{
\config{m, [\eif(\etrue, c_1, c_2)]^{l},t,w} 
\xrightarrow{} \config{m, c_1,  t,w} \to^{*} \config{m', \eskip, t', w'}
}
\]
 We choose the transformed memory ${\ssa{m}} $ so that  $ m =\delta^{-1}(\ssa{m})$.
 
 To show: $\config{\ssa{m}, [\eif(\etrue, [\bar{\ssa{x}}', \bar{\ssa{x_1}}, \bar{\ssa{x_2}}] ,[\bar{\ssa{y}}', \bar{\ssa{y_1}}, \bar{\ssa{y_2}}] ,[\bar{\ssa{z}}', \bar{\ssa{z_1}}, \bar{\ssa{z_2}}] , c_1, c_2)]^{l}, t, w} \to^{*} \config{\ssa{m'}, \eskip, t', w'} $ and $ m' = \delta'^{-1}(\ssa{m'}) $.

We use the corresponding rule $\textbf{SSA-IF-T}$.  
\[
\inferrule
{
}
{
\config{\ssa{ { m} , [\eif(\etrue, [\bar{\ssa{x}}', \bar{\ssa{x_1}}, \bar{\ssa{x_2}}] , [\bar{\ssa{y}}', \bar{\ssa{y_1}}, \bar{\ssa{y_2}}] ,[\bar{\ssa{z}}', \bar{\ssa{z_1}}, \bar{\ssa{z_2}}] , \ssa{c_1, c_2})]^{l}},t,w} 
\xrightarrow{} \\ \config{\ssa{ m, c_1}; \eifvar(\ssa{\bar{x}', \bar{x_1}});\eifvar(\ssa{\bar{y}', \bar{y_2}});\eifvar(\ssa{\bar{z}', \bar{z_1}}),  t,w } 
}~\textbf{ssa-if-t}
\]
By induction hypothesis on $ \Sigma;\delta ; c_1 \hookrightarrow  \ssa{c_1}; \delta_1;\Sigma_1$, and we know that $\config{m, c_1,  t,w} \to^{*} \config{m', \eskip, t', w'} $, from our assumption that $ m =\delta^{-1}(\ssa{m})$, we know that 
\[\config{\ssa{ { m}, c_1},  t,w} \to^{*} \config{ \ssa{ { m_1 }, \eskip,} t', w' } \land m' = \delta_1^{-1}(\ssa{m_1}) \]
and we then have:
\[
\inferrule
{
  \config{\ssa{ { m}, c_1},  t,w} \to^{*} \config{ \ssa{ { m_1 }, \eskip,} t', w' }
}
{
 \config{\ssa{  m, c_1;} \eifvar(\ssa{\bar{x}', \bar{x_1})};\eifvar(\ssa{\bar{y}', \bar{y_1})};\eifvar(\ssa{\bar{z}', \bar{z_1})},  t,w  }  \to^{*}
 \config{\ssa{ { m_1 [ \bar{x}' \mapsto {m_1}(\bar{x_1}),\bar{y}' \mapsto {m_1}(\bar{y_2}),\bar{z}' \mapsto {m_1}(\bar{z_1}) ] }, \eskip}, t', w'  }
}
\]
Now, we want to show that $ m' = \delta[\bar{x} \mapsto \ssa{\bar{x}'},\bar{y} \mapsto \ssa{\bar{y}'},\bar{z} \mapsto \ssa{\bar{z}'} ]^{-1}(\ssa{ m_1 [ \bar{x}' \mapsto {m_1}(\bar{x_1}),\bar{y}' \mapsto {m_1}(\bar{y_2}),\bar{z}' \mapsto {m_1}(\bar{z_1}) ] }) $.

Unfold the definition, we want to show that $$\forall x  \in ( \dom(\delta) \cup \bar{x} \cup \bar{y} \cup \bar{z} ), (\delta[\bar{x} \mapsto \ssa{\bar{x}'},\bar{y} \mapsto \ssa{\bar{y}'},\bar{z} \mapsto \ssa{\bar{z}'} ](x), m'(x)) \in \ssa{m_1 [ \bar{x}' \mapsto {m_1}(\bar{x_1}),\bar{y}' \mapsto {m_1}(\bar{y_2}),\bar{z}' \mapsto {m_1}(\bar{z_1}) ] } .$$
\begin{enumerate}
    \item For variable $x$ in $\bar{x}$, we can find a corresponding ssa variable $\ssa{x} \in \ssa{\bar{x}'}$, so that $( \ssa{x}, m'(x) ) \in \ssa{ m_1 [\bar{x}' \mapsto m_1(\bar{x_1})] } $. It is because we know $[x \mapsto \ssa{x_1}]$ for certain $\ssa{x_1} \in \ssa{\bar{x_1}}$ in $\delta_1$, then by unfolding  $m' = \delta_1^{-1}(\ssa{m_1})$ and $\ssa{\bar{x_1}} \in \codom(\delta_1)$, we know $(\ssa{x_1}, m'(x)) \in \ssa{m_1}$ so that $m'(x) = \ssa{m_1}(\ssa{x_1})$.
    \item For variable $y \in \bar{y}$, we know that $y \in \dom(\delta_1)$, then $[ y \mapsto \ssa{y_2} ]$ for certain $\ssa{y_2} \in \ssa{\bar{y_2}}$ in $\delta_1$.  So we know that $(\delta_1(y), m'(y) ) \in \ssa{m_1}$, and then $m'(y) = \ssa{m_1(y_2)}$. We can show $(\ssa{y}, m'(y)) \in \ssa{m_1[\bar{y}' \mapsto m_1(\bar{y_2})]}$.
    \item For variable $z \in \bar{z}$, we know that $z \not\in \dom(\delta_1)$ by the definition (otherwise $z$ will appear in $\bar{x}$), then $[ z \mapsto \ssa{z_1} ]$ for certain $ \ssa{z_1} \in \ssa{\bar{z_1}}$ in $\delta$.  We know $(\delta(z), m(z)) \in \ssa{m}$ from our assumption, so we have $ m(z) = \ssa{m(z_1)}$. Because $z$ is not modified in $c_1$, so that $m(z) = m'(z)$. Also $\ssa{m}$ will not shrink during execution and $\ssa{z_1}$ will not be written in $\ssa{c_1}$, so $(\ssa{z_1}, m'(z)) \in \ssa{m_1}$. Then we can show that $ (\ssa{z}, m'(z) ) \in \ssa{m_1[\bar{z}' \mapsto m_1(\bar{z_1})] }$.
    \item For variable $k \in \dom(\delta)- \bar{x} - \bar{y}-\bar{z}$. From our assumption $ m = \delta^{-1}(\ssa{m})$, we can show $(\delta(k), m(k) ) \in \ssa{m}$. We know that $k$ is not written in either branch from our definition, so $(\delta(k), m'(k) ) \in \ssa{m_1} $ .
\end{enumerate}

{
\caseL{
	Case
	$
	\inferrule{
    { \Sigma; \delta ; c \hookrightarrow \ssa{c_1} ; \delta_1; \Sigma_1 }
     \and
    { [ \bar{x}, \ssa{\bar{{x_1}}}, \ssa{\bar{{x_2}}} ] = \delta \bowtie \delta_1 }
    \\
     {\ssa{\bar{{x}}'} \ fresh(\Sigma_1 )}
    \and {\delta' = \delta[\bar{x} \mapsto \ssa{\bar{{x}}'}]}
    \and 
     {\delta' ; \bexpr \hookrightarrow \ssa{\bexpr} }
     \and
    {\ssa{c' = c_1[\bar{x}'/ \bar{x_1}]   } }
    % \and{ \delta' ; c \hookrightarrow \ssa{c'} ; \delta'' }
  }{ 
  \Sigma; \delta ;  \ewhile ~ [\bexpr]^{l} ~ \edo ~ c 
  \hookrightarrow 
  \ssa{\ewhile ~ [\bexpr]^{l}, 0, [\bar{{x}}', \bar{{x_1}}, \bar{{x_2}}] ~ \edo ~ {c} } ; \delta'; \Sigma_1 \cup \{\ssa{\bar{x}'}  \}
}~{\textbf{S-WHILE}}
$
}
}
\\
{
We choose an arbitrary memory $m$ so that $m \vDash \delta$, we choose a trace $t$ and a loop maps $w$. Suppose $ \config{m ,a} \to v_N $. There are two cases, when $v_N=0$, the loop body is not executed so we can easily show that the trace is not modified.
%
When the while loop is still running ($v_N > 0$), we have the following evaluation in the while language:
\[
\inferrule
{
 \empty
}
{
\config{
m, \ewhile ~ [b]^{l} ~ \edo [c]^{l + 1},  t, w 
}
\xrightarrow{} \config{m, c ; 
\eif_w (b, c ; 
\ewhile ~ [b]^{l} ~ \edo [c]^{l + 1},  \eskip),
t, w }
}
~\textbf{while-b}
\]
which follows by the following evaluation:
\[
	\inferrule
{
 m, b \xrightarrow{} b'
}
{
\config{m, \eif_w (b, c,  \eskip) ,  t, w }
\xrightarrow{} \config{m, 
 \eif_w (b', c,  \eskip), t, w }
}
~\textbf{ifw-b}
\]
In the corresponding ssa-form language, we have the corresponding evaluation in the same way by assuming 
$m = \delta^{-1}(\ssa{m})$.
%
\[
	\inferrule
{
 {n = 0 \rightarrow i = 1 }
 \and
 {n > 0 \rightarrow i = 2 }
}
{
\config{
\ssa{m},  
\ssa{\ewhile ~ [\bexpr]^{l}, n, 
[\bar{{x}}', \bar{{x_1}}, \bar{{x_2}}] 
~ \edo ~ {c} 
},  t, w 
}
\xrightarrow{} \\ 
\config{
\ssa{m},
\eif_w 
(\ssa{b[\bar{x_i}/\bar{x'}], [\bar{{x}}', \bar{{x_1}}, \bar{{x_2}}], n,  c[\bar{x_i}/\bar{x'}] }; 
\ssa{
\ewhile ~ [b]^{l}, n+1, 
[\bar{{x}}', \bar{{x_1}}, \bar{{x_2}}]  
~ \edo ~ c} ,  \eskip),
t, w
}
}
~\textbf{ssa-while-b}
\]
This evaluation is followed by the following evaluation:
\[
	\inferrule
{
 \ssa{m, b \xrightarrow{} b'}
}
{
\config{\ssa{m, \eif_w (b, [\bar{{x}}', \bar{{x_1}}, \bar{{x_2}}] , n,  c_1,  c_2)} ,  t, w }
\xrightarrow{} \config{\ssa{ m, 
 \eif_w (b', [\bar{{x}}', \bar{{x_1}}, \bar{{x_2}}] , n , c_1 , c_2 )}, t, w }
}
~\textbf{ssa-ifw-b}
\]
%
Depending on if the counter $n$ is equal to $0$ or not, there are two possible execution paths (the variables $\ssa{\bar{x}}$ is replaced by the $\ssa{\bar{x_1}}$ or $\ssa{\bar{x_2}}$). We start from the first iteration (when $n =0$) when $v_N >0$. 
}
{
By induction hypothsis on the premise $ { \Sigma; \delta ; c \hookrightarrow \ssa{c_1} ; \delta_1; \Sigma_1 }$, we know that 
\[ \config{\ssa{{m}, c'[ \bar{x_1}/\bar{x}'  ]}, t, (w+l)  } \to^{*} \config{\ssa{{m'}, \eskip}, t'_{i}, w'  } \land m' = \delta_1^{-1}(\ssa{m'})   \]
Hence we can conclude that:
\[
  \inferrule{
   \config{\ssa{{m}, c'[ \bar{x_1}/\bar{x}'  ]}, t, (w+l) }  \to^{*} \config{\ssa{{m'}, \eskip}, t'_{1}, w'  }
  }{
  \config{\ssa{ {m}, c'[ \bar{x_1}/\bar{x}'  ];  [\eloop ~ (\valr_N-1), n+1, [\bar{\ssa{x}}', \bar{\ssa{x_1}}, \bar{\ssa{x_2}}] ~  \edo ~ c' ]^{l} },  t, (w + l)  }  \to^{*} \\ \config{ \ssa{{m'}, [\eloop ~ (\valr_N-1), n+1, [\bar{\ssa{x}}', \bar{\ssa{x_1}}, \bar{\ssa{x_2}}] ~  \edo ~ c' ]^{l}}, t'_{1}, w'  } 
  }
\]
%
Then there are two cases, 
%
\begin{enumerate}
     \item  when guard in the $\eif_w$ expression evaluates to $\efalse$, the while loop terminates and exits.
     The execution in the while language is defined in the evaluation rule $\textbf{ifw-false}$ as follows.
     \[
		\inferrule
		{
		 \empty
		}
		{
		\config{\ssa{
		m, \eif_w (
		\efalse, [\bar{{x}}', \bar{{x_1}}, \bar{{x_2}}],   n, 
		c; {\ewhile ~ [b]^{l} ~ \edo ~ c},
		\eskip)
		)} ,  t, w }
		\\
		\xrightarrow{} 
		\config{\ssa{m, 
		{\eskip}; \eifvar(\bar{x'}, \bar{x_i}) }, t, (w - l) }
		}
		~\textbf{ifw-false}
	\]
%
	The corresponding ssa-form evaluation as follows:
	\[
		\inferrule
		{
		 { n = 0 \rightarrow i = 1 }
		 \and
		 {n > 0 \rightarrow i =2}
		}
		{
		\config{\ssa{
		m, \eif_w (
		\efalse, [\bar{{x}}', \bar{{x_1}}, \bar{{x_2}}],   n, 
		{  
		c; \ssa{\ewhile ~ [b]^{l}, n, [\bar{{x}}', \bar{{x_1}}, \bar{{x_2}}]  ~ \edo ~ c},
		\eskip)
		} 
		)} ,  t, w }
		\\
		\xrightarrow{} 
		\config{\ssa{m, 
		{\eskip}; \eifvar(\bar{x'}, \bar{x_i}) }, t, (w - l) }
		}
		~\textbf{ssa-ifw-false}
	\]
	We can see that both traces are not changed during the exit of the while. We need to show that $ m' = \delta^{-1} (\ssa{m'[\bar{x} \mapsto m'(\bar{x_2})]}) $. We know that $[ \bar{x} \mapsto \bar{x_2}]$ in $\delta_1$ from the definition, so we can show that for any variable $\ssa{x_2} \in \bar{x_2}$, $( \ssa{x_2}, m'(x) ) \in \ssa{m'}$. For variables $x \in {\dom(\delta) - \bar{x} } $, the variable is not modified during the execution of $c$ so that we know $m(x) = m'(x)$, and then we can show that $(\delta(x), m'(x)) \in \ssa{m'} $ because $\delta(x)$ is not written in $\ssa{c'[\bar{x_1}/ \bar{x}']}$ .
%
  	\item 
		when guard in the $\eif_w$ expression evaluates to $\etrue$, the while terminates and exits.
     The execution in the while language is defined in the evaluation rule $\textbf{ifw-true}$.
          %
     We want to show that : assuming in the $i-th$ $(i < \ssa{n})$ iteration, starting with $t_i$ and $w_i$ and $m_i = \delta_1^{-1}(\ssa{m_i})$,
     this command is evaluated according to the while language operation semantics as
     	$
		\config{m, \eif_w (\etrue, c ; \ewhile ~ [b]^{l} ~ \edo ~ c, ,  \eskip) ,  t, w }
		\xrightarrow{}^* \config{m, c 
		t, (w + l) }
 		$.
     %
     Then the corresponding ssa form evaluation as follows : 
     %
     \[ 
     \inferrule{}{
     	\config{
		\ssa{
			m, 
			{
			\eif_w (\etrue, [\bar{{x}}', \bar{{x_1}}, \bar{{x_2}}], n,  
			c; \ssa{\ewhile ~ [b]^{l}, n, [\bar{{x}}', \bar{{x_1}}, \bar{{x_2}}]  ~ \edo ~ c},
			\eskip)
			} 
		},  t, w 
		}
		\\
		\xrightarrow{} 
		\config{
		\ssa{m, 
		{
		\eif_w (\etrue, [\bar{{x}}', \bar{{x_1}}, \bar{{x_2}}], n,  
		c; \ssa{\ewhile ~ [b]^{l}, n, [\bar{{x}}', \bar{{x_1}}, \bar{{x_2}}]  ~ \edo ~ c},
		}
		}
		t, (w + l) }
		} 
     \]  
     and $m_i = \delta^{-1}(\ssa{m_i}) $.
     We then have the evaluation in the while language:
     \[
		\inferrule
		{
		 \empty
		}
		{
		\config{m, 
		\eif_w (b, 
		c ; \ewhile ~ [b]^{l} ~ \edo ~ c, 
		\eskip),
		t, w }
		\xrightarrow{} 
		\config{m, 
		c ; \ewhile ~ [b]^{l} ~ \edo ~ c,  
		t, (w + l) }
		}
		~\textbf{ifw-true}
	\]
	We then have the following evaluation:
	\[
		\inferrule
		{
		 \empty
		}
		{
		\config{
		\ssa{
		m, 
		{
		\eif_w (\etrue, [\bar{{x}}', \bar{{x_1}}, \bar{{x_2}}], n,  
		c; \ssa{\ewhile ~ [b]^{l}, n, [\bar{{x}}', \bar{{x_1}}, \bar{{x_2}}]  ~ \edo ~ c},
		\eskip)
		} 
		},  t, w 
		}
		\\
		\xrightarrow{} 
		\config{
		\ssa{m, 
		{
		\eif_w (\etrue, [\bar{{x}}', \bar{{x_1}}, \bar{{x_2}}], n,  
		c; \ssa{\ewhile ~ [b]^{l}, n, [\bar{{x}}', \bar{{x_1}}, \bar{{x_2}}]  ~ \edo ~ c},
		}
		}
		t, (w + l) }
		}
		~\textbf{ssa-ifw-true}
	\]
%
By induction hypothsis on the premise $  { \Sigma; \delta_1 ; c \hookrightarrow \ssa{c_2} ; \delta_1; \Sigma_1 }$, we know that
%
\[
\config{\ssa{{m_i}, c'[ \bar{x_2}/\bar{x}'  ]}, t_i, (w_i+l)  } \to^{*} \config{\ssa{{m_{i+1}}, \eskip}, t_{i+1}, w_{i+1}  } \land m_{i+1} = \delta_1^{-1}(\ssa{m_{i+1}})
\]
%
Hence we can conclude that:
\[
  \inferrule{
   \config{\ssa{{m_i}, c'[ \bar{x_2}/\bar{x}'  ]}, t_i, (w_i+l) }  \to^{*} \config{\ssa{{m_{i+1}}, \eskip}, t_{i+1}, w_{i+1}  }
  }{
  \config{\ssa{ {m_i}, c'[ \bar{x_2}/\bar{x}'  ];  [\eloop ~ (\valr_N-i-1), n+1, [\bar{\ssa{x}}', \bar{\ssa{x_1}}, \bar{\ssa{x_2}}] ~  \edo ~ c' ]^{l} },  t_i, (w_i + l)  }  \to^{*} \\ \config{ \ssa{{m_{i+1}}, [\eloop ~ (\valr_N-i-1), n+1, [\bar{\ssa{x}}', \bar{\ssa{x_1}}, \bar{\ssa{x_2}}] ~  \edo ~ c' ]^{l}}, t_{i+1}, w_{i+1}  } 
  }
\]
So we can show that before the exit of the loop after ($v_N= n $) iterations, we have $t_{n} = t_{n}$ and $m_{n} = \delta_1^{-1}(\ssa{m_{n}})$.
 \end{enumerate}
%
This proof is similar when it comes to the exit as in case 1. 
}
\end{itemize}
%
\end{proof}
%
\clearpage
%
\clearpage