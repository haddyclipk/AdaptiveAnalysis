\begin{lem}[Uniqueness of the Labeled Variables]
    % \label{lem:lvar_unique}
    For every program $c \in \cdom$ and every two labeled variables such that
    $x^i, y^j \in \lvar(c)$, then $x^i \neq y^j$.
    \[
      \forall c \in \cdom, x^i, y^j \in \mathcal{L} \st x^i, y^j \in \lvar(c)\implies x^i \neq y^j.
      \]
  \end{lem}
  \begin{proof}
  \end{proof}
  \begin{lem}
    [Trace Non-Decreasing]
    % \label{lem:tracenondec}
    For every program $c \in \cdom$ and traces $\trace, \trace' \in \mathcal{T}$, if 
    $\config{c, \trace} \rightarrow^{*} \config{\eskip, \trace'}$,
    then there exists a trace $\trace'' \in \mathcal{T}$ with $\trace \tracecat \trace'' = \trace'$
    %
    $$
    \forall \trace, \trace' \in \mathcal{T}, c \st
    \config{c, \trace} \rightarrow^{*} \config{\eskip, \trace'} 
    \implies \exists \trace'' \in \mathcal{T} \st \trace \tracecat \trace'' = \trace'
    $$
    \end{lem}
    \begin{proof}
      Taking arbitrary trace $\trace \in \mathcal{T}$, by induction on program $c$, we have the following cases:
      \caseL{$c = [\assign{x}{\expr}]^{l}$}
      By the evaluation rule $\rname{assn}$, we have
      $
      {
      \config{[\assign{{x}}{\aexpr}]^{l},  \trace } 
      \xrightarrow{} 
      \config{\eskip, \trace :: ({x}, l, v, \bullet)}
      }$, for some $v \in \mathbb{N}$.
      \\
      Picking $\trace' = \trace ::({x}, l, v, \bullet)$ and $\trace'' =  [({x}, l, v, \bullet) ]$,
      it is obvious that $\trace \tracecat \trace'' = \trace'$.
      % \\
      % There are 2 cases, where $l' = l$ and $l' \neq l$.
      % \\
      % In case of $l' \neq l$, we know $\event \not\eventin \trace$, then this Lemma is vacuously true.
      %   \\
      %   In case of $l' = l$, by the abstract Execution Trace computation, we know 
      %   $\absflow(c) = \absflow'([x := \expr]^{l}; \clabel{\eskip}^{l_e}) = \{(l, \absexpr(\expr), l_e)\}$  
      %   \\
      % Then we have $\absevent = (l, \absexpr(\expr), l_e) $ and $\absevent \in \absflow(c)$.
      \\
      This case is proved.
      \caseL{$c = [\assign{x}{\query(\qexpr)}]^{l'}$}
      This case is proved in the same way as \textbf{case: $c = [\assign{x}{\expr}]^l$}.
      \caseL{$\ewhile [b]^{l_w} \edo c$}
      By the first rule applied to $c$, there are two cases:
      \subcaseL{$\textbf{while-t}$}
      If the first rule applied to is $\rname{while-t}$, we have
      \\
      $\config{{\ewhile [b]^{l_w} \edo c_w, \trace}}
        \xrightarrow{} 
        \config{{
        c_w; \ewhile [b]^{l_w} \edo c_w,
        \trace :: (b, l_w, \etrue, \bullet)}}~ (1)
      $.
      \\
      Let $\trace_w' \in \mathcal{T}$ be the trace satisfying following execution:
      \\
      $
      \config{{
      c_w,
      \trace :: (b, l_w, \etrue, \bullet)}}
      \xrightarrow{*} 
      \config{{
      \eskip, \trace_w'}}
    $
    \\
    By induction hypothesis on sub program $c_w$ with starting trace 
    $\trace :: (b, l_w, \etrue, \bullet)$ and ending trace $\trace_w'$, 
    we know there exist
    $\trace_w \in \mathcal{T}$ such that $\trace_w' = \trace :: (b, l_w, \etrue, \bullet) \tracecat \trace_w$.
    \\
    Then we have the following execution continued from $(1)$:
    \\
    $
    \config{{\ewhile [b]^{l_w} \edo c_w, \trace}}
        \xrightarrow{} 
        \config{{
        c_w; \ewhile [b]^{l_w} \edo c_w,
        \trace :: (b, l_w, \etrue, \bullet)}}
        \xrightarrow{*} 
        \config{\ewhile [b]^{l_w} \edo c_w, \trace :: (b, l_w, \etrue, \bullet) \tracecat \trace_w}
        ~(2)
    $
    By repeating the execution (1) and (2) until the program is evaluated into $\eskip$,
    with trace $\trace_w^{i'} $ for $i = 1, \cdots, n n \geq 1$ in each iteration, we know 
    in the $i-th$ iteration,
     there exists  $\trace_w^i \in \mathcal{T}$ such that  
    $\trace_w^{i'} = \trace_w^{(i-1)'} :: (b, l_w, \etrue, \bullet) ++ \trace_w^{i'}$
    \\
    Then we have the following execution:
    \\
    $
    \config{{\ewhile [b]^{l_w} \edo c_w, \trace}}
        \xrightarrow{} 
        \config{{
        c_w; \ewhile [b]^{l_w} \edo c_w,
        \trace :: (b, l_w, \etrue, \bullet)}}
        \xrightarrow{*} 
        \config{\ewhile [b]^{l_w} \edo c_w, \trace_w^{n'}}
        \xrightarrow{}^\rname{{while-f}}
        \config{\eskip, \trace_w^{n'}:: (b, l_w, \efalse, \bullet)}
    $ and $\trace_w^{n'} = \trace :: (b, l_w, \etrue, \bullet) \tracecat \trace_w^{1} :: \cdots :: (b, l_w, \etrue, \bullet) \tracecat \trace_w^{n} $.
    \\
    Picking $\trace' = \trace_w^{n'} :: (b, l_w, \efalse, \bullet)$ and $\trace'' = [(b, l_w, \etrue, \bullet)] \tracecat \trace_w^{1} :: \cdots :: (b, l_w, \etrue, \bullet) \tracecat \trace_w^{n}$,
    we have 
    $\trace ++ \trace'' = \trace'$.
    \\
    This case is proved.
      \subcaseL{$\textbf{while-f}$}
      If the first rule applied to $c$ is $\rname{while-f}$, we have
      \\
      $
      {
        \config{{\ewhile [b]^{l_w} \edo c_w, \trace}}
        \xrightarrow{}^\rname{while-f}
        \config{{
        \eskip,
        \trace :: (b, l_w, \efalse, \bullet)}}
      }$,
      In this case, picking $\trace' = \trace ::(b, l_w, \efalse, \bullet)$ and $\trace'' =  [(b, l_w, \efalse, \bullet) ]$,
      it is obvious that $\trace \tracecat \trace'' = \trace'$.
      \\
      This case is proved.
      \caseL{$\eif([b]^l, c_t, c_f)$}
      This case is proved in the same way as \textbf{case: $c = \ewhile [b]^{l} \edo c$}.
      \caseL{$c = c_{s1};c_{s2}$}
     By the induction hypothesis on $c_{s1}$ and $c_{s2}$ separately,
     we have this case proved.
    \end{proof}
    %
    % \todo{more explanation}
    % \mg{This corollary needs some explanation. In particular, we should stress that $\event$ and $\event'$ may differ in the query value.}
    \begin{coro}
    % \label{coro:aqintrace}
    For every event and a trace $\trace \in \mathcal{T}$,
    if $\event \in \trace$, 
    then there exist another event $\event' \in \eventset$ and traces $\trace_1, \trace_2 \in \mathcal{T}$
    such that $\trace_1 \tracecat [\event'] \tracecat \trace_2 = \trace $
    with 
    $\event$ and $\event'$ equivalent but may differ in their query value.
    \[
      \forall \event \in \eventset, \trace \in \mathcal{T} \st
    \event \in \trace \implies \exists \trace_1, \trace_2 \in \mathcal{T}, 
    \event' \in \eventset \st (\event \in \event') \land \trace_1 \tracecat [\event'] \tracecat \trace_2 = \trace  
    \]
    \end{coro}
    \begin{proof}
    % Proof in File: {\tt ``coro\_aqintrace.tex''}
    % \begin{coro}
\label{coro:aqintrace}
\[
\aq \aqin t \implies \exists t_1, t_2, \aq'. ~ s.t., ~ (\aq \aqeq \aq') \land t_1 ++ [\aq'] ++ t_2 = t	
\]
\end{coro}
\begin{subproof}
Proof in File: {\tt ``coro\_aqintrace.tex''}
% \begin{coro}
\label{coro:aqintrace}
\[
\aq \aqin t \implies \exists t_1, t_2, \aq'. ~ s.t., ~ (\aq \aqeq \aq') \land t_1 ++ [\aq'] ++ t_2 = t	
\]
\end{coro}
\begin{subproof}
Proof in File: {\tt ``coro\_aqintrace.tex''}
% \begin{coro}
\label{coro:aqintrace}
\[
\aq \aqin t \implies \exists t_1, t_2, \aq'. ~ s.t., ~ (\aq \aqeq \aq') \land t_1 ++ [\aq'] ++ t_2 = t	
\]
\end{coro}
\begin{subproof}
Proof in File: {\tt ``coro\_aqintrace.tex''}
% \input{coro_aqintrace}
By unfolding the $\aq \aqin t$, we have the following cases:
%
\caseL{$t = []$} The hypothesis is $\efalse$, this case is proved.
%
\caseL{$t = \aq'::t' \land \aq' \aqeq \aq $}
%
Let $t_1 = []$ and $t_2 = t'$, by unfolding the list concatenation operation, we have:
%
\[
	t_1 ++ [\aq'] ++ t_2 = [] ++ [\aq'] ++ t' = t
\]
%
Since $\aq' \aqeq \aq$ by the hypothesis, this case is proved.
%
\caseL{$t = \aq'::t' \land \aq' \aqneq \aq $}
%
By induction hypothesis on $\aq \aqin t'$, we know:
%
\[
	\exists t_1', t_2', \aq''. ~ s.t., ~ (\aq \aqeq \aq'') \land t_1' ++ [\aq''] ++ t_2' = t'	
\]
%
Let $t_1 = \aq'::t_1'$ and $t_2 = t_2'$, by unfolding the list concatenation operation, we have:
%
\[
	t_1 ++ [\aq''] ++ t_2 = (\aq':: t_1') ++ [\aq''] ++ t_2' = \aq'::t' = t
\]
%
Since $\aq'' \aqeq \aq$ by the hypothesis, this case is proved.
%
\end{subproof}
By unfolding the $\aq \aqin t$, we have the following cases:
%
\caseL{$t = []$} The hypothesis is $\efalse$, this case is proved.
%
\caseL{$t = \aq'::t' \land \aq' \aqeq \aq $}
%
Let $t_1 = []$ and $t_2 = t'$, by unfolding the list concatenation operation, we have:
%
\[
	t_1 ++ [\aq'] ++ t_2 = [] ++ [\aq'] ++ t' = t
\]
%
Since $\aq' \aqeq \aq$ by the hypothesis, this case is proved.
%
\caseL{$t = \aq'::t' \land \aq' \aqneq \aq $}
%
By induction hypothesis on $\aq \aqin t'$, we know:
%
\[
	\exists t_1', t_2', \aq''. ~ s.t., ~ (\aq \aqeq \aq'') \land t_1' ++ [\aq''] ++ t_2' = t'	
\]
%
Let $t_1 = \aq'::t_1'$ and $t_2 = t_2'$, by unfolding the list concatenation operation, we have:
%
\[
	t_1 ++ [\aq''] ++ t_2 = (\aq':: t_1') ++ [\aq''] ++ t_2' = \aq'::t' = t
\]
%
Since $\aq'' \aqeq \aq$ by the hypothesis, this case is proved.
%
\end{subproof}
By unfolding the $\aq \aqin t$, we have the following cases:
%
\caseL{$t = []$} The hypothesis is $\efalse$, this case is proved.
%
\caseL{$t = \aq'::t' \land \aq' \aqeq \aq $}
%
Let $t_1 = []$ and $t_2 = t'$, by unfolding the list concatenation operation, we have:
%
\[
	t_1 ++ [\aq'] ++ t_2 = [] ++ [\aq'] ++ t' = t
\]
%
Since $\aq' \aqeq \aq$ by the hypothesis, this case is proved.
%
\caseL{$t = \aq'::t' \land \aq' \aqneq \aq $}
%
By induction hypothesis on $\aq \aqin t'$, we know:
%
\[
	\exists t_1', t_2', \aq''. ~ s.t., ~ (\aq \aqeq \aq'') \land t_1' ++ [\aq''] ++ t_2' = t'	
\]
%
Let $t_1 = \aq'::t_1'$ and $t_2 = t_2'$, by unfolding the list concatenation operation, we have:
%
\[
	t_1 ++ [\aq''] ++ t_2 = (\aq':: t_1') ++ [\aq''] ++ t_2' = \aq'::t' = t
\]
%
Since $\aq'' \aqeq \aq$ by the hypothesis, this case is proved.
%
\end{subproof}
    By unfolding the $\aq \aqin t$, we have the following cases:
    %
    \caseL{$t = []$} The hypothesis is $\efalse$, this case is proved.
    %
    \caseL{$t = \aq'::t' \land \aq' \aqeq \aq $}
    %
    Let $t_1 = []$ and $t_2 = t'$, by unfolding the list concatenation operation, we have:
    %
    \[
        t_1 ++ [\aq'] ++ t_2 = [] ++ [\aq'] ++ t' = t
    \]
    %
    Since $\aq' \aqeq \aq$ by the hypothesis, this case is proved.
    %
    \caseL{$t = \aq'::t' \land \aq' \aqneq \aq $}
    %
    By induction hypothesis on $\aq \aqin t'$, we know:
    %
    \[
        \exists t_1', t_2', \aq''. ~ s.t., ~ (\aq \aqeq \aq'') \land t_1' ++ [\aq''] ++ t_2' = t'	
    \]
    %
    Let $t_1 = \aq'::t_1'$ and $t_2 = t_2'$, by unfolding the list concatenation operation, we have:
    %
    \[
        t_1 ++ [\aq''] ++ t_2 = (\aq':: t_1') ++ [\aq''] ++ t_2' = \aq'::t' = t
    \]
    %
    Since $\aq'' \aqeq \aq$ by the hypothesis, this case is proved.    %
    \end{proof}