\documentclass[a4paper,11pt]{article}
\usepackage[table]{xcolor}



%Packages
\usepackage[T1]{fontenc}
\usepackage{fourier} 
\usepackage[english]{babel} 
\usepackage{amsmath,amsfonts,amsthm} 
\usepackage{lscape}
\usepackage{geometry}
\usepackage{amsmath}
\usepackage{algorithm}
\usepackage{algorithmic}
\usepackage{amssymb}
\usepackage{amsfonts}
\usepackage{times}
\usepackage{bm}
\usepackage{ stmaryrd }
\usepackage{ amssymb }
\usepackage{ textcomp }
\usepackage[normalem]{ulem}
% For derivation rules
\usepackage{mathpartir}
\usepackage{color}
\usepackage{a4wide}


\newcommand{\sctag}[1]{\tag{\textsc{#1}}\label{eq:#1}}
\newcommand{\ands}{~\wedge~}

%Variations
\newcommand{\astable}{\mathbb{S}}
\newcommand{\achange}{U}

%Index Terms
\newcommand{\scond}[3]{(\eif\im{#1}\ethen\im{#2}\eelse\im{#3})}
\newcommand{\keps}[2]{\epsilon(#1,#2)}
\newcommand{\spower}[2]{#1^{#2}}
\newcommand{\ssum}[4]{\sum\limits_{#1=#2}^{#3}#4}
\newcommand{\smin}[2]{\kw{min}({#1},{#2})}
\newcommand{\smax}[2]{\kw{max}(#1,#2)}
\newcommand{\smaxx}[3]{\kw{max}(#1,#2,#3)}
\newcommand{\smaxxx}[4]{\kw{max}(#1,#2,#3,#4)}

%Sizes
\newcommand{\szero}{0}
\newcommand{\sone}{1}
\newcommand{\splus}[2]{#1 + #2}
\newcommand{\ssucc}[1]{#1 {+} \sone}
\newcommand{\sminus}[2]{#1 - #2}
\newcommand{\sdiv}[2]{\frac{#1}{#2}}
\newcommand{\smult}[2]{#1\cdot#2}
\newcommand{\splusone}[1]{#1+1}
\newcommand{\sceil}[1]{\ceil*{#1}}
\newcommand{\sfloor}[1]{\floor*{#1}}
\newcommand{\size}[1]{|#1|}
\newcommand{\slog}[1]{\kw{log}_2(#1)}
\newcommand{\sinf}{\infty}

%Sorts
\newcommand{\ssize}{\mathbb{N}}
\newcommand{\svar}{\mathbb{V}}
\newcommand{\scost}{\mathbb{R}}
\newcommand{\sfun}[2]{#1\mbox{\ra} #2}
\newcommand{\sfunmon}[2]{#1\xrightarrow{\mbox{mon}} #2}
% \newcommand{\sort}{\varsigma}
\newcommand{\sort}{S}
\newcommand{\sorted}[1]{#1 \mathrel{::} \sort}
\newcommand{\sized}[1]{#1 \mathrel{::} \ssize}

% Types
\newcommand{\grt}{A}
\newcommand{\lbound}{\mathop{\uparrow}}
\newcommand{\tbool}{\mbox{bool}}
\newcommand{\trbool}{\mbox{bool}_r}
\newcommand{\tubool}{\mbox{bool}_u}
\newcommand{\trint}{\mbox{int}_r}
\newcommand{\tint}{\mbox{int}}
\newcommand{\tquery}{\mbox{query}}
\newcommand{\tunit}{\mbox{unit}}
\newcommand{\trunit}{\mbox{unit}_r}
\newcommand{\tlist}[3]{\mbox{list}[#1]^{#2}\,#3}
\newcommand{\tlists}[1]{ #1 \, \mbox{list} }
\newcommand{\ulist}[2]{\mbox{list}[#1]\,#2}
\newcommand{\tslist}[1]{\mbox{list}\,#1}
\newcommand{\ttree}[3]{\mbox{tree}[#1]^{#2}\,#3}
\newcommand{\utree}[2]{\mbox{tree}[#1]\,#2}
\newcommand{\tbase}{b}
\newcommand{\uarr}[2]{\mathrel{\xrightarrow[]{\wexec(#1,#2)}}} 
\newcommand{\uarrs}[1]{\mathrel{\xrightarrow[]{\mu(#1)}}} 
\newcommand{\uarrd}{\mathrel{\xrightarrow{\wdead}}}
\newcommand{\tarrd}[1]{\mathrel{\xrightarrow{\wdiff(#1)}}}



\newcommand{\tforall}[3]{\forall#1\overset{\wexec(#2,#3)}{::}S.\,}
\newcommand{\tforalld}[2]{\forall#1\overset{\wdiff(#2)}{::}S.\,}
\newcommand{\uforalls}[2]{\forall#1\overset{\mu(#2)}{::}S.\,}
\newcommand{\tforallS}[1]{\forall#1.\,}
\newcommand{\tsforall}[1]{\forall#1{::}S.\,}
\newcommand{\tforallN}[1]{\forall#1{::}\ssize.\,}
\newcommand{\texists}[1]{\exists#1{::}S.\,}
\newcommand{\texistsN}[1]{\exists#1{::}\ssize.\,}
\newcommand{\tcimpl}[2]{#1 \mathrel{\supset} #2}
\newcommand{\tcprod}[2]{#1 \mathrel{\&} #2}

\newcommand{\ttimes}{\mathrel{\times}}
\newcommand{\tsum}{\mathrel{+}}
\newcommand{\tinter}{\mathrel{\wedge}}
\newcommand{\tst}[1]{(#1)^{\astable}}
\newcommand{\tch}[2]{U\,(#1,#2)} 
\newcommand{\tchs}[1]{U\,(#1,#1)} 
\newcommand{\tcho}[1]{U\,#1} 
\newcommand{\tmu}[1]{(#1)^{\mu}}
\newcommand{\tno}[1]{(#1)^{\_}}
\newcommand{\trm}[2]{|#1|_{#2}}
\newcommand{\trmo}[1]{|#1|}
\newcommand{\tmup}[1]{(#1)^{\mu'}}
\newcommand{\tdual}[1]{d({#1})}
\newcommand{\tbox}[1]{\square\,#1}
\newcommand{\tdmu}[1]{#1^{\shortdownarrow {\mu}}}
\newcommand{\tmon}[1]{{\color{red}m(#1)}}
\newcommand{\tforce}[1]{#1^{\shortdownarrow \achange }}
\newcommand{\tlift}[2]{(#1,#2)^{\uparrow}}
\newcommand{\tpull}[1]{#1^{\nearrow}}
\newcommand{\tpushd}[1]{(#1)^{\downarrow\square}}

% Terms
\newcommand{\vbase}{r}
\newcommand{\vtrue}{\mbox{tt}}
\newcommand{\vfalse}{\mbox{ff}}

\newcommand{\la}{\langle} 
\newcommand{\ra}{\rangle}
\newcommand{\eapp}{\;} 
\newcommand{\eleft}{\pi_1}
\newcommand{\eright}{\pi_2} 
\newcommand{\econst}{\kw{n}}
\newcommand{\etrue}{\mbox{true}}
 \newcommand{\efalse}{\mbox{false}}
\newcommand{\eif}{\mbox{if\;}} 
\newcommand{\ethen}{\mbox{\;then\;}}
\newcommand{\eelse}{\mbox{\;else\;}} 
\newcommand{\einl}{\mbox{inl\;}}
\newcommand{\einr}{\mbox{inr\;}} 
\newcommand{\elet}{\mbox{let\;}}
\newcommand{\clet}{\mbox{clet}\;}
\newcommand{\ecimp}{\mbox{.}_c\;}
\newcommand{\eelimU}{\mbox{elim}_U\;} 
\newcommand{\ein}{\mbox{\;in\;}}
\newcommand{\ecase}{\mbox{\;case\;}} 
\newcommand{\eof}{\mbox{\;of\;}}
\newcommand{\eas}{\mbox{\;as\;}} 
\newcommand{\ecelim}{\mbox{celim\;}}
\newcommand{\enil}{\mbox{nil}} 
\newcommand{\epack}{\mbox{pack\;}}
\newcommand{\eunpack}{\mbox{unpack\;}}
\newcommand{\efix}{\mbox{fix\;}} 
\newcommand{\efixNC}{\mbox{fix$_{NC}$\;}} 
\newcommand{\eLam}{ \Lambda}
\newcommand{\elam}{ \lambda} 
\newcommand{\eApp}{ [\,]\,}
\newcommand{\eleaf}{\mbox{leaf}} 
\newcommand{\ewith}{\;\mbox{with}\;} 
\newcommand{\enode}{\mbox{node}}
\newcommand{\econs}{\mbox{cons}} 
\newcommand{\econsC}{\mbox{cons$_C$}} 
\newcommand{\econsNC}{\mbox{cons$_{NC}$}} 
\newcommand{\eunit}{()}
\newcommand{\eswitch}{\kw{switch}\;}
\newcommand{\enoch}{\kw{NC}\;}
\newcommand{\eder}{\kw{der}\;}
\newcommand{\esplit}{\kw{split}\;}
\newcommand{\ecoerce}[2]{\kw{coerce}_{#1,#2}\;}
\newcommand{\econtra}{\kw{contra}\;}

\newcommand{\ealloc}[2]{ \mathrel{ \mathsf{alloc}\, {#1} \, {#2} } }
\newcommand{\eallocB}[2]{ \mathrel{ \mathsf{alloc_b}\, {#1} \, {#2} } }
\newcommand{\eupdt}[3]{ \mathrel{ \mathsf{update} \ {#1} \ {#2} \ {#3} }  }
\newcommand{\ereadx}[2] { \mathrel{ \mathsf{read} \ {#1} \ {#2} }  }
\newcommand{\eupdtB}[3]{ \mathrel{ \mathsf{update_b} \ {#1} \ {#2} \ {#3} }  }
\newcommand{\ereadxB}[2] { \mathrel{ \mathsf{read_b} \ {#1} \ {#2} }  }
\newcommand{\eret}[1] {\mathrel{ \mathsf{return} \, {#1} }}
\newcommand{\eletx}[3]{  \mathrel{ \mathsf{let_m} \{ {#1} \} = {#2} \ \mathsf{in} \ {#3}  } }



\newcommand{\caseof}[1]{\mbox{case}~#1~\mbox{of}}
\newcommand{\tcaseof}[1]{\mathsf{case}~#1~\mathsf{of}}
\newcommand{\ofnil}[1]{~~\mbox{nil}~\to#1}
\newcommand{\ofzero}[1]{~~\kw{0}~\to#1}
\newcommand{\ofcons}[3]{|~#1::#2~\to~#3}

% Diff Rel
\newcommand{\udiff}{\gtrapprox}
\newcommand{\rdiff}{\ominus}
\newcommand{\rdiffs}{\lesssim}
\newcommand{\ldiff}{\lesssim}

% Evaluation
\newcommand{\red}[1]{\Downarrow^{#1}}

\newcommand{\wmax}{\mbox{\scriptsize max}}
\newcommand{\wmin}{\mbox{\scriptsize min}}
\newcommand{\wdiff}{\mbox{\scriptsize diff}}
\newcommand{\wexec}{\mbox{\scriptsize exec}}
\newcommand{\wdead}{\mbox{\scriptsize dead}}
%Logical relation
\newcommand{\step}{\text{Step index}}
\newcommand{\world}{\text{World}}
\newcommand{\values}{\text{Value}}
\newcommand{\expr}{\text{Expression}}
\newcommand{\ulr}[1]{\llbracket#1\rrbracket_{v}}
\newcommand{\ulrg}[1]{\llbracket#1\rrbracket_{\grt}}
\newcommand{\lr}[1]{\llparenthesis#1\rrparenthesis_{v}}
\newcommand{\lre}[2]{\llparenthesis#1\rrparenthesis_{\varepsilon}^{#2}}
\newcommand{\lrg}[1]{\llparenthesis#1\rrparenthesis_{\grt}}
\newcommand{\ulre}[3]{\llbracket#1\rrbracket_{\varepsilon}^{#2,#3}}
\newcommand{\ulrew}[1]{\llbracket#1\rrbracket_{\varepsilon}^{0,\sinf}}

\newcommand{\relwith}[2]{\{#1~|~#2\}}
\newcommand{\rel}[1]{\{#1\}}
\newcommand{\del}[1]{\mathcal{D}\llbracket#1\rrbracket}
\newcommand{\dd}[1]{\mathcal{D}\llbracket\Delta\rrbracket}
\newcommand{\ugsubst}[1]{\mathcal{G}\llbracket#1\rrbracket}
\newcommand{\gsubst}[1]{\mathcal{G}\llparenthesis#1\rrparenthesis}
\newcommand{\dsubst}[1]{\mathcal{D}\llbracket#1\rrbracket}
\newcommand{\s}{\sigma}
\newcommand{\peq}{\preceq}
\newcommand{\plt}{\prec}
\renewcommand{\d}{\delta}
\newcommand{\g}{\gamma}


% Typing judgments
\newcommand{\jiterm}[2]{\mathrel{\vdash {#1} :: #2}}
\newcommand{\jtype}[4]{\mathrel{\vdash_{#1}^{#2} {#3} : {#4}}}
\newcommand{\jtypeM}[4]{\mathrel{\vdash_{#1}^{#2} {#3} :^c {#4}}}

\newcommand{\jtypes}[3]{\mathrel{\vdash_{#1}^{\mu} {#2} : {#3}}}

\newcommand{\jstype}[3]{\mathrel{\vdash
    {#1} \backsim {#2} : {#3}}}


\newcommand{\jtypediff}[4]{\mathrel{\vdash% _{\wdiff}
    {#2} \ominus {#3} \ldiff #1 : {#4}}}

\newcommand{\jtypediffM}[4]{\mathrel{\vdash
    {#2} \ominus {#3} \ldiff #1 :^c {#4}}}
\newcommand{\jmintypesame}[3]{\mathrel{\vdash
    {#2} \ominus {#2} \ldiff #1 :^c {#3}}}

\newcommand{\jelab}[6]{\mathrel{\vdash
    {#2} \ominus {#3} \rightsquigarrow {#4} \ominus {#5} \ldiff #1 : {#6}}}
\newcommand{\jelabsame}[4]{\mathrel{\vdash
    {#2} \ominus {#3} \rightsquigarrow {#2} \ominus {#3} \ldiff #1 : {#4}}}

\newcommand{\jelabun}[5]{\mathrel{\vdash_{#1}^{#2}
    {#3} \rightsquigarrow {#4} : {#5}}}

\newcommand{\jelabc}[4]{\mathrel{\vdash
    {#2} \ominus {#3} \rightsquigarrow {#2}^* \ominus {#3}^* \ldiff #1 : {#4}}}


\newcommand{\jelabcu}[4]{\mathrel{\vdash_{#1}^{#2}
    {#3} \rightsquigarrow {#3}^* : {#4}}}

\newcommand{\jtypediffsym}[5]{\mathrel{\vdash
    #1 \ldiff {#3} \ominus {#4} \ldiff #2 : {#5}}}
\newcommand{\sty}[2]{\vdash#1 \mathrel{::} #2}


\newcommand{\rname}[1]{\mbox{\small{#1}}}

\newcommand{\vsem}[2]{\llbracket #1 \rrbracket_{V}^{#2}}
\newcommand{\esem}[2]{\llbracket #1 \rrbracket_{E}^{#2}}
\newcommand{\conj}{\mathrel{\wedge}}

\newcommand{\vusem}[1]{\llparenthesis #1 \rrparenthesis_{V}}
\newcommand{\eusem}[1]{\llparenthesis #1 \rrparenthesis_{E}}

\newcommand{\jsubtype}[2]{\sat#1\sqsubseteq#2}
\newcommand{\jasubtype}[2]{\sat^{\mathsf{\grt}}#1\sqsubseteq#2}
\newcommand{\jeqtype}[2]{\sat#1 \equiv#2}
\newcommand{\under}[2]{\sat #1 \trianglelefteq  #2}

\newcommand{\type}{\text{type}}
\newcommand{\rtype}{\text{relational type}}
\newcommand{\Type}{\text{Unary type}}
\newcommand{\Rtype}{\text{Binary type}}



% Cost Constants
\newcommand{\kvar}{c_{var}}
\newcommand{\kconst}{c_{n}}
\newcommand{\kinl}{c_{inl}}
\newcommand{\kinr}{c_{inl}}
\newcommand{\kcase}{c_{case}}
\newcommand{\kfix}{c_{fix}}
\newcommand{\kapp}{c_{app}}
\newcommand{\kLam}{c_{fix}}
\newcommand{\kiApp}{c_{iApp}}
\newcommand{\kpack}{c_{pack}}
\newcommand{\kunpack}{c_{unpack}}
\newcommand{\knil}{c_{nil}}
\newcommand{\kcons}{c_{cons}}
\newcommand{\kcaseL}{c_{caseL}}
\newcommand{\kleaf}{c_{leaf}}
\newcommand{\knode}{c_{node}}
\newcommand{\kcaseT}{c_{caseT}}
\newcommand{\kprod}{c_{prod}}
\newcommand{\kproj}{c_{proj}}
\newcommand{\klet}{c_{let}}


%Constraints
\newcommand{\creal}{\mathbb{R}}
\newcommand{\sat}[1]{\models#1}
\newcommand{\sata}[1]{\models_A#1}
\newcommand{\ceq}[2]{#1\mathrel{\doteq}#2}
\newcommand{\cleq}[2]{#1 \mathop{\leq} #2}
\newcommand{\cleqspec}[2]{#1 \overline{\mathop{\leq}} #2}
\newcommand{\clt}[2]{#1 \mathop{<} #2}
\newcommand{\cgt}[2]{#1 \mathop{>} #2}
\newcommand{\ceqz}[1]{#1 \mathrel{\doteq} 0}
\newcommand{\cneg}[1]{\mathop{\neg}#1}
\newcommand{\cand}[2]{#1 \wedge #2}
\newcommand{\cexists}[3]{\exists#1::#2.#3}
\newcommand{\cexistsK}[3]{\exists#1:#2.#3}
\newcommand{\cexistsS}[2]{\exists#1.#2}
\newcommand{\cexistsC}[2]{\exists#1::\scost.#2}
\newcommand{\cexistsall}[2]{\exists(#1).#2}
\newcommand{\cforall}[3]{\forall#1::#2.#3}
\newcommand{\cforallS}[3]{\forall#1:#2.#3}
\newcommand{\cimpl}[2]{#1\rightarrow#2}
\newcommand{\cor}[2]{#1 \vee #2}
\newcommand{\ctrue}{\top}
\newcommand{\cfalse}{\bottom}
\newcommand{\blank}[2][100]{\hfil\penalty#1\hfilneg }
\newcommand{\ccond}[3]{\im{#1}\mathrel{\mbox{?}}\im{#2}\mathrel{\colon}\im{#3}}


\newcommand{\wfty}[1]{\vdash   #1~\kw{wf}}
\newcommand{\awfty}[1]{\vdash^{\mathsf{\grt}}   #1~\kw{wf}}
\newcommand{\wfcs}[1]{\vdash   #1~\kw{wf}}
\newcommand{\wfctx}[1]{\vdash   #1~\kw{wf}}
\newcommand{\awfctx}[1]{\vdash^{\mathsf{\grt}}   #1~\kw{wf}}



\newcommand{\dc}{ downward closure (\lemref{lem:down-closure}) }
\newcommand{\ctx}{\Delta; \Phi_a; \Gamma}
\newcommand{\nctx}{\Delta; \Phi_a; \tbox{\Gamma}}
\newcommand{\primctx}{\Upsilon}
\newcommand{\octx}{\Delta; \Phi_a; \Omega}
\newcommand{\rctx}[1]{\Delta; \Phi_a; \trm{\Gamma}{#1}}

\newcommand{\shade}[1]{\colorbox{lightgray}{#1}}
\newcommand{\fv}[1]{\text{FV}(#1)}
\newcommand{\fcv}[1]{\text{dom}(#1)}
\newcommand{\fiv}[1]{\text{FIV}(#1)}
\newcommand{\fdv}[1]{\text{dom}(#1)}


\newcommand{\assC}[2]{\text{Assume that $\sat \s \Phi$ and there exists $\Gamma'$  s.t. $\fv{#2} \subseteq \fcv{\Gamma'} $ and $\Gamma' \subseteq \Gamma$ and $(m, \d) \in \ugsubst{\trm{\sigma \Gamma'}{#1}}$}}
\newcommand{\assCU}[1]{\text{Assume that $\sat \s \Phi$ and there exists $\Omega'$  s.t. $\fv{#1} \subseteq \fcv{\Omega'} $ and $\Omega' \subseteq \Omega$ and $(m, \d) \in \ugsubst{\s \Omega'}$}}
\newcommand{\IHassun}[1]{\text{$\fv{#1} \subseteq \fcv{\Omega'} $ and $\Omega' \subseteq \Omega$ and $(m, \d) \in \ugsubst{\s \Omega'}$}}


\newcommand{\IHassU}[2]{\text{$\fv{#2} \subseteq \fcv{\trm{\Gamma'}{#1}} $ and $\trm{\Gamma'}{#1} \subseteq \trm{\Gamma}{#1}$ and $(m, \d) \in \ugsubst{\trm{\sigma \Gamma'}{#1}}$}}


\newcommand{\IHass}[2]{\text{$\fv{#2} \subseteq \fcv{\Gamma'} $ and $\Gamma' \subseteq \Gamma$ and $(m, \d) \in \ugsubst{\trm{\sigma \Gamma'}{#1}}$}}
% Environment
\newcommand{\memory}{\Gamma}%\Delta \ | \ \Phi \ | \ \Gamma  \ | \
                            %\Sigma}
\newcommand{\senv}{\Delta}
\newcommand{\lenv}{\Sigma}
\newcommand{\uenv}{\Omega}
\newcommand{\renv}{\Gamma} 
\newcommand{\cenv}{\Phi} 
\newcommand{\sep}{ \ | \ }
\newcommand{\monad}[4]{\mathrel{ M( \overset{ \mathrel{\mathrm{exec}{#4 }}}{#2}  })}
\newcommand{\monadR}[4]{\mathrel{ \overset{\mathrm{diff}(#4)}{\{ {#1} \} \ {#2} \ \{ {#3} \} }}}
\newcommand{\depProd}[4]{ \mathrel{ \Pi {#1} \stackrel{\mathrm{exec} {#4}}{:}{#2} . \ {#3}}}
\newcommand{\depProdr}[4]{ \mathrel{ \Pi {#1} \stackrel{\mathrm{diff} {#4}}{:}{#2} . \ {#3}}}
\newcommand{\uarrow}[3]{ \mathrel{ \stackrel{\mathrm{exec} {#3}}{{#1} \longrightarrow#2}}}
\newcommand{\uforall}[4]{ \mathrel{ \stackrel{\mathrm{exec} {#4}}{\forall {#1} :#2 . \ #3}}}
\newcommand{\uexist}[3]{\mathrel{{\exists {#1} :: {#2} . \ {#3}}}}
\newcommand{\rarrow}[3]{ \mathrel{ \stackrel{\mathrm{diff}(#3)}{{#1} \longrightarrow {#2}}}}
\newcommand{\rarrowt}[3]{ \mathrel{ {#1} \stackrel{\mathrm{} {#3}}{\longrightarrow} {#2}}}
\newcommand{\rforall}[4]{ \mathrel{ \stackrel{\mathrm{diff}(#4)}{\forall {#1}{::}{#2} . \ {#3}}}}
\newcommand{\rexists}[3]{ \mathrel{ {\exists {#1} {::}{#2} . \ {#3}}}}
\newcommand{\rforallt}[4]{ \mathrel{ \forall {#1} \stackrel{\mathrm{} {#4}}{:}{#2} . \ {#3}}}
\newcommand{\arr}[3]{ \mathrel{ \mathsf{Array}_{#1}[{#2}] \ {#3}} }
\newcommand{\arrR}[3]{ \mathrel{ \mathsf{Array}_{#1}[{#2}] \ {#3}} }
\newcommand{\lst}[2]{ \mathrel{ \mathsf{list}[{#1}] \ {#2}} }
\newcommand{\lstR}[3]{ \mathrel{ \mathsf{list}^{#1}[{#2}] \ {#3}} }
\newcommand{\abs}[2]{\mathrel { \lambda {#1} . {#2} } }
\newcommand{\app}[2]{\mathrel{ {#1} \, {#2} }}
\newcommand{\ret}[1] {\mathrel{ \mathsf{return} \, {#1} }}

\newcommand{\letx}[3]{  \mathrel{ \mathsf{let}\   {#1} = {#2} \ \mathsf{in} \ {#3}  } }
\newcommand{\packx}[1]{  \mathrel{ \mathsf{pack} \, {#1}} }
\newcommand{\unpackx}[3]{  \mathrel{ \mathsf{unpack} \,  {#1} \, \mathsf{as} \, {#2} \, \mathsf{in} \ {#3}  } }
\newcommand{\alloc}[2]{ \mathrel{ \mathsf{alloc}\, {#1} \, {#2} } }
\newcommand{\updt}[3]{ \mathrel{ \mathsf{update} \ {#1} \ {#2} \ {#3} }  }
\newcommand{\readx}[2] { \mathrel{ \mathsf{read} \ {#1} \ {#2} }  }
\newcommand{\tTt}[3]{\mathrel{  {#1} \xrightarrow \ {#2} } }
\newcommand{\force}[1]{\mathrel{\mathsf{force} \ \ {#1}}}
\newcommand{\tfix}{\mathsf{Fix}}
\newcommand{\fix}[1]{\mathsf{fix} \, f(x). {#1}}

%Relational
\newcommand{\monadx}[3]{\mathrel{ \{ {#1} \} \ {#2} \ \{ {#3} \} }}
\newcommand{\monadu}[4]{\mathrel{ \overset{ \mathrel{\mathrm{exec}{#4 }}}{\{ {#1} \} \ #2 \ \{ {#3} \}} }}
\newcommand{\cmp}[4] {\mathrel{   \vdash  {#1} \ominus {#2} \ldiff {#4}  : {#3}  }}
\newcommand{\pair}[1]{\mathrel{ {#1}_{1}{#1}_{2}}}
\newcommand{\imp}[2]{\mathrel{  {#1} \Rightarrow {#2} }}
\newcommand{\eval}[3]{\mathrel{ {#1} \Downarrow^{#3} {#2}   }}
\newcommand{\evalf}[3]{\mathrel{ {#1} \Downarrow^{#3}_{f} {#2}   }}
\newcommand{\evalp}[3]{\mathrel{ {#1} \Downarrow^{#3}_{p} {#2}   }}
\newcommand{\heap}[1]{ ;  {#1}}
\newcommand {\spc} { \  \ }
\newcommand{\monadL}[3]{\mathrel{ \{ {#1} \}  \\  \ {#2} \ \\  \{ {#3} \} }}

\newcommand{\wfa}[1]{\mathrel{\vdash {#1} \quad wf}}
\newcommand{\wf}[1]{\mathrel{\vdash {#1} \quad wf}}
\newcommand{\subtypeA}[2]{\mathrel{ \models {#1} \sqsubseteq {#2} } }
\newcommand{\subtype}[2]{\mathrel{   \models {#1} \sqsubseteq {#2} }   }
\newcommand{\subcost}[3]{\mathrel{   \models {#1} {#3} {#2} }   }

\newcommand{\emptyhp}{\mathsf{empty}}
\newcommand{\llb}[1]{ \llbracket {#1} \rrbracket }
\newcommand{\llu}[2]{ \llb{#1}_{#2}}
\newcommand{\llp}[2]{ \llparenthesis {#1} \rrparenthesis_{#2} }
\newcommand{\llbe}[1]{ \llbracket {#1} \rrbracket^{E} }
\newcommand{\llpe}[2]{ \llparenthesis {#1} \rrparenthesis_{#2}^{E} }

\newcommand{\mg}[1]{\textcolor[rgb]{.90,0.00,0.00}{[MG: #1]}}
\newcommand{\dg}[1]{\textcolor[rgb]{0.00,0.5,0.5}{[DG: #1]}}
\newcommand{\wq}[1]{\textcolor[rgb]{.50,0.0,0.7}{[WQ: #1]}}

% Helpful shortcuts

\newcommand{\freshSize}[1]{#1\in\text{fresh}(\ssize)}
\newcommand{\freshCost}[1]{#1\in\text{fresh}(\scost)}
\newcommand{\freshVar}[1]{#1 \in \text{fresh}(S)}

\newcommand{\m}{M} 
%Bi-directional Typing Judgement
\newcommand{\chdiff}[5]{\vdash{#1}\rdiff{#2}~{\downarrow}~#3,#4 \Rightarrow
{\color{red}#5}}

\newcommand{\chsdiff}[3]{\vdash{#1}\backsim{#2}~{\downarrow}~#3}


\newcommand{\chdiffNC}[5]{\vdash^{\color{blue}NC}{#1}\rdiff{#2}~{\downarrow}~#3,#4 \Rightarrow
{{\color{red}#5}}}

\newcommand{\infdiff}[6]{\vdash{#1}\rdiff{#2}~{\uparrow}~{\color{red}{#3}}\Rightarrow[{\color{red}#4}],{\color{red}#5},{\color{red}#6}}

\newcommand{\infsdiff}[3]{\vdash{#1}\backsim{#2}~{\uparrow}~{\color{red}{#3}}}

\newcommand{\infdiffsimple}[5]{\vdash{#1}\rdiff{#2}~{\uparrow}~{\color{red}{#3}}\Rightarrow{\color{red}#4},{\color{red}#5}}

\newcommand{\chmax}[4]{\vdash^{\wmax}{#1}~{\downarrow}~#2, #3  \Rightarrow
{{\color{red}#4}}}
\newcommand{\chmin}[4]{\vdash^{\wmin}{#1}~{\downarrow}~#2, #3  \Rightarrow
{{\color{red}#4}}}

\newcommand{\chexec}[5]{\vdash{#1}~{\downarrow}~#2, #3, #4  \Rightarrow
{{\color{red}#5}}}

\newcommand{\infmax}[5]{\vdash^{\wmax}{#1}~{\uparrow}~{\color{red}{#2}}\Rightarrow[{\color{red}#3}], {\color{red}#4},{\color{red}#5}}
\newcommand{\infmin}[5]{\vdash^{\wmin}{#1}~{\uparrow}~{\color{red}{#2}}\Rightarrow[{\color{red}#3}], {\color{red}#4},{\color{red}#5}}

\newcommand{\infexec}[6]{\vdash{#1}~{\uparrow}~{\color{red}{#2}}\Rightarrow[{\color{red}#3}], {\color{red}#4},{\color{red}#5},{\color{red}#6}}

\newcommand{\infexecsimple}[5]{\vdash{#1}~{\uparrow}~{\color{red}{#2}}\Rightarrow{\color{red}#3}, {\color{red}#4},{\color{red}#5}}

\newcommand{\emptypsi}{.}

%Existential elimination
\newcommand{\elimExt}[3]{#1 \vdash \kw{elimExt}(#2)~\downarrow #3}
\newcommand{\solveVar}[6]{#1 \vdash \kw{solve}(#2;#3) \downarrow (#4;#5;#6)}



%Shortcuts
\newcommand{\al}{\alpha}
\newcommand{\algwf}[1]{\vdash  #1~\kw{wf}}
\newcommand{\algwfa}[1]{\vdash^{A} #1~\kw{wf}}
\newcommand{\jalgeqtype}[3]{\sat#1\equiv#2\Rightarrow {\color{red}#3}}
\newcommand{\jalgasubtype}[3]{\sat^{\mathsf{\grt}}#1\sqsubseteq#2\Rightarrow {\color{red}#3}}
\newcommand{\jalgsubtype}[3]{\sat#1\sqsubseteq#2\Rightarrow {\color{red}#3}}
\newcommand{\jalgssubtype}[2]{\sat#1\leq#2}


\newcommand{\fvars}[1]{\text{FV}(#1)}
\newcommand{\fivars}[1]{\text{FIV}(#1)}
\newcommand{\filtercost}[1]{\text{filterCost}(#1)}
\newcommand{\uctx}{\Delta; \psi_a; \Phi_a; \Omega}
\newcommand{\bctx}{\Delta; \psi_a; \Phi_a; \Gamma}


\newcommand{\suba}[1]{{#1}[\theta_a]}
\newcommand{\subaex}[2]{{#1}[\theta_a, #2]}
\newcommand{\subt}[1]{{#1}[\theta]}
\newcommand{\subta}[1]{{#1}[\theta\,\theta_a]}
\newcommand{\subsat}[3]{#1~ \rhd~ #2 : #3}

\newcommand{\erty}[1]{|#1|}
\newcommand{\eanno}[4]{(#1:#2,#3,#4)}
\newcommand{\eannobi}[3]{(#1:#2,#3)}
\newcommand{\e}{\overline{e}}
\newcommand{\trans}{\rightsquigarrow}
\newcommand{\tboxp}[1]{\square(#1)}
\newcommand{\tlr}[1]{\tlift{\trm{#1}{i}}}
\newcommand*\bang{!}



%%% Local Variables:
%%% mode: latex
%%% TeX-master: "main"
%%% End:

\newcommand{\nform}{\mathsf{F}}
\newcommand{\mechanism}{\mathsf{M}}
\newcommand{\depth}{\mathsf{depth}}
\newcommand{\query}{\text{Q}}

% \newcommand{\caseof}[2]{\mathsf{case} \ {#1} \ \mathsf{of}\ \{ {#2}\}}
\newtheorem{lemma}{Lemma}
\newtheorem{theorem}{Theorem}[section]
\newtheorem{corollary}{Corollary}[theorem]
%%% Attempt 1: Linear 1



\newcommand{\diam}{{\color{red}\diamond}}
\newcommand{\dagg}{{\color{blue}\dagger}}
\let\oldstar\star
\renewcommand{\star}{\oldstar}

\newcommand{\im}[1]{\ensuremath{#1}}

\newcommand{\kw}[1]{\im{\mathtt{#1}}}


\newcommand{\set}[1]{\im{\{{#1}\}}}

\newcommand{\mmax}{\ensuremath{\mathsf{max}}}

%%%%%%%%%%%%%%%%%%%%%%%%%%%%%%%%%%%%%%%%%%%%%%%%%%%%%%%%
% Comments
\newcommand{\omitthis}[1]{}

% Misc.
\newcommand{\etal}{\textit{et al.}}
\newcommand{\bump}{\hspace{3.5pt}}

% Text fonts
\newcommand{\tbf}[1]{\textbf{#1}}
%\newcommand{\trm}[1]{\textrm{#1}}

% Math fonts
\newcommand{\mbb}[1]{\mathbb{#1}}
\newcommand{\mbf}[1]{\mathbf{#1}}
\newcommand{\mrm}[1]{\mathrm{#1}}
\newcommand{\mtt}[1]{\mathtt{#1}}
\newcommand{\mcal}[1]{\mathcal{#1}}
\newcommand{\mfrak}[1]{\mathfrak{#1}}
\newcommand{\msf}[1]{\mathsf{#1}}
\newcommand{\mscr}[1]{\mathscr{#1}}









\newcommand{\defeq}{\mathrel{\doteq}}
\newcommand{\conj}{\mathrel{\wedge}}
\newcommand{\disj}{\mathrel{\vee}}

\newcommand{\lzero}{0}


% context
\newcommand{\tctx}{\Gamma}
\newcommand{\ictx}{ }


% expression
\newcommand{\expr}{e}
\newcommand{\aexpr}{a}
\newcommand{\bexpr}{b}
\newcommand{\sexpr}{\textrm{e} }
\newcommand{\qexpr}{\psi}
\newcommand{\qval}{\alpha}
\newcommand{\query}{{\tt query}}
\newcommand{\saexpr}{\textrm{a} }
\newcommand{\sbexpr}{\textrm{b} }
\newcommand{\vall}{w}
\newcommand{\valr}{v}
\newcommand{\eif}{\kw{if}}
\newcommand{\eapp}{\;}
\newcommand{\eprojl}{\kw{fst}}
\newcommand{\eprojr}{\kw{snd}}
\newcommand{\eifvar}{\kw{ifvar}}
%expression and commands for WHILE language
\newcommand{\ewhile}{\kw{while}}
\newcommand{\bop}{*}
\newcommand{\uop}{\circ}
\newcommand{\eskip}{\kw{skip}}

\newcommand{\eloop}{\kw{loop}}
\newcommand{\edo}{\kw{do}}
\newcommand{\qdom}{\mathcal{QD}}

%configuration
\newcommand{\config}[1]{\langle #1 \rangle}
\newcommand{\ematch}{\kw{match}}
\newcommand{\clabel}[1]{\left[ #1 \right]}


%\newcommand{\eprov}[1]{\eta_{#1}}
\newcommand{\etrue}{\kw{true}}
\newcommand{\efalse}{\kw{false}}
\newcommand{\econst}{c}
\newcommand{\eop}{\delta}
\newcommand{\efix}{\mathop{\kw{fix}}}
\newcommand{\elet}{\mathop{\kw{let}}}
\newcommand{\ein}{\mathop{ \kw{in}} }
\newcommand{\eas}{\mathop{ \kw{as}} }
\newcommand{\enil}{\kw{nil}}
\newcommand{\econs}{\mathop{\kw{cons}}}
%\newcommand{\labelA}{\ell}
%monad expressions / terms
\newcommand{\term}{t}
\newcommand{\return}{\kw{return}}
\newcommand{\bernoulli}{\kw{bernoulli}}
\newcommand{\uniform}{\kw{uniform}}
 \newcommand{\epack}{\mbox{pack\;}}
\newcommand{\eunpack}{\mbox{unpack\;}}
\newcommand{\eilam}{\Lambda.}

\newcommand{\evec}{\kw{dict}}
\newcommand{\eget}{\kw{get}}

% trace
\newcommand{\triapp}[2]{\kw{IApp}(#1,#2)}
\newcommand{\trow}{\text{row}}
\newcommand{\tr}{T}
\newcommand{\trift}{\eif^{\kw{t}}}
\newcommand{\triff}{\eif^{\kw{f}}}
\newcommand{\trprojl}{\eprojl}
\newcommand{\trprojr}{\eprojr}
\newcommand{\trtrue}{\etrue}
\newcommand{\trfalse}{\efalse}
\newcommand{\trconst}{\econst}
\newcommand{\trop}{\eop}
\newcommand{\trfix}{\efix}
\newcommand{\trapp}[5]{#1 \; #2 \mathrel{\triangleright} {\efix
#3(#4).#5}}
\newcommand{\trnil}{\enil}
\newcommand{\trcons}{\econs}
\newcommand{\trlet}{\elet}
%types for monad
\newcommand{\treal}{\kw{real}}
\newcommand{\tint}{\kw{int}}
\newcommand{\tmonad}{\kw{M}}
\newcommand{\tunit}{\kw{unit}}
\newcommand{\tdb}{\kw{tdb}}

% adaptivity
\newcommand{\adap}{\kw{adap}}
\newcommand{\ddep}[1]{\kw{depth}_{#1}}
\newcommand{\nat}{\mathbb{N}}
\newcommand{\natb}{\nat_{\bot}}
\newcommand{\natbi}{\natb^\infty}
\newcommand{\nnatA}{Z}
\newcommand{\nnatB}{m}
\newcommand{\nnatbA}{s}
\newcommand{\nnatbB}{t}
\newcommand{\nnatbiA}{q}
\newcommand{\nnatbiB}{r}

%type
\newcommand{\type}{\tau}
\newcommand{\tbase}{\kw{b}}
\newcommand{\tbool}{\kw{bool}}
\newcommand{\tbox}[1]{ \kw{\square} \, (#1) }
\newcommand{\tarr}[5]{#1; #3 \xrightarrow{#4; \, #5} #2}
\newcommand{\tlist}[1]{\kw{list} \, #1 }
\newcommand{\env}{\theta}
\newcommand{\tforall}[3]{\forall#3 \overset{#1, #2}{::} S.\, }
\newcommand{\texists}[1]{\exists#1 {::} S.\, }
\newcommand{\lto}{\multimap}
\newcommand{\bang}[1]{ !_{#1}}
\newcommand{\whynot}[1]{ ?_{#1} }
\newcommand{\ltype}{A}
\newcommand{\adapt}{R}
% index
\newcommand{\idx}{I }
\newcommand{\smax}[2]{\kw{max}(#1,#2)}
\newcommand{\ienv}{\sigma}

%evaluation
\newcommand{\bigstep}[1]{\mathrel{\to^{#1}}}

\newcommand{\dmap}{\rho}
\newcommand{\dmapb}{\bot_\dmap}
\newcommand{\supp}{\kw{supp}}
\newcommand{\dom}{\kw{dom}}
\newcommand{\codom}{\kw{codom}}

\newcommand{\tvdash}[1]{\vdash_{#1}}

\newcommand{\lrv}[1]{[\![ #1 ]\!]_{\text{V}}}
\newcommand{\lre}[3]{[\![ #3 ]\!]_{\text{E}}^{#1, #2}}
\newcommand{\stepiA}{k}
\newcommand{\stepiB}{j}
\newcommand{\size}[1]{|#1|}

%logic relations
\newcommand{\lr}[1]{[\![ #1 ]\!]}
\newcommand{\lrt}[1]{\mathcal{T}[\![ #1 ]\!]}


\newcommand{\wf}[1]{\vdash #1 \, \kw{wf} }
\newcommand{\sub}[2]{ #1 \, <: \, #2 }
\newcommand{\eqv}[3]{ #1 \, \equiv \, #2 \Rightarrow \textcolor{red}
{#3}  }
\newcommand{\eqvt}[3]{ #1 \, \sqsubseteq \, #2 \Rightarrow \textcolor{red}
{#3}  }
\newcommand{\eqvc}[2]{ #1 \, \equiv^c \, #2   }


%core calculus
\newcommand{\ctyping}[3]{ \tvdash{ #1} {#2} :^c #3 }
\newcommand{\cbox}{\mathsf{box}}
\newcommand{\cder}{\mathsf{der}}
\newcommand{\elab}[4]{ \vdash_{ #1} #2 \rightsquigarrow #3 : #4}
\newcommand{\coerce}[2]{\mathsf{coerce}_{#1, #2}}

%algorithmic typing rules
\newcommand{\infr}[4]{{#1} ~ {\textcolor{red}\uparrow} ~ {\color{red} #2} \Rightarrow
{ } {\color{red} #3} }
\newcommand{\chec}[3]{{#1} ~ {\downarrow} ~ {#2} \Rightarrow {\color{red} #3} }
% \newcommand{\restriction}{\Phi}
\newcommand{\fresh}{ \mathsf{fresh}}
\newcommand{\red}[1]{ \textcolor{red} {#1} }
\newcommand{\fiv}[1]{ \mathsf{FIV} (#1)   }
\newcommand{\fv}[1]{ \mathsf{FV} (#1)   }

\newcommand{\todo}[1]{{\small \color{red}\textbf{[[ #1 ]]}}}
\newcommand{\todomath}[1]{{\scriptstyle \color{red}\mathbf{[[ #1 ]]}}}

\newcommand{\caseL}[1]{\item \textbf{#1}\newline}

\newcommand{\attr}{\mathsf{attr}}
\newcommand{\weight}{\mathsf{W}}
\newcommand{\num}{\mathsf{n}}

\usepackage{enumitem}
\setenumerate{listparindent=\parindent}

\newlist{enumih}{enumerate}{3}
\setlist[enumih]{label=\alph*),before=\raggedright, topsep=1ex, parsep=0pt,  itemsep=1pt }

\newlist{enumconc}{enumerate}{3}
\setlist[enumconc]{leftmargin=0.5cm, label*= \arabic*.  , topsep=1ex, parsep=0pt,  itemsep=3pt }


\newlist{enumsub}{enumerate}{3}
\setlist[enumsub]{ leftmargin=0.7cm, label*= \textbf{subcase} \bf \arabic*: }

\newlist{enumsubsub}{enumerate}{3}
\setlist[enumsubsub]{ leftmargin=0.5cm, label*= \textbf{subsubcase} \bf \arabic*: }

\newlist{mainitem}{itemize}{3}
\setlist[mainitem]{ leftmargin=0cm , label= {\bf Case} }

%%%%COLORS
\definecolor{periwinkle}{rgb}{0.8, 0.8, 1.0}
\definecolor{powderblue}{rgb}{0.69, 0.88, 0.9}
\definecolor{sandstorm}{rgb}{0.93, 0.84, 0.25}
\definecolor{trueblue}{rgb}{0.0, 0.45, 0.81}


\usepackage{array}

\newlength\Origarrayrulewidth

% horizontal rule equivalent to \cline but with 2pt width
\newcommand{\Cline}[1]{%
 \noalign{\global\setlength\Origarrayrulewidth{\arrayrulewidth}}%
 \noalign{\global\setlength\arrayrulewidth{2pt}}\cline{#1}%
 \noalign{\global\setlength\arrayrulewidth{\Origarrayrulewidth}}%
}

% draw a vertical rule of width 2pt on both sides of a cell
\newcommand\Thickvrule[1]{%
  \multicolumn{1}{!{\vrule width 2pt}c!{\vrule width 2pt}}{#1}%
}

% draw a vertical rule of width 2pt on the left side of a cell
\newcommand\Thickvrulel[1]{%
  \multicolumn{1}{!{\vrule width 2pt}c|}{#1}%
}

% draw a vertical rule of width 2pt on the right side of a cell
\newcommand\Thickvruler[1]{%
  \multicolumn{1}{|c!{\vrule width 2pt}}{#1}%
}

\newcommand{\command}{c}
\newcommand{\green}[1]{{ \color{green} #1 } }

\newcommand{\func}[2]{\mathsf{AD}(#1) \to (#2)}
\newcommand{\varEst}{\bf{VetxEst}}
\newcommand{\graphGen}{\bf{GraphGen}}

\newcommand{\ag}[2]{\mathsf{VetxEst}{(#1)}\to {(#2)}}
\newcommand{\ad}[2]{\mathsf{GraphGen}{(#1)}\to {(#2)}}
\newcommand{\rb}{\mathsf{RechBound}}
\newcommand{\pathsearch}{\mathsf{AdaptPathSearch}}

\newcommand{\highlight}[1]{\textcolor[rgb]{.0,0.0,1.0}{ #1}}

\usepackage{tikz}
\usetikzlibrary{shapes,arrows}
\newcommand{\THESYSTEM}{\textsf{AdaptFun}}

% Define block styles
\tikzstyle{decision} = [diamond, draw, fill=blue!20, 
    text width=4.5em, text badly centered, node distance=3cm, inner sep=0pt]
\tikzstyle{block} = [rectangle, draw, fill=blue!20, 
    text width=5em, text centered, rounded corners, minimum height=4em]
\tikzstyle{line} = [draw, -latex']
\tikzstyle{cloud} = [draw, ellipse,fill=red!20, node distance=3cm,
    minimum height=2em]

\begin{document}
\title{Program Analysis for Adaptivity Analysis}

\author{}

\date{}

\maketitle
%
\tableofcontents

% \section{Examples for Understanding \cite{cousot2019abstract}}
\begin{example}
 \[
 \begin{array}{ll}
 & \sem{
 \ewhile {}^0 (x > 0) {}^1 x = x - 1;
 } \triangleq \\
 & \left\{ \left< \pi_1^0, ^0\right> \mid \pi_1^0 \in \mathbb{T^{+}} \right\}
 \\
 \cup & \left\{ \left< \pi_1^0, {}^0 \xrightarrow{(x > 0)} {}^1 \xrightarrow{(x = x - 1) = v} {}^0 \right> 
 \mid \pi_i^0 \in \mathbb{T^{+}} \land 
 \sem{(x > 0)}\env(\pi_1^0) = \etrue  \land
 \sem{(x - 1)}\env(\pi_1^0 \xrightarrow{(x > 0)} {}^1) = v \right\}
 \\
 \cup & \left\{ \left< \pi_1^0, {}^0 \xrightarrow{(x > 0)} {}^1 \xrightarrow{(x = x - 1) = v_1} {}^0 \xrightarrow{(x > 0)} {}^1 \xrightarrow{(x = x - 1) = v_2} {}^0 \right> 
 \left \vert 
 \begin{array}{l}
 \pi_i^0 \in \mathbb{T^{+}} \\
 \land 
 \sem{(x > 0)}\env(\pi_1^0) = \etrue  \\
 \land
 \sem{(x - 1)}\env(\pi_1^0 \xrightarrow{(x > 0)} {}^1) = v_1 \\
 \land 
 \sem{(x > 0)}\env(\pi_1{}^0 \xrightarrow{(x > 0)} {}^1 \xrightarrow{(x = x - 1) = v_1} {}^0) = \etrue \\
 \land
 \sem{(x - 1)}\env(\pi_1^0 \xrightarrow{(x > 0)} {}^1 \xrightarrow{(x = x - 1) = v_1} {}^0 \xrightarrow{(x > 0)} {}^1) = v_2
 \end{array}
 \right.
 \right\}
 \\
 \cup & \cdots 
 \\
 \cup & \left\{ \left< \pi_1^0, {}^0 \xrightarrow{(x > 0)} {}^1\xrightarrow{(x = x - 1) = v_1} {}^0 \cdots  {}^0 \xrightarrow{\neg(x > 0)} {}^l \right>  
 \left \vert 
 \begin{array}{l}
 \pi_1^0 \in \mathbb{T^{+}}\\
 \land 
 \sem{(x > 0)}\env(\pi_1^0) = \etrue  \\
 \land
 \sem{(x - 1)}\env(\pi_1^0 \xrightarrow{(x > 0)} {}^1) = v_1 \\
 \land \cdots \\
 % \land
 % \sem{(x - 1)}\env(\pi_1^0 \xrightarrow{(x > 0)} {}^1 \xrightarrow{(x = x - 1) = v} {}^0 \xrightarrow{(x > 0)} {}^1) = v_2\\
 \land 
 \sem{(x > 0)}\env(\pi_i{}^0 \xrightarrow{(x > 0)} {}^1\xrightarrow{(x = x - 1) = v_1} {}^0 \cdots  {}^0) = \efalse \\
 \land l = \kw{aft}[\ewhile {}^0 (x > 0) {}^1 x = x - 1;]
 \end{array}
 \right.
 \right\}
 \end{array}
 \]
 \end{example}
 %
 \begin{example}
 \[
 \begin{array}{ll}
 & \sem{ 
 \epsilon {}^0 x = 1; \ewhile {}^1  (x > 0) {}^2 x = x - 1 ; {}^3 }  \\
 \triangleq & 
 \sem{\epsilon {}^0 x = 1;} 
 \cup
 \left\{ \left< \pi_1{}^0, {}^0\pi_2^1\pi_3^3\right> 
 \left \vert 
 \begin{array}{l}
 \left< \pi_1^0, {}^0\pi_2^1 \right> \in \sem{\epsilon {}^0 x = 1;}
 \\
 \land \left< \pi_1{}^0\pi_2^1, {}^1\pi_3^3\right>  \in 
 \sem{\ewhile {}^1 (x > 0) {}^2 x = x - 1 ;} 
 \end{array}
 \right.
 \right\}
 \\
 \triangleq & 
 \sem{\epsilon}
 \cup
 \left\{ \left< \pi_1{}^0, {}^0 \pi_2 {}^1 \right> 
 \left \vert 
 \begin{array}{l}
 \left< \pi_1{}^0, {}^0 \right> \in \sem{\epsilon}  \land \\
  
 \left< \pi_1{}^0, {}^0 \pi_2{}^1 \right> \in \sem{ {}^0 x = 1;}
 \end{array}
 \right.
 \right\}
 \cup
 \left\{ \left< \pi_1{}^0, {}^0\pi_2^1\pi_3 {}^3\right> 
 \left \vert 
 \begin{array}{l}
 \left< \pi_1{}^0, {}^0\pi_2^1 \right> \in \sem{\epsilon {}^0 x = 1;}\land
 \\
  \left< \pi_1 {}^0\pi_2 {}^1, {}^1 \pi_3 {}^3\right>  \in 
 \sem{\ewhile {}^1  (x > 0) {}^2 x = x - 1 ;} 
 \end{array}
 \right.
 \right\}
 \\
 \triangleq 
 & \left\{ \left< \pi_1^0, ^0\right> \mid \pi_1^0 \in \mathbb{T^{+}} \right\} \\
 & \cup
 \left\{ \left< \pi_1{}^0, {}^0 \xrightarrow{(x = 1) = 1} {}^1\right> 
 \vert 
 \pi_1^0 \in \mathbb{T^{+}}
 \land 
 \sem{1}\env(\pi_1^0) = 1
 \right\} \\
 & \cup
 \left\{ \left< \pi_1{}^0, {}^0\xrightarrow{(x = 1) = 1} {}^1\pi_3^3\right> 
 \left \vert 
 % \begin{array}{l}
 \pi_1^0 \in \mathbb{T^{+}}
 \land
 \left< \pi_1 {}^0\xrightarrow{(x = 1) = 1} {}^1, {}^1\pi_3^3\right>  \in 
 \sem{\ewhile {}^1  (x > 0) {}^2 x = x - 1 ; {}^3} 
 % \end{array}
 \right.
 \right\}
 \\
 \triangleq
 & \left\{ \left< \pi_1^0, ^0\right> \mid \pi_1^0 \in \mathbb{T^{+}} \right\} \\
 & \cup
 \left\{ \left< \pi_1{}^0, {}^0 \xrightarrow{(x = 1) = 1} {}^1\right> 
 \vert 
 \pi_1^0 \in \mathbb{T^{+}}
 \land 
 \sem{1}\env(\pi_1^0) = 1
 \right\} 
 \\
 & \cup \left\{ \left< \pi_1^0, {}^0 \xrightarrow{(x = 1) = 1} {}^1 \xrightarrow{(x > 0)} {}^2 \xrightarrow{(x = x - 1) = 0} {}^1 \right> 
 \left \vert
 \begin{array}{l}
 \pi_i^0 \in \mathbb{T^{+}} \\
 \land 
 \sem{(x > 0)}\env(\pi_1{}^0 \xrightarrow{(x = 1) = 1} {}^1) = \etrue  \\
 \land
 \sem{(x - 1)}\env(\pi_1{}^0 \xrightarrow{(x = 1) = 1} {}^1 \xrightarrow{(x > 0)} {}^2) = 0
 \end{array}
 \right.
 \right\}
 \\
 & \cup \left\{ \left< \pi_1^0, {}^0 \xrightarrow{(x = 1) = 1} {}^1 \xrightarrow{(x > 0)} {}^2 \xrightarrow{(x = x - 1) = 0} {}^1 
 \xrightarrow{\neg(x > 0)} {}^3 \right> 
 \left \vert
 \begin{array}{l}
 \pi_i^0 \in \mathbb{T^{+}} \\
 \land 
 \sem{(x > 0)}\env(\pi_1{}^0 \xrightarrow{(x = 1) = 1} {}^1) = \etrue  \\
 \land
 \sem{(x - 1)}\env(\pi_1{}^0 \xrightarrow{(x = 1) = 1} {}^1 \xrightarrow{(x > 0)} {}^2) = 0 \\
 \land 
 \sem{(x > 0 )}\env(\pi_1{}^0 \xrightarrow{(x = 1) = 1} {}^1 \xrightarrow{(x > 0)} {}^2 \xrightarrow{(x = x - 1) = 0} {}^1) = \efalse
 \end{array}
 \right.
 \right\}
 \\
 \triangleq
 & \left\{ \left< \pi_1^0, ^0\right> \mid \pi_1^0 \in \mathbb{T^{+}} \right\} 
  \cup
 \left\{ \left< \pi_1{}^0, {}^0 \xrightarrow{(x = 1) = 1} {}^1\right> 
 \vert 
 \pi_1^0 \in \mathbb{T^{+}}
 \right\} 
  \cup \left\{ \left< \pi_1^0, {}^0 \xrightarrow{(x = 1) = 1} {}^1 \xrightarrow{(x > 0)} {}^2 \xrightarrow{(x = x - 1) = 0} {}^1 \right> 
 \vert \pi_i^0 \in \mathbb{T^{+}}
 \right\}
 \\
 & \cup \left\{ \left< \pi_1^0, {}^0 \xrightarrow{(x = 1) = 1} {}^1 \xrightarrow{(x > 0)} {}^2 \xrightarrow{(x = x - 1) = 0} {}^1 
 \xrightarrow{\neg(x > 0)} {}^3 \right> 
 \vert \pi_i^0 \in \mathbb{T^{+}}
 \right\}
 \end{array}
 \]
 \end{example}
 %
 \clearpage
 \begin{example}[Formal Example for Exclusion of Timing Channel].
 \label{ex:excltiming}
 \\
 $\mathsf{diff}(\omega, \omega')$ (in \cite{cousot2019abstract} Equation~(2)) excludes timing channel,(i.e., $\omega$ is a strict prefix of $\omega'$). There is an example showing this exclusion: 
 \[
 	\ewhile {}^1 (x > 0) \{{}^2 x = x - 1; {}^3 y = 1;\} 
 \]
 In this example, $y$'s execution times relies on value of $x$. Under the adaptivity scenario, $y$ depends on $x$ (control dependency).
 However, according to value dependency defined in Definition~2 \cite{cousot2019abstract}, we can never derive :
 \[
 	x \rightsquigarrow^{1}_{\ewhile {}^1 (x > 0) \{{}^2 x = x - 1; {}^3 y = 1;\}} y
 \]
 %
 \begin{proof}
 By semantics definition, we have:
 %
 \begin{equation}
 \label{eq:sem_timingdep}
 	\begin{array}{ll}
 & \sem{
 \ewhile {}^1 (x > 0) \{{}^2 x = x - 1; {}^3 y = 1;\} 
 } \triangleq \\
 & \left\{ \left< \pi_1^1, ^1\right> \mid \pi_1^1 \in \mathbb{T^{+}} \right\}
 \\
 \cup & \left\{ \left< \pi_1^1, {}^1 \xrightarrow{(x > 0)} {}^2 \xrightarrow{(x = x - 1) = v} {}^3 \xrightarrow{(y = 1) = 1} {}^1 \right> 
 \mid \pi_i^1 \in \mathbb{T^{+}} \land 
 \sem{(x > 0)}\env(\pi_1^1) = \etrue  \land
 \sem{(x - 1)}\env(\pi_1^1 \xrightarrow{(x > 0)} {}^2) = v\right\}
 \\
 \cup & \left\{ 
 \begin{array}{l}
 \left< \pi_1^1, {}^1 \xrightarrow{(x > 0)} {}^2 \xrightarrow{(x = x - 1) = v_1} {}^3 \xrightarrow{(y = 1) = 1} {}^1 \xrightarrow{(x > 0)} {}^2 \xrightarrow{(x = x - 1) = v_2} {}^3 \xrightarrow{(y = 1) = 1} {}^1 \right> 
  \\
  \left \vert
 \begin{array}{l}
 \pi_i^1 \in \mathbb{T^{+}} \\
 \land 
 \sem{(x > 0)}\env(\pi_1^1) = \etrue  \\
 \land
 \sem{(x - 1)}\env(\pi_1^1 \xrightarrow{(x > 0)} {}^2) = v_1 \\
 \land 
 \sem{(x > 0)}\env(\pi_1{}^1 \xrightarrow{(x > 0)} {}^2 \xrightarrow{(x = x - 1) = v_1} {}^3 \xrightarrow{(y = 1) = 1} {}^1) = \etrue \\
 \land
 \sem{(x - 1)}\env(\pi_1^1 \xrightarrow{(x > 0)} {}^2 \xrightarrow{(x = x - 1) = v_1} {}^3 \xrightarrow{(y = 1) = 1}
  {}^1 \xrightarrow{(x > 0)} {}^2) = v_2
 \end{array}
 \right.
 \end{array}
 \right\}
 \\
 \cup & \cdots 
 \\
 \cup & \left\{ 
 \begin{array}{l}
 \left< \pi_1^1, {}^1 \xrightarrow{(x > 0)} {}^2\xrightarrow{(x = x - 1) = v_1} {}^3 \xrightarrow{(y = 1) = 1} {}^1 \cdots  {}^1 \xrightarrow{\neg(x > 0)} {}^l \right>  
 \\
 \left \vert 
 \begin{array}{l}
 \pi_1^1 \in \mathbb{T^{+}}\\
 \land 
 \sem{(x > 0)}\env(\pi_1^1) = \etrue  \\
 \land
 \sem{(x - 1)}\env(\pi_1^1 \xrightarrow{(x > 0)} {}^2) = v_1 \\
 \land \cdots \\
 \land 
 \sem{(x > 0)}\env(\pi_i{}^1 \xrightarrow{(x > 0)} {}^2\xrightarrow{(x = x - 1) = v_1} {}^3 \xrightarrow{(y = 1) = 1} {}^1 \cdots  {}^1) = \efalse \\
 \land l = \kw{aft}[\ewhile {}^1 (x > 0) \{ {}^2 x = x - 1; {}^3 y = 1 \}]
 \end{array}
 \right.
 \end{array}
 \right\}
 \end{array}
 \end{equation}
 %
 %
 %
 Let $\pi_1^1, \pi_1'^1 \in \mathbb{T^{+}}$ be arbitrary traces s.t. 
 %
 $$\forall z \in \mathbb{V}\setminus \{x\}. ~ \env(\pi_1^1)(z) = \env(\pi_1'^1) $$ 
 %
 Let $v = \env(\pi_1^1)(x) $, $v' = \env(\pi_1'^1)(x')$, picking arbitrary $\pi_2, \pi_2'$ s.t.:
 \[
 	\left< \pi_1^1, \pi_2 \right>,  \left< \pi_1'^1, \pi'_2 \right> \in \sem{\ewhile {}^1 (x > 0) \{{}^2 x = x - 1; {}^3 y = 1;\} }
 \]
 %
 By Definition of $\mathsf{seqval}$, we have:
 \[
 	\mathsf{seqval}\sem{y}{}^2(\pi_1^1, \pi_2 ) = \underbrace{0 \cdots 1}_{0 \leq v}
 \]
 %
 \[
 	\mathsf{seqval}\sem{y}{}^2(\pi_1^1, \pi_2' ) = \underbrace{0 \cdots 1}_{0 \leq v'}
 \]
 %
  $\forall \omega_0, \omega_1', \omega_1, u, u' \st \underbrace{0 \cdots 1}_{0 \leq v} = \omega \cdot u \cdot \omega_1 \land \underbrace{0 \cdots 1}_{0 \leq v} = \omega \cdot u; \cdot \omega_1'$ we have:
 %
 \[
 	u = u' = 1
 \]
 %
 By $\mathsf{diff}(\omega, \omega')$ (in \cite{cousot2019abstract} Equation~(2) ), we have:
 %
 \[
 	\neg(\mathsf{diff}(\mathsf{seqval}\sem{y}{}^1(\pi_1^1, \pi_2 ), \mathsf{seqval}\sem{y}{}^1(\pi_1^1, \pi_2' )))
 \]
 %
 Then we have:
 \[
 \begin{array}{l}
 	\forall \left< \pi_1^1, \pi_2 \right>,  \left< \pi_1'^1, \pi'_2 \right> \in \mathcal{S}^{+\infty} \sem{\ewhile {}^2 (x > 0) {}^3 y = 1;} \st
 	\Big(
 	\forall z \in \mathbb{V}\setminus \{x\} \st \env(\pi_1^1)(z) = \env(\pi_1'^1) \st \\
 	\land \neg(\mathsf{diff}(\mathsf{seqval}\sem{y}{}^1(\pi_1^1, \pi_2 ), \mathsf{seqval}\sem{y}{}^1(\pi_1^1, \pi_2' )))\Big)
 \end{array}
 \] 
 \\
 %
 By Definition~2 (in \cite{cousot2019abstract}), 
 \[
 	\mathcal{S}^{+\infty}\sem{\ewhile {}^1 (x > 0) \{{}^2 x = x - 1; {}^3 y = 1;} 
 \notin \mathcal{D}^1(z, y)
 \]
 %
 By Definition~(4) (in \cite{cousot2019abstract}), we cannot derive:
 \[
 	x \rightsquigarrow^{1}_{\ewhile {}^1 (x > 0) \{{}^2 x = x - 1; {}^3 y = 1;\}} y
 \]
 %
 %
 \end{proof}
 \end{example}
 %
 \clearpage
 \begin{example}[Counter Example of Including Timing Channel].
 \label{ex:overapp}
 \\
 If $\mathsf{diff}(\omega, \omega')$ (in \cite{cousot2019abstract} Equation~(2)) simply includes timing channel,(i.e., $\omega$ is a strict prefix of $\omega'$) as follows:
 \[
 	\mathsf{diff}(\omega, \omega') \triangleq \exists \omega_0, \omega_1, \omega_1', v, v' 
 	\st \bigvee \left\{
 	\begin{array}{lr}
 	(\omega = \omega_0 \cdot v \omega_1
 		\land \omega' = \omega_0 \cdot v' \omega_1' \land v \neq v') & \mbox{original definition} \\
 	(\omega = \omega' \cdot v \cdot \omega_1) & \mbox{including timing channel} \\
 	\end{array}
 	\right\}
 \] 
 then by Definition~2 (in \cite{cousot2019abstract}), there is a over approximation example:
 \[
 	\ewhile {}^2 (x > 0) {}^3 y = 1; 
 \]
 Let $z \in \mathbb{V}\setminus \{x\}$ be arbitrary variable different from $x$,
 in this example, $y$ doesn't rely on $z$. 
 However, according to value dependency defined in Definition~2 \cite{cousot2019abstract} we can derive 
 \[
 	z \rightsquigarrow^{2}_{\ewhile {}^2 (x > 0) {}^3 y = 1;} y
 \]
 %
 \begin{proof}
 By semantics definition, we have:
 %
 \begin{equation}
 \label{eq:sem_timingoverapp}
 \begin{array}{ll}
 & \sem{ \ewhile {}^2 (x > 0) {}^3 y = 1; } \triangleq \\
 & \left\{ \left< \pi_1^2, ^2\right> \mid \pi_1^2 \in \mathbb{T^{+}} \right\}
 \\
 \cup & \left\{ \left< \pi_1^2, {}^2 \xrightarrow{(x > 0)} {}^3 \xrightarrow{(y = 1) = 1} {}^2 \right> 
 \mid \pi_i^2 \in \mathbb{T^{+}} \land 
 \sem{(x > 0)}\env(\pi_1^2) = \etrue  \land
 \sem{1}\env(\pi_1^2 \xrightarrow{(x > 0)} {}^3) = 1 \right\}
 \\
 \cup & \left\{ \left< \pi_1^2, {}^2 \xrightarrow{(x > 0)} {}^3 \xrightarrow{(y = 1) = 1} {}^2 \xrightarrow{(x > 0)} {}^3 \xrightarrow{(y = 1) = 1} {}^2 \right> 
 \left \vert 
 \begin{array}{l}
 \pi_i^2 \in \mathbb{T^{+}} \\
 \land 
 \sem{(x > 0)}\env(\pi_1^2) = \etrue  \\
 \land
 \sem{1}\env(\pi_1^2 \xrightarrow{(x > 0)} {}^3) = 1 \\
 \land 
 \sem{(x > 0)}\env(\pi_1{}^2 \xrightarrow{(x > 0)} {}^3 \xrightarrow{(y = 1) = 1} {}^2) = \etrue \\
 \land
 \sem{1}\env(\pi_1^2 \xrightarrow{(x > 0)} {}^3 \xrightarrow{(y = 1) = 1} {}^2 \xrightarrow{(x > 0)} {}^3) = 1
 \end{array}
 \right.
 \right\}
 \\
 \cup & \cdots 
 \\
 \cup & \left\{ \left< \pi_1^2, {}^2 \xrightarrow{(x > 0)} {}^3\xrightarrow{(x = x - 1) = v_1} {}^2 \cdots  {}^2 \xrightarrow{\neg(x > 0)} {}^l \right>  
 \left \vert 
 \begin{array}{l}
 \pi_1^2 \in \mathbb{T^{+}}\\
 \land 
 \sem{(x > 0)}\env(\pi_1^2) = \etrue  \\
 \land
 \sem{1}\env(\pi_1^2 \xrightarrow{(x > 0)} {}^3) = v_1 \\
 \land \cdots \\
 % \land
 % \sem{(x - 1)}\env(\pi_1^2 \xrightarrow{(x > 0)} {}^3 \xrightarrow{(x = x - 1) = v} {}^2 \xrightarrow{(x > 0)} {}^3) = v_2\\
 \land 
 \sem{(x > 0)}\env(\pi_i{}^2 \xrightarrow{(x > 0)} {}^3\xrightarrow{(y = 1) = 1} {}^2 \cdots {}^2) = \efalse \\
 \land l = \kw{aft}[\ewhile {}^2 (x > 0) {}^3 y = 1;]
 \end{array}
 \right.
 \right\}
 \end{array}
 \end{equation}
 %
 %
 Let $z \in \mathbb{V}\setminus \{x\}$ be arbitrary variable different from $x$,
 pick $\pi_0 = {}^0 \xrightarrow{z = 1} {}^1 \xrightarrow{x = 2} {}^2$, 
 $\pi_0' = {}^0 \xrightarrow{z = 2} {}^1 \xrightarrow{x = 2} {}^2$:
 \[
 	\left< \pi_0^2, {}^2 \xrightarrow{(x > 0)} {}^3 \xrightarrow{(y = 1) = 1} {}^2 \right> \in \sem{\ewhile {}^2 (x > 0) {}^3 y = 1;}
 \]
 %
 \[
 	\left< \pi_0'^2, {}^2 \xrightarrow{(x > 0)} {}^3 \xrightarrow{(y = 1) = 1} {}^2 \xrightarrow{(x > 0)} {}^3 \xrightarrow{(y = 1) = 1}{}^2\right> \in \sem{\ewhile {}^2 (x > 0) {}^3 y = 1; }
 \]
 %
 By Definition of $\mathsf{seqval}$, we have:
 \[
 	\mathsf{seqval}\sem{y}{}^2(\pi_0^2, {}^2 \xrightarrow{(x > 0)} {}^3 \xrightarrow{(y = 1) = 1} {}^2 ) = 1 \cdot
 \]
 %
 \[
 	\mathsf{seqval}\sem{y}{}^2(\pi_0'^2, {}^2 \xrightarrow{(x > 0)} {}^3 \xrightarrow{(y = 1) = 1} {}^2 \xrightarrow{(x > 0)} {}^3 \xrightarrow{(y = 1) = 1}{}^2 ) = 1 \cdot 1 \cdot
 \]
 %
 If $\mathsf{diff}(\omega, \omega')$ (in \cite{cousot2019abstract} Equation~(2) ) includes timing channel, i.e., 
 %
 \[
 	\mathsf{diff}(1 \cdot, 1 \cdot 1 \cdot) = \etrue
 \]
 %
 By Definition~2 (in \cite{cousot2019abstract}), 
 \[
 	\mathcal{S}^{+\infty}\sem{\ewhile {}^2 (x > 0) {}^3 y = 1;} 
 \in \mathcal{D}^2(z, y)
 \].
 %
 By Definition~(4) (in \cite{cousot2019abstract}):
 \[
 	z \rightsquigarrow^{2}_{\ewhile {}^2 (x > 0) {}^3 y = 1;} y
 \]
 %
 However, intuitively, the execution of $y$ doesn't relies on value of $z$ at all. This is a false positive example (or over approximation) of dependency.
 %
 \end{proof}
 \end{example}
 %
 \clearpage
 \clearpage
 \begin{example}[Approach-1 of Including Timing Channel]
 Modifying the Definition~2 (in \cite{cousot2019abstract}) as follows:
 \begin{defn}[Modified Data Dependency].
 \label{defn:mdfy_dep}
 \[
 	\begin{array}{lll}
 	\mathcal{D}^l(x, y) & \triangleq &
 	\Big\{ 
 	\Pi \in (\mathbb{T}^{+} \times \mathbb{T}^{+\infty})
 	\exists \config{\pi_1, \pi_2}, \config{\pi_1', \pi_2'} \in \Pi \st \\
 	&& \quad \big(
 	\forall z \in \mathbb{V}\setminus \{x\} \st \env(\pi_1)(z) = \env(\pi_1') \st \\
 	&& \quad  \quad \land \neg(\mathsf{diff}(\mathsf{seqval}\sem{y}{}^1(\pi_1, \pi_2 ), \mathsf{seqval}\sem{y}{}^1(\pi'_1, \pi_2')))
 	\big) \Big\} \\
 	& \cup & \Big\{ \Pi \in (\mathbb{T}^{+} \times \mathbb{T}^{+\infty})
 	\exists \config{\pi_1, \pi_2}, \config{\pi_1', \pi_2'} \in \Pi \st \forall \config{\pi_1', \pi_2''} \in \Pi\\
 	&& \quad \big(
 	\forall z \in \mathbb{V}\setminus \{x\} \st \env(\pi_1)(z) = \env(\pi_1') \st \\
 	&& \quad  \quad  \land |\mathsf{seqval}\sem{y}{}^1(\pi_1', \pi_2'')| \leq |\mathsf{seqval}\sem{y}{}^1(\pi_1', \pi_2')| <  
 	|\mathsf{seqval}\sem{y}{}^1(\pi_1, \pi_2)|\big) \Big\}
 	\end{array}
 \]
 \end{defn}
 This definition is capable to include the timing channel(control dependency) in Example~\ref{ex:excltiming},
 as well as exclude the over approximation in Example~\ref{ex:overapp}.
 %
 \begin{proof}
 \caseL{Including the Example~\ref{ex:excltiming}}
 By Example~\ref{ex:excltiming}, we have its semantics in Equation~\ref{eq:sem_timingdep}.
 %
 %
 Let $\pi_1^1, \pi_1'^1 \in \mathbb{T^{+}}$ be arbitrary traces s.t. 
 %
 $$\forall z \in \mathbb{V}\setminus \{x\}. ~ \env(\pi_1^1)(z) = \env(\pi_1'^1) $$ 
 %
 Let $v = \env(\pi_1^1)(x) $, $v' = \env(\pi_1'^1)(x')$, without lose of generalization, let $0 \leq v' < v$.
 \\
 Let 
 $t_i = {}^2 \xrightarrow{(x = x - 1) = v - i } {}^3 \xrightarrow{(y = 1) = 1} {}^1 \cdots  {}^1$, for $i = 1, \cdots, v$
 %
 $\pi_2 = {}^1 \xrightarrow{(x > 0)} {}^2 t_1 {}^1 \cdots   {}^2 t_v {}^1 \xrightarrow{\neg(x > 0)} {}^l $,
 %
  By Equation~\ref{eq:sem_timingdep}, we know:
 \[
 	\left< \pi_1^1, \pi_2 \right> \in \sem{\ewhile {}^1 (x > 0) \{{}^2 x = x - 1; {}^3 y = 1;\} }
 \]
 %
 Let 
 $t_i' = {}^2 \xrightarrow{(x = x - 1) = v' - i } {}^3 \xrightarrow{(y = 1) = 1} {}^1 \cdots  {}^1$, for $i = 1, \cdots, v'$
 %
 $\pi_2' = {}^1 \xrightarrow{(x > 0)} {}^2 t'_1 {}^1 \cdots  {}^2 t'_{v'} {}^1 \xrightarrow{\neg(x > 0)} {}^l $,
 %
  By Equation~\ref{eq:sem_timingdep}, we know:
 \[
 	\left< \pi_1'^1, \pi'_2 \right> \in \sem{\ewhile {}^1 (x > 0) \{{}^2 x = x - 1; {}^3 y = 1;\} }
 \]
 %
 \\
 By Definition of $\mathsf{seqval}$, we have:
 \[
 	|\mathsf{seqval}\sem{y}{}^1(\pi_1^1, \pi_2 )| = |\underbrace{0 \cdots 1}_{0 \leq v}| = v + 1
 \]
 %
 \[
 	|\mathsf{seqval}\sem{y}{}^1(\pi_1^1, \pi_2' )| = |\underbrace{0 \cdots 1}_{0 \leq v'}| = v' + 1
 \]
 %
 Since, $v' < v $, we have: 
 \[
 	|\mathsf{seqval}\sem{y}{}^1(\pi_1^1, \pi_2' )| < |\mathsf{seqval}\sem{y}{}^1(\pi_1^1, \pi_2 )|
 \]
 %
 Let $\Pi_2''$ defined as follows:
 \[
 	\Pi''_2 \triangleq \{
 	\left< \pi_1'^1, \pi''_2 \right> \vert
 	\pi''_2 =  {}^1 \xrightarrow{(x > 0)} {}^2 t'_1 {}^1 \cdots  {}^2 t'_{j} {}^1, \forall j < v'\} 
 \]
 %
 By Equation~\ref{eq:sem_timingdep}, we know:
 %
 \[
 		\Pi''_2 \subseteq \sem{\ewhile {}^1 (x > 0) \{{}^2 x = x - 1; {}^3 y = 1;\} }
 		\land
 		\forall \left< \pi_1', \pi''_2 \right> \in \sem{\ewhile {}^1 (x > 0) \{{}^2 x = x - 1; {}^3 y = 1;\} } 
 		\st \left< \pi_1', \pi''_2 \right> \in  \Pi''_2
 \]
 %

 %
 %
 Then for arbitrary $\left< \pi_1', \pi''_2 \right> \in \sem{\ewhile {}^1 (x > 0) \{{}^2 x = x - 1; {}^3 y = 1;\} }$, we know:
 \[
 	\exists \st \pi''_2 =  {}^1 \xrightarrow{(x > 0)} {}^2 t'_1 {}^1 \cdots  {}^2 t'_{j} {}^1 \land j \leq v'
 \]
 %
 \[
 	|\mathsf{seqval}\sem{y}{}^1(\pi_1^1, \pi_2'' )| = |\underbrace{0 \cdots 1}_{0 \leq j}| = j \leq v'
 \]
 %
 Then we have
 \[
 \begin{array}{l}
 	\forall \left< \pi_1'^1, \pi''_2 \right> \in \mathcal{S}^{+\infty} \sem{\ewhile {}^2 (x > 0) {}^3 y = 1;} \st
 	\Big(
 	|\mathsf{seqval}\sem{y}{}^1(\pi_1'^1, \pi_2'' )| \leq v' < v))\Big)
 \end{array}
 \] 
 %
 i.e.,
 \[
 	\begin{array}{l}
 	\exists \config{\pi_1, \pi_2}, \config{\pi_1', \pi_2'} \in \mathcal{S}^{+\infty}\sem{\ewhile {}^1 (x > 0) \{{}^2 x = x - 1; {}^3 y = 1;} \\
 	\quad \st \forall \config{\pi_1', \pi_2''} \in \mathcal{S}^{+\infty}\sem{\ewhile {}^1 (x > 0) \{{}^2 x = x - 1; {}^3 y = 1;} \\
 	\quad \big(
 	\forall z \in \mathbb{V}\setminus \{x\} \st \env(\pi_1)(z) = \env(\pi_1') \st \\
 	\quad  \quad  \land |\mathsf{seqval}\sem{y}{}^1(\pi_1', \pi_2'')| 
 	\leq |\mathsf{seqval}\sem{y}{}^1(\pi_1', \pi_2')| <  
 	|\mathsf{seqval}\sem{y}{}^1(\pi_1, \pi_2)|\big) 	
 	\end{array}
 \]
 \\
 %
 By modified data dependency in Definition~\ref{defn:mdfy_dep} from Definition~2 (in \cite{cousot2019abstract}), we have:
 \[
 	\mathcal{S}^{+\infty}\sem{\ewhile {}^1 (x > 0) \{{}^2 x = x - 1; {}^3 y = 1;} 
 	\in \mathcal{D}^1(x, y)
 \]
 %
 By Definition~(4) (in \cite{cousot2019abstract}), we have:
 \[
 	x \rightsquigarrow^{1}_{\ewhile {}^1 (x > 0) \{{}^2 x = x - 1; {}^3 y = 1;\}} y
 \]
 %
 \caseL{Excluding the Over Approximated Example~\ref{ex:overapp}}
 %
 Let $z \in \mathbb{V}\setminus \{x\}$ be arbitrary variable different from $x$, it is sufficient to show,
 by the modified data dependency in Definition~\ref{defn:mdfy_dep}, the following dependency cannot be derived:
 \[
 	z \rightsquigarrow^{2}_{\ewhile {}^2 (x > 0) {}^3 y = 1;} y
 \]
 Take arbitrary $\config{\pi_1, \pi_2}, \config{\pi_1', \pi_2'}$ s.t.:
 \[
 	\config{\pi_1, \pi_2}, \config{\pi_1', \pi_2'} \in \mathcal{S}^{+\infty}\sem{\ewhile {}^1 (x > 0) \{{}^2 x = x - 1; {}^3 y = 1;}  
 	\land \forall u \in \mathbb{V}\setminus \{z\} \st \env(\pi_1)(u) = \env(\pi_1')(u)
 	\]
 Then we know:
 \[
 	\env(\pi_1)(x) = \env(\pi_1')(x)
 \]
 Let $v = \env(\pi_1)(x) = \env(\pi_1')(x)$
 %
 By Definition of $\mathsf{seqval}$, we have:
 \[
 	|\mathsf{seqval}\sem{y}{}^2(\pi_1, \pi_2 )| = | 0 \cdot 1 \cdots| \leq v
 	\land
 	|\mathsf{seqval}\sem{y}{}^2(\pi_1', \pi_2' )| = | 0 \cdot 1 \cdots| \leq v
 \]
 %
 Without loss of generalization, let $\config{\pi_1', \pi_2'} \in \mathcal{S}^{+\infty}\sem{\ewhile {}^1 (x > 0) \{{}^2 x = x - 1; {}^3 y = 1;} $ s.t.
 \[
 |\mathsf{seqval}\sem{y}{}^2(\pi_1', \pi_2' )| = | 0 \cdot 1 \cdots 1| < v
 \]
 %
 There always exists $\config{\pi_1', \pi_2''} \in \mathcal{S}^{+\infty}\sem{\ewhile {}^1 (x > 0) \{{}^2 x = x - 1; {}^3 y = 1;}$ s.t.:
 \[
 	|\mathsf{seqval}\sem{y}{}^2(\pi_1', \pi_2'' )| = | 0 \cdot 1 \cdots 1| = v
 \]
 %
 Then we have:
 \[
 	\begin{array}{l}
 	\not\exists \config{\pi_1, \pi_2}, \config{\pi_1', \pi_2'} \in \mathcal{S}^{+\infty}\sem{\ewhile {}^1 (x > 0) \{{}^2 x = x - 1; {}^3 y = 1;} \\
 	\quad \st \forall \config{\pi_1', \pi_2''} \in \mathcal{S}^{+\infty}\sem{\ewhile {}^1 (x > 0) \{{}^2 x = x - 1; {}^3 y = 1;} \\
 	\quad \big(
 	\forall z \in \mathbb{V}\setminus \{x\} \st \env(\pi_1)(z) = \env(\pi_1') \st \\
 	\quad  \quad  \land |\mathsf{seqval}\sem{y}{}^1(\pi_1', \pi_2'')| 
 	\leq |\mathsf{seqval}\sem{y}{}^1(\pi_1', \pi_2')| <  
 	|\mathsf{seqval}\sem{y}{}^1(\pi_1, \pi_2)|\big) 	
 	\end{array}
 \]
 %
 By modified data dependency in Definition~\ref{defn:mdfy_dep},
 \[
 	\mathcal{S}^{+\infty}\sem{\ewhile {}^2 (x > 0) {}^3 y = 1;} 
 	\notin \mathcal{D}^2(z, y)
 \].
 %
 By Definition~(4) (in \cite{cousot2019abstract}), the following dependency cannot be derived.
 \[
 	z \rightsquigarrow^{2}_{\ewhile {}^2 (x > 0) {}^3 y = 1;} y
 \]
 %
 %
 \end{proof}
 \end{example}
% \clearpage
% %
\section{{\tt Labeled While} Language}
%
%
\subsection{Labeled Language}
\mg{It is ok to list all the operations in the appendix but for the main paper it is better to save space.}
\[
\begin{array}{llll}
\mbox{Arithmetic Operators} 
& \oplus_a & ::= & + ~|~ - ~|~ \times 
%
~|~ \div ~|~ \max ~|~ \min\\  
% \mbox{Unary Operators} 
% & \oplus_a & ::= & + ~|~ - ~|~ \times 
% %
% ~|~ \div \\  
\mbox{Boolean Operators} 
& \oplus_b & ::= & \lor ~|~ \land
\\
%
\mbox{Relational Operators} 
& \sim & ::= & < ~|~ \leq ~|~ == 
\\  
%
\mbox{Label} 
& l & \in & \mathbb{N} \mg{\text{This should be} \in \text{since l is a label and not a set.}}
\\ 
%
\mbox{Arithmetic Expression} 
& \aexpr & ::= & 
n ~|~ {x} ~|~ \aexpr \oplus_a \aexpr  
\\
&  &  & 
 ~|~ \elog \aexpr  ~|~ \esign \aexpr
\\
%
\mbox{Boolean Expression} & \bexpr & ::= & 
%
\etrue ~|~ \efalse  ~|~ \neg \bexpr
 ~|~ \bexpr \oplus_b \bexpr
%
~|~ \aexpr \sim \aexpr 
\\
%
\mbox{Expression} & \expr & ::= & v ~|~ \aexpr \sep \bexpr ~|~ [\expr, \dots, \expr]
\\  
%
\mbox{Value} 
& v & ::= & { n \sep \etrue \sep \efalse ~|~ [] ~|~ [v, \dots, v]}  
\\
%
\mbox{Query Expression} 
& {\qexpr} & ::= 
& { \qval ~|~ \aexpr ~|~ \qexpr \oplus_a \qexpr ~|~ \chi[\aexpr]} 
\\
%
\mbox{Query Value} & \qval & ::= 
& {n ~|~ \chi[n] ~|~ \chi[n] \oplus_a  \chi[n] ~|~ n \oplus_a  \chi[n]
    ~|~ \chi[n] \oplus_a  n}\\
&&& \text{\mg{I don’t think this is what we want. Isn’t $\chi[n+1]$ a query value?}}\\
&&& \text{\mg{What about $\chi[i] + \chi[i] + \chi[i]$? They are not in the grammar}}
\\
&&& \text{\jl{ $\chi[i] + \chi[i] + \chi[i]$ and $\chi[n+1]$ are both expressions, they will be evaluated to a value 
}}
\\%
\mbox{Labeled Command} 
& {c} & ::= &   [\assign {{x}}{ {\expr}}]^{l} ~|~  [\assign {{x} } {{\query(\qexpr)}}]^{l}
~|~ {\ewhile [ \bexpr ]^{l} \edo {c} }
\\
&&&
~|~ {c};{c}  
~|~ \eif([\bexpr]{}^l , {c}, {c}) 
~|~ [\eskip]^l\\ 
&&& \text{\mg{Is skip missing a label?}}
\\
&&& \text{\jl{I intentionally removed leabel for skip before}}
\\%
&&& \text{\jl{but I realized I need it after implementation}}
\\%
&&& \text{\jl{I haven't yet updated it here, thanks for reminding}}
\\% \mbox{Event} 
% & \event & ::= & 
%     ({x}, l, v) ~|~ ({x}, l, \qval, v)  ~~~~~~~~~~~ \mbox{Assignment Event} \\
% &&& ~|~(\bexpr, l, v)   ~~~~~~~~~~~~~~~~~~~~~~~~~~~~~~~~~~ \mbox{Testing Event}
% \\
\mbox{Event} 
& \event & ::= & 
    ({x}, l, v, \bullet) ~|~ ({x}, l, v, \qval)  ~~~~~~~~~~~ \mbox{Assignment Event} \\
&&& ~|~(\bexpr, l, v, \bullet)   ~~~~~~~~~~~~~~~~~~~~~~~~~~~~~~~~~~ \mbox{Testing Event}
\\
&&& \text{\mg{I think it would be better to use quadruples for events, where the}}\\
&&& \text{\mg{first element is either a variable or a boolean expression and }}\\
&&& \text{\mg{the last is either a query value or some default value $\bullet$}}\\
%
% \mbox{Trace} & \trace
% & ::= & \cdot | \trace \cdot \event | \trace \tracecat \trace 
% \\
%
% \mbox{Trace} & \trace
% & ::= & [] ~|~ \event:: \trace ~|~ \trace \tracecat \trace  \\
\mbox{Trace} & \trace
& ::= & [] ~|~ \trace :: \event\\
&&& \text{\mg{I don't understand why you need both :: and ++ as constructors.}}\\
&&& \text{\jl{Because append is to the left but we are adding element to the left in the OS}}\\
&&& \text{\jl{I was too sticky to the convention, it is a good idea to append to the left and just use $::$}}
% %
% \mbox{Event Signature} & \sig
% & ::= & (x, l, n) | (x, l, n, \query) | (b, l, n)
% \\
% %
\end{array}
\]
% \todo{change trace notation into list, and update corresponding operator nations}
% \\
% \wqside{"$\cdot$" has two meanings? empty, delimit. Trace is list of event?}
We use following notations to represent the set of corresponding terms:
\[
\begin{array}{lll}
\mathcal{VAR} & : & \mbox{Set of Variables}  
\\ 
%
\mathcal{VAL} & : & \mbox{Set of Values} 
\\ 
%
\mathcal{QVAL} & : & \mbox{Set of Query Values} 
\\ 
%
\cdom & : & \mbox{Set of Commands} 
\\ 
%
\eventset  & : & \mbox{Set of Events}  
\\
%
\eventset^{\asn}  & : & \mbox{Set of Assignment Events}  
\\
%
\eventset^{\test}  & : & \mbox{Set of Testing Events}  
\\
%
%%
\dbdom  & : & \mbox{{Set of Databases}} 
\\
%
{\mathcal{T}} & : & \mbox{Set of Traces}
\\
%
% \qdom = {[-1,1]} & : & \mbox{{Domain of Query Results}}\\
\qdom & : & \mbox{{Domain of Query Results}}\\
&&\text{\mg{I don't think you need to hard code [-1,1] here}}\\
\end{array}
\]
%
%
%
Environment $ \env : {\mathcal{T}}  \to \mathcal{VAR} \to \mathcal{VAL} \cup \{\bot\}$
\mgside{The following definition is missing one case, also it is better to say that $y\neq x$.}
% \[
% \begin{array}{lll}
% \env(\trace  \tracecat [(x, l, v, \cdot)]) x \triangleq v
% &
% \env(\trace \tracecat [(y, l, v, \cdot)]) x \triangleq \env(\trace) x, y \neq x
% &
% \env(\trace \tracecat [(b, l, v, \cdot)]) x \triangleq \env(\trace) x
% \\
% \env(\trace \tracecat [(x, l, v, \qval)]) x \triangleq v
% &
% \env(\trace \tracecat [(y, l, v, \qval)]) x \triangleq \env(\trace) x, y \neq x
% &
% \env({[]} ) x \triangleq \bot
% \end{array}
% \]
\[
\begin{array}{lll}
\env(\trace  \traceadd (x, l, v, \bullet)) x \triangleq v
&
\env(\trace \traceadd (y, l, v, \bullet)) x \triangleq \env(\trace) x, y \neq x
&
\env(\trace \traceadd (b, l, v, \bullet)) x \triangleq \env(\trace) x
\\
\env(\trace \traceadd (x, l, v, \qval)) x \triangleq v
&
\env(\trace \traceadd (y, l, v, \qval)) x \triangleq \env(\trace) x, y \neq x
&
\env({[]} ) x \triangleq \bot
\end{array}
\]
%
%
% Query Environment $\qenv: \mathcal{T}  \to \mathcal{SVAR} \to \mathcal{QVAL} \cup \{\bot\}$
% \[
% \begin{array}{lll}
% \qenv(\trace \cdot (x, l, n, v) ) x \triangleq \qenv(\trace) x
% &
% \qenv(\trace \cdot (x, l, n, \qval, v) ) x \triangleq \qval
% &
% \qenv(\trace \cdot (y, l, n, v)) x \triangleq \env(\trace) x
% \\
% \qenv(\trace \cdot (b, l, n, v)) x \triangleq \env(\trace) x
% &
% \qenv(\cdot ) x \triangleq \bot
% &
% \end{array}
% \]
%

%
% \subsection{Trace-based Operational Semantics for Language \mg{What is ``Language''?}}
\subsection{Trace-based Operational Semantics for {\tt Labeled While} Language}
%
%
%
{
\begin{mathpar}
\boxed{ \config{\trace,\aexpr} \aarrow v \, : \, \mbox{Trace  $\times$ Arithmetic Expr $\Rightarrow$ Arithmetic Value} }
\\
\text{\mg{Missing. Without these rules it is difficult to understand why we need a trace to evaluate expressions.}}
\\
\inferrule{ 
  \empty
}{
 \config{\trace,  n} 
 \aarrow n
}
\and
\inferrule{ 
  \env(\trace) x = v
}{
 \config{\trace,  x} 
 \aarrow v
}
\and
\inferrule{ 
  \config{\trace, \aexpr_1} \aarrow v_1
  \and 
  \config{\trace, \aexpr_2} \aarrow v_2
  \and 
   v_1 \oplus_a v_2 = v
}{
 \config{\trace,  \aexpr_1 \oplus_a \aexpr_2} 
 \aarrow v
}
\and
\inferrule{ 
  \config{\trace, \aexpr} \aarrow v'
  \and 
  \elog v' = v
}{
 \config{\trace,  \elog \aexpr} 
 \aarrow v
}
\and
\inferrule{ 
  \config{\trace, \aexpr} \aarrow v'
  \and 
  \esign v' = v
}{
 \config{\trace,  \esign \aexpr} 
 \aarrow v
}
\\
\boxed{ \config{\trace, \bexpr} \barrow v \, : \, \mbox{Trace $\times$ Boolean Expr $\Rightarrow$ Boolean Value} }
\\
\text{\mg{Missing. Without these rules it is difficult to understand why we need a trace to evaluate expressions.}}
\\
\inferrule{ 
  \empty
}{
 \config{\trace,  \efalse} 
 \barrow \efalse
}
\and 
\inferrule{ 
  \empty
}{
 \config{\trace,  \etrue} 
 \barrow \etrue
}
\and 
\inferrule{ 
  \config{\trace, \bexpr} \barrow v'
  \and 
  \neg v' = v
}{
 \config{\trace,  \neg \bexpr} 
 \barrow v
}
\and 
\inferrule{ 
  \config{\trace, \bexpr_1} \barrow v_1
  \and 
  \config{\trace, \bexpr_2} \barrow v_2
  \and 
   v_1 \oplus_b v_2 = v
}{
 \config{\trace,  \bexpr_1 \oplus_b \bexpr_2} 
 \barrow v
}
\and 
\inferrule{ 
  \config{\trace, \aexpr_1} \aarrow v_1
  \and 
  \config{\trace, \aexpr_2} \aarrow v_2
  \and 
   v_1 \sim v_2 = v
}{
 \config{\trace,  \aexpr_1 \sim \aexpr_2} 
 \barrow v
}
\\
\boxed{ \config{\trace, \expr} \earrow v \, : \, \mbox{Trace $\times$ Expression $\Rightarrow$ Value} }
\\
\inferrule{ 
  \config{\trace, \aexpr} \aarrow v
}{
 \config{\trace,  \aexpr} 
 \earrow v
}
\and
\inferrule{ 
  \config{\trace, \bexpr} \barrow v
}{
 \config{\trace,  \bexpr} 
 \earrow v
}
\and
\inferrule{ 
  \config{\trace, \expr_1} \earrow v_1
  \cdots
  \config{\trace, \expr_n} \earrow v_n
}{
 \config{\trace,  [\expr_1, \cdots, \expr_n]} 
 \earrow [v_1, \cdots, v_n]
}
\and
\inferrule{ 
  \empty
}{
 \config{\trace,  v} 
 \earrow v
}
\\
\boxed{ \config{\trace, \qexpr} \qarrow \qval \, : \, \mbox{Trace  $\times$ Query Expr $\Rightarrow$ Query Value} }
\\
\inferrule{ 
  \config{\trace, \aexpr} \aarrow n
}{
 \config{\trace,  \aexpr} 
 \qarrow n
}
\and
\inferrule{ 
  \config{\trace, \qexpr_1} \qarrow \qval_1
  \and
  \config{\trace, \qexpr_2} \qarrow \qval_2
}{
 \config{\trace,  \qexpr_1 \oplus_a \qexpr_2} 
 \qarrow \qval_1 \oplus_a \qval_2
}
\and
\inferrule{ 
  \config{\trace, \aexpr} \aarrow n
}{
 \config{\trace, \chi[\aexpr]} \qarrow \chi[n]
}
\and
\inferrule{ 
  \empty
}{
 \config{\trace,  \qval} 
 \qarrow \qval
}
 \end{mathpar}
%
The trace based operational semantics \mg{rules} are defined in Figure \ref{fig:os_ssa}.
%
\begin{figure}
  \text{\mg{Several skip are missing labels. Do we need fresh labels or we reuse l?}}
  \\
  \text{\jl{Both are good for OS, but generate fresh label will need extra arguments in soundness proof, so rescuing l is better}}
  \\
\text{\mg{Also, why we use ++, cannot we just define lists as adding elements on the right?}}  \\
\text{\jl{I was too sticky to the convention, it is a good idea to append to the left and just use $::$ as construtor}}  \\
\text{\mg{It is also unclear why we store the boolean expression in if and while, besides the boolean value.}}\\
\text{\jl{When proving the soundness of dependency between trace-based and program-based,}}\\
\text{\jl{The variable used in the boolean expression is useful in proving the inversion Lemmas.}}
{
\begin{mathpar}
\boxed{
\mbox{Command $\times$ Trace}
\xrightarrow{}
\mbox{Command $\times$ Trace}
}
\and
\boxed{\config{{c, \trace}}
\xrightarrow{} 
\config{{c',  \trace'}}
}
\\
\inferrule
{
\empty
}
{
\config{\clabel{\eskip}^l,  \trace } 
\xrightarrow{} 
\config{\clabel{\eskip}^l, \trace}
}
~\textbf{skip}
%
\and
%
\inferrule
{
\event = ({x}, l, v, \bullet)
}
{
\config{[\assign{{x}}{\aexpr}]^{l},  \trace } 
\xrightarrow{} 
\config{\clabel{\eskip}^l, \trace \traceadd \event}
}
~\textbf{assn}
%
\and
%
{
\inferrule
{
 \trace, \qexpr \qarrow \qval
 \and 
\query(\qval) = v
\and 
\event = ({x}, l, v, \qval)
}
{
\config{{[\assign{x}{\query(\qexpr)}]^l, \trace}}
\xrightarrow{} 
\config{{\clabel{\eskip}^l,  \trace \traceadd \event} }
}
~\textbf{query}
}
%
\and
%
\inferrule
{
 \trace, b \barrow \etrue
 \and 
 \event = (b, l, \etrue, \bullet)
}
{
\config{{\ewhile [b]^{l} \edo c, \trace}}
\xrightarrow{} 
\config{{
c; \ewhile [b]^{l} \edo c),
\trace \traceadd \event}}
}
~\textbf{while-t}
%
%
\and
%
\inferrule
{
 \trace, b \barrow \efalse
 \and 
 \event = (b, l, \efalse, \bullet)
}
{
\config{{\ewhile [b]^{l}, \edo c, \trace}}
\xrightarrow{} 
\config{{
  \clabel{\eskip}^l,
\trace \traceadd \event}}
}
~\textbf{while-f}
%
%
\and
%
%
\inferrule
{
\config{{c_1, \trace}}
\xrightarrow{}
\config{{c_1',  \trace'}}
}
{
\config{{c_1; c_2, \trace}} 
\xrightarrow{} 
\config{{c_1'; c_2, \trace'}}
}
~\textbf{seq1}
%
\and
%
\inferrule
{
  \config{{c_2, \trace}}
  \xrightarrow{}
  \config{{c_2',  \trace'}}
}
{
\config{{\clabel{\eskip}^l; c_2, \trace}} \xrightarrow{} \config{{ c_2', \trace'}}
}
~\textbf{seq2}
%
\and
%
%
\inferrule
{
   \trace, b \barrow \etrue
 \and 
 \event = (b, l, \etrue, \bullet)
}
{
 \config{{
\eif([b]^{l}, c_1, c_2), 
\trace}}
\xrightarrow{} 
\config{{c_1, \trace \traceadd \event}}
}
~\textbf{if-t}
%
\and
%
\inferrule
{
 \trace, b \barrow \efalse
 \and 
 \event = (b, l, \efalse, \bullet)
}
{
\config{{\eif([b]^{l}, c_1, c_2), \trace}}
\xrightarrow{} 
\config{{c_2, \trace \traceadd \event}}
}
~\textbf{if-f}
% %
%
%
\end{mathpar}
}
% \end{subfigure}
    \caption{Trace-based Operational Semantics for Language.}
    \label{fig:os}
\end{figure}
%
%
%
%
%
%
%
\clearpage

\clearpage
% %
\section{Event and Trace}
%
%
\subsection{Event}
%
Event Value : $\pi_v : \eventset \to \mathcal{VAL}$
\[
\begin{array}{ll}
\pi_v((x, l, n, v)) \triangleq v
&
\pi_v (x, l, n, \qval, v) \triangleq v
\\
\pi_v (b, l, n, v)  \triangleq v
&
\end{array}
\]
%
Event Query Value : $\pi_{q} : \eventset \to \mathcal{VAL}$
\[
\begin{array}{ll}
\pi_{q} (x, l, n, v) \triangleq v
&
\pi_{q} (x, l, n, \qval, v) \triangleq \qval
\\
\pi_{q} (b, l, n, v)  \triangleq v
&
\end{array}
\]%
% 
Event Signature : $\pi_{\sig} : \eventset \to \mathcal{VAL}$
\[
\begin{array}{ll}
\pi_{\sig} (x, l, n, v) \triangleq (x, l, n)
&
\pi_{\sig} (x, l, n, \qval, v) \triangleq (x, l, n, \query)
\\
\pi_{\sig} (b, l, n, v)  \triangleq (b, l, n)
&
\end{array}
\]
%
%
\begin{defn}[Equivalence of Query].
%
\label{def:query_equal}
Given a trace $\trace$ and 2 query expressions $\qexpr_1$, $\qexpr_2$:
$$
\qexpr_1 =_{q}^{\trace} \qexpr_2 \triangleq
\left\{
    \begin{array}{ll} 
      \etrue      
      & 
    \exists \qval_1, \qval_2.
    \begin{array}{l} 
      (\config{\trace,  \qexpr_1} \qarrow \qval_1 \land \config{\trace,  \qexpr_2 } \qarrow \qval_2) 
      \\
      \land (\forall r \in \qdom. \exists v. ~ s.t., ~ 
            \config{\trace, \qval_1[r/\chi]} \aarrow v \land \config{\trace,  \qval_2[r/\chi] } \aarrow v)  
    \end{array}\\
      \efalse         
      & \text{o.w.} 
    \end{array}
    \right.
$$
%
, where $FV(\qexpr)$ is the set of free variables in the query expression $\qexpr$.
$\qexpr_1 \neq_{q}^{\trace} \qexpr_2$  is defined vice versa.
%
We use $=_{q}$  and $\neq_{q}$ as the shorthands when $\trace$ is empty.
\end{defn}
%
Equivalence of 2 events : $\event_1 \eventeq \event_2$
\[
\event_1 \eventeq \event_2 \triangleq
\left\{
\begin{array}{ll}
\etrue & \event_1 = (x, l, n, v) \land \event_2 = (x, l, n, v) \\
\etrue & \event_1 = (x, l, n, \qval_1, v) \land \event_2 = (x, l, n, \qval_2, v)  
\land \qval_1 =_{q} \qval_2\\
\etrue & \event_1 = (b, l, n, v) \land \event_2 = (b, l, n, v) \\
\efalse & o.w.
\end{array}
\right.
\]
%
Signature Equivalence of 2 events : $\event_1 \sigeq \event_2$
\[
\event_1 \sigeq \event_2 \triangleq
\left\{
\begin{array}{ll}
\etrue & \event_1 = (x, l, n, v) \land \event_2 = (x, l, n, v') \\
\etrue & \event_1 = (x, l, n, \qval, v) \land \event_2 = (x, l, n, \qval', v') \\
\etrue & \event_1 = (b, l, n, v) \land \event_2 = (b, l, n, v') \\
\efalse & o.w.
\end{array}
\right.
\]
%
\begin{defn}[Order of Events].
\label{def:query_dir}
\\
Given 2 events 
$\event_1 = (x_1, l_1, n_1), 
\event_2 = (x_2, l_2, n_2)$
:
%
\[
\event_1 \eventlt \event_2
 \triangleq 
 \left\{
 \begin{array}{ll}
    l_1 < l_2 & n_1 = n_2
    \\
    n_1 < n_2  & o.w.
\end{array}  
\right.
\]
%
$\event_1 \eventgeq \event_2$  is defined vice versa.
\end{defn}
%
%
%
\subsection{Trace}
Trace appending:
\[
  \trace \cdot \event \triangleq
  \left\{
  \begin{array}{ll} 
    {} \cdot \event           & \trace =  \cdot \\
    \trace \cdot \event    & \trace =  \trace' \cdot \event\\ 
  \end{array}
  \right.
\]
%
Trace concatenation:
\[
  \trace_1 \cdot \trace_2 \triangleq
  \left\{
  \begin{array}{ll} 
    \trace_1                                  & \trace_2 =  \cdot \\
    (\trace_1 \cdot \event) \cdot \trace_2'    & \trace_2 =  {} \cdot \event \cdot \trace_2'\\ 
  \end{array}
  \right.
\]
%
An $\event \in \eventset$ belongs to a trace $\trace$, i.e., $\event \eventin \trace$ are defined as follows:
%
\begin{equation}
  \event \eventin \trace  
  \triangleq \left\{
  \begin{array}{ll} 
    \etrue                  & \trace =  (\trace' \cdot \event') \land (\event \eventeq \event') \\
    \event \eventin \trace' & \trace =  (\trace' \cdot \event') \land (\event \eventneq \event') \\ 
    \efalse                 & o.w.
  \end{array}
  \right.
\end{equation}
%
An $\event \in \eventset$ isomorphically belongs to a trace $\trace$, i.e., $\event \ismin \trace$ are defined as follows:
  %
\begin{equation}
  \event \ismin \trace  
  \triangleq \left\{
  \begin{array}{ll} 
    \etrue                  & \trace =  (\trace' \cdot \event') 
    \land (\pi_{\sig}(\event) = \pi_{\sig}(\event')) \\
    \event \ismin \trace'   & \trace =  (\trace' \cdot \event') \land (\pi_{\sig}(\event) \neq \pi_{\sig}(\event') ) \\ 
    \efalse                 & o.w.
  \end{array}
  \right.
\end{equation}
%
%
%
\begin{defn}[Equivalence of Program]
%
\label{def:aq_prog}
Given 2 programs $c_1$ and $c_2$:
\[
c_1 =_{c} c_2
\triangleq 
\left\{
  \begin{array}{ll} 
    \etrue        
    & c_1 = \eskip \land c_2 = \eskip
    \\ 
    \forall \trace \in \mathcal{T} \st \exists v \in \mathcal{VAL}
    \st \config{ \trace, \expr_1} \aarrow v \land \config{ \trace, \expr_1} \aarrow v     
    & c_1 = \assign{x}{\expr_1} \land c_2 = \assign{x}{\expr_2} 
    \\ 
    \qexpr_1 =_{q} \qexpr_2       
    & c_1 = \assign{x}{\query(\qexpr_1)} \land c_1 = \assign{x}{\query(\qexpr_2)} 
    \\
    c_1^f =_{c} c_2^f \land c_1^t =_{c} c_2^t
    & c_1 = \eif(b, c_1^t, c_1^f) \land c_2 = \eif(b, c_2^t, c_2^f)
    \\ 
    c_1' =_{c} c_2'         
    & c_1 = \ewhile b \edo c_1' \land c_2 = \ewhile b \edo c_2'
    \\ 
    c_1^h =_{c} c_2^h \land c_1^t =_{c} c_2^t
    & c_1 = c_1^h;c_1^t \land c_2 = c_2^h;c_2^t 
  \end{array}
  \right.
\]
%
$c_1 \neq_{c} c_2$  is defined vice versa.
%
\end{defn}
%
Given 2 programs $c$ and $c'$, $c'$ is a sub-program of$c$, i.e., $c' \in_{c} c$ is defined as:
\begin{equation}
c' \in_{c} c \triangleq \exists c_1, c_2, c''. ~ s.t.,~
c =_{c} c_1; c''; c_2 \land c' =_{c} c''
\end{equation} 
%
\begin{defn}[Well-formed Trace $\mathcal{T}$]
\label{def:wf_trace}
A trace $\trace$ is well formed, i.e., $\trace \in \mathcal{T}$ if and only if it preserves the following two properties:
\begin{itemize}
\item{\emph{(Uniqueness)}} 
$\forall \event_1, \event_2 \eventin \trace \st \event_1 \ismneq \event_2$
%
\item{\emph{(Ordering)}} $\forall \event_1, \event_2 \eventin \trace \st 
(\event_1 \eventlt \event_2) \Longleftrightarrow
\exists \trace_1, \trace_2, \trace_3 \in \mathbb{T},
 \event_1', \event_2' \in \eventset \st
(\event_1 \eventeq \event_1') \land (\event_2 \eventeq \event_2')
\land \trace_1 \cdot \event_1' \cdot \trace_2 \cdot \event_2' \cdot \trace_3 = \trace$
\end{itemize}
\end{defn}
%
%
\begin{thm}[Trace Generated from Operational Semantics is Well-formed $c \vDash \trace$].
\label{thm:os_wf_trace}
\\
\[
\forall \trace \in \mathcal{T}, c \st
\config{c, \vtrace} \to^{*} \config{\eskip, \trace \cdot \vtrace'}
\implies
\vtrace' \in \mathcal{T}
\]
%
\end{thm}
\begin{proof}
Proof in File: {\tt ``thm\_os\_wf\_trace.tex''}.
% {
By induction on the program $c$, we have following cases:
%
\caseL{ $[\assign x \expr]^{l}$}
%
We assume a configuration $\config{m,[\assign x \expr]^{l} , t, w}$ s.t. .
\\
From the evaluation rule \rname{assn1} and rule \rname{assn2}, we know there exists an evaluation as follows:
%
\[
	\config{m, [\assign x \expr]^{l}, t, w} \to \config{m', \eskip, t, w}
\]
%
when we assume $\config{m,e} \to^{*} v$. 
%
\\
The resulting trace is $t$. 
\\
By unfolding the trace subtraction operation, we know $t - t = []$ is an empty list, which is well formed according to Definition~\ref{def:wf_trace}.
}
\caseL{$[\assign{x}{\query(\qexpr)}]^{l}$}
By repeatedly applying the operational semantics rule \rname{query-e}, we know there exist $\qval$ s.t.,:
%
\[
\config{m, [\assign{x}{\query(\qexpr)}]^{l}, t, w} \to^{*} \config{m', [\assign{x}{\query(\qval)}]^{l}, t, w}
\]
%
According to the evaluation rule \rname{query-v}, we have the following transition:
\[
	\config{m', [\assign{x}{\query(\qval)}]^{l}, t, w} \to \config{m', \eskip, t ++ [(\qval, l, w)], w}
\]
%
The resulting trace is $t ++ [(\qval, l, w)]$.
\\
By unfolding the trace subtraction operation, we know $(t ++ [(\qval, l, w)]) - t = [(\qval, l, w)]$,
which is well formed by Definition~\ref{def:wf_trace}.
%%
\caseL{$\eif([b]^l, c_1, c_2)$}
Let $\config{m', \eskip, t, w} ~ (\star)$ be the configuration 
s.t. $\config{m, \eif([b]^l, c_1, c_2), t, w} \to^{*} \config{m', \eskip, t', w'}$.
It is sufficient to show that $t' - t$ is well-formed.
\\
According to the evaluation rule \rname{if}, we know there exists an evaluation by repeatedly applying \rname{if} 
%
\[
\config{m, \eif([b]^l, c_1, c_2), t, w} \to^{*} \config{m, \eif([v]^l, c_1, c_2), t, w} ~ (1)
\]
%
, where $\config{m, b} \barrow v$ and $v$ is either $\etrue$ or $\efalse$.
\\
Considering 2 subcases where $v$ is either $\etrue$ or $\efalse$:
%
\subcaseL{$v = \etrue$}
%
By applying the evaluation rule \rname{if-t}, we have:
\[
\config{m, \eif([\etrue]^l, c_1, c_2), t, w} \to \config{m, c_1, t, w} ~ (t1)
\]
%
By induction hypothesis on $c_1$ with starting memory $m$, trace $t$ and while map $w$, 
let $\config{m_1, \eskip, t_1, w_1}$ be the configuration s.t.
\[
	\config{m, c_1, t, w} \to^{*} \config{m_1, \eskip, t_1, w_1} ~(t2)
\]
we know $(t_1 - t)$ is a well formed trace.
\\
According to the evaluation $(a)$, $(t1)$ and $(t2)$, we know $\config{m_1, \eskip, t_1, w_1}$ 
is the assumed configuration $\config{m', \eskip, t', w'}$ such that
\[
	\config{m, \eif([b]^l, c_1, c_2), t, w} \to^{*} \config{m_1, \eskip, t_1, w_1}
\]
%
Since $(t_1 - t)$ is well-formed and $t' - t = t_1 - t$, this subcase is proved.
%
\subcaseL{$v = \efalse$}
%
By applying the evaluation rule \rname{if-f}, we have:
\[
\config{m, \eif([\efalse]^l, c_1, c_2), t, w} \to \config{m, c_2, t, w} ~ (f1)
\]
%
By induction hypothesis on $c_1$ with starting memory $m$, trace $t$ and while map $w$, 
let $\config{m_2, \eskip, t_2, w_2}$ be the configuration s.t.
\[
	\config{m, c_2, t, w} \to^{*} \config{m_2, \eskip, t_2, w_2} ~(f2)
\]
we know $(t_2 - t)$ is a well formed trace.
\\
According to the evaluation $(a)$, $(f1)$ and $(f2)$, we know $\config{m_2, \eskip, t_2, w_2}$ 
is the assumed configuration $\config{m', \eskip, t', w'}$ in $(\star)$ such that
\[
	\config{m, \eif([b]^l, c_1, c_2), t, w} \to^{*} \config{m_2, \eskip, t_2, w_2}
\]
%
Since $(t_2 - t)$ is well-formed and $t' = t_2$, this subcase is proved.
%
\caseL{$c_1; c_2$}
%
According to the evaluation rule \rname{seq1}, we have the following evaluation by repeatedly applying \rname{seq1} 
%
\[
\config{m, c_1;c_2, t, w} \to^{*} \config{m_1, [\eskip]^l;c_2,  t_1, w_1} ~ (1)
\]
%
where 
%
\[
	\config{m, c_1, t, w} \to^{*} \config{m_1, [\eskip]^l, t_1, w_1]} ~ (2)
\]
%
%
By induction hypothesis on $c_1$ and $(2)$, we know $(t_1 - t)$ is well-formed.
%
According to the evaluation rule \rname{seq2}, we have:
%
\[
\config{m_1, [\eskip]^l;c_2,  t_1, w_1} \to \config{m_1, c_2, t_1, w_1} ~ (3)
\]
%   
%
By induction hypothesis on $c_2$ and let $\config{m_2, \eskip, t_2, w_2]}$ be the configuration s.t.,:
%
\[
	\config{m_1, c_2, t_1, w_1} \to^{*} \config{m_2, \eskip, t_2, w_2]} ~ (4)
\]
%
we know $(t_2 - t_1)$ is well-formed.
%
\\
According to evaluations $(1), (2), (3)$ and $(4)$, we have following evaluation:
%
\[
	\config{m, c_1;c_2, t, w} \to^{*} \config{m_2, \eskip, t_2, w_2]}
\]
%
It is sufficient to show $(t_2 - t)$ is well-formed.
%
Unfolding the definition~\ref{def:wf_trace}, it is sufficient to show that the trace $(t_2 - t)$ holds both the \emph{Uniqueness} and \emph{Ordering} property.
\\
Let $t_1' = t_1 - t$ and $t_2' = t_2 - t_1$, unfolding the subtraction operations, we have:
\[
	(t_2 - t) = t_1' ++ t_2'
\]
%
%
\caseL{Uniqueness $(\diamond)$} 
By the distinction properties of labels in program $c$, 
\[
	\forall \aq_1 \in t_1', \aq_2 \in t_2'. ~ \aq_1 \aqneq \aq_2
\] 
%
Since $t = t_1' ++ t_2'$ and both $t_1'$ and $t_2'$ are well-formed, we know:
\[
	\forall \aq_1, \aq_2 \in (t_1' ++ t_2'). ~ \aq_1 \aqneq \aq_2
\]
%
THe Uniqueness property for $(t_2 - t)$ is proved.
%
\caseL{Ordering $(\square)$}
%
By the definition of ordering property, we need to show $\forall \aq_1, \aq_2 \aqin (t_2 - t)$:
\[
	(\aq_1 <_{aq} \aq_2) \Longleftrightarrow
	\exists t^1, t^2, t^3, \aq_1', \aq_2'. ~ s.t.,~ 
	(\aq_1 \aqeq \aq_1') \land (\aq_2 \aqeq \aq_2') \land t^1 ++ [\aq_1'] ++ t^2 ++ [\aq_2'] ++ t^3 = t
\]
%
Since $(t_2 - t) = t_1' ++ t_2'$, there are 4 cases need to be proved:
\\
$(1) ~ \aq_1, \aq_2 \aqin t_1'$
\\
$(2) ~ \aq_2, \aq_2 \aqin t_2'$
\\
$(3) ~ \aq_2 \aqin t_1', \aq_1 \aqin t_2'$ 
\\
$(4) ~ \aq_1 \aqin t_1', \aq_2 \aqin t_2'$
\\
Given both $t_1'$ and $t_2'$ are well-formed, we know this property is true in case $(1)$ and $(2)$.
\\
By the ordering properties of labels in program $c_1; c_2$,
we know all the labels in $c_2$ are greater than the labels in $c_1$, i.e.,
\[
	\forall \aq_1 \aqin t_1', \aq_2 \aqin t_2'. ~ \aq_1 <_{aq} \aq_2 ~ (a)
\]
%
Then we have the case $(3)$ proved because the premise is $\efalse$ in this case.
%
To prove the ordering property in case $(4)$, it is sufficient to show the following 2 implications:
\[
	(\aq_1 <_{aq} \aq_2) 
	\Longrightarrow
	\exists t^1, t^2, t^3, \aq_1', \aq_2'. ~ s.t.,~ 
	(\aq_1 \aqeq \aq_1') \land (\aq_2 \aqeq \aq_2') \land t^1 ++ [\aq_1'] ++ t^2 ++ [\aq_2'] ++ t^3 = t
\]
%
\[
	% \forall \aq_1, \aq_2 \aqin t. ~
	\exists t^1, t^2, t^3, \aq_1', \aq_2'. ~ s.t.,~ 
	(\aq_1 \aqeq \aq_1') \land (\aq_2 \aqeq \aq_2') \land t^1 ++ [\aq_1'] ++ t^2 ++ [\aq_2'] ++ t^3 = t
	 \Longrightarrow
	(\aq_1 <_{aq} \aq_2)
\]
%
By Corollary~\ref{coro:aqintrace} and the hypothesis $(4)$, we know:
%
\[
	\exists t_{11}, t_{12}, \aq_1'. ~ (\aq_1 \aqeq \aq_1') \land t_{11} ++ [\aq_1'] ++ t_{12} = t_1'
\]
%
\[
	\exists t_{21}, t_{22}, \aq_2'. ~ (\aq_2 \aqeq \aq_2') \land t_{21} ++ [\aq_2'] ++ t_{22} = t_2'
\]
%
Then according to the list concatenation operations, we have:
%
\[
	\exists t_{11}, t_{12}, \aq_1', t_{21}, t_{22}, \aq_2'. ~ (\aq_1 \aqeq \aq_1') \land (\aq_2 \aqeq \aq_2') \land 
	t_{11} ++ [\aq_1'] ++ t_{12} ++ t_{21} ++ [\aq_2'] ++ t_{22} = t_1' ++ t_2'
\]
%
Let $t^1 = t_{11}$, $t^2 = t_{12} ++ t_{21}$ and $t^3 = t_{22}$, since $(t_2 - t) = t_1' ++ t_2'$, we have:
%
\[
	\exists t^1, t^2, t^3, \aq_1', \aq_2'. ~ s.t.,~ 
	(\aq_1 \aqeq \aq_1') \land (\aq_2 \aqeq \aq_2') \land t^1 ++ [\aq_1'] ++ t^2 ++ [\aq_2'] ++ t^3 = (t_2 - t)
\]
%
This case is proved.
%
\\
%
For the other directions, according to the hypothesis $(4)$ and $(a)$ proved above, we know:
\[
	(\aq_1 <_{aq} \aq_2)
\]
%
This case is proved.
%
\caseL{$\ewhile [b]^l \edo c'$}
%
According to the \rname{while-b} and \rname{ifw-b} rule, we have following evaluation:
\[
	\config{m, \ewhile [b]^l \edo c', t, w} \to^{*}
	\config{m, \eif_w([v]^l, c', \eskip), t, w} ~ (w1)
\]
%
where $\config{m, b} \barrow^{*} v$ and $v$ is either $\etrue$ or $\efalse$.
%
\subcaseL{$v = \etrue$ }
%
By evaluation rule \rname{ifw-true}, we have:
%
\[
	\config{m, \eif_w([\etrue]^l, c';\ewhile [b]^l, 2 \edo c', \eskip), t, w} 
	\to^{*} \config{m, c';\ewhile [b]^l \edo c', t, [l \mapsto 1]} ~ (w2)
\]
%
By rule \rname{seq1}, we have the following evaluation:
%
\[
	\config{m, c';\ewhile [b]^l, 2 \edo c', t, [l \mapsto 1]} \to^{*} \config{m', \eskip;\ewhile [b]^l, 2 \edo c', t', w'} ~ (w3)
\] 
%
where $\config{m, c', t, [l \mapsto 1]} \to^{*} \config{m', \eskip, t', w'} ~ (w4)$. 
%
By induction hypothesis on $c'$ and the evaluation $(w4)$, we know $t' - t$ is well-formed.
\\
By the rule \rname{seq2}, we have:
%
\[
	\config{m', \eskip;\ewhile [b]^l, 2 \edo c', t', w'} \to \config{m', \ewhile [b]^l, 2 \edo c', t', w'} ~ (w6)
\]
%
By the termination of the while loop, we know there exist an evaluation by repeating the evaluation in $(w1), (w2), (w3)$ and $(w6)$:
%
\[
	\config{m', \ewhile [b]^l, 2 \edo c', t', w'} \to^{*} \config{m^n, \eif_w([\efalse]^l, c';\ewhile [b]^l, n + 1, \edo c', \eskip), t^n, w^n}
\]
%
By the evaluation $(w5)$, we know for every resulting trace $t^i$ in the evaluation of each iteration $i = 2, \ldots, n$,
the following invariant holds:
\\
$t^i - t^{i - 1}$ is well formed.
\\
%
By evaluation rule \rname{ifw-false}, we have:
%
\[
	\config{m, \eif_w([\efalse]^l, c';\ewhile [b]^l, n + 1 \edo c', \eskip), t^n, w + l} 
	\to^{*} \config{m, \eskip, t^n, (w + l)/l}
\]
%
%
It is sufficient to show $(t^n - t)$ is well-formed.
%
Unfolding the definition~\ref{def:wf_trace}, it is sufficient to show that the trace $(t^n - t)$ holds both the \emph{Uniqueness} and \emph{Ordering} property.
\\
Let $t_1 = t' - t$ and $t_i = t^{i} - t^{i - 1}$ for $i = 2, \ldots, n$, unfolding the subtraction operations, we have:
\[
	(t^n - t) = t_1 ++ t_2 ++ \ldots ++ t_n
\]
\\
By repeating the proof $(\diamond)$ and $(\square)$ in the sequence case, we have this case proved. 
%
\subcaseL{$v = \efalse$}
%
By evaluation rule \rname{ifw-false}, we have:
%
\[
	\config{m, \eif_w([\efalse]^l, c';\ewhile [b]^l \edo c', \eskip), t, w} 
	\to^{*} \config{m, \eskip, t, w/l}
\]
%
%
Because $t - t = []$, this sub-case is proved.
%

\end{proof}
%
\begin{lem}
[Trace Non-Decreasing].
\\
$$
\forall \trace \in \mathcal{T}, c \st
\config{c, \trace} \rightarrow^{*} \config{\eskip, \trace'} 
\implies \exists \trace'' \in \mathcal{T} \st \trace \cdot \trace'' = \trace'
$$
\end{lem}
%
\begin{coro}
\label{coro:aqintrace}
\[
\event \eventin \trace \implies \exists \trace_1, \trace_2 \in \mathcal{T}, 
\event' \in \eventset \st (\event \eventeq \event') \land \trace_1 \cdot \event' \cdot \trace_2 = t  
\]
\end{coro}
\begin{subproof}
Proof in File: {\tt ``coro\_aqintrace.tex''}
% \begin{coro}
\label{coro:aqintrace}
\[
\aq \aqin t \implies \exists t_1, t_2, \aq'. ~ s.t., ~ (\aq \aqeq \aq') \land t_1 ++ [\aq'] ++ t_2 = t	
\]
\end{coro}
\begin{subproof}
Proof in File: {\tt ``coro\_aqintrace.tex''}
% \begin{coro}
\label{coro:aqintrace}
\[
\aq \aqin t \implies \exists t_1, t_2, \aq'. ~ s.t., ~ (\aq \aqeq \aq') \land t_1 ++ [\aq'] ++ t_2 = t	
\]
\end{coro}
\begin{subproof}
Proof in File: {\tt ``coro\_aqintrace.tex''}
% \input{coro_aqintrace}
By unfolding the $\aq \aqin t$, we have the following cases:
%
\caseL{$t = []$} The hypothesis is $\efalse$, this case is proved.
%
\caseL{$t = \aq'::t' \land \aq' \aqeq \aq $}
%
Let $t_1 = []$ and $t_2 = t'$, by unfolding the list concatenation operation, we have:
%
\[
	t_1 ++ [\aq'] ++ t_2 = [] ++ [\aq'] ++ t' = t
\]
%
Since $\aq' \aqeq \aq$ by the hypothesis, this case is proved.
%
\caseL{$t = \aq'::t' \land \aq' \aqneq \aq $}
%
By induction hypothesis on $\aq \aqin t'$, we know:
%
\[
	\exists t_1', t_2', \aq''. ~ s.t., ~ (\aq \aqeq \aq'') \land t_1' ++ [\aq''] ++ t_2' = t'	
\]
%
Let $t_1 = \aq'::t_1'$ and $t_2 = t_2'$, by unfolding the list concatenation operation, we have:
%
\[
	t_1 ++ [\aq''] ++ t_2 = (\aq':: t_1') ++ [\aq''] ++ t_2' = \aq'::t' = t
\]
%
Since $\aq'' \aqeq \aq$ by the hypothesis, this case is proved.
%
\end{subproof}
By unfolding the $\aq \aqin t$, we have the following cases:
%
\caseL{$t = []$} The hypothesis is $\efalse$, this case is proved.
%
\caseL{$t = \aq'::t' \land \aq' \aqeq \aq $}
%
Let $t_1 = []$ and $t_2 = t'$, by unfolding the list concatenation operation, we have:
%
\[
	t_1 ++ [\aq'] ++ t_2 = [] ++ [\aq'] ++ t' = t
\]
%
Since $\aq' \aqeq \aq$ by the hypothesis, this case is proved.
%
\caseL{$t = \aq'::t' \land \aq' \aqneq \aq $}
%
By induction hypothesis on $\aq \aqin t'$, we know:
%
\[
	\exists t_1', t_2', \aq''. ~ s.t., ~ (\aq \aqeq \aq'') \land t_1' ++ [\aq''] ++ t_2' = t'	
\]
%
Let $t_1 = \aq'::t_1'$ and $t_2 = t_2'$, by unfolding the list concatenation operation, we have:
%
\[
	t_1 ++ [\aq''] ++ t_2 = (\aq':: t_1') ++ [\aq''] ++ t_2' = \aq'::t' = t
\]
%
Since $\aq'' \aqeq \aq$ by the hypothesis, this case is proved.
%
\end{subproof}
%
\end{subproof}
% \\
%
%
% \todo{
% \begin{lem}[While Map Remains Unchanged (Invariant)]
% \label{lem:wunchange}
% Given a program $c$ with a starting memory $m$, trace $t$ and while map $w$, s.t.,
% $\config{m, c, t, w} \to^{*} \config{m', \eskip, t', w'}$ and $Labels(c) \cap Keys(w) = \emptyset$, then 
% \[
%   w = w'
% \]
% \end{lem}
% \begin{subproof}[Proof of Lemma~\ref{lem:wunchange}]
% %
% Proof in File: {\tt ``lem\_wunchange.tex''}
% % 
%
\todo{
\begin{lem}[While Map Remains Unchanged (Invariant)]
\label{lem:wunchange}
Given a program $c$ with a starting memory $m$, trace $t$ and while map $w$, s.t.,
$\config{m, c, t, w} \to^{*} \config{m', \eskip, t', w'}$ and $Labels(c) \cap Keys(w) = \emptyset$, then 
\[
	w = w'
\]
\end{lem}
\begin{subproof}[Proof of Lemma~\ref{lem:wunchange}]
%
By induction on program $c$, we have following cases:
%
\caseL{$[\assign x \expr]^{l}$}
%
From the evaluation rule \rname{assn1} and rule \rname{assn2}, we know there exists an evaluation
$$
\config{m, [\assign x \expr]^{l}, t, w} \to^{*} \config{m[v/x], \eskip, t', w}
$$
%
, where $\config{m,e} \to^{*} v$.
%
The resulting while map is $w$, this case is proved because $w = w$.
%
\caseL{$[\assign{x}{\query(\qexpr)}]^{l}$}
%
By repeatedly applying the operational semantics rule \rname{query-e}, we know there exist $\qval$ s.t.,:
%
\[
\config{m, [\assign{x}{\query(\qexpr)}]^{l}, t, w} \to^{*} \config{m', [\assign{x}{\query(\qval)}]^{l}, t', w}
\]
%
According to the evaluation rule \rname{query-v}, we have the following transition:
\[
	\config{m', [\assign{x}{\query(\qval)}]^{l}, t', []} \to \config{m', \eskip, t' ++ [(\qval, l, w)], w}
\]
%
The resulting while map is $w$, this case is proved because $w = w$.
%%
\caseL{$\eif([b]^l, c_1, c_2)$}
%
According to the evaluation rule \rname{if}, we know there exists an evaluation by repeatedly applying the rule \rname{if}: 
%
\[
\config{m, \eif([b]^l, c_1, c_2), t, w} \to^{*} \config{m_b, \eif([v]^l, c_1, c_2), t', w} ~ (1)
\]
%
where $v$ is either $\etrue$ or $\efalse$.
\\
Considering 2 subcases where $v$ is either $\etrue$ or $\efalse$:
%
\subcaseL{$v = \etrue$}
%
By applying the evaluation rule \rname{if-t}, we have:
\[
\config{m_b, \eif([\etrue]^l, c_1, c_2), t', w} \to \config{m_b, c_1, t', w} ~ (t1)
\]
%
By induction hypothesis on $c_1$ with starting memory $m_b$, trace $t'$ and while map $w$, let $\config{m_1, \eskip, t_1, w_1}$ be the configuration s.t.
\[
	\config{m_b, c_1, t', w} \to^{*} \config{m_1, \eskip, t_1, w_1} ~(t2)
\]
we know $w_1 = w$.
\\
According to the evaluation $(a)$, $(t1)$ and $(t2)$, we have the following evaluation:
% 
\[
	\config{m, \eif([b]^l, c_1, c_2), t, w} \to^{*} \config{m_1, \eskip, t_1, w}
\]
%
, where the resulting while map is still $w$, this sub-case is proved.
%
\subcaseL{$v = \efalse$}
%
By applying the evaluation rule \rname{if-f}, we have:
\[
\config{m_b, \eif([\efalse]^l, c_1, c_2), t', w} \to \config{m', c_2, t', w} ~ (f1)
\]
%
By induction hypothesis on $c_1$ with starting memory $m_b$, trace $t'$ and while map $w$, 
let $\config{m_2, \eskip, t_2, w_2}$ be the configuration s.t.
\[
	\config{m_b, c_2, t', w} \to^{*} \config{m_2, \eskip, t_2, w_2} ~(f2)
\]
we know $w_2 = w$.
\\
According to the evaluation $(a)$, $(f1)$ and $(f2)$, we the following evaluation:
\[
	\config{m, \eif([b]^l, c_1, c_2), t, w} \to^{*} \config{m_2, \eskip, t_2, w}
\]
%
, where the resulting while map is still $w$, this sub-case is proved.
%
\caseL{$c_1; c_2$}
%
According to the evaluation rule \rname{seq1}, we have the following evaluation by repeatedly applying \rname{seq1} 
%
\[
\config{m, c_1;c_2, t, w} \to^{*} \config{m_1, [\eskip]^l;c_2,  t_1, w_1} ~ (1)
\]
%
where 
%
\[
	\config{m, c_1, t, w} \to^{*} \config{m_1, [\eskip]^l, t_1, w_1]} ~ (2)
\]
%
By induction hypothesis on $c_1$ and $(2)$, we know $w_1 = w$.
%
According to the evaluation rule \rname{seq2}, we have:
%
\[
\config{m_1, [\eskip]^l;c_2,  t_1, w} \to \config{m_1, c_2, t_1, w} ~ (3)
\]
%
let $\config{m_2, [\eskip]^l, t_2, w_2]}$ be the configuration s.t.,:
%
\[
	\config{m_1, c_2, t_1, w} \to^{*} \config{m_2, [\eskip]^l, t_2, w_2]}~ (4).
\]
%
%
By induction hypothesis on $c_2$, $m_1$, $t_1$, $w$ and $(4)$, we know $w_2 = w$.
%
\\
According to the evaluations $(1)$, $(3)$ and $(4)$, we have:
\[
	\config{m, c_1;c_2, t, w} \to^{*} \config{m', [\eskip]^l, t', w}
\]
%
This case is proved.
%
\caseL{$\ewhile [b]^l \edo c'$}
%
According to the \rname{while-b} and \rname{ifw-b} rule, we have following evaluation:
\[
	\config{m, \ewhile [b]^l \edo c', t, w} \to^{*}
	\config{m, \eif_w([v]^l, c', \eskip), t, w} ~ (w1)
\]
%
where $v$ is either $\etrue$ or $\efalse$.
%
\subcaseL{$v = \etrue$ }
%
By evaluation rule \rname{ifw-true}, we have:
%
\[
	\config{m, \eif_w([\etrue]^l, c';\ewhile [b]^l \edo c', \eskip), t, w} 
	\to^{*} \config{m, c';\ewhile [b]^l \edo c', t, w+l} ~ (w2)
\]
%
By rule \rname{seq1}, we have the following evaluation:
%
\[
	\config{m, c';\ewhile [b]^l \edo c', t, w+l} \to^{*} \config{m', \eskip;\ewhile [b]^l \edo c', t', w'} ~ (w3)
\] 
%
where $\config{m, c', t, w+l} \to^{*} \config{m', \eskip, t', w'} ~ (w4)$. 
%
By induction hypothesis on $c'$ and the evaluation $(w4)$, we know $w' = w + l$. Then we have $\config{m, c', t, w+l} \to^{*} \config{m', \eskip, t', w+l} ~ (w5)$.
%
By the rule \rname{seq2}, we have:
%
\[
	\config{m', \eskip;\ewhile [b]^l \edo c', t', w + l} \to \config{m', \ewhile [b]^l \edo c', t', w + l} ~ (w6)
\]
%
By the termination of the while loop, we know there exist an evaluation by repeating the evaluation in $(w1), (w2), (w3)$ and $(w6)$:
%
\[
	\config{m', \ewhile [b]^l \edo c', t', w + l} \to^{*} \config{m', \eif_w([\efalse]^l, c';\ewhile [b]^l \edo c', \eskip), t'', w''}
\]
%
By the evaluation $(w5)$, we know in the evaluation of each iterations, the resulting while map is always $w + l$. So we have $w'' = w + l$.
%
By evaluation rule \rname{ifw-false}, we have:
%
\[
	\config{m, \eif_w([\efalse]^l, c';\ewhile [b]^l \edo c', \eskip), t'', w + l} 
	\to^{*} \config{m, \eskip, t, (w + l)/l}
\]
%
By the hypothesis that $l \notin Keys(w)$, unfolding the $(w + l)/l$ operation, we have:
%
\[
	w = (w + l)/l.
\]
%
This sub-case is proved.
%
\subcaseL{$v = \efalse$}
%
By evaluation rule \rname{ifw-false}, we have:
%
\[
	\config{m, \eif_w([\efalse]^l, c';\ewhile [b]^l \edo c', \eskip), t, w} 
	\to^{*} \config{m, \eskip, t, w/l}
\]
%
By the hypothesis that $l \notin Keys(w)$, unfolding the $w/l$ operation, we have:
%
\[
	w = w/l.
\]
%
This sub-case is proved.
%
\end{subproof}
}



% %
% \end{subproof}
% }
%
% \todo{
% \begin{lem}[Trace is Written Only]
% \label{lem:twriteonly}
% Given a program $c$ with starting trace $t_1$ and $t_2$,
% for arbitrary starting memory $m$ and while map $w$,
% if there exist evaluations
% $$\config{m, c, t_1, w} \to^{*} \config{m_1', \eskip, t_1', w_1'}$$
% % 
% $$\config{m, c, t_2, w} \to^{*} \config{m_2', \eskip, t_2', w_2'}$$
% %
% then:
% %
% \[
%   m_1' = m_2' \land w_1' = w_2'
% \]
% \end{lem}
% %
% \begin{subproof}[Proof of Lemma~\ref{lem:twriteonly}]
% %
% Proof in File: {\tt ``lem\_twriteonly.tex''}
% % \todo{
\begin{lem}[Trace Uniqueness]
\label{lem:tunique}
Given a program $c$ with a starting memory $m$, 
for any starting trace $t_1$ and $t_2$, if there exist evaluations
$$\config{m, c, t_1, []} \to^{*} \config{m_1', \eskip, t_1', w_1'}$$
% 
$$\config{m, c, t_2, []} \to^{*} \config{m_2', \eskip, t_2', w_2'}$$
%
then:
%
\[
	t_1' - t_1 = t_2' - t_2
\]
\end{lem}
%
\begin{subproof}[Proof of Lemma~\ref{lem:tunique}]
%
By induction on program $c$, we have following cases:
%
\caseL{ $[\assign x \expr]^{l}$}
%
From the evaluation rule \rname{assn1} and rule \rname{assn2}, we know there exist evaluations
$$
\config{m, [\assign x \expr]^{l}, t_1, w} \to^{*} \config{m[v/x], \eskip, t_1, w}
$$
%
\[
\config{m, [\assign x \expr]^{l}, t_2, w} \to^{*} \config{m[v/x], \eskip, t_2, w}
\]
%
for starting trace $t_1$ and $t_2$ respectively, where $\config{m,e} \aarrow^{*} v$.
%
Since the resulting traces in the 2 evaluations are still $t_1$ and $t_2$, 
and $t_1 - t_1 = t_2 - t_2 = []$, this case is proved.
%
\caseL{$[\assign{x}{\query(\qexpr)}]^{l}$}
%
By repeatedly applying the operational semantics rule \rname{query-e}, we know there exist evaluations
%
\[
\config{m, [\assign{x}{\query(\qexpr)}]^{l}, t_1, w} \to^{*} \config{m, [\assign{x}{\query(\qval)}]^{l}, t_1, w}
\]
%
\[
\config{m, [\assign{x}{\query(\qexpr)}]^{l}, t_2, w} \to^{*} \config{m, [\assign{x}{\query(\qval)}]^{l}, t_2, w}
\]
%
for starting trace $t_1$ and $t_2$ respectively, where $\config{m, \qexpr} \qarrow^{*} \qval$.
%
\\
%
According to the evaluation rule \rname{query-v}, we then have the following 2 transitions with starting trace $t_1$ and $t_2$ respectively:
%
\[
\config{m, [\assign{x}{\query(\qval)}]^{l}, t_1, w} \to \config{m_1, \eskip, t_1 ++ [(\qval, l, w)], w}
\]
%
\[
\config{m, [\assign{x}{\query(\qval)}]^{l}, t_2, w} \to \config{m_2, \eskip, t_2 ++ [(\qval, l, w)], w}
\]
%
The resulting traces are $t_1 ++ [(\qval, l, w)]$, and $t_2 ++ [(\qval, l, w)]$. 
\\
By unfolding the trace subtraction operation, we have:
\[
(t_1 ++ [(\qval, l, w)]) - t_1 = [(\qval, l, w)]
%
\land
%
(t_2 ++ [(\qval, l, w)]) - t_2 = [(\qval, l, w)]
\]
%
This case is proved.
%%
\caseL{$\eif([b]^l, c_1, c_2)$}
%
According to the evaluation rule \rname{if}, we know there exist 2 evaluations for $t_1$ and $t_2$ respectively by repeatedly applying the rule \rname{if}: 
%
\[
\config{m, \eif([b]^l, c_1, c_2), t_1, w} \to^{*} \config{m, \eif([v]^l, c_1, c_2), t_1, w} ~ (1)
\]
%
\[
\config{m, \eif([b]^l, c_1, c_2), t_2, w} \to^{*} \config{m, \eif([v]^l, c_1, c_2), t_2, w} ~ (2)
\]
%
, where $\config{m, b} \barrow^{*} v$ and $v$ is either $\etrue$ or $\efalse$.
\\
Considering 2 subcases where $v$ is either $\etrue$ or $\efalse$:
%
\subcaseL{$v = \etrue$}
%
By applying the evaluation rule \rname{if-t}, we have:
\[
\config{m, \eif([\etrue]^l, c_1, c_2), t_1, w} \to \config{m, c_1, t_1, w} ~ (3)
\]
%
\[
\config{m, \eif([\etrue]^l, c_1, c_2), t_2, w} \to \config{m, c_1, t_2, w} ~ (4)
\]
%
By induction hypothesis on $c_1$ with $m$, $w$ $t_1$ and $t_2$, we know there exist evaluations:
\[
	\config{m, c_1, t_1, w} \to^{*} \config{m_1, \eskip, t_1', w_1} ~ (5)
\]
%
\[
	\config{m, c_1, t_2, w} \to^{*} \config{m_2, \eskip, t_2', w_2} ~ (6)
\]
%
, where $t_1' - t_1 = t_2' - t_2$.
\\
According to the evaluations $(1), (3), (5)$ and $(2), (4), (6)$, we have the following 2 evaluations:
% 
\[
	\config{m, \eif([b]^l, c_1, c_2), t_1, w} \to^{*} \config{m_1, \eskip, t_1', w_1}
\]
%
\[
	\config{m, \eif([b]^l, c_1, c_2), t_2, w} \to^{*} \config{m_2, \eskip, t_2', w_2}
\]
%
Since $t_1' - t_1 = t_2' - t_2$, this sub-case is proved.
%
\subcaseL{$v = \efalse$}
%
By applying the evaluation rule \rname{if-f}, we have:
%
\[
\config{m, \eif([\efalse]^l, c_1, c_2), t_1, w} \to \config{m, c_2, t_1, w} ~ (3)
\]
%
\[
\config{m, \eif([\efalse]^l, c_1, c_2), t_2, w} \to \config{m, c_2, t_2, w} ~ (4)
\]
%
By induction hypothesis on $c_2$ with $m$, $w$ $t_1$ and $t_2$, we know there exist evaluations:
\[
	\config{m, c_2, t_1, w} \to^{*} \config{m_1', \eskip, t_1'', w_1'} ~ (5)
\]
%
\[
	\config{m, c_2, t_2, w} \to^{*} \config{m_2', \eskip, t_2'', w_2'} ~ (6)
\]
%
, where $t_1'' - t_1 = t_2'' - t_2$.
\\
According to the evaluations $(1), (3), (5)$ and $(2), (4), (6)$, we have the following 2 evaluations:
% 
\[
	\config{m, \eif([b]^l, c_1, c_2), t_1, w} \to^{*} \config{m_1', \eskip, t_1'', w_1'}
\]
%
\[
	\config{m, \eif([b]^l, c_1, c_2), t_2, w} \to^{*} \config{m_2', \eskip, t_2'', w_2'}
\]
%
Since $t_1'' - t_1 = t_2'' - t_2$, this sub-case is proved.
%
\caseL{$c_1; c_2$}
%
According to the evaluation rule \rname{seq1}, we have the following 2 evaluations by repeatedly applying \rname{seq1} 
%
\[
\config{m, c_1;c_2, t_1, w} \to^{*} \config{m_1^1, [\eskip]^l;c_2,  t_1^1, w_1^1} ~ (1)
\]
%
%
\[
\config{m, c_1;c_2, t_2, w} \to^{*} \config{m_2^1, [\eskip]^l;c_2,  t_2^1, w_2^1} ~ (2)
\]
%
where 
%
\[
	\config{m, c_1, t_1, w} \to^{*} \config{m_1^1, [\eskip]^l, t_1^1, w_1^1]} ~ (3)
\]
%
%
\[
\config{m, c_1, t_2, w} \to^{*} \config{m_2^1, [\eskip]^l,  t_2^1, w_2^1} ~ (4)
\]
%
%
By induction hypothesis on $c_1$ and $(3), (4)$, we know $t_1^1 - t_1 = t_2^1 - t_2 ~ (a)$.
%
%
According to the evaluation rule \rname{seq2}, we have:
%
\[
\config{m_1^1, [\eskip]^l;c_2, t_1^1, w_1^1} \to \config{m_1^1, c_2, t_1^1, w_1^1}
\]
%
\[
\config{m_2^1, [\eskip]^l;c_2, t_2^1, w_2^1} \to \config{m_2^1, c_2, t_2^1, w_2^1}
\]
%
%
%
By induction hypothesis on $c_2$, $t_2^1$ and $t_1^1$ we have the following 2 evaluations 
%
\[
	\config{m_1^1, c_2, t_1^1, w} \to^{*} \config{m_1^2, \eskip, t_1^2, w_1^2]}
\]
%
\[
	\config{m_2^1, c_2, t_2^1, w} \to^{*} \config{m_2^2, \eskip, t_2^2, w_2^2]}
\]
%
and $t_1^2 - t_1^1 = t_2^2 - t_2^1 ~ (b)$.
%
\\
By $(a)$ and $(b)$, unfolding the trace subtraction operation, we know
%
\[
\exists t_1', t_1'', t_2', t_2''. ~ 
t_1 ++ t_1' = t_1^1 \land t_1^1 ++ t_1'' = t_1^2 
\land t_2 ++ t_2' = t_2^1 \land t_2^1 ++ t_2'' = t_2^2 
\]
%
By writing, we have:
%
\[
\exists t_1', t_1'', t_2', t_2''. ~ 
t_1 ++ t_1' ++ t_1'' = t_1^2 
\land t_2 ++ t_2' ++ t_2'' = t_2^2 
\]
%
Again, by the definition of trace subtraction, we have:
%
\[
	t_2^2 - t_2 = t_1^2 - t_1
\]
%
This case is proved.
%
\caseL{$\ewhile [b]^l \edo c'$}
%
According to the \rname{while-b} and \rname{ifw-b} rule, we have following 2 evaluations:
%
\[
	\config{m, \ewhile [b]^l \edo c', t_1, w} \to^{*}
	\config{m, \eif_w([v]^l, c', \eskip), t_1, w}
\]
%
\[
	\config{m, \ewhile [b]^l \edo c', t_2, w} \to^{*}
	\config{m, \eif_w([v]^l, c', \eskip), t_2, w}
\]
%
where $v$ is either $\etrue$ or $\efalse$.
%
\subcaseL{$v = \etrue$ }
%
By evaluation rule \rname{ifw-true}, we have:
%
\[
	\config{m, \eif_w([\etrue]^l, c';\ewhile [b]^l \edo c', \eskip), t_1, w} 
	\to^{*} \config{m, c';\ewhile [b]^l \edo c', t_1, w+l} ~ (w2)
\]
%
\[
	\config{m, \eif_w([\etrue]^l, c';\ewhile [b]^l \edo c', \eskip), t_2, w} 
	\to^{*} \config{m, c';\ewhile [b]^l \edo c', t_2, w+l} ~ (w2)
\]
%
%
By rule \rname{seq1}, we have the following evaluations:
%
\[
	\config{m, c';\ewhile [b]^l \edo c', t_1, w+l} \to^{*} \config{m', \eskip;\ewhile [b]^l \edo c', t_1', w'} ~ (w3)
\] 
%
%
\[
	\config{m, c';\ewhile [b]^l \edo c', t_2, w+l} \to^{*} \config{m', \eskip;\ewhile [b]^l \edo c', t_2', w'} ~ (w3)
\] 
%
, where 
$\config{m, c', t_1, w+l} \to^{*} \config{m_1', \eskip, t_1', w_1'}$
and
$\config{m, c', t_2, w+l} \to^{*} \config{m_2', \eskip, t_2', w_2'}$. 
%
\\
%
By induction hypothesis on $c'$, we know:
\[
t_1' - t_1 = t_2' - t_2.
\]
%
By the rule \rname{seq2}, we have:
%
\[
	\config{m_1', \eskip;\ewhile [b]^l \edo c', t_1', w + l} \to \config{m_1', \ewhile [b]^l \edo c', t_1', w_1'} ~ (w6)
\]
%
\[
	\config{m_2', \eskip;\ewhile [b]^l \edo c', t_2', w + l} \to \config{m_2', \ewhile [b]^l \edo c', t_2', w_2'} ~ (w6)
\]
%
By the termination of the while loop, we know there exist an evaluation by repeating the evaluation in $(w1), (w2), (w3)$ and $(w6)$:
%
\[
	\config{m_1', \ewhile [b]^l \edo c', t_1', w_1'} \to^{*} \config{m_1^n, \eif_w([\efalse]^l, c';\ewhile [b]^l \edo c', \eskip), t_1^n, w_1^n}
\]
%
\[
	\config{m_2', \ewhile [b]^l \edo c', t_2', w_2'} \to^{*} \config{m_2^n, \eif_w([\efalse]^l, c';\ewhile [b]^l \edo c', \eskip), t_2^n, w_2^n}
\]
%
By the evaluation $(w5)$ and induction hypothesis on each iteration, we know the resulting trace $t_1^i$ and $t_2^i$ in the evaluation of each iteration $i = 2, \ldots, k$ satisfies:
%
\[
t_1^i - t_1^{i - 1} = t_2^i - t_2^{i - 1}
\]
%
By unfolding the trace subtraction operation, we have:
%
\[
	t_1^n - t_1 = t_2^n - t_2
\]
%
By evaluation rule \rname{ifw-false}, we have:
%
\[
	\config{m_1^n, \eif_w([\efalse]^l, c';\ewhile [b]^l \edo c', \eskip), t_1^n, w_1^n} 
	\to^{*} \config{m_1^n, \eskip, t_1^n, w_1^n - l}
\]
%
\[
	\config{m_2^n, \eif_w([\efalse]^l, c';\ewhile [b]^l \edo c', \eskip), t_2^n, w_2^n} 
	\to^{*} \config{m_2^n, \eskip, t_2^n, w_2^n - l}
\]
%
Since $t_1^n - t_1 = t_2^n - t_2$, this sub-case is proved.
%
\subcaseL{$v = \efalse$}
%
By evaluation rule \rname{ifw-false}, we have:
%
\[
	\config{m, \eif_w([\efalse]^l, c';\ewhile [b]^l \edo c', \eskip), t_1, w} 
	\to^{*} \config{m, \eskip, t_1, w/l}
\]
%
%
\[
	\config{m, \eif_w([\efalse]^l, c';\ewhile [b]^l \edo c', \eskip), t_2, w} 
	\to^{*} \config{m, \eskip, t_2, w/l}
\]
%
Because $t_1 - t_1 = t_2 - t_2 = []$, this sub-case is proved.
%
\end{subproof}
}
% \end{subproof}
% }
%
% \todo{
% \begin{lem}[Trace Uniqueness]
% \label{lem:tunique}
% Given a program $c$ with a starting memory $m$, \wq{a while map w,}
% for any starting trace $t_1$ and $t_2$, if there exist evaluations
% $$\config{m, c, t_1, w} \to^{*} \config{m_1', \eskip, t_1', w_1'}$$
% % 
% $$\config{m, c, t_2, w} \to^{*} \config{m_2', \eskip, t_2', w_2'}$$
% %
% then:
% %
% \[
% t_1' - t_1 = t_2' - t_2
% \]
% \end{lem}
% %
% \begin{subproof}[Proof of Lemma~\ref{lem:tunique}]
% %
% Proof in File: {\tt ``lem\_tunique.tex''}
% % \todo{
\begin{lem}[Trace Uniqueness]
\label{lem:tunique}
Given a program $c$ with a starting memory $m$, 
for any starting trace $t_1$ and $t_2$, if there exist evaluations
$$\config{m, c, t_1, []} \to^{*} \config{m_1', \eskip, t_1', w_1'}$$
% 
$$\config{m, c, t_2, []} \to^{*} \config{m_2', \eskip, t_2', w_2'}$$
%
then:
%
\[
	t_1' - t_1 = t_2' - t_2
\]
\end{lem}
%
\begin{subproof}[Proof of Lemma~\ref{lem:tunique}]
%
By induction on program $c$, we have following cases:
%
\caseL{ $[\assign x \expr]^{l}$}
%
From the evaluation rule \rname{assn1} and rule \rname{assn2}, we know there exist evaluations
$$
\config{m, [\assign x \expr]^{l}, t_1, w} \to^{*} \config{m[v/x], \eskip, t_1, w}
$$
%
\[
\config{m, [\assign x \expr]^{l}, t_2, w} \to^{*} \config{m[v/x], \eskip, t_2, w}
\]
%
for starting trace $t_1$ and $t_2$ respectively, where $\config{m,e} \aarrow^{*} v$.
%
Since the resulting traces in the 2 evaluations are still $t_1$ and $t_2$, 
and $t_1 - t_1 = t_2 - t_2 = []$, this case is proved.
%
\caseL{$[\assign{x}{\query(\qexpr)}]^{l}$}
%
By repeatedly applying the operational semantics rule \rname{query-e}, we know there exist evaluations
%
\[
\config{m, [\assign{x}{\query(\qexpr)}]^{l}, t_1, w} \to^{*} \config{m, [\assign{x}{\query(\qval)}]^{l}, t_1, w}
\]
%
\[
\config{m, [\assign{x}{\query(\qexpr)}]^{l}, t_2, w} \to^{*} \config{m, [\assign{x}{\query(\qval)}]^{l}, t_2, w}
\]
%
for starting trace $t_1$ and $t_2$ respectively, where $\config{m, \qexpr} \qarrow^{*} \qval$.
%
\\
%
According to the evaluation rule \rname{query-v}, we then have the following 2 transitions with starting trace $t_1$ and $t_2$ respectively:
%
\[
\config{m, [\assign{x}{\query(\qval)}]^{l}, t_1, w} \to \config{m_1, \eskip, t_1 ++ [(\qval, l, w)], w}
\]
%
\[
\config{m, [\assign{x}{\query(\qval)}]^{l}, t_2, w} \to \config{m_2, \eskip, t_2 ++ [(\qval, l, w)], w}
\]
%
The resulting traces are $t_1 ++ [(\qval, l, w)]$, and $t_2 ++ [(\qval, l, w)]$. 
\\
By unfolding the trace subtraction operation, we have:
\[
(t_1 ++ [(\qval, l, w)]) - t_1 = [(\qval, l, w)]
%
\land
%
(t_2 ++ [(\qval, l, w)]) - t_2 = [(\qval, l, w)]
\]
%
This case is proved.
%%
\caseL{$\eif([b]^l, c_1, c_2)$}
%
According to the evaluation rule \rname{if}, we know there exist 2 evaluations for $t_1$ and $t_2$ respectively by repeatedly applying the rule \rname{if}: 
%
\[
\config{m, \eif([b]^l, c_1, c_2), t_1, w} \to^{*} \config{m, \eif([v]^l, c_1, c_2), t_1, w} ~ (1)
\]
%
\[
\config{m, \eif([b]^l, c_1, c_2), t_2, w} \to^{*} \config{m, \eif([v]^l, c_1, c_2), t_2, w} ~ (2)
\]
%
, where $\config{m, b} \barrow^{*} v$ and $v$ is either $\etrue$ or $\efalse$.
\\
Considering 2 subcases where $v$ is either $\etrue$ or $\efalse$:
%
\subcaseL{$v = \etrue$}
%
By applying the evaluation rule \rname{if-t}, we have:
\[
\config{m, \eif([\etrue]^l, c_1, c_2), t_1, w} \to \config{m, c_1, t_1, w} ~ (3)
\]
%
\[
\config{m, \eif([\etrue]^l, c_1, c_2), t_2, w} \to \config{m, c_1, t_2, w} ~ (4)
\]
%
By induction hypothesis on $c_1$ with $m$, $w$ $t_1$ and $t_2$, we know there exist evaluations:
\[
	\config{m, c_1, t_1, w} \to^{*} \config{m_1, \eskip, t_1', w_1} ~ (5)
\]
%
\[
	\config{m, c_1, t_2, w} \to^{*} \config{m_2, \eskip, t_2', w_2} ~ (6)
\]
%
, where $t_1' - t_1 = t_2' - t_2$.
\\
According to the evaluations $(1), (3), (5)$ and $(2), (4), (6)$, we have the following 2 evaluations:
% 
\[
	\config{m, \eif([b]^l, c_1, c_2), t_1, w} \to^{*} \config{m_1, \eskip, t_1', w_1}
\]
%
\[
	\config{m, \eif([b]^l, c_1, c_2), t_2, w} \to^{*} \config{m_2, \eskip, t_2', w_2}
\]
%
Since $t_1' - t_1 = t_2' - t_2$, this sub-case is proved.
%
\subcaseL{$v = \efalse$}
%
By applying the evaluation rule \rname{if-f}, we have:
%
\[
\config{m, \eif([\efalse]^l, c_1, c_2), t_1, w} \to \config{m, c_2, t_1, w} ~ (3)
\]
%
\[
\config{m, \eif([\efalse]^l, c_1, c_2), t_2, w} \to \config{m, c_2, t_2, w} ~ (4)
\]
%
By induction hypothesis on $c_2$ with $m$, $w$ $t_1$ and $t_2$, we know there exist evaluations:
\[
	\config{m, c_2, t_1, w} \to^{*} \config{m_1', \eskip, t_1'', w_1'} ~ (5)
\]
%
\[
	\config{m, c_2, t_2, w} \to^{*} \config{m_2', \eskip, t_2'', w_2'} ~ (6)
\]
%
, where $t_1'' - t_1 = t_2'' - t_2$.
\\
According to the evaluations $(1), (3), (5)$ and $(2), (4), (6)$, we have the following 2 evaluations:
% 
\[
	\config{m, \eif([b]^l, c_1, c_2), t_1, w} \to^{*} \config{m_1', \eskip, t_1'', w_1'}
\]
%
\[
	\config{m, \eif([b]^l, c_1, c_2), t_2, w} \to^{*} \config{m_2', \eskip, t_2'', w_2'}
\]
%
Since $t_1'' - t_1 = t_2'' - t_2$, this sub-case is proved.
%
\caseL{$c_1; c_2$}
%
According to the evaluation rule \rname{seq1}, we have the following 2 evaluations by repeatedly applying \rname{seq1} 
%
\[
\config{m, c_1;c_2, t_1, w} \to^{*} \config{m_1^1, [\eskip]^l;c_2,  t_1^1, w_1^1} ~ (1)
\]
%
%
\[
\config{m, c_1;c_2, t_2, w} \to^{*} \config{m_2^1, [\eskip]^l;c_2,  t_2^1, w_2^1} ~ (2)
\]
%
where 
%
\[
	\config{m, c_1, t_1, w} \to^{*} \config{m_1^1, [\eskip]^l, t_1^1, w_1^1]} ~ (3)
\]
%
%
\[
\config{m, c_1, t_2, w} \to^{*} \config{m_2^1, [\eskip]^l,  t_2^1, w_2^1} ~ (4)
\]
%
%
By induction hypothesis on $c_1$ and $(3), (4)$, we know $t_1^1 - t_1 = t_2^1 - t_2 ~ (a)$.
%
%
According to the evaluation rule \rname{seq2}, we have:
%
\[
\config{m_1^1, [\eskip]^l;c_2, t_1^1, w_1^1} \to \config{m_1^1, c_2, t_1^1, w_1^1}
\]
%
\[
\config{m_2^1, [\eskip]^l;c_2, t_2^1, w_2^1} \to \config{m_2^1, c_2, t_2^1, w_2^1}
\]
%
%
%
By induction hypothesis on $c_2$, $t_2^1$ and $t_1^1$ we have the following 2 evaluations 
%
\[
	\config{m_1^1, c_2, t_1^1, w} \to^{*} \config{m_1^2, \eskip, t_1^2, w_1^2]}
\]
%
\[
	\config{m_2^1, c_2, t_2^1, w} \to^{*} \config{m_2^2, \eskip, t_2^2, w_2^2]}
\]
%
and $t_1^2 - t_1^1 = t_2^2 - t_2^1 ~ (b)$.
%
\\
By $(a)$ and $(b)$, unfolding the trace subtraction operation, we know
%
\[
\exists t_1', t_1'', t_2', t_2''. ~ 
t_1 ++ t_1' = t_1^1 \land t_1^1 ++ t_1'' = t_1^2 
\land t_2 ++ t_2' = t_2^1 \land t_2^1 ++ t_2'' = t_2^2 
\]
%
By writing, we have:
%
\[
\exists t_1', t_1'', t_2', t_2''. ~ 
t_1 ++ t_1' ++ t_1'' = t_1^2 
\land t_2 ++ t_2' ++ t_2'' = t_2^2 
\]
%
Again, by the definition of trace subtraction, we have:
%
\[
	t_2^2 - t_2 = t_1^2 - t_1
\]
%
This case is proved.
%
\caseL{$\ewhile [b]^l \edo c'$}
%
According to the \rname{while-b} and \rname{ifw-b} rule, we have following 2 evaluations:
%
\[
	\config{m, \ewhile [b]^l \edo c', t_1, w} \to^{*}
	\config{m, \eif_w([v]^l, c', \eskip), t_1, w}
\]
%
\[
	\config{m, \ewhile [b]^l \edo c', t_2, w} \to^{*}
	\config{m, \eif_w([v]^l, c', \eskip), t_2, w}
\]
%
where $v$ is either $\etrue$ or $\efalse$.
%
\subcaseL{$v = \etrue$ }
%
By evaluation rule \rname{ifw-true}, we have:
%
\[
	\config{m, \eif_w([\etrue]^l, c';\ewhile [b]^l \edo c', \eskip), t_1, w} 
	\to^{*} \config{m, c';\ewhile [b]^l \edo c', t_1, w+l} ~ (w2)
\]
%
\[
	\config{m, \eif_w([\etrue]^l, c';\ewhile [b]^l \edo c', \eskip), t_2, w} 
	\to^{*} \config{m, c';\ewhile [b]^l \edo c', t_2, w+l} ~ (w2)
\]
%
%
By rule \rname{seq1}, we have the following evaluations:
%
\[
	\config{m, c';\ewhile [b]^l \edo c', t_1, w+l} \to^{*} \config{m', \eskip;\ewhile [b]^l \edo c', t_1', w'} ~ (w3)
\] 
%
%
\[
	\config{m, c';\ewhile [b]^l \edo c', t_2, w+l} \to^{*} \config{m', \eskip;\ewhile [b]^l \edo c', t_2', w'} ~ (w3)
\] 
%
, where 
$\config{m, c', t_1, w+l} \to^{*} \config{m_1', \eskip, t_1', w_1'}$
and
$\config{m, c', t_2, w+l} \to^{*} \config{m_2', \eskip, t_2', w_2'}$. 
%
\\
%
By induction hypothesis on $c'$, we know:
\[
t_1' - t_1 = t_2' - t_2.
\]
%
By the rule \rname{seq2}, we have:
%
\[
	\config{m_1', \eskip;\ewhile [b]^l \edo c', t_1', w + l} \to \config{m_1', \ewhile [b]^l \edo c', t_1', w_1'} ~ (w6)
\]
%
\[
	\config{m_2', \eskip;\ewhile [b]^l \edo c', t_2', w + l} \to \config{m_2', \ewhile [b]^l \edo c', t_2', w_2'} ~ (w6)
\]
%
By the termination of the while loop, we know there exist an evaluation by repeating the evaluation in $(w1), (w2), (w3)$ and $(w6)$:
%
\[
	\config{m_1', \ewhile [b]^l \edo c', t_1', w_1'} \to^{*} \config{m_1^n, \eif_w([\efalse]^l, c';\ewhile [b]^l \edo c', \eskip), t_1^n, w_1^n}
\]
%
\[
	\config{m_2', \ewhile [b]^l \edo c', t_2', w_2'} \to^{*} \config{m_2^n, \eif_w([\efalse]^l, c';\ewhile [b]^l \edo c', \eskip), t_2^n, w_2^n}
\]
%
By the evaluation $(w5)$ and induction hypothesis on each iteration, we know the resulting trace $t_1^i$ and $t_2^i$ in the evaluation of each iteration $i = 2, \ldots, k$ satisfies:
%
\[
t_1^i - t_1^{i - 1} = t_2^i - t_2^{i - 1}
\]
%
By unfolding the trace subtraction operation, we have:
%
\[
	t_1^n - t_1 = t_2^n - t_2
\]
%
By evaluation rule \rname{ifw-false}, we have:
%
\[
	\config{m_1^n, \eif_w([\efalse]^l, c';\ewhile [b]^l \edo c', \eskip), t_1^n, w_1^n} 
	\to^{*} \config{m_1^n, \eskip, t_1^n, w_1^n - l}
\]
%
\[
	\config{m_2^n, \eif_w([\efalse]^l, c';\ewhile [b]^l \edo c', \eskip), t_2^n, w_2^n} 
	\to^{*} \config{m_2^n, \eskip, t_2^n, w_2^n - l}
\]
%
Since $t_1^n - t_1 = t_2^n - t_2$, this sub-case is proved.
%
\subcaseL{$v = \efalse$}
%
By evaluation rule \rname{ifw-false}, we have:
%
\[
	\config{m, \eif_w([\efalse]^l, c';\ewhile [b]^l \edo c', \eskip), t_1, w} 
	\to^{*} \config{m, \eskip, t_1, w/l}
\]
%
%
\[
	\config{m, \eif_w([\efalse]^l, c';\ewhile [b]^l \edo c', \eskip), t_2, w} 
	\to^{*} \config{m, \eskip, t_2, w/l}
\]
%
Because $t_1 - t_1 = t_2 - t_2 = []$, this sub-case is proved.
%
\end{subproof}
}
% \end{subproof}
% }
%

%
%
%
%
%
%
%
% \subsection{SSA Transformation and Soundness of Transformation}
% in File {\tt ``ssa\_transform\_sound.tex''}
% %
\subsection{SSA Transformation}
We use a translation environment $\delta$, to map variables $x$ in the {\tt While} language to those variables $\ssa{x}$ in the SSA language.
We use a name environment denoted as $\Sigma$ as a set of ssa variables, to get a fresh variable by $fresh(\Sigma)$. 
We define $\delta_1 \bowtie \delta_2 $ in a similar way as
\cite{vekris2016refinement}.
%
\[ 
\delta_1 \bowtie \delta_2 = \{ ( x, {\ssa{x_1}, \ssa{x_2}} ) \in 
\mathcal{VAR} \times \mathcal{SVAR} \times \mathcal{SVAR} \mid x \mapsto {\ssa{x_1}} \in \delta_1 , x \mapsto {\ssa{x_2} } \in \delta_2, {\ssa{x_1} \not= {\ssa{x_2} }  }  \} 
\]
%
\[ 
\delta_1 \bowtie \delta_2 / \bar{x} = \{ ( x, {\ssa{x_1}, \ssa{x_2}} ) \in 
\mathcal{VAR} \times \mathcal{SVAR} \times \mathcal{SVAR}
 \mid x \not\in \bar{x} \land x \mapsto {\ssa{x_1}} \in \delta_1 , x \mapsto {\ssa{x_2} } \in \delta_2, {\ssa{x_1} \not= {\ssa{x_2} }   }  \} 
 \]
 %
We call a list of variables $\bar{x}$.
\[
 [\bar{x}, \bar{\ssa{x_1}}, \bar{\ssa{x_2}}] = \{ (x, x_1,x_2)  | \forall 0 \leq i < |\bar{x}|, x = \bar{x }[i] \land x_1 = \bar{x_1}[i] \land x_2 = \bar{x_2 }[i] \land |\bar{x}| = |\bar{x_1}| = |\bar{x_2}|   \}
\]
%
\begin{mathpar}
\boxed{ \delta ; e \hookrightarrow \ssa{e} }
\and
\inferrule{
}{
 \delta ; x \hookrightarrow \delta(x)
}~{\textbf{S-VAR}}
\and
\boxed{ \Sigma; \delta ; c  \hookrightarrow \ssa{c} ; \delta' ; \Sigma' }
\and
\inferrule{
  { \delta ; \bexpr \hookrightarrow \ssa{\bexpr} }
  \and
  { \Sigma; \delta ; c_1 \hookrightarrow \ssa{c_1} ; \delta_1;\Sigma_1 }
  \and
  {\Sigma_1; \delta ; c_2 \hookrightarrow \ssa{c_2} ; \delta_2 ; \Sigma_2 }
  \\
  {[\bar{x}, \ssa{\bar{{x_1}}, \bar{{x_2}}}] = \delta_1 \bowtie \delta_2  }
  \and
   {[\bar{y}, \ssa{\bar{{y_1}}, \bar{{y_2}}}] = \delta \bowtie \delta_1 / \bar{x} }
  \and
   {[\bar{z}, \ssa{\bar{{z_1}}, \bar{{z_2}}}] = \delta \bowtie \delta_2 / \bar{x} }
  \\
  { \delta' =\delta[\bar{x} \mapsto \ssa{\bar{{x}}'} ][\bar{y} \mapsto \ssa{\bar{{y}}'} ][\bar{z} \mapsto \ssa{\bar{{z}}'} ]}
  \and 
  {\ssa{\bar{{x}}', \bar{y}', \bar{z}'} \ fresh(\Sigma_2)
  }
  \and{\Sigma' = \Sigma_2 \cup \{ \ssa{ \bar{x}', \bar{y}', \bar{z}' } \} }
}{
 \Sigma; \delta ; [\eif(\bexpr, c_1, c_2)]^l  \hookrightarrow [\ssa{ \eif(\bexpr, [\bar{{x}}', \bar{{x_1}}, \bar{{x_2}}] ,[\bar{{y}}', \bar{{y_1}}, \bar{{y_2}}] ,[\bar{{z}}', \bar{{z_1}}, \bar{{z_2}}] , {c_1}, {c_2})}]^l; \delta';\Sigma'
}~{\textbf{S-IF}}
%
\and
%
\inferrule{
 {\delta ; \expr \hookrightarrow \ssa{\expr} }
 \and
 {\delta' = \delta[x \mapsto \ssa{{x}} ]}
 \and{ \ssa{x} \ fresh(\Sigma) }
 \and { \Sigma' = \Sigma \cup \{ \ssa{x} \} }
}{
 \Sigma;\delta ; [\assign x \expr]^{l} \hookrightarrow [\ssa{\assign {{x}}{ \expr}}]^{l} ; \delta'; \Sigma'
}~{\textbf{S-ASSN}}
%
\and
%
\inferrule{
 {\delta ; \query \hookrightarrow \ssa{\query}}
 \and
 {\delta ; \qexpr \hookrightarrow \ssa{\qexpr}}
 \and
 {\delta' = \delta[x \mapsto \ssa{x} ]}
 \and{ \ssa{x} \ fresh(\Sigma) }
  \and { \Sigma' = \Sigma \cup \{ \ssa{x} \} }
}{
 \Sigma;\delta ; [\assign{x}{\query(\qexpr)}]^{l} \hookrightarrow 
 [\assign {\ssa{x}}{ \ssa{\query(\qexpr)}}]^{l} ; \delta';\Sigma'
}~{\textbf{S-QUERY}}
%
%%
\and
%
%
\and
%
\inferrule{
    { \Sigma; \delta ; c \hookrightarrow \ssa{c_1} ; \delta_1; \Sigma_1 }
     \and
    { [ \bar{x}, \ssa{\bar{{x_1}}}, \ssa{\bar{{x_2}}} ] = \delta \bowtie \delta_1 }
    \\
     {
     \ssa{\bar{{x}}'} \ fresh(\Sigma_1 )}
    \and {\delta' = \delta[\bar{x} \mapsto \ssa{\bar{{x}}'}]}
    \and 
     {\delta' ; \bexpr \hookrightarrow \ssa{\bexpr} }
     \and
    {\ssa{c' = c_1[\bar{x}'/ \bar{x_1}]   } }
  }{ 
  \Sigma; \delta ;  \ewhile ~ [\bexpr]^{l} ~ \edo ~ c 
  \hookrightarrow 
  \ssa{\ewhile ~ [\bexpr]^{l}, 0, [\bar{{x}}', \bar{{x_1}}, \bar{{x_2}}] ~ \edo ~ {c} } ; \delta'; \Sigma_1 \cup \{\ssa{\bar{x}'}  \}
}~{\textbf{S-WHILE}
}
\and
%
\inferrule{
 {\Sigma;\delta ; c_1 \hookrightarrow \ssa{c_1} ; \delta_1; \Sigma_1} 
 \and
 {\Sigma_1; \delta_1 ; c_2 \hookrightarrow \ssa{c_2} ; \delta'; \Sigma'} 
}{
\Sigma;\delta ; c_1 ; c_2 \hookrightarrow \ssa{c_1} ; \ssa{c_2} \ ; \delta';\Sigma'
}~{\textbf{S-SEQ}}
\end{mathpar}

\paragraph{Concrete examples.}
\[
c_1 \triangleq
\begin{array}{l}
     \left[x \leftarrow \query(1) \right]^1; \\
     \eif \; (x ==0)^{2} \; \\
    \ethen \; \left[y \leftarrow \query(2) \right]^3\; \\
    \eelse \; \left[y \leftarrow 0 \right]^4 ; \\
    \eif \; (x == 1 )^5\; \\
    \ethen \; \left[ y \leftarrow 0 \right]^6\; \\
    \eelse \; \left[y \leftarrow \query(2) \right]^7\\
\end{array}
%
%
\hspace{20pt} \hookrightarrow  \hspace{20pt}
%
\begin{array}{l}
     \left[ \ssa{x_1} \leftarrow \query(1) \right]^1; \\
     \eif \; (\ssa{x_1 ==0})^{2}, [\ssa{ y_3, y_1,y_2  }],[],[]  \; \\
    \ethen \; \left[ \ssa{y_1} \leftarrow \query(2) \right]^3\; \\
    \eelse \; \left[\ssa{y_2 \leftarrow 0 } \right]^4 ; \\
    \eif \; (\ssa{x_1 == 1} )^{5} , [\ssa{ y_6, y_4, y_5 } ] \; \\
    \ethen \; \left[ \ssa{y_4 \leftarrow 0} \right]^6\; \\
    \eelse \; \left[\ssa{y_5} \leftarrow \query(2) \right]^7\\
\end{array}
\]
\[
c_2 \triangleq
\begin{array}{l}
   \left[ x \leftarrow \query(1) \right]^1; \\
   \left[y \leftarrow \query(2) \right]^2 ; \\
    \eif \;( x + y == 5 )^3\; \\
    \ethen \;\left[ z \leftarrow \query(3)\right]^4 \; \\
    \eelse \;\left[ \ssa{\eskip}\right]^5 ; \\
   \left[ w \leftarrow q_4 \right]^6; \\
\end{array}
\hspace{20pt} \hookrightarrow \hspace{20pt}
%
\begin{array}{l}
   \left[ \ssa{ x_1 } \leftarrow \query(1) \right]^1; \\
   \left[\ssa{ y_1} \leftarrow \query(2) \right]^2 ; \\
    \eif \;( \ssa{ x_1 + y_1 == 5} )^3, [ ],[] ,[ ]\; \\
    \ethen \;\left[ \ssa{ z_1 }
    \leftarrow \query(3)\right]^4 \; \\
    \eelse \;\left[ \eskip\right]^5 ; \\
   \left[ \ssa{ w_1} \leftarrow \query(4) \right]^6; \\
\end{array}
\]

{
\[
c_3 \triangleq
\begin{array}{l}
     \left[x \leftarrow \query(1) \right]^1 ; \\
     \left[i \leftarrow 0 \right]^2 ; \\
    \ewhile ~  [i < 100]^3 ~ \edo
    \\
    ~ \Big( 
    \left[z \leftarrow \query(3) \right]^4; \\
    \left[x \leftarrow z + x \right]^5; \\
    \left[i \leftarrow i + 1 \right]^6
    \Big) ;
\end{array}
%
\hspace{20pt} \hookrightarrow \hspace{20pt} 
%
\begin{array}{l}
     \left[\ssa{x_1} \leftarrow \query(1) \right]^1 ; \\
     \left[\ssa{i_1} \leftarrow 0 \right]^2 ; \\
    \ewhile
    ~ [\ssa{i_1} < 100]^3, 0,
    ~\ssa{[ x_3,x_1 ,x_2 ], [i_3, i_1, i_2] }~
    \edo \\
    ~ \Big( 
    \left[\ssa{z_1} \leftarrow \query(3) \right]^4; \\
    \left[ \ssa{x_2} \leftarrow \ssa{z_1 + x_3} \right]^5; \\
    \left[\ssa{i_2} \leftarrow \ssa{i_3} + 1 \right]^6
    \Big) ;
\end{array}
\]
}
%
\begin{figure}
   \[
 \begin{array}{lll}
    | \ewhile ~ [ \sbexpr ]^{l}, n, [\bar{\ssa{x}}, \bar{\ssa{x_1}}, \bar{\ssa{x_2}}] 
    ~ \edo ~  \ssa{c}|  
    &=& \ewhile ~ [|\sbexpr|]^{l},  ~ \edo ~ |\ssa{c}| 
	\\
    |\ssa{c_1 ; c_2}|  &=& |\ssa{c_1}| ; |\ssa{c_2}| 
    \\
    |[\eif(\sbexpr,
    [ \bar{\ssa{x}}, \bar{\ssa{x_1}}, \bar{\ssa{x_2}}] ,
    [ \bar{\ssa{y}}, \bar{\ssa{y_1}}, \bar{\ssa{y_2}}] , 
    [\bar{\ssa{z}}, \bar{\ssa{z_1}}, \bar{\ssa{z_2}}] , 
    \ssa{ c_1, c_2)}]^{l}|  
    &=&
    [\eif(|\sbexpr|, |\ssa{ c_1}|, |\ssa{c_2}|)]^{l}
    \\
    | [\assign {\ssa{x}}{\ssa{\expr}}]^{l}| & = & [\assign {|\ssa{x}|}{|\ssa{\expr}|} ]^{l}
    \\
    | [\assign {\ssa{x}}{\query(\ssa{\qexpr})} ]^{l} | & = & [\assign {|\ssa{x}|}{|\query(\ssa{\qexpr})|}]^{l}
    \\
    |\ssa{x}_i| & = & x 
    \\
    |n | & = & n 
    \\
    | \saexpr_1 \oplus_{a} \saexpr_2 | & = &  |\ssa{\aexpr_1}| \oplus_a |\ssa{\aexpr_2}| \\
    | \sbexpr_1 \oplus_{b} \sbexpr_2 | & = &  |\sbexpr_1| \oplus_b |\sbexpr_2|
 \end{array}
\]
    \caption{The Erasure of SSA}
    \label{fig:ssa_erasure-while}
\end{figure}
%
%
%
% 
%
\subsection{The Soundness of the Transformation}
In this subsection, we show our transformation from the {\tt While} language to its SSA form is sound with respect to the adaptivity. 
To be specific, a transformed program $\ssa{c}$ starting with appropriate configuration, generates the same trace as the program before the transformation $c$, in its corresponding configuration.
%
%
\begin{defn}[\todo{Written Variables}].
\\
We defined the assigned variables in the while language program $c$ as $\avars{c}$,the assigned variables in the ssa-form program $\ssa{c}$ as $\avarssa{\ssa{c}}$ defined as follows.
\[
\begin{array}{lll}
    \avars{\assign{x}{\expr}} & =& \{ x \} \\
    \avars{\assign{x}{\query(\qexpr)}} & =& \{ x \} \\
    \avars{c_1; c_2}  & = & \avars{c_1} \cup \avars{c_2} \\
    \avars{\ewhile ~ \bexpr ~ \edo ~ c} &= &  \avars{c} \\
    \avars{\eif(\bexpr, c_1, c_2)} & =&  \avars{c_1} \cup \avars{c_2}\\
\end{array} 
\]
%
\[
\begin{array}{lll}
    \avarssa{\ssa{\assign{x}{\expr}}} & =& \{ \ssa{x} \}
    \\
    \avarssa{\ssa{\assign{x}{\query(\ssa{\qexpr})}}} & =& \{ \ssa{x} \}
    \\
    \avarssa{\ssa{c_1; c_2 } }  & = & \avarssa{\ssa{c}_1} \cup \avarssa{\ssa{c}_2}
    \\
    \avarssa{\ewhile ~ \ssa{\bexpr, n, [\bar{x}, \bar{x_1}, \bar{x_2}] ~ \edo ~ \ssa{c}}}
    & = &  
    \{\ssa{\bar{x}}\} \cup \avarssa{\ssa{c}} 
    \\
    \avarssa{\eif(\ssa{\bexpr,[\bar{x}, \bar{x_1}, \bar{x_2}],[\bar{y}, \bar{y_1}, \bar{y_2}],[\bar{z}, \bar{z_1}, \bar{z_2}], c_1, c_2} )} 
    & =&  \{ \ssa{\bar{x}},\ssa{\bar{y}} , \ssa{\bar{z}} \} 
    \cup \avarssa{\ssa{c_1}} \cup \avarssa{\ssa{c_2}}\\
\end{array}
\]
\end{defn}
\begin{defn}[\todo{Read Variables}].
\\
{
The variables read in the while language programs $c$ as $\vars{c}$, variables used in ssa-form program $\ssa{c}$ : 
}
\[
\begin{array}{lll}
    \vars{\assign{x}{\expr}} & =& \vars{\expr}  \\
    \vars{\assign{x}{\query(\qexpr)}} & =&\{  \} \\
    \vars{ c_1; c_2  }  & = & \vars{c_1} \cup \vars{c_2} \\
    \vars{  \eloop ~ \aexpr ~ \edo ~ c  } &= &\vars{\aexpr} \cup \vars{c} \\
    \vars{\eif(\bexpr, c_1, c_2)} & =& \vars{\bexpr} \cup \vars{c_1} \cup \vars{c_2}\\
\end{array}
\]
\[
\begin{array}{lll}
    \varssa{\ssa{\assign{x}{\expr}}} & =& \varssa{\ssa{\expr}}  \\
    \varssa{\ssa{\assign{x}{\query(\qexpr)}}} & =& \{  \} \\
    \varssa{ \ssa{c_1; c_2}  }  & = & \varssa{\ssa{c}_1} \cup \varssa{\ssa{c}_2} \\
    % \varssa{  \eloop ~ \ssa{\aexpr, n, [\bar{x}, \bar{x_1}, \bar{x_2}] ~ \edo ~ c} } &= &\varssa{\ssa{\aexpr}} \cup \varssa{\ssa{c}}  \cup \{ \ssa{\bar{x_1}} \} \cup \{ \ssa{\bar{x_2}} \}\\
    {\varssa{  \ewhile ~ \ssa{\bexpr, n, [\bar{x}, \bar{x_1}, \bar{x_2}] ~ \edo ~ c} }} 
    &= &
    \varssa{\ssa{\bexpr}} \cup \varssa{\ssa{c}}  \cup \{ \ssa{\bar{x_1}} \} \cup \{ \ssa{\bar{x_2}} \}\\
    \varssa{\eif(\ssa{\bexpr,[\bar{x}, \bar{x_1}, \bar{x_2}], [\bar{y}, \bar{y_1}, \bar{y_2}],[\bar{z}, \bar{z_1}, \bar{z_2}], c_1, c_2} )} & =& \varssa{\ssa{\bexpr}} \cup \varssa{\ssa{c_1}} \cup \varssa{\ssa{c_2}} \cup \{\ssa{\bar{x_1}, \bar{x_2},\bar{y_1}, \bar{y_2},\bar{z_1}, \bar{z_2} }\}  \\
\end{array}
\]
\end{defn}
%
\begin{defn}[\todo{Necessary Variables}].
\\
{
We call the variables needed to be assigned before executing the program $c$ as necessary variables $\fv{c}$. Its ssa form is : $\fvssa{\ssa{c}}$.
}  
 \[
 \begin{array}{lll}
     \fvars{\assign{x}{\expr} }  & = & \vars{\expr}  \\
     \fvars{\assign{x}{\query(\qexpr)} }  & = & \{ \}  \\
     {\fvars{  \ewhile ~ \bexpr ~ \edo ~ c  } }&= & \vars{\bexpr} \cup \fvars{c} \\
     \fvars{\eif(\bexpr, c_1, c_2)} & =& \vars{\bexpr} \cup \fvars{c_1} \cup \fvars{c_2}  \\
      \fvars{c_1 ; c_2} & = & \fvars{c_1} \cup ( \fvars{c_2} - \avars{c_1})
 \end{array}
 \]
 \[
 \begin{array}{lll}
     \fvssa{\ssa{\assign{x}{\expr}} }  & = & \varssa{\ssa{\expr}}  \\
     \fvssa{ \ssa{ \assign{x}{\query(\qexpr)}} }  & = & \{ \}  \\
     {\fvssa{  \ewhile ~ \ssa{\bexpr, n, [\bar{x}, \bar{x_1}, \bar{x_2}] ~ \edo ~ c} } }
     &= & 
     \varssa{\ssa{\bexpr}} \cup \fvssa{\ssa{c}}[\ssa{ \bar{x_1}} / \ssa{\bar{x}}]\\
     \fvssa{\eif(\ssa{\bexpr,[\bar{x}, \bar{x_1}, \bar{x_2}],[\bar{y}, \bar{y_1}, \bar{y_2}],[\bar{z}, \bar{z_1}, \bar{z_2}], c_1, c_2} )} & =& \varssa{\ssa{\bexpr}} \cup \fvssa{\ssa{c_1}} \cup \fvssa{\ssa{c_2}}  \\
      \fvssa{\ssa{c_1 ; c_2}} & = & \fvssa{\ssa{c_1}} \cup ( \fvssa{\ssa{c_2}} - \avarssa{\ssa{c_1}})
 \end{array}
 \]
%
\end{defn}
%
The Lemma~\ref{lem:fv} and \ref{lem:same_value} proved the preserving properties for variables and values during the transformation.
%
\begin{lem}[Variable Preserving]
\label{lem:fv}
If $\Sigma;\delta ; c \hookrightarrow \ssa{c} ; \delta';\Sigma' $, $\fvssa{\ssa{c}} = \delta(\fvars{c})$. 
\end{lem}
\begin{proof}
 By induction on the transformation.
 \begin{itemize}
    \caseL{Case $\inferrule{
  { \delta ; \bexpr \hookrightarrow \ssa{\bexpr} }
  \and
  { \delta ; c_1 \hookrightarrow \ssa{c_1} ; \delta_1 }
  \and
  {\delta ; c_2 \hookrightarrow \ssa{c_2} ; \delta_2 }
  \\
  {[\bar{x}, \ssa{\bar{{x_1}}, \bar{{x_2}}}] = \delta_1 \bowtie \delta_2  }
  \and
   {[\bar{y}, \ssa{\bar{{y_1}}, \bar{{y_2}}}] = \delta \bowtie \delta_1 / \bar{x} }
  \and
   {[\bar{z}, \ssa{\bar{{z_1}}, \bar{{z_2}}}] = \delta \bowtie \delta_2 / \bar{x} }
  \\
  { \delta' =\delta[\bar{x} \mapsto \ssa{\bar{{x}}'} ]}
  \and 
  {\ssa{\bar{{x}}', \bar{y}', \bar{z}'} \ fresh }
}{
 \delta ; [\eif(\bexpr, c_1, c_2)]^l  \hookrightarrow [\ssa{ \eif(\bexpr, [\bar{{x}}', \bar{{x_1}}, \bar{{x_2}}] ,[\bar{{y}}', \bar{{y_1}}, \bar{{y_2}}] ,[\bar{{z}}', \bar{{z_1}}, \bar{{z_2}}] , {c_1}, {c_2})}]^l; \delta'
}~{\textbf{S-IF}} $}
From the definition of $\fvssa{[\eif(\sbexpr, [\bar{\ssa{x'}}, \bar{\ssa{x_1}}, \bar{\ssa{x_2}}] , \ssa{c_1}, \ssa{c_2})]^l} = \varssa{\ssa{\bexpr}} \cup \fvssa{\ssa{c_1}} \cup \fvssa{\ssa{c_2}}$. We want to show: \[\varssa{\ssa{\bexpr}}) \cup \fvssa{\ssa{c_1}} \cup \fvssa{\ssa{c_2}} = \delta( \vars{\bexpr}) \cup \delta(\fv{c_1}) \cup \delta(\fv{c_2}  )\]
By induction hypothosis on the second and third premise, we know that : $\fvssa{\ssa{c_1}} = \delta(\fv{c_1}) $ and $\fvssa{\ssa{c_2}} = \delta(\fv{c_2}) $.  We still need to show that: 
\[
  \varssa{\ssa{\bexpr}} = \delta(\vars{\bexpr})
\] 
From the first premise, we know that $\vars{b} \subseteq \dom(\delta)$. This is goal is proved by the rule $\textbf{S-VAR}$ on all the variables in $\bexpr$.\\
{\caseL{Case
$\inferrule{
    { \Sigma; \delta ; c \hookrightarrow \ssa{c_1} ; \delta_1; \Sigma_1 }
     \and
    { [ \bar{x}, \ssa{\bar{{x_1}}}, \ssa{\bar{{x_2}}} ] = \delta \bowtie \delta_1 }
    \\
     {\ssa{\bar{{x}}'} \ fresh(\Sigma_1 )}
    \and {\delta' = \delta[\bar{x} \mapsto \ssa{\bar{{x}}'}]}
    \and 
     {\delta' ; \bexpr \hookrightarrow \ssa{\bexpr} }
     \and
    {\ssa{c' = c_1[\bar{x}'/ \bar{x_1}]   } }
    % \and{ \delta' ; c \hookrightarrow \ssa{c'} ; \delta'' }
  }{ 
  \Sigma; \delta ;  \ewhile ~ [\bexpr]^{l} ~ \edo ~ c 
  \hookrightarrow 
  \ssa{\ewhile ~ [\bexpr]^{l}, 0, [\bar{{x}}', \bar{{x_1}}, \bar{{x_2}}] ~ \edo ~ {c} } ; \delta'; \Sigma_1 \cup \{\ssa{\bar{x}'}  \}
}~{\textbf{S-WHILE}}
$}
}
{
Unfolding the definition, we need to show:
\[\varssa{\ssa{\bexpr}} \cup \fvssa{\ssa{c'}}[\ssa{ \bar{x_1}} / \ssa{\bar{x}}] = \delta (\vars{\bexpr}) \cup \delta(\fv{c} ) \]
We can similarly show that $\varssa{\ssa{\bexpr}} = \delta(\vars{b})$ as in the if case. We still need to show that: 
\[
 \fvssa{\ssa{c_1[\bar{x}' / \bar{x_1}]}}[ \ssa{ \bar{x_1} } / \ssa{\bar{x}'}] =  \delta(\fv{c} )
\]
It is proved by induction hypothesis on $  { \Sigma; \delta ; c \hookrightarrow \ssa{c_1} ; \delta_1; \Sigma_1 }$.\\
}
%
\caseL{Case $\inferrule{
 {\Sigma;\delta ; c_1 \hookrightarrow \ssa{c_1} ; \delta_1; \Sigma_1} 
 \and
 {\Sigma_1; \delta_1 ; c_2 \hookrightarrow \ssa{c_2} ; \delta'; \Sigma'} 
}{
\Sigma;\delta ; c_1 ; c_2 \hookrightarrow \ssa{c_1} ; \ssa{c_2} \ ; \delta';\Sigma'
}~{\textbf{S-SEQ}}$}
To show:
  \[ \fvssa{\ssa{c_1}} \cup ( \fvssa{\ssa{c_2}} - \avarssa{\ssa{c_1}}) = \delta(\fv{c_1} )\cup \delta( \fv{c_2} - \avars{c_1}) \]
  By induction hypothesis on the first premise, we know that : $ \fvssa{\ssa{c_1}} = \delta(\fv{c_1} ) $, still to show: 
    \[ ( \fvssa{\ssa{c_2}} - \avarssa{\ssa{c_1}}) = \delta( \fv{c_2} - \avars{c_1})
    \]
    We know that $\delta_1 = \delta [\avars{c_1} \mapsto \avarssa{\ssa{c_1}} ]$, so by induction hypothesis, we know: $ \fvssa{\ssa{c_2}} = \delta[\avars{c_1} \mapsto \avarssa{\ssa{c_1}} ]( \fv{c_2})  = \delta(\fv{c_2}) \cup \avarssa{\ssa{c_1}} - \delta(\avars{c_1}) $.
    
    This case is proved.
 \end{itemize}
 
\end{proof}

{
We first define a good memory in the {\tt While} language $m$ or in the ssa language $\ssa{m}$ with respect to a translation environment $\delta$, denoted as $m \vDash \delta$ and $\ssa{m} \vDash \delta$ respectively. 
%
\begin{defn}[Well Defined Memory].
\begin{enumerate}
    % \item $m \vDash c \triangleq \forall x \in \fv{c}, \exists v, (x, v) \in m$.
    \item $ m \vDash \delta  \triangleq \forall x \in \dom(\delta), \exists v, (x,v) \in m$.
    % \item $\ssa{m} \vDash_{ssa} \ssa{c} \triangleq \forall \ssa{x} \in \fvssa{\ssa{c}}, \exists v, (\ssa{x}, v) \in \ssa{m}$.
    \item $ \ssa{m} \vDash_{ssa} \delta  \triangleq \forall \ssa{x} \in \codom(\delta), \exists v, (\ssa{x},v) \in \ssa{m}$.
\end{enumerate}
\end{defn}
%
The part declared in the translation environment $\delta$ in a ssa memory $\ssa{m}$ can be reverted to corresponding part of the memory $m$ with an inverse of $\delta$ as follows.
%
\begin{defn}[Inverse of Transformed memory]
 $m = \delta^{-1}(\ssa{m}) \triangleq \forall x \in \dom(\delta), (\delta(x), m(x)) \in \ssa{m} $.
\end{defn}
}
%
\begin{lem}[Value Preserving].
\label{lem:same_value}
{
Given $\delta; e \hookrightarrow \ssa{e}$,  $\forall m. m \vDash \delta. \forall \ssa{m}, \ssa{m} \vDash_{ssa} \delta \land m = \delta^{-1}(\ssa{m})$, then $\config{m, e} \to v $ and $\config{
\ssa{m}, \ssa{e}} \to {v}$.
}
\end{lem}

\begin{thm}[Soundness of transformation]
Given $\Sigma; \delta ; c \hookrightarrow \ssa{c} ; \delta';\Sigma' $, $\forall m. m \vDash \delta. \forall \ssa{m}, \ssa{m} \vDash_{ssa} \delta \land m = \delta^{-1}(\ssa{m})$, if there exist an execution of $c$ in the while language, starting with a trace $t$ and loop maps $w$, $\config{m, c, t, w} \to^{*} \config{m', \eskip, t', w' } $,  then there also exists a corresponding execution of $\ssa{c}$ in the ssa language so that 
  $\config{  {\ssa{m}}, \ssa{c}, t, w} \to^{*} \config{{  \ssa{m'}}, \eskip, t', w' } $ and $ m' = \delta'^{-1}(\ssa{m'}) $.
\end{thm}

\begin{proof}
 We assume that $q$ is the same when transformed to $\ssa{q}$, as the primitive in both languages. And the value remains the same during the transformation.  
 It is proved by induction on the transformation rules.
 \begin{itemize}
   \caseL{Case $\inferrule{
 {\Sigma;\delta ; c_1 \hookrightarrow \ssa{c_1} ; \delta_1;\Sigma_1} 
 \and
 {\Sigma_1; \delta_1 ; c_2 \hookrightarrow \ssa{c_2} ; \delta'; \Sigma'} 
}{
\Sigma;\delta ; c_1 ; c_2 \hookrightarrow \ssa{c_1} ; \ssa{c_2} \ ; \delta';\Sigma'
}~{\textbf{S-SEQ}}$}
We choose an arbitrary memory $m$ so that $m \vDash \delta$, we choose a trace $t$ and a loop maps $w$.
\[
\inferrule
{
{\config{m, c_1,  t,w} \xrightarrow{}^{*} \config{m_1, \eskip,  t_1,w_1}}
\and
{\config{m_1, c_2,  t_1,w_1} \xrightarrow{}^{*} \config{m', \eskip,  t',w'}}
}
{
\config{m, c_1; c_2,  t,w} \xrightarrow{}^{*} \config{m', \eskip, t',w'}
}
\]
 We choose the transformed memory ${\ssa{m}} $ so that  $ m =\delta^{-1}(\ssa{m})$.
 
 To show: $ \config{\ssa{ m, c_1;c_2 }, t, w } \xrightarrow{}^{*} \config{\ssa{m', \eskip}, t'. w' }$ and $ m' = \delta'^{-1} (\ssa{m'}) $.
 
 By induction hypothesis on the first premise, we have:
 \[ \config{\ssa{m, c_1}, t,w} \xrightarrow{}^{*} \config{\ssa{m_1, \eskip},t_1,w_1 } \land m_1 = \delta_1^{-1}(\ssa{m_1}) \]
  By induction hypothesis on the second premise, using the conclusion $ m_1 = \delta_1^{-1}(\ssa{m_1}) $.
  We have:
  \[
   \config{\ssa{m_1, c_2}, t_1,w_1} \xrightarrow{}^{*} \config{\ssa{m', \eskip},t',w' } \land m' = \delta'^{-1}(\ssa{m'})
  \]
  So we know that 
  \[
  \inferrule{
  { \config{\ssa{m, c_1}, t,w} \xrightarrow{}^{*} \config{\ssa{m_1, \eskip},t_1,w_1 }  }
  \and
  { \config{\ssa{m_1, c_2}, t_1,w_1} \xrightarrow{}^{*} \config{\ssa{m', \eskip},t',w' } }
  }{
  \config{\ssa{m, c_1;c_2 }, t,w} \xrightarrow{}^{*} \config{\ssa{m', \eskip}, t' , w' }
  }
  \]
 \caseL{Case $\inferrule{
 { \delta ; \expr \hookrightarrow \sexpr}
 \and
 {\delta' = \delta[x \mapsto \ssa{x} ]}
 \and{ \ssa{x} \ fresh(\Sigma) }
 \and {\Sigma' = \Sigma \cup \{x\} }
}{
 \Sigma;\delta ; [\assign x \expr]^{l} \hookrightarrow [\assign {\ssa{x}}{ \ssa{\expr}}]^{l} ; \delta';\Sigma'
}~{\textbf{S-ASSN}} $ }

 We choose an arbitrary memory $m$ so that $m \vDash \delta$, we choose a trace $t$ and a loop maps $w$, we know that the resulting trace is still $t$ from its evaluation rule $\textbf{assn}$ when we suppose $m, \expr \to v$.
 \[
 \inferrule
{
}
{
\config{m, [\assign x v]^{l},  t,w} \xrightarrow{} \config{m[v/x], [\eskip]^{l}, t,w}
}
~\textbf{assn}
 \]
 We choose the transformed memory ${\ssa{m}} $ so that  $ m =\delta^{-1}(\ssa{m})$.
 
 To show: $\config{\ssa{m}, [\assign {\ssa{x}}{ \ssa{\expr}}]^{l} , t, w} \to^{*} \config{\ssa{m'}, \eskip, t, w} $ and $ m' = \delta'^{-1}(\ssa{m'}) $.
 
 From the rule \textbf{ssa-assn}, we assume $\ssa{m}, \ssa{\expr} \to \ssa{v}$, we know that 
 \[
 \inferrule
{
}
{
\config{\ssa{ m, [\assign x v]^{l}},  t,w } \xrightarrow{} \config{\ssa{m[x \mapsto v], [\eskip]^{l}}, t,w }
}
~\textbf{ssa-assn}
 \]
 We also know that $\delta'= \delta[x \mapsto \ssa{x}]$ and $m = \delta^{-1}(\ssa{m})$, $m'= m[v/x]$. We can show that $ m[v/x] = \delta[x \mapsto \ssa{x}]^{-1}(\ssa{m}[\ssa{x} \mapsto v]) $.
 
\caseL{Case $\inferrule{
 {\delta ; q \hookrightarrow \ssa{q}}
 \and
 {\delta ; \expr \hookrightarrow \ssa{\expr}}
 \and
 {\delta' = \delta[x \mapsto \ssa{x} ]}
 \and{ \ssa{x} \ fresh(\Sigma) }
 \and{ \Sigma' = \Sigma \cup \{x\} }
}{
 \Sigma;\delta ; [\assign{x}{\query(\qexpr)}]^{l} \hookrightarrow [\assign {\ssa{x}}{ \ssa{\query(\qexpr)}}]^{l} ; \delta'
}~{\textbf{S-QUERY}}$} 
We choose an arbitrary memory $m$ so that $m \vDash \delta$, we choose a trace $t$ and a loop maps $w$, we know that when we suppose $\config{m, \expr} \to v$.
 \[
\inferrule
{
\query(v)(D) = \qval 
}
{
\config{m, [\assign{x}{\query(v)}]^l, t, w} \xrightarrow{} \config{m[ \qval/ x], \eskip,  t \mathrel{++} [\query(v),l,w )],w }
}
~\textbf{query}
 \]
 We choose the transformed memory ${\ssa{m}} $ so that  $ m =\delta^{-1}(\ssa{m})$.
 
 To show: $\config{\ssa{m}, [\assign {\ssa{x}}{ \ssa{\query(\qexpr)}}]^{l} , t, w} \to^{*} \config{\ssa{m'}, \eskip, t, w} $ and $ m' = \delta'^{-1}(\ssa{m'}) $.
 
 From the rule \textbf{ssa-query}, we know that 
 \[
 \inferrule
{
\ssa{\query(v)(D) = \qval} 
}
{
\config{ \ssa{ m, [\assign{\ssa{x}}{\ssa{\query(\qexpr)}}]^l}, t, w} \xrightarrow{} \config{\ssa{  m[  x \mapsto v], \eskip,}  t \mathrel{++} [(q^{(l,w )},v)],w }
}
~\textbf{ssa-query}
 \]
 We also know that $\delta'= \delta[x \mapsto \ssa{x}]$ and $m = \delta^{-1}(\ssa{m})$, $m'= m[v/x]$. We can show that $ m[v/x] = \delta[x \mapsto \ssa{x}]^{-1}(\ssa{m}[\ssa{x} \mapsto v]) $.

  \caseL{Case $\inferrule{
  { \delta ; \bexpr \hookrightarrow \ssa{\bexpr} }
  \and
  { \Sigma; \delta ; c_1 \hookrightarrow \ssa{c_1} ; \delta_1;\Sigma_1 }
  \and
  {\Sigma_1; \delta ; c_2 \hookrightarrow \ssa{c_2} ; \delta_2 ; \Sigma_2 }
  \\
  {[\bar{x}, \ssa{\bar{{x_1}}, \bar{{x_2}}}] = \delta_1 \bowtie \delta_2  }
  \and
   {[\bar{y}, \ssa{\bar{{y_1}}, \bar{{y_2}}}] = \delta \bowtie \delta_1 / \bar{x} }
  \and
   {[\bar{z}, \ssa{\bar{{z_1}}, \bar{{z_2}}}] = \delta \bowtie \delta_2 / \bar{x} }
  \\
  { \delta' =\delta[\bar{x} \mapsto \ssa{\bar{{x}}'} ][\bar{y} \mapsto \ssa{\bar{{y}}'} ][\bar{z} \mapsto \ssa{\bar{{z}}'} ]}
  \and 
  {\ssa{\bar{{x}}', \bar{y}', \bar{z}'} \ fresh(\Sigma_2)
  }
  \and{\Sigma' = \Sigma_2 \cup \{ \ssa{ \bar{x}', \bar{y}', \bar{z}' } \} }
}{
 \Sigma; \delta ; [\eif(\bexpr, c_1, c_2)]^l  \hookrightarrow [\ssa{ \eif(\bexpr, [\bar{{x}}', \bar{{x_1}}, \bar{{x_2}}] ,[\bar{{y}}', \bar{{y_1}}, \bar{{y_2}}] ,[\bar{{z}}', \bar{{z_1}}, \bar{{z_2}}] , {c_1}, {c_2})}]^l; \delta';\Sigma'
}~{\textbf{S-IF}}$}
We choose an arbitrary memory $m$ so that $m \vDash \delta$, we choose a trace $t$ and a loop maps $w$.
There are two possible evaluation rules depending on the the condition $b$, we choose the case when $b = \etrue$, we know there is an execution in ssa language so that $\ssa{\bexpr} = \etrue$, we use the rule $\textbf{if-t}$.  
 \[\inferrule
{
}
{
\config{m, [\eif(\etrue, c_1, c_2)]^{l},t,w} 
\xrightarrow{} \config{m, c_1,  t,w} \to^{*} \config{m', \eskip, t', w'}
}
\]
 We choose the transformed memory ${\ssa{m}} $ so that  $ m =\delta^{-1}(\ssa{m})$.
 
 To show: $\config{\ssa{m}, [\eif(\etrue, [\bar{\ssa{x}}', \bar{\ssa{x_1}}, \bar{\ssa{x_2}}] ,[\bar{\ssa{y}}', \bar{\ssa{y_1}}, \bar{\ssa{y_2}}] ,[\bar{\ssa{z}}', \bar{\ssa{z_1}}, \bar{\ssa{z_2}}] , c_1, c_2)]^{l}, t, w} \to^{*} \config{\ssa{m'}, \eskip, t', w'} $ and $ m' = \delta'^{-1}(\ssa{m'}) $.

We use the corresponding rule $\textbf{SSA-IF-T}$.  
\[
\inferrule
{
}
{
\config{\ssa{ { m} , [\eif(\etrue, [\bar{\ssa{x}}', \bar{\ssa{x_1}}, \bar{\ssa{x_2}}] , [\bar{\ssa{y}}', \bar{\ssa{y_1}}, \bar{\ssa{y_2}}] ,[\bar{\ssa{z}}', \bar{\ssa{z_1}}, \bar{\ssa{z_2}}] , \ssa{c_1, c_2})]^{l}},t,w} 
\xrightarrow{} \\ \config{\ssa{ m, c_1}; \eifvar(\ssa{\bar{x}', \bar{x_1}});\eifvar(\ssa{\bar{y}', \bar{y_2}});\eifvar(\ssa{\bar{z}', \bar{z_1}}),  t,w } 
}~\textbf{ssa-if-t}
\]
By induction hypothesis on $ \Sigma;\delta ; c_1 \hookrightarrow  \ssa{c_1}; \delta_1;\Sigma_1$, and we know that $\config{m, c_1,  t,w} \to^{*} \config{m', \eskip, t', w'} $, from our assumption that $ m =\delta^{-1}(\ssa{m})$, we know that 
\[\config{\ssa{ { m}, c_1},  t,w} \to^{*} \config{ \ssa{ { m_1 }, \eskip,} t', w' } \land m' = \delta_1^{-1}(\ssa{m_1}) \]
and we then have:
\[
\inferrule
{
  \config{\ssa{ { m}, c_1},  t,w} \to^{*} \config{ \ssa{ { m_1 }, \eskip,} t', w' }
}
{
 \config{\ssa{  m, c_1;} \eifvar(\ssa{\bar{x}', \bar{x_1})};\eifvar(\ssa{\bar{y}', \bar{y_1})};\eifvar(\ssa{\bar{z}', \bar{z_1})},  t,w  }  \to^{*}
 \config{\ssa{ { m_1 [ \bar{x}' \mapsto {m_1}(\bar{x_1}),\bar{y}' \mapsto {m_1}(\bar{y_2}),\bar{z}' \mapsto {m_1}(\bar{z_1}) ] }, \eskip}, t', w'  }
}
\]
Now, we want to show that $ m' = \delta[\bar{x} \mapsto \ssa{\bar{x}'},\bar{y} \mapsto \ssa{\bar{y}'},\bar{z} \mapsto \ssa{\bar{z}'} ]^{-1}(\ssa{ m_1 [ \bar{x}' \mapsto {m_1}(\bar{x_1}),\bar{y}' \mapsto {m_1}(\bar{y_2}),\bar{z}' \mapsto {m_1}(\bar{z_1}) ] }) $.

Unfold the definition, we want to show that $$\forall x  \in ( \dom(\delta) \cup \bar{x} \cup \bar{y} \cup \bar{z} ), (\delta[\bar{x} \mapsto \ssa{\bar{x}'},\bar{y} \mapsto \ssa{\bar{y}'},\bar{z} \mapsto \ssa{\bar{z}'} ](x), m'(x)) \in \ssa{m_1 [ \bar{x}' \mapsto {m_1}(\bar{x_1}),\bar{y}' \mapsto {m_1}(\bar{y_2}),\bar{z}' \mapsto {m_1}(\bar{z_1}) ] } .$$
\begin{enumerate}
    \item For variable $x$ in $\bar{x}$, we can find a corresponding ssa variable $\ssa{x} \in \ssa{\bar{x}'}$, so that $( \ssa{x}, m'(x) ) \in \ssa{ m_1 [\bar{x}' \mapsto m_1(\bar{x_1})] } $. It is because we know $[x \mapsto \ssa{x_1}]$ for certain $\ssa{x_1} \in \ssa{\bar{x_1}}$ in $\delta_1$, then by unfolding  $m' = \delta_1^{-1}(\ssa{m_1})$ and $\ssa{\bar{x_1}} \in \codom(\delta_1)$, we know $(\ssa{x_1}, m'(x)) \in \ssa{m_1}$ so that $m'(x) = \ssa{m_1}(\ssa{x_1})$.
    \item For variable $y \in \bar{y}$, we know that $y \in \dom(\delta_1)$, then $[ y \mapsto \ssa{y_2} ]$ for certain $\ssa{y_2} \in \ssa{\bar{y_2}}$ in $\delta_1$.  So we know that $(\delta_1(y), m'(y) ) \in \ssa{m_1}$, and then $m'(y) = \ssa{m_1(y_2)}$. We can show $(\ssa{y}, m'(y)) \in \ssa{m_1[\bar{y}' \mapsto m_1(\bar{y_2})]}$.
    \item For variable $z \in \bar{z}$, we know that $z \not\in \dom(\delta_1)$ by the definition (otherwise $z$ will appear in $\bar{x}$), then $[ z \mapsto \ssa{z_1} ]$ for certain $ \ssa{z_1} \in \ssa{\bar{z_1}}$ in $\delta$.  We know $(\delta(z), m(z)) \in \ssa{m}$ from our assumption, so we have $ m(z) = \ssa{m(z_1)}$. Because $z$ is not modified in $c_1$, so that $m(z) = m'(z)$. Also $\ssa{m}$ will not shrink during execution and $\ssa{z_1}$ will not be written in $\ssa{c_1}$, so $(\ssa{z_1}, m'(z)) \in \ssa{m_1}$. Then we can show that $ (\ssa{z}, m'(z) ) \in \ssa{m_1[\bar{z}' \mapsto m_1(\bar{z_1})] }$.
    \item For variable $k \in \dom(\delta)- \bar{x} - \bar{y}-\bar{z}$. From our assumption $ m = \delta^{-1}(\ssa{m})$, we can show $(\delta(k), m(k) ) \in \ssa{m}$. We know that $k$ is not written in either branch from our definition, so $(\delta(k), m'(k) ) \in \ssa{m_1} $ .
\end{enumerate}

{
\caseL{
	Case
	$
	\inferrule{
    { \Sigma; \delta ; c \hookrightarrow \ssa{c_1} ; \delta_1; \Sigma_1 }
     \and
    { [ \bar{x}, \ssa{\bar{{x_1}}}, \ssa{\bar{{x_2}}} ] = \delta \bowtie \delta_1 }
    \\
     {\ssa{\bar{{x}}'} \ fresh(\Sigma_1 )}
    \and {\delta' = \delta[\bar{x} \mapsto \ssa{\bar{{x}}'}]}
    \and 
     {\delta' ; \bexpr \hookrightarrow \ssa{\bexpr} }
     \and
    {\ssa{c' = c_1[\bar{x}'/ \bar{x_1}]   } }
    % \and{ \delta' ; c \hookrightarrow \ssa{c'} ; \delta'' }
  }{ 
  \Sigma; \delta ;  \ewhile ~ [\bexpr]^{l} ~ \edo ~ c 
  \hookrightarrow 
  \ssa{\ewhile ~ [\bexpr]^{l}, 0, [\bar{{x}}', \bar{{x_1}}, \bar{{x_2}}] ~ \edo ~ {c} } ; \delta'; \Sigma_1 \cup \{\ssa{\bar{x}'}  \}
}~{\textbf{S-WHILE}}
$
}
}
\\
{
We choose an arbitrary memory $m$ so that $m \vDash \delta$, we choose a trace $t$ and a loop maps $w$. Suppose $ \config{m ,a} \to v_N $. There are two cases, when $v_N=0$, the loop body is not executed so we can easily show that the trace is not modified.
%
When the while loop is still running ($v_N > 0$), we have the following evaluation in the while language:
\[
\inferrule
{
 \empty
}
{
\config{
m, \ewhile ~ [b]^{l} ~ \edo [c]^{l + 1},  t, w 
}
\xrightarrow{} \config{m, c ; 
\eif_w (b, c ; 
\ewhile ~ [b]^{l} ~ \edo [c]^{l + 1},  \eskip),
t, w }
}
~\textbf{while-b}
\]
which follows by the following evaluation:
\[
	\inferrule
{
 m, b \xrightarrow{} b'
}
{
\config{m, \eif_w (b, c,  \eskip) ,  t, w }
\xrightarrow{} \config{m, 
 \eif_w (b', c,  \eskip), t, w }
}
~\textbf{ifw-b}
\]
In the corresponding ssa-form language, we have the corresponding evaluation in the same way by assuming 
$m = \delta^{-1}(\ssa{m})$.
%
\[
	\inferrule
{
 {n = 0 \rightarrow i = 1 }
 \and
 {n > 0 \rightarrow i = 2 }
}
{
\config{
\ssa{m},  
\ssa{\ewhile ~ [\bexpr]^{l}, n, 
[\bar{{x}}', \bar{{x_1}}, \bar{{x_2}}] 
~ \edo ~ {c} 
},  t, w 
}
\xrightarrow{} \\ 
\config{
\ssa{m},
\eif_w 
(\ssa{b[\bar{x_i}/\bar{x'}], [\bar{{x}}', \bar{{x_1}}, \bar{{x_2}}], n,  c[\bar{x_i}/\bar{x'}] }; 
\ssa{
\ewhile ~ [b]^{l}, n+1, 
[\bar{{x}}', \bar{{x_1}}, \bar{{x_2}}]  
~ \edo ~ c} ,  \eskip),
t, w
}
}
~\textbf{ssa-while-b}
\]
This evaluation is followed by the following evaluation:
\[
	\inferrule
{
 \ssa{m, b \xrightarrow{} b'}
}
{
\config{\ssa{m, \eif_w (b, [\bar{{x}}', \bar{{x_1}}, \bar{{x_2}}] , n,  c_1,  c_2)} ,  t, w }
\xrightarrow{} \config{\ssa{ m, 
 \eif_w (b', [\bar{{x}}', \bar{{x_1}}, \bar{{x_2}}] , n , c_1 , c_2 )}, t, w }
}
~\textbf{ssa-ifw-b}
\]
%
Depending on if the counter $n$ is equal to $0$ or not, there are two possible execution paths (the variables $\ssa{\bar{x}}$ is replaced by the $\ssa{\bar{x_1}}$ or $\ssa{\bar{x_2}}$). We start from the first iteration (when $n =0$) when $v_N >0$. 
}
{
By induction hypothsis on the premise $ { \Sigma; \delta ; c \hookrightarrow \ssa{c_1} ; \delta_1; \Sigma_1 }$, we know that 
\[ \config{\ssa{{m}, c'[ \bar{x_1}/\bar{x}'  ]}, t, (w+l)  } \to^{*} \config{\ssa{{m'}, \eskip}, t'_{i}, w'  } \land m' = \delta_1^{-1}(\ssa{m'})   \]
Hence we can conclude that:
\[
  \inferrule{
   \config{\ssa{{m}, c'[ \bar{x_1}/\bar{x}'  ]}, t, (w+l) }  \to^{*} \config{\ssa{{m'}, \eskip}, t'_{1}, w'  }
  }{
  \config{\ssa{ {m}, c'[ \bar{x_1}/\bar{x}'  ];  [\eloop ~ (\valr_N-1), n+1, [\bar{\ssa{x}}', \bar{\ssa{x_1}}, \bar{\ssa{x_2}}] ~  \edo ~ c' ]^{l} },  t, (w + l)  }  \to^{*} \\ \config{ \ssa{{m'}, [\eloop ~ (\valr_N-1), n+1, [\bar{\ssa{x}}', \bar{\ssa{x_1}}, \bar{\ssa{x_2}}] ~  \edo ~ c' ]^{l}}, t'_{1}, w'  } 
  }
\]
%
Then there are two cases, 
%
\begin{enumerate}
     \item  when guard in the $\eif_w$ expression evaluates to $\efalse$, the while loop terminates and exits.
     The execution in the while language is defined in the evaluation rule $\textbf{ifw-false}$ as follows.
     \[
		\inferrule
		{
		 \empty
		}
		{
		\config{\ssa{
		m, \eif_w (
		\efalse, [\bar{{x}}', \bar{{x_1}}, \bar{{x_2}}],   n, 
		c; {\ewhile ~ [b]^{l} ~ \edo ~ c},
		\eskip)
		)} ,  t, w }
		\\
		\xrightarrow{} 
		\config{\ssa{m, 
		{\eskip}; \eifvar(\bar{x'}, \bar{x_i}) }, t, (w - l) }
		}
		~\textbf{ifw-false}
	\]
%
	The corresponding ssa-form evaluation as follows:
	\[
		\inferrule
		{
		 { n = 0 \rightarrow i = 1 }
		 \and
		 {n > 0 \rightarrow i =2}
		}
		{
		\config{\ssa{
		m, \eif_w (
		\efalse, [\bar{{x}}', \bar{{x_1}}, \bar{{x_2}}],   n, 
		{  
		c; \ssa{\ewhile ~ [b]^{l}, n, [\bar{{x}}', \bar{{x_1}}, \bar{{x_2}}]  ~ \edo ~ c},
		\eskip)
		} 
		)} ,  t, w }
		\\
		\xrightarrow{} 
		\config{\ssa{m, 
		{\eskip}; \eifvar(\bar{x'}, \bar{x_i}) }, t, (w - l) }
		}
		~\textbf{ssa-ifw-false}
	\]
	We can see that both traces are not changed during the exit of the while. We need to show that $ m' = \delta^{-1} (\ssa{m'[\bar{x} \mapsto m'(\bar{x_2})]}) $. We know that $[ \bar{x} \mapsto \bar{x_2}]$ in $\delta_1$ from the definition, so we can show that for any variable $\ssa{x_2} \in \bar{x_2}$, $( \ssa{x_2}, m'(x) ) \in \ssa{m'}$. For variables $x \in {\dom(\delta) - \bar{x} } $, the variable is not modified during the execution of $c$ so that we know $m(x) = m'(x)$, and then we can show that $(\delta(x), m'(x)) \in \ssa{m'} $ because $\delta(x)$ is not written in $\ssa{c'[\bar{x_1}/ \bar{x}']}$ .
%
  	\item 
		when guard in the $\eif_w$ expression evaluates to $\etrue$, the while terminates and exits.
     The execution in the while language is defined in the evaluation rule $\textbf{ifw-true}$.
          %
     We want to show that : assuming in the $i-th$ $(i < \ssa{n})$ iteration, starting with $t_i$ and $w_i$ and $m_i = \delta_1^{-1}(\ssa{m_i})$,
     this command is evaluated according to the while language operation semantics as
     	$
		\config{m, \eif_w (\etrue, c ; \ewhile ~ [b]^{l} ~ \edo ~ c, ,  \eskip) ,  t, w }
		\xrightarrow{}^* \config{m, c 
		t, (w + l) }
 		$.
     %
     Then the corresponding ssa form evaluation as follows : 
     %
     \[ 
     \inferrule{}{
     	\config{
		\ssa{
			m, 
			{
			\eif_w (\etrue, [\bar{{x}}', \bar{{x_1}}, \bar{{x_2}}], n,  
			c; \ssa{\ewhile ~ [b]^{l}, n, [\bar{{x}}', \bar{{x_1}}, \bar{{x_2}}]  ~ \edo ~ c},
			\eskip)
			} 
		},  t, w 
		}
		\\
		\xrightarrow{} 
		\config{
		\ssa{m, 
		{
		\eif_w (\etrue, [\bar{{x}}', \bar{{x_1}}, \bar{{x_2}}], n,  
		c; \ssa{\ewhile ~ [b]^{l}, n, [\bar{{x}}', \bar{{x_1}}, \bar{{x_2}}]  ~ \edo ~ c},
		}
		}
		t, (w + l) }
		} 
     \]  
     and $m_i = \delta^{-1}(\ssa{m_i}) $.
     We then have the evaluation in the while language:
     \[
		\inferrule
		{
		 \empty
		}
		{
		\config{m, 
		\eif_w (b, 
		c ; \ewhile ~ [b]^{l} ~ \edo ~ c, 
		\eskip),
		t, w }
		\xrightarrow{} 
		\config{m, 
		c ; \ewhile ~ [b]^{l} ~ \edo ~ c,  
		t, (w + l) }
		}
		~\textbf{ifw-true}
	\]
	We then have the following evaluation:
	\[
		\inferrule
		{
		 \empty
		}
		{
		\config{
		\ssa{
		m, 
		{
		\eif_w (\etrue, [\bar{{x}}', \bar{{x_1}}, \bar{{x_2}}], n,  
		c; \ssa{\ewhile ~ [b]^{l}, n, [\bar{{x}}', \bar{{x_1}}, \bar{{x_2}}]  ~ \edo ~ c},
		\eskip)
		} 
		},  t, w 
		}
		\\
		\xrightarrow{} 
		\config{
		\ssa{m, 
		{
		\eif_w (\etrue, [\bar{{x}}', \bar{{x_1}}, \bar{{x_2}}], n,  
		c; \ssa{\ewhile ~ [b]^{l}, n, [\bar{{x}}', \bar{{x_1}}, \bar{{x_2}}]  ~ \edo ~ c},
		}
		}
		t, (w + l) }
		}
		~\textbf{ssa-ifw-true}
	\]
%
By induction hypothsis on the premise $  { \Sigma; \delta_1 ; c \hookrightarrow \ssa{c_2} ; \delta_1; \Sigma_1 }$, we know that
%
\[
\config{\ssa{{m_i}, c'[ \bar{x_2}/\bar{x}'  ]}, t_i, (w_i+l)  } \to^{*} \config{\ssa{{m_{i+1}}, \eskip}, t_{i+1}, w_{i+1}  } \land m_{i+1} = \delta_1^{-1}(\ssa{m_{i+1}})
\]
%
Hence we can conclude that:
\[
  \inferrule{
   \config{\ssa{{m_i}, c'[ \bar{x_2}/\bar{x}'  ]}, t_i, (w_i+l) }  \to^{*} \config{\ssa{{m_{i+1}}, \eskip}, t_{i+1}, w_{i+1}  }
  }{
  \config{\ssa{ {m_i}, c'[ \bar{x_2}/\bar{x}'  ];  [\eloop ~ (\valr_N-i-1), n+1, [\bar{\ssa{x}}', \bar{\ssa{x_1}}, \bar{\ssa{x_2}}] ~  \edo ~ c' ]^{l} },  t_i, (w_i + l)  }  \to^{*} \\ \config{ \ssa{{m_{i+1}}, [\eloop ~ (\valr_N-i-1), n+1, [\bar{\ssa{x}}', \bar{\ssa{x_1}}, \bar{\ssa{x_2}}] ~  \edo ~ c' ]^{l}}, t_{i+1}, w_{i+1}  } 
  }
\]
So we can show that before the exit of the loop after ($v_N= n $) iterations, we have $t_{n} = t_{n}$ and $m_{n} = \delta_1^{-1}(\ssa{m_{n}})$.
 \end{enumerate}
%
This proof is similar when it comes to the exit as in case 1. 
}
\end{itemize}
%
\end{proof}
%
\clearpage
%
\clearpage
\clearpage
% %
\section{Dependency and Adapativity}
%
%
%
\subsection{Events Dependency}
%
\begin{defn}[Variable May Dependency].
  \label{def:var_dep}
  \\
  An variable ${x}_2^{l_2} \in \lvar_{{c}}$ is in the \emph{variable may-dependency} relation with another
  variable ${x}_1^{l_1} \in \lvar_{{c}}$ in a program ${c}$, denoted as 
  %
  $\vardep({x}_1^{l_1}, {x}_2^{l_2}, {c})$, if an only if.
  %
  \[
    \begin{array}{l}
  \exists \event_1, \event_2 \in \eventset^{\asn}, D \in \dbdom. ~
  (\pi_{1}{(\event_1)}, \pi_{2}{(\event_1)}) = ({x}_1, l_1)
  \land
  (\pi_{1}{(\event_2)}, \pi_{2}{(\event_2)}) = ({x}_2, l_2)
  \\ \quad 
  \land 
 \left(
  \exists \trace \in \mathcal{T} \st 
  \eventdep^{val}(\event_1, \event_2, \trace, c, D) 
  \lor
  \left( \exists \event_b \in \eventset^{\test} \st \eventdep^{val}(\event_1, \event_b, \trace, c, D) \land \eventdep^{\ctl}(\event_b, \event_2, c, D)  \right)  
 \right)
    \end{array}
  \]
  %
  , where $\eventdep^{val}$ and $\eventdep^{\test}$ is defined in \ref{def:event_valdep} and \ref{def:event_ctldep}.
  %
  %
  \end{defn}
  %
  explanation: Among all events corresponding to the evaluations of the assignment commands associated to the two labelled variables respectively, 
as long as there is one pair of events satisfying the relation, then we say they have \emph{Variable May-Dependency} relation.
% \begin{defn}
% [Value Dependency of Events \todo{Explicit Dependency}]
% \label{def:event_valdep}.
% \\
% An event $\event_2$ is in the \emph{value may-dependency} relation with an assignment
% event $\event_1 \in \eventset^{\asn}$ in a program ${c}$
% with hidden database $D$, denoted as 
% %
% $\eventdep^{val}(\event_1, \event_2, c, D)$, if and only if
% %
% \[
% \exists \vtrace_0,
% \vtrace_1, \vtrace_2, \vtrace_2' \in \mathcal{T}, \event_2' \in \eventset, \event_1' \in \eventset^{\asn}, {c}_1, {c}_2,  {c}_2'.
%   \left(
%   \begin{array}{ll}   
%  & \config{{c}, \vtrace_0} \rightarrow^{*} 
% \config{{c}_1, \vtrace_1 \cdot \event_1}  \rightarrow^{*} 
%   \config{{c}_2,  \vtrace_1 \cdot \event_1 \cdot \vtrace_2 \cdot \event_2 } 
%   % 
%  \\ 
%  \bigwedge &
%   \config{{c}_1, \vtrace_1 \cdot \event_1'}  \rightarrow^{*} 
%   \config{{c}_2',  \vtrace_1 \cdot \event_1'  \cdot \vtrace_2' \cdot \event_2' } 
% \\
% \bigwedge & \event_1 \sigeq \event_1' \land \diff(\event_2,\event_2')
% \end{array}
% \right)
%  \] 
%  % \wq{$\forall \vtrace_0$? is there any requirement of $c_2$ and $c_2'$? For example, you go 2 steps to get $\event_2$ while $\event_2'$ can be empty if it just goes 0 steps. Shall we set some requirement on $\event_2'$, for example, both goes to the end of $c_2$ or certain line?}
% %
% \end{defn}
%
\\
\jl{Better to Combine these two  \emph{may-dependency} definitions}
\begin{defn}
[Value Dependency of Events \todo{Explicit Dependency}]
\label{def:event_valdep}.
\\
An event $\event_2$ is in the \emph{value may-dependency} relation with an assignment
event $\event_1 \in \eventset^{\asn}$ in a program ${c}$
with hidden database $D$ and a trace $\trace \in \mathcal{T}$ denoted as 
%
$\eventdep^{val}(\event_1, \event_2, [\event_1 ] \tracecat \trace \tracecat [\event_2], c, D)$, if and only if
%
\[
\exists \vtrace_0,
\vtrace_1, \vtrace' \in \mathcal{T}, \event_2' \in \eventset, \event_1' \in \eventset^{\asn}, {c}_1, {c}_2  \in \cdom  \st
  \left(
  \begin{array}{ll}   
 & \config{{c}, \vtrace_0} \rightarrow^{*} 
\config{{c}_1, \vtrace_1 \tracecat [\event_1]}  \rightarrow^{*} 
  \config{{c}_2,  \vtrace_1 \tracecat [\event_1] \tracecat \vtrace \tracecat [\event_2] } 
  % 
 \\ 
 \bigwedge &
  \config{{c}_1, \vtrace_1 \tracecat [\event_1']}  \rightarrow^{*} 
  \config{{c}_2,  \vtrace_1 \tracecat[ \event_1'] \tracecat \vtrace' \tracecat [\event_2'] } 
\\
\bigwedge &  \pi_1(\event_1) = \pi_1(\event_1') \land \pi_2(\event_1) = \pi_2(\event_1') 
\\
\bigwedge & 
\diff(\event_2,\event_2' ) \land 
\vcounter(\vtrace) ~ \pi_2(\event_2)
= 
\vcounter(\vtrace') ~ \pi_2(\event_2)\\
\end{array}
\right)
 \]
%  \wq{I realize Diff is little bit unclear, it means same variable, label, but only different in value:- Maybe DiffValue?}
%
\end{defn}
% \begin{defn}
% [Testing Dependency of Events]
% \label{def:event_testdep}.
% \\
% One event $\event_2$ may have a testing dependency on a testing event $\event_1 = ({b}_1, l_1, n_1, v_1)$
% in a program ${c}$, with a hidden database $D$, 
% denoted as 
% %
% $\eventdep^{\ctl}(\event_1, \event_2, c, D)$, is defined as follows: 
% %
% \[
% \exists \event_1' \in \eventset^{\test}, \vtrace_0,
% \vtrace_1, \vtrace_2, \vtrace_2', {c}_1.
%   \left(
%   \begin{array}{ll}   
%   & \config{{c}, \vtrace_0} \rightarrow^{*} 
%     \config{{c}_1, \vtrace_1 \cdot \event_1}  \rightarrow^{*} 
%     \config{\eskip,  \vtrace_1 \cdot \event_1 \vtrace_2} 
%   \\ 
%   \bigwedge &
%   \config{{c}_1, \vtrace_1 \cdot \event_1'}  \rightarrow^{*} 
%   \config{\eskip,  \vtrace_1 \cdot \event_1 \vtrace_2' } 
%   \\
%   \bigwedge &
%   \event_2 \sigin \vtrace_2 \land \event_2 \notsigin \vtrace_2'
% \end{array}
% \right)
%  \]
% %
% \end{defn}
%
% \begin{defn}
% [Control Dependency of Events \todo{Implicit Dependency}]
% \label{def:event_ctldep}.
% \\
% An event $\event_2$ is in the \emph{control may-dependency} relation with an assignment
% event $\event_1 \in \eventset^{\asn}$ in a program ${c}$
% with hidden database $D$, denoted as 
% %
% $\eventdep^{\ctl}(\event_1, \event_2, c, D)$, if and only if: 
% %
% \[
% \exists \vtrace_1, \vtrace_2, \vtrace_2', \vtrace_0 \in \mathcal{T}, 
% \event_1' \in \eventset^{\asn}, {c}_1.
% \left(
% \begin{array}{ll}   
%   & \config{{c}, \vtrace_0} \rightarrow^{*} 
%     \config{{c}_1, \vtrace_1 \cdot \event_1}  \rightarrow^{*} 
%     \config{\eskip,  \vtrace_1 \cdot \event_1 \cdot \vtrace_2} 
%   \\ 
%   \bigwedge &
%   \config{{c}_1, \vtrace_1 \cdot \event_1'}  \rightarrow^{*} 
%   \config{\eskip,  \vtrace_1 \cdot \event_1' \cdot \vtrace_2' } 
%   \\
%   \bigwedge & \event_1 \sigeq \event_1' \land 
%   \event_2 \sigin \vtrace_2 \land \event_2 \notsigin \vtrace_2'
% \end{array}
% \right)
%  \]
% %
% \end{defn}
%
\begin{defn}
[Control Dependency of Events \todo{Implicit Dependency}]
\label{def:event_ctldep}.
\\
An event $\event$ is in the \emph{control may-dependency} relation with an assignment
event $\event_b \in \eventset^{\test}$ in a program ${c}$
with hidden database $D$, denoted as 
%
% doesn't rely on trace
% $\eventdep^{\ctl}(\event_1, \event_2, \wq{\tau,} c, D)$, if and only if: 
%
% \[
% \exists \vtrace_1, \vtrace_2, \vtrace_2', \vtrace_0 \in \mathcal{T}, 
% \event_1' \in \eventset^{\asn}, {c}_1.
% \left(
% \begin{array}{ll}   
%   & \config{{c}, \vtrace_0} \rightarrow^{*} 
%     \config{{c}_1, \vtrace_1 \cdot \event_1}  \rightarrow^{*} 
%     \config{\eskip,  \vtrace_1 \cdot \event_1 \tracecat \vtrace_2} 
%   \\ 
%   \bigwedge &
%   \config{{c}_1, \vtrace_1 \cdot \event_1'}  \rightarrow^{*} 
%   \config{\eskip,  \vtrace_1 \cdot \event_1' \tracecat \vtrace_2' } 
%   \\
%   \bigwedge &  \pi_1(\event_1) = \pi_1(\event_1') \land \pi_2(\event_1) = \pi_2(\event_1') \\
%   \bigwedge & l_1 = \pi_2(\event_1) \land l_2 = \pi_2(\event_2)
%   \\
%   \bigwedge &  \land \vcounter(\vtrace_2') l_2 \neq \vcounter(\trace_2) l_2 
% \end{array}
% \right)
%  \]%
%
\\
$\eventdep^{\ctl}(\event_b, \event, c, D)$, if and only if: 
%
% \[
% \exists \vtrace_1, \vtrace_2', \vtrace_0 \in \mathcal{T}, 
% \event_1, \event_1' \in \eventset^{\asn},  {c}_1.
% \left(
% \begin{array}{ll}   
%   & \config{{c}, \vtrace_0} \rightarrow^{*} 
%     \config{{c}_1, \vtrace_1 \cdot \event_1}  \rightarrow^{*} 
%     \config{c_2,  \vtrace_1 \cdot \event_1 \tracecat \trace_2 \cdot \event_b \cdot \trace_3} 
%   \\ 
%   \bigwedge &
%   \config{{c}_1, \vtrace_1 \cdot \event_1'}  \rightarrow^{*} 
%   \config{c_2,  \vtrace_1 \cdot \event_1 {\wq{\event_1'?}} \tracecat \trace_2' \cdot (\neg \event_b)\wq{+ \cdot \event_3' ?}} 
%   \\
%   \bigwedge &  \diff(\event_1, \event_1') \\ 
%   \bigwedge & l_b = \pi_2(\event_b) \land \vcounter(\vtrace_2') l_b = \vcounter(\trace_2) l_b
%     \land \event \in \trace_3
% \end{array}
% \right)
%  \]
%  %
%  \[
% \exists \vtrace_0, \vtrace_1, \vtrace_2, \trace_2', \vtrace_3 \in \mathcal{T}, 
% \event_1, \event_1' \in \eventset^{\asn},  {c}_1, c_2 \in \cdom.
% \left(
% \begin{array}{ll}   
%   & \config{{c}, \vtrace_0} \rightarrow^{*} 
%     \config{{c}_1, \vtrace_1 \cdot \event_1}  \rightarrow^{*} 
%     \config{c_2,  \vtrace_1 \cdot \event_1 \tracecat \trace_2 \cdot \event_b \tracecat  \trace_3} 
%   \\ 
%   \bigwedge &
%   \config{{c}_1, \vtrace_1 \cdot \event_1'}  \rightarrow^{*} 
%   \config{c_2,  \vtrace_1 \cdot \event_1' \tracecat \trace_2' \cdot (\neg \event_b)} 
%   \\
%   \bigwedge &  \diff(\event_1, \event_1') \\ 
%   \bigwedge & l_b = \pi_2(\event_b) \land \vcounter(\vtrace_2') l_b = \vcounter(\trace_2) l_b
%     \land \event \in \trace_3
% \end{array}
% \right)
%  \]
 %
 \[
\exists \vtrace_0, \vtrace_1, \vtrace_2, \trace_2', \vtrace_3, \vtrace_3'  \in \mathcal{T}, 
\event_1, \event_1' \in \eventset^{\asn},  {c}_1, c_2 \in \cdom.
\left(
\begin{array}{ll}   
  & \config{{c}, \vtrace_0} \rightarrow^{*} 
    \config{{c}_1, \vtrace_1 \tracecat [\event_1]}  \rightarrow^{*} 
    \config{c_2,  \vtrace_1 \tracecat [\event_1] \tracecat \trace_2 \tracecat [\event_b] \tracecat  \trace_3} 
  \\ 
  \bigwedge &
  \config{{c}_1, \vtrace_1 \tracecat [\event_1']}  \rightarrow^{*} 
  \config{c_2,  \vtrace_1 \tracecat [\event_1'] \tracecat \trace_2' \tracecat [(\neg \event_b)] \tracecat \trace_3'} 
  \\
  \bigwedge &  \diff(\event_1, \event_1') \land \tlabel_{\trace_3} \cap \tlabel_{\trace_3'} = \emptyset\\ 
  \bigwedge & l_b = \pi_2(\event_b) \land \vcounter(\vtrace_2') l_b = \vcounter(\trace_2) l_b
    \land \event \in \trace_3
\end{array}
\right)
 \]
 %
\end{defn}
%
% \begin{defn}[Event May Dependency].
% \label{def:event_dep}
% \\ 
% An event $\event_2 \in \eventset^{\asn}$ is in the \emph{may-dependency} relation with another
% event $\event_1 \in \eventset^{\asn}$ in a program ${c}$ with a hidden database $D$, denoted as $\eventdep(\event_1, \event_2, c, D)$,
% if and only if
% \[
% \exists \trace \in \mathcal{T} \st 
% \eventdep^{val}(\event_1, \event_2, \trace, c, D) 
% \lor
% \left( \exists \event_b \in \eventset^{\test} \st \eventdep^{val}(\event_1, \event_b, \trace, c, D) \land \eventdep^{\ctl}(\event_b, \event_2, c, D)  \right)
% \] %
% %
% \end{defn}
%

%
\begin{defn}[labelled Variables ($\lvar_{c} \subseteq \mathcal{VAR} \times \mathbb{N}$ or 
$\lvar : c \to \mathcal{P}(\mathcal{VAR} \times \mathbb{N})$]
$$
  \lvar_{c} \triangleq
  \left\{
  \begin{array}{ll}
      \{{x}^l\}                   
      & {c} = [{\assign x e}]^{l} 
      \\
      \{{x}^l\}                   
      & {c} = [{\assign x \query(\qexpr)}]^{l} 
      \\
      \lvar_{{c_1}} \cup \lvar_{{c_2}}  
      & {c} = {c_1};{c_2}
      \\
      \lvar_{{c}} \cup \lvar_{{c_2}} 
      & {c} =\eif([\bexpr]^{l}, c_1, c_2) 
      \\
      \lvar_{{c}'}
      & {c}   = \ewhile ([\bexpr]^{l}, {c}')
\end{array}
\right.
$$
\end{defn}
%
\begin{defn}[Query Variables ($\qvar_{c} \subseteq \mathcal{VAR} \times \mathbb{N}$)].
\\
Given a program $c$, its query variables $\qvar$ is a vector containing all variables newly assigned by a query in the programm, $\qvar \subset \mathcal{VAR}$.
It is defined as follows:
$$
  \qvar_{{c}} \triangleq
  \left\{
  \begin{array}{ll}
      \{\}                  
      & {c} = [{\assign x e}]^{(l, w)} 
      \\
      \{{x}^l\}                  
      & {c} = [{\assign x \query(\qexpr)}]^{(l, w)} 
      \\
      \qvar_{{c_1}} \cup \qvar_{{c_2}}  
      & {c} = {c_1};{c_2}
      \\
      \qvar_{{c_1}} \cup \qvar_{{c_2}} 
      & {c} =\eif([\bexpr]^{l}, c_1, c_2) 
      \\
      \qvar_{{c}'}
      & {c}   = \ewhile ([\bexpr]^{l}, {c}')
\end{array}
\right.
$$
\end{defn}
%
%
\begin{defn}[Execution Based Dependency Graph].
\\
Given a program ${c}$ with its assigned variables $\lvar_c$ 
the dependency graph $\traceG({c}, D) = (\vertxs, \edges, \weights, \qflag)$ is defined as:
%
\[
\begin{array}{rlcl}
  \text{Vertices} &
  \vertxs & := & \left\{ 
  x^l \in \mathcal{VAR} \times \mathbb{N} 
  ~ \middle\vert ~ x^l \in \lvar_{c}
  \right\}
  \\
  \text{Directed Edges} &
  \edges & := & 
  \left\{ 
  (x^i, y^j) \in (\mathcal{VAR} \times \mathbb{N}) \times (\mathcal{VAR} \times \mathbb{N})
  ~ \middle\vert ~
  x^i, y^j \in \lvar_{{c}}, \vardep(x^i, y^j, c)
  
  \right\}
  \\
  \text{Weights} &
  \weights & := & 
  \left\{ 
  (x^l, n) \in \mathcal{VAR} \times \mathbb{N} \times \mathbb{N}
  ~ \middle\vert ~ 
  n = \max \left\{ \vcounter(\vtrace') l ~ \middle\vert ~ x^l \in \lvar_{c},
  \forall \vtrace \in \mathcal{T} \st \config{{c}, \trace} \to^{*} \config{\eskip, \trace\cdot\vtrace'} 
   \right\}
  \right\}
  \\
  \text{Query Flags} &
  \qflag & := & 
  \left\{(x^l, n)  \in \mathcal{VAR} \times \mathbb{N}  \times \{0, 1\} 
  ~ \middle\vert ~
   x^l \in \lvar_{c},
   \left\{
  \begin{array}{ll}
  n = 1 & x^l \in \qvar_{c} \\ 
  n = 0 & o.w.
  \end{array}
  \right\}
  \right\}
\end{array}
\]
\end{defn}
%
%
\begin{defn}[Finite Walk ($k$)].
\label{def:finitewalk}
\\
Given a labelled weighted graph $G = (\vertxs, \edges, \weights, \qflag)$, a \emph{finite walk} $k$ in $G$ is a sequence of edges $(e_1 \ldots e_{n - 1})$ 
for which there is a sequence of vertices $(v_1, \ldots, v_{n})$ such that:
\begin{itemize}
    \item $e_i = (v_{i},v_{i + 1})$ for every $1 \leq i < n$.
    \item every vertex $v \in \vertxs$ appears in this vertices sequence $(v_1, \ldots, v_{n})$ of $k$ at most $W(v)$ times.  
\end{itemize}
$(v_1, \ldots, v_{n})$ is the vertex sequence of this walk.
\\
%
Length of this finite walk $k$ is the number of vertices in its vertex sequence, i.e., $\len(k) = n$.
\end{defn}
%
Given a labelled weighted graph $G = (\vertxs, \edges, \weights, \qflag)$, 
we use $\walks(G)$ to denote a set containing all finite walks $k$ in $G$;
and $k_{v_1 \to v_2} \in \walks(G)$where $v_1, v_2 \in \vertxs$ denotes the walk from vertex $v_1$ to $v_2$ .
%
%
\begin{defn}[Length of Finite Walk w.r.t. Query ($\qlen$)].
\label{def:qlen}
\\
Given a labelled weighted graph $G = (\vertxs, \edges, \weights, \qflag)$ and a \emph{finite walk} $k$ in $G$ with its vertex sequence $(v_1, \ldots, v_{n})$, the length of $k$ w.r.t query is defined as:
\[
  \qlen(k) = \len\big(
  v \mid v \in (v_1, \ldots, v_{n}) \land \flag(v) = 1 \big)
\]
, where $\big(v \mid v \in (v_1, \ldots, v_{n}) \land \flag(v) = 1 \big)$ is a subsequence of $k$'s vertex sequence.
\end{defn}
%
%
% \subsection{SSA Transformation and Soundness of Transformation}
% in File {\tt ``ssa\_transform\_sound.tex''}
% %
\subsection{SSA Transformation}
We use a translation environment $\delta$, to map variables $x$ in the {\tt While} language to those variables $\ssa{x}$ in the SSA language.
We use a name environment denoted as $\Sigma$ as a set of ssa variables, to get a fresh variable by $fresh(\Sigma)$. 
We define $\delta_1 \bowtie \delta_2 $ in a similar way as
\cite{vekris2016refinement}.
%
\[ 
\delta_1 \bowtie \delta_2 = \{ ( x, {\ssa{x_1}, \ssa{x_2}} ) \in 
\mathcal{VAR} \times \mathcal{SVAR} \times \mathcal{SVAR} \mid x \mapsto {\ssa{x_1}} \in \delta_1 , x \mapsto {\ssa{x_2} } \in \delta_2, {\ssa{x_1} \not= {\ssa{x_2} }  }  \} 
\]
%
\[ 
\delta_1 \bowtie \delta_2 / \bar{x} = \{ ( x, {\ssa{x_1}, \ssa{x_2}} ) \in 
\mathcal{VAR} \times \mathcal{SVAR} \times \mathcal{SVAR}
 \mid x \not\in \bar{x} \land x \mapsto {\ssa{x_1}} \in \delta_1 , x \mapsto {\ssa{x_2} } \in \delta_2, {\ssa{x_1} \not= {\ssa{x_2} }   }  \} 
 \]
 %
We call a list of variables $\bar{x}$.
\[
 [\bar{x}, \bar{\ssa{x_1}}, \bar{\ssa{x_2}}] = \{ (x, x_1,x_2)  | \forall 0 \leq i < |\bar{x}|, x = \bar{x }[i] \land x_1 = \bar{x_1}[i] \land x_2 = \bar{x_2 }[i] \land |\bar{x}| = |\bar{x_1}| = |\bar{x_2}|   \}
\]
%
\begin{mathpar}
\boxed{ \delta ; e \hookrightarrow \ssa{e} }
\and
\inferrule{
}{
 \delta ; x \hookrightarrow \delta(x)
}~{\textbf{S-VAR}}
\and
\boxed{ \Sigma; \delta ; c  \hookrightarrow \ssa{c} ; \delta' ; \Sigma' }
\and
\inferrule{
  { \delta ; \bexpr \hookrightarrow \ssa{\bexpr} }
  \and
  { \Sigma; \delta ; c_1 \hookrightarrow \ssa{c_1} ; \delta_1;\Sigma_1 }
  \and
  {\Sigma_1; \delta ; c_2 \hookrightarrow \ssa{c_2} ; \delta_2 ; \Sigma_2 }
  \\
  {[\bar{x}, \ssa{\bar{{x_1}}, \bar{{x_2}}}] = \delta_1 \bowtie \delta_2  }
  \and
   {[\bar{y}, \ssa{\bar{{y_1}}, \bar{{y_2}}}] = \delta \bowtie \delta_1 / \bar{x} }
  \and
   {[\bar{z}, \ssa{\bar{{z_1}}, \bar{{z_2}}}] = \delta \bowtie \delta_2 / \bar{x} }
  \\
  { \delta' =\delta[\bar{x} \mapsto \ssa{\bar{{x}}'} ][\bar{y} \mapsto \ssa{\bar{{y}}'} ][\bar{z} \mapsto \ssa{\bar{{z}}'} ]}
  \and 
  {\ssa{\bar{{x}}', \bar{y}', \bar{z}'} \ fresh(\Sigma_2)
  }
  \and{\Sigma' = \Sigma_2 \cup \{ \ssa{ \bar{x}', \bar{y}', \bar{z}' } \} }
}{
 \Sigma; \delta ; [\eif(\bexpr, c_1, c_2)]^l  \hookrightarrow [\ssa{ \eif(\bexpr, [\bar{{x}}', \bar{{x_1}}, \bar{{x_2}}] ,[\bar{{y}}', \bar{{y_1}}, \bar{{y_2}}] ,[\bar{{z}}', \bar{{z_1}}, \bar{{z_2}}] , {c_1}, {c_2})}]^l; \delta';\Sigma'
}~{\textbf{S-IF}}
%
\and
%
\inferrule{
 {\delta ; \expr \hookrightarrow \ssa{\expr} }
 \and
 {\delta' = \delta[x \mapsto \ssa{{x}} ]}
 \and{ \ssa{x} \ fresh(\Sigma) }
 \and { \Sigma' = \Sigma \cup \{ \ssa{x} \} }
}{
 \Sigma;\delta ; [\assign x \expr]^{l} \hookrightarrow [\ssa{\assign {{x}}{ \expr}}]^{l} ; \delta'; \Sigma'
}~{\textbf{S-ASSN}}
%
\and
%
\inferrule{
 {\delta ; \query \hookrightarrow \ssa{\query}}
 \and
 {\delta ; \qexpr \hookrightarrow \ssa{\qexpr}}
 \and
 {\delta' = \delta[x \mapsto \ssa{x} ]}
 \and{ \ssa{x} \ fresh(\Sigma) }
  \and { \Sigma' = \Sigma \cup \{ \ssa{x} \} }
}{
 \Sigma;\delta ; [\assign{x}{\query(\qexpr)}]^{l} \hookrightarrow 
 [\assign {\ssa{x}}{ \ssa{\query(\qexpr)}}]^{l} ; \delta';\Sigma'
}~{\textbf{S-QUERY}}
%
%%
\and
%
%
\and
%
\inferrule{
    { \Sigma; \delta ; c \hookrightarrow \ssa{c_1} ; \delta_1; \Sigma_1 }
     \and
    { [ \bar{x}, \ssa{\bar{{x_1}}}, \ssa{\bar{{x_2}}} ] = \delta \bowtie \delta_1 }
    \\
     {
     \ssa{\bar{{x}}'} \ fresh(\Sigma_1 )}
    \and {\delta' = \delta[\bar{x} \mapsto \ssa{\bar{{x}}'}]}
    \and 
     {\delta' ; \bexpr \hookrightarrow \ssa{\bexpr} }
     \and
    {\ssa{c' = c_1[\bar{x}'/ \bar{x_1}]   } }
  }{ 
  \Sigma; \delta ;  \ewhile ~ [\bexpr]^{l} ~ \edo ~ c 
  \hookrightarrow 
  \ssa{\ewhile ~ [\bexpr]^{l}, 0, [\bar{{x}}', \bar{{x_1}}, \bar{{x_2}}] ~ \edo ~ {c} } ; \delta'; \Sigma_1 \cup \{\ssa{\bar{x}'}  \}
}~{\textbf{S-WHILE}
}
\and
%
\inferrule{
 {\Sigma;\delta ; c_1 \hookrightarrow \ssa{c_1} ; \delta_1; \Sigma_1} 
 \and
 {\Sigma_1; \delta_1 ; c_2 \hookrightarrow \ssa{c_2} ; \delta'; \Sigma'} 
}{
\Sigma;\delta ; c_1 ; c_2 \hookrightarrow \ssa{c_1} ; \ssa{c_2} \ ; \delta';\Sigma'
}~{\textbf{S-SEQ}}
\end{mathpar}

\paragraph{Concrete examples.}
\[
c_1 \triangleq
\begin{array}{l}
     \left[x \leftarrow \query(1) \right]^1; \\
     \eif \; (x ==0)^{2} \; \\
    \ethen \; \left[y \leftarrow \query(2) \right]^3\; \\
    \eelse \; \left[y \leftarrow 0 \right]^4 ; \\
    \eif \; (x == 1 )^5\; \\
    \ethen \; \left[ y \leftarrow 0 \right]^6\; \\
    \eelse \; \left[y \leftarrow \query(2) \right]^7\\
\end{array}
%
%
\hspace{20pt} \hookrightarrow  \hspace{20pt}
%
\begin{array}{l}
     \left[ \ssa{x_1} \leftarrow \query(1) \right]^1; \\
     \eif \; (\ssa{x_1 ==0})^{2}, [\ssa{ y_3, y_1,y_2  }],[],[]  \; \\
    \ethen \; \left[ \ssa{y_1} \leftarrow \query(2) \right]^3\; \\
    \eelse \; \left[\ssa{y_2 \leftarrow 0 } \right]^4 ; \\
    \eif \; (\ssa{x_1 == 1} )^{5} , [\ssa{ y_6, y_4, y_5 } ] \; \\
    \ethen \; \left[ \ssa{y_4 \leftarrow 0} \right]^6\; \\
    \eelse \; \left[\ssa{y_5} \leftarrow \query(2) \right]^7\\
\end{array}
\]
\[
c_2 \triangleq
\begin{array}{l}
   \left[ x \leftarrow \query(1) \right]^1; \\
   \left[y \leftarrow \query(2) \right]^2 ; \\
    \eif \;( x + y == 5 )^3\; \\
    \ethen \;\left[ z \leftarrow \query(3)\right]^4 \; \\
    \eelse \;\left[ \ssa{\eskip}\right]^5 ; \\
   \left[ w \leftarrow q_4 \right]^6; \\
\end{array}
\hspace{20pt} \hookrightarrow \hspace{20pt}
%
\begin{array}{l}
   \left[ \ssa{ x_1 } \leftarrow \query(1) \right]^1; \\
   \left[\ssa{ y_1} \leftarrow \query(2) \right]^2 ; \\
    \eif \;( \ssa{ x_1 + y_1 == 5} )^3, [ ],[] ,[ ]\; \\
    \ethen \;\left[ \ssa{ z_1 }
    \leftarrow \query(3)\right]^4 \; \\
    \eelse \;\left[ \eskip\right]^5 ; \\
   \left[ \ssa{ w_1} \leftarrow \query(4) \right]^6; \\
\end{array}
\]

{
\[
c_3 \triangleq
\begin{array}{l}
     \left[x \leftarrow \query(1) \right]^1 ; \\
     \left[i \leftarrow 0 \right]^2 ; \\
    \ewhile ~  [i < 100]^3 ~ \edo
    \\
    ~ \Big( 
    \left[z \leftarrow \query(3) \right]^4; \\
    \left[x \leftarrow z + x \right]^5; \\
    \left[i \leftarrow i + 1 \right]^6
    \Big) ;
\end{array}
%
\hspace{20pt} \hookrightarrow \hspace{20pt} 
%
\begin{array}{l}
     \left[\ssa{x_1} \leftarrow \query(1) \right]^1 ; \\
     \left[\ssa{i_1} \leftarrow 0 \right]^2 ; \\
    \ewhile
    ~ [\ssa{i_1} < 100]^3, 0,
    ~\ssa{[ x_3,x_1 ,x_2 ], [i_3, i_1, i_2] }~
    \edo \\
    ~ \Big( 
    \left[\ssa{z_1} \leftarrow \query(3) \right]^4; \\
    \left[ \ssa{x_2} \leftarrow \ssa{z_1 + x_3} \right]^5; \\
    \left[\ssa{i_2} \leftarrow \ssa{i_3} + 1 \right]^6
    \Big) ;
\end{array}
\]
}
%
\begin{figure}
   \[
 \begin{array}{lll}
    | \ewhile ~ [ \sbexpr ]^{l}, n, [\bar{\ssa{x}}, \bar{\ssa{x_1}}, \bar{\ssa{x_2}}] 
    ~ \edo ~  \ssa{c}|  
    &=& \ewhile ~ [|\sbexpr|]^{l},  ~ \edo ~ |\ssa{c}| 
	\\
    |\ssa{c_1 ; c_2}|  &=& |\ssa{c_1}| ; |\ssa{c_2}| 
    \\
    |[\eif(\sbexpr,
    [ \bar{\ssa{x}}, \bar{\ssa{x_1}}, \bar{\ssa{x_2}}] ,
    [ \bar{\ssa{y}}, \bar{\ssa{y_1}}, \bar{\ssa{y_2}}] , 
    [\bar{\ssa{z}}, \bar{\ssa{z_1}}, \bar{\ssa{z_2}}] , 
    \ssa{ c_1, c_2)}]^{l}|  
    &=&
    [\eif(|\sbexpr|, |\ssa{ c_1}|, |\ssa{c_2}|)]^{l}
    \\
    | [\assign {\ssa{x}}{\ssa{\expr}}]^{l}| & = & [\assign {|\ssa{x}|}{|\ssa{\expr}|} ]^{l}
    \\
    | [\assign {\ssa{x}}{\query(\ssa{\qexpr})} ]^{l} | & = & [\assign {|\ssa{x}|}{|\query(\ssa{\qexpr})|}]^{l}
    \\
    |\ssa{x}_i| & = & x 
    \\
    |n | & = & n 
    \\
    | \saexpr_1 \oplus_{a} \saexpr_2 | & = &  |\ssa{\aexpr_1}| \oplus_a |\ssa{\aexpr_2}| \\
    | \sbexpr_1 \oplus_{b} \sbexpr_2 | & = &  |\sbexpr_1| \oplus_b |\sbexpr_2|
 \end{array}
\]
    \caption{The Erasure of SSA}
    \label{fig:ssa_erasure-while}
\end{figure}
%
%
%
% 
%
\subsection{The Soundness of the Transformation}
In this subsection, we show our transformation from the {\tt While} language to its SSA form is sound with respect to the adaptivity. 
To be specific, a transformed program $\ssa{c}$ starting with appropriate configuration, generates the same trace as the program before the transformation $c$, in its corresponding configuration.
%
%
\begin{defn}[\todo{Written Variables}].
\\
We defined the assigned variables in the while language program $c$ as $\avars{c}$,the assigned variables in the ssa-form program $\ssa{c}$ as $\avarssa{\ssa{c}}$ defined as follows.
\[
\begin{array}{lll}
    \avars{\assign{x}{\expr}} & =& \{ x \} \\
    \avars{\assign{x}{\query(\qexpr)}} & =& \{ x \} \\
    \avars{c_1; c_2}  & = & \avars{c_1} \cup \avars{c_2} \\
    \avars{\ewhile ~ \bexpr ~ \edo ~ c} &= &  \avars{c} \\
    \avars{\eif(\bexpr, c_1, c_2)} & =&  \avars{c_1} \cup \avars{c_2}\\
\end{array} 
\]
%
\[
\begin{array}{lll}
    \avarssa{\ssa{\assign{x}{\expr}}} & =& \{ \ssa{x} \}
    \\
    \avarssa{\ssa{\assign{x}{\query(\ssa{\qexpr})}}} & =& \{ \ssa{x} \}
    \\
    \avarssa{\ssa{c_1; c_2 } }  & = & \avarssa{\ssa{c}_1} \cup \avarssa{\ssa{c}_2}
    \\
    \avarssa{\ewhile ~ \ssa{\bexpr, n, [\bar{x}, \bar{x_1}, \bar{x_2}] ~ \edo ~ \ssa{c}}}
    & = &  
    \{\ssa{\bar{x}}\} \cup \avarssa{\ssa{c}} 
    \\
    \avarssa{\eif(\ssa{\bexpr,[\bar{x}, \bar{x_1}, \bar{x_2}],[\bar{y}, \bar{y_1}, \bar{y_2}],[\bar{z}, \bar{z_1}, \bar{z_2}], c_1, c_2} )} 
    & =&  \{ \ssa{\bar{x}},\ssa{\bar{y}} , \ssa{\bar{z}} \} 
    \cup \avarssa{\ssa{c_1}} \cup \avarssa{\ssa{c_2}}\\
\end{array}
\]
\end{defn}
\begin{defn}[\todo{Read Variables}].
\\
{
The variables read in the while language programs $c$ as $\vars{c}$, variables used in ssa-form program $\ssa{c}$ : 
}
\[
\begin{array}{lll}
    \vars{\assign{x}{\expr}} & =& \vars{\expr}  \\
    \vars{\assign{x}{\query(\qexpr)}} & =&\{  \} \\
    \vars{ c_1; c_2  }  & = & \vars{c_1} \cup \vars{c_2} \\
    \vars{  \eloop ~ \aexpr ~ \edo ~ c  } &= &\vars{\aexpr} \cup \vars{c} \\
    \vars{\eif(\bexpr, c_1, c_2)} & =& \vars{\bexpr} \cup \vars{c_1} \cup \vars{c_2}\\
\end{array}
\]
\[
\begin{array}{lll}
    \varssa{\ssa{\assign{x}{\expr}}} & =& \varssa{\ssa{\expr}}  \\
    \varssa{\ssa{\assign{x}{\query(\qexpr)}}} & =& \{  \} \\
    \varssa{ \ssa{c_1; c_2}  }  & = & \varssa{\ssa{c}_1} \cup \varssa{\ssa{c}_2} \\
    % \varssa{  \eloop ~ \ssa{\aexpr, n, [\bar{x}, \bar{x_1}, \bar{x_2}] ~ \edo ~ c} } &= &\varssa{\ssa{\aexpr}} \cup \varssa{\ssa{c}}  \cup \{ \ssa{\bar{x_1}} \} \cup \{ \ssa{\bar{x_2}} \}\\
    {\varssa{  \ewhile ~ \ssa{\bexpr, n, [\bar{x}, \bar{x_1}, \bar{x_2}] ~ \edo ~ c} }} 
    &= &
    \varssa{\ssa{\bexpr}} \cup \varssa{\ssa{c}}  \cup \{ \ssa{\bar{x_1}} \} \cup \{ \ssa{\bar{x_2}} \}\\
    \varssa{\eif(\ssa{\bexpr,[\bar{x}, \bar{x_1}, \bar{x_2}], [\bar{y}, \bar{y_1}, \bar{y_2}],[\bar{z}, \bar{z_1}, \bar{z_2}], c_1, c_2} )} & =& \varssa{\ssa{\bexpr}} \cup \varssa{\ssa{c_1}} \cup \varssa{\ssa{c_2}} \cup \{\ssa{\bar{x_1}, \bar{x_2},\bar{y_1}, \bar{y_2},\bar{z_1}, \bar{z_2} }\}  \\
\end{array}
\]
\end{defn}
%
\begin{defn}[\todo{Necessary Variables}].
\\
{
We call the variables needed to be assigned before executing the program $c$ as necessary variables $\fv{c}$. Its ssa form is : $\fvssa{\ssa{c}}$.
}  
 \[
 \begin{array}{lll}
     \fvars{\assign{x}{\expr} }  & = & \vars{\expr}  \\
     \fvars{\assign{x}{\query(\qexpr)} }  & = & \{ \}  \\
     {\fvars{  \ewhile ~ \bexpr ~ \edo ~ c  } }&= & \vars{\bexpr} \cup \fvars{c} \\
     \fvars{\eif(\bexpr, c_1, c_2)} & =& \vars{\bexpr} \cup \fvars{c_1} \cup \fvars{c_2}  \\
      \fvars{c_1 ; c_2} & = & \fvars{c_1} \cup ( \fvars{c_2} - \avars{c_1})
 \end{array}
 \]
 \[
 \begin{array}{lll}
     \fvssa{\ssa{\assign{x}{\expr}} }  & = & \varssa{\ssa{\expr}}  \\
     \fvssa{ \ssa{ \assign{x}{\query(\qexpr)}} }  & = & \{ \}  \\
     {\fvssa{  \ewhile ~ \ssa{\bexpr, n, [\bar{x}, \bar{x_1}, \bar{x_2}] ~ \edo ~ c} } }
     &= & 
     \varssa{\ssa{\bexpr}} \cup \fvssa{\ssa{c}}[\ssa{ \bar{x_1}} / \ssa{\bar{x}}]\\
     \fvssa{\eif(\ssa{\bexpr,[\bar{x}, \bar{x_1}, \bar{x_2}],[\bar{y}, \bar{y_1}, \bar{y_2}],[\bar{z}, \bar{z_1}, \bar{z_2}], c_1, c_2} )} & =& \varssa{\ssa{\bexpr}} \cup \fvssa{\ssa{c_1}} \cup \fvssa{\ssa{c_2}}  \\
      \fvssa{\ssa{c_1 ; c_2}} & = & \fvssa{\ssa{c_1}} \cup ( \fvssa{\ssa{c_2}} - \avarssa{\ssa{c_1}})
 \end{array}
 \]
%
\end{defn}
%
The Lemma~\ref{lem:fv} and \ref{lem:same_value} proved the preserving properties for variables and values during the transformation.
%
\begin{lem}[Variable Preserving]
\label{lem:fv}
If $\Sigma;\delta ; c \hookrightarrow \ssa{c} ; \delta';\Sigma' $, $\fvssa{\ssa{c}} = \delta(\fvars{c})$. 
\end{lem}
\begin{proof}
 By induction on the transformation.
 \begin{itemize}
    \caseL{Case $\inferrule{
  { \delta ; \bexpr \hookrightarrow \ssa{\bexpr} }
  \and
  { \delta ; c_1 \hookrightarrow \ssa{c_1} ; \delta_1 }
  \and
  {\delta ; c_2 \hookrightarrow \ssa{c_2} ; \delta_2 }
  \\
  {[\bar{x}, \ssa{\bar{{x_1}}, \bar{{x_2}}}] = \delta_1 \bowtie \delta_2  }
  \and
   {[\bar{y}, \ssa{\bar{{y_1}}, \bar{{y_2}}}] = \delta \bowtie \delta_1 / \bar{x} }
  \and
   {[\bar{z}, \ssa{\bar{{z_1}}, \bar{{z_2}}}] = \delta \bowtie \delta_2 / \bar{x} }
  \\
  { \delta' =\delta[\bar{x} \mapsto \ssa{\bar{{x}}'} ]}
  \and 
  {\ssa{\bar{{x}}', \bar{y}', \bar{z}'} \ fresh }
}{
 \delta ; [\eif(\bexpr, c_1, c_2)]^l  \hookrightarrow [\ssa{ \eif(\bexpr, [\bar{{x}}', \bar{{x_1}}, \bar{{x_2}}] ,[\bar{{y}}', \bar{{y_1}}, \bar{{y_2}}] ,[\bar{{z}}', \bar{{z_1}}, \bar{{z_2}}] , {c_1}, {c_2})}]^l; \delta'
}~{\textbf{S-IF}} $}
From the definition of $\fvssa{[\eif(\sbexpr, [\bar{\ssa{x'}}, \bar{\ssa{x_1}}, \bar{\ssa{x_2}}] , \ssa{c_1}, \ssa{c_2})]^l} = \varssa{\ssa{\bexpr}} \cup \fvssa{\ssa{c_1}} \cup \fvssa{\ssa{c_2}}$. We want to show: \[\varssa{\ssa{\bexpr}}) \cup \fvssa{\ssa{c_1}} \cup \fvssa{\ssa{c_2}} = \delta( \vars{\bexpr}) \cup \delta(\fv{c_1}) \cup \delta(\fv{c_2}  )\]
By induction hypothosis on the second and third premise, we know that : $\fvssa{\ssa{c_1}} = \delta(\fv{c_1}) $ and $\fvssa{\ssa{c_2}} = \delta(\fv{c_2}) $.  We still need to show that: 
\[
  \varssa{\ssa{\bexpr}} = \delta(\vars{\bexpr})
\] 
From the first premise, we know that $\vars{b} \subseteq \dom(\delta)$. This is goal is proved by the rule $\textbf{S-VAR}$ on all the variables in $\bexpr$.\\
{\caseL{Case
$\inferrule{
    { \Sigma; \delta ; c \hookrightarrow \ssa{c_1} ; \delta_1; \Sigma_1 }
     \and
    { [ \bar{x}, \ssa{\bar{{x_1}}}, \ssa{\bar{{x_2}}} ] = \delta \bowtie \delta_1 }
    \\
     {\ssa{\bar{{x}}'} \ fresh(\Sigma_1 )}
    \and {\delta' = \delta[\bar{x} \mapsto \ssa{\bar{{x}}'}]}
    \and 
     {\delta' ; \bexpr \hookrightarrow \ssa{\bexpr} }
     \and
    {\ssa{c' = c_1[\bar{x}'/ \bar{x_1}]   } }
    % \and{ \delta' ; c \hookrightarrow \ssa{c'} ; \delta'' }
  }{ 
  \Sigma; \delta ;  \ewhile ~ [\bexpr]^{l} ~ \edo ~ c 
  \hookrightarrow 
  \ssa{\ewhile ~ [\bexpr]^{l}, 0, [\bar{{x}}', \bar{{x_1}}, \bar{{x_2}}] ~ \edo ~ {c} } ; \delta'; \Sigma_1 \cup \{\ssa{\bar{x}'}  \}
}~{\textbf{S-WHILE}}
$}
}
{
Unfolding the definition, we need to show:
\[\varssa{\ssa{\bexpr}} \cup \fvssa{\ssa{c'}}[\ssa{ \bar{x_1}} / \ssa{\bar{x}}] = \delta (\vars{\bexpr}) \cup \delta(\fv{c} ) \]
We can similarly show that $\varssa{\ssa{\bexpr}} = \delta(\vars{b})$ as in the if case. We still need to show that: 
\[
 \fvssa{\ssa{c_1[\bar{x}' / \bar{x_1}]}}[ \ssa{ \bar{x_1} } / \ssa{\bar{x}'}] =  \delta(\fv{c} )
\]
It is proved by induction hypothesis on $  { \Sigma; \delta ; c \hookrightarrow \ssa{c_1} ; \delta_1; \Sigma_1 }$.\\
}
%
\caseL{Case $\inferrule{
 {\Sigma;\delta ; c_1 \hookrightarrow \ssa{c_1} ; \delta_1; \Sigma_1} 
 \and
 {\Sigma_1; \delta_1 ; c_2 \hookrightarrow \ssa{c_2} ; \delta'; \Sigma'} 
}{
\Sigma;\delta ; c_1 ; c_2 \hookrightarrow \ssa{c_1} ; \ssa{c_2} \ ; \delta';\Sigma'
}~{\textbf{S-SEQ}}$}
To show:
  \[ \fvssa{\ssa{c_1}} \cup ( \fvssa{\ssa{c_2}} - \avarssa{\ssa{c_1}}) = \delta(\fv{c_1} )\cup \delta( \fv{c_2} - \avars{c_1}) \]
  By induction hypothesis on the first premise, we know that : $ \fvssa{\ssa{c_1}} = \delta(\fv{c_1} ) $, still to show: 
    \[ ( \fvssa{\ssa{c_2}} - \avarssa{\ssa{c_1}}) = \delta( \fv{c_2} - \avars{c_1})
    \]
    We know that $\delta_1 = \delta [\avars{c_1} \mapsto \avarssa{\ssa{c_1}} ]$, so by induction hypothesis, we know: $ \fvssa{\ssa{c_2}} = \delta[\avars{c_1} \mapsto \avarssa{\ssa{c_1}} ]( \fv{c_2})  = \delta(\fv{c_2}) \cup \avarssa{\ssa{c_1}} - \delta(\avars{c_1}) $.
    
    This case is proved.
 \end{itemize}
 
\end{proof}

{
We first define a good memory in the {\tt While} language $m$ or in the ssa language $\ssa{m}$ with respect to a translation environment $\delta$, denoted as $m \vDash \delta$ and $\ssa{m} \vDash \delta$ respectively. 
%
\begin{defn}[Well Defined Memory].
\begin{enumerate}
    % \item $m \vDash c \triangleq \forall x \in \fv{c}, \exists v, (x, v) \in m$.
    \item $ m \vDash \delta  \triangleq \forall x \in \dom(\delta), \exists v, (x,v) \in m$.
    % \item $\ssa{m} \vDash_{ssa} \ssa{c} \triangleq \forall \ssa{x} \in \fvssa{\ssa{c}}, \exists v, (\ssa{x}, v) \in \ssa{m}$.
    \item $ \ssa{m} \vDash_{ssa} \delta  \triangleq \forall \ssa{x} \in \codom(\delta), \exists v, (\ssa{x},v) \in \ssa{m}$.
\end{enumerate}
\end{defn}
%
The part declared in the translation environment $\delta$ in a ssa memory $\ssa{m}$ can be reverted to corresponding part of the memory $m$ with an inverse of $\delta$ as follows.
%
\begin{defn}[Inverse of Transformed memory]
 $m = \delta^{-1}(\ssa{m}) \triangleq \forall x \in \dom(\delta), (\delta(x), m(x)) \in \ssa{m} $.
\end{defn}
}
%
\begin{lem}[Value Preserving].
\label{lem:same_value}
{
Given $\delta; e \hookrightarrow \ssa{e}$,  $\forall m. m \vDash \delta. \forall \ssa{m}, \ssa{m} \vDash_{ssa} \delta \land m = \delta^{-1}(\ssa{m})$, then $\config{m, e} \to v $ and $\config{
\ssa{m}, \ssa{e}} \to {v}$.
}
\end{lem}

\begin{thm}[Soundness of transformation]
Given $\Sigma; \delta ; c \hookrightarrow \ssa{c} ; \delta';\Sigma' $, $\forall m. m \vDash \delta. \forall \ssa{m}, \ssa{m} \vDash_{ssa} \delta \land m = \delta^{-1}(\ssa{m})$, if there exist an execution of $c$ in the while language, starting with a trace $t$ and loop maps $w$, $\config{m, c, t, w} \to^{*} \config{m', \eskip, t', w' } $,  then there also exists a corresponding execution of $\ssa{c}$ in the ssa language so that 
  $\config{  {\ssa{m}}, \ssa{c}, t, w} \to^{*} \config{{  \ssa{m'}}, \eskip, t', w' } $ and $ m' = \delta'^{-1}(\ssa{m'}) $.
\end{thm}

\begin{proof}
 We assume that $q$ is the same when transformed to $\ssa{q}$, as the primitive in both languages. And the value remains the same during the transformation.  
 It is proved by induction on the transformation rules.
 \begin{itemize}
   \caseL{Case $\inferrule{
 {\Sigma;\delta ; c_1 \hookrightarrow \ssa{c_1} ; \delta_1;\Sigma_1} 
 \and
 {\Sigma_1; \delta_1 ; c_2 \hookrightarrow \ssa{c_2} ; \delta'; \Sigma'} 
}{
\Sigma;\delta ; c_1 ; c_2 \hookrightarrow \ssa{c_1} ; \ssa{c_2} \ ; \delta';\Sigma'
}~{\textbf{S-SEQ}}$}
We choose an arbitrary memory $m$ so that $m \vDash \delta$, we choose a trace $t$ and a loop maps $w$.
\[
\inferrule
{
{\config{m, c_1,  t,w} \xrightarrow{}^{*} \config{m_1, \eskip,  t_1,w_1}}
\and
{\config{m_1, c_2,  t_1,w_1} \xrightarrow{}^{*} \config{m', \eskip,  t',w'}}
}
{
\config{m, c_1; c_2,  t,w} \xrightarrow{}^{*} \config{m', \eskip, t',w'}
}
\]
 We choose the transformed memory ${\ssa{m}} $ so that  $ m =\delta^{-1}(\ssa{m})$.
 
 To show: $ \config{\ssa{ m, c_1;c_2 }, t, w } \xrightarrow{}^{*} \config{\ssa{m', \eskip}, t'. w' }$ and $ m' = \delta'^{-1} (\ssa{m'}) $.
 
 By induction hypothesis on the first premise, we have:
 \[ \config{\ssa{m, c_1}, t,w} \xrightarrow{}^{*} \config{\ssa{m_1, \eskip},t_1,w_1 } \land m_1 = \delta_1^{-1}(\ssa{m_1}) \]
  By induction hypothesis on the second premise, using the conclusion $ m_1 = \delta_1^{-1}(\ssa{m_1}) $.
  We have:
  \[
   \config{\ssa{m_1, c_2}, t_1,w_1} \xrightarrow{}^{*} \config{\ssa{m', \eskip},t',w' } \land m' = \delta'^{-1}(\ssa{m'})
  \]
  So we know that 
  \[
  \inferrule{
  { \config{\ssa{m, c_1}, t,w} \xrightarrow{}^{*} \config{\ssa{m_1, \eskip},t_1,w_1 }  }
  \and
  { \config{\ssa{m_1, c_2}, t_1,w_1} \xrightarrow{}^{*} \config{\ssa{m', \eskip},t',w' } }
  }{
  \config{\ssa{m, c_1;c_2 }, t,w} \xrightarrow{}^{*} \config{\ssa{m', \eskip}, t' , w' }
  }
  \]
 \caseL{Case $\inferrule{
 { \delta ; \expr \hookrightarrow \sexpr}
 \and
 {\delta' = \delta[x \mapsto \ssa{x} ]}
 \and{ \ssa{x} \ fresh(\Sigma) }
 \and {\Sigma' = \Sigma \cup \{x\} }
}{
 \Sigma;\delta ; [\assign x \expr]^{l} \hookrightarrow [\assign {\ssa{x}}{ \ssa{\expr}}]^{l} ; \delta';\Sigma'
}~{\textbf{S-ASSN}} $ }

 We choose an arbitrary memory $m$ so that $m \vDash \delta$, we choose a trace $t$ and a loop maps $w$, we know that the resulting trace is still $t$ from its evaluation rule $\textbf{assn}$ when we suppose $m, \expr \to v$.
 \[
 \inferrule
{
}
{
\config{m, [\assign x v]^{l},  t,w} \xrightarrow{} \config{m[v/x], [\eskip]^{l}, t,w}
}
~\textbf{assn}
 \]
 We choose the transformed memory ${\ssa{m}} $ so that  $ m =\delta^{-1}(\ssa{m})$.
 
 To show: $\config{\ssa{m}, [\assign {\ssa{x}}{ \ssa{\expr}}]^{l} , t, w} \to^{*} \config{\ssa{m'}, \eskip, t, w} $ and $ m' = \delta'^{-1}(\ssa{m'}) $.
 
 From the rule \textbf{ssa-assn}, we assume $\ssa{m}, \ssa{\expr} \to \ssa{v}$, we know that 
 \[
 \inferrule
{
}
{
\config{\ssa{ m, [\assign x v]^{l}},  t,w } \xrightarrow{} \config{\ssa{m[x \mapsto v], [\eskip]^{l}}, t,w }
}
~\textbf{ssa-assn}
 \]
 We also know that $\delta'= \delta[x \mapsto \ssa{x}]$ and $m = \delta^{-1}(\ssa{m})$, $m'= m[v/x]$. We can show that $ m[v/x] = \delta[x \mapsto \ssa{x}]^{-1}(\ssa{m}[\ssa{x} \mapsto v]) $.
 
\caseL{Case $\inferrule{
 {\delta ; q \hookrightarrow \ssa{q}}
 \and
 {\delta ; \expr \hookrightarrow \ssa{\expr}}
 \and
 {\delta' = \delta[x \mapsto \ssa{x} ]}
 \and{ \ssa{x} \ fresh(\Sigma) }
 \and{ \Sigma' = \Sigma \cup \{x\} }
}{
 \Sigma;\delta ; [\assign{x}{\query(\qexpr)}]^{l} \hookrightarrow [\assign {\ssa{x}}{ \ssa{\query(\qexpr)}}]^{l} ; \delta'
}~{\textbf{S-QUERY}}$} 
We choose an arbitrary memory $m$ so that $m \vDash \delta$, we choose a trace $t$ and a loop maps $w$, we know that when we suppose $\config{m, \expr} \to v$.
 \[
\inferrule
{
\query(v)(D) = \qval 
}
{
\config{m, [\assign{x}{\query(v)}]^l, t, w} \xrightarrow{} \config{m[ \qval/ x], \eskip,  t \mathrel{++} [\query(v),l,w )],w }
}
~\textbf{query}
 \]
 We choose the transformed memory ${\ssa{m}} $ so that  $ m =\delta^{-1}(\ssa{m})$.
 
 To show: $\config{\ssa{m}, [\assign {\ssa{x}}{ \ssa{\query(\qexpr)}}]^{l} , t, w} \to^{*} \config{\ssa{m'}, \eskip, t, w} $ and $ m' = \delta'^{-1}(\ssa{m'}) $.
 
 From the rule \textbf{ssa-query}, we know that 
 \[
 \inferrule
{
\ssa{\query(v)(D) = \qval} 
}
{
\config{ \ssa{ m, [\assign{\ssa{x}}{\ssa{\query(\qexpr)}}]^l}, t, w} \xrightarrow{} \config{\ssa{  m[  x \mapsto v], \eskip,}  t \mathrel{++} [(q^{(l,w )},v)],w }
}
~\textbf{ssa-query}
 \]
 We also know that $\delta'= \delta[x \mapsto \ssa{x}]$ and $m = \delta^{-1}(\ssa{m})$, $m'= m[v/x]$. We can show that $ m[v/x] = \delta[x \mapsto \ssa{x}]^{-1}(\ssa{m}[\ssa{x} \mapsto v]) $.

  \caseL{Case $\inferrule{
  { \delta ; \bexpr \hookrightarrow \ssa{\bexpr} }
  \and
  { \Sigma; \delta ; c_1 \hookrightarrow \ssa{c_1} ; \delta_1;\Sigma_1 }
  \and
  {\Sigma_1; \delta ; c_2 \hookrightarrow \ssa{c_2} ; \delta_2 ; \Sigma_2 }
  \\
  {[\bar{x}, \ssa{\bar{{x_1}}, \bar{{x_2}}}] = \delta_1 \bowtie \delta_2  }
  \and
   {[\bar{y}, \ssa{\bar{{y_1}}, \bar{{y_2}}}] = \delta \bowtie \delta_1 / \bar{x} }
  \and
   {[\bar{z}, \ssa{\bar{{z_1}}, \bar{{z_2}}}] = \delta \bowtie \delta_2 / \bar{x} }
  \\
  { \delta' =\delta[\bar{x} \mapsto \ssa{\bar{{x}}'} ][\bar{y} \mapsto \ssa{\bar{{y}}'} ][\bar{z} \mapsto \ssa{\bar{{z}}'} ]}
  \and 
  {\ssa{\bar{{x}}', \bar{y}', \bar{z}'} \ fresh(\Sigma_2)
  }
  \and{\Sigma' = \Sigma_2 \cup \{ \ssa{ \bar{x}', \bar{y}', \bar{z}' } \} }
}{
 \Sigma; \delta ; [\eif(\bexpr, c_1, c_2)]^l  \hookrightarrow [\ssa{ \eif(\bexpr, [\bar{{x}}', \bar{{x_1}}, \bar{{x_2}}] ,[\bar{{y}}', \bar{{y_1}}, \bar{{y_2}}] ,[\bar{{z}}', \bar{{z_1}}, \bar{{z_2}}] , {c_1}, {c_2})}]^l; \delta';\Sigma'
}~{\textbf{S-IF}}$}
We choose an arbitrary memory $m$ so that $m \vDash \delta$, we choose a trace $t$ and a loop maps $w$.
There are two possible evaluation rules depending on the the condition $b$, we choose the case when $b = \etrue$, we know there is an execution in ssa language so that $\ssa{\bexpr} = \etrue$, we use the rule $\textbf{if-t}$.  
 \[\inferrule
{
}
{
\config{m, [\eif(\etrue, c_1, c_2)]^{l},t,w} 
\xrightarrow{} \config{m, c_1,  t,w} \to^{*} \config{m', \eskip, t', w'}
}
\]
 We choose the transformed memory ${\ssa{m}} $ so that  $ m =\delta^{-1}(\ssa{m})$.
 
 To show: $\config{\ssa{m}, [\eif(\etrue, [\bar{\ssa{x}}', \bar{\ssa{x_1}}, \bar{\ssa{x_2}}] ,[\bar{\ssa{y}}', \bar{\ssa{y_1}}, \bar{\ssa{y_2}}] ,[\bar{\ssa{z}}', \bar{\ssa{z_1}}, \bar{\ssa{z_2}}] , c_1, c_2)]^{l}, t, w} \to^{*} \config{\ssa{m'}, \eskip, t', w'} $ and $ m' = \delta'^{-1}(\ssa{m'}) $.

We use the corresponding rule $\textbf{SSA-IF-T}$.  
\[
\inferrule
{
}
{
\config{\ssa{ { m} , [\eif(\etrue, [\bar{\ssa{x}}', \bar{\ssa{x_1}}, \bar{\ssa{x_2}}] , [\bar{\ssa{y}}', \bar{\ssa{y_1}}, \bar{\ssa{y_2}}] ,[\bar{\ssa{z}}', \bar{\ssa{z_1}}, \bar{\ssa{z_2}}] , \ssa{c_1, c_2})]^{l}},t,w} 
\xrightarrow{} \\ \config{\ssa{ m, c_1}; \eifvar(\ssa{\bar{x}', \bar{x_1}});\eifvar(\ssa{\bar{y}', \bar{y_2}});\eifvar(\ssa{\bar{z}', \bar{z_1}}),  t,w } 
}~\textbf{ssa-if-t}
\]
By induction hypothesis on $ \Sigma;\delta ; c_1 \hookrightarrow  \ssa{c_1}; \delta_1;\Sigma_1$, and we know that $\config{m, c_1,  t,w} \to^{*} \config{m', \eskip, t', w'} $, from our assumption that $ m =\delta^{-1}(\ssa{m})$, we know that 
\[\config{\ssa{ { m}, c_1},  t,w} \to^{*} \config{ \ssa{ { m_1 }, \eskip,} t', w' } \land m' = \delta_1^{-1}(\ssa{m_1}) \]
and we then have:
\[
\inferrule
{
  \config{\ssa{ { m}, c_1},  t,w} \to^{*} \config{ \ssa{ { m_1 }, \eskip,} t', w' }
}
{
 \config{\ssa{  m, c_1;} \eifvar(\ssa{\bar{x}', \bar{x_1})};\eifvar(\ssa{\bar{y}', \bar{y_1})};\eifvar(\ssa{\bar{z}', \bar{z_1})},  t,w  }  \to^{*}
 \config{\ssa{ { m_1 [ \bar{x}' \mapsto {m_1}(\bar{x_1}),\bar{y}' \mapsto {m_1}(\bar{y_2}),\bar{z}' \mapsto {m_1}(\bar{z_1}) ] }, \eskip}, t', w'  }
}
\]
Now, we want to show that $ m' = \delta[\bar{x} \mapsto \ssa{\bar{x}'},\bar{y} \mapsto \ssa{\bar{y}'},\bar{z} \mapsto \ssa{\bar{z}'} ]^{-1}(\ssa{ m_1 [ \bar{x}' \mapsto {m_1}(\bar{x_1}),\bar{y}' \mapsto {m_1}(\bar{y_2}),\bar{z}' \mapsto {m_1}(\bar{z_1}) ] }) $.

Unfold the definition, we want to show that $$\forall x  \in ( \dom(\delta) \cup \bar{x} \cup \bar{y} \cup \bar{z} ), (\delta[\bar{x} \mapsto \ssa{\bar{x}'},\bar{y} \mapsto \ssa{\bar{y}'},\bar{z} \mapsto \ssa{\bar{z}'} ](x), m'(x)) \in \ssa{m_1 [ \bar{x}' \mapsto {m_1}(\bar{x_1}),\bar{y}' \mapsto {m_1}(\bar{y_2}),\bar{z}' \mapsto {m_1}(\bar{z_1}) ] } .$$
\begin{enumerate}
    \item For variable $x$ in $\bar{x}$, we can find a corresponding ssa variable $\ssa{x} \in \ssa{\bar{x}'}$, so that $( \ssa{x}, m'(x) ) \in \ssa{ m_1 [\bar{x}' \mapsto m_1(\bar{x_1})] } $. It is because we know $[x \mapsto \ssa{x_1}]$ for certain $\ssa{x_1} \in \ssa{\bar{x_1}}$ in $\delta_1$, then by unfolding  $m' = \delta_1^{-1}(\ssa{m_1})$ and $\ssa{\bar{x_1}} \in \codom(\delta_1)$, we know $(\ssa{x_1}, m'(x)) \in \ssa{m_1}$ so that $m'(x) = \ssa{m_1}(\ssa{x_1})$.
    \item For variable $y \in \bar{y}$, we know that $y \in \dom(\delta_1)$, then $[ y \mapsto \ssa{y_2} ]$ for certain $\ssa{y_2} \in \ssa{\bar{y_2}}$ in $\delta_1$.  So we know that $(\delta_1(y), m'(y) ) \in \ssa{m_1}$, and then $m'(y) = \ssa{m_1(y_2)}$. We can show $(\ssa{y}, m'(y)) \in \ssa{m_1[\bar{y}' \mapsto m_1(\bar{y_2})]}$.
    \item For variable $z \in \bar{z}$, we know that $z \not\in \dom(\delta_1)$ by the definition (otherwise $z$ will appear in $\bar{x}$), then $[ z \mapsto \ssa{z_1} ]$ for certain $ \ssa{z_1} \in \ssa{\bar{z_1}}$ in $\delta$.  We know $(\delta(z), m(z)) \in \ssa{m}$ from our assumption, so we have $ m(z) = \ssa{m(z_1)}$. Because $z$ is not modified in $c_1$, so that $m(z) = m'(z)$. Also $\ssa{m}$ will not shrink during execution and $\ssa{z_1}$ will not be written in $\ssa{c_1}$, so $(\ssa{z_1}, m'(z)) \in \ssa{m_1}$. Then we can show that $ (\ssa{z}, m'(z) ) \in \ssa{m_1[\bar{z}' \mapsto m_1(\bar{z_1})] }$.
    \item For variable $k \in \dom(\delta)- \bar{x} - \bar{y}-\bar{z}$. From our assumption $ m = \delta^{-1}(\ssa{m})$, we can show $(\delta(k), m(k) ) \in \ssa{m}$. We know that $k$ is not written in either branch from our definition, so $(\delta(k), m'(k) ) \in \ssa{m_1} $ .
\end{enumerate}

{
\caseL{
	Case
	$
	\inferrule{
    { \Sigma; \delta ; c \hookrightarrow \ssa{c_1} ; \delta_1; \Sigma_1 }
     \and
    { [ \bar{x}, \ssa{\bar{{x_1}}}, \ssa{\bar{{x_2}}} ] = \delta \bowtie \delta_1 }
    \\
     {\ssa{\bar{{x}}'} \ fresh(\Sigma_1 )}
    \and {\delta' = \delta[\bar{x} \mapsto \ssa{\bar{{x}}'}]}
    \and 
     {\delta' ; \bexpr \hookrightarrow \ssa{\bexpr} }
     \and
    {\ssa{c' = c_1[\bar{x}'/ \bar{x_1}]   } }
    % \and{ \delta' ; c \hookrightarrow \ssa{c'} ; \delta'' }
  }{ 
  \Sigma; \delta ;  \ewhile ~ [\bexpr]^{l} ~ \edo ~ c 
  \hookrightarrow 
  \ssa{\ewhile ~ [\bexpr]^{l}, 0, [\bar{{x}}', \bar{{x_1}}, \bar{{x_2}}] ~ \edo ~ {c} } ; \delta'; \Sigma_1 \cup \{\ssa{\bar{x}'}  \}
}~{\textbf{S-WHILE}}
$
}
}
\\
{
We choose an arbitrary memory $m$ so that $m \vDash \delta$, we choose a trace $t$ and a loop maps $w$. Suppose $ \config{m ,a} \to v_N $. There are two cases, when $v_N=0$, the loop body is not executed so we can easily show that the trace is not modified.
%
When the while loop is still running ($v_N > 0$), we have the following evaluation in the while language:
\[
\inferrule
{
 \empty
}
{
\config{
m, \ewhile ~ [b]^{l} ~ \edo [c]^{l + 1},  t, w 
}
\xrightarrow{} \config{m, c ; 
\eif_w (b, c ; 
\ewhile ~ [b]^{l} ~ \edo [c]^{l + 1},  \eskip),
t, w }
}
~\textbf{while-b}
\]
which follows by the following evaluation:
\[
	\inferrule
{
 m, b \xrightarrow{} b'
}
{
\config{m, \eif_w (b, c,  \eskip) ,  t, w }
\xrightarrow{} \config{m, 
 \eif_w (b', c,  \eskip), t, w }
}
~\textbf{ifw-b}
\]
In the corresponding ssa-form language, we have the corresponding evaluation in the same way by assuming 
$m = \delta^{-1}(\ssa{m})$.
%
\[
	\inferrule
{
 {n = 0 \rightarrow i = 1 }
 \and
 {n > 0 \rightarrow i = 2 }
}
{
\config{
\ssa{m},  
\ssa{\ewhile ~ [\bexpr]^{l}, n, 
[\bar{{x}}', \bar{{x_1}}, \bar{{x_2}}] 
~ \edo ~ {c} 
},  t, w 
}
\xrightarrow{} \\ 
\config{
\ssa{m},
\eif_w 
(\ssa{b[\bar{x_i}/\bar{x'}], [\bar{{x}}', \bar{{x_1}}, \bar{{x_2}}], n,  c[\bar{x_i}/\bar{x'}] }; 
\ssa{
\ewhile ~ [b]^{l}, n+1, 
[\bar{{x}}', \bar{{x_1}}, \bar{{x_2}}]  
~ \edo ~ c} ,  \eskip),
t, w
}
}
~\textbf{ssa-while-b}
\]
This evaluation is followed by the following evaluation:
\[
	\inferrule
{
 \ssa{m, b \xrightarrow{} b'}
}
{
\config{\ssa{m, \eif_w (b, [\bar{{x}}', \bar{{x_1}}, \bar{{x_2}}] , n,  c_1,  c_2)} ,  t, w }
\xrightarrow{} \config{\ssa{ m, 
 \eif_w (b', [\bar{{x}}', \bar{{x_1}}, \bar{{x_2}}] , n , c_1 , c_2 )}, t, w }
}
~\textbf{ssa-ifw-b}
\]
%
Depending on if the counter $n$ is equal to $0$ or not, there are two possible execution paths (the variables $\ssa{\bar{x}}$ is replaced by the $\ssa{\bar{x_1}}$ or $\ssa{\bar{x_2}}$). We start from the first iteration (when $n =0$) when $v_N >0$. 
}
{
By induction hypothsis on the premise $ { \Sigma; \delta ; c \hookrightarrow \ssa{c_1} ; \delta_1; \Sigma_1 }$, we know that 
\[ \config{\ssa{{m}, c'[ \bar{x_1}/\bar{x}'  ]}, t, (w+l)  } \to^{*} \config{\ssa{{m'}, \eskip}, t'_{i}, w'  } \land m' = \delta_1^{-1}(\ssa{m'})   \]
Hence we can conclude that:
\[
  \inferrule{
   \config{\ssa{{m}, c'[ \bar{x_1}/\bar{x}'  ]}, t, (w+l) }  \to^{*} \config{\ssa{{m'}, \eskip}, t'_{1}, w'  }
  }{
  \config{\ssa{ {m}, c'[ \bar{x_1}/\bar{x}'  ];  [\eloop ~ (\valr_N-1), n+1, [\bar{\ssa{x}}', \bar{\ssa{x_1}}, \bar{\ssa{x_2}}] ~  \edo ~ c' ]^{l} },  t, (w + l)  }  \to^{*} \\ \config{ \ssa{{m'}, [\eloop ~ (\valr_N-1), n+1, [\bar{\ssa{x}}', \bar{\ssa{x_1}}, \bar{\ssa{x_2}}] ~  \edo ~ c' ]^{l}}, t'_{1}, w'  } 
  }
\]
%
Then there are two cases, 
%
\begin{enumerate}
     \item  when guard in the $\eif_w$ expression evaluates to $\efalse$, the while loop terminates and exits.
     The execution in the while language is defined in the evaluation rule $\textbf{ifw-false}$ as follows.
     \[
		\inferrule
		{
		 \empty
		}
		{
		\config{\ssa{
		m, \eif_w (
		\efalse, [\bar{{x}}', \bar{{x_1}}, \bar{{x_2}}],   n, 
		c; {\ewhile ~ [b]^{l} ~ \edo ~ c},
		\eskip)
		)} ,  t, w }
		\\
		\xrightarrow{} 
		\config{\ssa{m, 
		{\eskip}; \eifvar(\bar{x'}, \bar{x_i}) }, t, (w - l) }
		}
		~\textbf{ifw-false}
	\]
%
	The corresponding ssa-form evaluation as follows:
	\[
		\inferrule
		{
		 { n = 0 \rightarrow i = 1 }
		 \and
		 {n > 0 \rightarrow i =2}
		}
		{
		\config{\ssa{
		m, \eif_w (
		\efalse, [\bar{{x}}', \bar{{x_1}}, \bar{{x_2}}],   n, 
		{  
		c; \ssa{\ewhile ~ [b]^{l}, n, [\bar{{x}}', \bar{{x_1}}, \bar{{x_2}}]  ~ \edo ~ c},
		\eskip)
		} 
		)} ,  t, w }
		\\
		\xrightarrow{} 
		\config{\ssa{m, 
		{\eskip}; \eifvar(\bar{x'}, \bar{x_i}) }, t, (w - l) }
		}
		~\textbf{ssa-ifw-false}
	\]
	We can see that both traces are not changed during the exit of the while. We need to show that $ m' = \delta^{-1} (\ssa{m'[\bar{x} \mapsto m'(\bar{x_2})]}) $. We know that $[ \bar{x} \mapsto \bar{x_2}]$ in $\delta_1$ from the definition, so we can show that for any variable $\ssa{x_2} \in \bar{x_2}$, $( \ssa{x_2}, m'(x) ) \in \ssa{m'}$. For variables $x \in {\dom(\delta) - \bar{x} } $, the variable is not modified during the execution of $c$ so that we know $m(x) = m'(x)$, and then we can show that $(\delta(x), m'(x)) \in \ssa{m'} $ because $\delta(x)$ is not written in $\ssa{c'[\bar{x_1}/ \bar{x}']}$ .
%
  	\item 
		when guard in the $\eif_w$ expression evaluates to $\etrue$, the while terminates and exits.
     The execution in the while language is defined in the evaluation rule $\textbf{ifw-true}$.
          %
     We want to show that : assuming in the $i-th$ $(i < \ssa{n})$ iteration, starting with $t_i$ and $w_i$ and $m_i = \delta_1^{-1}(\ssa{m_i})$,
     this command is evaluated according to the while language operation semantics as
     	$
		\config{m, \eif_w (\etrue, c ; \ewhile ~ [b]^{l} ~ \edo ~ c, ,  \eskip) ,  t, w }
		\xrightarrow{}^* \config{m, c 
		t, (w + l) }
 		$.
     %
     Then the corresponding ssa form evaluation as follows : 
     %
     \[ 
     \inferrule{}{
     	\config{
		\ssa{
			m, 
			{
			\eif_w (\etrue, [\bar{{x}}', \bar{{x_1}}, \bar{{x_2}}], n,  
			c; \ssa{\ewhile ~ [b]^{l}, n, [\bar{{x}}', \bar{{x_1}}, \bar{{x_2}}]  ~ \edo ~ c},
			\eskip)
			} 
		},  t, w 
		}
		\\
		\xrightarrow{} 
		\config{
		\ssa{m, 
		{
		\eif_w (\etrue, [\bar{{x}}', \bar{{x_1}}, \bar{{x_2}}], n,  
		c; \ssa{\ewhile ~ [b]^{l}, n, [\bar{{x}}', \bar{{x_1}}, \bar{{x_2}}]  ~ \edo ~ c},
		}
		}
		t, (w + l) }
		} 
     \]  
     and $m_i = \delta^{-1}(\ssa{m_i}) $.
     We then have the evaluation in the while language:
     \[
		\inferrule
		{
		 \empty
		}
		{
		\config{m, 
		\eif_w (b, 
		c ; \ewhile ~ [b]^{l} ~ \edo ~ c, 
		\eskip),
		t, w }
		\xrightarrow{} 
		\config{m, 
		c ; \ewhile ~ [b]^{l} ~ \edo ~ c,  
		t, (w + l) }
		}
		~\textbf{ifw-true}
	\]
	We then have the following evaluation:
	\[
		\inferrule
		{
		 \empty
		}
		{
		\config{
		\ssa{
		m, 
		{
		\eif_w (\etrue, [\bar{{x}}', \bar{{x_1}}, \bar{{x_2}}], n,  
		c; \ssa{\ewhile ~ [b]^{l}, n, [\bar{{x}}', \bar{{x_1}}, \bar{{x_2}}]  ~ \edo ~ c},
		\eskip)
		} 
		},  t, w 
		}
		\\
		\xrightarrow{} 
		\config{
		\ssa{m, 
		{
		\eif_w (\etrue, [\bar{{x}}', \bar{{x_1}}, \bar{{x_2}}], n,  
		c; \ssa{\ewhile ~ [b]^{l}, n, [\bar{{x}}', \bar{{x_1}}, \bar{{x_2}}]  ~ \edo ~ c},
		}
		}
		t, (w + l) }
		}
		~\textbf{ssa-ifw-true}
	\]
%
By induction hypothsis on the premise $  { \Sigma; \delta_1 ; c \hookrightarrow \ssa{c_2} ; \delta_1; \Sigma_1 }$, we know that
%
\[
\config{\ssa{{m_i}, c'[ \bar{x_2}/\bar{x}'  ]}, t_i, (w_i+l)  } \to^{*} \config{\ssa{{m_{i+1}}, \eskip}, t_{i+1}, w_{i+1}  } \land m_{i+1} = \delta_1^{-1}(\ssa{m_{i+1}})
\]
%
Hence we can conclude that:
\[
  \inferrule{
   \config{\ssa{{m_i}, c'[ \bar{x_2}/\bar{x}'  ]}, t_i, (w_i+l) }  \to^{*} \config{\ssa{{m_{i+1}}, \eskip}, t_{i+1}, w_{i+1}  }
  }{
  \config{\ssa{ {m_i}, c'[ \bar{x_2}/\bar{x}'  ];  [\eloop ~ (\valr_N-i-1), n+1, [\bar{\ssa{x}}', \bar{\ssa{x_1}}, \bar{\ssa{x_2}}] ~  \edo ~ c' ]^{l} },  t_i, (w_i + l)  }  \to^{*} \\ \config{ \ssa{{m_{i+1}}, [\eloop ~ (\valr_N-i-1), n+1, [\bar{\ssa{x}}', \bar{\ssa{x_1}}, \bar{\ssa{x_2}}] ~  \edo ~ c' ]^{l}}, t_{i+1}, w_{i+1}  } 
  }
\]
So we can show that before the exit of the loop after ($v_N= n $) iterations, we have $t_{n} = t_{n}$ and $m_{n} = \delta_1^{-1}(\ssa{m_{n}})$.
 \end{enumerate}
%
This proof is similar when it comes to the exit as in case 1. 
}
\end{itemize}
%
\end{proof}
%
\clearpage
%
The following lemma describes a property of the trace-based dependency graph.
For any program $c$ with a database $D$ and a initial trace $\trace$,
the directed edges in its trace-based dependency graph can only be constructed from variable with  
smaller labels variables of greater ones.
There doesn't exist backward edges with direction from greater labelled variables to smaller ones.
% \begin{lem}
% \label{lem:edgeforwarding}
% [Edges are Forwarding Only].
% \\
% %
% %
% $$
% \forall \trace \in \mathcal{T}, D \in \dbdom \st G(c, D) =  (\vertxs, \edges, \weights, \qflag) 
% \implies
% \forall (\event', \event) \in \edges \st \event' \eventleq \event
% $$
% %
% \end{lem}
% %
% \begin{proof}
% Proof in File: {\tt ``edge\_forward.tex''}.
% % \begin{proof}
This lemma is proved by showing there is a contradiction. 

\jl{
Assume there exists an edge  $(\av', \av) \in \edges$ and $\av' \avgeq \av$, where $\av' = ({\qval}',l',w')$ and $\av = ({\qval},l,w)$.
%
According to the Definition~\ref{def:trace-based_graph}, we have:
%
$$
DEP(\av', \av, c, m, D) ~ (1)
$$
%
%
Unfolding the Definition \ref{def:query_dep} in $(1)$,
we know: there exists $t_1, t_3, m_1, m_3, c_2$ s.t.,
%
\[
\config{m, c, [], []} \rightarrow^{*} 
\config{m_1, [\assign{x}{\query({\qval'})}]^{l'} ; c_2,
  t_1, w'} 
\rightarrow^{\textbf{query-v}} 
\config{m_1[v_1/x], c_2,
(t_1 ++ [\av'], w_1} \rightarrow^{*} \config{m_3, \eskip, t_3,w_3} ~ (a)  
\]
%
and
%
\[
 \bigwedge
 \begin{array}{l}   
  % 
  \left( 
  \begin{array}{l}
  \av \avin (t_3 - (t_1 ++ [\av'])) 
  % 
  \\
  \implies 
  \exists v \in \qdom, v \neq v_1, m_3', t_3', w_3'. ~  
  \config{m_1[v/x], {c_2}, t_1 ++ [\av'], w_1} 
  \\ 
  \quad \quad 
  \rightarrow^{*}
  (\config{m_3', \eskip, t_3', w_3'} 
  \land 
  \av \not \avin (t_3'-(t_1 ++ [\av'])))
\end{array} \right ) ~(b)
\\
\left( 
  \begin{array}{l}
  \av \not\avin (t_3 - (t_1 ++ [\av']))
    % 
    \\
    \implies 
  \exists v \in \qdom, v \neq v_1, m_3', t_3', w_3'. 
  \config{m_1[v/x], {c_2}, t_1 ++ [\av'], w_1}
  \\ 
  \quad \quad 
  \rightarrow^{*} 
  (\config{m_3', \eskip, t_3', w_3'} 
  \land 
  \av  \avin (t_3' - (t_1 ++ [\av'])))
\end{array} \right ) ~ (c)
\end{array}
\]
%
%
According to the Theorem \ref{thm:os_wf_trace} and $(a)$, we know both $t_1$, $t_3$ are well-formed traces.
% }
% \\
% \jl{
\\
Consider 2 cases:
 \[\av \avin (t_3 - (t_1 ++[\av'])) ~(d) ~~ \lor ~~ \av \notin_{aq} (t_3 - (t_1 ++[\av'])) ~(e)\]
%
\begin{itemize}
%
  \caseL{ (d) \[\av \avin (t_3 - (t_1 ++[\av'])) ~ (d)\]}
  By unfolding the trace subtraction operations, we have;
  \[
    \exists t_2. ~ s.t., ~ t_1 ++[\av'] ++ t_2 = t_3 \land \av \avin t_2 ~ (3)
  \]
%
%
According to the Corollary~\ref{coro:aqintrace} and $\av \avin t_2$, we have:
%
\[
  \exists t_{21}, t_{22}, \av' ~ s.t.,~ (\av \aveq \av') \land t_{21} ++ [\av'] ++ t_{22} = t_2 ~ (4)
\]
%
By rewriting $(4)$ inside $(3)$, we have:
%
\[
  \exists t_{21}, t_{22}, \av'' ~ s.t.,~ (\av \aveq \av'') \land t_1 ++[\av'] ++ t_{21} ++ [\av''] ++ t_{22} = t_3 ~ \star
\]
%
By the \emph{ordering} property in definition \ref{def:wf_trace} and $(\star)$, we know
%
\[
  \av' <_{aq} \av
\]
%
This is contradict to our assumption, where $\av' \avgeq \av$. 
%
%
\caseL{(e), \[\av \notin_{aq} (t_3 - (t_1 ++[\av'])) ~(e)\]} 
%
According to the condition $(c)$, we know: $\exists v \in \qdom, v \neq v_1, m_3', t_3', w_3'.$
    % 
\[ 
  \config{m_1[v/x], {c_2}, t_1 ++ [\av'], w_1} 
  \rightarrow^{*} (\config{m_3', \eskip, t_3', w_3'} ~ (5)
  \land \av  \in_{q} (t_3' - (t_1 ++ [\av']))) ~ (6)
\] 
%
According to the Theorem \ref{thm:os_wf_trace}, Sub-Lemma-t and $(5)$,
we know $t_3'- (t_1 ++ [\av'])$ is a well-formed trace.
\\
From $(a)$, we know that the minimal line number of $c_2$ is greater than $l'$, so we know that : 
\[
  \forall \av'' \in t_3'- (t_1 ++ [\av']), \av''>_{aq} \av'
\]
%
Unfolding the trace subtraction operations in $(6)$, we have;
\[
  \exists t_2'. ~ s.t., ~ t_1 ++[\av'] ++ t_2' = t_3' \land \av \avin t_2' ~ (7)
\]
%
%
According to the sub-lemma and $(7)$, we have:
%
\[
  \exists t_{21}', t_{22}' ~ s.t.,~ t_{21}' ++ [\av] ++ t_{22}' = t_2' ~ (8)
\]
%
By rewriting $(8)$ inside $(7)$, we have:
%
\[
  t_1 ++[\av'] ++ t_{21}' ++ [\av] ++ t_{22}' = t_3' ~ \diamond
\]
%
By the \emph{ordering} property in definition \ref{def:wf_trace} and $(\diamond)$, we know
%
\[
  \av' <_{aq} \av
\]
%
This is contradict to the assumption,  $\av' \avgeq \av$.
%
\end{itemize}
%
From both cases, we derive $\av' \avgeq \av$, which is contradict to the hypothesis, i.e., $\av' \avgeq \av$.
Then, we can conclude that 
for any directed edge $(\av', \av) \in \edges$, 
this is not the case that:
%
$$\av' \avgeq \av$$
%
}
\end{proof}
%
%
%
% \end{proof}
%
%
% %
% \begin{lem}
% \label{lem:DAG}
% [Trace-based Dependency Graph is Directed Acyclic].
% \\
% %
% $\forall \trace \in \mathcal{T}, D \in \dbdom $, $G(c, D)$ is a directed acyclic graph.
% \end{lem}
%
%
\subsection{Trace-based Adaptivity}
Given a program $c$ with a database $D$, we generate its program-based graph 
$\traceG(\ssa{c}, D) = (\vertxs, \edges, \weights, \qflag)$.
%
Then the adaptivity bound based on program analysis for $\ssa{c}$ is the number of query vertices on a finite walk in $\progG(\ssa{c})$. This finite walk satisfies:
%
\begin{itemize}
\item the number of query vertices on this walk is maximum
\item the visiting times of each vertex $v$ on this walk is bound by its weight $\weights(v)$.
\end{itemize}
%
It is formally defined in \ref{def:trace_adapt}.
%
\begin{defn}
[Adaptivity of A Program].
\label{def:trace_adapt}
\\
Given a program $\ssa{c}$ in SSA language, 
its adaptivity is defined for all possible starting SSA memory $\ssa{m}$ and database $D$ as follows:
%
$$
A(c) = \max \big 
\{ \qlen(k) \mid D \in \dbdom , k \in \walks(\traceG(c, D) \big \} 
$$
\end{defn}
%
%
%
%
\begin{lem}
[Adaptivity is Bounded].
\\
{
Given the program $c$ with a certain database $D$ and starting memory $m$, the $A(c)$ w.r.t. the $D$ and $m$ is bounded, i.e.,:
%
\[
\config{} 
\rightarrow^{*} 
\config{m', \eskip, t', w'} 
\implies
A_{D, m}(c) \leq |t'|
\]
}
\end{lem}
%
\begin{proof}
{
Proof is obvious based on the Lemma \ref{lem:DAG}.
}
\end{proof}
%
%
%
%
% \subsection{SSA Transformation and Soundness of Transformation}
% in File {\tt ``ssa\_transform\_sound.tex''}
% %
\subsection{SSA Transformation}
We use a translation environment $\delta$, to map variables $x$ in the {\tt While} language to those variables $\ssa{x}$ in the SSA language.
We use a name environment denoted as $\Sigma$ as a set of ssa variables, to get a fresh variable by $fresh(\Sigma)$. 
We define $\delta_1 \bowtie \delta_2 $ in a similar way as
\cite{vekris2016refinement}.
%
\[ 
\delta_1 \bowtie \delta_2 = \{ ( x, {\ssa{x_1}, \ssa{x_2}} ) \in 
\mathcal{VAR} \times \mathcal{SVAR} \times \mathcal{SVAR} \mid x \mapsto {\ssa{x_1}} \in \delta_1 , x \mapsto {\ssa{x_2} } \in \delta_2, {\ssa{x_1} \not= {\ssa{x_2} }  }  \} 
\]
%
\[ 
\delta_1 \bowtie \delta_2 / \bar{x} = \{ ( x, {\ssa{x_1}, \ssa{x_2}} ) \in 
\mathcal{VAR} \times \mathcal{SVAR} \times \mathcal{SVAR}
 \mid x \not\in \bar{x} \land x \mapsto {\ssa{x_1}} \in \delta_1 , x \mapsto {\ssa{x_2} } \in \delta_2, {\ssa{x_1} \not= {\ssa{x_2} }   }  \} 
 \]
 %
We call a list of variables $\bar{x}$.
\[
 [\bar{x}, \bar{\ssa{x_1}}, \bar{\ssa{x_2}}] = \{ (x, x_1,x_2)  | \forall 0 \leq i < |\bar{x}|, x = \bar{x }[i] \land x_1 = \bar{x_1}[i] \land x_2 = \bar{x_2 }[i] \land |\bar{x}| = |\bar{x_1}| = |\bar{x_2}|   \}
\]
%
\begin{mathpar}
\boxed{ \delta ; e \hookrightarrow \ssa{e} }
\and
\inferrule{
}{
 \delta ; x \hookrightarrow \delta(x)
}~{\textbf{S-VAR}}
\and
\boxed{ \Sigma; \delta ; c  \hookrightarrow \ssa{c} ; \delta' ; \Sigma' }
\and
\inferrule{
  { \delta ; \bexpr \hookrightarrow \ssa{\bexpr} }
  \and
  { \Sigma; \delta ; c_1 \hookrightarrow \ssa{c_1} ; \delta_1;\Sigma_1 }
  \and
  {\Sigma_1; \delta ; c_2 \hookrightarrow \ssa{c_2} ; \delta_2 ; \Sigma_2 }
  \\
  {[\bar{x}, \ssa{\bar{{x_1}}, \bar{{x_2}}}] = \delta_1 \bowtie \delta_2  }
  \and
   {[\bar{y}, \ssa{\bar{{y_1}}, \bar{{y_2}}}] = \delta \bowtie \delta_1 / \bar{x} }
  \and
   {[\bar{z}, \ssa{\bar{{z_1}}, \bar{{z_2}}}] = \delta \bowtie \delta_2 / \bar{x} }
  \\
  { \delta' =\delta[\bar{x} \mapsto \ssa{\bar{{x}}'} ][\bar{y} \mapsto \ssa{\bar{{y}}'} ][\bar{z} \mapsto \ssa{\bar{{z}}'} ]}
  \and 
  {\ssa{\bar{{x}}', \bar{y}', \bar{z}'} \ fresh(\Sigma_2)
  }
  \and{\Sigma' = \Sigma_2 \cup \{ \ssa{ \bar{x}', \bar{y}', \bar{z}' } \} }
}{
 \Sigma; \delta ; [\eif(\bexpr, c_1, c_2)]^l  \hookrightarrow [\ssa{ \eif(\bexpr, [\bar{{x}}', \bar{{x_1}}, \bar{{x_2}}] ,[\bar{{y}}', \bar{{y_1}}, \bar{{y_2}}] ,[\bar{{z}}', \bar{{z_1}}, \bar{{z_2}}] , {c_1}, {c_2})}]^l; \delta';\Sigma'
}~{\textbf{S-IF}}
%
\and
%
\inferrule{
 {\delta ; \expr \hookrightarrow \ssa{\expr} }
 \and
 {\delta' = \delta[x \mapsto \ssa{{x}} ]}
 \and{ \ssa{x} \ fresh(\Sigma) }
 \and { \Sigma' = \Sigma \cup \{ \ssa{x} \} }
}{
 \Sigma;\delta ; [\assign x \expr]^{l} \hookrightarrow [\ssa{\assign {{x}}{ \expr}}]^{l} ; \delta'; \Sigma'
}~{\textbf{S-ASSN}}
%
\and
%
\inferrule{
 {\delta ; \query \hookrightarrow \ssa{\query}}
 \and
 {\delta ; \qexpr \hookrightarrow \ssa{\qexpr}}
 \and
 {\delta' = \delta[x \mapsto \ssa{x} ]}
 \and{ \ssa{x} \ fresh(\Sigma) }
  \and { \Sigma' = \Sigma \cup \{ \ssa{x} \} }
}{
 \Sigma;\delta ; [\assign{x}{\query(\qexpr)}]^{l} \hookrightarrow 
 [\assign {\ssa{x}}{ \ssa{\query(\qexpr)}}]^{l} ; \delta';\Sigma'
}~{\textbf{S-QUERY}}
%
%%
\and
%
%
\and
%
\inferrule{
    { \Sigma; \delta ; c \hookrightarrow \ssa{c_1} ; \delta_1; \Sigma_1 }
     \and
    { [ \bar{x}, \ssa{\bar{{x_1}}}, \ssa{\bar{{x_2}}} ] = \delta \bowtie \delta_1 }
    \\
     {
     \ssa{\bar{{x}}'} \ fresh(\Sigma_1 )}
    \and {\delta' = \delta[\bar{x} \mapsto \ssa{\bar{{x}}'}]}
    \and 
     {\delta' ; \bexpr \hookrightarrow \ssa{\bexpr} }
     \and
    {\ssa{c' = c_1[\bar{x}'/ \bar{x_1}]   } }
  }{ 
  \Sigma; \delta ;  \ewhile ~ [\bexpr]^{l} ~ \edo ~ c 
  \hookrightarrow 
  \ssa{\ewhile ~ [\bexpr]^{l}, 0, [\bar{{x}}', \bar{{x_1}}, \bar{{x_2}}] ~ \edo ~ {c} } ; \delta'; \Sigma_1 \cup \{\ssa{\bar{x}'}  \}
}~{\textbf{S-WHILE}
}
\and
%
\inferrule{
 {\Sigma;\delta ; c_1 \hookrightarrow \ssa{c_1} ; \delta_1; \Sigma_1} 
 \and
 {\Sigma_1; \delta_1 ; c_2 \hookrightarrow \ssa{c_2} ; \delta'; \Sigma'} 
}{
\Sigma;\delta ; c_1 ; c_2 \hookrightarrow \ssa{c_1} ; \ssa{c_2} \ ; \delta';\Sigma'
}~{\textbf{S-SEQ}}
\end{mathpar}

\paragraph{Concrete examples.}
\[
c_1 \triangleq
\begin{array}{l}
     \left[x \leftarrow \query(1) \right]^1; \\
     \eif \; (x ==0)^{2} \; \\
    \ethen \; \left[y \leftarrow \query(2) \right]^3\; \\
    \eelse \; \left[y \leftarrow 0 \right]^4 ; \\
    \eif \; (x == 1 )^5\; \\
    \ethen \; \left[ y \leftarrow 0 \right]^6\; \\
    \eelse \; \left[y \leftarrow \query(2) \right]^7\\
\end{array}
%
%
\hspace{20pt} \hookrightarrow  \hspace{20pt}
%
\begin{array}{l}
     \left[ \ssa{x_1} \leftarrow \query(1) \right]^1; \\
     \eif \; (\ssa{x_1 ==0})^{2}, [\ssa{ y_3, y_1,y_2  }],[],[]  \; \\
    \ethen \; \left[ \ssa{y_1} \leftarrow \query(2) \right]^3\; \\
    \eelse \; \left[\ssa{y_2 \leftarrow 0 } \right]^4 ; \\
    \eif \; (\ssa{x_1 == 1} )^{5} , [\ssa{ y_6, y_4, y_5 } ] \; \\
    \ethen \; \left[ \ssa{y_4 \leftarrow 0} \right]^6\; \\
    \eelse \; \left[\ssa{y_5} \leftarrow \query(2) \right]^7\\
\end{array}
\]
\[
c_2 \triangleq
\begin{array}{l}
   \left[ x \leftarrow \query(1) \right]^1; \\
   \left[y \leftarrow \query(2) \right]^2 ; \\
    \eif \;( x + y == 5 )^3\; \\
    \ethen \;\left[ z \leftarrow \query(3)\right]^4 \; \\
    \eelse \;\left[ \ssa{\eskip}\right]^5 ; \\
   \left[ w \leftarrow q_4 \right]^6; \\
\end{array}
\hspace{20pt} \hookrightarrow \hspace{20pt}
%
\begin{array}{l}
   \left[ \ssa{ x_1 } \leftarrow \query(1) \right]^1; \\
   \left[\ssa{ y_1} \leftarrow \query(2) \right]^2 ; \\
    \eif \;( \ssa{ x_1 + y_1 == 5} )^3, [ ],[] ,[ ]\; \\
    \ethen \;\left[ \ssa{ z_1 }
    \leftarrow \query(3)\right]^4 \; \\
    \eelse \;\left[ \eskip\right]^5 ; \\
   \left[ \ssa{ w_1} \leftarrow \query(4) \right]^6; \\
\end{array}
\]

{
\[
c_3 \triangleq
\begin{array}{l}
     \left[x \leftarrow \query(1) \right]^1 ; \\
     \left[i \leftarrow 0 \right]^2 ; \\
    \ewhile ~  [i < 100]^3 ~ \edo
    \\
    ~ \Big( 
    \left[z \leftarrow \query(3) \right]^4; \\
    \left[x \leftarrow z + x \right]^5; \\
    \left[i \leftarrow i + 1 \right]^6
    \Big) ;
\end{array}
%
\hspace{20pt} \hookrightarrow \hspace{20pt} 
%
\begin{array}{l}
     \left[\ssa{x_1} \leftarrow \query(1) \right]^1 ; \\
     \left[\ssa{i_1} \leftarrow 0 \right]^2 ; \\
    \ewhile
    ~ [\ssa{i_1} < 100]^3, 0,
    ~\ssa{[ x_3,x_1 ,x_2 ], [i_3, i_1, i_2] }~
    \edo \\
    ~ \Big( 
    \left[\ssa{z_1} \leftarrow \query(3) \right]^4; \\
    \left[ \ssa{x_2} \leftarrow \ssa{z_1 + x_3} \right]^5; \\
    \left[\ssa{i_2} \leftarrow \ssa{i_3} + 1 \right]^6
    \Big) ;
\end{array}
\]
}
%
\begin{figure}
   \[
 \begin{array}{lll}
    | \ewhile ~ [ \sbexpr ]^{l}, n, [\bar{\ssa{x}}, \bar{\ssa{x_1}}, \bar{\ssa{x_2}}] 
    ~ \edo ~  \ssa{c}|  
    &=& \ewhile ~ [|\sbexpr|]^{l},  ~ \edo ~ |\ssa{c}| 
	\\
    |\ssa{c_1 ; c_2}|  &=& |\ssa{c_1}| ; |\ssa{c_2}| 
    \\
    |[\eif(\sbexpr,
    [ \bar{\ssa{x}}, \bar{\ssa{x_1}}, \bar{\ssa{x_2}}] ,
    [ \bar{\ssa{y}}, \bar{\ssa{y_1}}, \bar{\ssa{y_2}}] , 
    [\bar{\ssa{z}}, \bar{\ssa{z_1}}, \bar{\ssa{z_2}}] , 
    \ssa{ c_1, c_2)}]^{l}|  
    &=&
    [\eif(|\sbexpr|, |\ssa{ c_1}|, |\ssa{c_2}|)]^{l}
    \\
    | [\assign {\ssa{x}}{\ssa{\expr}}]^{l}| & = & [\assign {|\ssa{x}|}{|\ssa{\expr}|} ]^{l}
    \\
    | [\assign {\ssa{x}}{\query(\ssa{\qexpr})} ]^{l} | & = & [\assign {|\ssa{x}|}{|\query(\ssa{\qexpr})|}]^{l}
    \\
    |\ssa{x}_i| & = & x 
    \\
    |n | & = & n 
    \\
    | \saexpr_1 \oplus_{a} \saexpr_2 | & = &  |\ssa{\aexpr_1}| \oplus_a |\ssa{\aexpr_2}| \\
    | \sbexpr_1 \oplus_{b} \sbexpr_2 | & = &  |\sbexpr_1| \oplus_b |\sbexpr_2|
 \end{array}
\]
    \caption{The Erasure of SSA}
    \label{fig:ssa_erasure-while}
\end{figure}
%
%
%
% 
%
\subsection{The Soundness of the Transformation}
In this subsection, we show our transformation from the {\tt While} language to its SSA form is sound with respect to the adaptivity. 
To be specific, a transformed program $\ssa{c}$ starting with appropriate configuration, generates the same trace as the program before the transformation $c$, in its corresponding configuration.
%
%
\begin{defn}[\todo{Written Variables}].
\\
We defined the assigned variables in the while language program $c$ as $\avars{c}$,the assigned variables in the ssa-form program $\ssa{c}$ as $\avarssa{\ssa{c}}$ defined as follows.
\[
\begin{array}{lll}
    \avars{\assign{x}{\expr}} & =& \{ x \} \\
    \avars{\assign{x}{\query(\qexpr)}} & =& \{ x \} \\
    \avars{c_1; c_2}  & = & \avars{c_1} \cup \avars{c_2} \\
    \avars{\ewhile ~ \bexpr ~ \edo ~ c} &= &  \avars{c} \\
    \avars{\eif(\bexpr, c_1, c_2)} & =&  \avars{c_1} \cup \avars{c_2}\\
\end{array} 
\]
%
\[
\begin{array}{lll}
    \avarssa{\ssa{\assign{x}{\expr}}} & =& \{ \ssa{x} \}
    \\
    \avarssa{\ssa{\assign{x}{\query(\ssa{\qexpr})}}} & =& \{ \ssa{x} \}
    \\
    \avarssa{\ssa{c_1; c_2 } }  & = & \avarssa{\ssa{c}_1} \cup \avarssa{\ssa{c}_2}
    \\
    \avarssa{\ewhile ~ \ssa{\bexpr, n, [\bar{x}, \bar{x_1}, \bar{x_2}] ~ \edo ~ \ssa{c}}}
    & = &  
    \{\ssa{\bar{x}}\} \cup \avarssa{\ssa{c}} 
    \\
    \avarssa{\eif(\ssa{\bexpr,[\bar{x}, \bar{x_1}, \bar{x_2}],[\bar{y}, \bar{y_1}, \bar{y_2}],[\bar{z}, \bar{z_1}, \bar{z_2}], c_1, c_2} )} 
    & =&  \{ \ssa{\bar{x}},\ssa{\bar{y}} , \ssa{\bar{z}} \} 
    \cup \avarssa{\ssa{c_1}} \cup \avarssa{\ssa{c_2}}\\
\end{array}
\]
\end{defn}
\begin{defn}[\todo{Read Variables}].
\\
{
The variables read in the while language programs $c$ as $\vars{c}$, variables used in ssa-form program $\ssa{c}$ : 
}
\[
\begin{array}{lll}
    \vars{\assign{x}{\expr}} & =& \vars{\expr}  \\
    \vars{\assign{x}{\query(\qexpr)}} & =&\{  \} \\
    \vars{ c_1; c_2  }  & = & \vars{c_1} \cup \vars{c_2} \\
    \vars{  \eloop ~ \aexpr ~ \edo ~ c  } &= &\vars{\aexpr} \cup \vars{c} \\
    \vars{\eif(\bexpr, c_1, c_2)} & =& \vars{\bexpr} \cup \vars{c_1} \cup \vars{c_2}\\
\end{array}
\]
\[
\begin{array}{lll}
    \varssa{\ssa{\assign{x}{\expr}}} & =& \varssa{\ssa{\expr}}  \\
    \varssa{\ssa{\assign{x}{\query(\qexpr)}}} & =& \{  \} \\
    \varssa{ \ssa{c_1; c_2}  }  & = & \varssa{\ssa{c}_1} \cup \varssa{\ssa{c}_2} \\
    % \varssa{  \eloop ~ \ssa{\aexpr, n, [\bar{x}, \bar{x_1}, \bar{x_2}] ~ \edo ~ c} } &= &\varssa{\ssa{\aexpr}} \cup \varssa{\ssa{c}}  \cup \{ \ssa{\bar{x_1}} \} \cup \{ \ssa{\bar{x_2}} \}\\
    {\varssa{  \ewhile ~ \ssa{\bexpr, n, [\bar{x}, \bar{x_1}, \bar{x_2}] ~ \edo ~ c} }} 
    &= &
    \varssa{\ssa{\bexpr}} \cup \varssa{\ssa{c}}  \cup \{ \ssa{\bar{x_1}} \} \cup \{ \ssa{\bar{x_2}} \}\\
    \varssa{\eif(\ssa{\bexpr,[\bar{x}, \bar{x_1}, \bar{x_2}], [\bar{y}, \bar{y_1}, \bar{y_2}],[\bar{z}, \bar{z_1}, \bar{z_2}], c_1, c_2} )} & =& \varssa{\ssa{\bexpr}} \cup \varssa{\ssa{c_1}} \cup \varssa{\ssa{c_2}} \cup \{\ssa{\bar{x_1}, \bar{x_2},\bar{y_1}, \bar{y_2},\bar{z_1}, \bar{z_2} }\}  \\
\end{array}
\]
\end{defn}
%
\begin{defn}[\todo{Necessary Variables}].
\\
{
We call the variables needed to be assigned before executing the program $c$ as necessary variables $\fv{c}$. Its ssa form is : $\fvssa{\ssa{c}}$.
}  
 \[
 \begin{array}{lll}
     \fvars{\assign{x}{\expr} }  & = & \vars{\expr}  \\
     \fvars{\assign{x}{\query(\qexpr)} }  & = & \{ \}  \\
     {\fvars{  \ewhile ~ \bexpr ~ \edo ~ c  } }&= & \vars{\bexpr} \cup \fvars{c} \\
     \fvars{\eif(\bexpr, c_1, c_2)} & =& \vars{\bexpr} \cup \fvars{c_1} \cup \fvars{c_2}  \\
      \fvars{c_1 ; c_2} & = & \fvars{c_1} \cup ( \fvars{c_2} - \avars{c_1})
 \end{array}
 \]
 \[
 \begin{array}{lll}
     \fvssa{\ssa{\assign{x}{\expr}} }  & = & \varssa{\ssa{\expr}}  \\
     \fvssa{ \ssa{ \assign{x}{\query(\qexpr)}} }  & = & \{ \}  \\
     {\fvssa{  \ewhile ~ \ssa{\bexpr, n, [\bar{x}, \bar{x_1}, \bar{x_2}] ~ \edo ~ c} } }
     &= & 
     \varssa{\ssa{\bexpr}} \cup \fvssa{\ssa{c}}[\ssa{ \bar{x_1}} / \ssa{\bar{x}}]\\
     \fvssa{\eif(\ssa{\bexpr,[\bar{x}, \bar{x_1}, \bar{x_2}],[\bar{y}, \bar{y_1}, \bar{y_2}],[\bar{z}, \bar{z_1}, \bar{z_2}], c_1, c_2} )} & =& \varssa{\ssa{\bexpr}} \cup \fvssa{\ssa{c_1}} \cup \fvssa{\ssa{c_2}}  \\
      \fvssa{\ssa{c_1 ; c_2}} & = & \fvssa{\ssa{c_1}} \cup ( \fvssa{\ssa{c_2}} - \avarssa{\ssa{c_1}})
 \end{array}
 \]
%
\end{defn}
%
The Lemma~\ref{lem:fv} and \ref{lem:same_value} proved the preserving properties for variables and values during the transformation.
%
\begin{lem}[Variable Preserving]
\label{lem:fv}
If $\Sigma;\delta ; c \hookrightarrow \ssa{c} ; \delta';\Sigma' $, $\fvssa{\ssa{c}} = \delta(\fvars{c})$. 
\end{lem}
\begin{proof}
 By induction on the transformation.
 \begin{itemize}
    \caseL{Case $\inferrule{
  { \delta ; \bexpr \hookrightarrow \ssa{\bexpr} }
  \and
  { \delta ; c_1 \hookrightarrow \ssa{c_1} ; \delta_1 }
  \and
  {\delta ; c_2 \hookrightarrow \ssa{c_2} ; \delta_2 }
  \\
  {[\bar{x}, \ssa{\bar{{x_1}}, \bar{{x_2}}}] = \delta_1 \bowtie \delta_2  }
  \and
   {[\bar{y}, \ssa{\bar{{y_1}}, \bar{{y_2}}}] = \delta \bowtie \delta_1 / \bar{x} }
  \and
   {[\bar{z}, \ssa{\bar{{z_1}}, \bar{{z_2}}}] = \delta \bowtie \delta_2 / \bar{x} }
  \\
  { \delta' =\delta[\bar{x} \mapsto \ssa{\bar{{x}}'} ]}
  \and 
  {\ssa{\bar{{x}}', \bar{y}', \bar{z}'} \ fresh }
}{
 \delta ; [\eif(\bexpr, c_1, c_2)]^l  \hookrightarrow [\ssa{ \eif(\bexpr, [\bar{{x}}', \bar{{x_1}}, \bar{{x_2}}] ,[\bar{{y}}', \bar{{y_1}}, \bar{{y_2}}] ,[\bar{{z}}', \bar{{z_1}}, \bar{{z_2}}] , {c_1}, {c_2})}]^l; \delta'
}~{\textbf{S-IF}} $}
From the definition of $\fvssa{[\eif(\sbexpr, [\bar{\ssa{x'}}, \bar{\ssa{x_1}}, \bar{\ssa{x_2}}] , \ssa{c_1}, \ssa{c_2})]^l} = \varssa{\ssa{\bexpr}} \cup \fvssa{\ssa{c_1}} \cup \fvssa{\ssa{c_2}}$. We want to show: \[\varssa{\ssa{\bexpr}}) \cup \fvssa{\ssa{c_1}} \cup \fvssa{\ssa{c_2}} = \delta( \vars{\bexpr}) \cup \delta(\fv{c_1}) \cup \delta(\fv{c_2}  )\]
By induction hypothosis on the second and third premise, we know that : $\fvssa{\ssa{c_1}} = \delta(\fv{c_1}) $ and $\fvssa{\ssa{c_2}} = \delta(\fv{c_2}) $.  We still need to show that: 
\[
  \varssa{\ssa{\bexpr}} = \delta(\vars{\bexpr})
\] 
From the first premise, we know that $\vars{b} \subseteq \dom(\delta)$. This is goal is proved by the rule $\textbf{S-VAR}$ on all the variables in $\bexpr$.\\
{\caseL{Case
$\inferrule{
    { \Sigma; \delta ; c \hookrightarrow \ssa{c_1} ; \delta_1; \Sigma_1 }
     \and
    { [ \bar{x}, \ssa{\bar{{x_1}}}, \ssa{\bar{{x_2}}} ] = \delta \bowtie \delta_1 }
    \\
     {\ssa{\bar{{x}}'} \ fresh(\Sigma_1 )}
    \and {\delta' = \delta[\bar{x} \mapsto \ssa{\bar{{x}}'}]}
    \and 
     {\delta' ; \bexpr \hookrightarrow \ssa{\bexpr} }
     \and
    {\ssa{c' = c_1[\bar{x}'/ \bar{x_1}]   } }
    % \and{ \delta' ; c \hookrightarrow \ssa{c'} ; \delta'' }
  }{ 
  \Sigma; \delta ;  \ewhile ~ [\bexpr]^{l} ~ \edo ~ c 
  \hookrightarrow 
  \ssa{\ewhile ~ [\bexpr]^{l}, 0, [\bar{{x}}', \bar{{x_1}}, \bar{{x_2}}] ~ \edo ~ {c} } ; \delta'; \Sigma_1 \cup \{\ssa{\bar{x}'}  \}
}~{\textbf{S-WHILE}}
$}
}
{
Unfolding the definition, we need to show:
\[\varssa{\ssa{\bexpr}} \cup \fvssa{\ssa{c'}}[\ssa{ \bar{x_1}} / \ssa{\bar{x}}] = \delta (\vars{\bexpr}) \cup \delta(\fv{c} ) \]
We can similarly show that $\varssa{\ssa{\bexpr}} = \delta(\vars{b})$ as in the if case. We still need to show that: 
\[
 \fvssa{\ssa{c_1[\bar{x}' / \bar{x_1}]}}[ \ssa{ \bar{x_1} } / \ssa{\bar{x}'}] =  \delta(\fv{c} )
\]
It is proved by induction hypothesis on $  { \Sigma; \delta ; c \hookrightarrow \ssa{c_1} ; \delta_1; \Sigma_1 }$.\\
}
%
\caseL{Case $\inferrule{
 {\Sigma;\delta ; c_1 \hookrightarrow \ssa{c_1} ; \delta_1; \Sigma_1} 
 \and
 {\Sigma_1; \delta_1 ; c_2 \hookrightarrow \ssa{c_2} ; \delta'; \Sigma'} 
}{
\Sigma;\delta ; c_1 ; c_2 \hookrightarrow \ssa{c_1} ; \ssa{c_2} \ ; \delta';\Sigma'
}~{\textbf{S-SEQ}}$}
To show:
  \[ \fvssa{\ssa{c_1}} \cup ( \fvssa{\ssa{c_2}} - \avarssa{\ssa{c_1}}) = \delta(\fv{c_1} )\cup \delta( \fv{c_2} - \avars{c_1}) \]
  By induction hypothesis on the first premise, we know that : $ \fvssa{\ssa{c_1}} = \delta(\fv{c_1} ) $, still to show: 
    \[ ( \fvssa{\ssa{c_2}} - \avarssa{\ssa{c_1}}) = \delta( \fv{c_2} - \avars{c_1})
    \]
    We know that $\delta_1 = \delta [\avars{c_1} \mapsto \avarssa{\ssa{c_1}} ]$, so by induction hypothesis, we know: $ \fvssa{\ssa{c_2}} = \delta[\avars{c_1} \mapsto \avarssa{\ssa{c_1}} ]( \fv{c_2})  = \delta(\fv{c_2}) \cup \avarssa{\ssa{c_1}} - \delta(\avars{c_1}) $.
    
    This case is proved.
 \end{itemize}
 
\end{proof}

{
We first define a good memory in the {\tt While} language $m$ or in the ssa language $\ssa{m}$ with respect to a translation environment $\delta$, denoted as $m \vDash \delta$ and $\ssa{m} \vDash \delta$ respectively. 
%
\begin{defn}[Well Defined Memory].
\begin{enumerate}
    % \item $m \vDash c \triangleq \forall x \in \fv{c}, \exists v, (x, v) \in m$.
    \item $ m \vDash \delta  \triangleq \forall x \in \dom(\delta), \exists v, (x,v) \in m$.
    % \item $\ssa{m} \vDash_{ssa} \ssa{c} \triangleq \forall \ssa{x} \in \fvssa{\ssa{c}}, \exists v, (\ssa{x}, v) \in \ssa{m}$.
    \item $ \ssa{m} \vDash_{ssa} \delta  \triangleq \forall \ssa{x} \in \codom(\delta), \exists v, (\ssa{x},v) \in \ssa{m}$.
\end{enumerate}
\end{defn}
%
The part declared in the translation environment $\delta$ in a ssa memory $\ssa{m}$ can be reverted to corresponding part of the memory $m$ with an inverse of $\delta$ as follows.
%
\begin{defn}[Inverse of Transformed memory]
 $m = \delta^{-1}(\ssa{m}) \triangleq \forall x \in \dom(\delta), (\delta(x), m(x)) \in \ssa{m} $.
\end{defn}
}
%
\begin{lem}[Value Preserving].
\label{lem:same_value}
{
Given $\delta; e \hookrightarrow \ssa{e}$,  $\forall m. m \vDash \delta. \forall \ssa{m}, \ssa{m} \vDash_{ssa} \delta \land m = \delta^{-1}(\ssa{m})$, then $\config{m, e} \to v $ and $\config{
\ssa{m}, \ssa{e}} \to {v}$.
}
\end{lem}

\begin{thm}[Soundness of transformation]
Given $\Sigma; \delta ; c \hookrightarrow \ssa{c} ; \delta';\Sigma' $, $\forall m. m \vDash \delta. \forall \ssa{m}, \ssa{m} \vDash_{ssa} \delta \land m = \delta^{-1}(\ssa{m})$, if there exist an execution of $c$ in the while language, starting with a trace $t$ and loop maps $w$, $\config{m, c, t, w} \to^{*} \config{m', \eskip, t', w' } $,  then there also exists a corresponding execution of $\ssa{c}$ in the ssa language so that 
  $\config{  {\ssa{m}}, \ssa{c}, t, w} \to^{*} \config{{  \ssa{m'}}, \eskip, t', w' } $ and $ m' = \delta'^{-1}(\ssa{m'}) $.
\end{thm}

\begin{proof}
 We assume that $q$ is the same when transformed to $\ssa{q}$, as the primitive in both languages. And the value remains the same during the transformation.  
 It is proved by induction on the transformation rules.
 \begin{itemize}
   \caseL{Case $\inferrule{
 {\Sigma;\delta ; c_1 \hookrightarrow \ssa{c_1} ; \delta_1;\Sigma_1} 
 \and
 {\Sigma_1; \delta_1 ; c_2 \hookrightarrow \ssa{c_2} ; \delta'; \Sigma'} 
}{
\Sigma;\delta ; c_1 ; c_2 \hookrightarrow \ssa{c_1} ; \ssa{c_2} \ ; \delta';\Sigma'
}~{\textbf{S-SEQ}}$}
We choose an arbitrary memory $m$ so that $m \vDash \delta$, we choose a trace $t$ and a loop maps $w$.
\[
\inferrule
{
{\config{m, c_1,  t,w} \xrightarrow{}^{*} \config{m_1, \eskip,  t_1,w_1}}
\and
{\config{m_1, c_2,  t_1,w_1} \xrightarrow{}^{*} \config{m', \eskip,  t',w'}}
}
{
\config{m, c_1; c_2,  t,w} \xrightarrow{}^{*} \config{m', \eskip, t',w'}
}
\]
 We choose the transformed memory ${\ssa{m}} $ so that  $ m =\delta^{-1}(\ssa{m})$.
 
 To show: $ \config{\ssa{ m, c_1;c_2 }, t, w } \xrightarrow{}^{*} \config{\ssa{m', \eskip}, t'. w' }$ and $ m' = \delta'^{-1} (\ssa{m'}) $.
 
 By induction hypothesis on the first premise, we have:
 \[ \config{\ssa{m, c_1}, t,w} \xrightarrow{}^{*} \config{\ssa{m_1, \eskip},t_1,w_1 } \land m_1 = \delta_1^{-1}(\ssa{m_1}) \]
  By induction hypothesis on the second premise, using the conclusion $ m_1 = \delta_1^{-1}(\ssa{m_1}) $.
  We have:
  \[
   \config{\ssa{m_1, c_2}, t_1,w_1} \xrightarrow{}^{*} \config{\ssa{m', \eskip},t',w' } \land m' = \delta'^{-1}(\ssa{m'})
  \]
  So we know that 
  \[
  \inferrule{
  { \config{\ssa{m, c_1}, t,w} \xrightarrow{}^{*} \config{\ssa{m_1, \eskip},t_1,w_1 }  }
  \and
  { \config{\ssa{m_1, c_2}, t_1,w_1} \xrightarrow{}^{*} \config{\ssa{m', \eskip},t',w' } }
  }{
  \config{\ssa{m, c_1;c_2 }, t,w} \xrightarrow{}^{*} \config{\ssa{m', \eskip}, t' , w' }
  }
  \]
 \caseL{Case $\inferrule{
 { \delta ; \expr \hookrightarrow \sexpr}
 \and
 {\delta' = \delta[x \mapsto \ssa{x} ]}
 \and{ \ssa{x} \ fresh(\Sigma) }
 \and {\Sigma' = \Sigma \cup \{x\} }
}{
 \Sigma;\delta ; [\assign x \expr]^{l} \hookrightarrow [\assign {\ssa{x}}{ \ssa{\expr}}]^{l} ; \delta';\Sigma'
}~{\textbf{S-ASSN}} $ }

 We choose an arbitrary memory $m$ so that $m \vDash \delta$, we choose a trace $t$ and a loop maps $w$, we know that the resulting trace is still $t$ from its evaluation rule $\textbf{assn}$ when we suppose $m, \expr \to v$.
 \[
 \inferrule
{
}
{
\config{m, [\assign x v]^{l},  t,w} \xrightarrow{} \config{m[v/x], [\eskip]^{l}, t,w}
}
~\textbf{assn}
 \]
 We choose the transformed memory ${\ssa{m}} $ so that  $ m =\delta^{-1}(\ssa{m})$.
 
 To show: $\config{\ssa{m}, [\assign {\ssa{x}}{ \ssa{\expr}}]^{l} , t, w} \to^{*} \config{\ssa{m'}, \eskip, t, w} $ and $ m' = \delta'^{-1}(\ssa{m'}) $.
 
 From the rule \textbf{ssa-assn}, we assume $\ssa{m}, \ssa{\expr} \to \ssa{v}$, we know that 
 \[
 \inferrule
{
}
{
\config{\ssa{ m, [\assign x v]^{l}},  t,w } \xrightarrow{} \config{\ssa{m[x \mapsto v], [\eskip]^{l}}, t,w }
}
~\textbf{ssa-assn}
 \]
 We also know that $\delta'= \delta[x \mapsto \ssa{x}]$ and $m = \delta^{-1}(\ssa{m})$, $m'= m[v/x]$. We can show that $ m[v/x] = \delta[x \mapsto \ssa{x}]^{-1}(\ssa{m}[\ssa{x} \mapsto v]) $.
 
\caseL{Case $\inferrule{
 {\delta ; q \hookrightarrow \ssa{q}}
 \and
 {\delta ; \expr \hookrightarrow \ssa{\expr}}
 \and
 {\delta' = \delta[x \mapsto \ssa{x} ]}
 \and{ \ssa{x} \ fresh(\Sigma) }
 \and{ \Sigma' = \Sigma \cup \{x\} }
}{
 \Sigma;\delta ; [\assign{x}{\query(\qexpr)}]^{l} \hookrightarrow [\assign {\ssa{x}}{ \ssa{\query(\qexpr)}}]^{l} ; \delta'
}~{\textbf{S-QUERY}}$} 
We choose an arbitrary memory $m$ so that $m \vDash \delta$, we choose a trace $t$ and a loop maps $w$, we know that when we suppose $\config{m, \expr} \to v$.
 \[
\inferrule
{
\query(v)(D) = \qval 
}
{
\config{m, [\assign{x}{\query(v)}]^l, t, w} \xrightarrow{} \config{m[ \qval/ x], \eskip,  t \mathrel{++} [\query(v),l,w )],w }
}
~\textbf{query}
 \]
 We choose the transformed memory ${\ssa{m}} $ so that  $ m =\delta^{-1}(\ssa{m})$.
 
 To show: $\config{\ssa{m}, [\assign {\ssa{x}}{ \ssa{\query(\qexpr)}}]^{l} , t, w} \to^{*} \config{\ssa{m'}, \eskip, t, w} $ and $ m' = \delta'^{-1}(\ssa{m'}) $.
 
 From the rule \textbf{ssa-query}, we know that 
 \[
 \inferrule
{
\ssa{\query(v)(D) = \qval} 
}
{
\config{ \ssa{ m, [\assign{\ssa{x}}{\ssa{\query(\qexpr)}}]^l}, t, w} \xrightarrow{} \config{\ssa{  m[  x \mapsto v], \eskip,}  t \mathrel{++} [(q^{(l,w )},v)],w }
}
~\textbf{ssa-query}
 \]
 We also know that $\delta'= \delta[x \mapsto \ssa{x}]$ and $m = \delta^{-1}(\ssa{m})$, $m'= m[v/x]$. We can show that $ m[v/x] = \delta[x \mapsto \ssa{x}]^{-1}(\ssa{m}[\ssa{x} \mapsto v]) $.

  \caseL{Case $\inferrule{
  { \delta ; \bexpr \hookrightarrow \ssa{\bexpr} }
  \and
  { \Sigma; \delta ; c_1 \hookrightarrow \ssa{c_1} ; \delta_1;\Sigma_1 }
  \and
  {\Sigma_1; \delta ; c_2 \hookrightarrow \ssa{c_2} ; \delta_2 ; \Sigma_2 }
  \\
  {[\bar{x}, \ssa{\bar{{x_1}}, \bar{{x_2}}}] = \delta_1 \bowtie \delta_2  }
  \and
   {[\bar{y}, \ssa{\bar{{y_1}}, \bar{{y_2}}}] = \delta \bowtie \delta_1 / \bar{x} }
  \and
   {[\bar{z}, \ssa{\bar{{z_1}}, \bar{{z_2}}}] = \delta \bowtie \delta_2 / \bar{x} }
  \\
  { \delta' =\delta[\bar{x} \mapsto \ssa{\bar{{x}}'} ][\bar{y} \mapsto \ssa{\bar{{y}}'} ][\bar{z} \mapsto \ssa{\bar{{z}}'} ]}
  \and 
  {\ssa{\bar{{x}}', \bar{y}', \bar{z}'} \ fresh(\Sigma_2)
  }
  \and{\Sigma' = \Sigma_2 \cup \{ \ssa{ \bar{x}', \bar{y}', \bar{z}' } \} }
}{
 \Sigma; \delta ; [\eif(\bexpr, c_1, c_2)]^l  \hookrightarrow [\ssa{ \eif(\bexpr, [\bar{{x}}', \bar{{x_1}}, \bar{{x_2}}] ,[\bar{{y}}', \bar{{y_1}}, \bar{{y_2}}] ,[\bar{{z}}', \bar{{z_1}}, \bar{{z_2}}] , {c_1}, {c_2})}]^l; \delta';\Sigma'
}~{\textbf{S-IF}}$}
We choose an arbitrary memory $m$ so that $m \vDash \delta$, we choose a trace $t$ and a loop maps $w$.
There are two possible evaluation rules depending on the the condition $b$, we choose the case when $b = \etrue$, we know there is an execution in ssa language so that $\ssa{\bexpr} = \etrue$, we use the rule $\textbf{if-t}$.  
 \[\inferrule
{
}
{
\config{m, [\eif(\etrue, c_1, c_2)]^{l},t,w} 
\xrightarrow{} \config{m, c_1,  t,w} \to^{*} \config{m', \eskip, t', w'}
}
\]
 We choose the transformed memory ${\ssa{m}} $ so that  $ m =\delta^{-1}(\ssa{m})$.
 
 To show: $\config{\ssa{m}, [\eif(\etrue, [\bar{\ssa{x}}', \bar{\ssa{x_1}}, \bar{\ssa{x_2}}] ,[\bar{\ssa{y}}', \bar{\ssa{y_1}}, \bar{\ssa{y_2}}] ,[\bar{\ssa{z}}', \bar{\ssa{z_1}}, \bar{\ssa{z_2}}] , c_1, c_2)]^{l}, t, w} \to^{*} \config{\ssa{m'}, \eskip, t', w'} $ and $ m' = \delta'^{-1}(\ssa{m'}) $.

We use the corresponding rule $\textbf{SSA-IF-T}$.  
\[
\inferrule
{
}
{
\config{\ssa{ { m} , [\eif(\etrue, [\bar{\ssa{x}}', \bar{\ssa{x_1}}, \bar{\ssa{x_2}}] , [\bar{\ssa{y}}', \bar{\ssa{y_1}}, \bar{\ssa{y_2}}] ,[\bar{\ssa{z}}', \bar{\ssa{z_1}}, \bar{\ssa{z_2}}] , \ssa{c_1, c_2})]^{l}},t,w} 
\xrightarrow{} \\ \config{\ssa{ m, c_1}; \eifvar(\ssa{\bar{x}', \bar{x_1}});\eifvar(\ssa{\bar{y}', \bar{y_2}});\eifvar(\ssa{\bar{z}', \bar{z_1}}),  t,w } 
}~\textbf{ssa-if-t}
\]
By induction hypothesis on $ \Sigma;\delta ; c_1 \hookrightarrow  \ssa{c_1}; \delta_1;\Sigma_1$, and we know that $\config{m, c_1,  t,w} \to^{*} \config{m', \eskip, t', w'} $, from our assumption that $ m =\delta^{-1}(\ssa{m})$, we know that 
\[\config{\ssa{ { m}, c_1},  t,w} \to^{*} \config{ \ssa{ { m_1 }, \eskip,} t', w' } \land m' = \delta_1^{-1}(\ssa{m_1}) \]
and we then have:
\[
\inferrule
{
  \config{\ssa{ { m}, c_1},  t,w} \to^{*} \config{ \ssa{ { m_1 }, \eskip,} t', w' }
}
{
 \config{\ssa{  m, c_1;} \eifvar(\ssa{\bar{x}', \bar{x_1})};\eifvar(\ssa{\bar{y}', \bar{y_1})};\eifvar(\ssa{\bar{z}', \bar{z_1})},  t,w  }  \to^{*}
 \config{\ssa{ { m_1 [ \bar{x}' \mapsto {m_1}(\bar{x_1}),\bar{y}' \mapsto {m_1}(\bar{y_2}),\bar{z}' \mapsto {m_1}(\bar{z_1}) ] }, \eskip}, t', w'  }
}
\]
Now, we want to show that $ m' = \delta[\bar{x} \mapsto \ssa{\bar{x}'},\bar{y} \mapsto \ssa{\bar{y}'},\bar{z} \mapsto \ssa{\bar{z}'} ]^{-1}(\ssa{ m_1 [ \bar{x}' \mapsto {m_1}(\bar{x_1}),\bar{y}' \mapsto {m_1}(\bar{y_2}),\bar{z}' \mapsto {m_1}(\bar{z_1}) ] }) $.

Unfold the definition, we want to show that $$\forall x  \in ( \dom(\delta) \cup \bar{x} \cup \bar{y} \cup \bar{z} ), (\delta[\bar{x} \mapsto \ssa{\bar{x}'},\bar{y} \mapsto \ssa{\bar{y}'},\bar{z} \mapsto \ssa{\bar{z}'} ](x), m'(x)) \in \ssa{m_1 [ \bar{x}' \mapsto {m_1}(\bar{x_1}),\bar{y}' \mapsto {m_1}(\bar{y_2}),\bar{z}' \mapsto {m_1}(\bar{z_1}) ] } .$$
\begin{enumerate}
    \item For variable $x$ in $\bar{x}$, we can find a corresponding ssa variable $\ssa{x} \in \ssa{\bar{x}'}$, so that $( \ssa{x}, m'(x) ) \in \ssa{ m_1 [\bar{x}' \mapsto m_1(\bar{x_1})] } $. It is because we know $[x \mapsto \ssa{x_1}]$ for certain $\ssa{x_1} \in \ssa{\bar{x_1}}$ in $\delta_1$, then by unfolding  $m' = \delta_1^{-1}(\ssa{m_1})$ and $\ssa{\bar{x_1}} \in \codom(\delta_1)$, we know $(\ssa{x_1}, m'(x)) \in \ssa{m_1}$ so that $m'(x) = \ssa{m_1}(\ssa{x_1})$.
    \item For variable $y \in \bar{y}$, we know that $y \in \dom(\delta_1)$, then $[ y \mapsto \ssa{y_2} ]$ for certain $\ssa{y_2} \in \ssa{\bar{y_2}}$ in $\delta_1$.  So we know that $(\delta_1(y), m'(y) ) \in \ssa{m_1}$, and then $m'(y) = \ssa{m_1(y_2)}$. We can show $(\ssa{y}, m'(y)) \in \ssa{m_1[\bar{y}' \mapsto m_1(\bar{y_2})]}$.
    \item For variable $z \in \bar{z}$, we know that $z \not\in \dom(\delta_1)$ by the definition (otherwise $z$ will appear in $\bar{x}$), then $[ z \mapsto \ssa{z_1} ]$ for certain $ \ssa{z_1} \in \ssa{\bar{z_1}}$ in $\delta$.  We know $(\delta(z), m(z)) \in \ssa{m}$ from our assumption, so we have $ m(z) = \ssa{m(z_1)}$. Because $z$ is not modified in $c_1$, so that $m(z) = m'(z)$. Also $\ssa{m}$ will not shrink during execution and $\ssa{z_1}$ will not be written in $\ssa{c_1}$, so $(\ssa{z_1}, m'(z)) \in \ssa{m_1}$. Then we can show that $ (\ssa{z}, m'(z) ) \in \ssa{m_1[\bar{z}' \mapsto m_1(\bar{z_1})] }$.
    \item For variable $k \in \dom(\delta)- \bar{x} - \bar{y}-\bar{z}$. From our assumption $ m = \delta^{-1}(\ssa{m})$, we can show $(\delta(k), m(k) ) \in \ssa{m}$. We know that $k$ is not written in either branch from our definition, so $(\delta(k), m'(k) ) \in \ssa{m_1} $ .
\end{enumerate}

{
\caseL{
	Case
	$
	\inferrule{
    { \Sigma; \delta ; c \hookrightarrow \ssa{c_1} ; \delta_1; \Sigma_1 }
     \and
    { [ \bar{x}, \ssa{\bar{{x_1}}}, \ssa{\bar{{x_2}}} ] = \delta \bowtie \delta_1 }
    \\
     {\ssa{\bar{{x}}'} \ fresh(\Sigma_1 )}
    \and {\delta' = \delta[\bar{x} \mapsto \ssa{\bar{{x}}'}]}
    \and 
     {\delta' ; \bexpr \hookrightarrow \ssa{\bexpr} }
     \and
    {\ssa{c' = c_1[\bar{x}'/ \bar{x_1}]   } }
    % \and{ \delta' ; c \hookrightarrow \ssa{c'} ; \delta'' }
  }{ 
  \Sigma; \delta ;  \ewhile ~ [\bexpr]^{l} ~ \edo ~ c 
  \hookrightarrow 
  \ssa{\ewhile ~ [\bexpr]^{l}, 0, [\bar{{x}}', \bar{{x_1}}, \bar{{x_2}}] ~ \edo ~ {c} } ; \delta'; \Sigma_1 \cup \{\ssa{\bar{x}'}  \}
}~{\textbf{S-WHILE}}
$
}
}
\\
{
We choose an arbitrary memory $m$ so that $m \vDash \delta$, we choose a trace $t$ and a loop maps $w$. Suppose $ \config{m ,a} \to v_N $. There are two cases, when $v_N=0$, the loop body is not executed so we can easily show that the trace is not modified.
%
When the while loop is still running ($v_N > 0$), we have the following evaluation in the while language:
\[
\inferrule
{
 \empty
}
{
\config{
m, \ewhile ~ [b]^{l} ~ \edo [c]^{l + 1},  t, w 
}
\xrightarrow{} \config{m, c ; 
\eif_w (b, c ; 
\ewhile ~ [b]^{l} ~ \edo [c]^{l + 1},  \eskip),
t, w }
}
~\textbf{while-b}
\]
which follows by the following evaluation:
\[
	\inferrule
{
 m, b \xrightarrow{} b'
}
{
\config{m, \eif_w (b, c,  \eskip) ,  t, w }
\xrightarrow{} \config{m, 
 \eif_w (b', c,  \eskip), t, w }
}
~\textbf{ifw-b}
\]
In the corresponding ssa-form language, we have the corresponding evaluation in the same way by assuming 
$m = \delta^{-1}(\ssa{m})$.
%
\[
	\inferrule
{
 {n = 0 \rightarrow i = 1 }
 \and
 {n > 0 \rightarrow i = 2 }
}
{
\config{
\ssa{m},  
\ssa{\ewhile ~ [\bexpr]^{l}, n, 
[\bar{{x}}', \bar{{x_1}}, \bar{{x_2}}] 
~ \edo ~ {c} 
},  t, w 
}
\xrightarrow{} \\ 
\config{
\ssa{m},
\eif_w 
(\ssa{b[\bar{x_i}/\bar{x'}], [\bar{{x}}', \bar{{x_1}}, \bar{{x_2}}], n,  c[\bar{x_i}/\bar{x'}] }; 
\ssa{
\ewhile ~ [b]^{l}, n+1, 
[\bar{{x}}', \bar{{x_1}}, \bar{{x_2}}]  
~ \edo ~ c} ,  \eskip),
t, w
}
}
~\textbf{ssa-while-b}
\]
This evaluation is followed by the following evaluation:
\[
	\inferrule
{
 \ssa{m, b \xrightarrow{} b'}
}
{
\config{\ssa{m, \eif_w (b, [\bar{{x}}', \bar{{x_1}}, \bar{{x_2}}] , n,  c_1,  c_2)} ,  t, w }
\xrightarrow{} \config{\ssa{ m, 
 \eif_w (b', [\bar{{x}}', \bar{{x_1}}, \bar{{x_2}}] , n , c_1 , c_2 )}, t, w }
}
~\textbf{ssa-ifw-b}
\]
%
Depending on if the counter $n$ is equal to $0$ or not, there are two possible execution paths (the variables $\ssa{\bar{x}}$ is replaced by the $\ssa{\bar{x_1}}$ or $\ssa{\bar{x_2}}$). We start from the first iteration (when $n =0$) when $v_N >0$. 
}
{
By induction hypothsis on the premise $ { \Sigma; \delta ; c \hookrightarrow \ssa{c_1} ; \delta_1; \Sigma_1 }$, we know that 
\[ \config{\ssa{{m}, c'[ \bar{x_1}/\bar{x}'  ]}, t, (w+l)  } \to^{*} \config{\ssa{{m'}, \eskip}, t'_{i}, w'  } \land m' = \delta_1^{-1}(\ssa{m'})   \]
Hence we can conclude that:
\[
  \inferrule{
   \config{\ssa{{m}, c'[ \bar{x_1}/\bar{x}'  ]}, t, (w+l) }  \to^{*} \config{\ssa{{m'}, \eskip}, t'_{1}, w'  }
  }{
  \config{\ssa{ {m}, c'[ \bar{x_1}/\bar{x}'  ];  [\eloop ~ (\valr_N-1), n+1, [\bar{\ssa{x}}', \bar{\ssa{x_1}}, \bar{\ssa{x_2}}] ~  \edo ~ c' ]^{l} },  t, (w + l)  }  \to^{*} \\ \config{ \ssa{{m'}, [\eloop ~ (\valr_N-1), n+1, [\bar{\ssa{x}}', \bar{\ssa{x_1}}, \bar{\ssa{x_2}}] ~  \edo ~ c' ]^{l}}, t'_{1}, w'  } 
  }
\]
%
Then there are two cases, 
%
\begin{enumerate}
     \item  when guard in the $\eif_w$ expression evaluates to $\efalse$, the while loop terminates and exits.
     The execution in the while language is defined in the evaluation rule $\textbf{ifw-false}$ as follows.
     \[
		\inferrule
		{
		 \empty
		}
		{
		\config{\ssa{
		m, \eif_w (
		\efalse, [\bar{{x}}', \bar{{x_1}}, \bar{{x_2}}],   n, 
		c; {\ewhile ~ [b]^{l} ~ \edo ~ c},
		\eskip)
		)} ,  t, w }
		\\
		\xrightarrow{} 
		\config{\ssa{m, 
		{\eskip}; \eifvar(\bar{x'}, \bar{x_i}) }, t, (w - l) }
		}
		~\textbf{ifw-false}
	\]
%
	The corresponding ssa-form evaluation as follows:
	\[
		\inferrule
		{
		 { n = 0 \rightarrow i = 1 }
		 \and
		 {n > 0 \rightarrow i =2}
		}
		{
		\config{\ssa{
		m, \eif_w (
		\efalse, [\bar{{x}}', \bar{{x_1}}, \bar{{x_2}}],   n, 
		{  
		c; \ssa{\ewhile ~ [b]^{l}, n, [\bar{{x}}', \bar{{x_1}}, \bar{{x_2}}]  ~ \edo ~ c},
		\eskip)
		} 
		)} ,  t, w }
		\\
		\xrightarrow{} 
		\config{\ssa{m, 
		{\eskip}; \eifvar(\bar{x'}, \bar{x_i}) }, t, (w - l) }
		}
		~\textbf{ssa-ifw-false}
	\]
	We can see that both traces are not changed during the exit of the while. We need to show that $ m' = \delta^{-1} (\ssa{m'[\bar{x} \mapsto m'(\bar{x_2})]}) $. We know that $[ \bar{x} \mapsto \bar{x_2}]$ in $\delta_1$ from the definition, so we can show that for any variable $\ssa{x_2} \in \bar{x_2}$, $( \ssa{x_2}, m'(x) ) \in \ssa{m'}$. For variables $x \in {\dom(\delta) - \bar{x} } $, the variable is not modified during the execution of $c$ so that we know $m(x) = m'(x)$, and then we can show that $(\delta(x), m'(x)) \in \ssa{m'} $ because $\delta(x)$ is not written in $\ssa{c'[\bar{x_1}/ \bar{x}']}$ .
%
  	\item 
		when guard in the $\eif_w$ expression evaluates to $\etrue$, the while terminates and exits.
     The execution in the while language is defined in the evaluation rule $\textbf{ifw-true}$.
          %
     We want to show that : assuming in the $i-th$ $(i < \ssa{n})$ iteration, starting with $t_i$ and $w_i$ and $m_i = \delta_1^{-1}(\ssa{m_i})$,
     this command is evaluated according to the while language operation semantics as
     	$
		\config{m, \eif_w (\etrue, c ; \ewhile ~ [b]^{l} ~ \edo ~ c, ,  \eskip) ,  t, w }
		\xrightarrow{}^* \config{m, c 
		t, (w + l) }
 		$.
     %
     Then the corresponding ssa form evaluation as follows : 
     %
     \[ 
     \inferrule{}{
     	\config{
		\ssa{
			m, 
			{
			\eif_w (\etrue, [\bar{{x}}', \bar{{x_1}}, \bar{{x_2}}], n,  
			c; \ssa{\ewhile ~ [b]^{l}, n, [\bar{{x}}', \bar{{x_1}}, \bar{{x_2}}]  ~ \edo ~ c},
			\eskip)
			} 
		},  t, w 
		}
		\\
		\xrightarrow{} 
		\config{
		\ssa{m, 
		{
		\eif_w (\etrue, [\bar{{x}}', \bar{{x_1}}, \bar{{x_2}}], n,  
		c; \ssa{\ewhile ~ [b]^{l}, n, [\bar{{x}}', \bar{{x_1}}, \bar{{x_2}}]  ~ \edo ~ c},
		}
		}
		t, (w + l) }
		} 
     \]  
     and $m_i = \delta^{-1}(\ssa{m_i}) $.
     We then have the evaluation in the while language:
     \[
		\inferrule
		{
		 \empty
		}
		{
		\config{m, 
		\eif_w (b, 
		c ; \ewhile ~ [b]^{l} ~ \edo ~ c, 
		\eskip),
		t, w }
		\xrightarrow{} 
		\config{m, 
		c ; \ewhile ~ [b]^{l} ~ \edo ~ c,  
		t, (w + l) }
		}
		~\textbf{ifw-true}
	\]
	We then have the following evaluation:
	\[
		\inferrule
		{
		 \empty
		}
		{
		\config{
		\ssa{
		m, 
		{
		\eif_w (\etrue, [\bar{{x}}', \bar{{x_1}}, \bar{{x_2}}], n,  
		c; \ssa{\ewhile ~ [b]^{l}, n, [\bar{{x}}', \bar{{x_1}}, \bar{{x_2}}]  ~ \edo ~ c},
		\eskip)
		} 
		},  t, w 
		}
		\\
		\xrightarrow{} 
		\config{
		\ssa{m, 
		{
		\eif_w (\etrue, [\bar{{x}}', \bar{{x_1}}, \bar{{x_2}}], n,  
		c; \ssa{\ewhile ~ [b]^{l}, n, [\bar{{x}}', \bar{{x_1}}, \bar{{x_2}}]  ~ \edo ~ c},
		}
		}
		t, (w + l) }
		}
		~\textbf{ssa-ifw-true}
	\]
%
By induction hypothsis on the premise $  { \Sigma; \delta_1 ; c \hookrightarrow \ssa{c_2} ; \delta_1; \Sigma_1 }$, we know that
%
\[
\config{\ssa{{m_i}, c'[ \bar{x_2}/\bar{x}'  ]}, t_i, (w_i+l)  } \to^{*} \config{\ssa{{m_{i+1}}, \eskip}, t_{i+1}, w_{i+1}  } \land m_{i+1} = \delta_1^{-1}(\ssa{m_{i+1}})
\]
%
Hence we can conclude that:
\[
  \inferrule{
   \config{\ssa{{m_i}, c'[ \bar{x_2}/\bar{x}'  ]}, t_i, (w_i+l) }  \to^{*} \config{\ssa{{m_{i+1}}, \eskip}, t_{i+1}, w_{i+1}  }
  }{
  \config{\ssa{ {m_i}, c'[ \bar{x_2}/\bar{x}'  ];  [\eloop ~ (\valr_N-i-1), n+1, [\bar{\ssa{x}}', \bar{\ssa{x_1}}, \bar{\ssa{x_2}}] ~  \edo ~ c' ]^{l} },  t_i, (w_i + l)  }  \to^{*} \\ \config{ \ssa{{m_{i+1}}, [\eloop ~ (\valr_N-i-1), n+1, [\bar{\ssa{x}}', \bar{\ssa{x_1}}, \bar{\ssa{x_2}}] ~  \edo ~ c' ]^{l}}, t_{i+1}, w_{i+1}  } 
  }
\]
So we can show that before the exit of the loop after ($v_N= n $) iterations, we have $t_{n} = t_{n}$ and $m_{n} = \delta_1^{-1}(\ssa{m_{n}})$.
 \end{enumerate}
%
This proof is similar when it comes to the exit as in case 1. 
}
\end{itemize}
%
\end{proof}
%
\clearpage
%
\clearpage
% \begin{example}[Execution Trace of a Program with While Command].
\\
\[
\ewhile [(x > 0)]{}^0 \edo [x = x - 1]{}^1;
\]
Let $\trace_0 \in \mathbb{T}$ be the initial trace, 
without loss of generalization, let $\env(\trace_0) x = 1$.
\\
By operational semantics rules, we have following evaluation:
  \begin{mathpar}
\inferrule
  {
   \vtrace_0, (x > 0) \barrow \etrue
   \and 
   \event = ((x > 0), 0, 1, \etrue)
  }
  {
  \config{\ssa{\ewhile [(x > 0)]{}^0, 0 \edo [x = x - 1]{}^1;, \vtrace_0}}
  \\
  \xrightarrow{} 
  \config{\ssa{[x = x - 1]{}^1; \ewhile [(x > 0)]{}^0, 0 \edo [x = x - 1]{}^1;, \vtrace_0 \cdot ((x > 0), 0, 1, \etrue)}}
  }
  ~\textbf{while-t}
\and
\inferrule
  {
  \inferrule
  {
	\config{x - 1, \vtrace_0 \cdot ((x > 0), 0, 1, \etrue)} \aarrow 0
  }
   {
   \config{\ssa{[x = x - 1]{}^1;, \vtrace_0 \cdot ((x > 0), 0, 1, \etrue)}}
  \xrightarrow{} 
  \config{\ssa{\eskip;, \vtrace_0 \cdot ((x > 0), 0, 1, \etrue) \cdot (x, 1, 1, 0)}}
   }~\textbf{asn}
  }
  {
  \config{\ssa{[x = x - 1]{}^1; \ewhile [(x > 0)]{}^0, 0 \edo [x = x - 1]{}^1;, \vtrace_0 \cdot \event}}
  \\
  \xrightarrow{} 
  \config{\ssa{\eskip; \ewhile [(x > 0)]{}^0, 0 \edo [x = x - 1]{}^1;, \vtrace_0 \cdot ((x > 0), 0, 1, \etrue) \cdot (x, 1, 1, 0)}}
  }
  ~\textbf{seq1}
\and
\inferrule
  {
  \inferrule
  {
   \vtrace_0, (x > 0) \barrow \efalse
   \and 
   \event = ((x > 0), 0, 2, \efalse)
  }
   {
   \config{\ssa{\ewhile [(x > 0)]{}^0, 1 \edo [x = x - 1]{}^1;, \vtrace_0 \cdot ((x > 0), 0, 1, \etrue)}}
  \\
  \xrightarrow{} 
  \config{\ssa{\eskip;, \vtrace_0 \cdot ((x > 0), 0, 1, \etrue) \cdot (x, 1, 1, 0)\cdot ((x > 0), 0, 2, \efalse)}}
   }~\textbf{while-f}
  }
  {
  \config{\ssa{\eskip; \ewhile [(x > 0)]{}^0, 0 \edo [x = x - 1]{}^1;, \vtrace_0 \cdot ((x > 0), 0, 1, \etrue) \cdot (x, 1, 1, 0)}}
  \\
  \xrightarrow{} 
  \config{\ssa{\eskip;, 
  \vtrace_0 \cdot ((x > 0), 0, 1, \etrue) \cdot (x, 1, 1, 0) \cdot ((x > 0), 0, 2, \efalse)}}
  }
  ~\textbf{seq2}
\end{mathpar}
%
Then we have following execution, where $\trace_0 \in \mathbb{T}$ and $\env(\trace_0) x = 1$.:
\[
\config{\ewhile [(x > 0)]{}^0 \edo [x = x - 1]{}^1;, \trace_0}
	\rightarrow^{*}
\config{\ssa{\eskip;, 
  \vtrace_0 \cdot ((x > 0), 0, 1, \etrue) \cdot (x, 1, 1, 0) \cdot ((x > 0), 0, 2, \efalse)}}
\]
\end{example}
%
%
\clearpage
\begin{example}[Example of Timing Channel under Trace Semantics].
\label{ex:timingdep}
\\
Using the same example from Example.~\ref{ex:excltiming}
\[
	\ewhile  [(x > 0)]{}^1 \edo [x = x - 1;]{}^2  [y = 1;]{}^3
\]
In this example, $y$'s execution times relies on value of $x$. Under the adaptivity scenario, $y$ depends on $x$ (control dependency).
%
This example shows we can derive by Definition.~\ref{def:var_dep}:
\[
	\vardep(x^2, y^3, {\ewhile  [(x > 0)]{}^1 \edo [x = x - 1;]{}^2  [y = 1;]{}^3})
\]
%
\begin{proof}
Let $\trace_0 = \cdot (x, 0, 1, 2) \in \mathbb{T}$, by semantics definition, we have:
%
\begin{equation}
\label{eq:os_timingdep}
\begin{array}{ll}
& \config{\ewhile  [(x > 0)]{}^1 \edo [x = x - 1;]{}^2  [y = 1;]{}^3 , \trace_0} \\
& \rightarrow^\rname{while-t}
\config{[x = x - 1;]{}^2  [y = 1;]{}^3; \ewhile  [(x > 0)]{}^1 \edo [x = x - 1;]{}^2  [y = 1;]{}^3, \trace_0 \cdot ((x > 0), 1, 1, \etrue)} \\
& \rightarrow^\rname{seq1}
\config{[\eskip]{}^2  [y = 1;]{}^3; \ewhile  [(x > 0)]{}^1 \edo [x = x - 1;]{}^2  [y = 1;]{}^3, \trace_0 \cdot ((x > 0), 1, 1, \etrue) \cdot(x, 2, 1, 1)} \\
& \rightarrow^\rname{seq2}
\config{[\eskip]{}^3; \ewhile  [(x > 0)]{}^1 \edo [x = x - 1;]{}^2  [y = 1;]{}^3, \trace_0 \cdot ((x > 0), 1, 1, \etrue) \cdot(x, 2, 1, 1) \cdot(y, 3, 1, 1)} \\
&\rightarrow^\rname{seq2}
\config{[x = x - 1;]{}^2  [y = 1;]{}^3; \ewhile  [(x > 0)]{}^1 \edo [x = x - 1;]{}^2  [y = 1;]{}^3, \trace_0 \cdot ((x > 0), 1, 1, \etrue) \cdot (x, 2, 1, 1) \\
& \quad \cdot(y, 3, 1, 1) \cdot ((x > 0), 1, 2, \etrue)} \\
& \rightarrow^\rname{seq1}
\config{[\eskip]{}^2  [y = 1;]{}^3; \ewhile  [(x > 0)]{}^1 \edo [x = x - 1;]{}^2  [y = 1;]{}^3, \trace_0 \cdot ((x > 0), 1, 1, \etrue) \cdot(x, 2, 1, 1) \\
& \quad \cdot(y, 3, 1, 1) \cdot ((x > 0), 1, 2, \etrue) \cdot(x, 2, 2, 0)} \\
& \rightarrow^\rname{seq2}
\config{[\eskip]{}^3; \ewhile  [(x > 0)]{}^1 \edo [x = x - 1;]{}^2  [y = 1;]{}^3, \trace_0 \cdot ((x > 0), 1, 1, \etrue) \cdot (x, 2, 1, 1) \\
& \quad \cdot(y, 3, 1, 1) \cdot ((x > 0), 1, 2, \etrue) \cdot(x, 2, 2, 0) \cdot(y, 3, 2, 1)} \\
& \rightarrow^\rname{seq2}
\config{[\eskip]{}^1;, \trace_0 \cdot ((x > 0), 1, 1, \etrue) \cdot(x, 2, 1, 1) \\
& \quad \cdot(y, 3, 1, 1) \cdot ((x > 0), 1, 1, \etrue) \cdot(x, 2, 2, 0) \cdot(y, 3, 2, 1) \cdot ((x > 0), 1, 3, \efalse)} \\
\end{array}
\end{equation}
%
Let $\event_1 = (x, 2, 1, 1)$,  $\event_1' = (x, 2, 1, 0)$ and $\event_b = ((x > 0), 1, 2, \etrue)$, then we have another execution as follows:
\[
\begin{array}{ll}
& \config{[\eskip]{}^2  [y = 1;]{}^3; \ewhile  [(x > 0)]{}^1 \edo [x = x - 1;]{}^2  [y = 1;]{}^3, \trace_0 \cdot ((x > 0), 1, 1, \etrue) \cdot \event_1'} \\
& \rightarrow^\rname{seq2}
\config{[\eskip]{}^3; \ewhile  [(x > 0)]{}^1 \edo [x = x - 1;]{}^2  [y = 1;]{}^3, \trace_0 \cdot ((x > 0), 1, 1, \etrue) \cdot \event_1' \cdot(y, 3, 1, 1)} \\
&\rightarrow^\rname{seq2}
\config{[\eskip;]{}^1, \trace_0 \cdot ((x > 0), 1, 1, \etrue) \cdot \event_1' \cdot(y, 3, 1, 1) \cdot ((x > 0), 1, 2, \efalse)} \\
\end{array}
\]
%
Then, we have $\event_b' = ((x > 0), 1, 2, \efalse)$ where $\event_b \sigeq \event_b'$ and $\event_b \eventneq \event_b'$.
\\
By Definition~\ref{def:event_valdep}, we have:
\\
\[
	\eventdep^{val}(\event_1, \event_b, \ewhile  [(x > 0)]{}^1 \edo [x = x - 1;]{}^2  [y = 1;]{}^3, D)
\]
\\
Let $\event_2 = (y, 3, 2, 1)$ then we have another execution as follows:
\[
\begin{array}{ll}
\config{[\eskip;]{}^1, \trace_0 \cdot ((x > 0), 1, 1, \etrue) \cdot \event_1' \cdot(y, 3, 1, 1) \cdot ((x > 0), 1, 2, \efalse)} \\
\rightarrow^{*} 
\config{[\eskip;]{}^1, \trace_0 \cdot ((x > 0), 1, 1, \etrue) \cdot \event_1' \cdot(y, 3, 1, 1) \cdot \event_b'}
\end{array}
\]
%
where $\event_2 \notsigin \trace_0 \cdot ((x > 0), 1, 1, \etrue) \cdot \event_1' \cdot(y, 3, 1, 1) \cdot \event_b'$
\\
By Definition~\ref{def:event_testdep}, we have:
\\
\[
	\eventdep^{\test}(\event_b, \event_2, \ewhile  [(x > 0)]{}^1 \edo [x = x - 1;]{}^2  [y = 1;]{}^3, D)
\]
\\

By Definition~\ref{def:event_dep}, we have: 
\[
	\eventdep(\event_1, \event_2, \ewhile  [(x > 0)]{}^1 \edo [x = x - 1;]{}^2  [y = 1;]{}^3, D)
\]
%
By Definition~\ref{def:event_dep}, we have:
\[
	\vardep(x^2, y^3, {\ewhile  [(x > 0)]{}^1 \edo [x = x - 1;]{}^2  [y = 1;]{}^3})
\]
%
%
\end{proof}
\end{example}
%
\clearpage
\begin{example}[Excluding the Over approximation in Example.~\ref{eq:sem_timingoverapp}].
\\
If $\mathsf{diff}(\omega, \omega')$ (in \cite{cousot2019abstract} Equation~(2)) simply includes timing channel,(i.e., $\omega$ is a strict prefix of $\omega'$) as follows:
\[
	\mathsf{diff}(\omega, \omega') \triangleq \exists \omega_0, \omega_1, \omega_1', v, v' 
	\st \bigvee \left\{
	\begin{array}{lr}
	(\omega = \omega_0 \cdot v \omega_1
		\land \omega' = \omega_0 \cdot v' \omega_1' \land v \neq v') & \mbox{original definition} \\
	(\omega = \omega' \cdot v \cdot \omega_1) & \mbox{including timing channel} \\
	\end{array}
	\right\}
\] 
then by Definition~2 (in \cite{cousot2019abstract}), there is a over approximation example:
\[
	\ewhile {}^2 (x > 0) {}^3 y = 1; 
\]
Let $z \in \mathbb{V}\setminus \{x\}$ be arbitrary variable different from $x$,
in this example, $y$ doesn't rely on $z$. 
However, according to value dependency defined in Definition~2 \cite{cousot2019abstract} we can derive 
\[
	z \rightsquigarrow^{2}_{\ewhile {}^2 (x > 0) {}^3 y = 1;} y
\]
%
\begin{proof}
By semantics definition, we have:
%
%
\end{proof}
\end{example}
%

\clearpage
% % 
\section{\THESYSTEM}
\label{sec:adpfun}
% In this section, we present our algorithm for computing the upper bound for a program $c$'s adaptivity
% $A(c)$ defined~\ref{def:trace_adapt} through static program analysis.
% This section presents the key definitions
% for the static analysis algorithm in Section~\ref{sec:algorithm-keys} before going into the detail of the algorithm,
% then shows the complete static analysis algorithm.
% \mg{
% In this section, we present our static program analysis for computing an upper bound on the adaptivity a program $c$
% }
In this section, we present our static program analysis for computing an upper bound on the adaptivity a program $c$.
%
\subsection{A guide to the algorithm}
In order to have the upper bound of adaptivity:
\\
1. $\THESYSTEM$  first build a program-based dependency graph to {over-}approximate the
trace-based dependency graph
through Section~\ref{sec:alg_vertexgen}, Section~\ref{sec:alg_weightedgegen} and~\ref{sec:alg_graphgen}:
% in the phases one to phases four of $\THESYSTEM$ in Section~\ref{sec:abscfg} to Section~\ref{sec:alg_graphgen}:
\\
1.1 approximate the vertices: 
% in the forth step of
in the first phase of $\THESYSTEM$ in
% the algorithm in Section~\ref{sec:alg_graphgen} 
Section~\ref{sec:alg_vertexgen}
without extra static analysis technique.
\\
1.2 approximate the vertices weights:
in the second phase of 
% the algorithm in  
of $\THESYSTEM$ specifically in Section~\ref{sec:alg_weightgen}
\\
1.3 approximate the edges between vertices:
also in the second phase of $\THESYSTEM$ specifically in 
% Section~\ref{sec:alg_weightgen}
Section~\ref{sec:alg_edgegen}
\\
1.4 generate the final approximated program-based dependency graph in Section~\ref{sec:alg_graphgen}
%  to {over-}approximate the
% approximate the query annotation: 
% in the forth step of
in the third phase of $\THESYSTEM$.
% the algorithm  without extra static analysis technique.
\\
2. Then in the last phase in Section~\ref{sec:alg_adaptcompute}, $\THESYSTEM$
% we compute the upper bound for adaptivity over this approximated graph:
% , as an upper bound for
% program's adaptivity
computes the upper bound for adaptivity over this approximated graph.
% in the last phase of this algorithm in Section~\ref{sec:alg_adaptcompute}.
\\

% \subsection{Adaptivity Based on Program Analysis in \THESYSTEM}
% In order to give a bound on the program's adaptivity, we first build a
% program-based data-dependency graph to {over-}approximate the
% trace-based dependency graph.  Then, we define a program-based
% adaptivity over this approximated graph, as an upper bound for
% $A(c)$.

This program-based graph has a similar topology structure as the one
of the trace-based (semantic) dependency graph. It has the same
vertices and query annotations, and approximated edges and weights.  
% An
% approximated edge correspond to a program-based data dependency
% relation ($\flowsto$ in Definition~\ref{def:flowsto}) and an approximated
% weight corresponds to a reachability bound analysis results from
% Definition~\ref{def:transition_closure}.

% %
% \subsubsection{Program-Based Variable Dependency}
% The program-based dependency relation over two labeled variables ($x^i, y^j)$ is defined as a $\flowsto$ relation with respect to the program $c$ as follows.
% %
% \begin{defn}[Data Flow Relation between Assigned Variables ($\flowsto$)].
% \label{def:flowsto}
% \\
% Given a program  ${c}$,
% a variable ${x^i}  \in \lvar_c $ is in the \emph{flows to} relation with another variable ${y^j} \in \lvar_c$, if and only if:
% \mg{I cannot even parse the next formula. Why there is a big disjunction on the left? Disjunction is a binary operation, or n-ary if given a set, what is this disjunction between?}\\
% \mg{please, remove the underscript $c$ in the exists. It just makes everything mroe difficult to parse.}\\
% \mg{The use of $\lor$ is odd. E.g. $\exists {(\expr \lor \qexpr)}$ or
%   $ [{\assign{y}{\expr \lor \query(\qexpr)}}]^{j} $. I suggest to write the whole formula instead of using weird shortenings.}\\
% \mg{Also, now it is too late to change this but instead of breaking down the definition using the subterm relation and then defining the flowto relation, it would have been better to give just one inductive definition of Flowto - I imagine that this makes also the proof more awkward.}
% %
% {\footnotesize 
% \[
% \begin{array}{l}
% \flowsto({x^i, y^j, c}) \triangleq 
% \\
% \left( \bigvee
% \begin{array}{l}
% (\exists \expr \st \clabel{\assign{y}{\expr}}^j \in_{c} {c} 
% \land {x} \in FV(\expr) \land (x^i \in \live(j, c)))
% \\
% (\exists {\qexpr} \st [\assign{y}{\query({\qexpr})}]^j \in_{c} {c} 
% \land x \in FV({\qexpr}) \land (x^i \in \live(j,c))))
% \\
% \Big(\exists {c_w} \in \cdom, l \in \mathcal{L}, \bexpr \st
% 	\ewhile [\bexpr]^l \edo {c_w} \in_{c} {c}
% 	\land \flowsto(x^i, y^j, c_w)
% 	\\ \qquad	
%      \lor 
% 	\big( \exists {(\expr \lor \qexpr)} \st
% 	[{\assign{y}{\expr \lor \query(\qexpr)}}]^{j} \in_{c}  {c_w}  \land {x} \in FV(\bexpr) \land x^i \in \live(l, c)
% 	\big)
% 	\Big)
% \\
% \Big(
% \exists {c_1}, {c_2} \in \cdom, l \in \mathcal{L}, \bexpr 
% \st 
% 	\eif([\bexpr]^l, {c_1}, {c_2}) \in_{c} {c} \land
% 	\flowsto(x^i, y^j, c_1) \lor \flowsto(x^i, y^j, c_2)
% 	\\ \qquad 
% 	\lor 
% 	\big( \exists {(\expr \lor \qexpr)} \st
% 	\land {x} \in FV(\bexpr) \land x^i \in \live(l, c) \land
% 	([{\assign{y}{\expr \lor \query(\qexpr)}}]^{j} \in_{c}  {c_1}  
% 	\lor [{\assign{y}{\expr \lor \query(\qexpr)}}]^{j} \in_{c}  {c_2})
% 	\big)
% \Big)
% % \\
% \end{array}
% \right).
% \end{array}
% \]
% }
% %
% \end{defn}
% %
% \mg{The next notation is inconsistent with the one used above. Also, this definition should be given before the definition of flowto. From the description I have no cluse what this notion of reachability means. Also, the definition is referred to does not define this notation.}\\
% \mg{the definition somehow seems to make sense but until when the or notation is fixed and I don't see the definition of RD, I cannot tell for sure.}
% $\live^l(c) \subseteq \lvar_c$,
% which is the set of all the reachable variables at location of label $l$ in the program $c$.
% For every labelled variable $x^l$ in this set, 
% the value assigned to that variable
% in the assignment command associated to that label is reachable at the entry point of  executing the command of label $l$.
% This is formally defined , formally computed in Definition~\ref{def:feasible_flowsto}
% \\
% \mg{This description seems inconsistent with the definition. I suggest to use the same variables and terms.}
% To understand the $\flowsto$ intuition, 
% given a program  ${c}$ with its labelled variables $\lvar_c$, and two variables ${x^i}, y^j  \in \lvar_c $ 
% % showing up as $i$-th, $j$-th elements in $\lvar$ 
% % (i.e., ${x} = \lvar(i)$ and ${y} = \lvar(j)$),
% we say $y^j$ flows to ${x^i}$ in ${c}$ if and only if 
% the value of $y^j$ directly or indirectly influence the evaluation of the value of ${x}$ as follows:
% %
% \begin{itemize}
% \item (Explicit Influence) The program ${c}$ contains either 
% a command $[\assign{{x}}{\aexpr}]^i$ or $[\assign{{x}}{\query({\qexpr})}]^i$,
% such that ${y}$ shows up as a free variable in $\expr$ or ${\qexpr}$.
% We use $\flowsto({x^i, y^j, c})$ to denote $y^j$ flows to $x^i$ in ${c}$.
% %
% \item (Implicit Influence) The program ${c}$ contains either a while loop
% command
% or if command, 
% such that $x$ shows up in the guard
% and $y$ shows up in the left hand of an assignment command and this assignment command showing up
%  in the body of the while loop, or branches of if command.
% \end{itemize}
% %
% % This is formally defined in \ref{def:flowsto}.
% % We use $FV(\expr)$, $FV(\sbexpr)$ and $FV(\qexpr)$ denote the set of free variables in 
% % expression $\expr$, boolean expression $\sbexpr$ and query expression $\qexpr$ respectively.
% %
% %
% \mg{I don't understand what this definition of equivalence means. It is not observational equivalence
% and it is not syntactic equivalence. What are we trying to capture here? Also, it is equivalence of programs, not of program.}
% \begin{defn}[Equivalence of Program]
% %
% \label{def:aq_prog}
% Given 2 programs $c_1$ and $c_2$:
% \[
% c_1 =_{c} c_2
% \triangleq 
% \left\{
%   \begin{array}{ll} 
%     \etrue        
%     & c_1 = \eskip \land c_2 = \eskip
%     \\ 
%     \forall \trace \in \mathcal{T} \st \exists v \in \mathcal{VAL}
%     \st \config{ \trace, \expr_1} \aarrow v \land \config{ \trace, \expr_1} \aarrow v     
%     & c_1 = \assign{x}{\expr_1} \land c_2 = \assign{x}{\expr_2} 
%     \\ 
%     \qexpr_1 =_{q} \qexpr_2       
%     & c_1 = \assign{x}{\query(\qexpr_1)} \land c_1 = \assign{x}{\query(\qexpr_2)} 
%     \\
%     c_1^f =_{c} c_2^f \land c_1^t =_{c} c_2^t
%     & c_1 = \eif(b, c_1^t, c_1^f) \land c_2 = \eif(b, c_2^t, c_2^f)
%     \\ 
%     c_1' =_{c} c_2'         
%     & c_1 = \ewhile b \edo c_1' \land c_2 = \ewhile b \edo c_2'
%     \\ 
%     c_1^h =_{c} c_2^h \land c_1^t =_{c} c_2^t
%     & c_1 = c_1^h;c_1^t \land c_2 = c_2^h;c_2^t 
%   \end{array}
%   \right.
% \]
% %
% As usual, we denote by $c_1 \neq_{c} c_2$ the negation of the equivalence.
% %
% \end{defn}
% %
% \mg{This definition needs to go before it is used. }
% Given 2 programs $c$ and $c'$, we denote by $c' \in_{c} c$  that $c'$ is a sub-program of $c$ defined as follows,
% \begin{equation}
% c' \in_{c} c \triangleq \exists c_1, c_2, c''. ~ s.t.,~
% c =_{c} c_1; c''; c_2 \land c' =_{c} c''
% \end{equation} 
% %

% \subsubsection{Program Analysis Based Dependency Graph}
% We give the formal definition for the program-based dependency graph for a program $c$, 
% $\progG({c}) = (\vertxs, \edges, \weights, \flag)$ as follows.
% \begin{defn}
%     [Program-Based Dependency Graph].
%     \label{def:prog_graph}
%     \\
% Given a program ${c}$
% its program-based graph 
% $\progG({c}) = (\vertxs, \edges, \weights, \qflag)$ is defined as:
% {\footnotesize
% \[
% \begin{array}{rlcl}
% \text{Vertices} &
% \vertxs & := & \left\{ 
% x^l \in \mathcal{LV} 
% ~ \middle\vert ~
% x^l \in \lvar_{c}
% \right\}
% \\
% \text{Directed Edges} &
% \edges & := & 
% \left\{ 
%   ({x}_1^{i}, {x}_2^{j}) \in \mathcal{LV} \times \mathcal{LV}
%   ~ \middle\vert ~
%   \begin{array}{l}
%     {x}_1^{i}, {x}_2^{j} \in \vertxs
% 	\land
%     % \\
%     \exists n \in \mathbb{N}, z_1^{r_1}, \cdots, z_n^{r_n} \in \lvar_{{c}} \st 
%     n \geq 0 \land
%     \\
%     \flowsto(x^i,  z_1^{r_1}, c) 
%     \land \cdots \land \flowsto(z_n^{r_n}, y^j, c) 
%   \end{array}
% \right\}
% \\
% \text{Weights} &
% \weights & := &
% % \bigcup
% % \begin{array}{l}
% 	\left\{ (x^l, w) \in  \mathcal{LV} \times EXPR(\constdom)
% 	\mid
% 	x^l \in \lvar_{{c}} \land w = \absW(l)
% 	\right\}
% % \end{array} 
% \\
% \text{Query Annotation} &
% \qflag & := & 
% \left\{(x^l, n)  \in  \mathcal{LV} \times \{0, 1\} 
% ~ \middle\vert ~
%  x^l \in \lvar_{c},
% n = 1 \iff x^l \in \qvar_{c} \land n = 0 \iff  x^l \in \qvar_{c} .
% \right\}
% \end{array}
% \] 
% }
% , where the $\absW(l)$ is the symbolic reachability bound in domain of $EXPR(\constdom)$,
% % for the assignment command of label $l$ to which  
% the labeled variable $x^l$, 
% % is associated, 
% computed from the $\THESYSTEM$ algorithm 
% in Definition~\ref{def:transition_closure}.
% The $EXPR(\constdom)$ is an expression over symbolic constants containing the
% input variables and natural number.
% \end{defn} 
% %
% \paragraph{Program-Based Adaptivity ($\progA(c)$)}
% %
% Given a program ${c}$, we generate its program-based graph 
% $\progG({c}) = (\vertxs, \edges, \weights, \qflag)$.
% %
% Then the adaptivity bound based on program analysis for ${c}$ 
% % is the number of query vertices on a finite walk in $\progG({c})$. This finite walk satisfies:
% % \begin{itemize}
% % \item the number of query vertices on this walk is maximum
% % \item the visiting times of each vertex $v$ on this walk is bound by its reachability bound $\weights(v)$.
% % \end{itemize}
% is computed as the maximum query length over all finite walks in $\walks(\progG({c}))$,
% %
% % It is formally defined in \ref{def:prog_adapt}.
% defined formally as follows.
% %
% %
% \begin{defn}
% [{Program-Based Adaptivity}].
% \label{def:prog_adapt}
% \\
% {
% Given a program ${c}$ and its program-based graph 
% $\progG({c}) = (\vertxs, \edges, \weights, \qflag)$,
% %
% the program-based adaptivity for $c$ is defined as%
% \[
% \progA({c}) 
% := \max
% \left\{ \qlen(k)\ \mid \  k\in \walks(\progG({c}))\right \}.
% \]
% }
% \end{defn}  
%
%
% {
% \begin{defn}[Variable Flags ($\flag$)].
% \\
% Given a program  ${c}$ with its labelled variables $\lvar$, the $\flag$ is a vector of the same length as $\lvar$, s.t. for each variable ${x}$ showing up as the $i$-th element in $\lvar$ (i.e., ${x} = \lvar(i)$), 
% $\flag(i) \in \{0, 1, 2\}$ is defined as follows:
% %
% %
% \[
% \flag(i) := 
% \left\{
% \begin{array}{ll}
% 2 & 
% {x^l} \in \lvar_{c} \land 
% (\exists {\qexpr}. ~ s.t., ~
% [\assign{{x}}{\query({\qexpr})}]^l \in_{c} {c})
% \\
% 1 &  
% \begin{array}{l}
% {x^l} \in \lvar_{c} \bigwedge \\
% \left(
% \begin{array}{l}
% \big(\exists  ~ {c'}, {\expr}, \sbexpr, l, l'. ~
% 	\ewhile [\sbexpr]^l \edo {c'} \in_{c} {c}
% 	\land 
% 	[{\assign{x}{\expr}}]^{l'} \in_{c}  {c'}
% \big) \bigvee
% \\
% \big(\exists ~ \sbexpr, l, l_1, l_2, {c_1}, {c_2}, {\expr}_1, {\expr}_2. ~
% 	\eif([\sbexpr]^l, {c_1}, {c_2}) \in_{c} {c} \land
% 	([{\assign{x}{\expr_1}}]^{l1} \in_{c} {c_1} \lor 
% 	[{\assign{x}{\expr_2}}]^{l2} \in_{c} {c_2})
% \big)
% \end{array}
% \right)
% \end{array}
% \\
% 0 & \text{o.w.}
% \end{array}
% \right\}. 
% \] 
% %
% \end{defn}
%
% Operations on $\flag$ are defined as follows:
% \begin{equation}
% \begin{array}{llll}
% {\flag_1 \uplus \flag_2}(i) & := &
% \left\{
% \begin{array}{ll}
% k & k = \max{\big\{\flag_1(i), \flag_2(i)\big\}} 
% \land |\flag_1| = |\flag_2|\\
% 0 & o.w.
% \end{array}\right.
% & i = 1, \cdots, |\flag_1|  
% \\
% {\flag \uplus n}(i) & := & 
% \max\big\{ \flag(i), n \big\} 
% & i = 1, \ldots, |\flag|    
% \\
% \left[ n \right]^k (i) & := &  n
% & i = 1, \ldots, k ~ \land ~ |\left[ n \right]^k| = k
% \end{array}
% \end{equation}
%
%
%
% \begin{defn}[Data Flow Matrix ($\Mtrix$)]
% The data flow matrix $\Mtrix$ of a program $c$ is a matrix of size $|\lvar_c| \times |\lvar_c|$ 
% s.t.,
% %
% \[
% \Mtrix(i, j) \triangleq
% \left\{
% \begin{array}{ll}
% 1	&	\flowsto({x^i, y^j, c}) \\
% 0	& o.w.
% \end{array}
% \right., {x^i}, y^j  \in \lvar_c.
% \]
% %
% \end{defn}
% %
% Operations on the data flow matrices are defined as follows:
% %
% \begin{equation}
% \Mtrix_1 ; \Mtrix_2 
% := \Mtrix_2 \cdot \Mtrix_1 + \Mtrix_1 + \Mtrix_2
% \end{equation}
% %
% Consider the same program $c$ as above, its data flow matrix $\Mtrix$ and $\flag$ for the program $c$ is as follows:
% $$
% {c} = 
% \begin{array}{l}
% \left[{\assign {x_1} {\query(0)}}	\right]^1;
% \\
% \left[{\assign {x_2} {x_1 + 1}}		\right]^2;
% \\
% \left[{\assign {x_3} {x_2 + 2}}		\right]^3
% \end{array}
% ~~~~~~~~~~~~
% \Mtrix
% =  \left[ 
% \begin{matrix}
% 0 & 0 & 0 \\
% 1 & 0 & 0 \\
% 1 & 1 & 0 \\
% \end{matrix} \right] ~ , 
% \flag = \left [ \begin{matrix}
% 1 \\
% 0 \\
% 0 \\
% \end{matrix} \right ]
% $$
% %
% % There are two special matrices used for generating the data flow matrix $\Mtrix$ in the analysis algorithm. They are the left matrix $\lMtrix_i$ and right matrix $\mathsf{R_{(e, i)}}$.

% % Given a program  ${c}$ with its labelled variables $\lvar$ of length $N$,
% % the left matrix $\lMtrix_i$ generates a matrix of $1$ column, $N$ rows, 
% % where the $i$-th row is $1$ and all the other rows are $0$.
% % %
% % \begin{defn}[Left Matrix ($\lMtrix_i$)].
% % \\
% % Given a program  ${c}$ with its labelled variables $\lvar$ of size $N$, 
% % the left matrix $\lMtrix_i$ is defined as follows:
% % \[
% % \lMtrix_i(j) : = 
% % \left
% % \{
% % \begin{array}{ll}
% % 1 & j = i \\
% % 0 & o.w.
% % \end{array}
% % \right.,
% % j = 1, \ldots, N.
% % \]
% % \end{defn}
% % %
% % Given a program  ${c}$ with its labelled variables $\lvar$ of length $N$,
% % the right matrix $\rMtrix_{\expr, i}$ generates a matrix of one row and $N$ columns, 
% % where the locations of free variables in $\expr$ is marked as $1$. 
% % %
% % %
% % \begin{defn}[Right Matrix ($\rMtrix_{\expr}$)].
% % \\
% % Given a program  ${c}$ with its labelled variables $\lvar$ of length $N$, 
% % the right matrix $\rMtrix_{\expr}$ is defined as follows:
% % \[
% % \rMtrix_{\expr}(j) : = 
% % \left\{
% % \begin{array}{ll}
% % 1 & {x} \in FV(\expr) 
% % \\
% % 0 & o.w.
% % \end{array}
% % \right.,
% % {x} = \lvar(j) ~ , ~ j = 1, \ldots, N.
% % \]
% % %
% % %
% % \end{defn}
% % %
% % Using the same example program ${c}$ as above with labelled variables $\lvar = [ {x_1 , x_2 , x_3} ] $,
% % the left and right matrices w.r.t. its $2$-nd command 
% % $\left[{\assign {x_2} {x_1 + 1}}\right]^2$  are as follows:
% % \[
% % \lMtrix_1 = \left[ \begin{matrix}
% % 0   \\
% % 1 	 \\
% % 0   \\
% % \end{matrix}   \right ] 
% % ~~~~~~~~~~~~~~
% % \rMtrix_{{x}_1 + 1}
% % = \left[ \begin{matrix} 
% % 1 & 0 & 0 \\
% % \end{matrix}  \right]
% % \]
% %
% %
% %
% \subsection{ $\THESYSTEM$ Analysis Algorithm}
% \subsection{Dependency Graph Estimation}
\subsection{Vertices Estimation}
\label{sec:alg_vertexgen}
The vertices and query annotations in the execution based graph is built on static information, 
we reuse this information and construct the vertices and query annotations on 
this  {over-}approximate graph identical to the execution-based graph, specifically
%  in the last step of
% the algorithm in Section~\ref{}
as follows,
  \highlight{
\[
    \progV(c) \triangleq \left\{ 
  x^l \in \mathcal{LV} 
  ~ \middle\vert ~
  x^l \in \lvar_{c}
  \right\}
  \]
  }
% \wq{To do: Add $\THESYSTEM$, a data flow analysis algorithm to scan the program and give a graph.}
% {\THESYSTEM} consists of three phases: 
% \begin{enumerate}
%     \item Generating an abstract control flow graph with each edge representing an abstract event transiting between two command labels. 
%     \item Computing the value bound invariant for each variable in the event and 
%     the event transition closure over the abstract control flow graph,
%     we get the reachability bound for each labeled command.
%     \item Refining the abstract control flow graph with data-flow, by performing the reaching definition analysis, we generate a weighted data control flow graph.
%     \item An algorithm to find the appropriate path in the weighted data control flow graph
% \end{enumerate}

% \begin{enumerate}
%     \item An algorithm to generate a precise data control flow graph
%     \item An algorithm to perform a Reachability number analysis to calculate the weight of each node in the graph generated in phase 1.
%     \item An algorithm to find the appropriate path in the weighted data control flow graph
% \end{enumerate}
\subsection{Weight and Edge Estimation}
\label{sec:alg_weightedgegen}
The edge and weight in execution-based graph is built on all possible execution traces,
in order to approximate them statically, we show how to analyze the program to 
get an accurate approximation.

% This analysis first 
%  generate an abstract control flow graph
%  over all program labels, 
% in order to analyzing the data flow relations through variables assigned in every labeled command,
% and the reaching time of each variable.
% Then, it refines this control flow graph 
% % into a weighted data-dependency graph, 
% and generate the Program-Based Dependency Graph,
% through the data flow and reaching bound analysis results.
% In the last step, it finds the longest finite walk in this weighted data control flow graph w.r.t. the query variables,
% and return the number of query vertices traversed alongside.
% % \wq{To do: Add $\THESYSTEM$, a data flow analysis algorithm to scan the program and give a graph.}
% To be more specific, {\THESYSTEM} consists of five phases as follows,
% \\
% % \jl{Better to have a graph or picture of overview of the algorithm}
% \todo{graph}
% \todo{pass again}
This analysis
\begin{enumerate}
    % \item Generating 
    \item first generate 
    an abstract control flow graph over all labels, 
    % which are used as program's control locations,
    %
    \item then compute the weight of every vertex in $\progV(c)$ by computing a symbolic reachability bound for each label,
    % \\
    \item then estimate the edges between every vertex in $\progV(c)$ by computing the feasible data flow relation between every labeled variables.
  
\end{enumerate}
\subsubsection{Abstract Execution Control Flow graph}
\label{sec:abscfg}
We first show how to generate the abstract control flow graph in this part, in order to estimate 
the weight for vertex in $\progV(c)$ and edges between vertices in $\progV(c)$.
\\
In this abstract control flow graph, every vertex is a label,
 corresponding to a label command in the program.
Each directed 
 edge represents an abstract transition 
 between two control locations, i.e., the labels of two commands (we call the labels also control location and they refer to the same thing), 
 where the second labeled command will be executed after execution of the command with first label.
 The abstract transition contains a set of difference constraints for variables, generated by abstracting the command of the first labeled.
 We present the detail of program abstraction and the graph generation in following steps.
%
\paragraph*{Expression Abstraction}
We introduce the following notations and operations first
% an expression abstraction method based on the expression abstraction in paper \cite{sinn2017complexity}.
\\
% is enriched into $\constdom \triangleq \mathbb{N} \cup \inpvar \cup \{\max{(\dbdom)}\} $.
The Symbolic Constant:  $\constdom \triangleq \mathbb{N} \cup \inpvar \cup \{\max{(\dbdom)}\} $.
%  is enriched into $\constdom \triangleq \mathbb{N} \cup \inpvar \cup \{\max{(\dbdom)}\} $.
It consists of 
natural numbers $\mathbb{N}$,
% the symbolic constants
the program's input variables $\inpvar$ 
% (i.e., the set of the program's input variables), 
and a constant value $\max(\dbdom)$ for estimating the upper bound of variables which are
assigned by queries.
\\
Difference constraint: $DC : \mathcal{VAR} \cup \constdom \to \mathcal{\mathcal{VAR} \times (\mathcal{VAR} \cup \constdom) } \times (\mathbb{Z} \cup \{\infty\})$
 \\
$DC(\mathcal{VAR}  \cup \constdom) \cup \{\top\}$ represents all the difference constraints over the 
variable and symbolic constants.
% The difference constraint $DC$ over $\mathcal{VAR} \cup \constdom$ 
It is a set of the inequality of form $x \leq y + v$ where $x \in \mathcal{VAR} $, 
$y \in \mathcal{VAR}  \cup \constdom$ and $v \in \mathbb{Z}$. 
\\
This difference constraint is defined in the same way as
\cite{sinn2017complexity}. 
% represents the set of inequality over all $\mathcal{VAR}  \cup \constdom$. 
\\
% The symbolic constant is enriched into $\constdom \triangleq \mathbb{N} \cup \inpvar \cup \{\max{(\dbdom)}\} $.
% It consists of 
% natural number $\mathbb{N}$,
% the symbolic constants $\inpvar$ (i.e., the set of the program's input variables), 
% and a constant value $\max(\dbdom)$ for estimating the upper bound of variables which are
% assigned by queries.
% \\
% The symbolic constant is enriched into $\constdom \triangleq \mathbb{N} \cup \inpvar \cup \{\max{(\dbdom)}\} $.
% \\
For concise, we use $\dcdom^{\top}$ to represent the $DC(\mathcal{VAR}  \cup \constdom) \cup \{\top\}$ .
\\
% % $ \absdom: \mathcal{P}(DC(\mathcal{VAR}  \cup \constdom) \cup \{\top \})$:
% \\
% $\constdom: \mathbb{N} \cup \inpvar \cup \{\max{(\dbdom)}\} $ 
% The  constant 
\\
% % $DC(\mathcal{VAR}  \cup \constdom)$ represents the set of inequality over all $\mathcal{VAR}  \cup \constdom$.
% \\
Expression Abstraction: (adopted from the expression abstraction method in paper \cite{sinn2017complexity})
% We simplify the expression abstraction method from paper \cite{sinn2017complexity}, where 
$\absexpr : \expr \to \mathcal{VAR} \to DC(\mathcal{VAR}  \cup \constdom) \cup \{\top\} $,
computes the abstract value for each variable 
according to the expression $\expr$ assigned to it in the command.
% \[
%   \begin{array}{ll} 
%     \absexpr(y + c, x)  = x' \leq y + c  & c \in \mathbb{N} \land y \in (VAR \cup \constdom) \\
%     \absexpr(y - c, x)  = x' \leq y - c  & c \in \mathbb{N} \land y \in (VAR \cup \constdom) \\
%     \absexpr(v, x)  = x' \leq v + 0  & v \in (VAR \cup \constdom) \\
%     \absexpr(\aexpr, x) = x' \leq 0 + \infty   & \aexpr \text{ doesn't have any of the forms as above} \\
%     \absexpr(\qexpr, x)  = x' \leq 0 + \max(\dbdom) & \qexpr \text{ is a query expression}  \\
%     \absexpr(\bexpr, x) = x' \leq 0 + 1   & \bexpr \text{ is a boolean expression} \\
%   \end{array}
%   \]
  \[
    \begin{array}{ll} 
      \absexpr(x - v, x)  = x' \leq x - v  & x \in \grdvar \land v \in \mathbb{N} \\
      \absexpr(y + v, x)  = x' \leq y + v  & x \in \grdvar \land v \in \mathbb{Z} \land y \in (\grdvar \cup \constdom) \\
      \absexpr(v, x)  = x' \leq v + 0  & x \in \grdvar \land v \in (\grdvar \cup \constdom) \\
      \absexpr(y + v, x)  = x' \leq y + v & \\
      \grdvar = \grdvar \cup \{y\} & x \in \grdvar \land v \in \mathbb{Z} \land y \notin (\grdvar \cup \constdom)  \\
      \absexpr(\qexpr, x)  = x' \leq 0 + \max(\dbdom) & x \in \grdvar \land \qexpr \text{ is a query expression}  \\
      \absexpr(\bexpr, x) = x' \leq 0 + 1   & x \in \grdvar \land \bexpr \text{ is a boolean expression} \\
      \absexpr(\expr, x) = x' \leq \infty  &  x \in \grdvar \land \expr \text{ doesn't have any of the forms as above} \\
      \absexpr(\expr, x) = \top  &  x \notin \grdvar \\
    \end{array}
    \]
  In line:4 of case where variable in the $\grdvar$ is updated by a variable $y$ not in this set, we add $y$ into the set $\grdvar$ and repeat 
  above procedure  until $\grdvar$ and $\absexpr(\expr, x)$ is stabilized for every variable.
  \\
More specifically in 
% understanding the intuition, 
we handle a simplified normalized guard expression ($ x > 0$ for $x^l \in \lvar_c$)
 in $\ewhile$ and the same forms of the counter modification as in paper \cite{sinn2017complexity}.
\\
The counter variables only increase, decrease or reset by expression in the form of arithmetic minus and plus (able to extend to max and min.)
\\
For more complex expression assignments, where the counter reset, or calculated from $\elog$, 
multiplication or division, and expressions involving multiple variables, the constraint is approximated as reset of $\infty$.
\\
This simplification doesn't affect our analysis results in our examples. It is easy to extend the normalized expression 
into more complex forms as in \cite{sinn2017complexity}, as well as the 
counter variable manipulation with more advanced expressions as in \cite{}.
% \\ 
% The boolean expression in the guard of $\ewhile$ command is normalized into form of $ x > 0$ where $x^l \in \lvar_c$ for some $l$.
\paragraph{Program Event Abstraction}
We first compute the abstract initial and final state for program and then generate the abstract event for program in 
computing the abstract execution trace.
\\
The abstract initial state is a label from $\ldom$.
\\
The abstract final state is a pair from $\ldom \times \dcdom^{\top}$,  
where first component is a label from $\ldom$ and the second component is a constraint from $\dcdom^{\top}$.
%
\begin{defn}[Abstract Event]
  \label{def:abs_event}
  Abstract Event: 
  $\absevent \in $
  $\ldom \times \dcdom^{\top} \times \ldom$
  is a pair of abstract initial state and final state.
  % computed from program's abstract final and initial state, $\absfinal(c)$ and $\absinit(c)$ with formal definition, and algorithm detail in Appendix.
  \end{defn}
  The abstract event for program is generated when computing its abstract execution trace.
\\
%
Given a program $c$, its abstract initial state,
and the set of its abstract final state is computed as follows,
%
\[
  \begin{array}{ll}
    \absinit(\clabel{\assign{x}{\expr}}{}^l)  & = l  \\
    \absinit(\clabel{\assign{x}{\expr}}{}^l)  & = l \\
    \absinit(\clabel{\eskip}^{l})  & = l \\
    \absinit(\eif [b]^l \ethen c_1 \eelse c_2)  & = l \\
    \absinit(\ewhile [b]^l \edo c)  & = l \\
    \absinit(c_1 ; c_2)  & = \absinit(c_1) \\
 \end{array}
 \]
%
Final State Abstraction: 
$\absfinal: \cdom \to \mathcal{P}(\ldom \times \dcdom^{\top})$,
computes the set of Abstract Final State for the command. 
 \[
  \begin{array}{ll}
    \absfinal(\clabel{\assign{x}{\expr}}{}^l)  & = \{(l, \absexpr\eapp (\expr, x))\}  \\
     \absfinal(\clabel{\assign{x}{\query(\qexpr)}}{}^l)  & = \{
      (l, x' \leq 0 + \max(\dbdom) )\}  \\
     \absfinal(\clabel{\eskip}^{l})  
     & = \{(l, \top)\} \\
     \absfinal(\eif [b]^l \ethen c_1 \eelse c_2)  & = \absfinal(c_1) \cup \absfinal(c_2) \\
     \absfinal(\ewhile [b]^l \edo c)  & = \{(l, \top)\} \\
     \absfinal(c_1 ; c_2)  & =  \absfinal(c_2) \\
 \end{array}
 \]
 %
 \paragraph{Abstract Execution Trace}
 Then, we  extract the abstract execution trace  $\absflow(c)$ for a program, which computes the Abstract Execution Trace for program $c$, as a set of the abstract events $\absevent$.
 %
 \begin{defn}[Abstract Execution Trace]
 \label{def:abs_trace}
  $\absflow \in \cdom \to \mathcal{P}($Label $\times DC(\mathcal{VAR}  \cup \constdom) \cup \{\top\}) \times $ Label )
  \end{defn}
 %
 Abstract Execution Trace Computation:
  \\
  For simplicity, we use $\mathcal{P}(\absevent)$ represent the power set of all abstract events, and we have $\absflow(c) \in \mathcal{P}(\absevent)$.
 \\
 We first append a skip command with a symbolic label $l_e$, i.e., $\clabel{\eskip}^{l_e}$ at the end of the program $c$, and compute the $\absflow(c) = \absflow'(c')$ for $c'$, where $c' = c;\clabel{\eskip}^{l_e}$ as follows,
 %
 {\footnotesize
 \[
   \begin{array}{ll}
      \absflow'(\clabel{\assign{x}{\expr}}{}^l)  & = \emptyset  \\
      \absflow'(\clabel{\assign{x}{\query(\qexpr)}}{}^l)  & = \emptyset  \\
      \absflow'([\eskip]^{l})  & = \emptyset \\
      \absflow'(\eif [b]^l \ethen c_t \eelse c_f)  & =  \absflow'(c_t) \cup \absflow'(c_f)
      %   \\ & \quad 
        \cup \{(l, \top,  \absinit(c_t) ) ,  (l, \top, \absinit(c_f)) \} \\
       \absflow'(\ewhile [b]^l \edo c_w)  & =  \absflow'(c_w) \cup \{(l, \top, \absinit(c_w)) \} 
      %  \\ & \quad 
       \cup \{(l', dc, l)| (l', dc) \in \absfinal(c_w) \} \\
       \absflow'(c_1 ; c_2)  & = \absflow'(c_1) \cup  \absflow'(c_2) 
      %  \\ & \quad 
       \cup \{ (l, dc, \absinit(c_2)) | (l, dc) \in \absfinal(c_1) \} \\
   \end{array}
   \]
   }

   The $\absflow'([x := \expr]^{l})$, $\absflow'([x := \query(\qexpr)]^{l})$ and $\absflow'([\eskip]^{l})$ are all empty set. 
   For every event $\event$ with label $l$ in an execution trace $\trace$ of program $c$, 
   there is an abstract event in program's abstract execution trace of form $(l, \_, \_)$. 
   \begin{lem}[Soundness of the Abstract Execution Trace]
     \label{lem:abscfg_sound}
   Given a program ${c}$, we have:
   %
   \[
     \begin{array}{l}
       \forall \vtrace_0, \trace \in \mathcal{T} ,  \event = (\_, l, \_) \in \eventset \st
   \config{{c}, \trace_0} \to^{*} \config{\eskip, \trace_0 \tracecat \vtrace} 
   \land \event \in \trace 
   \\
   \qquad \implies \exists \absevent = (l, \_, \_) \in (\ldom\times \dcdom^{\top} \times \ldom) \st 
   \absevent \in \absflow(c)
   \end{array}
   \]
   \end{lem}
%    This lemma is proved formally in Appendix~\ref{apdx:reachability_soundness}.
% For every event $\event$ with label $l$ in an execution trace $\trace$ of program $c$, 
% there is an abstract event in program's abstract execution trace of form $(l, \_, \_)$. 
This lemma is proved formally in Lemma~\ref{lem:abscfg_sound} in Appendix~\ref{apdx:reachability_soundness}.
\\
For every labeled variable in program $c$, $x^l \in \lvar_c$, there is a unique abstract event in program's abstract execution trace $\absevent \in \absflow(c)$ of form $(l, \_, \_)$. 
\begin{lem}[Uniqueness of the Abstract Execution Trace]
  \label{lem:abscfg_unique}
Given a program ${c}$, we have:
%
\[
  \begin{array}{l}
    \forall \vtrace_0, \trace \in \mathcal{T} ,  \event = (\_, l, \_, \_) \in \eventset^{\asn} \st
\config{{c}, \trace_0} \to^{*} \config{\eskip, \trace_0 \tracecat \vtrace} 
\land \event \in \trace 
\\
\qquad \implies \exists! \absevent = (l, \_, \_) \in (\ldom\times \dcdom^{\top} \times \ldom) \st 
\absevent \in \absflow(c)
\end{array}
\]
\end{lem}
This lemma and proof is also 
formalized in Lemma~\ref{lem:absevent_unique} in Appendix~\ref{apdx:reachability_soundness}.

% We have a pre-processing algorithm to go through the programs and returns the list of labels associating with a loop and whose visiting times need to be analyzed.
%
\paragraph{Abstract Control Flow Graph} 
Through a program $c$'s abstract execution trace, its abstract control flow graph is computed in 
Definition~\ref{def:abs_cfg}.
% Given program $c$ with its abstract control flow $\absflow(c)$, the Abstract Control Flow Graph:
% \\
\begin{defn}[Abstract Control Flow Graph]
\label{def:abs_cfg}
Given a program $c$, 
with its abstract control flow $\absflow(c)$
its abstract control flow graph $\absG(c) =(\absV, \absE, \absW)$ is defined as follows,
\\
% \highlight{
% :
%
% \\
$\absE = \{(l_1, dc, l_2) | (l_1, dc, l_2) \in \absflow(c)\}$,
\\
$\absV = labels(c)\cup\{l_e\}$
\\
 $\absW 
\triangleq \left\{ (l, w) \in \mathbb{L} \times EXPR(\constdom) \right\}$.
% }
\\
, where the weight of every label to be computed in the next step.
\end{defn}
% 
The vertices in this graph are program's labels with an exit label $l_{ex}$.
Each directed 
 edge $(l_1, dc, l_2)$ from $l_1$ to $l_2$,
 represents an abstract transition 
 between two control locations with a set of difference constraints on it.
%  , i.e., the labels of two commands (we call the labels also control location and they refer to the same thing), 
%  where 
In this transition, the  command labeled with the second location $l_2$, 
 will be executed after execution of the command with label $l_1$,
%  The abstract transition contains a set of difference constraints for variables, 
with the difference constraints generated by abstracting the command of the first label.
% \\
% It is easy to show for every $(l_1, dc, l_2) \in \absflow(c)$ such that $l_2 \neq l_e$, $(l_1, l_2) \in flow(c)$. The formal Lemma and proof can be found in Lemma~\ref{lem:flow_to_absflow} in Appendix~\ref{apdx:reachability_soundness}.
The weight for every vertex is a symbolic expression over the symbolic constant, 
which is the estimated upper bound on the number of visiting time for every control location
through the reachability bound analysis as follows.
%
% It is easy to show for every $(l_1, dc, l_2) \in \absflow(c)$ such that $l_2 \neq l_e$, $(l_1, l_2) \in flow(c)$. The formal Lemma and proof can be found in Lemma~\ref{lem:flow_to_absflow} in Appendix~\ref{apdx:reachability_soundness}.
%
\subsubsection{Weight Estimation}
\label{sec:alg_weightgen}
%
In order to estimate weight for every vertex in $\progV(c)$,
we first show how to compute the reachability bound for every label in $c$,
then show how to compute the weight for every vertex in $\progV(c)$.
\\
we infer the invariant for every variable, and compute the transition closure for every abstract transition. By solving the closure
with the invariants of variables involved in this closure for every transition, we compute
the symbolic reachability bound of every commands corresponding to this transition.
\\
We present the details of invariant, closure generation, and reachability bound computation as follows.
%
\paragraph*{Variable Modification Tracking}
Identify the abstract events where each variable is increased, decreased and reset:
\\
$\inc: \mathcal{VAR} \to \mathcal{P}(\absevent) $
the set of the abstract events where the variable increase.
\\
$\inc(x) = \{(\absevent, c) | \absevent = (l, l', x' \leq x + v)\}$
\\
$\reset: \mathcal{VAR} \to \mathcal{P}(\absevent) $
The set of the abstract events where the variable is reset.
\\
$\dec: \mathcal{VAR} \to \mathcal{P}(\absevent) $
The set of abstract events where the variable decrease.
% \\
% $\dec(x) = \{(\absevent, c) | \absevent = (l, l', x' \leq x - v)\}$
\\
$Incr(v) \triangleq \sum\limits_{(\absevent, c) \in \inc(v)}\{\absclr(\absevent) \times v\}$
%
\paragraph*{Local Bounds}
Given a program $c$ with its abstract control flow graph 
$\absG(c) = (\absV, \absE)$
\\
Local Bounds Computation:
$\locbound: \absevent \to \mathcal{VAR} \cup \constdom$.
%
\[ 
\begin{array}{ll}
  \locbound(\absevent) \triangleq 1 
  & \absevent \notin SCC(\absG(c))
  \\
  \locbound(\absevent) \triangleq (x, v) 
  & \absevent \in SCC(\absG(c)) \land \absevent \in \dec(x) \land  \absevent = (\_, \_ , x' \leq x - v) \\
  \locbound(\absevent) \triangleq (x, \max(\dec(x))) 
  & \absevent \in SCC(\absG(c)) \land 
  \absevent  \notin \bigcup_{x \in \mathcal{VAR}} \dec(x)
  \land \absevent \notin SCC(\absG(c) \setminus \dec(x)) 
\end{array}
  \]
  The first case is straightforward. Since variable's visiting time outside of any while loop is at most 1, we do not need to analyze the visiting times of every node in the graph from phase 1.
  The second and third step is guaranteed by the \emph{Discussion on Soundness} in Section 4 of \cite{sinn2017complexity}.
  Then soundness proof is in Lemma~\ref{lem:local_bound_sound} in Appendix~\ref{apdx:reachability_soundness}.
%
\paragraph*{Abstract Event Closure and Bound Invariant}
Then, computing the bound invariants for variables and the transition closures for abstract events:
\\ 
$ \varinvar: \mathcal{VAR} \cup \constdom \to EXPR(\constdom)$
\\
$\absclr: \absevent \to EXPR(\constdom)$
\\
Then, the symbolic invariant for each variable 
as well as the symbolic transition closure for each transition is calculated as follows:
\[ 
\begin{array}{lll}
  \varinvar(x) & \triangleq c & c \in \constdom \\
  \varinvar(x) & \triangleq Incr(v) + \max(\{\varinvar(a) + c | (t, a, c) \in \reset(x)\}) & c \notin \constdom
\end{array}
\]
%
\begin{defn}
  \label{def:transition_closure_base}
\[ 
\begin{array}{lll}
  \absclr(\absevent) 
  & \triangleq x / v & \\ 
  & \locbound(\absevent) = (x, v) \in \constdom \times \mathbb{N} & \\
  \absclr(\absevent) 
  & \triangleq (Incr(x) + 
  \sum\limits_{(\absevent', y, v') \in \reset(x)}
  \absclr(\absevent') \times \max(\varinvar(y) + v', 0) ) / v & \\
  & \locbound(\absevent) = (x, v) \land x \notin \constdom & 
\end{array}
  \]
\end{defn}
%
\paragraph*{Improved Variable Modification Tracking}
Instead of just identifying the abstract events where each variable is reset,
this improvement identifies the chain of the events where a given variable is reset by the 
variables of the abstract events through the chain.
\\
$\resetchain: \mathcal{VAR} \to \mathcal{P}(\mathcal{P}(\absevent)) $
The set of the chain of abstract events where the variable is reset through the chain.
% \\
% $Incr(v) \triangleq \sum\limits_{(\absevent, c) \in \inc(v)}\{\absclr(\absevent) \times v\}$
%
\paragraph*{Improved Bound Invariant Computation}
Then, computing the bound invariants for variables and the transition closures for abstract events:
\\ 
$ \varinvar: \mathcal{VAR} \cup \constdom \to EXPR(\constdom)$
\\
$\absclr: \absevent \to EXPR(\constdom)$
\\
Then, the symbolic invariant for each variable 
as well as the symbolic transition closure for each transition is calculated as follows:
\[ 
\begin{array}{lll}
  \varinvar(x) & \triangleq c & c \in \constdom \\
  \varinvar(x) & \triangleq Incr(v) + \max(\{\varinvar(a) + c | (t, a, c) \in \reset(x)\}) & c \notin \constdom
\end{array}
\]
%
\begin{defn}
  \label{def:transition_closure}
\[ 
\begin{array}{lll}
  \absclr(\absevent) 
  & \triangleq x / v & \\ 
  & \locbound(\absevent) = (x, v) \in \constdom \times \mathbb{N} & \\
  \absclr(\absevent) 
  & \triangleq \Big(
    \sum\limits_{y \in \{ y ~|~ 
    ch \in \resetchain(x), (l_1, x, y, v, l_2) \in ch \} } Incr(x) & \\
    & \quad + 
  \sum\limits_{ch \in \resetchain(x)}
  \big( \min\limits_{\absevent' \in ch}({\absclr(\absevent')}) \times 
  \max(\varinvar(y) + \sum\limits_{(l_1, x, y, v, l_2) \in ch } v, 0)\big) \Big) / v & \\
  & \locbound(\absevent) = (x, v) \land x \notin \constdom & 
\end{array}
  \]
\end{defn}
  %
% \paragraph*{Adding the Reachability Bounds for Every Vertex in the Data-Control Flow Graph}
% Updating the weight of every vertex in the $\progG(c) = (\progV, \progE, \progW, \progF)$ for program $c$ generated from phase 1. 
% For every $x^l \in \progV$, find the abstract event $\absevent \in \absflow(c)$ of the form $(l, \_, \_)$, updating the $\progW(x^l) $ by the transition closure of this event.
% \\
$
\progW(x^l) 
  \triangleq \absclr(\absevent)
$
\paragraph*{Reachability Bound for Every program Label}
Through the transition closure computed above, 
The weight of every label in 
% Then we update 
the program $c$'s abstract control flow graph,
$\absG(c) =(\absV, \absE, \absW)$
is 
computed as the maximum over all the abstract events $\absevent \in \absE$ heading out from this vertex, formally as follows.
% by annotating each vertex with a symbolic weight. 
% This weight corresponds to 
%reachability bounds of
\\
$\absW 
\triangleq \left\{ (l, w) \in \mathbb{N} \times EXPR(\constdom) | w = \max\limits_{\absevent = (l, \_, \_)} \{ \absclr(\absevent)\} \right\}$.
% \\
\paragraph{Vertex Weight Computation}
Then compute the weight for each vertex in $\progV$ as follows,
\highlight{
% :
% \\
 \[\progW(c) \triangleq
   \left\{ (x^l, w) \in  \mathcal{LV} \times EXPR(\constdom)
\mid
x^l \in \progV(c) \land w = \absW(l)
\right\}
\]
}
%
\subsubsection{Edge Estimation}
\label{sec:alg_edgegen}
We show how to estimation the edge in this part.
\\
In this step,
 $\THESYSTEM$ performs a feasible data-flow analysis 
 through the reachable definition algorithm,
%  and then Then we generated the set of feasible data-flow between labeled variables based on that.
and generates the set of feasible data-flow between labeled variables.
\\
%  By generating set of all the reachable variables at location of label $l$ in the program $c$.
% For every labelled variable $x^l$ in this set, 
% the value assigned to that variable
% in the assignment command associated to that label is reachable at the entry point of  executing the command of label $l$.
% \\
In the first step, 
it performs the standard reaching definition analysis given a program $c$, on its every label $l$.  This step generates set of all the reachable variables at location of label $l$ in the program $c$.
The $\live(l, c)$ represent the analysis result, which is the set of 
reachable labeled variables in program $c$ at the location of label $l$.
For every labelled variable $x^l$ in this set, 
the value assigned to that variable
in the assignment command associated to that label is reachable at the entry point of  executing the command of label $l$.
% \\
% it performs the standard reaching definition analysis given a program $c$, on its every label $l$.
% \\
% Another operator \mathsf{blocks} 
The block, 
is either the command of the form of assignment, skip, or a test of the form of $[b]^{l}$, 
% and $block$ of program $c$ is 
denoted by $\mathsf{blocks}(c)$
the set of all the blocks 
in program $c$, where  $\mathsf{blocks}: \cdom \to \mathcal{P}(\cdom \cup \clabel{\bexpr}^{l})$.
Then it generates the set of feasible data-flow between labeled variables with detail in Definition~\ref{def:feasible_flowsto}, 
based on $\live(l, c)$ for every label in a program $c$ and its blocks $\mathsf{blocks}$.
\\
The details are as follows.
%
% Performing a feasible data-flow analysis through the reachable definition algorithm. 
%  By generating set of all the reachable variables at location of label $l$ in the program $c$.
% For every labelled variable $x^l$ in this set, 
% the value assigned to that variable
% in the assignment command associated to that label is reachable at the entry point of  executing the command of label $l$.
% \paragraph{Generate CFG}
%  \begin{def}
%   \label{def:init_label}
%   Define $\mathsf{init}$: Command -> label, which returns the initial label of the statement. 
% \[
%  \begin{array}{ll}
%     init([x := e]^{l})  & = l  \\
%      init([x := q(e)]^{l})  & = l \\
%      init([skip]^{l})  & = l \\
%      init([if [b]^l then C_1 else C_2]^{l})  & = l \\
%      init([while [b]^l do C]^{l})  & = l \\
%      init(C_1 ; C_2)  & = init(C_1) \\
%  \end{array}
%  \]
% \end{def}
%   Define $\mathsf{final}$: Command -> Powerset(label), which returns the final labels of the statement. 
%  \[
%  \begin{array}{ll}
%     final([x := e]^{l})  & = \{l\}  \\
%      final([x := q(e)]^{l})  & = \{l\}  \\
%      final([skip]^{l})  & = \{l\} \\
%      final([if [b]^l then C_1 else C_2]^{l})  & = final(C_1) \cup final(C_2) \\
%      final([while [b]^l do C]^{l})  & = \{l\} \footnote{while terminates after b evaluates to false} \\
%      final(C_1 ; C_2)  & =  final(C_2) \\
%  \end{array}
%  \]
\paragraph*{Blocks and Defs}
 Define block B to be either the command of the form of assignment, skip, or test of the form of $[b]^{l}$.\\
 Define $\mathsf{blocks}$ : command -> Powerset(Block)
 \[
 \begin{array}{ll}
    blocks([x := e]^{l})  & = \{[x := e]^{l}\}  \\
     block([x := q(e)]^{l})  & = \{[x := q(e)]^{l}\}  \\
     blocks([skip]^{l})  & = \{[skip]^{l}\} \\
     blocks([if [b]^l then C_1 else C_2]^{l})  & = {[b]^{l}} \cup blocks(C_1) \cup blocks(C_2) \\
     blocks([while [b]^l do C]^{l})  & = \{[b]^{l}\} \cup blocks(C) \\
     blocks(C_1 ; C_2)  & = blocks(C_1) \cup  blocks(C_2) \\
 \end{array}
 \]
 Define $\mathsf{labels}$ to get the labels of blocks.
 \[
   labels(C) = \{l | [B]^{l} \in blocks(C) \}
 \]  

% The control flow graph is generated by edges between labels. Define $\mathsf{flow}$: command -> P (label $\times$ label ).

% \[
%  \begin{array}{ll}
%     flow([x := e]^{l})  & = \emptyset  \\
%      flow([x := q(e)]^{l})  & = \emptyset  \\
%      flow([skip]^{l})  & = \emptyset \\
%      flow([if [b]^l then C_1 else C_2)  & =  flow(C_1) \cup flow(C_2)\cup \{(l, init(C_1)) , (l, init(C_2)) \} \\
%      flow([while [b]^l do C)  & =  flow(C) \cup \{(l, init(C)) \} \cup \{(l', l)| l' \in final(C) \} \\
%      flow(C_1 ; C_2)  & = flow(C_1) \cup  flow(C_2) \cup \{ (l,init(C_2)) | l \in final(C_1) \} \\
%  \end{array}
%  \]
 
 \paragraph{Reaching definition analysis}
 Set $?$ to be undefined, $label^{?}$ is label $\cup \{?\}$.\\
 Define $\mathsf{kill}$: $blocks \to \mathcal{P}(\mathcal{VAR} \times LABEL \cup \{?\})$, which produces the set of labelled variables of assignment destroyed by the block.
 \\
  Define $\mathsf{gen}$: $blocks \to \mathcal{P}(\mathcal{VAR} \times LABEL \cup \{?\})$, which generates the set of labelled variables generated by the block.
  \\\
  Define $defs(x)(c): \mathcal{VAR} \to LABEL$, gives all the labels where assigns value to variable x in the target program $c$. \[
 \begin{array}{ll}
    kill([x := e]^{l})  & = \{ (x, ?) \} \cup \{ (x, l') | l' \in defs(x) \} \\
     kill([x := q(e)]^{l})  & = \{ (x, ?) \} \cup \{ (x, l') | l' \in defs(x) \}  \\
     kill([skip]^{l})  & = \emptyset \\
     kill([ [b]^l ]^{l})  & =  \emptyset
 \end{array}
 ~~
  \begin{array}{ll}
      gen([x := e]^{l})  & = \{ (x, l) \}  \\
     gen([x := q(e)]^{l})  & = \{ (x, l) \}  \\
     gen([skip]^{l})  & = \emptyset \\
     gen([ [b]^l ]^{l})  & =  \emptyset 
 \end{array}
 \]
 Define $in(l)$, $out(l)$: LABEL$ \to \mathcal{VAR} \times LABEL \cup \{?\}$ for every block in program $c$ is computed as follows,
 \[
 \begin{array}{lll}
    in(l)  & = \{ (x, ?) | x^l \in \lvar_c \land  l = \absinit(c) \}  
    \cup \{ out(l')|  | (l',\_, l) \in \absE \land  l \neq \absinit(c)\}  \\
     out(l)  & =  gen(B^{l}) \cup \{ in(l) \setminus kill(B^l)  \} & B^l \in blocks(c)   
 \end{array}
 \]
 computing $in(l)$ and $out(l)$ for every $B^l \in blocks(c) $, and repeating these two step
until the $in(l)$ and $out(l)$ are stabilized for every $B^l \in blocks(c) $
We use $\live_{in}(l,c)$ and $\live_{out}(l, c)$ denote the stabilized results for the command of label $l$ in program $c$. 
\\
The $\live_{in}(l,c)$ and $\live_{out}(l, c)$ is computed by the Standard worklist algorithm with detail as below.
\begin{enumerate}
    \item initial in[l]=out[l]=$\emptyset$
    \item initial in[entry label] = $\emptyset$
    \item initialize a work queue, contains all the blocks in C
    \item while |W| != 0 \\
         pop l in W\\
          old = out[l]\\
          in(l) =  out(l') where (l',l) in flow(C)\\
           out(l) = gen($b^l$) $\cup$ (in(l) - kill($b^l$) ) where $b^l$ in block(C)   \\
          if (old != out(l)) W= W $\cup$ \{l'| (l,l') in flow(C)\}\\
          end while
\end{enumerate}
%
computing $in(l)$ and $out(l)$ for every $B^l \in blocks(c) $, and repeating these two step
until the $in(l)$ and $out(l)$ are stabilized for every $B^l \in blocks(c) $
We use $\live_{in}(l,c)$ and $\live_{out}(l, c)$ denote the stabilized results for the command of label $l$ in program $c$. 
The $\live_{in}(l,c)$ and $\live_{out}(l, c)$ is computed by the Standard worklist algorithm. (For simplicity, we use $\live(l,c)$ to represent $\live_{in}(l,c)$ in the other part of the paper.
%%
\paragraph{Feasible Data-Flow Generation}
by using the results of Reaching definition analysis results, specifically $\live(l, c)$ for every label in a program $c$, we refine the vertices and edges in the $\absG$ graph 
by generating the set of feasible data-flow between labeled variables as follows,
%
%   \[
%  \begin{array}{ll}
%     dcdg([x := e]^{l})  & = \{ (y^i, x^l) | y \in VAR(e) \land (y,i) \in \live_{in}(l) \}  \\
%      dcdg([x := q(e)]^{l})  & = \{ (y^i, x^l) | y \in VAR(e) \land (y,i) \in \live_{in}(l) \}  \\
%      dcdg([skip]^{l})  & = \emptyset \\
%      dcdg([if [b]^l then C_1 else C_2)  & =  dcdg(c_1) \cup dcdg(c_2)\\ & \cup \{(x^i,y^j) | x \in VAR(b) \land (x,i) \in \live_{in}(l) \land ([y = \_]^j) \in blocks(c_1) \} \\
%      &\cup \{(x^i,y^j) | x \in VAR(b) \land (x,i) \in \live_{in}(l) \land ([y = \_]^j) \in blocks(c_2) \} \\
%      dcdg([while [b]^l do c)  & =  dcdg(c) \cup \{(x^i,y^j) | x \in VAR(b) \land (x,i) \in \live_{in}(l) \land ([y = \_]^j) \in blocks(C) \} \\
%      dcdg(c_1 ;c_2)  & = dcdg(c_1) \cup  dcdg(c_2) \\
%  \end{array}
%  \]
%
\begin{defn}[Feasible Data-Flow]
  \label{def:feasible_flowsto}
  Given a program $c$ and two labeled variables $x^i, y^j$  in this program, 
  $\flowsto(x^i, y^j, c)$ is 
    {\footnotesize
    \[
   \begin{array}{ll}
    \flowsto(x^i, y^j, \clabel{\assign{x}{\expr}}{}^l)  & \triangleq (x^i, y^j) \in \{ (y^i, x^l) | y \in FV(e) \land (y,i) \in \live_{in}(l, \clabel{\assign{x}{\expr}}^l) \}  \\
    \flowsto(x^i, y^j, \clabel{\assign{x}{\query(\qexpr)}}{}^l)  & \triangleq (x^i, y^j) \in \{ (y^i, x^l) | y \in FV(e) \land (y,i) \in \live_{in}(l,\clabel{\assign{x}{\query(\qexpr)}}^l) \}  \\
    \flowsto(x^i, y^j, [\eskip]^{l})  & = \emptyset \\
    \flowsto(x^i, y^j, \eif [b]^l \ethen c_1 \eelse c_2)  & \triangleq \flowsto(x^i, y^j, c_1) \lor \flowsto(x^i, y^j, c_2) \\ 
        & \lor (x^i, y^j) \in
       \{(x^i,y^j) | x \in FV(b) \land (x,i) \in \live_{in}(l) \land ([y = \_]^j) \in blocks(c_1) \} \\
       &\lor (x^i, y^j) \in \{(x^i,y^j) | x \in FV(b) \land (x,i) \in \live_{in}(l) \land ([y = \_]^j) \in blocks(c_2) \} \\
       \flowsto(x^i, y^j, \ewhile [b]^l \edo c_w)  & \triangleq  \flowsto(x^i, y^j, c_w) \lor (x^i, y^j) \in  \{(x^i,y^j) | x \in FV(b) \land (x,i) \in \live_{in}(l) \land ([y = \_]^j) \in blocks(c_w) \} \\
       \flowsto(x^i, y^j, c_1 ;c_2)  & \triangleq \flowsto(x^i, y^j, c_1) \lor \flowsto(x^i, y^j, c_2) \\
   \end{array}
   \]
   }
   \end{defn}
%
\paragraph*{Edges Generation}
\highlight{
  \[
    \progE(c) \triangleq 
    \left\{ 
    ({x}_1^{i}, {x}_2^{j}) \in \mathcal{LV} \times \mathcal{LV}
    ~ \middle\vert ~
    \begin{array}{l}
      {x}_1^{i}, {x}_2^{j} \in \progV(c)
    \land
      % \\
      \exists n \in \mathbb{N}, z_1^{r_1}, \cdots, z_n^{r_n} \in \lvar_{{c}} \st 
      n \geq 0 \land
      \\
      \flowsto(x^i,  z_1^{r_1}, c) 
      \land \cdots \land \flowsto(z_n^{r_n}, y^j, c) 
    \end{array}
    \right\}
    \]
}
%  \begin{defn}[Feasible Data-Flow]
%   \label{def:feasible_flowsto}
%     {\footnotesize
%     \[
%    \begin{array}{ll}
%       dcdg(\clabel{\assign{x}{\expr}}{}^l)  & = \{ (y^i, x^l) | y \in FV(e) \land (y,i) \in \live_{in}(l, \clabel{\assign{x}{\expr}}^l) \}  \\
%        dcdg(\clabel{\assign{x}{\query(\qexpr)}}{}^l)  & = \{ (y^i, x^l) | y \in FV(e) \land (y,i) \in \live_{in}(l,\clabel{\assign{x}{\query(\qexpr)}}^l) \}  \\
%        dcdg([\eskip]^{l})  & = \emptyset \\
%        dcdg([\eif [b]^l \ethen c_1 \eelse c_2)  & =  dcdg(c_1) \cup dcdg(c_2)\\ & \cup 
%        \{(x^i,y^j) | x \in FV(b) \land (x,i) \in \live_{in}(l) \land ([y = \_]^j) \in blocks(c_1) \} \\
%        &\cup \{(x^i,y^j) | x \in FV(b) \land (x,i) \in \live_{in}(l) \land ([y = \_]^j) \in blocks(c_2) \} \\
%        dcdg([\ewhile [b]^l \edo c)  & =  dcdg(c) \cup \{(x^i,y^j) | x \in FV(b) \land (x,i) \in \live_{in}(l) \land ([y = \_]^j) \in blocks(C) \} \\
%        dcdg(c_1 ;c_2)  & = dcdg(c_1) \cup  dcdg(c_2) \\
%    \end{array}
%    \]
%    }
%    \end{defn}
%    For any two labeled variables $x^i, y^j$ in a program $c$, 
%   %  it is easy to see that there is a one-on-one correspondence between 
%   %  $\flowsto$ relation of the two variables, and the $dcdg$ analysis result on $c$.
%   we use $\flowsto()$ denote if they have a feasible data-flow relation in Definition~\ref{def:flowsto}.
%    \begin{defn}[Feasible Data-Flow ($\flowsto$)]
%    \label{def:flowsto}
%    \[
%    \forall c \in \cdom, x^i, y^j \in \lvar_c \st 
%    \flowsto(x^i, y^j, c) \iff (x^i, y^j) \in dcdg(c)
%    \]
%    \end{defn}
  %  This soundness is proved in Proof~\ref{pf:rd_soundness} in Appendix~\ref{apdx:rd_soundness}.
  %  For any two labeled variables in a program $c$, it is easy to see that there is a one-on-one correspondence between 
  %  $\flowsto$ relation of the two variables, and the $dcdg$ analysis result on $c$.
  %  \begin{thm}[Soundness of the Feasible Data-Flow Analysis]
  %  \label{thm:rd_soundness}
  %  \[
  %  \forall c \in \cdom, x^i, y^j \in \lvar_c \st 
  %  \flowsto(x^i, y^j, c) \iff (x^i, y^j) \in dcdg(c)
  %  \]
  %  \end{thm}
  %  This soundness is proved in Proof~\ref{pf:rd_soundness} in Appendix~\ref{apdx:rd_soundness}.
   \subsection{Program-Based Data Dependency Graph Generation}
  %  Weighted Data Dependency Graph Generation}
   \label{sec:alg_graphgen}
   %
%    Each directed edge represents an abstract transition 
%    between two control locations, i.e., the labels of two commands (we call the labels also control location and they refer to the same thing in the follows), 
%    where the second labeled command will be executed after execution of the command with first label.
%    The abstract transition contains a set of difference constraints for variables, generated by abstracting the command of the first label.
%   \item Computing 
%   % we get the reachability bound for each command.
%   the symbolic reachability bound for each command,
%   % the value bound invariant for each variable in the event and 
%   by inferring the value bound invariant for each variable 
%   % the event transition closure over the abstract control flow graph,
%   and the transition closure for every abstract transition through the constraints over the abstract control flow graph.
%   % \\
%   % Through this graph and constraint for every transition, we infer the  invariant for every variable,
%   % and compute the transition closure for every abstract transition.
%   % By solving the closure with the invariants of variables involved in this closure for every transition, 
%   % we compute the symbolic reachability bound of every commands corresponding to 
% %     % this transition.
% %     \item Performing a feasible data-flow analysis from the reachable definition algorithm. 
% % %  By generating set of all the reachable variables at location of label $l$ in the program $c$.
% % and generating the set of all the reachable variables for every program location.
% % For every labelled variable $x^l$ in this set, 
% % the value assigned to that variable
% % in the assignment command associated to that label is reachable at the entry point of  executing the command of label $l$.
% % \item Refining the abstract control flow graph into a weighted-data dependency graph, 
% % by annotating each vertex with reachability bounds and 
% % removing unfeasible edges and redundant edges and vertices.
% % adding edges between
% %     variables having data-flow relations, and
% % removing the edges between locations where the variables associated to that labeled command isn't reachable from the second location.
% % \\
% % first annotate each vertex of label $l$ with the variable 
% % assigned in that labeled command, and remove the rest doesn't correspond to an assignment command.
% % Then 
% % add direct edge between two labeled variables,
% % where the first variable 
% % is directly used in the assignment expression to the second variable, by restricting 
% % the first labeled variable is reachable at the the second label.
% %
% \item Computing the adaptivity through this weighted data dependency graph,
%   by finding a finite walk on this weighted graph, 
% traversing the maximum times of query variables, by restricting the visiting time of every vertex on this walk to its weight.
% The maximum number of vertices corresponding to a query variables visited on this walk is the estimated upper bound, for program's adaptivity.

%    In this step, $\THESYSTEM$ refines the abstract control flow graph into the program-based weighted-data dependency graph, 
% by annotating each vertex with reachability bounds and 
% removing unfeasible edges and redundant edges and vertices,
% % This graph is used 
% for approximating the trace-based weight-data dependency graph.
% \\
% Specifically, we first annotate each vertex of label $l$ with the variable 
% assigned in that labeled command, and remove the rest doesn't correspond to an assignment command.
% Then 
% add direct edge between two labeled variables,
% where the first variable 
% is directly used in the assignment expression to the second variable, by restricting 
% the first labeled variable is reachable at the second label.
% % \\
% The formal definition is as follows.
Finally we build the estimated data dependency graph based on program static analysis as follows:
\\
\highlight{
  \[
    \progG(c) = (\progV(c), \progE(c), \progW(c), \progF(c))
    \]
}
with $\progV(c), \progE(c), \progW(c)$ as computed in each steps above,\\
and $\progF(c) =\left\{(x^l, n)  \in  \mathcal{LV} \times \{0, 1\} 
~ \middle\vert ~
x^l \in \lvar_{c},
n = 1 \iff x^l \in \qvar_{c} \land n = 0 \iff  x^l \in \qvar_{c} .
\right\} $,
% The algorithm computation is 
formally as follows,
% Through the reachable definition set on every label,
% we remove the edges between labels where the variables associated to that labeled command isn't reachable from the second location.
%\absG(c) =(\absV, \absE, \absW)
\begin{defn}
  [Program-Based Dependency Graph]
  \label{def:prog_graph}
  % [Program-Based Weighted Data Dependency Graph Generation Algorithm]
% \label{def:analyz_dcfg}
Given a program $c$, with its abstract weighted control flow graph $\absG(c) = (\absV, \absE, \absW)$ and 
feasible data flow relation $\flowsto(x^i, y^j, c)$ for every $x^i, y^j \in \lvar_c$, its Program-Based Weighted Data Dependency Graph
$\progG(c) = (\progV, \progE, \progW, \progF)$,
is generated as follows,
% \\
% \highlight{
% $\progV =\{x^l | x^l \in \lvar_c\} $
% \\
% $\progE = \{(y^i, x^l) | (y^i, x^l)  \in dcdg(c) \}$
% \\
% $\progW = \{(x^l, w ) | (l, w ) \in \absW \land x^l \in \lvar_c\}$
% \\
% $\progF = \{(l, q) \in \mathcal{L} \times \{0, 1\}| q = 1 \iff l \in \qvar_c, q = 0 \iff l \notin \qvar_c \}$.
% }
% \end{defn}
% \begin{defn}
  % [Program-Based Dependency Graph].
  % \label{def:prog_graph}
%   % \\
% Given a program ${c}$
% its program-based graph 
% $\progG({c}) = (\vertxs, \edges, \weights, \qflag)$ is defined as:
{\footnotesize
\[
\begin{array}{rlcl}
\text{Vertices} &
\progV & := & \left\{ 
x^l \in \mathcal{LV} 
~ \middle\vert ~
x^l \in \lvar_{c}
\right\}
\\
\text{Directed Edges} &
\progE & := & 
\left\{ 
({x}_1^{i}, {x}_2^{j}) \in \mathcal{LV} \times \mathcal{LV}
~ \middle\vert ~
\begin{array}{l}
  {x}_1^{i}, {x}_2^{j} \in \vertxs
\land
  % \\
  \exists n \in \mathbb{N}, z_1^{r_1}, \cdots, z_n^{r_n} \in \lvar_{{c}} \st 
  n \geq 0 \land
  \\
  \flowsto(x^i,  z_1^{r_1}, c) 
  \land \cdots \land \flowsto(z_n^{r_n}, y^j, c) 
\end{array}
\right\}
\\
\text{Weights} &
\progW & := &
% \bigcup
% \begin{array}{l}
\left\{ (x^l, w) \in  \mathcal{LV} \times EXPR(\constdom)
\mid
x^l \in \lvar_{{c}} \land w = \absW(l)
\right\}
% \end{array} 
\\
\text{Query Annotation} &
\progF & := & 
\left\{(x^l, n)  \in  \mathcal{LV} \times \{0, 1\} 
~ \middle\vert ~
x^l \in \lvar_{c},
n = 1 \iff x^l \in \qvar_{c} \land n = 0 \iff  x^l \in \qvar_{c} .
\right\}
\end{array}
\] }
\end{defn}
% In last phase, we get a dependency graph whose node is computation blocks uniquely decided by its label. In this phase, we want to add more information to every node in the graph, which is the approximated visiting times (how many times this block is exeucted). The algorithm is defined in Algorithm~\ref{alg:add_weights}, with 3 main functions, the Control Location Insertion, 
% Program Transition Abstraction, 
% Bound Calculation 
% and ADDWEIGHTs.
% \begin{algorithm}
% \caption{
% {Add weights to dependency graph (the main algorithm of phase 2)}
% \label{alg:add_weights}
% }
% \begin{algorithmic}[1]
% \REQUIRE the program $c$, the dependency graph $G = (\vertxs, \edges, \weights, \qflag)$ from phase 1
% \STATE  prel = PREPROCESSING(c) 
% \STATE {\bf for} $x^l \in \vertxs$: 
% \STATE \qquad {\bf if} $l \in prel$:
% \STATE \qquad \qquad $\weights(x^l) = \rb(c, l)$
% \STATE \qquad {\bf else}:
% \STATE \qquad \qquad $\weights(x^l) = 1$
% \RETURN $G$
% \end{algorithmic}
% \end{algorithm}

% \paragraph{Getting Reachability Bounds}
% This is defined in Algorithm~\ref{alg:rb}.
% To be precise, we use static analysis method from \cite{Sumit2010rechability}, which is able to "provide the symbolic worst case bound on the number of times a block is reached", let us call it reachability bound analysis. 

% This analysis only works to find the symbolic bound of one block (in our graph, it corresponds to one node). This algorithm is summarized as follows.
% \begin{enumerate}
%     \item Build a transition system which describes the relation between variables in this target block and these variables in the successive visit to this block. There are defined translate functions which translate the statement to transition system and the corresponding operations such as the composition of transition systems, merging two transition systems from two control branches and so on. It is worth to mention that it calculates the transitive closure of the transition system obtained from a loop body, which can be analogy to computing the invariant of a loop.  
%     \item Use Ranking function which takes the transition system and outputs the bound
% \end{enumerate}

% \begin{algorithm}
% \caption{
% {Reachability Bound Analysis ($\rb$)}
% \label{alg:rb}
% }
% \begin{algorithmic}[1]
% \REQUIRE the program $C$, the target while loop with label $l$.
% \STATE  T  = GenerateTransitionSystem(C,l) 
% \STATE B = 1 + ComputeBound(T)
% \RETURN B
% \end{algorithmic}
% \end{algorithm}

% The algorithm $GenerateTransitionSystem(C,l)$ can be described as follows. It uses the control flow graph generated from the program $C$, and splits the node marked by $l$ into two nodes $l_1$ and $l_2$ to generate a new control flow graph from the $l_1$ to $l_2$. It translates the node in the graph into a transition systems by the its translate function and replaces node with transition systems. For loops in the graph, the loop itself is replaced by the transitive closure of the transition systems of its body. Finally, the new generated control flow graph can be transformed to a transition system. The transition system is a disjunction of transitions, and every transition is expressed as a conjuction of formulas over program variables $x,y,z$ in the target block (l) and its successive visits $x',y',z'$ in the same block.
% \\
% The transition formula for each command are as follows:
% \\
% \todo{adding the naive transition formula for assignment and if}
% \\
% \jl{$translate(c)$  is defined as follows:}
% \\
% $translate(\assign{x}{e}) = \{x := e\} if x \in VAR(b)$
% \\
% % $translate(\assign{x}{e}) = \{\} if x \notin VAR(b)$
% % \\
% $translate(\assign{x}{\qexpr}) = \{x \in VAR(b) \implies x := \qexpr\} $
% \\
% % $translate(\assign{x}{\qexpr}) = \{\} if x \notin VAR(b)$
% % \\
% $translate(\eif(\bexpr, c_1, c_2)) = \{\bexpr \implies translate(c_1), \neg\bexpr \implies translate(c_2)\}$
% \\
% $translate(\ewhile(\bexpr, c_w)) = \{ ComputeBound(GenerateTransitionSystem(C,l)) \times translate(c_w) \}$
% \\
% $translate(c_1;c_2) = translate(c_1) + translate(c_2)$
% \\
% \todo{adding the compose method for composing 2 transition formulas}
% $compose\{x := e_1, \cdots, x := e_2 \} = \{x := e_2\}$
% \\
% $\{y := e_1, \cdots, x := e_2 \} \land \flowsto(y, x, c) = \{x := [y \to e_1] e_2\}$
% \\
% \todo{add the $ComputeBound$ function, specifically the ranking function}
% \\
% The function $computeBound$ takes into a transition system (a disjuction of transitions), and computes the bound. There are ranking functions which take a transition and return the bound, that is used by $computeBound$. There are some heuristics in compute the bound based on the transitions systems, if interested, please look at the paper for more details.

% \paragraph{Add Weight} We also need to take care about the situation when a bound can not be predicted by {$\rb$}, we need to use another loose analysis to get a loose bound.



\clearpage
\subsection{Adaptivity Upper Bound Computation}
%  from refined weighted-labeled data-flow graph}
\label{sec:alg_adaptcompute}
This phase compute the adaptivity upper bound for a program $c$,
% Given a program ${c}$, we generate
based on its program-based data dependency graph 
$\progG({c}) = (\vertxs, \edges, \weights, \qflag)$. 
%
Defined in Definition~\ref{def:prog_adapt} as 
%
% Then the adaptivity bound based on program analysis for ${c}$ 
% is the number of query vertices on a finite walk in $\progG({c})$. This finite walk satisfies:
% \begin{itemize}
% \item the number of query vertices on this walk is maximum
% \item the visiting times of each vertex $v$ on this walk is bound by its reachability bound $\weights(v)$.
% \end{itemize}
the maximum query length over all finite walks in $\walks(\progG({c}))$, and computed 
% is computed as the maximum query length over all finite walks in $\walks(\progG({c}))$, and computed 
in Algorithm~\ref{alg:adpt_alg}.
%
% It is formally defined in \ref{def:prog_adapt}.
% defined formally as follows.
%
%
\begin{defn}
[{Program-Based Adaptivity}].
\label{def:prog_adapt}
\\
{
Given a program ${c}$ and its program-based graph 
$\progG({c}) = (\vertxs, \edges, \weights, \qflag)$,
%
the program-based adaptivity for $c$ is defined as%
\[
\progA({c}) 
:= \max
\left\{ \qlen(k)\ \mid \  k\in \walks(\progG({c}))\right \}.
\]
}
\end{defn} 

The following algorithm finds the walk with the longest query length on a program $c$'s execution-based dependency graph 
$\progG(c) = (\vertxs, \edges, \weights, \qflag)$, through a combination of 
% DFS and BFS algorithm 
deep first search and breath first search strategy
% as defined 
in Algorithm~\ref{alg:adpt_alg} and Algorithm~\ref{alg:adaptscc}.

\paragraph*{Challenge}
Following is the challenge of computing the adaptivity on a program based dependency graph.
In order to search for the finite walk having the longest query length, which isn't a simple longest weighted path.
\\
The visiting times of every vertex on this walk should be no more than its weight, which is a symbolic expression.
So we cannot simply search for the longest weight path where the visiting times of the vertex on it could possibly exceed its weights.
We can neither simply traverse on this graph by decreasing the weight of every node by 1 after every visiting,
because the weight is symbolic and simply traversing leads to non-termination.
\\
We can simply adopt either a deep first strategy to estimate the adaptivity as the length of the longest weight path, as in Algorithm~\ref{alg:overadp_alg}.
However, this gives us over-approximation to a large extend.
% In Algorithm~\ref{alg:adpt_alg}, 
% we first find all the strong connected components of this graph, 

\begin{algorithm}
    \caption{
    {Longest Adaptivity Search Algorithm ($\pathsearch$)}
    \label{alg:adpt_alg}
    }
    \begin{algorithmic}[1]
    \REQUIRE $G = (\vertxs, \edges, \weights, \qflag)$ \#\{The program based dependency graph\}
    % with a start vertex $s$ and destination vertex $t$ .
    \STATE  {\bf {$\kw{\pathsearch(G)}$}:}  
    \STATE {\bf init} 
    % \\
    % current node: $c$, 
    \\
    $q$: empty queue.
    % \\
    % $\kw{visited}$: List of length $|\vertxs|$, initialize with $\efalse$.
    % \\
    % $\kw{SSCvisited}$: List of length $|\vertxs|$, initialize with $\efalse$.
    % \\ 
    % $\kw{adapt_{scc}(SCC_i) = \pathsearch_{scc}(SCC_i)}$.
    \\
    $\kw{adapt}$ : the adaptivity of this graph initialize with $0$.
    \\
    \STATE Find all Strong Connected Components (SCC) in $G$: $\kw{SCC_1}, \cdots, \kw{SCC_n}, 0 \leq n \leq |\vertxs|$, 
    % where $\kw{SCC_i} = (\vertxs_i, \edges_i, \weights_i, \qflag_i)$.
    % and assign each vertex $x^i$ with an SCC number $\kw{SCC}(x^i)$
    \STATE {\bf for} every SCC: $\kw{SCC_i}$, compute its Adaptivity $\kw{SCC_i}$:
    \STATE \quad $\kw{adapt_{scc}[SCC_i] = \pathsearch_{scc}(SCC_i)}$;
    \STATE {\bf for} every $\kw{SCC_i}$:
    \STATE \qquad $q.append(\kw{SCC_i})$;
    \STATE \qquad $\kw{adapt_{tmp}} = 0$;
    \STATE \qquad {\bf while} $q$ isn't empty:
    \STATE \qquad \qquad $\kw{s} = q.pop()$;  \#\{take the top SCC from head of queue\}
    \STATE \qquad \qquad  $\kw{adapt_{tmp}}_0= \kw{adapt_{tmp}}$; \#\{record the adaptivity of last level\}
    \STATE \qquad \qquad  $\kw{SCC_{max}}$;  \#\{record the SCC with longest walk in this level\}
    % initialize cycle-adapt = 0.
    \STATE \qquad \qquad {\bf for} every 
    % SCC having a directed edge from $s$ of $s$: $\kw{SCC'}$:
    % directed edge goes out of $\kw{s}$ and connects a 
    different SCC, $\kw{s'}$ connected by $\kw{s}$ by a directed edge from $\kw{s}$:
    % \STATE \qquad \qquad   cycle-adapt$ = \max($cycle-adapt, $\kw{dfs_{refine}(G, v, v)})$;
    % \STATE \qquad \qquad \qquad \#\{compute the adaptivity of vertex $v$  on $\kw{SCC}(v)$, and update r[v] with the SCC-adapt\}
    % \STATE \qquad \qquad \qquad $ r[v] = r[s] + \kw{dfs_{refine}(G, v, visited)})$; 
    \STATE \qquad \qquad \qquad {\bf if} $(\kw{adapt_{tmp}} < \kw{adapt_{tmp}}_0 + \kw{adapt_{scc}[s']})$:
    \STATE \qquad \qquad \qquad \qquad $\kw{adapt_{tmp}} = \kw{adapt_{tmp}}_0 + \kw{adapt_{scc}[s']}$; 
    \STATE \qquad \qquad \qquad \qquad $\kw{SCC_{max} = s'} $; \#\{update the SCC with longest walk in this level\} 
    % \STATE \qquad   $r[c] = r[c] + $cycle-adapt;
    % \STATE \qquad for all unvisited vertex $v$ having directed edge from c and $! \kw{cycle}(c)$:
    % \STATE \qquad \qquad $r[v] = r[c] + \flag(v)$; 
    % \STATE \qquad \qquad \qquad  \#\{mark all the nodes with the same $\kw{SCC}$ number as visited\} 
    % \STATE \qquad \qquad \qquad  \#\{append the unvisited vertex to the rear of the queue\}
    % \STATE \qquad \qquad \qquad  \#\{mark all the nodes with the same $\kw{SCC}$ number as visited\} 
    % \STATE \qquad \qquad for $v \in V$,   $\kw{visited}[s] = 1$;
    \STATE \qquad \qquad \qquad $q.append(\kw{SCC_{max}})$;
    \STATE \qquad $\kw{adapt} = \max(\kw{adapt}, \kw{adapt_{tmp}})$;    
    \RETURN $\kw{adapt}$.
    \end{algorithmic}
    \end{algorithm}
    %

    In Algorithm~\ref{alg:adpt_alg}, 
    it first finds all the strong connected components (SCC) of this graph using the Kosaraju’s algorithm in line:3.
    Every $\kw{SCC_1}, \cdots, \kw{SCC_n}, 0 \leq n \leq |\vertxs|$ is a sub-graph of $\progG(c)$, where $\kw{SCC_i} = (\vertxs_i, \edges_i, \weights_i, \qflag_i)$.
    % where $\kw{SCC_i} = (\vertxs_i, \edges_i, \weights_i, \qflag_i)$.
    Then, 
    % we compute the adaptivity on every SCC, which is a subgraph of the $\progG(c)$, in line:4-5 by Algorithm~\ref{alg:adaptscc}.
    it computes the adaptivity on every SCC
    % , which is a subgraph of the $\progG(c)$, 
    in line:4-5 by Algorithm~\ref{alg:adaptscc}.
    We guarantee the soundness of the adaptivity on SCC by Lemma~\ref{lem:sound_adaptalg_scc} with proof in Appendix~\ref{apdx:adaptalg_soundness}.
    The $\progG(c)$ is then shrank into an acyclic directed graph where 
    % vertices are all the SCCs and edges are between every SCCs with their adaptivities as weights.
    $\kw{SCC_1}, \cdots, \kw{SCC_n}$ are vertices with the adaptivities as weights.
    % , and directed edges are .
    For every $(v_i, v_j) \in \edges$ such that $v_1 \in \vertxs_i$, $v_j \in \vertxs_j$ and $i \neq j$,
    there is a edge $(s_i, s_j)$ in this shrank graph. \\ 
    Then, we use the standard breath first search strategy to find the longest weighted path
    %  w.r.t. all the SCCs and their adaptivities.
    on this shrank graph and return the length as adaptivity.
    \\
    We guarantee that 
    % this longest weighted path is a sound computation of the adaptivity on this,
    the length of this longest weighted path is a sound computation of the adaptivity for program $c$,
    % as well as 
    and this longest weighted path a sound computation of the finite walk having the longest query length 
    % on this graph, in Theorem~\ref{thm:sound_adaptalg}
    on $c$'s program based dependency graph, in Theorem~\ref{thm:sound_adaptalg}
    in Appendix~\ref{apdx:adaptalg_soundness}.
%    
    % for every vertex which isn't on any SCC, it is easy to know that it will be visited 
    % at most once given no edges going back to this vertex. We can know the adaptivity on the SCC 
     %
    % \begin{algorithm}
    % \caption{
    % {Longest Adaptivity Search Algorithm ($\pathsearch$)}
    % \label{alg:adpt_alg}
    % }
    % \begin{algorithmic}
    % \REQUIRE Weighted Directed Graph $G = (\vertxs, \edges, \weights, \flag)$ with a start vertex $s$ and destination vertex $t$ .
    % \STATE  {\bf {bfs $(G)$}:}  
    % \STATE {\bf init} 
    % \\
    % current node: $c$, 
    % \\
    % queue: $q$ : List, add into $a$ an arbitrary v from $\vertxs$. 
    % \\
    % visited: List of length $|\vertxs|$, initialize with $\efalse$.
    % \\
    % results: $r$ : List of length $|\vertxs|$, initialize with -1.
    % \\
    % curr$\kw{flowcapacity}$: INT, initialize MAXINT.
    % \\
    % querynum: INT, initialize 0. \#\{To count the query numbers when we are walking inside a cycle\}
    % \\
    % \STATE \qquad {\bf while} $q$ isn't empty:
    % \STATE \qquad \qquad take the vertex from head $c= q.pop()$
    % \STATE \qquad \qquad mark $c$ as visited, visited $[c] = 1$.
    % \STATE \qquad \qquad {\bf if} $\kw{cycle}(c)$  \#\{we are inside a cycle\}
    % \STATE \qquad \qquad \qquad curr$\kw{flowcapacity}$ = min($\weights$(c), curr$\kw{flowcapacity}$).
    % \STATE \qquad \qquad \qquad querynum += $\flag(c)$.
    % \STATE \qquad \qquad  \qquad for all unvisited vertex $v$ having directed edge from c:
    % \STATE \qquad \qquad \qquad \qquad r[v] = r[c]; q.add(v)
    % \STATE \qquad \qquad \qquad  {\bf if}  $v$ is visited, then the circle finished
    % \STATE \qquad \qquad \qquad \qquad update the result $r[v] =  \max(r[v], r[c] + $curr$\kw{flowcapacity}$*querynum)
    % \STATE \qquad \qquad \qquad \qquad curr$\kw{flowcapacity}$ = MAXINT
    % \STATE \qquad \qquad \qquad \qquad querynum = 0.  
    % \STATE \qquad \qquad {\bf else} 
    % \STATE \qquad \qquad \qquad for all unvisited vertex $v$ having directed edge from c:
    % \STATE \qquad \qquad \qquad  \qquad $r[v] = \max(r[v], r[c] + \flag(c))$; q.add(v)
    % \RETURN max($r$)
    % \end{algorithmic}
    % \end{algorithm}
    %
    \begin{algorithm}
      \caption{
      {Over-Approximated Adaptivity on SCC}
      \label{alg:overadp_alg}
      }
      \begin{algorithmic}[1]
      \REQUIRE $G = (\vertxs, \edges, \weights, \qflag)$ \#\{An Strong Connected Symbolic Weighted Directed Graph\}
      % with a start vertex $s$ and destination vertex $t$ .
      \STATE {\bf {$\kw{\pathsearch_{scc-naive}(G)}$}:}  
      \STATE {\bf init} 
      \\
      $\kw{r_{scc}}$: the Adaptivity of this SCC
      % \STATE  {\bf def} {$\kw{dfs_{naive}(G, c,visited)}$}: 
      % % \STATE {\bf init} 
      % % \\
      % % current node: $c$, 
      % % \\
      % % visited: List of length $|\vertxs|$, initialize with $\efalse$.
      % % \\
      % % \STATE {\bf if} $c = s$:
      % % \RETURN \qquad  $\weights(s)*\flag(s) $.
      % \STATE \qquad $r[c] = \weights(c)*\qflag(c) $
      % \STATE \qquad {\bf for}  all vertex $v$ having directed edge from $c$:
      % \STATE \qquad \qquad {\bf if}  $v$ is unvisited:
      % \STATE \qquad \qquad \qquad  \#\{mark $v$ as visited\} $\kw{visited}[v] = 1$;
      % \STATE \qquad \qquad \qquad $r[c] += \kw{dfs_{naive}(G, v, visited)}$;
      % \STATE \qquad {\bf else}: \#\{There is a cycle finished\}
      % \RETURN \qquad \qquad $\weights(v)*\flag(v) $.
      \STATE  {\bf for} every vertex $v$ in $\vertxs$:
      % \STATE  \qquad initialize \kw{visited} with $\efalse$.
      \STATE  \qquad $r_{scc} += \weights(v)*\qflag(v)$  
      \RETURN $r[c]$
      \end{algorithmic}
      \end{algorithm}
      %
%
    % \begin{algorithm}
    %     \caption{
    %     {Over-Approximated Adaptivity on SCC}
    %     \label{alg:overadp_alg}
    %     }
    %     \begin{algorithmic}
    %     \REQUIRE Weighted Directed Graph $G = (\vertxs, \edges, \weights, \qflag)$ with a start vertex $s$ and destination vertex $t$ .
    %     \STATE  {\bf {$\kw{dfs_{naive}(G, c,visited)}$}:}  
    %     % \STATE {\bf init} 
    %     % \\
    %     % current node: $c$, 
    %     % \\
    %     % visited: List of length $|\vertxs|$, initialize with $\efalse$.
    %     % \\
    %     % \STATE {\bf if} $c = s$:
    %     % \RETURN \qquad  $\weights(s)*\flag(s) $.
    %     \STATE $r[c] = \weights(c)*\qflag(c) $
    %     \STATE {\bf for}  all vertex $v$ having directed edge from $c$:
    %     \STATE \qquad {\bf if}  $v$ is unvisited:
    %     \STATE \qquad \qquad  \#\{mark $v$ as visited\} $\kw{visited}[v] = 1$;
    %     \STATE \qquad \qquad $r[c] += \kw{dfs_{naive}(G, v, visited)}$;
    %     % \STATE \qquad {\bf else}: \#\{There is a cycle finished\}
    %     % \RETURN \qquad \qquad $\weights(v)*\flag(v) $.
    %     \RETURN $r[c]$
    %     \end{algorithmic}
    %     \end{algorithm}%
        %
    \begin{algorithm}
            \caption{
            {Adaptivity on $\kw{SCC}$}
            \label{alg:adaptscc}
            }
            \begin{algorithmic}[1]
              \REQUIRE $G = (\vertxs, \edges, \weights, \qflag)$ \#\{An Strong Connected program based dependency Graph\}
            \STATE  {\bf {$\kw{\pathsearch_{scc}(G)}$}:}  
            \STATE {\bf init} 
            \\
            $\kw{r_{scc}}$: $EXPR(\constdom)$, initialized $0$, the Adaptivity of this SCC
            \STATE \qquad {\bf init} 
            % \STATE \qquad current node: $c$, 
            % \\
            % visited: List of length $|\vertxs|$, initialize with $\efalse$.
            % \\ \qquad  $\kw{r_{scc}}$ : initialize $0$, the adaptivity of this graph
            \\ \qquad  $\kw{visited}$ : $\{0, 1\}$ List, 
            \\ \qquad  \#\{length $|\vertxs|$, initialize with $0$ for every vertex, recording whether a vertex is visted.\}
            \\ \qquad  $\kw{r}$ : $EXPR(\constdom)$ List, 
            \\ \qquad  \#\{length $|\vertxs|$, initialize with $\qflag(v)$ for every vertex, recording the adaptivity reaching each vertex.\}
            \\ \qquad  $\kw{flowcapacity}$: $EXPR(\constdom)$ List, 
            % INT List of length $|\vertxs|$, initialize MAXINT. 
            \\ \qquad  \#\{length $|\vertxs|$, initialize with $\infty$ for every vertex,
            % \#\{For every vertex, 
            recording the minimum weight when the walk reaching 
            that vertex, inside a cycle\}
            \\ \qquad  $\kw{querynum[v]}$: INT List,
            %  of length $|\vertxs|$, initialize with $\qflag(v)$ for every vertex. 
            \\ \qquad  \#\{length $|\vertxs|$, initialize with $\qflag(v)$ for every vertex, 
            % \#\{For every vertex, 
            recording the query numbers when the path reaching 
            that vertex, inside a cycle\}
            \STATE {\bf if} $|\vertxs| = 1$ and $|\edges| = 0$:
            \STATE \qquad {\bf return}  $\qflag(v)$
            \STATE  {\bf def} {$\kw{dfs(G, c,visited)}$}:
            % \STATE \qquad update the length of the longest path reaching this vertex
            % $r[s] =  r[s] + $$\kw{flowcapacity}$[s] * querynum[s].
            % \RETURN  \qquad $r[s]$.      
            \STATE \qquad {\bf for} every vertex $v$ 
            % having directed edge from $c$:
            connected by a directed edge from $c$:
            \STATE \qquad \qquad {\bf if} $\kw{visited}[v] = \efalse$:
            \STATE \qquad \qquad \qquad $\kw{flowcapacity[v] = \min(\weights(v), {flowcapacity}[c])}$;
            \STATE \qquad \qquad \qquad $\kw{querynum[v] = querynum[c] + \qflag(v)}$;
            % \STATE \qquad \qquad \qquad \#\{do not update the length of the longest walk reaching $v$ until the cycle is finished\}
            % \STATE \qquad \qquad \qquad $\kw{r[v] =  r[c] + flowcapacity[v] \times querynum[v]} $; \#\{do not update the length of the longest walk reaching $v$ until the cycle is finished\}
            \STATE \qquad \qquad \qquad $\kw{r[v] =  \max(r[v], flowcapacity[v] \times querynum[v]}) $; 
            % \#\{do not update the length of the longest walk reaching $v$ until the cycle is finished\}
            \STATE \qquad \qquad \qquad  $\kw{visited}[v] = 1$; %\#\{mark $v$ as visited\}
            \STATE \qquad \qquad \qquad $\kw{dfs(G, v, visited)}$;
            \STATE \qquad \qquad {\bf else}: \#\{There is a cycle finished\}
            % \STATE \qquad \qquad \qquad \#\{update the length of the longest path reaching this vertex\}
            \STATE \qquad \qquad \qquad 
            $\kw{r[v] =  \max(r[v], r[c] +  \min(\weights(v), {flowcapacity}[c]) * (querynum[c] + \qflag(v)))}$; \#\{update the length of the longest walk reaching this vertex on this cycle\}
            %  $\kw{r[v] =  \max(r[v], r[c] + flowcapacity[v] * querynum[v])}$; \#\{update the length of the longest walk reaching this vertex on this cycle\}
            %  \STATE \qquad \qquad \qquad \#\{Recover the $\kw{flowcapacity}$ and querynumber to previous state, for different loops\}
            % \STATE \qquad \qquad \qquad $\kw{flowcapacity[v] = flowcapacity[c]}$; \#\{Recover the $\kw{flowcapacity}$\}
            % \STATE \qquad \qquad \qquad $\kw{querynum[v] = querynum[c]}$;\#\{Recover the $\kw{querynum}$\}
            \STATE \qquad {\bf return}  $\kw{r[c]}$
            \STATE  {\bf for} every vertex $v$ in $\vertxs$:
            \STATE  \qquad initialize the $\kw{visited, r, flowcapacity, querynum}$;
            \STATE  \qquad $\kw{r_{scc} = \max(r_{scc}, dfs(G, v, \kw{visited} ))}$ ; 
            \RETURN  $\kw{r_{scc}}$
            \end{algorithmic}
            \end{algorithm}
            \\
Following is the challenge of computing the adaptivity on a program based dependency graph.
In order to search for the finite walk having the longest query length, which isn't a simple longest weighted path.
\\
the visiting times of every vertex on this walk should be no more than its weight, which is a symbolic expression.
So we cannot simply search for the longest weight path where the visiting times of the vertex on it could possibly exceed its weights.
We can neither simply traverse on this graph by decreasing the weight of every node by 1 after every visiting,
because the weight is symbolic and simply traversing leads to non-termination.
\\
In Algorithm~\ref{alg:adaptscc}, if an SCC contains only one vertex without any edge, 
then it is easy to know that it will be visited 
at most once since there isn't edge going back to this vertex. 
So we can know that the adaptivity on this SCC is at most one if it is a query vertex,
and zero otherwise.
\\
For the SCC contains at least one edge, we are searching for the finite walk having the longest query length through a deep first search strategy.
The difficulty is, the visiting times of every vertex on this walk should be no more than its weight, which is a symbolic expression.
So we cannot simply search for the longest weight path where the visiting times of the vertex on it could possibly exceed its weights.
We can neither simply traverse on this graph by decreasing the weight of every node by 1 after every visiting,
because the weight is symbolic and simply traversing leads to non-termination.
\\
So we use the dfs to search for the finite path with a capacity limitation and use special parameter to compute the adaptivity
for every path.
We use a special parameter $\kw{flowcapacity}$  to track the minimum weight during the dfs process, and $\kw{querynum[v]}$
to track the total number of vertices which are query vertices.

we first initialize some parameters,
the $\kw{visited}$ a list of $\etrue$ or $\efalse$ for every vertex on this SCC, to guarantee the termination;
\\ 
$\kw{r}$ : INT List of length $|\vertxs|$, initialize with $\qflag(v)$ for every vertex. The adaptivity reaching each vertex.
\\ 
$\kw{flowcapacity}$ a list of symbolic expressions for every vertex, recording the minimum weight when the walk reaching 
that vertex, inside a cycle\}
\\ 
$\kw{querynum[v]}$: INT List of length $|\vertxs|$, initialize with $\qflag(v)$ for every vertex. 
\#\{For every vertex, recording the query numbers when the path reaching.
\\
Then from line5:11, we record the minimum weight and number of query vertices alone the path and update the adaptivity reaching 
this vertex, and then recursively dfs on all vertices heading out from this vertex.
\\
At line 12 where this vertex is visited the second time, 
we only update the adaptivity reaching this vertex and neither recursion nor update the $\kw{flowcapacity}$  and 
$\kw{querynum[v]}$.
\\
not recursion in order to guarantee the termination,
do not update the $\kw{querynum[v]}$ because a second visiting of the same vertex indicates there is a cycle goes back to this vertex, 
then, when we continuously search path heading out of this vertex, the minimum weight on this cycle will not affect the walks going out of this vertex that not pass this cycle.
However, if we keep recording the minimum weight, we are restricting the visiting times on a walk By
 using the minimum weight of vertices not on this walk, it is unsound anymore.
 %
 Then we compute the adaptivity of this SCC by taking the maximum adaptivity reaching every vertex on this SCC.
%
The soundness is formally guaranteed in Lemma~\ref{lem:sound_adaptalg_scc} in Appendix~\ref{apdx:adaptalg_soundness}.
            % \begin{algorithm}
        % \caption{
        % {Refined Adaptivity on $\kw{SCC}$}
        % \label{alg:dfscycle_alg}
        % }
        % \begin{algorithmic}
        % \REQUIRE Weighted Directed Graph $G = (\vertxs, \edges, \weights, \qflag)$ with a start vertex $s$ and destination vertex $t$ .
        % \STATE  {\bf {$\kw{dfs_{refine}(G, c, visited)}$}:}  
        % \STATE {\bf init} 
        % \\
        % current node: $c$, 
        % % \\
        % % visited: List of length $|\vertxs|$, initialize with $\efalse$.
        % \\
        % results: $r$ : INT List of length $|\vertxs|$, initialize with $\qflag(v)$ for every vertex.
        % \\
        % $\kw{flowcapacity}$: INT List of length $|\vertxs|$, initialize MAXINT. 
        % \#\{For every vertex, recording the minimum weight when the walk reaching 
        % that vertex, inside a cycle\}
        % \\
        % querynum: INT List of length $|\vertxs|$, initialize with $\qflag(v)$ for every vertex. 
        % \#\{For every vertex, recording the query numbers when the walk reaching 
        % that vertex, inside a cycle\}
        % \\
        % % \STATE {\bf if} $c = s$:
        % % \STATE \qquad update the length of the longest path reaching this vertex
        % % $r[s] =  r[s] + $$\kw{flowcapacity}$[s] * querynum[s].
        % % \RETURN  \qquad $r[s]$.      
        % \STATE {\bf for}  all vertex $v$ having directed edge from $c$:
        % \STATE \qquad \qquad $\kw{flowcapacity}$[v] = min($\weights(v)$, $\kw{flowcapacity}$[c]);
        % \STATE \qquad \qquad querynum[v] = querynum[c] + $\qflag(v)$;
        % \STATE \qquad \qquad \#\{do not update the length of the longest walk reaching $v$ until the cycle is finished\}
        % \STATE \qquad \qquad $r[v] =  r[c] $;
        % \STATE \qquad {\bf if}  $v$ is unvisited:
        % \STATE \qquad \qquad \#\{mark $v$ as visited\} $\kw{visited}[v] = 1$;
        % \STATE \qquad \qquad $\kw{dfs_{refine}(G, v, visited)}$;
        % \STATE \qquad {\bf else}: \#\{There is a cycle finished\}
        % \STATE \qquad \qquad \#\{update the length of the longest path reaching this vertex\}
        % \STATE \qquad \qquad 
        %  $r[v] =  \max(r[v], r[c] + $$\kw{flowcapacity}$[v] * querynum[v]);
        %  \STATE \qquad \qquad \#\{Recover the $\kw{flowcapacity}$ and querynumber to previous state, for different loops\}
        %  \STATE \qquad \qquad $\kw{flowcapacity}$[v] = $\kw{flowcapacity}$[c];
        %  \STATE \qquad \qquad querynum[v] = querynum[c];
        % \RETURN  $r[c]$
        % \end{algorithmic}
        % \end{algorithm}
        % %

\begin{thm}[Soundness of $\pathsearch$]
    \label{thm:sound_adaptalg}
    For every program $c$, given its \emph{Program-Based Dependency Graph} $\progG$,
     $$\pathsearch(\progG) \geq \progA(\progG).$$
\end{thm}
% \begin{thm}[Soundness of $\pathsearch$]
  \label{thm:sound_adaptalg}
  For every program $c$, given its \emph{Program-Based Dependency Graph} $\progG$,
   $$\pathsearch(\progG) \geq \progA(\progG).$$
\end{thm}
proof Summary:
1. for each SCC, a subgraph of $\progG$, $\pathsearch_{scc}(SCC) \geq \progA(SCC)$
2. for every two nodes with a path $x, \cdots, y$, let $\walks(k_{x,y})$ be all the walks from $x$ to $y$ on $\progG$,
then $adapt[SCC(x)] + \cdots + adapt[SCC(y)] \geq \max\{\qlen(k)\}$
\begin{proof}

\end{proof}
% % \paragraph{Variable Collection Algorithm, $\varCol$}
% % % The $\varCol$ algorithm shows how the labelled variables $\lvar$ are collected 
% % % (via the command ${\assign{x}{\expr}}$ or ${\assign{x}{\query(\qexpr)}}$) from the program ${c}$ in the first step.
% % % The algorithmic rules for $\varCol$ algorithm is defined in Figure~\ref{fig:var_col}. 
% % % It has the form: $\ag{\lvar; w; {c}}{ \lvar'; w'} $. 
% % % The input of $\varCol$ is the labelled variables $\lvar$ collected before the program ${c}$, a while map $w$ consistent with previous estimation, a program ${c}$. 
% % % The output of the algorithm is the updated labelled variables $\lvar'$, along with the updated while map $w$ for next steps' collecting.   
% % The $\varCol$ algorithm shows how the labelled variables $\lvar$ are collected 
% % (via the command ${\assign{x}{\expr}}$ or ${\assign{x}{\query(\qexpr)}}$) from the program ${c}$ in the first step, 
% % along with constructing the flag for each variable, i.e., $\flag$.
% % The algorithmic rules for $\varCol$ algorithm is defined in Figure~\ref{fig:var_col}. 
% % It has the form: 
% % {$\ag{\lvar; \flag; {c}}{ \lvar'; \flag'} $}. 
% % The input of $\varCol$ is a program ${c}$, 
% % the labelled variables $\lvar$ collected before the program ${c}$ 
% % as well as the flags $\flag$ for every corresponding variable .
% % The output of the algorithm is the updated labelled variables $\lvar'$ and flags $\flag'$ thorough the program ${c}$
% % %
% % % We have the algorithmic rules for $\varCol$ algorithm of the form: $\ag{\lvar; w; {c}}{\lvar';w'} $ as in Figure \ref{fig:var_col}. 
% % %
% % \begin{figure}
% % {
% % \begin{mathpar}
% % \inferrule
% % {
% % \empty
% % }
% % { \ag{\lvar ; \flag; {[\assign {x}{\expr}]^{l}}}
% % {\lvar ++ [{x}]; \flag++[0]}
% % }
% % ~\textbf{\varCol-asgn}
% % \and
% % \inferrule
% % {
% % }
% % { \ag{\lvar; \flag; [ \assign{{x}}{\query({\qexpr})}]^{l}}
% % {\lvar ++ [{x}]; \flag ++ [2]} 
% % }~\textbf{\varCol-query}
% % %
% % \and 
% % %
% % \inferrule
% % {
% % \ag{\lvar; [];  {c_1}}{\lvar_1; \flag_1}
% % \and 
% % \ag{\lvar_1; []; {c_2}}{ \lvar_2; \flag_2}
% % \and
% % \lvar_3 = \lvar_2 ++ \lvar'
% % \and
% % \flag_3 = \flag ++ ((\flag_1 ++ \flag_2) \uplus 1)
% % }
% % {
% % \ag{\lvar; \flag;
% % [\eif({\bexpr}, { c_1, c_2)}]^{l} }
% % {\lvar_3; \flag_3}
% % }~\textbf{\varCol-if}
% % %
% % %
% % %
% % \and 
% % %
% % \inferrule
% % {
% % \ag{\lvar; \flag {c_1}}{\lvar_1; \flag_1}
% % \and 
% % \ag{\lvar_1; \flag_1 ; {c_2}}{\lvar_2; \flag_2}
% % }
% % {
% % \ag{\lvar; \flag;
% % {(c_1 ; c_2)}}{\lvar_2 ; \flag_2}
% % }
% % ~\textbf{\varCol-seq}
% % \and 
% % %
% % %
% % {
% % \inferrule
% % {
% % { \ag{\lvar; [] ; {c}}
% % {\lvar'; \flag' }  }
% % \\
% % \lvar'' = \lvar'
% % \and 
% % \flag'' = \flag ++ (\flag' \uplus 1)
% % }
% % {
% % \ag{\lvar; \flag;  
% % \ewhile [{b}]^{l}
% % \edo  {c} }{\lvar''; \flag''}
% % }
% % ~\textbf{\varCol-while}
% % }
% % \end{mathpar}
% % }
% % \caption{The Algorithmic Rules of $\varCol$ }
% % \label{fig:var_col}
% % \end{figure}
% % %
% % %
% % The assignment commands are the source of variables $\varCol$ collecting, 
% % in the case $\textbf{\varCol-asgn}$ and $\textbf{\varCol-query}$, 
% % the output labelled variables are extended by ${x}$. 
% % \\
% % \todo{
% % When it comes to the $\eif \ldots \ethen \ldots \eelse$ command in the rule $\textbf{\varCol-if}$, variables assigned in the then branch ${c_1}$, as well as the variables assigned in the else branch ${c_2}$, and the new generated variables $\bar{{x}},\bar{{y}},\bar{{z}}$ in $ [ \bar{{x}}, \bar{{x_1}}, \bar{{x_2}}] ,[ \bar{{y}}, \bar{{y_1}}, \bar{{y_2}}],[ \bar{{z}}, \bar{{z_1}}, \bar{{z_2}}]$.
% % \\ 
% % The sequence command ${c_1;c_2}$ is standard by accumulating the predicted variables in the two commands ${c_1}$ and ${c_2}$ preserving their order. 
% % \\
% % The while command $\ewhile {\bexpr}, [{\bar{x}}] \ldots \edo {c}$ considers the newly generated variables by SSA transformation ${\bar{x}}$
% % as well and the newly labelled variables in its body ${c}$.
% % \\
% % %
% % Below we present the definition for a valid index, to have a clear understanding on the variable collecting algorithm:
% % }
% % %
% % %
% % \todo{
% % \begin{defn}[Valid Index (Remove?)]
% % Given an assigned variable list $\lvar$, $\lvar; \vDash ({c},i_1,i_2)$ iff 
% % $\lvar' = \lvar[0,\ldots, i_1-1], \lvar';{c} \to \lvar'' \land \lvar'' = \lvar[0, \ldots, i_2-1] $.  
% % \end{defn}}
% % %
% % %
% \todo{Data Dependency Analysis Algorithm Needed: (Possibly modify based on existing one, or a different one) get the more precise dependency information. 
% i.e., instead of dependency on all the over-approximated variables, 
% but dependency on only the variables assumed to be live.
% }
% \paragraph{Data Dependency Analysis Algorithm}
% %
% In this data flow matrix generating algorithm, we analyze the data flow information among all labelled variables $\lvar$ collected via the the $\varCol$ algorithm of length $N$.
% %
% We track the data flow relations between all these labelled variables. These informations are stored in a matrix $\Mtrix$, whose size is $N \times N$. 
% % We also track whether arbitrary variable is assigned with a query result in a vector $\flag$ with size $|\lvar|$. 
% %
% The algorithm to fill in the matrix is of the form: 
% {$\ad{\Gamma ; {c} ; \lvar}{\Mtrix}$}
% $\ad{\Gamma ; {c} ; i_1, i_2}{\Mtrix; \flag}$. 
% $\Gamma$ is a vector records the variables the current program ${c}$ depends on, the index $i_1$ is a pointer which refers to the position of the first new-generated variable in ${c}$ in the labelled variables $\lvar$, and $i_2$ points to the first new variable that is not in ${c}$ (if exists). 
% % %
% % %
% % {
% % \begin{defn}[Valid Gamma (Remove?)]
% % $\Gamma \vDash i_1$ iff $\forall i \geq i_1, \Gamma(i_1)=0 $.  
% % \end{defn}
% % }
% %%
% %
% % \framebox{$ {\Gamma} \vdash^{i_1, i_2}_{\Mtrix, \flag} ~ c $}
% % \begin{mathpar}
% % \inferrule
% % {\Mtrix = \lMtrix_i * ( \rMtrix_{{\expr},i} + \Gamma )
% % }
% % {
% %  \ad{\Gamma;[\assign {{x}}{{\expr}} ]^{l}; i }{\Mtrix; \flag_{0}; i+1 }
% % }
% % ~\textbf{\graphGen-asgn}
% % \and
% % {
% % \inferrule
% % {\Mtrix = \lMtrix_i * ( \rMtrix_{{\expr},i} + \Gamma )
% % \\
% % \flag = \lMtrix_i \and \flag(i) = 1
% % }
% % { 
% % \ad{\Gamma;[ \assign{{x}}{\query({\expr})} ]^{l} ; i }
% % {\Mtrix;\flag;i+1}
% % }~\textbf{\graphGen-query}}
% % %
% % \and 
% % %
% % {
% % \inferrule
% % {
% % {\ad{\Gamma + \rMtrix_{{\bexpr}, i_1}; {c_1} ; i_1 }{ \Mtrix_1;\flag_1;i_2 }}
% % \and 
% % {\ad{\Gamma + \rMtrix_{{\bexpr}, i_1};{c_2} ; i_2 }{ \Mtrix_2; \flag_2 ;i_3}}
% % \\
% % {\ad{\Gamma; [ \bar{{x}}, \bar{{x_1}}, \bar{{x_2}}]; i_3 }{ M_x; \flag_{\emptyset}; i_3+|\bar{{x}}| }}
% % %
% % \\
% % %
% % {\ad{\Gamma; [ \bar{{y}}, \bar{{y_1}}, \bar{{y_2}}]; i_3+|\bar{{x}}| }{ \Mtrix_y; \flag_{\emptyset}; i_3+|\bar{{x}}|+|\bar{{y}}| }}
% % %
% % \\
% % %
% % {\ad{\Gamma; [ \bar{{z}}, \bar{{z_1}}, \bar{{z_2}}]; i_3+|\bar{{x}}|+ |\bar{{y}}|}{ \Mtrix_y; \flag_{\emptyset}; i_3+|\bar{{x}}|+|\bar{{y}}| + |\bar{{z}}| }}
% % \\
% % {\Mtrix = (\Mtrix_1 + \Mtrix_2)+ \Mtrix_x+ \Mtrix_y + \Mtrix_z }
% % }
% % {
% % \ad{\Gamma ; \eif([{\bexpr}]^{l},[ \bar{{x}}, \bar{{x_1}},
% % \bar{{x_2}}] ,[ \bar{{y}}, \bar{{y_1}}, \bar{{y_2}}], 
% % [ \bar{{z}}, \bar{{z_1}}, \bar{{z_2}}],
% % { c_1, c_2)} ; i_1}{ \Mtrix ; \flag_1 \uplus \flag_2 \uplus 2  ; i_3+|\bar{x}|+|\bar{y}|+|\bar{z}| }
% % }
% % ~\textbf{\graphGen-if}
% % }
% % %
% % %
% % %
% % \and 
% % %
% % \inferrule
% % {
% % {\ad{\Gamma; {c_1} ; i_1 }{ \Mtrix_1 ; \flag_1; i_2 }  }
% % \and 
% % {
% % \ad{\Gamma;{c_2}; i_2}{ \Mtrix_2; \flag_2 ;i_3 }}
% % }
% % {
% % \ad{\Gamma ; ({c_1 ; c_2} ) ; i_1}{( \Mtrix_1 {;} \Mtrix_2) ; \flag_1 \uplus V_2 ; i_3  }
% % }
% % ~\textbf{\graphGen-seq}
% % %
% % \and 
% % %
% % \and 
% % %
% % { 
% % \inferrule
% % {
% % B= |{\bar{x}}| \and {A = |{c}|}
% % \\
% % {\ad{\Gamma;[\bar{{x}}, \bar{{x_1}}, \bar{{x_2}}]; i+ (B+A) }{ \Mtrix_{1};V_{1}; i+B+(B+A) }}
% % \\
% % {
% % \ad{\Gamma;{c} ; i+B+(B+A)  }{ \Mtrix_{2}; \flag_{2}; i+B+A+(B+A) }
% % }
% % \\
% % {
% % \ad{\Gamma ; [\bar{{x}}, \bar{{x_1}}, \bar{{x_2}}] ; i+(B+A) }{ \Mtrix; \flag ;i+(B+A)+B}
% % }
% % \\
% % { \Mtrix' = \Mtrix + ( \Mtrix_{1} + \Mtrix_{2}) }
% % \and
% % {
% % \flag' = \flag \uplus (( \flag_{1} \uplus \flag_{2}) \uplus 2)  }
% % }
% % {
% % \ad{\Gamma;
% % \ewhile ~ [ b ]^{l} ~ {n} ~
% % [\bar{{x}}, \bar{{x_1}}, \bar{{x_2}}] 
% % ~ \edo ~  c;
% % i }{ \Mtrix'; \flag' ;i+(B+A)+B }
% % }~\textbf{\graphGen-while}
% % }
% % \end{mathpar}
% {
% \framebox{$ \ad{\Gamma; c; \lvar_c}{\Mtrix}$}
% \begin{mathpar}
% \inferrule
% {
% {x}^l \in \lvar_c
% \and 
% \Mtrix = \lMtrix_i * ( \rMtrix_{{\expr}} + \Gamma )
% }
% {
% \ad{\Gamma; [\assign {{x}}{{\expr}} ]^{l}; \lvar_c}
% {\Mtrix}
% }
% ~\textbf{\graphGen-asgn}
% \and
% {
% \inferrule
% {
% {x}^l \in \lvar_c
% \and 
% \Mtrix = \lMtrix_i * ( \rMtrix_{{\expr}} + \Gamma )
% }
% { 
% \ad{\Gamma;[ \assign{{x}}{\query({\qexpr})} ]^{l} ; \lvar_c }
% {\Mtrix}
% }~\textbf{\graphGen-query}}
% %
% \and 
% %
% {
% \inferrule
% {
% {\ad{\Gamma + \rMtrix_{{\bexpr}}; {c_1} ; \lvar_c }{ \Mtrix_1}}
% \and 
% {\ad{\Gamma + \rMtrix_{{\bexpr}}; {c_2}; \lvar_c }{ \Mtrix_2}}
% \and
% {\Mtrix = (\Mtrix_1 + \Mtrix_2)}
% }
% {
% \ad{\Gamma ; \eif([{\bexpr}]^{l},{ c_1, c_2)}}
% { \Mtrix }
% }
% ~\textbf{\graphGen-if}
% }
% %
% %
% %
% \and 
% %
% \inferrule
% {
% {\ad{\Gamma; {c_1}; \lvar_c }{ \Mtrix_1}  }
% \and 
% {
% \ad{\Gamma;{c_2}; \lvar_c }{ \Mtrix_2}}
% }
% {
% \ad{\Gamma ; ({c_1 ; c_2} ); \lvar_c}
% {( \Mtrix_1 {;} \Mtrix_2) }
% }
% ~\textbf{\graphGen-seq}
% %
% \and 
% %
% \and 
% %
% { 
% \inferrule
% {
% {
% \ad{\Gamma + \rMtrix_{{\bexpr}};{c}; \lvar_c  }{ \Mtrix'}
% }
% }
% {
% \ad{\Gamma;
% \ewhile [ \sbexpr ]^{l} \edo  {c}; \lvar_c }{\Mtrix'}
% }~\textbf{\graphGen-while}
% }
% \end{mathpar}
% }
% %
% Below we define the valid data flow matrix, to have a clear understanding on the data flow generating algorithm:
% \begin{defn}[Valid Matrix]
% For a labelled variables $\lvar$, $\lvar \vDash (\Mtrix,\flag)$ iff the cardinality of $\lvar$ equals to the one of $\flag$, $|\lvar| = |\flag|$ 
% and the matrix $\Mtrix$ is of size $|\flag| \times |\flag|$.
% \end{defn}
% %
% \todo{Improvement if possible: Combining reachability bounds analysis into the static dependency analysis algorithm above, rather than adopting an external tool entirely.}
% %
% \paragraph{Reachability Bounds}
% Given a program $c$ with its labelled variables $\lvar$,
% we use the $\rb({x}, {c})$ algorithm, from paper \cite{10.1145/1806596.1806630}, to estimate the reachability bound for each variable ${x} \in \lvar$. 
% The input of $\rb$ is a program ${c}$ in SSA language and a variable ${x} $ from ${c}$.
% The output of $\rb({x}, {c})$ is an integer representing the reachability bound of ${x}$ in ${c}$.
% %

% %
% The following example programs ${c}2$ and ${c}3$ with while loop illustrate how the algorithm works.
% The collected labelled variables, $\lvar_{{c}2}$ and $\lvar_{{c}3}$,
% data flow matrix $\Mtrix_{{c}2}$ and  $\Mtrix_{{c}3}$
% and variable flags $\flag_{{c}2}$ and $\flag_{{c}3}$
% for program ${c}2$ and ${c}3$
% are presented in the right hand side.
%
% \[
% {{c}2 \triangleq
% \begin{array}{l}
% \left[{ x_1} \leftarrow \query(1)  \right]^1 ; 
% \\
% \left[{i_1} \leftarrow 0 \right]^2 ; 
% \\
% \ewhile
% ~ [{i_1} < 2]^3
% 	\\
% ~{[ x_3,x_1 ,x_2 ], [i_3, i_1, i_2] }
% ~ \edo 
% \\
% ~ \Big( 
% \left[{y}_1 \leftarrow \query(2) \right]^4;
% \\
% \left[{x_2 \leftarrow y_1  + x_3 } \right]^5;
% \\
% \left[{i_2 \leftarrow 1  + i_3 } \right]^6
% \Big) ; 
% \\
% \left[ {\assign{z_1}{x_3}} + 2  \right]^{7}
% \end{array}
% ,
% ~~~~
% \lvar_{{c}2} = \left [ \begin{matrix}
% {x}_1 \\
% {x}_3 \\
% {y}_1 \\
% {x}_2 \\
% {z}_1 \\
% {i}_1 \\
% {i}_2 \\
% {i}_3 
% \end{matrix} \right ]
% % \Mtrix =  \left[ \begin{matrix}
% %  & (x_1)  & (y_1) & (x_2) & (x_3) &  (z_1) & i_1 & i_2 & i_3\\
% % (x_1) & 0 & 0 & 0 & 0 & 0 & 0 & 0 & 0 \\
% % (y_1) & 0 & 0 & 0 & 0 & 0 & 1 & 1 & 1 \\
% % (x_2) & 0 & 1 & 0 & 1 & 0 & 1 & 1 & 1 \\
% % (x_3) & 1 & 0  & 1& 0 & 0 & 1 & 1 & 1 \\
% % (z_1) & 0 & 0 & 0 & 1 & 0 & 0 & 0 & 0 \\
% % (i_1) & 0 & 0 & 0 & 0 & 0 & 0 & 0 & 0 \\
% % (i_2) & 0 & 1 & 0 & 1 & 0 & 1 & 0 & 1 \\
% % (i_3) & 1 & 0  & 1& 0 & 0 & 1 & 1 & 1 \\
% % \end{matrix} \right]
% ,
% ~~~~~~
% \Mtrix_{{c}2} =  \left[ \begin{matrix}
% 0 & 0 & 0 & 0 & 0 & 0 & 0 & 0 \\
% 0 & 0 & 0 & 0 & 0 & 1 & 1 & 1 \\
% 0 & 1 & 0 & 1 & 0 & 1 & 1 & 1 \\
% 1 & 0  & 1& 0 & 0 & 1 & 1 & 1 \\
% 0 & 0 & 0 & 1 & 0 & 0 & 0 & 0 \\
% 0 & 0 & 0 & 0 & 0 & 0 & 0 & 0 \\
% 0 & 1 & 0 & 1 & 0 & 1 & 0 & 1 \\
% 1 & 0  & 1& 0 & 0 & 1 & 1 & 1 \\
% \end{matrix} \right]
% ,
% ~~~~
% \flag_{{c}2} = \left [ \begin{matrix}
% 1 \\
% 2 \\
% 1 \\
% 2 \\
% 0 \\
% 0 \\
% 2 \\
% 1 
% \end{matrix} \right ]
% }
% \]
% %
% %
% \[
% {{{c}3}  \triangleq
% \begin{array}{l}
% \left[{ x}_1 \leftarrow \query(1)  \right]^1 ;
% \\
% \left[{i_1} \leftarrow 1 \right]^2 ; 
% \\
% \ewhile ~ [i < 0]^{3} ,
% \\
% ~{[ x_3,x_1 ,x_2 ], [i_3, i_1, i_2] }
% ~ \edo
% \\
% ~ \Big( 
% \left[{ y_1} \leftarrow \query(2) \right]^3; \\
% \left[{x_2 \leftarrow y_1  + x_3 } \right]^5
% \Big) ; \\
% \left[ {\assign{z_1}{x_3}} + 2  \right]^{6}
% \end{array},
% ~~~~~~
% \lvar_{{c}3} = \left [ \begin{matrix}
% {x}_1 \\
% {i}_1 \\
% {x}_3 \\
% {i}_3 \\
% {z}_1 \\
% \end{matrix} \right ]
% ,~~~~~~
% \Mtrix_{{c}3}  =  \left[ \begin{matrix}
% 0 & 0 & 0 & 0 & 0 \\
% 0 & 0 & 0 & 0 & 0 \\
% 1 & 0 & 0 & 0 & 0 \\
% 0 & 1 & 0 & 0 & 0 \\
% 0 & 0 & 1 & 0 & 0 \\
% \end{matrix} \right]
% ,~~~~~~
% \flag_{{c}3} = \left [ \begin{matrix}
% 1 \\
% 0 \\
% 2 \\
% 2 \\
% 0 \\
% \end{matrix} \right ]
% }
% \]
% %
% We can now look at the if statement.
% \[ 
% %
% {c}4 \triangleq
% \begin{array}{l}
% 	\left[ {x}_1 \leftarrow \query(1) \right]^1; 
% 	\\
% 	\left[{y}_1 \leftarrow \query(2) \right]^2 ; 
% 	\\
% \eif \;( { x_1 + y_1 == 5} )^3,  \\
% {[ x_4,x_2,x_3 ],[] ,[y_3,y_1,y_2 ]} 
% \\
% \mathsf{then} ~ \left[ 
% {x}_2 \leftarrow \query(3) \right]^4 
% \\
% \mathsf{else} ~ \left[ 
% {x}_3 \leftarrow \query(4) \right]^5 ; 
% \\
% {y}_2 \leftarrow 2 ) \\
% \left[ { z_1 \leftarrow x_4 +y_3 }\right]^6
% \end{array},
% % \]
% % \[
% ~~~~~~
% \lvar_{{c}4} =  \left[ \begin{matrix}
% {x}_1 \\
% {y}_1 \\
% {x}_2 \\
% {x}_3 \\
% {y}_2 \\
% {x}_4 \\
% {y}_3 \\
% {z}_1 \\
% \end{matrix} \right], 
% ~~~~~ 
% \Mtrix_{{c}4} =  \left[ \begin{matrix}
% 0 & 0 & 0 & 0 & 0 & 0 & 0 & 0 \\
% 0 & 0 & 0 & 0 & 0 & 0 & 0 & 0 \\
% 0 & 0 & 0 & 0 & 0 & 0 & 0 & 0 \\
% 0 & 0 & 0 & 0 & 0 & 0 & 0 & 0 \\
% 0 & 0 & 0 & 0 & 0 & 0 & 0 & 0 \\
% 0 & 0 & 1 & 1 & 0 & 0 & 0 & 0 \\
% 0 & 1 & 0 & 0 & 1 & 0 & 0 & 0 \\
% 0 & 0 & 0 & 0 & 0 & 1 & 1 & 0 \\
% \end{matrix} \right], 
% ~~~~~ 
% \flag_{{c}4} = \left [ \begin{matrix}
% 1 \\
% 1 \\
% 1 \\
% 1 \\
% 0 \\
% 0 \\
% 0 \\
% 0 \\
% \end{matrix} \right ]
% \]
%
%
%
%


% By specifying the departure and destination vertices $s$ and $t$, the $\pathssearch(\progG, s, t)$ algorithm will 
% give the number of query vertices on a finite walk from $s$ to $t$, which contains the maximum number of query vertices.
% The pseudo-code of $\pathssearch(\progG, s, t)$ algorithm is defined in the Algorithm \ref{alg:adpt_alg}.
% %
% \begin{algorithm}
% \caption{
% {Walk Search Algorithm ($\pathssearch$)}
% \label{alg:adpt_alg}
% }
% \begin{algorithmic}
% \REQUIRE Weighted Directed Graph $G = (\vertxs, \edges, \weights, \flag)$ with a start vertex $s$ and destination vertex $t$ .
% \STATE  {\bf {bfs $(G, s, t)$}:}  
% \STATE \qquad {\bf init} 
% current node: $c = s$, 
% queue: $q = [c]$, 
% vector recoding if the vertex is visited: 
% visited$ = [0]*|\vertxs|$,
% result: $r$
% \STATE \qquad {\bf while} $q$ isn't empty:
% \STATE \qquad \qquad take the vertex from beginning $c= q.pop()$
% \STATE \qquad \qquad mark $c$ as visited, visited $[c] = 1$
% \STATE \qquad \qquad curr$\kw{flowcapacity}$ = min($\weights$(c), curr$\kw{flowcapacity}$).
% \STATE \qquad \qquad put all unvisited vertex $v$ having directed edge from c into $q$. 
% \STATE \qquad \qquad if $v$ is visited, then there is a circle in the graph, we update the result $r = r + $curr$\kw{flowcapacity}$
% \RETURN $r$
% \end{algorithmic}
% \end{algorithm}
%
%
% \subsection{\todo{Soundness of the \THESYSTEM}}

% {
% 	\begin{thm}[Soundness of the \THESYSTEM].
% 	Given a program ${c}$, we have:
% 	%
% 	\[
% 	\progA({c}) \geq A({c}).
% 	\]
% 	\end{thm}
% }
% {
% \begin{proof}
% Given a program ${c}$, 
% we construct its program-based graph $\progG({c}) = (\vertxs, \edges, \weights, \qflag)$
% by Definition~\ref{def:prog-based_graph}
% According to the Definition \ref{def:prog_adapt}, we have:
% %
% \[
% 	\progA({c}) 
% 	:= \max\left\{ \qlen(k)\ \mid \  k\in \walks(\progG({c}))\right \}.
% \]
% %
% According to the Definition \ref{def:trace-based_adapt}, we have the trace-based adaptivity as follows:
% $$
% A({c}) = \max \big 
% \{ \len(p) \mid {m} \in \mathcal{SM},D \in \dbdom ,p \in \paths(\traceG({c}, \text{D}, {m}) \big \} 
% $$
% %
% Then, we need to show:
% \[
% \max \big 
% \{ \len(p) \mid {m} \in \mathcal{SM},D \in \dbdom ,p \in \paths(\traceG({c}, \text{D}, {m}) \big \} 
% \leq
% \max\left\{ \qlen(k) \ \mid \  k\in \walks(\progG({c}))\right \}
% \]
% %
% It is sufficient to show that:
% \[
% 	\forall p, {m}, D, ~ s.t., ~ p \in \paths(\traceG({c}, \text{D}, {m}),
% 	\exists k \in \walks(\progG({c})) \land 
% 	\len(p) \leq \qlen(k)
% \]
% %
% Taking an arbitrary starting memory $m$ and an arbitrary underlying database $D$,
% we construct a trace-based graph $\traceG({c}, \text{D}, {m}) = (\vertxs, \edges)$ by the definition \ref{def:trace-based_graph}.
% %
% \\
% %
% Let $\midG({c},{m},\text{D}) = \{\midV, \midE, \midF\}$ be the intermediate graph by Definition~\ref{def:midgraph}.
% \\
% By Lemma~\ref{lem:bie_trace_to_mid}, we know:
% \[
% 	\forall p, {m}, D, ~ s.t., ~ p \in \paths(\traceG({c}, \text{D}, {m}),
% 	\exists p' \in \paths(\midG({c},{m},\text{D})) \land 
% 	\len(p) = \len_q(p')
% \]
% %
% Then it is sufficient to show that:
% %
% \[
% 	\forall p, {m}, D, ~ s.t., ~ p \in \paths(\midG({c}, \text{D}, {m}),
% 	\exists k \in \walks(\progG({c})) \land 
% 	\qlen(p) \leq \qlen(k)
% \]
% %
% We prove a stronger statement instead:
% \[
% 	\forall p, {m}, D, ~ s.t., ~ p \in \paths(\midG({c}, \text{D}, {m}),
% 	\exists k \in \walks(\progG({c})) \land 
% 	\qlen(p) = \qlen(k)	
% \]
% %
% %
% By Lemma~\ref{lem:sujv_mid_to_prog}, let $g$ be the surjective function $g: \progV \to \midV$ s.t.:
% %
% $$
% \forall \av \in \midV. ~ \progF(f(\av)) = \midF(\av) 
% \land |\kw{image}(f(\av))| \leq W(f(\av)).
% $$
% %
% %
% % \item(1) $\len(p_{\av_1 \to \av_2}) = \len(k_{f(\av_1) \to f(\av_2)})$
% % %
% % \item(2) $\forall \av \in p_{\av_1 \to \av_2}. ~ f(\av) \in k_{f(\av_1) \to f(\av_2)}$
% % %
% % \item(3) $\forall \av \in p_{\av_1 \to \av_2}. ~ 
% % \kw{image}(f(\av)) \cap {p_{\av_1 \to \av_2}}| = \# \{f(\av) \mid f(\av) \in k_{f(\av_1) \to f(\av_2)}\}$
% %
% Let ${m}$ and $D$ be an arbitrary memory and database $D$,
% taking an arbitrary path $p_{\av_1 \to \av_n} \in \paths(\midG({c}, \text{D}, {m})$ with:
% %
% \item Edge sequence: $(e, \ldots, e_{n-1})$
% %
% \item Vertices sequence: $(\av_1, \ldots, \av_n)$.
% \\
% By Lemma~\ref{lem:sujpathwalk_mid_to_prog}, let $h: \paths(\midG({c}, \text{D}, {m})) \to \walks(\progG({c}))$ be the surjective function satisfies:
% %
% \[
% 	\forall p_{\av_1 \to \av_n} \in \paths(\midG({c}, \text{D}, {m}))
% 	\text{ with }
% 	\left\{
% 	\begin{array}{ll}
% 	\mbox{edge sequence:} & (e, \ldots, e_{n-1})
% 	\\ 
% 	\mbox{vertices sequence:} & (\av_1, \ldots, \av_n)
% 	\end{array}
% 	\right.
% \]
% %
% \[
% 	\exists k_{f(\av_1) \to f(\av_n)} \in \walks(\progG({c}))
% 	\text{ with }
% 	\left\{
% 	\begin{array}{ll}
% 	\mbox{edge sequence:} & (g(e), \ldots, g(e_{n-1}) 
% 	\\ 
% 	\mbox{vertices sequence:} & (f(\av_1), \ldots, f(\av_{n}))
% 	\end{array}
% 	\right.
% \]
% %
% We have the walk:
% $k_{f(\av_1) \to f(\av_n)} \in \walks(\progG({c}))$ with:
% %
% \item Edges sequence: $(g(e), \ldots, g(e_{n-1}) $
% %
% \item Vertices sequence: $(f(\av_1), \ldots, f(\av_{n}))$.
% \\
% It is sufficient to show 
% %
% \[
% 	\qlen(p_{\av_1 \to \av_n}) = \qlen(k_{f(\av_1) \to f(\av_n)})
% \]
% %
% Unfold the definition of $\qlen$, it is suffice to show:
% \[
% \len \big( \av \mid \av \in (\av_1, \ldots, \av_n) \land \midF(\av) = 2 \big) 
% = \len \big(f(\av) \mid f(\av) \in (f(\av_1), \ldots, f(\av_{n})) \land \progF(f(\av)\big) = 2)	
% ~ (a)
% \]
% %
% By Lemma~\ref{lem:sujv_mid_to_prog}, we know:
% %
% \[
% 	\forall \av \in \midV. ~ \midF(\av) = \progF(f(\av)) ~(b)
% \]
% By rewriting $(b)$ in $(a)$, we have this case proved.
% %
% \\
% \todo{
% \begin{defn}[Intermediate Graph $\midG$].
% 	\label{def:midgraph}
% 	\\
% 	$\mathcal{AV}$ : Annotated Variables based on program execution
% 	\\
% 	Given a program ${c}$ with its labelled variables $\lvar$ of length $N$,
% 	a database $D$, a starting memory ${m}$,
% 	s.t., $\Gamma \vdash_{\Mtrix_c, \flag_c} {c}$,
% 	the intermediate graph 
% 	$\midG({c},{m},\text{D}) = (\vertxs, \edges, \flag)$ is defined as:%
% \[
% \begin{array}{rlcl}
% 	\text{Vertices} &
% 	\vertxs & := & \left\{ 
% 	\av \in \mathcal{AV} \middle\vert
% 	\exists {m'},  w', \qtrace, \vtrace.  ~ s.t., ~  
% 	\config{{m} ,{c}, [], [], []}  \to^{*}  \config{{m'} , \eskip, \qtrace, \vtrace, w' }
% 	\land \av \in \vtrace
% 	\right\}
% 	\\
% 	\text{Directed Edges} &
% 	\edges & := & 
% 	\left\{ 
% 	(\av, \av') \in \mathcal{AV} \times \mathcal{AV} 
% 	~ \middle\vert ~
% 	\flowsto(\av, \av', {c},{m},D) 
% 	\right\}
% 	\\
% 	\text{Flags} &
% 	\flag & := & 
% 	\big\{ (\av, n)  \in \vertxs \times \{0, 1, 2\} 
% 	\mid 
% 	(\pi_1(\av) = \lvar(i) \land n = \flag_c(i)); ~
% 	i = 1, \ldots, N
% 	\big\}
% \end{array}
% \]
% \end{defn}
% }
% %
% \\
% \todo{
% 	\begin{lem}[$\vardep$ is Transitive].
% 	\label{lem:vardep_trans}
% 	\\
% 	Given a program ${c}$, with a starting memory ${m}$ and a hidden database $D$, s.t., 
% 	$\config{{m}, {c}, [], [], []} \rightarrow^{*} \config{{m}', \eskip, \qtrace, \vtrace, w} $.
% 	Then, $\forall \av_1, \av_2, \av_3 \in \vtrace$:
% \[
% 	\Big(\vardep(\av_1, \av_2, {c}, {m}, D) \land 
% 	\vardep(\av_2, \av_3, {c}, {m}, D) \Big)
% 	\implies
% 	\vardep(\av_1, \av_3, {c}, {m}, D)
% \]
% 	\end{lem}
% 	\begin{subproof}[of Lemma~\ref{lem:vardep_trans}]
% 	Proof by unfolding and rewriting the Definition~\ref{def:var_dep}.
% 	\end{subproof}
% }
% \\
% %
% \todo{
% 	\begin{lem}[$\flowsto$ is Transitive ??].
% 	\label{lem:flowsto_trans}
% 	\\
% 	Given a program ${c}$ with its labelled variables $\lvar$ of length $N$. 
% 	Then $\forall x_1, x_2, x_3 \in \lvar$
% \[
% 	\Big(\flowsto(x_1, x_2) \land \flowsto(x_2, x_3) \Big)
% 	\implies
% 	\flowsto(x_1, x_3)
% \]
% 	\end{lem}
% 	\begin{subproof}[of Lemma~\ref{lem:flowsto_trans}]
% 	Proof by unfolding the Definition~\ref{def:flowsto}.
% 	\end{subproof}
% }
% \\
% %
% \todo{
% 	\begin{lem}[$\qdep$ Implies $\vardep$].
% 	\label{lem:querydep_vardep}
% 	\\
% 	Given a program ${c}$, with a starting memory ${m}$ and a hidden database $D$, s.t., 
% 	$\config{{m}, {c}, [], [], []} \rightarrow^{*} \config{{m}', \eskip, \qtrace, \vtrace, w} $.
% 	Then, $\forall \av_1, \av_2 \in \qtrace$
% \[
% 	\qdep(\av_1, \av_2, {c}, {m}, D) \implies 
% 	\vardep(\pi_2(\av_1), \pi_2(\av_2), {c}, {m}, D)
% \]
% 	\end{lem}
% 	\begin{subproof}[of Lemma~\ref{lem:querydep_vardep}]
% 	Proof by unfolding the Definition~\ref{def:var_dep} and Definition~\ref{def:query_dep}.
% 	\end{subproof}
% }
% \\
% %
% \todo{
% 	\begin{lem}[$\vardep$ Implies \flowsto].
% 	\label{lem:vardep_flows}
% 	\\
% 	Given a program ${c}$, with a starting memory ${m}$ and a hidden database $D$, s.t., 
% 	$\config{{m}, {c}, [], [], []} \rightarrow^{*} \config{{m}', \eskip, \qtrace, \vtrace, w} $.
% 	Then, $\forall \av_1, \av_2 \in \vtrace$
% \[
% 	\vardep(\av_1, \av_2, {c}, {m}, D) \implies 
% 	\flowsto(\pi_1(\av_1), \pi_1(\av_2))
% \]
% 	\end{lem}
% 	\begin{subproof}[of Lemma~\ref{lem:querydep_vardep}]
% 	Proof by showing contradiction based on the Definition~\ref{def:var_dep} and Definition~\ref{def:flowsto}.
% 	Let $\av_1, \av_2 \in \vtrace$ be 2 arbitrary annotated variables in the variable trace $\vtrace$,
% 	s.t., $\vardep(\av_1, \av_2, {c}, {m}, D)$.
% 	\\
% 	Unfolding the $\vardep$ definition, we have:	
% 	\end{subproof}
% }
% \\
% %
% \todo{
% 	\begin{lem}[Injective Mapping of vertices from $\traceG$ to $\midG$].
% 	\label{lem:injv_trace_to_mid}
% 	\\
% 	$\traceG({c}) = \{\traceV, \traceE\}$
% 	\\
% 	$\midG({c},{m},\text{D}) = \{\midV, \midE, \midF\}$
% \[
% 	\exists ~ \kw{injective} ~ f: \mathcal{AQ} \to \mathcal{AV}. 
% 	~ \forall \av \in \traceV. ~ 
% 	f(\av) \in \midV \land \midF(f(\av)) = 2
% \]
% 	\end{lem}
% \begin{subproof}
% Proving by Definition~\ref{def:midgraph} and Definition~\ref{def:prog_adapt}.
% \end{subproof}
% }
% \\
% \todo{
% 	\begin{lem}[One-on-One Mapping from $\edges$ of $\traceG$ to $\paths(\midG)$].
% 	\label{lem:bie_trace_to_mid}
% 	\\
% 	$\traceG({c}) = \{\traceV, \traceE\}$
% 	\\
% 	$\midG({c},{m},\text{D}) = \{\midV, \midE, \midF\}$
% 	\\
% 	An injective function $ f: \traceV \to \midV$ s.t.,
% 	$\forall \av \in \traceV. ~ \midF(f(\av)) = 2$ 
% \[
% 	\forall e = (\av_1, \av_2) \in \traceE. ~ 
% 	\exists p_{f(\av_1) \to f(\av_2)} \in \paths(\midG({c}, \text{D}, {m}))
% \]
% 	\end{lem}
% \begin{subproof}
% Proving by Lemma~\ref{lem:injv_trace_to_mid} and Definition~\ref{def:midgraph} and acyclic property of $\traceG$ and $\midG$.
% \end{subproof}
% }
% \\
% \todo{
% 	\begin{lem}[Surjective Mapping of Vertices from $\midG$ to $\progG$].
% 	\label{lem:sujv_mid_to_prog}
% 	\\
% 	$\midG({c},{m},\text{D}) = \{\midV, \midE, \midF\}$
% 	\\
% 	$\progG({c}) = \{\progV, \progE, \progF, \progW\}$
% 	\\
% 	$\exists ~ \kw{surjective} ~ f: \mathcal{AV} \to \mathcal{SVAR}.$
% 	%
% \[
% 	\forall \av \in \midV. ~ 
% 	f(\av) \in \progV \land \progF(f(\av)) = \midF(\av) \land
% 	|\kw{image}(f(\av))| \leq W(f(\av))
% \]
% \end{lem}
% \begin{subproof}
% Proving by Definition~\ref{def:midgraph}.
% \end{subproof}
% }
% \\
% \todo{
% 	\begin{lem}[Surjective Mapping from $\edges$ of $\midG)$ to $\edges$ of $\progG$].
% 	\label{lem:suje_mid_to_prog}
% 	\\
% 	$\midG({c},{m},\text{D}) = \{\midV, \midE, \midF\}$
% 	\\
% 	$\progG({c}) = \{\progV, \progE, \progF, \progW\}$
% 	\\
% 	A surjective function $f: \progV \to \midV$ s.t.,
% 	$\forall \av \in \midV. ~ \progF(f(\av)) = \midF(\av) \land |\kw{image}(f(\av))| \leq W(f(\av))$
% 	%
% \[
% 	\exists ~ \kw{surjective} ~ g: \midE \to \progE. ~
% 	\forall e_{mid} = (\av_1, \av_2) \in \midE. 
% 	\exists e_{prog} = ({f(\av_1), f(\av_2)}) \in \progE
% \]
% \end{lem}
% \begin{subproof}
% Proving by Lemma~\ref{lem:sujv_mid_to_prog}.
% \end{subproof}
% }
% \\
% \todo{
% 	\begin{lem}[Surjective Mapping from $\paths(\midG)$ to $\walks(\progG)$].
% 	\label{lem:sujpathwalk_mid_to_prog}
% 	\\
% 	$\midG({c},{m},\text{D}) = \{\midV, \midE, \midF\}$
% 	\\
% 	$\progG({c}) = \{\progV, \progE, \progF, \progW\}$
% 	\\
% 	A surjective function $f: \progV \to \midV$ s.t.,
% 	$\forall \av \in \midV. ~ \progF(f(\av)) = \midF(\av) \land |\kw{image}(f(\av))| \leq W(f(\av))$
% 	\\
% 	A surjective function $g: \midE \to \progE$ s.t.,
% 	$\forall e_{mid} = (\av_1, \av_2) \in \midE. 
% 	\exists e_{prog} = ({f(\av_1) \to f(\av_2)}) \in \progE$
% 	\\
% 	$\exists ~ \kw{surjective} ~ h: \paths(\midG({c},{m},\text{D})) \to \walks(\progG({c}))$ s.t.:
% 	%
% \[
% 	\forall p_{\av_1 \to \av_2} \in \paths(\midG({c},{m},\text{D}))
% 	\text{ with }
% 	\left\{
% 	\begin{array}{ll}
% 	\mbox{edge sequence:} & (e, \ldots, e_{n-1})
% 	\\ 
% 	\mbox{vertices sequence:} & (\av_1, \ldots, \av_n)
% 	\end{array}
% 	\right.
% \]
% \[
% 	\exists k_{f(\av_1) \to f(\av_2)} \in \walks(\progG({c}))
% 	\text{ with }
% 	\left\{
% 	\begin{array}{ll}
% 	\mbox{edge sequence:} & (g(e), \ldots, g(e_{n-1}) 
% 	\\ 
% 	\mbox{vertices sequence:} & (f(\av_1), \ldots, f(\av_{n}))
% 	\end{array}
% 	\right.
% \]
% % \item $(e, \ldots, e_{n-1})$, $(\av_1, \ldots, \av_n)$ are the edges sequence and vertices sequence of $p_{\av_1 \to \av_2}$.
% % then, 
% %  $\len(p_{\av_1 \to \av_2}) = \len(k_{f(\av_1) \to f(\av_2)})$
% % %
% % \item $\forall \av \in p_{\av_1 \to \av_2}. ~ f(\av) \in k_{f(\av_1) \to f(\av_2)}$
% % %
% % \item $\forall \av \in p_{\av_1 \to \av_2}. ~ 
% % \kw{image}(f(\av)) \cap {p_{\av_1 \to \av_2}}| = \# \{f(\av) \mid f(\av) \in k_{f(\av_1) \to f(\av_2)}\}
% % $
% \end{lem}
% %
% \begin{subproof}
% Proving by induction on the length of $l = p_{\av_1 \to \av_2} \in \paths(\midG({c},{m},\text{D}))$, and Lemma~\ref{lem:suje_mid_to_prog} and Lemma~\ref{lem:sujv_mid_to_prog}.
% \caseL{ $l = 1$: }
% \caseL{ $l = l' + 1$, $l' \geq 1$: }
% \end{subproof}
% }
% \end{proof}
% %

% %
% }

% % % % 


% \wq{dfs: Trace -> command -> DB -> trace set?  May we can a few sentences to explain it. It seems useful:-) }
% \jl{added types for them}
% \wq{setfilter returns a set of trace? (fun l -> ....): trace -> trace?  So the top level is: setmap (function f: trace -> trace) (set of trace), setmap applies the function to the input set of traces and returns a set of traces. }

% \wqside{I can see this algirithm is over approximated, frome case $\event' $ is a test event.}

% \wq{To be honest, soundness to me is that.  If an algorithm A is sound w.r.t to Dep, then we mean for every output of A, say, (x,y) in A implies Dep(x,y). Do you mean that or you mean soundness of DEP w.r.t algorithm instead? }
\clearpage
\section{Soundness}
\begin{thm}[Soundness of the \THESYSTEM]
    \label{thm:adaptfun_soundness}
  Given a program ${c}$, we have:
  %
  \[
  \progA({c}) \geq A({c}).
  \]
  \end{thm}
\clearpage
%
\section{Examples and Experimental Results}
The two round strategy works well in our framework, we explore further to look at an advanced adaptive data analysis algorithm - multiple round algorithm.
\begin{example}[Two Round Algorithm]
    \[
    %
        \kw{twoRounds(k)} \triangleq
    \begin{array}{l}
           \clabel{ a \leftarrow []}^{1} ; \\
            \clabel{\assign{j}{k} }^{2} ; \\
            \ewhile ~ \clabel{j > 0}^{3} ~ \edo ~ \\
            \Big(
             \clabel{\assign{x}{\query(\chi[k - j]\cdot \chi[k])} }^{4}  ; \\
             \clabel{\assign{j}{j-1}}^{5} ;\\
            \clabel{a \leftarrow x :: a}^{6}       \Big);\\
            \clabel{l \leftarrow (\mathrm{sign}\big (\sum_{i\in [k]} \chi[i]\times\ln\frac{1+a[i]}{1-a[i]} \big ))}^{7}\\
        \end{array}
    \]
    %
    \begin{algorithm}
    \footnotesize
    \caption{A two-round analyst strategy for random data (The example in  \cite{dwork2015preserving})}
    \label{alg:twoRound}
    \begin{algorithmic}
    \REQUIRE Mechanism $\mathcal{M}$ with a hidden data set $D \in \{-1,+1\}^{n\times (k+1)} \subset \dbdom$.
    \STATE  {\bf for}\ $j\in [k]$\ {\bf do}.  
    \STATE \qquad {\bf define} $q_j(d)=d(j)\cdot d(k)$ where $d \in \{D(i) ~|~ i = 0, \cdots, n\} \subseteq \{-1,+1\}^{k+1}$.
    \STATE \qquad {\bf let} $a_j=\mathcal{M}(q_j)$ 
    \STATE \qquad \COMMENT{In the line above, $\mathcal{M}$ computes approx. the exp. value  of $q_j$ over $D$. So, $a_j\in [-1,+1]$.}
    \STATE {\bf define} $q_{k}(d)= d(k) \cdot \mathrm{sign}\big (\sum_{i\in [k]} x(i) \cdot \ln\frac{1+a_i}{1-a_i} \big )$ where $x\in \{-1,+1\}^{k+1}$.
    \STATE\COMMENT{In the line above,  $\mathrm{sign}(y)=\left \{ \begin{array}{lr} +1 & \mathrm{if}\ y\geq 0\\ -1 &\mathrm{otherwise} \end{array} \right . $.}
    \STATE {\bf let} $a_{k+1}=\mathcal{M}(q_{k+1})$
    \STATE\COMMENT{In the line above,  $\mathcal{M}$ computes approx. the exp. value  of $q_{k+1}$ over $X$. So, $a_{k+1}\in [-1,+1]$.}
    \RETURN $a_{k+1}$.
    \ENSURE $a_{k+1}\in [-1,+1]$
        % \ENSURE 
    \end{algorithmic}
    \end{algorithm}
    %
%
    \end{example}

\begin{example}[Multiple Round Algorithm]
%
\[
%
\kw{multipleRounds(k, c)} \triangleq
\begin{array}{l}
     \left[j \leftarrow k \right]^1 ; \\
    \left[I \leftarrow [] \right]^2; \\
    \ewhile ~ \clabel{j > 0}^{3} ~ \edo ~ \\
    \Big(
    \clabel{\assign{j}{j-1}}^{4} ;\\
    \left[p \leftarrow c \right]^5; \\
    \left[ a \leftarrow \query (\chi[I]) \right]^6;\\
    \left[I \leftarrow \mathrel{\mathsf{update}} ( {I}, (a, p))  \right]^7
    \Big) 
\end{array}
\]
%
\begin{algorithm}
\footnotesize
\caption{A multi-round analyst strategy for random data base \cite{dwork2015preserving}}
\label{alg:multiRound}
\begin{algorithmic}
\REQUIRE Mechanism $\mathcal{M}$ with a hidden state $X\in [N]^{n}$ sampled u.a.r., control set size $c$
\STATE Define control dataset $C = \{0,1, \cdots, c - 1\}$
\STATE Initialize $Nscore(i) = 0$ for $i \in [N]$, $I = \emptyset$ and $Cscore(C(i)) = 0$ for $i \in [c]$
\STATE  {\bf for}\ $j\in [k]$\ {\bf do} 
\STATE \qquad {\bf let} $p=\uniform(0,1)$ 
\STATE \qquad {\bf define} $q (x) = \bernoulli ( p )$ .
\STATE \qquad {\bf define} $qc (x) = \bernoulli ( p )$ .
\STATE \qquad {\bf let} $a = \mathcal{M}(q)$ 
\STATE \qquad {\bf for}\ $i \in [N]$\ {\bf do}
\STATE \qquad \qquad $Nscore(i) = Nscore(i) + (a - p)*(q (i) - p)$ if $i \notin I$
\STATE \qquad {\bf for}\ $i \in [c]$\ {\bf do}
\STATE \qquad \qquad $Cscore(C(i)) = Cscore(C(i)) + (a - p)*(qc (i) - p)$
\STATE \qquad {\bf let} $I = \{i | i\in [N] \land Nscore(i) > \max(Cscore)\}$
\STATE \qquad {\bf let} $D = D \setminus I$ 
\RETURN $D$.
\end{algorithmic}
\end{algorithm}
%
\[
%
\kw{multipleRounds(k, c, N)} \triangleq
\begin{array}{l}
    \clabel{\assign{j}{N}}^0 ; \\
     \clabel{\assign{cs}{0}}^1; \\
     \clabel{\assign{ns}{0}}^2; \\
     \clabel{\assign{I}{0}}^3; \\
     \ewhile ~ \clabel{j > 0}^{4} ~ \edo ~ \\
     \Big(
     \clabel{\assign{j}{j-1}}^{5} ;\\
     \clabel{\assign{cs}{0 + cs}}^6; \\
     \clabel{\assign{ns}{0 + ns}}^7
     \Big); \\
     \clabel{\assign{w}{k}}^{8} ;\\
     \ewhile ~ \clabel{w > 0}^{9} ~ \edo ~ \\
    \Big(
    \clabel{\assign{w}{w-1}}^{10} ;\\
    \left[p \leftarrow c \right]^{11}; \\
    \left[q \leftarrow c \right]^{12}; \\
    \left[ a \leftarrow \query (\chi[I]) \right]^{13};\\
    \clabel{\assign{i}{N}}^{14} ; \\
    \ewhile ~ \clabel{i > 0}^{15} ~ \edo ~ \\
    \Big(
    \clabel{\assign{i}{i-1}}^{16} ;\\
    \clabel{\assign{cs(i)}{cs(i) + (a - p) * (q - p)}}^{17}; \\
    \eif (\clabel{ I < i}^{18}, \clabel{\assign{ns(i)}{{ns(i) + (a - p) * (q - p)}}}^{19},
    \clabel{\assign{ns}{ns(i)}}^{20}    )
    \Big); \\
    \clabel{\assign{i2}{N}}^{21} ; \\
    \ewhile ~ \clabel{i2 > 0}^{22} ~ \edo ~ \\
    \Big(
    \clabel{\assign{i2}{i2-1}}^{23} ;\\
    \eif (\clabel{ns(i2) > \kw{max}(cs)}^{24}, 
    \clabel{\assign{I}{i + I}}^{25},
    \clabel{\assign{I}{I}}^{26})
    \Big)
    \Big) 
\end{array}
\]
weight for Variable: cs of label 6 is: 0 + 1 * N \\
weight for Variable: j of label 5 is: 0 + 1 * N \\
weight for Variable: ns of label 7 is: 0 + 1 * N \\
weight for Variable: csi of label 17 is: 0 + 0 + 1 * k * N \\
weight for Variable: i of label 16 is: 0 + 0 + 1 * k * N \\
weight for Variable: nsi of label 19 is: 0 + 0 + 1 * k * N \\
weight for Variable: nsi of label 20 is: 0 + 0 + 1 * k * N \\
weight for Variable: i2 of label 23 is: 0 + 0 + 1 * k * N \\
weight for Variable: l of label 25 is: 0 + 0 + 1 * k * N \\
weight for Variable: l of label 26 is: 0 + 0 + 1 * k * N \\
weight for Variable: i2 of label 21 is: 0 + 1 * k \\
weight for Variable: i of label 14 is: 0 + 1 * k \\
weight for Variable: a of label 13 is: 0 + 1 * k \\
weight for Variable: q of label 12 is: 0 + 1 * k \\
weight for Variable: p of label 11 is: 0 + 1 * k \\
weight for Variable: w of label 10 is: 0 + 1 * k \\
weight for Variable: w of label 8 is: 1 \\
weight for Variable: ns of label 3 is: 1 \\
weight for Variable: cs of label 2 is: 1 \\
weight for Variable: l of label 1 is: 1 \\
weight for Variable: j of label 0 is: 1 \\
%
\end{example}
  We have seen the two round algorithm above. We show the multiple-round algorithm, which is an advanced algorithm.
 \\
\textbf{Description:}
The multiple round algorithm starts from an initialized empty tracking list $I$, a score called Nscore $ns=0$ , another score Cscore $cs=0$.
There is a hidden database $D$ as well.
% a score called Nscore $ns=0$ , another score Cscore $cs=0$. There is a hidden database $X$ as well.
% It goes $k$ rounds and every round, the two scores $ns$ and $cs$ are updated by a query result. 
% Then the list $I$ is updated by the two scores for every round. After the $r$ rounds, the algorithm returns the columns of the hidden database $X$ not specified in the tracking list $I$, which is $X\setminus I$. 
It goes $k$ rounds and every round, the two scores $ns$ and $cs$ are updated by a query result. 
Then the tracking list is updated by the two scores for every round.  
% Then the list $I$ is updated by the two scores for every round. 
After the $r$ rounds, the algorithm returns the columns of the hidden database $D$ not specified in the tracking list $I$, which is $D \setminus I$. 
\\
The algorithm is written in the while language as $\kw{multipleRounds(k, c)} $ taking two parameters $k$ and $c$ for 
number of iterations and the distribution sampling primitive $c$.
It starts from an initialized empty tracking list $I$ as well,
% a score called Nscore $ns=0$ , another score Cscore $cs=0$. There is a hidden database $X$ as well.
% It goes $k$ rounds and every round, the two scores $ns$ and $cs$ are updated by a query result. 
% Then the list $I$ is updated by the two scores for every round. After the $r$ rounds, the algorithm returns the columns of the hidden database $X$ not specified in the tracking list $I$, which is $X\setminus I$. 
It goes $k$ rounds and every round, construct the query $\query(\chi[I])$
and obtain the query result $a$.
Then, the tracking list $I$ is updated by a query result. 
% Then the list $I$ is updated by the two scores for every round. 
After the $r$ rounds, the algorithm returns the columns of the hidden database $D$ not specified in the tracking list $I$.
The $\mathrel{\mathsf{update}} ( {I}, (a, p))$ function takes $I, a, p$ as input and compute the updated results for $I$.
$\mathsf{update}$ function is used here to simplify the complex update computation of Nscore, Cscore and the tracking list $I$.
It will not change our analysis because these functions provides enough information through their arguments.%

% It uses a loop for the $k$ rounds computation and. We use functions $update\_nscore(p,a)$,$update\_cscore(p,a)$,$update(I,ns,cs)$ to simplify the complex update computation of Nscore, Cscore and the tracking list $I$. It will not change our analysis because these functions provides enough information through their arguments.
% As described in the two round algorithm, the multi-round algorithm has a loop as well.
% compare to two round algorithm
In comparison with the two round algorithm, the query asked in each iteration is not independent  in the multiple round one any more. 
The query in one iteration $j$ now depends on the tracking list $I$ from its previous iteration $j-1$, which is updated by the query result at the same iteration $j-1$. We can easily see the connection between queries from different iterations.
% the result of the query from previous iteration,
% so that the query ask at the $j^{th}$ iteration is
% $q(p, I)$.
%
%
%
\begin{figure}
\begin{center}
%
\begin{tikzpicture}[scale=\textwidth/17cm,samples=200]
%%% The nodes represents the k query in the first round
\filldraw[red] (0, 3) circle (2pt) node [anchor=south]{$a_1^{(4,1)}$};
\filldraw[black] (3, 4) circle (2pt) node [anchor=south]{$p_1^{(3,1)}$};
% \filldraw[black] (6, 2) circle (2pt) node [anchor=south]{$q^4_3$};
\filldraw[black] (6, 4) circle (2pt) node [anchor=south]{$p_1^{(3,2)}$};
\filldraw[black] (8, 3) circle (2pt) node [anchor=south]{$I_3^{(2,1)}$};
%%%%%% The nodes represents the n^k queries in the second round
\filldraw[red] (0, 2) circle (2pt) node [anchor=north]{$a_1^{(4,2)}$};
\filldraw[black] (3, 0) circle (2pt) node [anchor=north]{$I_2^{(5,1)}$};
% \filldraw[black] (6, 0) circle (2pt) node [anchor=north]{$q^{3, 7}_{k+1}$};
\filldraw[black] (6, 0) circle (2pt) node [anchor=north]{$I_2^{(5,2)}$};
\filldraw[black] (8, 1) circle (2pt) node [anchor=north]{$I_3^{(2,3)}$};
\filldraw[black] (8, 2) circle (2pt) node [anchor=south]{$I_3^{(2,2)}$};
\filldraw[black] (12, 2) circle (2pt) node [anchor=south]{$I_1^{1}$};
%%%%%The edges between a and I
%%%%% (a1(4,1), I3(2,1))
\draw[very thick, ->] (0, 3)  -- (7.9, 3) ;
%%%%% (a1(4,2), I3(2,2))
\draw[very thick, ->] (0, 2)  -- (7.9, 2) ;
%%%%%% The edges represents their dependency relations GROUP between I3 and I1
\draw[very thick,<-] (12, 2)  -- (8, 2) ;
\draw[very thick,->] (8, 2) -- (3.1, 0) ;
%
\draw[very thick,<-] (12, 2)  -- (8, 1) ;
\draw[very thick,->] (8, 1) -- (6.1, 0) ;
%
\draw[very thick,<-] (12, 2)  -- (8, 3) ;
%
%%%%%% The edges represents their dependency relations GROUP between I2 and others
%%%%%% The edges represents their dependency relations GROUP between I2(5,1) and others
\draw[very thick, ->] (3, 0)  -- (0, 2.9) ;
\draw[very thick, ->] (3, 0)  -- (3, 3.9) ;
\draw[very thick, ->] (3, 0)  -- (7.9, 2.9) ;
%%%%%% The edges represents their dependency relations GROUP between I2(5,2) and others
\draw[very thick, ->] (6, 0)  -- (0, 1.9) ;
\draw[very thick, ->] (6, 0)  -- (6, 3.9) ;
\draw[very thick, ->] (6, 0)  -- (7.9, 1.9) ;
%%%% The longest path representing the adaptivity
\draw[ultra thick, red, ->, dashed] (0, 2) -- (7.9, 2);
\draw[ultra thick, red, ->, dashed] (8, 2) -- (3.1, 0);
\draw[ultra thick, red, ->, dashed] (3, 0)  -- (0, 2.9);
\end{tikzpicture}
\end{center}
    \caption{the variable dependency graph for multi round algorithm}
    \label{fig:multi-round-graph-ssa}
\end{figure}
%
The adaptivity is 1 computed from the graph.
The query-based dependency graph is a subgraph of the variable dependency graph for multi round algorithm.



\begin{example}[Sequence with Single Variable Linear Data Value Dependency]
    %
    %
    \[
    %
        \kw{seq()} \triangleq 
    \begin{array}{l} 
           \clabel{ \assign{x}{\chi[0]}}^{0} ; \\
            \clabel{\assign{y}{\chi[x + 1]} }^{1} ; \\
            \clabel{\assign{z}{\chi[y + 1]}}^{2}; \\
             \clabel{\assign{w}{\chi[z + 1]} }^{3}
        \end{array}
    \]
    Analysis Result: $ \progA( \kw{seq()}) = 4$
    \end{example}
%
\begin{example}[Sequence with Multiple Variables Data Value Dependency]
    %
    %
    \[
    %
        \kw{seqMultiVar()} \triangleq 
    \begin{array}{l} 
           \clabel{ \assign{x}{\chi[0]}}^{0} ; \\
            \clabel{\assign{y}{\chi[x + 1]} }^{1} ; \\
            \clabel{\assign{z}{\chi[y + x]}}^{2}; \\
             \clabel{\assign{w}{\chi[z + 1] \cdot \chi[y]} }^{3}
        \end{array}
    \]
    Analysis Result: $ \progA(\kw{seqMultiVar()}) = 4$
\end{example}
%
    \begin{example}[If with Data-Value Dependency Separated]
        %
        %
        \[
        %
        \kw{ifValueDependency}(k) \triangleq 
        \begin{array}{l}
           \quad \clabel{ \assign{z}{\query(\chi[0])}}^{0} ; \\
           \quad \clabel{\assign{x}{k / 2} }^{1} ; \\
           \quad \eif(\clabel{x < 0}^2, \\
           \quad \clabel{\assign{y}{\query(\chi[z])}}^{3}, \\ 
           \quad \clabel{\assign{y}{\query(\chi[0])}}^{4})
            \end{array}
        \]
        Analysis Result: $ \progA( \kw{ifControlDependency()}) = 3$
    \end{example}

        \begin{example}[If with Data-Control Dependency Overlapped]
            %
            %
            \[
            %
            \kw{ifControlDependency()} \triangleq 
            \begin{array}{l}
                \clabel{ \assign{z}{\query(\chi[0])}}^{0} ; \\
                \clabel{\assign{x}{\query(\chi[z])} }^{1} ; \\
                \eif(\clabel{x < 0}^{2}, 
                \clabel{\assign{y}{\query(\chi[0] + \chi[1])}}^{3}, 
                \clabel{\assign{y}\query{(\chi[0])}}^{4})
            \end{array}
            \]
            %
            Analysis Result: $ \progA( \kw{ifControlDependency()}) = 3$
            \end{example}


            \begin{example}[Simple While with Recursive Data-Value Dependency]
                %
                %
                \[
                %
                \kw{whileRec}() \triangleq
                \begin{array}{l}
                    \clabel{ \assign{j}{10000} }^{0} ; \\
                    \clabel{ \assign{a}{\query(\chi[0])} }^{1} ; \\
                        \ewhile ~ \clabel{j > 0}^{2} ~ \edo ~ \\
                        \Big(
        \clabel{\assign{x}{\query(\chi[a]) }}^{3}  ; \\
        \clabel{\assign{a}{x + a}}^{4} ;\\
                        \clabel{\assign{j}{j-1}}^{5}       \Big)
                    \end{array}
                \]
                Analysis Results: $ \progA(\kw{whileRec}(k)) = 1 + k$
            \end{example}
%
        \begin{example}[Simple While with Multi-Path Data-Value Dependency]
        %
        %
        \[
        %
        \kw{whileMultiplePath(k)} \triangleq 
        \begin{array}{l}
            \clabel{ \assign{j}{k}}^{0} ; \\
            \clabel{ \assign{x}{\query(\chi[0])} }^{1} ; \\
                \ewhile ~ \clabel{j > 0}^{2} ~ \edo ~ \\
                \Big(
                 \clabel{\assign{j}{j-1}}^{3} ;\\
                 \eif(\clabel{j \% 2 == 0}^{4}, 
                 \clabel{\assign{y}{\chi[x]}}^{5}, 
                 \clabel{\assign{w}{\chi[x]}}^{6});\\           
                 \clabel{\assign{x}{\query(\chi(\ln(y)))} }^{7} \Big)
            \end{array}
        \]
        Analysis Results: $ \progA(\kw{whileMultiplePath}(k)) = 1 + 2 * k $ --> Over-Approximated
    \end{example}
%
        \begin{example}[Simple While with Recursive Multiple-Variable Data-Value Dependency]
            \[
            %
            \kw{whileMultipleVar}(k) \triangleq 
            \begin{array}{l}
            \clabel{\assign{j}{k} }^{0} ; \\
            \clabel{ \assign{x}{\query(\chi[0])}}^{1} ; \\
                \clabel{ \assign{y}{\query(\chi[1])}}^{2} ; \\
                    \ewhile ~ \clabel{j > 0}^{3} ~ \edo ~ \\
                    \Big(
                     \clabel{\assign{j}{j-1}}^{4} ;\\
                     \clabel{\assign{z}{\query(\chi(x + \ln(y)))} }^{5}  ; \\
                     \clabel{ \assign{x}{\query(\chi[z])}}^{6} ; \\
                     \clabel{ \assign{y}{\query(\chi[z])}}^{7} 
                    \Big)
                \end{array}
            \]
            Analysis Results: $ \progA(\kw{whileMultipleVar}(k)) = 1 + 2 * k $
        \end{example}
            %
            %
            \begin{example}[Simple While with Data-Value and Data-Control Dependency]
                %
                \[
                \kw{whileValueControlDependency}() \triangleq
                \begin{array}{l}
                    \clabel{ \assign{x}{\query(\chi[0])} }^{0} ; \\
                    \clabel{ \assign{z}{\query(\chi[0])} }^{1} ; \\
                        \ewhile ~ \clabel{x > 0}^{2} ~ \edo ~ \\
                        \Big(
                        \clabel{\assign{x}{\query(\chi(z))} }^{3}  ; \\
                        \clabel{\assign{z}{\query(\chi(x))}}^{4}
                      \Big)
                    \end{array}
                \]
                Analysis Results: $ \progA(\kw{whileValueControlDependency}(k)) = 1 + 2 * k $
            \end{example}
%
            \begin{example}[Simple While with MultiplePath Data-Value and Data-Control Dependency]
                %
                \[
                    %
                    \begin{array}{l}
                    \kw{whileMultiplePathValueControlDependency}(k) \triangleq\\
                        \clabel{ \assign{x}{\query(k)}}^{0} ; \\
                        \clabel{\assign{y}{0} }^{1} ; \\
           \ewhile ~ \clabel{x > 0}^{2} ~ \edo ~ \\
           \Big(
            \eif(\clabel{y > 0}^{3}, 
            \clabel{\assign{y}{\query(\chi[12])}}^{4}, 
            \clabel{\assign{w}{\query(\chi[9])}}^{5});           
            \\
            \clabel{\assign{x}{x-1}}^{6}\Big);\\
            \clabel{\assign{y}{\query(\chi(\ln(y)))} }^{7} 
                        \end{array}
                    \]
                    Analysis Results: $ \progA(\kw{whileMultiplePathValueControlDependency}(k)) = 2 + k $
                \end{example}
               %
                \begin{example}[Nested While with Recursive Data-Value Dependency]
                    %
                    %
                    \[
                    %
                    \kw{nestWhileValueDependency}(k) \triangleq 
                    \begin{array}{l}
                        \clabel{ \assign{i}{k} }^{0} ; \\
                        \clabel{\assign{x}{\query(\chi[0])}}^{1} ; \\
           \ewhile ~ \clabel{i > 0}^{2} ~ \edo ~ \\
           \Big(
            \clabel{\assign{i}{i-1}}^{3} ;\\
            \clabel{\assign{j}{k}}^{4} ;\\
            \clabel{\assign{y}{\query(\chi(\ln(x)))} }^{5}  ; \\
            \ewhile ~ \clabel{j > 0}^{6} ~ \edo ~ \\
            \Big(
             \clabel{\assign{j}{j-1}}^{7};\\
             \clabel{\assign{x}{\query(\chi(\ln(x)))} }^{8}
             \Big) \Big)
                        \end{array}
                    \]
                    Analysis Results: $ \progA(\kw{nestWhileValueDependency}(k)) = 2 + k^2 $
                \end{example}

                    \begin{example}[Nested While with Nested Recursive Data-Value Dependency Across Outer and Inner Loop]
                        %
                        %
                        \[
                        %
           \kw{nestedWhileRecAcross}(k) \triangleq 
                        \begin{array}{l}
           \clabel{ \assign{i}{k} }^{0} ; \\
           \clabel{\assign{x}{\query(\chi[0])}}^{1} ; \\
               \ewhile ~ \clabel{i > 0}^{2} ~ \edo ~ \\
               \Big(
                \clabel{\assign{i}{i-1}}^{3} ;\\
                \clabel{\assign{j}{k}}^{4} ;\\
                \ewhile ~ \clabel{j > 0}^{5} ~ \edo ~ \\
                \Big(
                 \clabel{\assign{j}{j-1}}^{6};\\
                 \clabel{\assign{y}{\query(\chi(x) + \chi(1))} }^{7}
                 \Big); \\
                \clabel{\assign{x}{\query(\chi(\ln(y)))} }^{8}
                 \Big)
           \end{array}
                        \]
                        Analysis Results: $ \progA(\kw{nestedWhileRecAcross}(k)) = 1 + 2 * k $
                    \end{example}
                %
            
                        \begin{example}[Nested While with Nested Recursive Multiple Variable 
           Data-Value Dependency Across Outer and Inner Loop]
           %
           \[
           %
           \kw{nestedWhileMultiVarRecAcross}(k) \triangleq 
           \begin{array}{l}
               \clabel{\assign{i}{k} }^{0} ; \\
               \clabel{ \assign{x}{\query(\chi[0])}}^{1} ; \\
               \clabel{ \assign{y}{\query(\chi[1])}}^{2} ; \\
          \ewhile ~ \clabel{i > 0}^{3} ~ \edo ~ \\
          \Big(
           \clabel{\assign{i}{i-1}}^{4} ;\\
           \clabel{\assign{j}{k}}^{5} ;\\
           \clabel{\assign{y}{\query(\chi(\ln(x) + y))} }^{6}  ; \\
           \ewhile ~ \clabel{j > 0}^{7} ~ \edo ~ \\
           \Big(
            \clabel{\assign{j}{j-1}}^{8};\\
            \clabel{\assign{x}{\query(\chi(\ln(y))+\chi[x])} }^{9}
            \Big) \Big)
               \end{array}
           \]
           Analysis Results: $ \progA(\kw{nestedWhileMultiVarRecAcross}(k)) = 1 + k + k^2$
           \\
           weight for Variable: j of label 6 is: 0 + 0 + 1 * k * k\\
           weight for Variable: y of label 7 is: 0 + 0 + 1 * k * k\\
           weight for Variable: j of label 4 is: 0 + 1 * k\\
           weight for Variable: i of label 3 is: 0 + 1 * k\\
           weight for Variable: x of label 8 is: 0 + 1 * k\\
           weight for Variable: x of label 1 is: 1\\
           weight for Variable: i of label 0 is: 1\\
           \end{example}
                    
           \begin{example}[Nested While with MultiplePath and Nested Recursive Multiple Variable 
               Data-Value Dependency Across Outer and Inner Loop]
               %
               We then show a more complex example with nested while command and nested data-flow across the outer and inner while loop through multiple variables.
               This example also contains the if command with data dependency occurred through the if guard.
               %
               \[
               %
               \begin{array}{l}
               \kw{nestedWhileMultiPathMultiVarRecAcross}(k) \triangleq \\
          \clabel{\assign{i}{k} }^{0} ; \\
          \clabel{ \assign{x}{\query(\chi[0])}}^{1} ; \\
          \clabel{ \assign{y}{\query(\chi[1])}}^{2} ; \\
              \ewhile ~ \clabel{i > 0}^{3} ~ \edo ~ \\
              \Big(
               \clabel{\assign{i}{i-1}}^{4} ;\\
               \clabel{\assign{j}{k}}^{5} ;\\
               \eif(\clabel{x > 0}^6, \clabel{\assign{y}{\query(\chi(\ln(x) + y))} }^{7},
               \clabel{\assign{y}{\query(\chi(x))} }^{8} )
                ; \\
               \ewhile ~ \clabel{j > 0}^{9} ~ \edo ~ \\
               \Big(
                \clabel{\assign{j}{j-1}}^{10};\\
                \clabel{\assign{x}{\query(\chi(\ln(y))+\chi[x])} }^{11}
                \Big) \Big)
          \end{array}
               \]
               Analysis Results: $ \progA(\kw{nestedWhileMultiPathMultiVarRecAcross}(k)) = 1 + k + k^2$
               \\
               weight for Variable: j of label 10 is: 0 + 0 + 1 * k * k \\
               weight for Variable: x of label 11 is: 0 + 0 + 1 * k * k \\
               weight for Variable: y of label 7 is: 0 + 1 * k \\
               weight for Variable: y of label 8 is: 0 + 1 * k \\
               weight for Variable: j of label 5 is: 0 + 1 * k \\
               weight for Variable: i of label 4 is: 0 + 1 * k \\
               weight for Variable: y of label 2 is: 1 \\
               weight for Variable: x of label 1 is: 1 \\
               weight for Variable: i of label 0 is: 1 \\
               \end{example}

               \begin{example}[Gradient Decent Algorithm]
          \[
          %
          \kw{gradientDecent(step, rate, t, n)} \triangleq
          \begin{array}{l}
                 \clabel{ a \leftarrow []}^{0} ; \\
                  \clabel{\assign{j}{step} }^{1} ; \\
                %   \clabel{\assign{d}{10000000} }^{2} ; \\
                  \ewhile ~ \clabel{j > 0 \land d < t}^{3} ~ \edo ~ \\
                  \Big(
                      \clabel{\assign{d}{\query(2 * (\chi[1] - (\chi[0]\times x )) * (-\chi[0]))} }^{4}  ; \\
                      \clabel{\assign{x}{x - rate * d} }^{4}  ; \\
                   \clabel{\assign{j}{j-1}}^{5} ;\\
                  \clabel{a \leftarrow x :: a}^{6} 
                  \Big);
              \end{array}
          \]
          %
          %
               %
          \end{example}

               \begin{example}[Two Round Odd Elements ]
          We present a variant of the previous two round example in Figure~\ref{fig:tworound_odd}. In this odd example, only the data at odd index of the database is used.
          %
          {\small
          \begin{figure}
          \begin{subfigure}{.4\textwidth}
          \begin{centering}
          $\begin{array}{l}
              % \left[j \leftarrow 0 \right]^1 ; \\
             \clabel{ \assign{a}{0}  }^{1} ; \\
              \clabel{\assign{j}{0} }^{2} ; \\
              \eloop ~ \clabel{3}^{3} ~ \edo ~ \\
             \quad 
               \eif ( \clabel{( j \% 2 == 0)}^{4}\\
               \quad \quad ,\clabel{ \assign{x}{ q(\chi[j+1])} }^{5}  \\
               \quad \quad ,\clabel{\assign{x}{ q(\chi[j]) }}^{6} ) ;\\
               \quad \clabel{\assign{a}{ a+x} }^{7}  ;\\
               \quad \clabel{ \assign{j}{j+1} }^{8}      ;\\
           \clabel{l \leftarrow q(a*\chi[3]) }^{9}\\
          \end{array} $
          \caption{}
          \end{centering}
          \end{subfigure}
          \begin{subfigure}{0.5\textwidth}
          \begin{centering}
          $
          \begin{array}{l}
              \clabel{ \assign{a_1}{0}  }^{1} ; \\
              \clabel{\assign{j_1}{0} }^{2} ; \\
              \eloop ~ \clabel{3}^{3} , 0,  ~ \edo ~ [(j_3, j_1,j_2), (a_3, a_1,a_2)], \\
            \quad 
               \eif ( \clabel{( j_3 \% 2 == 0)}^{4}, [x_3,x_1,x_2], [],[]\\
               \quad \quad ,\clabel{ \assign{x_1}{ q(\chi[j_3+1])} }^{5}  \\
               \quad \quad ,\clabel{\assign{x_2}{ q(\chi[j_3]) }}^{6} ) ;\\
               \quad \clabel{\assign{a_2}{ a_3+x_3} }^{7}  ;\\
               \quad \clabel{ \assign{j_2}{j_3+1} }^{8}      ;\\
           \clabel{l_1 \leftarrow q(a_3*\chi[3]) }^{9}\\
          \end{array}$
          \caption{}
          \end{centering}
          \end{subfigure}
              \vspace{-0.2cm}
              \caption{(a) The two round odd algorithm in labeled {\tt Loop} language, (b) The SSA program for the same example. }
              \label{fig:tworound_odd}
              \vspace{-0.5cm}
          \end{figure}
          }
          \end{example}
          %
          This algorithm only touches the odd part of the database, by adding an extra if statement to checking the index $j$ in Figure~\ref{fig:tworound_odd}(a). The extra complexity is added to handle the newly generated variables in the loop and if statement in the SSA version in Figure~\ref{fig:tworound_odd}(b). 
          The query-based dependency graph does not change a lot compared to the previous two rounds example in Figure~\ref{fig:simpl-two-round-graph}(c), but the node does change according to the trace. We assume $a = n$ in the final memory is the result of the sum of previous query results in the loop.
          We give the trace $t = [q(\chi[1])^{5,[3:1]}, q(\chi[1])^{6,[3:2]}, q(\chi[3])^{5,[3:3]}, q(n* \chi[3])^{9,\emptyset} ]$ and use $q_1$ for $q(\chi[1])$, $q_3$ representing for $q(\chi[3])$, $q_4$ for $q(n* \chi[3])$. The query-based dependency graph based on this trace is shown in Figure~\ref{fig:odd_graphs}(a). We show the red path, which is a sequence of adaptively chosen queries of length $2$. So among the total $4$ queries, we have 2-round adaptive queries. According to the Theorem\ref{thm:gaussiannoise} and \ref{thm:gaussiannoise2}, we will have a tighter upper bound on the generalization error if we know the adaptivity $2$, obtained from the red path in Figure~\ref{fig:odd_graphs}(a). 
          
          Our algorithm {\THESYSTEM} gives us the upper bound on the aforementioned adaptivity $2$. We construct the variable-based weighted dependency graph in Figure~\ref{fig:odd_graphs}(b). The weighted nodes are in the red dashed circle and the red dashed paths show the most weighted path in the graph, with the weight $2$. So, for this two rounds odd algorithm, our system gives a tight upper bound $2$, which can be used to get a better generalization error bound.
          
          
          
          
          \begin{figure}
              \begin{subfigure}{0.4\textwidth}
              \begin{centering}
              \begin{tikzpicture}[scale=\textwidth/18cm,samples=200]
          %%% The nodes represents the k query in the first round
          % \draw[very thick] (-1,6)  -- (13,6) -- (13,3) -- (-1,3) -- (-1,6);
          % \draw[black] (-2.5, 4) circle (0pt) node [anchor=south]{\textbf{line 4:}};
          \draw[thick] (1, 4.1) circle (30pt) node
          % node[label={above: \small{iteration 1:}}] 
          {\tiny{$q_1^{(5,1)}$}} ;
          \draw[thick] (6, 4.1) circle (30pt) node
          {\tiny{ $q_1^{(6,2)}$}};
           \draw[thick] (11, 4.1) circle (30pt) node 
          {\tiny{$q_3^{(5,3)}$}};
          % \filldraw[black] (-2.5, 0) circle (0pt) node [anchor=south]{\textbf{line 7:}};
          \draw[thick] (6, 0) circle (30pt) node {\tiny{$q_4^7$}};
          \draw[ thick,->, blue] (6, 0.5)  -- (6, 3.2) ;
          \draw[very thick,->, red, dashed] (6, 0.5)  to [out=30,in=240] (11, 3.2) ;
          \draw[ thick,->, blue] (6, 0.5)  to [out=150,in=300]  (1, 3.2) ;
          \end{tikzpicture}
          \caption{}
              \end{centering}
              \end{subfigure}
              \begin{subfigure}{0.5\textwidth}
              \begin{centering}
              \begin{tikzpicture}[scale=\textwidth/16cm,samples=200]
          %%% The nodes represents the k query in the first round
          % \draw[very thick] (-1,6)  -- (13,6) -- (13,3) -- (-1,3) -- (-1,6);
          % \draw[black] (-2.5, 4) circle (0pt) node [anchor=south]{\textbf{line 4:}};
          % \draw[thick] (1, 1.1) circle (25pt) node
          % % node[label={above: \small{iteration 1:}}] 
          % {\tiny{$q_1^{(5,1)}$}} ;
          \draw[] (2, 5.1) circle (15pt) node
          {\tiny{ $a_1$}};
          \draw[] (6, 8.1) circle (15pt) node
          {\tiny{ $a_3^{1}$}};
          \draw[very thick, red, dotted] (14, 8.1) circle (15pt) node
          {\tiny{ $x_{1}^{1}$}};
          \draw[very thick, red, dotted] (12, 7.1) circle (15pt) node
          {\tiny{ $x_{2}^{1}$}};
          \draw[] (10, 8.1) circle (15pt) node
          {\tiny{ $x_{3}^{1}$}};
          \draw[] (8, 7.1) circle (15pt) node
          {\tiny{ $a_{2}^{1}$}};
          \draw[] (6, 6.1) circle (15pt) node
          {\tiny{ $a_3^{2}$}};
          \draw[very thick, red, dotted] (14, 6.1) circle (15pt) node
          {\tiny{ $x_{1}^{2}$}};
          \draw[very thick, red, dotted] (12, 5.1) circle (15pt) node
          {\tiny{ $x_{2}^{2}$}};
          \draw[] (10, 6.1) circle (15pt) node
          {\tiny{ $x_{3}^{2}$}};
          \draw[] (8, 5.1) circle (15pt) node
          {\tiny{ $a_{2}^{2}$}};
          \draw[very thick, red, dotted] (14, 4.1) circle (15pt) node
          {\tiny{ $x_1^{3}$}};
          \draw[] (6, 4.1) circle (15pt) node
          {\tiny{ $a_3^{3}$}};
          \draw[] (10, 4.1) circle (15pt) node
          {\tiny{ $x_3^{3}$}};
          \draw[very thick, red, dotted] (12, 3.1) circle (15pt) node
          {\tiny{ $x_2^{3}$}};
          \draw[] (8, 3.1) circle (15pt) node
          {\tiny{ $a_{2}^{3}$}};
           \draw[] (6, 2.1) circle (15pt) node 
          {\tiny{$a_3$}};
          % \filldraw[black] (-2.5, 0) circle (0pt) node [anchor=south]{\textbf{line 7:}};
          \draw[very thick, red, dotted] (3, 2.2) circle (15pt) node {\tiny{$l_1$}};
           \draw[very thick,->, red, dashed] (3.5, 2)  -- (5.5, 2) ;
           \draw[very thick,->, red, dashed] (6.5, 2.1)  -- (7.5, 2.8) ;
           \draw[very thick,->, red, dashed] (8.5, 3.1)  -- (9.5, 3.8) ;
           \draw[thick,->, blue] (10.5, 4.1)  -- (13.5, 4.1) ;
            \draw[very thick,->, red, dashed] (10.5, 4.1)  -- (11.5, 3.3) ;
             \draw[thick,->, blue] (7.5, 3.5)  -- (6.5, 4.0) ;
             \draw[thick,->, blue] (6.5, 4.1)  -- (7.5, 4.8) ;
              \draw[thick,->, blue] (8.5, 5.1)  -- (9.5, 5.8) ;
               \draw[thick,->, blue] (10.5, 6.1)  -- (11.5, 5.3) ;
          \draw[thick,->, blue] (10.5, 6.1)  -- (13.5, 6.1) ;
          \draw[thick,->, blue] (7.5, 5.5)  -- (6.5, 6.0) ;
          \draw[thick,->, blue] (6.5, 6.1)  -- (7.5, 6.8) ;
          \draw[thick,->, blue] (8.5, 7.1)  -- (9.5, 7.8) ;
          \draw[thick,->, blue] (10.5, 8.1)  -- (11.5 , 7.3) ;
          \draw[thick,->, blue] (10.5, 8.1)  -- (13.5 , 8.1) ;
          % \draw[thick,->, blue] (8.5, 9.1)  -- (9.5 , 9.8) ;
          % \draw[thick,->, blue] (8, 9.6)  -- (8, 10.6) ;
          \draw[thick,->, blue] (7.5, 7.5)  -- (6.5, 8.0) ;
          \draw[thick,->, blue] (5.5, 8.0)  -- (2.6, 5.3) ;
          \draw[thick,->, blue] (5.5, 6.0)  -- (2.6, 5.1) ;
          \draw[thick,->, blue] (5.5, 4.0)  -- (2.6, 4.9) ;
          \draw[thick,->, blue] (5.5, 2.0)  -- (2.6, 4.7) ;
          % \draw[very thick,->, red] (6, 0.5)  to [out=30,in=240] (11, 3.2) ;
          % \draw[very thick,->, blue] (6, 0.5)  to [out=150,in=300]  (1, 3.2) ;
          \end{tikzpicture}
              \caption{}
              \end{centering}
              \end{subfigure}
              \vspace{-0.3cm}
              \caption{(a) The query-based dependency graph for odd example (b) The SSA variable-based weighted dependency graph for the same example, the node in red dashed circle is weighted.}
              \label{fig:odd_graphs}
              \vspace{-0.3cm}
          \end{figure}
\subsection{Implementation Results}  
    %
    \begin{center}
        \begin{tabular}{ c c c }
         programs & adaptivity rounds & $\THESYSTEM$ results \\ 
         $\kw{seq()}$ & $4$ & $4$ \\ 
         $\kw{seqMultiVar()}$ & $4$ & $4$ \\  
         $ \kw{ifValueDependency}$ & $3$ & $3$  \\
         $\kw{ifControlDependency()}$ & $3$ & $3$  \\
         $ \kw{whileRec()}$ & $1+k$ & $1+k$  \\
         $ \kw{whileMultiplePath(k)}$ & $1 + k$ & $1 + 2 * k$  \\
         $ \kw{whileMultipleVar(k)}$ & $1 + 2*k$ & $1 + 2*k$  \\
         $ \kw{whileValueControlDependency()}$ & $1 + 2*k$ & $1 + 2*k$  \\
         $ \kw{whileMultiplePathValueControlDependency(k)}$ & $2 + k$ & $2 + k$  \\
         $ \kw{nestWhileValueDependency(k)}$ & $2 + k^2$ & $2 + k^2$  \\
         $ \kw{nestedWhileRecAcross(k)}$ & $1 + 2*k$ & $1 + 2*k$  \\
         $ \kw{nestedWhileMultiVarRecAcross(k)}$ & $1 + k + k^2$ & $1 + k + k^2$  \\
         $ \kw{nestedWhileMultiPathMultiVarRecAcross(k)}$ & $1 + k + k^2$ & $1 + k + k^2$  \\
        %  $ \kw{sorting(k)}$ & cell8 & cell9  \\
        %  $ \kw{gradientDescent(k)}$ & cell8 & cell9  \\
         $ \kw{linearRegressionGD(k)}$ & $k$ & $k$  \\
         $ \kw{twoRoundComplete(k)}$ & $2$ & $2$  \\
         $ \kw{multipleRoundComplete(k)}$ & $k$ & $k$  \\
        \end{tabular}
        \end{center}
        %
    \begin{example}[Complete Two Round Algorithm]
        \[
        %
            \kw{twoRounds(k)} \triangleq
        \begin{array}{l}
               \clabel{ a \leftarrow []}^{1} ; \\
                \clabel{\assign{j}{k} }^{2} ; \\
                \ewhile ~ \clabel{j > 0}^{3} ~ \edo ~ \\
                \Big(
                 \clabel{\assign{x}{\query(\chi[k - j]\cdot \chi[k])} }^{4}  ; \\
                 \clabel{\assign{j}{j-1}}^{5} ;\\
                \clabel{a \leftarrow x :: a}^{6}       \Big);\\
                \clabel{l \leftarrow (\mathrm{sign}\big (\sum_{i\in [k]} \chi[i]\times\ln\frac{1+a[i]}{1-a[i]} \big ))}^{7}\\
            \end{array}
        \]
        %
        \begin{algorithm}
        \footnotesize
        \caption{A two-round analyst strategy for random data (The example in  \cite{dwork2015generalization})}
        \label{alg:twoRound}
        \begin{algorithmic}
        \REQUIRE Mechanism $\mathcal{M}$ with a hidden data set $D \in \{-1,+1\}^{n\times (k+1)} \subset \dbdom$.
        \STATE  {\bf for}\ $j\in [k]$\ {\bf do}.  
        \STATE \qquad {\bf define} $q_j(d)=d(j)\cdot d(k)$ where $d \in \{D(i) ~|~ i = 0, \cdots, n\} \subseteq \{-1,+1\}^{k+1}$.
        \STATE \qquad {\bf let} $a_j=\mathcal{M}(q_j)$ 
        \STATE \qquad \COMMENT{In the line above, $\mathcal{M}$ computes approx. the exp. value  of $q_j$ over $D$. So, $a_j\in [-1,+1]$.}
        \STATE {\bf define} $q_{k}(d)= d(k) \cdot \mathrm{sign}\big (\sum_{i\in [k]} x(i) \cdot \ln\frac{1+a_i}{1-a_i} \big )$ where $x\in \{-1,+1\}^{k+1}$.
        \STATE\COMMENT{In the line above,  $\mathrm{sign}(y)=\left \{ \begin{array}{lr} +1 & \mathrm{if}\ y\geq 0\\ -1 &\mathrm{otherwise} \end{array} \right . $.}
        \STATE {\bf let} $a_{k+1}=\mathcal{M}(q_{k+1})$
        \STATE\COMMENT{In the line above,  $\mathcal{M}$ computes approx. the exp. value  of $q_{k+1}$ over $X$. So, $a_{k+1}\in [-1,+1]$.}
        \RETURN $a_{k+1}$.
        \ENSURE $a_{k+1}\in [-1,+1]$
            % \ENSURE 
        \end{algorithmic}
        \end{algorithm}
        %
    %
        \end{example}
    
    \begin{example}[Complete Multiple Round Algorithm]
    %
    \begin{algorithm}
    \footnotesize
    \caption{A multi-round analyst strategy for random data base \cite{dwork2015generalization}}
    \label{alg:multiRound}
    \begin{algorithmic}
    \REQUIRE Mechanism $\mathcal{M}$ with a hidden state $X\in [N]^{n}$ sampled u.a.r., control set size $c$
    \STATE Define control dataset $C = \{0,1, \cdots, c - 1\}$
    \STATE Initialize $Nscore(i) = 0$ for $i \in [N]$, $I = \emptyset$ and $Cscore(C(i)) = 0$ for $i \in [c]$
    \STATE  {\bf for}\ $j\in [k]$\ {\bf do} 
    \STATE \qquad {\bf let} $p=\uniform(0,1)$ 
    \STATE \qquad {\bf define} $q (x) = \bernoulli ( p )$ .
    \STATE \qquad {\bf define} $qc (x) = \bernoulli ( p )$ .
    \STATE \qquad {\bf let} $a = \mathcal{M}(q)$ 
    \STATE \qquad {\bf for}\ $i \in [N]$\ {\bf do}
    \STATE \qquad \qquad $Nscore(i) = Nscore(i) + (a - p)*(q (i) - p)$ if $i \notin I$
    \STATE \qquad {\bf for}\ $i \in [c]$\ {\bf do}
    \STATE \qquad \qquad $Cscore(C(i)) = Cscore(C(i)) + (a - p)*(qc (i) - p)$
    \STATE \qquad {\bf let} $I = \{i | i\in [N] \land Nscore(i) > \max(Cscore)\}$
    \STATE \qquad {\bf let} $D = D \setminus I$ 
    \RETURN $D$.
    \end{algorithmic}
    \end{algorithm}
    %
    {\small
    \begin{figure}
        \begin{subfigure}{0.3\textwidth}
        \begin{centering}
        $
    %     \begin{array}{l}
    %     %  \left[j \leftarrow 0 \right]^1 ; \\
    %     \clabel{I \leftarrow [] }^1; \\
    %     \clabel{\assign{ns}{0} }^{2}; \\
    %      \clabel{\assign{cs}{0} }^{3}; \\
    %     \eloop ~ [3]^{4} ~  
    %     \ ~ \edo ~ \\ 
    %     \quad \clabel{a \leftarrow q(f( I)) }^{5}; \\
    %     \quad \clabel{\assign{ns}{update\_nscore(a)}; }^{6}\\
    %     \quad \clabel{\assign{cs}{update\_cscore(a)}; }^{7}\\
    %     \quad \clabel{I \leftarrow \mathsf{update} ( I, ns, cs)  }^{8}
    % \end{array}
    \kw{multipleRoundsSimp(k, c)} \triangleq
    \begin{array}{l}
         \left[j \leftarrow k \right]^1 ; \\
        \left[I \leftarrow [] \right]^2; \\
        \ewhile ~ \clabel{j > 0}^{3} ~ \edo ~ \\
        \Big(
        \clabel{\assign{j}{j-1}}^{4} ;\\
        \left[p \leftarrow c \right]^5; \\
        \left[ a \leftarrow \query (\chi[I]) \right]^6;\\
        \left[I \leftarrow \mathrel{\mathsf{update}} ( {I}, (a, p))  \right]^7
        \Big) 
    \end{array}
        $
        \caption{}
        \end{centering}
        \end{subfigure}
        \begin{subfigure}{0.6\textwidth}
        \begin{centering}
        $
    \kw{multipleRounds(k, c, N)} \triangleq
    \begin{array}{l}
        \clabel{\assign{j}{N}}^0 ; 
         \clabel{\assign{cs}{0}}^1; 
         \clabel{\assign{ns}{0}}^2;
         \clabel{\assign{I}{0}}^3; 
         \clabel{\assign{w}{k}}^{4} ;\\
         \ewhile ~ \clabel{j > 0}^{5} ~ \edo ~ \\
         \Big(
         \clabel{\assign{j}{j-1}}^{6} ;
         \clabel{\assign{cs}{0 + cs}}^7; 
         \clabel{\assign{ns}{0 + ns}}^8
         \Big); \\
    
         \ewhile ~ \clabel{w > 0}^{9} ~ \edo ~ \\
        \Big(
        \clabel{\assign{w}{w-1}}^{10} ;
        \left[p \leftarrow c \right]^{11}; 
        \left[q \leftarrow c \right]^{12}; 
        \left[ a \leftarrow \query (\chi[I]) \right]^{13};\\
        \clabel{\assign{i}{N}}^{14} ; 
        \ewhile ~ \clabel{i > 0}^{15} ~ \edo ~ \\
        \Big(
        \clabel{\assign{i}{i-1}}^{16} ;
        \clabel{\assign{cs(i)}{cs(i) + (a - p) * (q - p)}}^{17}; \\
        \eif (\clabel{ I < i}^{18}, \clabel{\assign{ns(i)}{{ns(i) + (a - p) * (q - p)}}}^{19},
        \clabel{\assign{ns}{ns(i)}}^{20}    )
        \Big); \\
        \clabel{\assign{i2}{N}}^{21} ; \\
        \ewhile ~ \clabel{i2 > 0}^{22} ~ \edo ~ \\
        \Big(
        \clabel{\assign{i2}{i2-1}}^{23} ;
        \eif (\clabel{ns(i2) > \kw{max}(cs)}^{24}, 
        \clabel{\assign{I}{i + I}}^{25},
        \clabel{\assign{I}{I}}^{26})
        \Big)
        \Big) 
    \end{array}
       $
       \caption{}
        \end{centering}
        \end{subfigure}
        \vspace{-0.3cm}
        \caption{(a) The labeled program implementing the multiple round algorithm (b)The same program in the SSA version}
        \vspace{-0.5cm}
        \label{fig:multiround_complete}
        \end{figure}
    }
    %
    \end{example}
      We have seen the two round algorithm above. We show the multiple-round algorithm, which is an advanced algorithm.
    
    
    \begin{example}[Gradient Decent Optimization Algorithm]
        This example is the gradient decent algorithm example is a generalization of the linear regression on a higher degree data relation.
        It uses gradient decent algorithm to minimize 
        the mean square loss function
        for a two-degree relation
         $y = a_1 \times x_1^2 + a_2 \times x_2 + c$
        on the dataset of two feature columns and one indicator column.
                  \[
                  %
                  \begin{array}{l}
                  \kw{gradientDecent(step, rate, t, n)} \triangleq \\
                         \clabel{ a_1 \leftarrow 0}^{0} ; \\
                         \clabel{ a_2 \leftarrow 0}^{1} ; \\
                         \clabel{ c \leftarrow 0}^{2} ; \\
                          \clabel{\assign{j}{\kw{step}} }^{3} ; \\
                        %   \clabel{\assign{d}{10000000} }^{2} ; \\
                          \ewhile ~ \clabel{j > 0}^{4} ~ \edo ~ \\
                          \Big(
                              \clabel{\assign{da1}{\query(-2 * (\chi[2] - (\chi[0]^2 \times a_1 + \chi[1] \times a_2 + c)) \times (\chi[0]))} }^{5}  ; \\
                              \clabel{\assign{da2}{\query(-2 * (\chi[2] - (\chi[0]^2 \times a_1 + \chi[1] \times a_2 + c)) \times (\chi[1]))} }^{6}  ; \\                      \clabel{\assign{dc}{\query(-2 * (\chi[2] - (\chi[0]^2 \times a_1 + \chi[1] \times a_2 + c)))} }^{5}  ; \\
                              \clabel{\assign{a_1}{a_1 - \kw{rate} * da1} }^{7}  ; \\
                              \clabel{\assign{a_2}{a_2 - \kw{rate} * da2} }^{8}  ; \\
                              \clabel{\assign{c}{c - \kw{rate} * dc} }^{9}  ; \\
                           \clabel{\assign{j}{j-1}}^{10} 
                        %   \clabel{a \leftarrow x :: a}^{6} 
                          \Big);
                      \end{array}
                  \]
                  %
                  %
        It is easy to see, this approach can be generalized to the regression of a variety of 
        relations in machine learning area.
                       %
                  \end{example}
         
           
                  
                  
                                  \begin{example}[convex optimization Algorithm]
                  \[
                  %
                  \begin{array}{l}
                  \kw{gradientDecent(step, rate, t, n)} \triangleq \\
                         \clabel{ a \leftarrow []}^{0} ; \\
                          \clabel{\assign{j}{\kw{step}} }^{1} ; \\
                        %   \clabel{\assign{d}{10000000} }^{2} ; \\
                          \ewhile ~ \clabel{j > 0 \land d < t}^{3} ~ \edo ~ \\
                          \Big(
                              \clabel{\assign{d}{\query(2 * (\chi[1] - (\chi[0]\times x )) * (-\chi[0]))} }^{4}  ; \\
                              \clabel{\assign{x}{x - \kw{rate} * d} }^{4}  ; \\
                           \clabel{\assign{j}{j-1}}^{5} ;\\
                          \clabel{a \leftarrow x :: a}^{6} 
                          \Big);
                      \end{array}
                  \]
                  %
                  %
                       %
                  \end{example}    
    
    \begin{example}[Sequence with Single Variable Linear Data Value Dependency]
        %
        %
        \[
        %
            \kw{seq()} \triangleq 
        \begin{array}{l} 
               \clabel{ \assign{x}{\chi[0]}}^{0} ; \\
                \clabel{\assign{y}{\chi[x + 1]} }^{1} ; \\
                \clabel{\assign{z}{\chi[y + 1]}}^{2}; \\
                 \clabel{\assign{w}{\chi[z + 1]} }^{3}
            \end{array}
        \]
        Analysis Result: $ \progA( \kw{seq()}) = 4$
        \end{example}
    %
    \begin{example}[Sequence with Multiple Variables Data Value Dependency]
        %
        %
        \[
        %
            \kw{seqMultiVar()} \triangleq 
        \begin{array}{l} 
               \clabel{ \assign{x}{\chi[0]}}^{0} ; \\
                \clabel{\assign{y}{\chi[x + 1]} }^{1} ; \\
                \clabel{\assign{z}{\chi[y + x]}}^{2}; \\
                 \clabel{\assign{w}{\chi[z + 1] \cdot \chi[y]} }^{3}
            \end{array}
        \]
        Analysis Result: $ \progA(\kw{seqMultiVar()}) = 4$
    \end{example}
    %
        \begin{example}[If with Data-Value Dependency Separated]
            %
            %
            \[
            %
            \kw{ifValueDependency}(k) \triangleq 
            \begin{array}{l}
               \quad \clabel{ \assign{z}{\query(\chi[0])}}^{0} ; \\
               \quad \clabel{\assign{x}{k / 2} }^{1} ; \\
               \quad \eif(\clabel{x < 0}^2, \\
               \quad \clabel{\assign{y}{\query(\chi[z])}}^{3}, \\ 
               \quad \clabel{\assign{y}{\query(\chi[0])}}^{4})
                \end{array}
            \]
            Analysis Result: $ \progA( \kw{ifControlDependency()}) = 3$
        \end{example}
    
            \begin{example}[If with Data-Control Dependency Overlapped]
                %
                %
                \[
                %
                \kw{ifControlDependency()} \triangleq 
                \begin{array}{l}
                    \clabel{ \assign{z}{\query(\chi[0])}}^{0} ; \\
                    \clabel{\assign{x}{\query(\chi[z])} }^{1} ; \\
                    \eif(\clabel{x < 0}^{2}, 
                    \clabel{\assign{y}{\query(\chi[0] + \chi[1])}}^{3}, 
                    \clabel{\assign{y}\query{(\chi[0])}}^{4})
                \end{array}
                \]
                %
                Analysis Result: $ \progA( \kw{ifControlDependency()}) = 3$
                \end{example}
    
    
                \begin{example}[Simple While with Recursive Data-Value Dependency]
                    %
                    %
                    \[
                    %
                    \kw{whileRec}() \triangleq
                    \begin{array}{l}
                        \clabel{ \assign{j}{10000} }^{0} ; \\
                        \clabel{ \assign{a}{\query(\chi[0])} }^{1} ; \\
                            \ewhile ~ \clabel{j > 0}^{2} ~ \edo ~ \\
                            \Big(
                             \clabel{\assign{x}{\query(\chi[a]) }}^{3}  ; \\
                             \clabel{\assign{a}{x + a}}^{4} ;\\
                            \clabel{\assign{j}{j-1}}^{5}       \Big)
                        \end{array}
                    \]
                    Analysis Results: $ \progA(\kw{whileRec}(k)) = 1 + k$
                \end{example}
    %
            \begin{example}[Simple While with Multi-Path Data-Value Dependency]
            %
            %
            \[
            %
            \kw{whileMultiplePath}(k) \triangleq 
            \begin{array}{l}
                \clabel{ \assign{j}{k}}^{0} ; \\
                \clabel{ \assign{x}{\query(\chi[0])} }^{1} ; \\
                    \ewhile ~ \clabel{j > 0}^{2} ~ \edo ~ \\
                    \Big(
                     \clabel{\assign{j}{j-1}}^{3} ;\\
                     \eif(\clabel{j \% 2 == 0}^{4}, 
                     \clabel{\assign{y}{\chi[x]}}^{5}, 
                     \clabel{\assign{w}{\chi[x]}}^{6});\\                            
                     \clabel{\assign{x}{\query(\chi(\ln(y)))} }^{7} \Big)
                \end{array}
            \]
            Analysis Results: $ \progA(\kw{whileMultiplePath}(k)) = 1 + 2 * k $ --> Over-Approximated
        \end{example}
    %
            \begin{example}[Simple While with Recursive Multiple-Variable Data-Value Dependency]
                \[
                %
                \kw{whileMultipleVar}(k) \triangleq 
                \begin{array}{l}
                \clabel{\assign{j}{k} }^{0} ; \\
                \clabel{ \assign{x}{\query(\chi[0])}}^{1} ; \\
                    \clabel{ \assign{y}{\query(\chi[1])}}^{2} ; \\
                        \ewhile ~ \clabel{j > 0}^{3} ~ \edo ~ \\
                        \Big(
                         \clabel{\assign{j}{j-1}}^{4} ;\\
                         \clabel{\assign{z}{\query(\chi(x + \ln(y)))} }^{5}  ; \\
                         \clabel{ \assign{x}{\query(\chi[z])}}^{6} ; \\
                         \clabel{ \assign{y}{\query(\chi[z])}}^{7} 
                        \Big)
                    \end{array}
                \]
                Analysis Results: $ \progA(\kw{whileMultipleVar}(k)) = 1 + 2 * k $
            \end{example}
                %
                %
                \begin{example}[Simple While with Data-Value and Data-Control Dependency]
                    %
                    \[
                    \kw{whileValueControlDependency}() \triangleq
                    \begin{array}{l}
                        \clabel{ \assign{x}{\query(\chi[0])} }^{0} ; \\
                        \clabel{ \assign{z}{\query(\chi[0])} }^{1} ; \\
                            \ewhile ~ \clabel{x > 0}^{2} ~ \edo ~ \\
                            \Big(
                            \clabel{\assign{x}{\query(\chi(z))} }^{3}  ; \\
                            \clabel{\assign{z}{\query(\chi(x))}}^{4}
                          \Big)
                        \end{array}
                    \]
                    Analysis Results: $ \progA(\kw{whileValueControlDependency}(k)) = 1 + 2 * k $
                \end{example}
    %
                \begin{example}[Simple While with MultiplePath Data-Value and Data-Control Dependency]
                    %
                    \[
                        %
                        \begin{array}{l}
                        \kw{whileMultiplePathValueControlDependency}(k) \triangleq\\
                            \clabel{ \assign{x}{\query(k)}}^{0} ; \\
                            \clabel{\assign{y}{0} }^{1} ; \\
                                \ewhile ~ \clabel{x > 0}^{2} ~ \edo ~ \\
                                \Big(
                                 \eif(\clabel{y > 0}^{3}, 
                                 \clabel{\assign{y}{\query(\chi[12])}}^{4}, 
                                 \clabel{\assign{w}{\query(\chi[9])}}^{5});                            
                                 \\
                                 \clabel{\assign{x}{x-1}}^{6}\Big);\\
                                 \clabel{\assign{y}{\query(\chi(\ln(y)))} }^{7} 
                            \end{array}
                        \]
                        Analysis Results: $ \progA(\kw{whileMultiplePathValueControlDependency}(k)) = 2 + k $
                    \end{example}
                                    %
                    \begin{example}[Nested While with Recursive Data-Value Dependency]
                        %
                        %
                        \[
                        %
                        \kw{nestWhileValueDependency}(k) \triangleq 
                        \begin{array}{l}
                            \clabel{ \assign{i}{k} }^{0} ; \\
                            \clabel{\assign{x}{\query(\chi[0])}}^{1} ; \\
                                \ewhile ~ \clabel{i > 0}^{2} ~ \edo ~ \\
                                \Big(
                                 \clabel{\assign{i}{i-1}}^{3} ;\\
                                 \clabel{\assign{j}{k}}^{4} ;\\
                                 \clabel{\assign{y}{\query(\chi(\ln(x)))} }^{5}  ; \\
                                 \ewhile ~ \clabel{j > 0}^{6} ~ \edo ~ \\
                                 \Big(
                                  \clabel{\assign{j}{j-1}}^{7};\\
                                  \clabel{\assign{x}{\query(\chi(\ln(x)))} }^{8}
                                  \Big) \Big)
                            \end{array}
                        \]
                        Analysis Results: $ \progA(\kw{nestWhileValueDependency}(k)) = 2 + k^2 $
                    \end{example}
    
                        \begin{example}[Nested While with Nested Recursive Data-Value Dependency Across Outer and Inner Loop]
                            %
                            %
                            \[
                            %
                                \kw{nestedWhileRecAcross}(k) \triangleq 
                            \begin{array}{l}
                                \clabel{ \assign{i}{k} }^{0} ; \\
                                \clabel{\assign{x}{\query(\chi[0])}}^{1} ; \\
                                    \ewhile ~ \clabel{i > 0}^{2} ~ \edo ~ \\
                                    \Big(
                                     \clabel{\assign{i}{i-1}}^{3} ;\\
                                     \clabel{\assign{j}{k}}^{4} ;\\
                                     \ewhile ~ \clabel{j > 0}^{5} ~ \edo ~ \\
                                     \Big(
                                      \clabel{\assign{j}{j-1}}^{6};\\
                                      \clabel{\assign{y}{\query(\chi(x) + \chi(1))} }^{7}
                                      \Big); \\
                                     \clabel{\assign{x}{\query(\chi(\ln(y)))} }^{8}
                                      \Big)
                                \end{array}
                            \]
                            Analysis Results: $ \progA(\kw{nestedWhileRecAcross}(k)) = 1 + 2 * k $
                        \end{example}
                    %
                
                            \begin{example}[Nested While with Nested Recursive Multiple Variable 
                                Data-Value Dependency Across Outer and Inner Loop]
                                %
                                \[
                                %
                                \kw{nestedWhileMultiVarRecAcross}(k) \triangleq 
                                \begin{array}{l}
                                    \clabel{\assign{i}{k} }^{0} ; \\
                                    \clabel{ \assign{x}{\query(\chi[0])}}^{1} ; \\
                                    \clabel{ \assign{y}{\query(\chi[1])}}^{2} ; \\
                                        \ewhile ~ \clabel{i > 0}^{3} ~ \edo ~ \\
                                        \Big(
                                         \clabel{\assign{i}{i-1}}^{4} ;\\
                                         \clabel{\assign{j}{k}}^{5} ;\\
                                         \clabel{\assign{y}{\query(\chi(\ln(x) + y))} }^{6}  ; \\
                                         \ewhile ~ \clabel{j > 0}^{7} ~ \edo ~ \\
                                         \Big(
                                          \clabel{\assign{j}{j-1}}^{8};\\
                                          \clabel{\assign{x}{\query(\chi(\ln(y))+\chi[x])} }^{9}
                                          \Big) \Big)
                                    \end{array}
                                \]
                                Analysis Results: $ \progA(\kw{nestedWhileMultiVarRecAcross}(k)) = 1 + k + k^2$
                                \\
                                weight for Variable: j of label 6 is: 0 + 0 + 1 * k * k\\
                                weight for Variable: y of label 7 is: 0 + 0 + 1 * k * k\\
                                weight for Variable: j of label 4 is: 0 + 1 * k\\
                                weight for Variable: i of label 3 is: 0 + 1 * k\\
                                weight for Variable: x of label 8 is: 0 + 1 * k\\
                                weight for Variable: x of label 1 is: 1\\
                                weight for Variable: i of label 0 is: 1\\
                                \end{example}
                        

                                \begin{example}[Nested While with MultiplePath and Nested Recursive Multiple Variable 
                                    Data-Value Dependency Across Outer and Inner Loop]
                                    %
                                    We then show a more complex example with nested while command and nested data-flow across the outer and inner while loop through multiple variables.
                                    This example also contains the if command with data dependency occurred through the if guard.
                                    %
                                    \[
                                    %
                                    \begin{array}{l}
                                    \kw{nestedWhileMultiPathMultiVarRecAcross}(k) \triangleq \\
                                        \clabel{\assign{i}{k} }^{0} ; \\
                                        \clabel{ \assign{x}{\query(\chi[0])}}^{1} ; \\
                                        \clabel{ \assign{y}{\query(\chi[1])}}^{2} ; \\
                                            \ewhile ~ \clabel{i > 0}^{3} ~ \edo ~ \\
                                            \Big(
                                             \clabel{\assign{i}{i-1}}^{4} ;\\
                                             \clabel{\assign{j}{k}}^{5} ;\\
                                             \eif(\clabel{x > 0}^6, \clabel{\assign{y}{\query(\chi(\ln(x) + y))} }^{7},
                                             \clabel{\assign{y}{\query(\chi(x))} }^{8} )
                                              ; \\
                                             \ewhile ~ \clabel{j > 0}^{9} ~ \edo ~ \\
                                             \Big(
                                              \clabel{\assign{j}{j-1}}^{10};\\
                                              \clabel{\assign{x}{\query(\chi(\ln(y))+\chi[x])} }^{11}
                                              \Big) \Big)
                                        \end{array}
                                    \]
                                    \end{example}
                                    Analysis Results: $ \progA(\kw{nestedWhileMultiPathMultiVarRecAcross}(k)) = 1 + k + k^2$
                                    \\
                                    weight for Variable: j of label 10 is: 0 + 0 + 1 * k * k \\
                                    weight for Variable: x of label 11 is: 0 + 0 + 1 * k * k \\
                                    weight for Variable: y of label 7 is: 0 + 1 * k \\
                                    weight for Variable: y of label 8 is: 0 + 1 * k \\
                                    weight for Variable: j of label 5 is: 0 + 1 * k \\
                                    weight for Variable: i of label 4 is: 0 + 1 * k \\
                                    weight for Variable: y of label 2 is: 1 \\
                                    weight for Variable: x of label 1 is: 1 \\
                                    weight for Variable: i of label 0 is: 1 \\


               \begin{example}[Two Round Odd Elements ]
          We present a variant of the previous two round example in Figure~\ref{fig:tworound_odd}. In this odd example, only the data at odd index of the database is used.
          %
          {\small
          \begin{figure}
          \begin{subfigure}{.4\textwidth}
          \begin{centering}
          $\begin{array}{l}
              % \left[j \leftarrow 0 \right]^1 ; \\
             \clabel{ \assign{a}{0}  }^{1} ; \\
              \clabel{\assign{j}{0} }^{2} ; \\
              \eloop ~ \clabel{3}^{3} ~ \edo ~ \\
             \quad 
               \eif ( \clabel{( j \% 2 == 0)}^{4}\\
               \quad \quad ,\clabel{ \assign{x}{ q(\chi[j+1])} }^{5}  \\
               \quad \quad ,\clabel{\assign{x}{ q(\chi[j]) }}^{6} ) ;\\
               \quad \clabel{\assign{a}{ a+x} }^{7}  ;\\
               \quad \clabel{ \assign{j}{j+1} }^{8}      ;\\
           \clabel{l \leftarrow q(a*\chi[3]) }^{9}\\
          \end{array} $
          \caption{}
          \end{centering}
          \end{subfigure}
          \begin{subfigure}{0.5\textwidth}
          \begin{centering}
          $
          \begin{array}{l}
              \clabel{ \assign{a_1}{0}  }^{1} ; \\
              \clabel{\assign{j_1}{0} }^{2} ; \\
              \eloop ~ \clabel{3}^{3} , 0,  ~ \edo ~ [(j_3, j_1,j_2), (a_3, a_1,a_2)], \\
            \quad 
               \eif ( \clabel{( j_3 \% 2 == 0)}^{4}, [x_3,x_1,x_2], [],[]\\
               \quad \quad ,\clabel{ \assign{x_1}{ q(\chi[j_3+1])} }^{5}  \\
               \quad \quad ,\clabel{\assign{x_2}{ q(\chi[j_3]) }}^{6} ) ;\\
               \quad \clabel{\assign{a_2}{ a_3+x_3} }^{7}  ;\\
               \quad \clabel{ \assign{j_2}{j_3+1} }^{8}      ;\\
           \clabel{l_1 \leftarrow q(a_3*\chi[3]) }^{9}\\
          \end{array}$
          \caption{}
          \end{centering}
          \end{subfigure}
              \vspace{-0.2cm}
              \caption{(a) The two round odd algorithm in labeled {\tt Loop} language, (b) The SSA program for the same example. }
              \label{fig:tworound_odd}
              \vspace{-0.5cm}
          \end{figure}
          }
          \end{example}
          %
          This algorithm only touches the odd part of the database, by adding an extra if statement to checking the index $j$ in Figure~\ref{fig:tworound_odd}(a). The extra complexity is added to handle the newly generated variables in the loop and if statement in the SSA version in Figure~\ref{fig:tworound_odd}(b). 
          The query-based dependency graph does not change a lot compared to the previous two rounds example in Figure~\ref{fig:simpl-two-round-graph}(c), but the node does change according to the trace. We assume $a = n$ in the final memory is the result of the sum of previous query results in the loop.
          We give the trace $t = [q(\chi[1])^{5,[3:1]}, q(\chi[1])^{6,[3:2]}, q(\chi[3])^{5,[3:3]}, q(n* \chi[3])^{9,\emptyset} ]$ and use $q_1$ for $q(\chi[1])$, $q_3$ representing for $q(\chi[3])$, $q_4$ for $q(n* \chi[3])$. The query-based dependency graph based on this trace is shown in Figure~\ref{fig:odd_graphs}(a). We show the red path, which is a sequence of adaptively chosen queries of length $2$. So among the total $4$ queries, we have 2-round adaptive queries. According to the Theorem\ref{thm:gaussiannoise} and \ref{thm:gaussiannoise2}, we will have a tighter upper bound on the generalization error if we know the adaptivity $2$, obtained from the red path in Figure~\ref{fig:odd_graphs}(a). 
          
          Our algorithm {\THESYSTEM} gives us the upper bound on the aforementioned adaptivity $2$. We construct the variable-based weighted dependency graph in Figure~\ref{fig:odd_graphs}(b). The weighted nodes are in the red dashed circle and the red dashed paths show the most weighted path in the graph, with the weight $2$. So, for this two rounds odd algorithm, our system gives a tight upper bound $2$, which can be used to get a better generalization error bound.
          
          
          
          
          \begin{figure}
              \begin{subfigure}{0.4\textwidth}
              \begin{centering}
              \begin{tikzpicture}[scale=\textwidth/18cm,samples=200]
          %%% The nodes represents the k query in the first round
          % \draw[very thick] (-1,6)  -- (13,6) -- (13,3) -- (-1,3) -- (-1,6);
          % \draw[black] (-2.5, 4) circle (0pt) node [anchor=south]{\textbf{line 4:}};
          \draw[thick] (1, 4.1) circle (30pt) node
          % node[label={above: \small{iteration 1:}}] 
          {\tiny{$q_1^{(5,1)}$}} ;
          \draw[thick] (6, 4.1) circle (30pt) node
          {\tiny{ $q_1^{(6,2)}$}};
           \draw[thick] (11, 4.1) circle (30pt) node 
          {\tiny{$q_3^{(5,3)}$}};
          % \filldraw[black] (-2.5, 0) circle (0pt) node [anchor=south]{\textbf{line 7:}};
          \draw[thick] (6, 0) circle (30pt) node {\tiny{$q_4^7$}};
          \draw[ thick,->, blue] (6, 0.5)  -- (6, 3.2) ;
          \draw[very thick,->, red, dashed] (6, 0.5)  to [out=30,in=240] (11, 3.2) ;
          \draw[ thick,->, blue] (6, 0.5)  to [out=150,in=300]  (1, 3.2) ;
          \end{tikzpicture}
          \caption{}
              \end{centering}
              \end{subfigure}
              \begin{subfigure}{0.5\textwidth}
              \begin{centering}
              \begin{tikzpicture}[scale=\textwidth/16cm,samples=200]
          %%% The nodes represents the k query in the first round
          % \draw[very thick] (-1,6)  -- (13,6) -- (13,3) -- (-1,3) -- (-1,6);
          % \draw[black] (-2.5, 4) circle (0pt) node [anchor=south]{\textbf{line 4:}};
          % \draw[thick] (1, 1.1) circle (25pt) node
          % % node[label={above: \small{iteration 1:}}] 
          % {\tiny{$q_1^{(5,1)}$}} ;
          \draw[] (2, 5.1) circle (15pt) node
          {\tiny{ $a_1$}};
          \draw[] (6, 8.1) circle (15pt) node
          {\tiny{ $a_3^{1}$}};
          \draw[very thick, red, dotted] (14, 8.1) circle (15pt) node
          {\tiny{ $x_{1}^{1}$}};
          \draw[very thick, red, dotted] (12, 7.1) circle (15pt) node
          {\tiny{ $x_{2}^{1}$}};
          \draw[] (10, 8.1) circle (15pt) node
          {\tiny{ $x_{3}^{1}$}};
          \draw[] (8, 7.1) circle (15pt) node
          {\tiny{ $a_{2}^{1}$}};
          \draw[] (6, 6.1) circle (15pt) node
          {\tiny{ $a_3^{2}$}};
          \draw[very thick, red, dotted] (14, 6.1) circle (15pt) node
          {\tiny{ $x_{1}^{2}$}};
          \draw[very thick, red, dotted] (12, 5.1) circle (15pt) node
          {\tiny{ $x_{2}^{2}$}};
          \draw[] (10, 6.1) circle (15pt) node
          {\tiny{ $x_{3}^{2}$}};
          \draw[] (8, 5.1) circle (15pt) node
          {\tiny{ $a_{2}^{2}$}};
          \draw[very thick, red, dotted] (14, 4.1) circle (15pt) node
          {\tiny{ $x_1^{3}$}};
          \draw[] (6, 4.1) circle (15pt) node
          {\tiny{ $a_3^{3}$}};
          \draw[] (10, 4.1) circle (15pt) node
          {\tiny{ $x_3^{3}$}};
          \draw[very thick, red, dotted] (12, 3.1) circle (15pt) node
          {\tiny{ $x_2^{3}$}};
          \draw[] (8, 3.1) circle (15pt) node
          {\tiny{ $a_{2}^{3}$}};
           \draw[] (6, 2.1) circle (15pt) node 
          {\tiny{$a_3$}};
          % \filldraw[black] (-2.5, 0) circle (0pt) node [anchor=south]{\textbf{line 7:}};
          \draw[very thick, red, dotted] (3, 2.2) circle (15pt) node {\tiny{$l_1$}};
           \draw[very thick,->, red, dashed] (3.5, 2)  -- (5.5, 2) ;
           \draw[very thick,->, red, dashed] (6.5, 2.1)  -- (7.5, 2.8) ;
           \draw[very thick,->, red, dashed] (8.5, 3.1)  -- (9.5, 3.8) ;
           \draw[thick,->, blue] (10.5, 4.1)  -- (13.5, 4.1) ;
            \draw[very thick,->, red, dashed] (10.5, 4.1)  -- (11.5, 3.3) ;
             \draw[thick,->, blue] (7.5, 3.5)  -- (6.5, 4.0) ;
             \draw[thick,->, blue] (6.5, 4.1)  -- (7.5, 4.8) ;
              \draw[thick,->, blue] (8.5, 5.1)  -- (9.5, 5.8) ;
               \draw[thick,->, blue] (10.5, 6.1)  -- (11.5, 5.3) ;
          \draw[thick,->, blue] (10.5, 6.1)  -- (13.5, 6.1) ;
          \draw[thick,->, blue] (7.5, 5.5)  -- (6.5, 6.0) ;
          \draw[thick,->, blue] (6.5, 6.1)  -- (7.5, 6.8) ;
          \draw[thick,->, blue] (8.5, 7.1)  -- (9.5, 7.8) ;
          \draw[thick,->, blue] (10.5, 8.1)  -- (11.5 , 7.3) ;
          \draw[thick,->, blue] (10.5, 8.1)  -- (13.5 , 8.1) ;
          % \draw[thick,->, blue] (8.5, 9.1)  -- (9.5 , 9.8) ;
          % \draw[thick,->, blue] (8, 9.6)  -- (8, 10.6) ;
          \draw[thick,->, blue] (7.5, 7.5)  -- (6.5, 8.0) ;
          \draw[thick,->, blue] (5.5, 8.0)  -- (2.6, 5.3) ;
          \draw[thick,->, blue] (5.5, 6.0)  -- (2.6, 5.1) ;
          \draw[thick,->, blue] (5.5, 4.0)  -- (2.6, 4.9) ;
          \draw[thick,->, blue] (5.5, 2.0)  -- (2.6, 4.7) ;
          % \draw[very thick,->, red] (6, 0.5)  to [out=30,in=240] (11, 3.2) ;
          % \draw[very thick,->, blue] (6, 0.5)  to [out=150,in=300]  (1, 3.2) ;
          \end{tikzpicture}
              \caption{}
              \end{centering}
              \end{subfigure}
              \vspace{-0.3cm}
              \caption{(a) The query-based dependency graph for odd example (b) The SSA variable-based weighted dependency graph for the same example, the node in red dashed circle is weighted.}
              \label{fig:odd_graphs}
              \vspace{-0.3cm}
          \end{figure}
%
\clearpage
\appendix
\addcontentsline{toc}{section}{Appendices}
\section*{Appendices}
\section{Proofs of Lemmas in Section 1, 2 and 3 }
\label{apdx:lemma_sec123}
\begin{lem}[Uniqueness of the Labeled Variables]
    % \label{lem:lvar_unique}
    For every program $c \in \cdom$ and every two labeled variables such that
    $x^i, y^j \in \lvar(c)$, then $x^i \neq y^j$.
    \[
      \forall c \in \cdom, x^i, y^j \in \mathcal{L} \st x^i, y^j \in \lvar(c)\implies x^i \neq y^j.
      \]
  \end{lem}
  \begin{proof}
  \end{proof}
  \begin{lem}
    [Trace Non-Decreasing]
    % \label{lem:tracenondec}
    For every program $c \in \cdom$ and traces $\trace, \trace' \in \mathcal{T}$, if 
    $\config{c, \trace} \rightarrow^{*} \config{\eskip, \trace'}$,
    then there exists a trace $\trace'' \in \mathcal{T}$ with $\trace \tracecat \trace'' = \trace'$
    %
    $$
    \forall \trace, \trace' \in \mathcal{T}, c \st
    \config{c, \trace} \rightarrow^{*} \config{\eskip, \trace'} 
    \implies \exists \trace'' \in \mathcal{T} \st \trace \tracecat \trace'' = \trace'
    $$
    \end{lem}
    \begin{proof}
      Taking arbitrary trace $\trace \in \mathcal{T}$, by induction on program $c$, we have the following cases:
      \caseL{$c = [\assign{x}{\expr}]^{l}$}
      By the evaluation rule $\rname{assn}$, we have
      $
      {
      \config{[\assign{{x}}{\aexpr}]^{l},  \trace } 
      \xrightarrow{} 
      \config{\eskip, \trace :: ({x}, l, v, \bullet)}
      }$, for some $v \in \mathbb{N}$.
      \\
      Picking $\trace' = \trace ::({x}, l, v, \bullet)$ and $\trace'' =  [({x}, l, v, \bullet) ]$,
      it is obvious that $\trace \tracecat \trace'' = \trace'$.
      % \\
      % There are 2 cases, where $l' = l$ and $l' \neq l$.
      % \\
      % In case of $l' \neq l$, we know $\event \not\eventin \trace$, then this Lemma is vacuously true.
      %   \\
      %   In case of $l' = l$, by the abstract Execution Trace computation, we know 
      %   $\absflow(c) = \absflow'([x := \expr]^{l}; \clabel{\eskip}^{l_e}) = \{(l, \absexpr(\expr), l_e)\}$  
      %   \\
      % Then we have $\absevent = (l, \absexpr(\expr), l_e) $ and $\absevent \in \absflow(c)$.
      \\
      This case is proved.
      \caseL{$c = [\assign{x}{\query(\qexpr)}]^{l'}$}
      This case is proved in the same way as \textbf{case: $c = [\assign{x}{\expr}]^l$}.
      \caseL{$\ewhile [b]^{l_w} \edo c$}
      By the first rule applied to $c$, there are two cases:
      \subcaseL{$\textbf{while-t}$}
      If the first rule applied to is $\rname{while-t}$, we have
      \\
      $\config{{\ewhile [b]^{l_w} \edo c_w, \trace}}
        \xrightarrow{} 
        \config{{
        c_w; \ewhile [b]^{l_w} \edo c_w,
        \trace :: (b, l_w, \etrue, \bullet)}}~ (1)
      $.
      \\
      Let $\trace_w' \in \mathcal{T}$ be the trace satisfying following execution:
      \\
      $
      \config{{
      c_w,
      \trace :: (b, l_w, \etrue, \bullet)}}
      \xrightarrow{*} 
      \config{{
      \eskip, \trace_w'}}
    $
    \\
    By induction hypothesis on sub program $c_w$ with starting trace 
    $\trace :: (b, l_w, \etrue, \bullet)$ and ending trace $\trace_w'$, 
    we know there exist
    $\trace_w \in \mathcal{T}$ such that $\trace_w' = \trace :: (b, l_w, \etrue, \bullet) \tracecat \trace_w$.
    \\
    Then we have the following execution continued from $(1)$:
    \\
    $
    \config{{\ewhile [b]^{l_w} \edo c_w, \trace}}
        \xrightarrow{} 
        \config{{
        c_w; \ewhile [b]^{l_w} \edo c_w,
        \trace :: (b, l_w, \etrue, \bullet)}}
        \xrightarrow{*} 
        \config{\ewhile [b]^{l_w} \edo c_w, \trace :: (b, l_w, \etrue, \bullet) \tracecat \trace_w}
        ~(2)
    $
    By repeating the execution (1) and (2) until the program is evaluated into $\eskip$,
    with trace $\trace_w^{i'} $ for $i = 1, \cdots, n n \geq 1$ in each iteration, we know 
    in the $i-th$ iteration,
     there exists  $\trace_w^i \in \mathcal{T}$ such that  
    $\trace_w^{i'} = \trace_w^{(i-1)'} :: (b, l_w, \etrue, \bullet) ++ \trace_w^{i'}$
    \\
    Then we have the following execution:
    \\
    $
    \config{{\ewhile [b]^{l_w} \edo c_w, \trace}}
        \xrightarrow{} 
        \config{{
        c_w; \ewhile [b]^{l_w} \edo c_w,
        \trace :: (b, l_w, \etrue, \bullet)}}
        \xrightarrow{*} 
        \config{\ewhile [b]^{l_w} \edo c_w, \trace_w^{n'}}
        \xrightarrow{}^\rname{{while-f}}
        \config{\eskip, \trace_w^{n'}:: (b, l_w, \efalse, \bullet)}
    $ and $\trace_w^{n'} = \trace :: (b, l_w, \etrue, \bullet) \tracecat \trace_w^{1} :: \cdots :: (b, l_w, \etrue, \bullet) \tracecat \trace_w^{n} $.
    \\
    Picking $\trace' = \trace_w^{n'} :: (b, l_w, \efalse, \bullet)$ and $\trace'' = [(b, l_w, \etrue, \bullet)] \tracecat \trace_w^{1} :: \cdots :: (b, l_w, \etrue, \bullet) \tracecat \trace_w^{n}$,
    we have 
    $\trace ++ \trace'' = \trace'$.
    \\
    This case is proved.
      \subcaseL{$\textbf{while-f}$}
      If the first rule applied to $c$ is $\rname{while-f}$, we have
      \\
      $
      {
        \config{{\ewhile [b]^{l_w} \edo c_w, \trace}}
        \xrightarrow{}^\rname{while-f}
        \config{{
        \eskip,
        \trace :: (b, l_w, \efalse, \bullet)}}
      }$,
      In this case, picking $\trace' = \trace ::(b, l_w, \efalse, \bullet)$ and $\trace'' =  [(b, l_w, \efalse, \bullet) ]$,
      it is obvious that $\trace \tracecat \trace'' = \trace'$.
      \\
      This case is proved.
      \caseL{$\eif([b]^l, c_t, c_f)$}
      This case is proved in the same way as \textbf{case: $c = \ewhile [b]^{l} \edo c$}.
      \caseL{$c = c_{s1};c_{s2}$}
     By the induction hypothesis on $c_{s1}$ and $c_{s2}$ separately,
     we have this case proved.
    \end{proof}
    %
    % \todo{more explanation}
    % \mg{This corollary needs some explanation. In particular, we should stress that $\event$ and $\event'$ may differ in the query value.}
    \begin{coro}
    % \label{coro:aqintrace}
    For every event and a trace $\trace \in \mathcal{T}$,
    if $\event \in \trace$, 
    then there exist another event $\event' \in \eventset$ and traces $\trace_1, \trace_2 \in \mathcal{T}$
    such that $\trace_1 \tracecat [\event'] \tracecat \trace_2 = \trace $
    with 
    $\event$ and $\event'$ equivalent but may differ in their query value.
    \[
      \forall \event \in \eventset, \trace \in \mathcal{T} \st
    \event \in \trace \implies \exists \trace_1, \trace_2 \in \mathcal{T}, 
    \event' \in \eventset \st (\event \in \event') \land \trace_1 \tracecat [\event'] \tracecat \trace_2 = \trace  
    \]
    \end{coro}
    \begin{proof}
    % Proof in File: {\tt ``coro\_aqintrace.tex''}
    % \begin{coro}
\label{coro:aqintrace}
\[
\aq \aqin t \implies \exists t_1, t_2, \aq'. ~ s.t., ~ (\aq \aqeq \aq') \land t_1 ++ [\aq'] ++ t_2 = t	
\]
\end{coro}
\begin{subproof}
Proof in File: {\tt ``coro\_aqintrace.tex''}
% \begin{coro}
\label{coro:aqintrace}
\[
\aq \aqin t \implies \exists t_1, t_2, \aq'. ~ s.t., ~ (\aq \aqeq \aq') \land t_1 ++ [\aq'] ++ t_2 = t	
\]
\end{coro}
\begin{subproof}
Proof in File: {\tt ``coro\_aqintrace.tex''}
% \input{coro_aqintrace}
By unfolding the $\aq \aqin t$, we have the following cases:
%
\caseL{$t = []$} The hypothesis is $\efalse$, this case is proved.
%
\caseL{$t = \aq'::t' \land \aq' \aqeq \aq $}
%
Let $t_1 = []$ and $t_2 = t'$, by unfolding the list concatenation operation, we have:
%
\[
	t_1 ++ [\aq'] ++ t_2 = [] ++ [\aq'] ++ t' = t
\]
%
Since $\aq' \aqeq \aq$ by the hypothesis, this case is proved.
%
\caseL{$t = \aq'::t' \land \aq' \aqneq \aq $}
%
By induction hypothesis on $\aq \aqin t'$, we know:
%
\[
	\exists t_1', t_2', \aq''. ~ s.t., ~ (\aq \aqeq \aq'') \land t_1' ++ [\aq''] ++ t_2' = t'	
\]
%
Let $t_1 = \aq'::t_1'$ and $t_2 = t_2'$, by unfolding the list concatenation operation, we have:
%
\[
	t_1 ++ [\aq''] ++ t_2 = (\aq':: t_1') ++ [\aq''] ++ t_2' = \aq'::t' = t
\]
%
Since $\aq'' \aqeq \aq$ by the hypothesis, this case is proved.
%
\end{subproof}
By unfolding the $\aq \aqin t$, we have the following cases:
%
\caseL{$t = []$} The hypothesis is $\efalse$, this case is proved.
%
\caseL{$t = \aq'::t' \land \aq' \aqeq \aq $}
%
Let $t_1 = []$ and $t_2 = t'$, by unfolding the list concatenation operation, we have:
%
\[
	t_1 ++ [\aq'] ++ t_2 = [] ++ [\aq'] ++ t' = t
\]
%
Since $\aq' \aqeq \aq$ by the hypothesis, this case is proved.
%
\caseL{$t = \aq'::t' \land \aq' \aqneq \aq $}
%
By induction hypothesis on $\aq \aqin t'$, we know:
%
\[
	\exists t_1', t_2', \aq''. ~ s.t., ~ (\aq \aqeq \aq'') \land t_1' ++ [\aq''] ++ t_2' = t'	
\]
%
Let $t_1 = \aq'::t_1'$ and $t_2 = t_2'$, by unfolding the list concatenation operation, we have:
%
\[
	t_1 ++ [\aq''] ++ t_2 = (\aq':: t_1') ++ [\aq''] ++ t_2' = \aq'::t' = t
\]
%
Since $\aq'' \aqeq \aq$ by the hypothesis, this case is proved.
%
\end{subproof}
    By unfolding the $\aq \aqin t$, we have the following cases:
    %
    \caseL{$t = []$} The hypothesis is $\efalse$, this case is proved.
    %
    \caseL{$t = \aq'::t' \land \aq' \aqeq \aq $}
    %
    Let $t_1 = []$ and $t_2 = t'$, by unfolding the list concatenation operation, we have:
    %
    \[
        t_1 ++ [\aq'] ++ t_2 = [] ++ [\aq'] ++ t' = t
    \]
    %
    Since $\aq' \aqeq \aq$ by the hypothesis, this case is proved.
    %
    \caseL{$t = \aq'::t' \land \aq' \aqneq \aq $}
    %
    By induction hypothesis on $\aq \aqin t'$, we know:
    %
    \[
        \exists t_1', t_2', \aq''. ~ s.t., ~ (\aq \aqeq \aq'') \land t_1' ++ [\aq''] ++ t_2' = t'	
    \]
    %
    Let $t_1 = \aq'::t_1'$ and $t_2 = t_2'$, by unfolding the list concatenation operation, we have:
    %
    \[
        t_1 ++ [\aq''] ++ t_2 = (\aq':: t_1') ++ [\aq''] ++ t_2' = \aq'::t' = t
    \]
    %
    Since $\aq'' \aqeq \aq$ by the hypothesis, this case is proved.    %
    \end{proof}
\section{Proof of the Soundness of $\THESYSTEM$}
\label{apdx:adapt_soundness}
  \begin{thm}[Soundness of the \THESYSTEM]
    \label{thm:adaptfun_soundness}
  Given a program ${c}$, we have:
  %
  \[
  \progA({c}) \geq A({c})
  \]
  \end{thm}
  %
Proof Summary:
\\
construct the
program-based graph $\progG({c}) = (\progV, \progE, \progW, \progF)$
\\
and trace-based graph $\traceG({c}) = (\traceV, \traceE, \traceW, \traceF)$ 
\\
1. prove the one-on-one mapping from $\progV$ to $\traceV$, in Lemma~\ref{lem:vertex_map};
\\
2. prove the total map from $\traceE$ to $\progE$, in Lemma~\ref{lem:edge_map};
\\
3. prove that the weight of every vertex in $\traceG$ is bounded by the weight of the same vertex in $\progG$, in 
Lemma~\ref{lem:weights_map};
\\
4. prove the one-on-one mapping from $\progF$ to $\traceF$, in Lemma~\ref{lem:queryvertex_map};
\\
5. show every walk in $\walks(\traceG)$ is bounded by a walk in $\walks(\progG)$ of the same $\qlen$.
\\
6. get the conclusion that $A(c)$ is bounded by 
the $\progA(c)$.
%
\begin{proof}
Given a program ${c}$, 
we construct its 
\\
program-based graph $\progG({c}) = (\progV, \progE, \progW, \progF)$
by Definition~\ref{def:prog_graph}
\\ and 
trace-based graph $\traceG({c}) = (\traceV, \traceE, \traceW, \traceF)$  by Definition~\ref{def:trace_graph}.
\\
To show $\progA({c}) \geq A({c})$, according to the Definition \ref{def:prog_adapt} and Definition~\ref{def:trace_adapt}, it is sufficient to show:
%
$$
\max \big 
\{ \qlen(k) \mid k \in \walks(\traceG(c)) \big \} 
\leq
\max\left\{ \qlen(k) \ \mid \  k\in \walks(\progG({c}))\right \}
$$
%
%
It is sufficient to show that:
\[
  \forall k_t \st 
  k_t \in \walks(\traceG({c}) \implies
  \exists k_p \in \walks(\progG({c})) \land 
  \qlen(k_t) \leq \qlen(k_p)
\]
%
Let $k_t = (e_1, \cdots, e_{n-1}) \in \walks(\traceG(c))$ be an arbitrary walk in $\traceG(c)$ with its vertices sequence 
$(v_1, \cdots, v_n)$.
\\
By Lemma~\ref{lem:vertex_map},
we know 
% for every $v_1$, there exists a unique image of $v_1$ in vertices set of $\progG(c)$.
% Let $f_v : \mathcal{LV} \to \mathcal{LV}$ be this mapping function, we have:
\[
  \forall v_i \in (v_1, \cdots, v_n) \st v_i \in \progV
  \]
%
% By 3 and 4, we know:
% $   
% \forall v_i \in k_t \st \traceF(v_i) = \progF(f_v(v_i))
% \land 
% \traceW(v_i) \leq \progW(f_v(v_i))$.
% \\
By Lemma~\ref{lem:edge_map}, we know
%
\[
  \forall e_i \in k_t \st e_i = (v_i, v_{i + 1}) \implies
  \exists e_{pi} \st e_{pi} = (v_i, v_{i + 1}) \land e_{pi} \in \progE
  \]
  %
Then we construct a walk $k_p$ with an edge sequence $(e_{p1}, \cdots, e_{p(n-1)}) $ 
with a vertices sequence $(v_1, \cdots, v_n)$ where 
$e_{pi} = (v_i, v_{i + 1}) \in \progE$ for all $e_{pi} \in (e_{p1}, \cdots, e_{p(n-1)})$.
\\
By Definition~\ref{def:finitewalk} for finite walk, 
we know
% \[
%   \forall (v_i, n_i) \in \traceW \st v_i \in (v_1, \cdots, v_n)
%   \implies \visit((v_1, \cdots, v_n), (v_i)) \leq n_i
% \]
\[
  \forall v_i \in (v_1, \cdots, v_n) \st
  \visit((v_1, \cdots, v_n), (v_i)) \leq \traceW(v_i)
\]
%
By Lemma~\ref{lem:weights_map}, we know for every $ v_i \in (v_1, \cdots, v_n)$,
$\visit((v_1, \cdots, v_n), (v_i)) \leq \progW(v_i)$.
\\
Then, by Definition~\ref{def:finitewalk}, we know $k_p \in \walks(\progG)$.
\\
By Lemma~\ref{lem:queryvertex_map} and Definition~\ref{def:qlen}, we know $\qlen(k_t) = \qlen(k_p)$.
% Taking an arbitrary starting memory $m$ and an arbitrary underlying database $D$,
% we construct a trace-based graph $\traceG({c}, \text{D}, {m}) = (\vertxs, \edges)$ by the definition \ref{def:trace-based_graph}.
% %
% \\
% %
% Let $\midG({c},{m},\text{D}) = \{\midV, \midE, \midF\}$ be the intermediate graph by Definition~\ref{def:midgraph}.
% \\
% By Lemma~\ref{lem:bie_trace_to_mid}, we know:
% \[
%   \forall p, {m}, D, ~ s.t., ~ p \in \paths(\traceG({c}, \text{D}, {m}),
%   \exists p' \in \paths(\midG({c},{m},\text{D})) \land 
%   \len(p) = \len_q(p')
% \]
% %
% Then it is sufficient to show that:
% %
% \[
%   \forall p, {m}, D, ~ s.t., ~ p \in \paths(\midG({c}, \text{D}, {m}),
%   \exists k \in \walks(\progG({c})) \land 
%   \qlen(p) \leq \qlen(k)
% \]
% %
% We prove a stronger statement instead:
% \[
%   \forall p, {m}, D, ~ s.t., ~ p \in \paths(\midG({c}, \text{D}, {m}),
%   \exists k \in \walks(\progG({c})) \land 
%   \qlen(p) = \qlen(k) 
% \]
%
%
% By Lemma~\ref{lem:sujv_mid_to_prog}, let $g$ be the surjective function $g: \progV \to \midV$ s.t.:
% %
% $$
% \forall \av \in \midV. ~ \progF(f(\av)) = \midF(\av) 
% \land |\kw{image}(f(\av))| \leq W(f(\av)).
% $$
%
%
% \item(1) $\len(p_{\av_1 \to \av_2}) = \len(k_{f(\av_1) \to f(\av_2)})$
% %
% \item(2) $\forall \av \in p_{\av_1 \to \av_2}. ~ f(\av) \in k_{f(\av_1) \to f(\av_2)}$
% %
% \item(3) $\forall \av \in p_{\av_1 \to \av_2}. ~ 
% \kw{image}(f(\av)) \cap {p_{\av_1 \to \av_2}}| = \# \{f(\av) \mid f(\av) \in k_{f(\av_1) \to f(\av_2)}\}$
\\
This theorem is proved.
\end{proof}
The following are the four lemmas used in the proof of Theorem~\ref{thm:adaptfun_soundness} above,
showing the correspondence properties between the program based graph and trace based graph.
	\begin{lem}[One-on-One Mapping of vertices from $\traceG$ to $\progG$]
	\label{lem:vertex_map}
	Given a program $c$ with its
	program-based graph $\progG({c}) = (\progV, \progE, \progW, \progF)$
	and 
	trace-based graph $\traceG({c}) = (\traceV, \traceE, \traceW, \traceF)$,
	then for every $v \in \mathcal{VAR} \times \mathbb{N}$,
	$v \in \progV$ if and only if $v \in \traceG$.
	\[
	\begin{array}{l}
	\forall c \in \cdom , v \in \mathcal{VAR} \times \mathbb{N} \st 
	\progG({c}) = (\progV, \progE, \progW, \progF)
	\land 
	\traceG({c}) = (\traceV, \traceE, \traceW, \traceF)
	\\ \quad
	\implies
	v \in \progV \Longleftrightarrow v \in \traceV
	\end{array}
	\]
	%
	\end{lem}
\begin{proof}
Proof Summary: Proving by Definition~\ref{def:prog_graph} and Definition~\ref{def:trace_graph}.
\\
Taking arbitrary program $c$,
by Definition~\ref{def:prog_graph} and Definition~\ref{def:trace_graph}, 
we have   
\\
its program-based graph $\progG({c}) = (\progV, \progE, \progW, \progF)$ 
\\
and 
trace-based graph $\traceG({c}) = (\traceV, \traceE, \traceW, \traceF)$.
\\
By the two definitions, we also know 
$\traceV  = \lvar_c$ and $\progV = \lvar_c$.
\\
Then we know $\traceV  = \progV$, i.e., 
for arbitrary $v \in \mathcal{VAR} \times \mathbb{N}$, $v \in \progV \Longleftrightarrow v \in \traceV$.
\end{proof}
%
	\begin{lem}[Mapping from Egdes of $\traceG$ to $\progG$]
	\label{lem:edge_map}
	Given a program $c$ with its
	program-based graph $\progG({c}) = (\progV, \progE, \progW, \progF)$
	and 
	trace-based graph $\traceG({c}) = (\traceV, \traceE, \traceW, \traceF)$,
	then for every $e = (v_1, v_2) \in \traceE$, there exists an edge 
	$e' = (v_1', v_2') \in \progE$ with 
	$v_1 = v_1' \land v_2 = v_2'$.
	\[
	\begin{array}{l}
	\forall c \in \cdom \st
	 \progG({c}) = (\progV, \progE, \progW, \progF)
	\land 
	\traceG({c}) = (\traceV, \traceE, \traceW, \traceF)
	\\ \quad
	\implies
	\forall e = (v_1, v_2) \in \traceE
	\st 
	% \exists e' \in \mathcal{VAR} \times \mathbb{N} \times (\mathcal{VAR} \times \mathbb{N})\st 
	\exists e' \in \progE \st e' = (v_1, v_2)
	\end{array}
	\]
	\end{lem}
\begin{proof}
Proof Summary: Proving by Lemma~\ref{lem:vertex_map}, Lemma~\ref{thm:flowsto_soundness} Definition~\ref{def:prog_graph} and Definition~\ref{def:trace_graph}
\\
Taking arbitrary program $c$,
by Definition~\ref{def:prog_graph} and Definition~\ref{def:trace_graph}, 
we have   
\\
its program-based graph $\progG({c}) = (\progV, \progE, \progW, \progF)$ 
\\
and 
trace-based graph $\traceG({c}) = (\traceV, \traceE, \traceW, \traceF)$.
\\
Taking arbitrary edge $e = (x^i, y^j) \in \traceE$, it is sufficient to show $(x^i, y^j) \in \progE$.
\\
By Lemma~\ref{lem:vertex_map}, we know $x^i, y^j \in \progV$.
\\
By definition of $\traceE$, we know $\vardep(x^i, y^j, c)$.
\\
By Theorem~\ref{thm:flowsto_soundness}, we know $ \exists n \in \mathbb{N}, z_1^{r_1}, \cdots, z_n^{r_n} \in \lvar_{{c}} \st 
n \geq 0 \land
\flowsto(x^i,  z_1^{r_1}, c) 
\land \cdots \land \flowsto(z_n^{r_n}, y^j, c) $.
\\
Then by definition of $\progE$, we know $(x^i, y^j) \in \progE$. This Lemma is proved.
\end{proof}
%
\begin{lem}[Weights are bounded]
	\label{lem:weights_map}
	Given a program $c$ with its
	program-based graph $\progG({c}) = (\progV, \progE, \progW, \progF)$
	and 
	trace-based graph $\traceG({c}) = (\traceV, \traceE, \traceW, \traceF)$,
	for every $v \in \traceV$, there is $v \in \progV$ and $\traceW(v) \leq \progW(v)$.
%
	% then for every $(v, n) \in \traceW$, 
	% there exists $(v, n') \in \progW$ with $n \leq n'$.
	% \[
	% \begin{array}{l}
	% \forall c \in \cdom , (v, n) \in \mathcal{VAR} \times \mathbb{N} \times \mathbb{N}
	%  \st 
	%  \\ \quad
	%  \progG({c}) = (\progV, \progE, \progW, \progF)
	% \implies 
	% \traceG({c}) = (\traceV, \traceE, \traceW, \traceF)
	% \\ \quad
	% \implies
	% (v, n) \in \traceW
	% \implies 
	% \exists n' \in \mathbb{N}\st 
	% (v, n') \in \progW  \land n \leq n'
	% \end{array}
	% \]
%
\[
	\begin{array}{l}
	\forall c \in \cdom 
	% , (v, n) \in \mathcal{VAR} \times \mathbb{N} \times \mathbb{N}
	 \st 
	%  \\ \quad
	 \progG({c}) = (\progV, \progE, \progW, \progF)
	\land 
	\traceG({c}) = (\traceV, \traceE, \traceW, \traceF)
	\\ \quad
	\implies
	\forall v \in \traceV \st 
	v \in \progV \land
	\traceW(v) \leq \progW(v)
	\end{array}
	\]
	\[
		\begin{array}{l}
			\forall c \in \cdom 
			% , (v, n) \in \mathcal{VAR} \times \mathbb{N} \times \mathbb{N}
			 \st 
			%  \\ \quad
			 \progG({c}) = (\progV, \progE, \progW, \progF)
			\land 
			\traceG({c}) = (\traceV, \traceE, \traceW, \traceF)
			\\ \quad
			\implies
			% \forall v \in \traceV \st 
			% v \in \progV \land
			% \traceW(v) \leq \progW(v)
			\forall (x^l, w_{t}) \in \traceW,
			(x^l, w_{p}) \in \progW, \vtrace, \trace' \in \mathcal{T}, v \in \mathbb{N} \st
			% \lvar_c \st 
			%  \vcounter(\vtrace') l ~ \middle\vert~
			% \forall \vtrace \in \mathcal{T} \st 
			% \config{{c}, \trace} \to^{*} \config{\eskip, \trace\tracecat\vtrace'} 
			% \land 
			\config{w_{p}, \trace} \earrow v
			\implies
			% \right\} 
			w_{t}(\trace) \leq v
		\end{array}
		% 		\forall (x^l, w_{t}) \in \traceW,
		% (x^l, w_{p}) \in \progW, \vtrace, \trace' \in \mathcal{T} \st
		% % \lvar_c \st 
		% %  \vcounter(\vtrace') l ~ \middle\vert~
		% % \forall \vtrace \in \mathcal{T} \st 
		% \config{{c}, \trace} \to^{*} \config{\eskip, \trace\tracecat\vtrace'} 
		% \land 
		% \config{w_{p}, \trace} \earrow v
		% \implies
		% % \right\} 
		% \leq 
		% w_{t}(\trace) \leq v
		\]
	\end{lem}
\begin{proof}
Proof Summary: Proving by Definition~\ref{def:prog_graph}, Definition~\ref{def:trace_graph} and relying on the soundness of Reachability Bound 
Analysis.
\\
Taking arbitrary program $c$,
by Definition~\ref{def:prog_graph} and Definition~\ref{def:trace_graph}, 
we have   
\\
its program-based graph $\progG({c}) = (\progV, \progE, \progW, \progF)$ 
\\
and 
trace-based graph $\traceG({c}) = (\traceV, \traceE, \traceW, \traceF)$.
\\
Taking arbitrary 
% $v \in \traceV$, by Lemma~\ref{lem:vertex_map}, we know $v \in \progV$.
$(x^l, w_{t}) \in \traceW, (x^l, w_{p}) \in \progW, \vtrace, \trace' \in \mathcal{T}$, satisfying:
\\
$\config{{c}, \trace} \to^{*} \config{\eskip, \trace\tracecat\vtrace'} 
\land 
\config{w_{p}, \trace} \earrow v$
\\
By soundness of reachability bound analysis in Theorem~\ref{thm:addweight_soundness}, we know 
% $\traceW(v) \leq \progW(v)$. This lemma is proved.
$\vcounter(\vtrace', l) \leq v$
\\
By definition~\ref{def:trace_graph}, we know $w_t(\trace) = \vcounter(\vtrace', l)$,
then we have $w_t(\trace) \leq v$ and this is proved.
\end{proof}
%
\begin{lem}[One-on-One Mapping for Query Vertices]
	\label{lem:queryvertex_map}
	Given a program $c$ with its
	program-based graph $\progG({c}) = (\progV, \progE, \progW, \progF)$
	and 
	trace-based graph $\traceG({c}) = (\traceV, \traceE, \traceW, \traceF)$,
	then for every $(x^i, n) \in \mathcal{VAR} \times \mathbb{N}  \times \{0, 1\} $,
	 $(x^i, n) \in \traceF$ if and only if $ (x^i, n) \in \progF$.
	\[
	\begin{array}{l}
	\forall c \in \cdom , (x^i, n) \in \mathcal{VAR} \times \mathbb{N}  \times \{0, 1\} 
	 \st 
	 \\ \quad
	 \progG({c}) = (\progV, \progE, \progW, \progF)
	\land 
	\traceG({c}) = (\traceV, \traceE, \traceW, \traceF)
	\\ \quad
	\implies
	(x^i, n) \in \traceF \Longleftrightarrow  (x^i, n) \in \progF
	\end{array}
	\]
	\end{lem}
\begin{subproof}
Proving by Definition~\ref{def:prog_graph}, Definition~\ref{def:trace_graph}.
\\
Taking arbitrary program $c$,
by Definition~\ref{def:prog_graph} and Definition~\ref{def:trace_graph}, 
we have   
\\
its program-based graph $\progG({c}) = (\progV, \progE, \progW, \progF)$ 
\\
and 
trace-based graph $\traceG({c}) = (\traceV, \traceE, \traceW, \traceF)$.
\\
By the two definitions, we also know $\traceF  = \progF$, 
i.e., 
for arbitrary $ (x^i, n) \in \mathcal{VAR} \times \mathbb{N}  \times \{0, 1\} $,
 $(x^i, n) \in \traceF \Longleftrightarrow  (x^i, n) \in \progF$.
 \\
 This lemma is proved.
\end{subproof}
% \todo{
% 	\begin{lem}[Surjective Mapping from $\edges$ of $\progG)$ to $\edges$ of $\progG$].
% 	\label{lem:suje_prog_to_prog}
% 	\\
% 	$\progG({c},{m},\text{D}) = \{\progV, \progE, \progF\}$
% 	\\
% 	$\progG({c}) = \{\progV, \progE, \progF, \progW\}$
% 	\\
% 	A surjective function $f: \progV \to \progV$ s.t.,
% 	$\forall \av \in \progV. ~ \progF(f(\av)) = \progF(\av) \land |\kw{image}(f(\av))| \leq W(f(\av))$
% 	%
% \[
% 	\exists ~ \kw{surjective} ~ g: \progE \to \progE. ~
% 	\forall e_{prog} = (\av_1, \av_2) \in \progE. 
% 	\exists e_{prog} = ({f(\av_1), f(\av_2)}) \in \progE
% \]
% \end{lem}
% \begin{subproof}
% Proving by Lemma~\ref{lem:sujv_prog_to_prog}.
% \end{subproof}
% }
% \\
% \todo{
% 	\begin{lem}[Surjective Mapping from $\paths(\progG)$ to $\walks(\progG)$].
% 	\label{lem:sujpathwalk_prog_to_prog}
% 	\\
% 	$\progG({c},{m},\text{D}) = \{\progV, \progE, \progF\}$
% 	\\
% 	$\progG({c}) = \{\progV, \progE, \progF, \progW\}$
% 	\\
% 	A surjective function $f: \progV \to \progV$ s.t.,
% 	$\forall \av \in \progV. ~ \progF(f(\av)) = \progF(\av) \land |\kw{image}(f(\av))| \leq W(f(\av))$
% 	\\
% 	A surjective function $g: \progE \to \progE$ s.t.,
% 	$\forall e_{prog} = (\av_1, \av_2) \in \progE. 
% 	\exists e_{prog} = ({f(\av_1) \to f(\av_2)}) \in \progE$
% 	\\
% 	$\exists ~ \kw{surjective} ~ h: \paths(\progG({c},{m},\text{D})) \to \walks(\progG({c}))$ s.t.:
% 	%
% \[
% 	\forall p_{\av_1 \to \av_2} \in \paths(\progG({c},{m},\text{D}))
% 	\text{ with }
% 	\left\{
% 	\begin{array}{ll}
% 	\mbox{edge sequence:} & (e, \ldots, e_{n-1})
% 	\\ 
% 	\mbox{vertices sequence:} & (\av_1, \ldots, \av_n)
% 	\end{array}
% 	\right.
% \]
% \[
% 	\exists k_{f(\av_1) \to f(\av_2)} \in \walks(\progG({c}))
% 	\text{ with }
% 	\left\{
% 	\begin{array}{ll}
% 	\mbox{edge sequence:} & (g(e), \ldots, g(e_{n-1}) 
% 	\\ 
% 	\mbox{vertices sequence:} & (f(\av_1), \ldots, f(\av_{n}))
% 	\end{array}
% 	\right.
% \]
% % \item $(e, \ldots, e_{n-1})$, $(\av_1, \ldots, \av_n)$ are the edges sequence and vertices sequence of $p_{\av_1 \to \av_2}$.
% % then, 
% %  $\len(p_{\av_1 \to \av_2}) = \len(k_{f(\av_1) \to f(\av_2)})$
% % %
% % \item $\forall \av \in p_{\av_1 \to \av_2}. ~ f(\av) \in k_{f(\av_1) \to f(\av_2)}$
% % %
% % \item $\forall \av \in p_{\av_1 \to \av_2}. ~ 
% % \kw{image}(f(\av)) \cap {p_{\av_1 \to \av_2}}| = \# \{f(\av) \prog f(\av) \in k_{f(\av_1) \to f(\av_2)}\}
% % $
% \end{lem}
% %
% \begin{subproof}
% Proving by induction on the length of $l = p_{\av_1 \to \av_2} \in \paths(\progG({c},{m},\text{D}))$, and Lemma~\ref{lem:suje_prog_to_prog} and Lemma~\ref{lem:sujv_prog_to_prog}.
% \caseL{ $l = 1$: }
% \caseL{ $l = l' + 1$, $l' \geq 1$: }
% \end{subproof}
% % \end{proof}
% %
% %
% %
% }
\clearpage
\section{Proof of the Soundness of $\flowsto$}
\label{apdx:flowsto_soundness}
\begin{thm}[$\vardep$ implies $\flowsto$]
\label{thm:flowsto_soundness}
Given a program ${c}$, for all  $ x^i, y^j \in \lvar_{{c}}$, if $\vardep(x^i, y^j, {c})$,
then
%  either $\flowsto(x^i, y^j, c)$ 
% or 
there exist $z_1^{r_1}, \cdots, z_n^{r_n} \in \lvar_{{c}}$ with $n \geq 0$ such that   
$\flowsto(x^i,  z_1^{r_1}, c) 
\land \cdots \land \flowsto(z_n^{r_n}, y^j, c)$
%
\[
\begin{array}{l}
  \forall x^i, y^j \in \lvar_{{c}}.
  \vardep(x^i, y^j, {c})
  \\ \quad \implies
  \Big( \exists n \in \mathbb{N}, z_1^{r_1}, \cdots, z_n^{r_n} \in \lvar_{{c}} \st n \geq 0 \land
  \flowsto(x^i,  z_1^{r_1}, c) 
  \land \cdots \land \flowsto(z_n^{r_n}, y^j, c) \Big)
\end{array}
\]
\end{thm}
Proof Summary:
\\
Taking arbitrary $x^i, y^j \in \lvar_c$,
  it is sufficient to show:
  \\
  1. $x^i$ is directly used in the expression of the assignment command associated to $y^j$, or a boolean
  expression of the guard for a $\eif$ or $\ewhile$ command with the assignment command associated to $y^j$ showing up in the body of that command, 
  we call it, $x^i$ directly flows to $y^j$,
  i.e.,
  $ \flowsto(x^i, y^j, {c})$;
  \\
  2. otherwise, there exists another labelled variable $z^l$ with \emph{variable May-Dependency} relation on $x^i$ and 
  $z^l$ directly flows to $y^j$, where the \emph{variable May-Dependency} relation between $x^i$ and $z^l$
   implies a "sub flowsto-chain" from $z^i$ to $z^l$, 
  i.e., 
  \\
  $\Big(
    \exists z^l \in \lvar_{{c}}.
  \left(\vardep(x^i, z^l, {c})   
  \implies
   \exists n \in \mathbb{N}, z_1^{r_1}, \cdots, z_n^{r_n} \in \lvar_{{c}} \st n \geq 0 \land
  \flowsto(x^i,  z_1^{r_1}, c) 
  \land \cdots \land \flowsto(z_n^{r_n}, z^l, c)
  \right)
  \land  \flowsto(z^l, y^j, {c})
  \Big)$.
%
\\ Formally as follows:
\[
  \begin{array}{l}
    \flowsto(x^i, y^j, {c})
    \lor 
      \exists z^l \in \lvar_{{c}} \st
    \\
    \Big(
    \vardep(x^i, z^l, {c})   
  \implies
     \exists n \in \mathbb{N}, z_1^{r_1}, \cdots, z_n^{r_n} \in \lvar_{{c}} \st n \geq 0 \land
    \flowsto(x^i,  z_1^{r_1}, c) 
    \\ \qquad \qquad 
    \land \cdots  \land \flowsto(z_n^{r_n}, z^l, c) 
    \Big)
    \land  \flowsto(z^l, y^j, {c})
  \end{array}
  \]
\\
%
Without induction hypothesis, the existence of $z^l$, and the conclusion that there is a "sub flowsto-chain" from $x^i$ to $z^l$ in step 2 
is not proved in previous soundness proof.
\\
%
So, I involved trace in the proof and adopt the way of induction on trace.
By restricting the domain of $z^l$ on the variables "showing up" in the 
trace between evaluating $x^i$ and $y^j$, 
I firstly show the existence of this $z^l$, and then get the conclusion of the  "sub flowsto-chain" from induction hypothesis. 
\\
%
By definition of $\vardep(x^i, y^j, {c})$, 
let $D \in \dbdom$ be the dataset,
and $\trace \in \mathcal{T}$, $\event_x, \event_y$ be the trace and two events satisfying the definition, 
with $\pi_1(\event_x) = x$ and $\pi_1(\event_y) = y$, then informally it is sufficient to show:
% then it is sufficient to show:
%     \[
%       \begin{array}{l}
%       \Big(
%          \exists \trace \in \mathcal{T} \st 
%         \eventdep^{val}(\event_x, \event_y, \trace, c, D) \lor 
%         (\exists \event_b \in \eventset^{\test} \st 
%         \eventdep^{val}(\event_x, \event_b, \trace, c, D)
%         \land \eventdep^{\ctl}(\event_b, \event_y, c, D))  
%         \Big)
%         \\ \quad \implies
%   \Big( \exists n \in \mathbb{N}, z_1^{r_1}, \cdots, z_n^{r_n} \in \lvar_{{c}} \st n \geq 0 \land
%   \flowsto(x^i,  z_1^{r_1}, c) 
%   \land \cdots \land \flowsto(z_n^{r_n}, y^j, c) \Big)
%       \end{array}
%     \]
%     \[
%         (\exists \event_b \in \eventset^{\test} \st x^i \in VAR(\pi_1(\event_b)) 
%         \land \eventdep^{\ctl}(\event_b, \event_y, c, D))    
%     \implies
%   \Big( \exists n \in \mathbb{N}, z_1^{r_1}, \cdots, z_n^{r_n} \in \lvar_{{c}} \st n \geq 0 \land
%   \flowsto(x^i,  z_1^{r_1}, c) 
%   \land \cdots \land \flowsto(z_n^{r_n}, y^j, c) \Big)
% \]
% Proved by applying Theorem.~\ref{thm:algvardep_sound} and Correctness of $\mathcal{A}$ in Theorem.~\ref{thm:alg_correct}.
% \[
% 	\begin{array}{l}
% 		\forall D \in \dbdom , c \in \cdom, \trace \in \mathcal{T} \st \forall \event_1, \event_2 \in \eventset^{\asn} \st
% 		 \exists \trace' \in \mathcal{T} \st \trace = [\event_1] \tracecat \trace' \tracecat [\event_2]
% 		\implies
% 		\eventdep(\event_1, \event_2, \trace, c, D) 
% 		\\ \quad 
% 		\implies 
% 		\flowsto(\pi_1(\event_1)^{\pi_2(\event_1)}, \pi_1(\event_2)^{\pi_2(\event_2)}, c) 
% 		\\ \qquad \quad \lor
% 		\exists \event \in \trace' \st 
% 		\left( 		
% 			\eventdep(\event_1, \event, \trace[\event_1:\event], c, D)
% 		\land 
% 		\flowsto(\pi_1(\event)^{\pi_2(\event)}, \pi_1(\event_2)^{\pi_2(\event_2)}, c) 
% 	\right) 
% 		% \\ \qquad \qquad \lor
% 		% \flowsto(\pi_1(\event_1)^{\pi_2(\event_1)}, \pi_1(\event_2)^{\pi_2(\event_2)}, c) 
% 	\end{array}
% 	\]
%
$$
\begin{array}{l}
(\flowsto(\pi_1(\event_1)^{\pi_2(\event_1)}, \pi_1(\event_2)^{\pi_2(\event_2)}, c)
\\ \quad 
\lor 
(\exists \event_z \in \trace' \st \event_z \in \eventset^{\asn} \land 
    \eventdep(\event_x, \event_z, \trace[\event_x: \event_z], c, D)
    \\ \qquad 
    \implies 
    \exists n \in \mathbb{N}, z_1^{r_1}, \cdots, z_n^{r_n} \in \lvar_{{c}} \st n \geq 0 \land
\flowsto(x^i,  z_1^{r_1}, c) 
\land \cdots \land \flowsto(z_n^{r_n},  \pi_1(\event_z)^{\pi_2(\event_z)}, c)
    )
    \\ \quad
    \land 
    \flowsto(\pi_1(\event_z)^{\pi_2(\event_z)}, \pi_1(\event_2)^{\pi_2(\event_2)}, c) 
\end{array}
$$
%
It is clearer to show it in two lemmas:
\\
1. Existence of a middle event:
\[
	\begin{array}{l}
		\forall D \in \dbdom , c \in \cdom, \trace \in \mathcal{T} \st \forall \event_1, \event_2 \in \eventset^{\asn} \st
		 \exists \trace' \in \mathcal{T} \st \trace = [\event_1] \tracecat \trace' \tracecat [\event_2]
		\implies
		\eventdep(\event_1, \event_2, \trace, c, D) 
		\\ \quad 
		\implies 
		\flowsto(\pi_1(\event_1)^{\pi_2(\event_1)}, \pi_1(\event_2)^{\pi_2(\event_2)}, c) 
		\\ \qquad \quad \lor
		\exists \event \in \trace' \st 
		\left( 		\event_z \in \eventset^{\asn} \land 
			\eventdep(\event_1, \event, \trace[\event_1:\event], c, D)
		\land 
		\flowsto(\pi_1(\event)^{\pi_2(\event)}, \pi_1(\event_2)^{\pi_2(\event_2)}, c) 
	\right) 
		% \\ \qquad \qquad \lor
		% \flowsto(\pi_1(\event_1)^{\pi_2(\event_1)}, \pi_1(\event_2)^{\pi_2(\event_2)}, c) 
	\end{array}
\]
Proved by showing a contradiction, with detail in Lemma~\ref{lem:depevents_exist}.
%
\\
2. The middle event with a sub-trace implies a "sub flowsto-chain", informally:
%
\[
  \begin{array}{l}
    \eventdep(\event_x, \event_z, \trace[\event_x: \event_z], c, D)
    \\ \qquad 
    \implies 
    \exists n \in \mathbb{N}, z_1^{r_1}, \cdots, z_n^{r_n} \in \lvar_{{c}} \st n \geq 0 \land
\flowsto(x^i,  z_1^{r_1}, c) 
\land \cdots \land \flowsto(z_n^{r_n},  \pi_1(\event_z)^{\pi_2(\event_z)}, c)
\end{array}
\]
%
Proved formally as follows with detail in Theorem~\ref{thm:alg_correct}, by induction on the trace $\trace$
%
\[
  \begin{array}{l}
		\forall D \in \dbdom , c \in \cdom, \trace \in \mathcal{T} \st \forall \event_1, \event_2 \in \eventset \st
    \event_1, \event_2 \in \eventset^{\asn} \land 
		 \exists \trace' \in \mathcal{T} \st \trace = [\event_1] \tracecat \trace' \tracecat [\event_2]
		\implies
		\eventdep(\event_1, \event_2, \trace, c, D) 
		\\ \quad 
		\implies 
    \exists n \in \mathbb{N}, z_1^{r_1}, \cdots, z_n^{r_n} \in \lvar_{{c}} \st n \geq 0 \land
		\flowsto(\pi_1(\event_1)^{\pi_2(\event_1)},  z_1^{r_1}, c) 
    \land \cdots \land \flowsto(z_n^{r_n}, \pi_1(\event_2)^{\pi_2(\event_2)}, c) 
	\end{array}
\]
with the induction hypothesis:
\[
  \begin{array}{l}
    \forall \event_{ih1}, \event_{ih2} \in \trace \st \event_{ih1}, \event_{ih2} \in \eventset^{\asn} 
    \land
		 \exists \trace' \in \mathcal{T} \st 
     \trace[\event_{ih1}:\event_{ih2}] = [\event_{ih1}] \tracecat \trace' \tracecat [\event_{ih2}]
		\implies
		\eventdep(\event_{ih1}, \event_{ih2}, \trace[\event_{ih1}:\event_{ih2}], c, D) 
		\\ \quad 
		\implies 
    \exists n \in \mathbb{N}, z_1^{r_1}, \cdots, z_n^{r_n} \in \lvar_{{c}} \st n \geq 0 \land
		\flowsto(\pi_1(\event_{ih1})^{\pi_2(\event_{ih1})},  z_1^{r_1}, c) 
    \land \cdots \land \flowsto(z_n^{r_n}, \pi_1(\event_{ih2})^{\pi_2(\event_{ih2})}, c) 
  \end{array}
  \]
%
Main proof logic:
\\
(1). obtaining the existence of $\event_z \in \eventset^{\asn} $ with dependency on $\event_x$, and a "direct flowsto" from $\event_z$ to $\event_y$
 from step 1;
 \\
(2). from the dependency of the $\event_z$ with $\event_x$ on the subtrace,
 obtaining a "sub flowsto-chain" by induction  hypothesis;
 \\
(3). composing the "sub flowsto-chain" from (2) with the  "direct flowsto" from (1), and getting the conclusion of
 the complete "flowsto chain".
%
\begin{proof}
  Taking arbitrary $x^i, y^j \in \lvar_c$,
%
%
by definition of $\vardep(x^i, y^j, {c})$, 
let $D \in \dbdom$ be the dataset,
and $\trace \in \mathcal{T}$, $\event_x, \event_y$ be the trace and two events satisfying the definition, 
with $\pi_1(\event_x)^{\pi_2(\event_x)} = x^i$ and $\pi_1(\event_y)^{\pi_2(\event_y)} = y^j$,  it is sufficient to show:
$$
\begin{array}{l}
  \eventdep(\event_x, \event_y, \trace, c, D) 
  \\ \quad 
  \implies 
  \exists n \in \mathbb{N}, z_1^{r_1}, \cdots, z_n^{r_n} \in \lvar_{{c}} \st n \geq 0 \land
  \flowsto(x^i,  z_1^{r_1}, c) 
  \land \cdots \land \flowsto(z_n^{r_n}, y^j, c) 
\end{array}
$$
%
By Theorem~\ref{thm:flowsto_event_soundness}, we have this theorem proved.
\end{proof}
\clearpage
\section{Proof of the Soundness of $\flowsto$ w.r.t. the Event}
\label{apdx:flowsto_event_soundness}
\subsection{Main Theorem Proof}
For concise of the proof, we introduce some conventional operators as follows.
\begin{defn}[Subtrace]
  Subtrace: $[ : ] : \mathcal{{T} \to \eventset \to \eventset \to \mathcal{T}}$ 
  \[
    \trace[\event_1 : \event_2] \triangleq
    \left\{
    \begin{array}{ll} 
    \trace'[\event_1: \event_2]             & \trace = \event :: \trace' \land \event \eventneq \event_1 \\
    \event_1 :: \trace'[:\event_2]  & \trace = \event :: \trace' \land \event \eventeq \event_1 \\
    {[]} & \trace = [] \\
    \end{array}
    \right.
  \]
  For any trace $\trace$ and two events $\event_1, \event_2 \in \eventset$,
  $\trace[\event_1 : \event_2]$ takes the subtrace of $\trace$ starting with $\event_1$ and ending with $\event_2$ including $\event_1$ and $\event_2$.
  \\
  We use $\trace[:\event_2] $ as the shorthand of subtrace starting from head and ending with $\event_2$, and similary for $\trace[\event_1:]$.
  \[
    \trace[:\event] \triangleq
    \left\{
    \begin{array}{ll} 
   \event' :: \trace'[: \event]             & \trace = \event' :: \trace' \land \event' \eventneq \event \\
    \event'  & \trace = \event' :: \trace' \land \event' \eventeq \event \\
    {[]}  & \trace = [] 
    \end{array}
    \right.
  % \]
  % \[
    \quad
    \trace[\event: ] \triangleq
    \left\{
    \begin{array}{ll} 
    \trace'[\event: ]     & \trace =  \event' :: \trace' \land \event \eventneq \event' \\
    \event' :: \trace'  & \trace = \event' :: \trace' \land \event \eventeq \event' \\
    {[ ] } & \trace = []
    \end{array}
    \right.
  \]
\end{defn}
%
\begin{thm}[$\eventdep$ implies $\flowsto$]
\label{thm:flowsto_event_soundness}
For every $D \in \dbdom , c \in \cdom, \trace \in \mathcal{T} \st \forall \event_1, \event_2 \in \eventset \st
\event_1, \event_2 \in \eventset^{\asn}$, 
if $\exists \trace' \in \mathcal{T} \st \trace = [\event_1] \tracecat \trace' \tracecat [\event_2]$ and 
$\eventdep(\event_1, \event_2, \trace, c, D) $, then
 $z_1^{r_1}, \cdots, z_n^{r_n} \in \lvar_{{c}}$ with $n \geq 0$ such that   
$\flowsto(x^i,  z_1^{r_1}, c) 
\land \cdots \land \flowsto(z_n^{r_n}, y^j, c)$
%
\[
  \begin{array}{l}
		\forall D \in \dbdom , c \in \cdom, \trace \in \mathcal{T} \st \forall \event_1, \event_2 \in \eventset \st
    \event_1, \event_2 \in \eventset^{\asn} \land 
		 \exists \trace' \in \mathcal{T} \st \trace = [\event_1] \tracecat \trace' \tracecat [\event_2]
		\implies
		\eventdep(\event_1, \event_2, \trace, c, D) 
		\\ \quad 
		\implies 
    \exists n \in \mathbb{N}, z_1^{r_1}, \cdots, z_n^{r_n} \in \lvar_{{c}} \st n \geq 0 \land
		\flowsto(\pi_1(\event_1)^{\pi_2(\event_1)},  z_1^{r_1}, c) 
    \land \cdots \land \flowsto(z_n^{r_n}, \pi_1(\event_2)^{\pi_2(\event_2)}, c) 
	\end{array}
\]
\end{thm}
Proof Summary:
I. Vacuously True cases, where trace doesn't satisfy the hypothesis 
\\
II. Base case where $\trace = [\event_1;\event_2]$
\\
III. inductive case where $\trace = [\event_1, \cdots, \event_2]$.
\\
1. Existence of a middle event:
\\
Proved by showing a contradiction, with detail in Lemma~\ref{lem:depevents_exist}.
%
\\
2. The middle event with a sub-trace implies a "sub flowsto-chain", informally:
%
\\
(1). obtaining the existence of $\event_z \in \eventset^{\asn} $ with dependency on $\event_x$, and a "direct flowsto" from $\event_z$ to $\event_y$
by Lemma~\ref{lem:depevents_exist}.
 \\
(2). from the dependency of the $\event_z$ with $\event_x$ on the subtrace,
 obtaining a "sub flowsto-chain" by induction  hypothesis;
 \\
(3). composing the "sub flowsto-chain" from (2) with the  "direct flowsto" from (1), and getting the conclusion of
 the complete "flowsto chain".
%
\begin{proof}
  Taking arbitrary $D \in \dbdom , c \in \cdom,$ by induction on the trace $\trace$ we have the following cases:
  \begin{case}($\trace = {[]}$)
    \\
    Since for all $\event_1, \event_2 \in \eventset^{\asn}$,
     $\not\exists \trace' \in \mathcal{T}$,satisfies $
    {[]}  = [\event_1] \tracecat \trace' \tracecat [\event_2]$, the theorem is vacuously true.
    \end{case}
    %
    \begin{case}($\event \in \eventset, \trace = [\event]$)
    \\
    Since for all $\event_1, \event_2 \in \eventset^{\asn}$,
     $\not\exists \trace' \in \mathcal{T}$,satisfies $
    {[]} = [\event_1] \tracecat \trace' \tracecat [\event_2]$, the theorem is vacuously true.
    \end{case}
    %
    \begin{case}
      \label{case:soundness_basecase}
      ($\event_1', \event_2' \in \eventset $, $\trace = [\event_1'; \event_2']$)
      \\
      To show:
      % \wq{$\forall l \in \mathcal{A}(\event_1, \event_2, \trace, c, D)$ below?}
      %\jl{yes, there is a typo}
      \[
      \begin{array}{l}
        \forall \event_1, \event_2 \in \eventset^{\asn} \st
        \exists \trace' \in \mathcal{T} \st [\event_1'; \event_2'] = [\event_1] \tracecat \trace' \tracecat [\event_2]
        \\ \qquad 
            \implies    
        % \forall  z^i, y^j \in \lvar_c, l_h, l_t \st 
          \eventdep(\event_1, \event_2, [\event_1; \event_2], c, D)
         \implies \flowsto(\pi_1(\event_1)^{\pi_2(\event_1)}, \pi_1(\event_2)^{\pi_2(\event_2)}, c)
      \end{array}
      \]
      %
      Taking arbitrary $ \event_1, \event_2 \in \eventset^{\asn}$, by law of excluded middle, there are 2 cases:
      \\
      $\event_1 = \event_1' \land  \event_2 = \event_2'$
      \\
      $\neg(\event_1 = \event_1' \land  \event_2 = \event_2')$
      \\
      In case of $\neg(\event_1 = \event_1' \land  \event_2 = \event_2')$, since 
      $\not\exists \trace' \in \mathcal{T}$,satisfies $
      [\event_1'; \event_2'] = [\event_1] \tracecat \trace' \tracecat [\event_2]$, 
      the theorem is vacuously true.
      \\
      %
      In case of $\event_1 = \event_1' \land  \event_2 = \event_2'$,
      let $\trace' = []$, we know $\exists \trace' \in \mathcal{T}$ satisfying 
      $[\event_1; \event_2] = [\event_1] \tracecat \trace' \tracecat [\event_2]$.
      \\
      % By Inversion Lemma~\ref{lem:inv_alg2}, we have either one of the two following cases:
      % \begin{enumerate}
      %   \item $\mathcal{A}(\event_1, \event_2, [\event_1; \event_2], c, D) = 
      %   \left\{[\pi_1(\event_1)^{\pi_2(\event_1)}, \pi_1(\event_2)^{\pi_2(\event_2)}] \right \}$ 
      %   and $\eventdep(\event_1, \event_2, \cdot  \event_1 \cdot \event_2, c, D)$.
      %   \item  $\mathcal{A}(\event_1, \event_2, [\event_1; \event_2], c, D) = \{\}$ 
      %   and $\neg \eventdep(\event_1, \event_2, \cdot  \event_1 \cdot \event_2, c, D)$;
      % \end{enumerate}
      % %
      % In case of $\mathcal{A}(\event_1, \event_2, [\event_1; \event_2], c, D) = \{\}$,
      % since $\not\exists  z^i, y^j \in \lvar_c, l_h, l_t \st l_h ++ [z^i, y^j] ++ l_t \in \{\}$, the theorem is vacuously true.
      % %
      % \\
      % Then in the case 1., 
      Then it is sufficient to show: 
      % \wq{Because $l \in A \implies DEP$ }
      % \jl{yes}
      \[
        \eventdep(\event_1, \event_2, [\event_1; \event_2], c, D) 
        \implies \flowsto(\pi_1(\event_1)^{\pi_2(\event_1)}, \pi_1(\event_2)^{\pi_2(\event_2)}, c)
        % \land \eventdep(\event_1, \event_2, \cdot  \event_1 \cdot \event_2, c, D)
      \]
      %
     By Lemma~\ref{lem:flowsto_soundness_emptytrace}, we have this case proved.
      %
      \end{case}
      %
        %
        \begin{case}
        ($\event_1', \event_2' \in \eventset$, $\trace_{ih} \in \mathcal{T}, \trace = [\event_1'] \tracecat \trace_{ih} \tracecat [\event_2'] \land \trace_{ih} \neq {[]}$)
        \\
        It is sufficient to show:
        \[    \begin{array}{l}
         \forall \event_1, \event_2 \in \eventset \st
          \event_1, \event_2 \in \eventset^{\asn} \land 
           \exists \trace' \in \mathcal{T} \st \trace = [\event_1] \tracecat \trace' \tracecat [\event_2]
          \implies
          \eventdep(\event_1, \event_2, [\event_1'] \tracecat \trace_{ih} \tracecat [\event_2'], c, D) 
          \\ \quad 
          \implies 
          \exists n \in \mathbb{N}, z_1^{r_1}, \cdots, z_n^{r_n} \in \lvar_{{c}} \st n \geq 0 \land
          \flowsto(\pi_1(\event_1)^{\pi_2(\event_1)},  z_1^{r_1}, c) 
          \land \cdots \land \flowsto(z_n^{r_n}, \pi_1(\event_2)^{\pi_2(\event_2)}, c) 
        \end{array}
        \]
        %
        with the induction hypothesis:
        %
        \[
          \begin{array}{l}
            \forall \event_{ih1}, \event_{ih2} \in \trace \st \event_{ih1}, \event_{ih2} \in \eventset^{\asn} 
            \land
             \exists \trace' \in \mathcal{T} \st 
             \trace[\event_{ih1}:\event_{ih2}] = [\event_{ih1}] \tracecat \trace' \tracecat [\event_{ih2}]
            \implies
            \eventdep(\event_{ih1}, \event_{ih2}, \trace[\event_{ih1}:\event_{ih2}], c, D) 
            \\ \quad 
            \implies 
            \exists n \in \mathbb{N}, z_1^{r_1}, \cdots, z_n^{r_n} \in \lvar_{{c}} \st n \geq 0 \land
            \flowsto(\pi_1(\event_{ih1})^{\pi_2(\event_{ih1})},  z_1^{r_1}, c) 
            \land \cdots \land \flowsto(z_n^{r_n}, \pi_1(\event_{ih2})^{\pi_2(\event_{ih2})}, c) 
          \end{array}
        \]
        %
        %
        Taking arbitrary $ \event_1, \event_2 \in \eventset^{\asn}$, by law of excluded middle, there are 2 cases:
        \\
        $\event_1 = \event_1' \land  \event_2 = \event_2'$
        \\
        $\neg(\event_1 = \event_1' \land  \event_2 = \event_2')$
        \\
        In case of $\neg(\event_1 = \event_1' \land  \event_2 = \event_2')$, since 
        $\not\exists \trace' \in \mathcal{T}$,satisfies $
        [\event_1'] \tracecat \trace_{ih} \tracecat [\event_2']  = [\event_1] \tracecat \trace' \tracecat [\event_2]$, 
        the theorem is vacuously true.
        \\
        %
        In case of $\event_1 = \event_1' \land  \event_2 = \event_2'$,
        let $\trace' = \trace_{ih}$, we know $\exists \trace' \in \mathcal{T}$ satisfying 
        $[\event_1'] \tracecat \trace_{ih} \tracecat [\event_2'] = [\event_1] \tracecat \trace' \tracecat [\event_2]$.
        \\
        To show:
        \[
          \begin{array}{l}
            \eventdep(\event_1, \event_2, [\event_1] \tracecat \trace_{ih} \tracecat [\event_2], c, D) 
          \\ \quad 
          \implies 
          \exists n \in \mathbb{N}, z_1^{r_1}, \cdots, z_n^{r_n} \in \lvar_{{c}} \st n \geq 0 \land
          \flowsto(\pi_1(\event_1)^{\pi_2(\event_1)},  z_1^{r_1}, c) 
          \land \cdots \land \flowsto(z_n^{r_n}, \pi_1(\event_2)^{\pi_2(\event_2)}, c) 
        \end{array}
        \]
        %
        By Lemma~\ref{lem:depevents_exist}, we know:
        \[
          \begin{array}{l}
            \flowsto(\pi_1(\event_1)^{\pi_2(\event_1)}, \pi_1(\event_2)^{\pi_2(\event_2)}, c) 
        \\ \qquad \quad \lor
        \exists \event \in \trace_{ih} \st 
        % \left( 		
          \eventdep(\event_1, \event, \trace[\event_1:\event], c, D)
        \land 
        \flowsto(\pi_1(\event)^{\pi_2(\event)}, \pi_1(\event_2)^{\pi_2(\event_2)}, c) 
      \end{array}
      \]
        \\
        In first case, we have $\flowsto(\pi_1(\event_1)^{\pi_2(\event_1)}, \pi_1(\event_2)^{\pi_2(\event_2)}, c) $ proved directly.
        \\
        In the second case, let $\event_{ih}$ be this event, from the induction hypothesis, we know:
        \[
          \exists n \in \mathbb{N}, z_1^{r_1}, \cdots, z_n^{r_n} \in \lvar_{{c}} \st n \geq 0 \land
            \flowsto(\pi_1(\event_1)^{\pi_2(\event_1)},  z_1^{r_1}, c) 
            \land \cdots \land \flowsto(z_n^{r_n}, \pi_1(\event_{ih})^{\pi_2(\event_{ih})}, c) 
          \]
          % \\
          Then we know:
          \[
            \begin{array}{l}
              \exists n \in \mathbb{N}, z_1^{r_1}, \cdots, z_n^{r_n} \in \lvar_{{c}} \st n \geq 0 \land
            \flowsto(\pi_1(\event_1)^{\pi_2(\event_1)},  z_1^{r_1}, c) 
            \land \cdots \land \flowsto(z_n^{r_n}, \pi_1(\event_{ih})^{\pi_2(\event_{ih})}, c) )
             \\ \quad  
             \land  \flowsto(\pi_1(\event)^{\pi_2(\event)}, \pi_1(\event_2)^{\pi_2(\event_2)}, c) 
            \end{array}
            \]
            This case is proved.
      \end{case}
      %
  % \[
  %   \begin{array}{l}
  %     \forall D \in \dbdom , c \in \cdom, \trace \in \mathcal{T} \st \forall \event_1, \event_2 \in \eventset \st
  %     \event_1, \event_2 \in \eventset^{\asn} \land 
  %      \exists \trace' \in \mathcal{T} \st \trace = [\event_1] \tracecat \trace' \tracecat [\event_2]
  %     \implies
  %     \eventdep(\event_1, \event_2, \trace, c, D) 
  %     \\ \quad 
  %     \implies 
  %     \exists n \in \mathbb{N}, z_1^{r_1}, \cdots, z_n^{r_n} \in \lvar_{{c}} \st n \geq 0 \land
  %     \flowsto(\pi_1(\event_1)^{\pi_2(\event_1)},  z_1^{r_1}, c) 
  %     \land \cdots \land \flowsto(z_n^{r_n}, \pi_1(\event_2)^{\pi_2(\event_2)}, c) 
  %   \end{array}
  % \]

%
\end{proof}

\subsection{Inversion Lemmas and Helper Lemmas}
The following are the inversion lemmas and helper lemmas used in the proof of Theorem~\ref{thm:flowsto_event_soundness} above,
showing the correspondence properties between the trace based semantics and the program analysis results.
\begin{lem}[The One-Step Event Dependency Inversion]
	\label{lem:flowsto_soundness_emptytrace}
	For every $ c \in \cdom, D \in \dbdom$ and two assignment events $\event_1, \event_2 \in \eventset^{\asn}$,
	if $\eventdep(\event_1, \event_2, [\event_1; \event_2],  c, D) $,
	then, $\flowsto(\pi_1(\event_1)^{\pi_2(\event_1)}, \pi_1(\event_2)^{\pi_2(\event_2)}, c)$.
	%
	\[
	\begin{array}{l}
		\forall \event_1, \event_2 \in \eventset^{\asn}, c \in \cdom, D \in \dbdom 
		\st 
		\eventdep(\event_1, \event_2, [\event_1; \event_2],  c, D) 
		\\ \quad 
		\implies 
		\flowsto(\pi_1(\event_1)^{\pi_2(\event_1)}, \pi_1(\event_2)^{\pi_2(\event_2)}, c)
	\end{array}
	\]
\end{lem}
Proof Summary:
\\
1. case of (the labelled unique assignment command associated to the $\event_2$ 
is executed but the value assigned to the variable in this event is changed in second execution)
\\
show x directly used by the assignment of the second event
\\
2.(the labelled unique assignment command associated to the $\event_2$ isn't executed in second execution)
\\
show x is directly used by the boolean expression for a conditional command and second event shows in the body of that conditional command 
%
\begin{proof}
	By the Definition~\ref{def:event_dep} for $\eventdep(\event_1, \event_2, [\event_1; \event_2], c, D)$, 
	we know there are 2 cases:
	%
	\caseL{1}
		\textbf{(the labelled unique assignment command associated to the $\event_2$ 
		is executed but the value assigned to the variable in this event is changed in second execution).}
	\begin{subproof}
%
\label{pf:eventdep_base_val}
We have the following by the definition $\eventdep(\event_1, \event_2, [\event_1; \event_2], c, D)$ of case 1:
\begin{equation}
  \label{eq:eventdep_def_base_val}
  \exists \vtrace_0,
    \vtrace_1, \vtrace' \in \mathcal{T},\event_1' \in \eventset^{\asn}, \event_2' \in \eventset, {c}_1, {c}_2  \in \cdom  \st
    \diff(\event_1, \event_1') \land
      \left(
      \begin{array}{ll}   
     & \config{{c}, \vtrace_0} \rightarrow^{*} 
    \config{{c}_1, \vtrace_1 \tracecat [\event_1]}  \rightarrow^{*} 
      \config{{c}_2,  \vtrace_1 \tracecat [\event_1; \event_2] } 
      % 
     \\ 
     \bigwedge &
      \config{{c}_1, \vtrace_1 \tracecat [\event_1']}  \rightarrow^{*} 
      \config{{c}_2,  \vtrace_1 \tracecat[ \event_1'] \tracecat \vtrace' \tracecat [\event_2'] } 
    \\
    \bigwedge & 
    \diff(\event_2,\event_2') \land 
    \vcounter(\vtrace) ~ \pi_2(\event_2)
    = 
    \vcounter(\vtrace') ~ \pi_2(\event_2)\\
    \end{array}
    \right)
  \end{equation}
  %
Let $\vtrace_0,
\vtrace_1, \vtrace' \in \mathcal{T}, \event_2' \in \eventset, \event_1' \in \eventset^{\asn}, {c}_1, {c}_2$ be the traces, events and commands satisfying the executions,
by Inversion Lemma~\ref{lem:inv_event} on 
$\event_1$, $\event_2$, we have the following instance of the first execution in Eq.~\ref{eq:eventdep_def_base_val},
 %
%
 %
\begin{equation}
\label{eq:valdep_inv1}
  \begin{array}{l}   
\config{{c}, \vtrace_0} \rightarrow^{*} 
\config{[\assign{x_1}{\expr_1 / \query(\qexpr_1)}]^{\pi_2(\event_1)} ; {c}_1
\footnote{
$\assign{x}{\expr / \query(\qexpr)}$ denotes variable $x$ is assigned by either an expression $\expr$ or query $\query(\qexpr)$
}, 
\vtrace_1}  
\rightarrow^\rname{assn/query}
 \config{c_1, \vtrace_1 \tracecat [\event_1]} \\
  \qquad \rightarrow^{*} 
  \config{[\assign{{x}_2}{\expr_2 / \query(\qexpr_2)}]^{l_2};{c}_2, 
  \vtrace_1 \tracecat [\event_1]} 
  \rightarrow^\rname{assn/query} 
  \config{{c}_2,  \vtrace_1 \tracecat [\event_1; \event_2]} 
  % 
\end{array}
\end{equation}
%
% \wqside{Some typo in equation 4, but I can follow:-)}
% \jl{thanks}
, where $x_1 = \pi_1(\event_1)$, $l_1 = {\pi_2(\event_1)}$, $x_2 = \pi_1(\event_2)$, $l_2 = \pi_2(\event_2)$, 
and $\expr_1 / \qexpr_1$, $\expr_2 / \qexpr_2$ are the expressions of the assignment commands 
associated to the $\event_1$ and $\event_2$ from  Lemma~\ref{lem:inv_event}.
\\
%
By $\diff(\event_2,\event_2')$ and the command label consistency,
we also have the instance of second execution in Eq.~\ref{eq:eventdep_def_base_val} as follows:
% \[
% \config{{c}_1, \vtrace_1 \tracecat [\event_1']}  \rightarrow^{*} 
%   \config{{c}_2',  \vtrace_1 \tracecat [\event_1']\cdot \vtrace_2' \tracecat [\event_2'] } 
%   \], 
% we know there exists $\expr_2'$ or $\qexpr_2'$ and following execution instance,
%  \[
%   \begin{array}{l}   
%   \config{c_1, \vtrace_1 \tracecat [\event_1']} 
%   \rightarrow^{*} 
%   \config{[\assign{{x}_2'}{\expr_2' / \query(\qexpr_2')}]^{l_2'} ; {c}_2', \vtrace_1 \tracecat [\event_1']\tracecat \vtrace'} 
%   \rightarrow^\rname{assn/query} 
%   \config{{c}_2',   \vtrace_1 \tracecat [\event_1'] \tracecat \vtrace' \tracecat [\event_2']} 
%   % 
% \end{array}
%  \]
%  , where  $x_2' = \pi_1(\event_2')$ and $l_2' = \pi_2(\event_2')$.
% %
% Unfolding $\diff(\event_2,\event_2')$, we have:
% \[
%   x_2 = x_2' \land l_2 = l_2' 
% \] 
% %
% Since each command in $c$ has a unique label, we have $\expr_2' = \expr_2$, $\qexpr_2 = \qexpr_2'$, and following execution instance:
\begin{equation}
\label{eq:valdep_inv2}
  \config{c_1, \vtrace_1 \tracecat [\event_1']} 
  \rightarrow^{*} 
  \config{[\assign{{x}_2}{\expr_2 / \query(\qexpr_2)}]^{l_2} ; {c}_2, \vtrace_1 \tracecat [\event_1']\cdot \vtrace_2'} 
  \rightarrow^\rname{assn/query} 
  \config{{c}_2,  \vtrace_1 \tracecat [\event_1']\cdot \vtrace_2' \tracecat [\event_2']} 
\end{equation}
%
From Eq.~\ref{eq:eventdep_def_base_val}, we also have
\begin{equation}
\label{eq:valdep_invn}
  \vcounter(\vtrace') l_2 = \vcounter( [] ) l_2 = 0
\end{equation}
%
%
% By Induction on the operational semantics rules on following execution from Eq.~\ref{eq:valdep_inv1}: 
% \wqside{Surprised we do induction here:-)}
 %
%  \[\config{c_1, \vtrace_1 \tracecat [\event_1]}
%   \rightarrow^{*} 
%   \config{[\assign{{x}_2}{\expr_2 / \query(\qexpr_2)}]^{l_2};{c}_2, 
%   \vtrace_1 \tracecat [\event_1]} 
% \]
 %
By Inversion Lemma~\ref{lem:inv_skip} and the execution in Eq.~\ref{eq:valdep_inv1}, we know:
 \[
 c_1 =_c 
 [\eskip]{}^*;[\assign{{x}_2}{\expr_2 / \query(\qexpr_2)}]^{l_2};{c}_2
 \]
 %
By substituting $c_1$ in Eq.~\ref{eq:valdep_inv2}, the following subproof shows there is only 1 possible instance of the execution in Eq.~\ref{eq:valdep_inv2}.
\begin{subproof}[Subproof]
\label{pf:noiteration_inv2}
There are two possibilities by the law of excluded middle:
\\
$[\assign{{x}_2}{\expr_2 / \query(\qexpr_2)}]^{l_2} \in_c c_2$ 
\\
or $[\assign{{x}_2}{\expr_2 / \query(\qexpr_2)}]^{l_2} \notin_c c_2$.
%
\begin{enumerate}
\item{$[\assign{{x}_2}{\expr_2 / \query(\qexpr_2)}]^{l_2}\notin_c c_2$}
\\
In this case, we have the following execution instance:
%
\footnote{$\rightarrow^{\rname{skip}^*}$ denotes the rule applied on 
every evaluation step of this execution is the $\rname{skip}$ rule.}
 %
  \[
  \config{c_1, \vtrace_1 \tracecat [\event_1']} 
  \rightarrow^{\rname{skip}^*} 
  \config{[\assign{{x}_2}{\expr_2 / \query(\qexpr_2)}]^{l_2} ; {c}_2, \vtrace_1 \tracecat [\event_1']} 
  \rightarrow^\rname{assn/query} 
  \config{{c}_2,  \vtrace_0 \tracecat \vtrace_1 \tracecat [\event_1'; \event_2']} 
 \]
%
\item{$[\assign{{x}_2}{\expr_2 / \query(\qexpr_2)}]^{l_2} \in_c c_2$}
\\
By Inversion Lemma~\ref{lem:inv_while}, 
we have a $\ewhile$ conditional command
 $(\ewhile [b_w]^l_w \edo c_w)$ in $c_2$, where
% $(\ewhile [b_w]^l_w \edo c_w) \in_c c_2$ and 
$[\assign{{x}_2}{\expr_2 / \query(\qexpr_2)}]^{l_2} \in_c c_w$.
% \\
Then, we have the following possible execution instances
 %
  \[
  \config{c_1, \vtrace_1 \tracecat [\event_1']} 
  \rightarrow^{\rname{skip}^*} 
  \config{[\assign{{x}_2}{\expr_2 / \query(\qexpr_2)}]^{l_2} ; {c}_2, \vtrace_1 \tracecat [\event_1']} 
  \rightarrow^\rname{assn/query} 
  \config{{c}_2,  \vtrace_1 \tracecat [\event_1']\tracecat [\event_2']} 
 \]
%
  \[
  \begin{array}{l}
  \config{c_1, \vtrace_1 \tracecat [\event_1']} 
  \rightarrow^{\rname{skip}^*} 
  \config{[\assign{{x}_2}{\expr_2 / \query(\qexpr_2)}]^{l_2} ; {c}_2, \vtrace_1 \tracecat [\event_1']} 
  \rightarrow^\rname{assn/query} 
  \config{{c}_2,  \vtrace_1 \tracecat [\event_1']\tracecat [(x_2, l_2,  v_2')]} 
  \\ \qquad
  \rightarrow^{*} 
  \config{[\assign{{x}_2}{\expr_2 / \query(\qexpr_2)}]^{l_2} ; {c}_2, 
  \vtrace_1 \tracecat [\event_1']\tracecat [(x_2, l_2,  v_2')] \tracecat \trace_3} 
  \\ \qquad
  \rightarrow^\rname{assn/query} 
  \config{{c}_2,  \vtrace_1 \tracecat [\event_1']\tracecat [(x_2, l_2,  v_2')] \tracecat \trace_3 \tracecat [\event_2']} 
 \end{array}
 \]
\[
  \cdots
\] 
, where each execution instance iterates the conditional command 
$(\ewhile [b_w]^l_w \edo c_w)$ in $c_2$, $0, 1$ or more times.
%
\\
%
For each execution instance, we have the corresponding instance of $\trace'$ as follows:
\\
$\trace'  = [] $
\\
$\trace' = \tracecat [(x_2, l_2,  v_2')] \tracecat \trace_3 $
%
\\
$\cdots$
%
\\
%
By Eq.~\ref{eq:valdep_invn} where $\vcounter(\trace') l_2 = 0$,
%
we know only the first execution instance with 0 iteration of $\ewhile$ command in $c_2$ satisfies this restriction, 
i.e., $\trace' = []$.
%
\end{enumerate}
In conclusion, we have the only qualified execution instance as follows where $\trace' = []$.
  \[
    \config{c_1, \vtrace_1 \tracecat [\event_1']} 
    \rightarrow^{\rname{skip}^*} 
    \config{[\assign{{x}_2}{\expr_2 / \query(\qexpr_2)}]^{l_2} ; {c}_2, \vtrace_1 \tracecat [\event_1']} 
    \rightarrow^\rname{assn/query} 
    \config{{c}_2,  \vtrace_1 \tracecat [\event_1']\tracecat [\event_2']} 
 \]
\end{subproof}
%
Then we know by the environment definition,
$\env$ obtains different values only for variable $x_1$ 
from trace $\vtrace_1 \tracecat [\event_1]$ and 
$\vtrace_1 \tracecat [\event_1']$, i.e.,
\[
  \forall z^r \in \lvar_c \setminus \{x_1^{l_1}\} ,
  \env(\vtrace_1 \tracecat [\event_1]) (z) =  
  \env(\vtrace_1 \tracecat [\event_1']) (z)
\]
%
By {Inversion Lemma~\ref{lem:inv_expr}} of arithmetic expression evaluation, we have
\[
  x_1 \in VAR(\expr_2 / \qexpr_2) 
\]
Since $\llabel(\vtrace_1 \tracecat [\event_1]) x_1 = l_1$, 
by Inversion Lemma~\ref{lem:inv_live} we know $x_1^{l_1} \in \live^{l_2}(c)$.
%
\\
%
By $\flowsto$ definition, we have:
%
\[
\flowsto(x_1^{l_1}, {x}_2^{l_2}, c)
\]
i.e.,
%
\[
\flowsto(\pi_1(\event_1)^{\pi_2(\event_1)}, \pi_1(\event_2)^{\pi_2(\event_2)}, c)
 \]
%
This case is proved.
\end{subproof}
	% \end{case}
	\caseL{2}
		\textbf{(the labelled unique assignment command associated to the $\event_2$ isn't executed in second execution).}
		\\
		Proof Summary:
		\\
		1. Let $\event_b$ be the testing event,
		in the same way of case 1, we get:
		 $\pi_1(\event_1) \in VAR(\pi_1(\event_b)) 
		 \land 
		 l_1 \in \live^{l_b}$
		 %
		 \\
		 2. By Lemma~\ref{lem:ctldep_inv}, we know:
		   $\forall z \in VAR(\pi_1(\event_b)) \st \exists i \in \mathbb{N} \st
		 \flowsto(z^i, \pi_1(\event)^{\pi_2(\event)}, c)$
		 %
		 \\
		 3. By $\flowsto$ definition we have:
		   $\flowsto(\pi_1(\event_1)^{\pi_2(\event_1)}, \pi_1(\event_2)^{\pi_2(\event_2)}, c)$
		\begin{subproof}[Proof of the Basecase: Case 2]
%
\label{pf:eventdep_base_ctl}
We have the following by the definition $\eventdep(\event_1, \event_2, [\event_1; \event_2], c, D)$ of case 2:
\begin{equation}
  \label{eq:eventdep_def_base_ctl}
  \begin{array}{ll}   
    & \exists \vtrace_0,
    \vtrace_1, \vtrace', \vtrace_3, \vtrace_3' \in \mathcal{T},\event_1' \in \eventset^{\asn}, {c}_1, {c}_2  \in \cdom, 
    \event_b \in \eventset^{\test}
    \st
    \\ 
   &   \qquad    \diff(\event_1, \event_1') 
\land
   \Big(
  \config{{c}, \vtrace_0} \rightarrow^{*} 
      \config{{c}_1, \vtrace_1 \tracecat [\event_1]}  \rightarrow^{*} 
      \config{c_2,  \trace_1 \tracecat [\event_1;\event_b] \tracecat  \trace_3} 
    \\   
   & \qquad \bigwedge 
    \config{{c}_1, \vtrace_1 \tracecat [\event_1']}  \rightarrow^{*} 
    \config{c_2,  \vtrace_1 \tracecat [\event_1'] \tracecat \trace' \tracecat [(\neg \event_b)] \tracecat \trace_3'} 
    \\
    & \qquad \bigwedge  \tlabel_{\trace_3} \cap \tlabel_{\trace_3'} = \emptyset
     \land \vcounter(\trace') ~  \pi_2(\event_b) = \vcounter(\trace) ~  \pi_2(\event_b)
      \land \event_2 \eventin \trace_3
    \land \event_2 \not\eventin \trace_3'
   \Big)
 \end{array}
  \end{equation}
  %
%
Let $\vtrace_0,
\vtrace_1, \vtrace', \vtrace_3, \vtrace_3' \in \mathcal{T}, 
\event_2' \in \eventset, \event_1' \in \eventset^{\asn}, \event_b, {c}_1, {c}_2$ be the traces, events and commands satisfying the executions,
by Inversion Lemma~\ref{lem:inv_event} on 
$\event_1$, $\event_2$, and $\event_b$,
we have the following instance of the first execution in Eq.~\ref{eq:eventdep_def_base_ctl},
 %
%
% Let $\event_{ih} = (b, l_b, n_b, v_b)$, by Eq.~\ref{eq:ctldep_inv1} and {Inversion Lemma~\ref{lem:inv_test}}, we have:
\begin{equation}
\label{eq:ctldep_inv1}
  \begin{array}{l}   
\config{{c}, \vtrace_0} \rightarrow^{*} 
\config{[\assign{{x}_1}{\expr_1 / \query(\qexpr_1)}]^{l_1} ; {c}_1, \vtrace_1}  
\rightarrow^{assn/query}
 \config{c_1, \vtrace_1 \tracecat [\event_1]} 
 \\
  \qquad \rightarrow^{*} 
  \config{\eif ([b]^{l_b}, c_t, c_f) / \ewhile [b]^{l_b} \edo c_w;{c}_3', 
  \vtrace_1 \tracecat [\event_1]} 
  \\
  \qquad 
   \rightarrow^{\rname{if-b / while-b}} 
  \config{(c_t;c_3' / c_f;c_3') /(c_3' / c_w; \ewhile [b]^{l_b} \edo c_w;{c}_3'), 
  \trace_1 \tracecat [\event_1;\event_b]} 
  \\
  \qquad   \rightarrow^{*} 
  \config{c_3, 
  \trace_1 \tracecat [\event_1;\event_b] \tracecat  \trace_3}
  % 
\end{array}
\end{equation}
%  %
% \begin{equation}
% \label{eq:ctldep_inv1}
%   \begin{array}{l}   
% \config{{c}, \vtrace_0} \rightarrow^{*} 
% \config{[\assign{x_1}{\expr_1 / \query(\qexpr_1)}]^{\pi_2(\event_1)} ; {c}_1
% \footnote{
% $\assign{x}{\expr / \query(\qexpr)}$ denotes variable $x$ is assigned by either an expression $\expr$ or query $\query(\qexpr)$
% }, 
% \vtrace_1}  
% \rightarrow^\rname{assn/query}
%  \config{c_1, \vtrace_1 \tracecat [\event_1]} \\
%   \qquad \rightarrow^{*} 
%   \config{[\assign{{x}_2}{\expr_2 / \query(\qexpr_2)}]^{l_2};{c}_2, 
%   \vtrace_1 \tracecat [\event_1]} 
%   \rightarrow^\rname{assn/query} 
%   \config{{c}_2,  \vtrace_1 \tracecat [\event_1; \event_2]} 
%   % 
% \end{array}
% \end{equation}
% %
% % \wqside{Some typo in equation 4, but I can follow:-)}
% % \jl{thanks}
, where $x_1 = \pi_1(\event_1)$, $l_1 = {\pi_2(\event_1)}$, 
% $x_2 = \pi_1(\event_2)$, $l_2 = \pi_2(\event_2)$, 
and $\eif ([b]^{l_b}, c_t, c_f) / \ewhile [b]^{l_b} \edo c_w$ 
is the conditional command of the assignment commands 
associated to the $\event_b$ from Inversion Lemma~\ref{lem:inv_event} of testing event.
\\
%
By the command label consistency,
we also have the instance of second execution in Eq.~\ref{eq:eventdep_def_base_ctl} as follows:
% \[
% \config{{c}_1, \vtrace_1 \tracecat [\event_1']}  \rightarrow^{*} 
%   \config{{c}_2',  \vtrace_1 \tracecat [\event_1']\cdot \vtrace_2' \tracecat [\event_2'] } 
%   \], 
% we know there exists $\expr_2'$ or $\qexpr_2'$ and following execution instance,
%  \[
%   \begin{array}{l}   
%   \config{c_1, \vtrace_1 \tracecat [\event_1']} 
%   \rightarrow^{*} 
%   \config{[\assign{{x}_2'}{\expr_2' / \query(\qexpr_2')}]^{l_2'} ; {c}_2', \vtrace_1 \tracecat [\event_1']\tracecat \vtrace'} 
%   \rightarrow^\rname{assn/query} 
%   \config{{c}_2',   \vtrace_1 \tracecat [\event_1'] \tracecat \vtrace' \tracecat [\event_2']} 
%   % 
% \end{array}
%  \]
%  , where  $x_2' = \pi_1(\event_2')$ and $l_2' = \pi_2(\event_2')$.
% %
% Unfolding $\diff(\event_2,\event_2')$, we have:
% \[
%   x_2 = x_2' \land l_2 = l_2' 
% \] 
% %
% Since each command in $c$ has a unique label, we have $\expr_2' = \expr_2$, $\qexpr_2 = \qexpr_2'$, and following execution instance:
\begin{equation}
\label{eq:ctldep_inv2}
\begin{array}{l}   
  \config{{c}, \vtrace_0} \rightarrow^{*} 
  \config{[\assign{{x}_1}{\expr_1 / \query(\qexpr_1)}]^{l_1} ; {c}_1, \vtrace_1}  
  \rightarrow^{assn/query}
   \config{c_1, \vtrace_1 \tracecat [\event_1]} 
   \\
    \qquad \rightarrow^{*} 
    \config{\eif ([b]^{l_b}, c_t, c_f) / \ewhile [b]^{l_b} \edo c_w;{c}_3', 
    \vtrace_1 \tracecat [\event_1] \tracecat \trace'} 
    \\
    \qquad 
     \rightarrow^{\rname{if-b / while-b}} 
    \config{(c_f;c_3' / c_t;c_3') /(c_w; \ewhile [b]^{l_b} \edo c_w;{c}_3' / c_3'), 
    \trace_1 \tracecat [\event_1]  \tracecat \trace' \tracecat [\neg \event_b]} 
    \\
    \qquad   \rightarrow^{*} 
    \config{c_3, 
    \trace_1 \tracecat [\event_1]  \tracecat \trace' \tracecat [\neg \event_b] \tracecat  \trace_3'}
    % 
  \end{array}
\end{equation}
%
From Eq.~\ref{eq:eventdep_def_base_ctl}, we also have
  $\vcounter(\vtrace') l_b = \vcounter( [] ) l_b = 0$.
\\
%
%
By the same proof steps from case 1 in Subproof~\ref{pf:eventdep_base_val}, we have
\[
  x_1 \in VAR(b)  \land x_1^{l_1} \in \live^{l_b}(c)
\]
%
By Lemma~\ref{lem:ctldep_inv}, we also know:
\[
  \forall z \in VAR(\pi_1(\event_b)) \st \exists i \in \mathbb{N} \st
\flowsto(z^i, \pi_1(\event)^{\pi_2(\event)}, c)
\]
%
Then by $\flowsto$ definition, we have $\flowsto(x_1^{l_1}, {x}_2^{l_2}, c)$
%
i.e.,
%
\[
\flowsto(\pi_1(\event_1)^{\pi_2(\event_1)}, \pi_1(\event_2)^{\pi_2(\event_2)}, c)
 \]
%
This case is proved.
\end{subproof}
\end{proof}
%
\begin{lem}[Control Dependency Inversion]
	\label{lem:ctldep_inv}
	For all $c \in \cdom$, $D \in \dbdom, \event_b \in \eventset^{\test}, \event \in \eventset^{\asn} $, if 
	$\eventdep^{\ctl}(\event_b, \event, c, D)$, 
	then for all  $z \in VAR(\pi_1(\event_b))$ there exists a label $i \in \mathbb{N}$ such that 
	$\flowsto(z^i, \pi_1(\event)^{\pi_2(\event)}, c)$
	\begin{equation}
		\label{eq:ctlflowsto_inv},		
		\begin{array}{l}
			\forall D \in \dbdom , c \in \cdom, \trace \in \mathcal{T},
			\event_1, \event_2 \in \eventset^{\asn} \st 
			\\ 
			\exists \vtrace_0,
			\vtrace_1, \vtrace', \vtrace_3, \vtrace_3' \in \mathcal{T},\event_1' \in \eventset^{\asn}, {c}_1, {c}_2  \in \cdom, 
			\event_b \in \eventset^{\test},
			\trace_{ih} \in \mathcal{T} \st 
		\trace = [\event_1] \tracecat \trace_{ih} \tracecat [\event_2]
		\\ \quad \implies	  
			  \config{{c}, \vtrace_0} \rightarrow^{*} 
				\config{{c}_1, \vtrace_1 \tracecat [\event_1]}  \rightarrow^{*} 
				\config{c_2,  \vtrace_1 \tracecat [\event_1] \tracecat \trace \tracecat [\event_b] \tracecat  \trace_3} 
			  \\ \qquad \land
			  \config{{c}_1, \vtrace_1 \tracecat [\event_1']}  \rightarrow^{*} 
			  \config{c_2,  \vtrace_1 \tracecat [\event_1'] \tracecat \trace' \tracecat [(\neg \event_b)] \tracecat \trace_3'} 
			  \\ \qquad \land
			\tlabel_{\trace_3} \cap \tlabel_{\trace_3'} = \emptyset
			   \land \vcounter(\trace') ~  \pi_2(\event_b) = \vcounter(\trace) ~  \pi_2(\event_b)
				\land \event_2 \eventin \trace_3
			  \land \event_2 \not\eventin \trace_3'
		\\ \quad \implies	
		\forall z \in VAR(\pi_1(\event_b)) \st 
		\exists l \in \mathbb{N} \st 
		\flowsto(z^l, \pi_1(\event_2)^{\pi_2(\event_2)},c)
	\end{array}
\end{equation}
	\end{lem}
	Proof Summary:
	\\
	Proving by using the Inversion Lemmas~\ref{lem:inv_expr}, \ref{lem:inv_expr_gnl}, 
	\ref{lem:inv_event}, and \ref{lem:inv_live}, and control dependency definition.
%
	\begin{proof}
		Take arbitrary $D \in \dbdom , c \in \cdom, \trace \in \mathcal{T},
		\event_1, \event_2 \in \eventset^{\asn} $,
%
let $\vtrace_0,
\vtrace_1, \vtrace', \vtrace_3, \vtrace_3' \in \mathcal{T}, 
\event_2' \in \eventset, \event_1' \in \eventset^{\asn}, \event_b, {c}_1, {c}_2$ be the traces, 
events and commands satisfying the executions,
by Inversion Lemma~\ref{lem:inv_event} on 
$\event_2$, and $\event_b$,
we have the following instance of the first execution in Eq.~\ref{eq:ctlflowsto_inv},
 %
%
% Let $\event_{ih} = (b, l_b, n_b, v_b)$, by Eq.~\ref{eq:ctldep_inv1} and {Inversion Lemma~\ref{lem:inv_test}}, we have:
\begin{equation}
% \label{eq:ctldep_inv1}
  \begin{array}{l}   
\config{{c}, \vtrace_0} \rightarrow^{*} 
% \config{[\assign{{x}_1}{\expr_1 / \query(\qexpr_1)}]^{l_1} ; {c}_1, \vtrace_1}  
% \rightarrow^{\rname{assn/query}}
%  \config{c_1, \vtrace_1 \tracecat [\event_1]} 
%  \\ \qquad 
%  \rightarrow^{*} 
  \config{\eif ([b]^{l_b}, c_t, c_f) / \ewhile [b]^{l_b} \edo c_w;{c}_3', 
  \vtrace_1 \tracecat [\event_1] \tracecat \trace} 
  \\
  \qquad 
   \rightarrow^{\rname{if-b / while-b}} 
  \config{(c_t;c_3' / c_f;c_3') /(c_w; \ewhile [b]^{l_b} \edo c_w;c_3'/[\eskip]; c_3'), 
  \trace_1 \tracecat [\event_1] \tracecat \trace \tracecat [\event_b]} 
  \\
  \qquad  \rightarrow^{*} 
  \config{[\assign{{x}_2}{\expr_2 / \query(\qexpr_2)}]^{l_2}; c_{3b}', 
  \trace_1 \tracecat [\event_1] \tracecat \trace \tracecat [\event_b] \tracecat  \trace_{3a}}
  \\ \qquad \rightarrow^{\rname{assn/query}}
  \config{ c_{3b}', 
  \trace_1 \tracecat [\event_1] \tracecat \trace \tracecat [\event_b] \tracecat  \trace_{3a} \tracecat [\event_2]}
  \rightarrow^{*} 
  \config{c_3, 
  \trace_1 \tracecat [\event_1] \tracecat \trace \tracecat [\event_b] \tracecat  \trace_{3a} \tracecat [\event_2] \tracecat \trace_{3b}}
  % 
\end{array}
\end{equation}
%  %
, where $\trace_3 = \trace_{3a} \tracecat [\event_2] \tracecat \trace_{3b}$,
% $x_1 = \pi_1(\event_1)$, $l_1 = {\pi_2(\event_1)}$, 
$x_2 = \pi_1(\event_2)$, $l_2 = \pi_2(\event_2)$, 
and $\eif ([b]^{l_b}, c_t, c_f) / \ewhile [b]^{l_b} \edo c_w$ 
is the conditional command of the assignment commands associated to the 
$\event_b$ from Inversion Lemma~\ref{lem:inv_event} of testing event.
\\
% \[\begin{array}{l}   
% 	  \config{\eif ([b]^{l_b}, c_t, c_f) / \ewhile [b]^{l_b} \edo c_w;{c}_3', 
% 	  \vtrace_1 \tracecat [\event_1] \tracecat \trace} 
% 	  \\
% 	  \qquad 
% 	   \rightarrow^{\rname{if-b / while-b}} 
% 	  \config{(c_t;c_3' / c_f;c_3') /(c_3' / c_w; \ewhile [b]^{l_b} \edo c_w;{c}_3'), 
% 	  \trace_1 \tracecat [\event_1] \tracecat \trace \tracecat [\event_b]} 
% 	  % 
% 	\end{array}
% 	\]
% $\config{\eif ([b]^{l_b}, c_t, c_f) / \ewhile [b]^{l_b} \edo c_w;{c}_3', 
% \vtrace_1 \tracecat [\event_1] \tracecat \trace} 
%  \rightarrow^{\rname{if-b / while-b}} 
% \config{(c_t;c_3' / c_f;c_3') /(c_3' / c_w; \ewhile [b]^{l_b} \edo c_w;{c}_3'), 
% \trace_1 \tracecat [\event_1] \tracecat \trace \tracecat [\event_b]} 
% $
The notation $(c_t;c_3' / c_f;c_3') /(c_w; \ewhile [b]^{l_b} \edo c_w;c_3' / [\eskip]; c_3')$ represents:
\\
In case of $\eif ([b]^{l_b}, c_t, c_f)$, if $\pi_3(\event_b) = \etrue$, we have the evaluation:
$$
\config{\eif ([b]^{l_b}, c_t, c_f) ;{c}_3', 
\vtrace_1 \tracecat [\event_1] \tracecat \trace} 
 \rightarrow^{\rname{if-b}} 
\config{c_t;c_3' 
\trace_1 \tracecat [\event_1] \tracecat \trace \tracecat [\event_b]} 
$$
%
The same for case of $\eif ([b]^{l_b}, c_t, c_f)$ with $\pi_3(\event_b) = \efalse$,
and case of $\ewhile [b]^{l_b} \edo c_w$ with $\pi_3(\event_b) = \etrue$ and $\pi_3(\event_b) = \efalse$.
%
\\
By the command label consistency,
we also have the instance of second execution as follows:
\begin{equation}
\label{eq:ctldep_inv2}
\begin{array}{l}   
  \config{{c}, \vtrace_0} \rightarrow^{*} 
%   \config{[\assign{{x}_1}{\expr_1 / \query(\qexpr_1)}]^{l_1} ; {c}_1, \vtrace_1}  
%   \rightarrow^{\rname{\rname{assn/query}}}
%    \config{c_1, \vtrace_1 \tracecat [\event_1]} 
%    \\
%     \qquad \rightarrow^{*} 
    \config{\eif ([b]^{l_b}, c_t, c_f) / \ewhile [b]^{l_b} \edo c_w;{c}_3', 
    \vtrace_1 \tracecat [\event_1] \tracecat \trace'} 
    \\
    \qquad 
     \rightarrow^{\rname{if-b / while-b}} 
    \config{(c_f;c_3' / c_t;c_3') /([\eskip]; c_3' / c_w; \ewhile [b]^{l_b} \edo c_w;{c}_3' ), 
    \trace_1 \tracecat [\event_1]  \tracecat \trace' \tracecat [\neg \event_b]} 
    \\
    \qquad   \rightarrow^{*} 
    \config{c_3, 
    \trace_1 \tracecat [\event_1]  \tracecat \trace' \tracecat [\neg \event_b] \tracecat  \trace_3'}
    % 
  \end{array}
\end{equation}
%
By the label consistency, and $\tlabel_{\trace_3} \cap \tlabel_{\trace_3'} = \emptyset$, 
% i.e., 
% $ \trace_{3a} \tracecat [\event_2] \tracecat \trace_{3b} \cap \tlabel_{\trace_3'} = \emptyset$
we know $\trace_3$ and $\trace_3'$ doesn't contain any event of evaluating the commands in $c_3'$.
Otherwise, $\tlabel_{\trace_3} \cap \tlabel_{\trace_3'} \neq \emptyset$, which is a contradiction.
\\
Since $\trace_3= \trace_{3a} \tracecat [\event_2] \tracecat \trace_{3b} $, we know $\event_2$ doesn't comes from evaluating
of $c_3'$, i.e.,:
\\
In the case of $\eif ([b]^{l_b}, c_t, c_f)$, $\event_2$ comes from the evaluation of $c_t$ or $c_f$,
i.e., $[\assign{{x}_2}{\expr_2 / \query(\qexpr_2)}]^{l_2} \in_c c_t$ or $c_f$;
\\
and in the case of $\ewhile [b]^{l_b} \edo c_w$, $\event_2$ comes from the evaluation of $c_w$,
i.e., $[\assign{{x}_2}{\expr_2 / \query(\qexpr_2)}]^{l_2} \in_c c_w$.
\\
In both of the two cases, we know $\forall z \in VAR(\pi_1(\event_b)) $ there is a label $l \in \mathbb{N}$ for this variable,
and by the $\flowsto$ definition, 
$\flowsto(z^l, \pi_1(\event_2)^{\pi_2(\event_2)},c)$.
\\
This lemma is proved.
	\end{proof}
	%
	% \begin{lem}[One Step Dependency Inversion]
	% 	\label{lem:onestepdep_inv}
	% For all $ c \in \cdom, D \in \dbdom, x^i \in \lvar_c$, and $\event_y \in \eventset^{\asn}$, 
	% if $x^i \in VAR(\expr_y)$, 
	% or there exists $\event_b \in \eventset^{\test}$ such that 
	% $x^i \in VAR(\pi_1(\event_b)$ and 
	% $\eventdep^{\ctl}(\event_b, \event_y, c, D)$, then $\flowsto(x^i, \pi_1(\event_y)^{\pi_2(\event_y)}, c)$.
	% %
	% 	\[
	% 	\begin{array}{l}
	% 		\forall c \in \cdom, D \in \dbdom, x^i \in \lvar_c, \event_y \in \eventset^{\asn}
	% 		\st
	% 		\\ \quad
	% 		(x^i \in VAR(\expr_y)\lor 
	% 		(\exists \event_b \in \eventset^{\test} \st x^i \in VAR(\pi_1(\event_b)) 
	% 		\land \eventdep^{\ctl}(\event_b, \event_y, c, D)))
	% 		\implies \flowsto(x^i, \pi_1(\event_y)^{\pi_2(\event_y)}, c)
	% 	\end{array}
	% \]
	% \end{lem}
	% \begin{proof}
	% 	proving by using the Inversion Lemmas~\ref{lem:inv_expr_gnl}, ~\ref{lem:inv_expr},
	% 	\ref{lem:inv_event}, and \ref{lem:inv_live}, 
	% 	and Control Dependency Inversion Lemmas~\ref{lem:ctldep_inv}.
	% \end{proof}
	%
	%
\begin{lem}[The Multiple-Steps Event Dependency Inversion]
	\label{lem:depevents_exist}
For every $D \in \dbdom , c \in \cdom, \trace \in \mathcal{T}$, and two assignment events 
$\event_1, \event_2 \in \eventset^{\asn}$,
if the trace $trace$ has the form $\trace = [\event_1] \tracecat \trace' \tracecat [\event_2]$ with $\trace' \in \mathcal{T}$, and $\eventdep(\event_1, \event_2, \trace, c, D)$
then $\flowsto(\pi_1(\event_1)^{\pi_2(\event_1)}, \pi_1(\event_2)^{\pi_2(\event_2)}, c) $,
or otherwise there exists
$\event \in \trace'$ such that
$\left( 		
   \eventdep(\event_1, \event, \trace[\event_1:\event], c, D)
\land 
\flowsto(\pi_1(\event)^{\pi_2(\event)}, \pi_1(\event_2)^{\pi_2(\event_2)}, c) 
\right)$.
%
	\[
	\begin{array}{l}
		\forall D \in \dbdom , c \in \cdom, \trace \in \mathcal{T} \st \forall \event_1, \event_2 \in \eventset^{\asn} \st
		 \exists \trace' \in \mathcal{T} \st \trace = [\event_1] \tracecat \trace' \tracecat [\event_2]
		\implies
		\eventdep(\event_1, \event_2, \trace, c, D) 
		\\ \quad 
		\implies 
		\flowsto(\pi_1(\event_1)^{\pi_2(\event_1)}, \pi_1(\event_2)^{\pi_2(\event_2)}, c) 
		\\ \qquad \quad \lor
		\exists \event \in \trace' \st 
		\left( 		
			\eventdep(\event_1, \event, \trace[\event_1:\event], c, D)
		\land 
		\flowsto(\pi_1(\event)^{\pi_2(\event)}, \pi_1(\event_2)^{\pi_2(\event_2)}, c) 
	\right) 
		% \\ \qquad \qquad \lor
		% \flowsto(\pi_1(\event_1)^{\pi_2(\event_1)}, \pi_1(\event_2)^{\pi_2(\event_2)}, c) 
	\end{array}
	\]
\end{lem}
Proof Summary: 
\\
Proving by using Lemma~\ref{lem:inv_indepevents}, Lemma~\ref{lem:ctldep_inv}, and the Inversion Lemmas~\ref{lem:inv_expr}, \ref{lem:inv_expr_gnl},
\ref{lem:inv_event}, and \ref{lem:inv_live}
and showing a contradiction.
\begin{proof}
	Taking arbitrary 
	$ D \in \dbdom , c \in \cdom, \trace \in \mathcal{T} $ and two events 
	$\event_1, \event_2 \in \eventset^{\asn}$, where $\trace$ has the form 
	$\trace = [\event_1] \tracecat \trace' \tracecat [\event_2]$ 
	for some 
	$\trace' \in \mathcal{T}$ and $\eventdep(\event_1, \event_2, \trace, c, D)$
	\\ 
	Assume 
	\[
		\begin{array}{l}
	\neg \flowsto(\pi_1(\event_1)^{\pi_2(\event_1)}, \pi_1(\event_2)^{\pi_2(\event_2)}, c) ~ (1)
	\\ \quad 
	\land 
	\forall \event \in \trace' \st 
	\left( 		
		\neg \eventdep(\event_1, \event, \trace[\event_1:\event], c, D)
	\lor 
		\neg \flowsto(\pi_1(\event)^{\pi_2(\event)}, \pi_1(\event_2)^{\pi_2(\event_2)}, c) 
	\right) ~ (2)
	\end{array}
	\]
	Then, by Lemma~\ref{lem:inv_indepevents} and $(2)$, we know 
	$$\flowsto(\pi_1(\event_1)^{\pi_2(\event_1)}, \pi_1(\event_2)^{\pi_2(\event_2)}, c)$$
	, which is contradict to $(1)$.
	\\
	This Lemma is proved.
\end{proof}
%
%
%
%
\begin{lem}[Independent Events Doesn't Block $\flowsto$ ]
		\label{lem:inv_indepevents}
		For every $D \in \dbdom , c \in \cdom, \trace \in \mathcal{T}$, one assignment events 
		$\event_1\in \eventset^{\asn}$, and another event $\event_2 \in \eventset$,
		if the trace $\trace$ has the form $\trace = [\event_1] \tracecat \trace' \tracecat [\event_2]$ with $\trace' \in \mathcal{T}$, 
		and $\eventdep(\event_1, \event_2, \trace, c, D)$,
		then the following two conclusions hold when $\event_2$ is an assignment event and a testing event respectively.
	\begin{itemize}
		\item
		If $\event_2 \in \eventset^{\asn}$,
		% For every $D \in \dbdom , c \in \cdom, \trace \in \mathcal{T}$, and two assignment events 
		% $\event_1, \event_2 \in \eventset^{\asn}$,
		% if the trace $\trace$ has the form $\trace = [\event_1] \tracecat \trace' \tracecat [\event_2]$ with $\trace' \in \mathcal{T}$, and $\eventdep(\event_1, \event_2, \trace, c, D)$,
		then for every $\event \in \trace'$, if it either doesn't have the \emph{Event May-Dependency} relation on $\event_1$, 
		or $\pi_1(\event)^{\pi_2(\event)}$ doesn't have the $\flowsto$ relation with $ \pi_1(\event_2)^{\pi_2(\event_2)}$,
		then the labelled variable $\pi_1(\event_1)^{\pi_2(\event_1)}$ directly flows to the other one $\pi_1(\event_2)^{\pi_2(\event_2)}$, 
		i.e., $\flowsto(\pi_1(\event_1)^{\pi_2(\event_1)}, \pi_1(\event_2)^{\pi_2(\event_2)}, c)$.
		%
		\[
		\begin{array}{l}
			\forall D \in \dbdom , c \in \cdom, \trace \in \mathcal{T} \st \forall \event_1, \event_2 \in \eventset^{\asn} \st
			 \exists \trace' \in \mathcal{T} \st \trace = [\event_1] \tracecat \trace' \tracecat [\event_2]
			\implies
			\eventdep(\event_1, \event_2, \trace, c, D) 
			\\ \quad 
			\implies 
			\left( \forall \event \in \trace' \st \neg \eventdep(\event_1, \event, \trace[\event_1:\event], c, D)
			\lor \neg \flowsto(\pi_1(\event)^{\pi_2(\event)}, \pi_1(\event_2)^{\pi_2(\event_2)}, c) 
			\right) 
			\\ \quad 
			\implies 
			\flowsto(\pi_1(\event_1)^{\pi_2(\event_1)}, \pi_1(\event_2)^{\pi_2(\event_2)}, c)
		\end{array}
		\]
		\item
If $\event_2 \in \eventset^{\test}$, 
then for every $\event \in \trace'$, if it either doesn't have the \emph{Event May-Dependency} relations on $\event_1$,
or $\pi_1(\event) \notin VAR(\pi_1(\event_2)) $,
then 
$\pi_1(\event_1) \in VAR(\pi_1(\event_2))$, and $ {\pi_2(\event_1)} = \llabel(\trace)$
%
\[
\begin{array}{l}
	\forall D \in \dbdom , c \in \cdom, \trace \in \mathcal{T} \st \forall \event_1,\in \eventset^{\asn}, \event_2 \in \eventset^{\test} \st
	 \exists \trace' \in \mathcal{T} \st \trace = [\event_1] \tracecat \trace' \tracecat [\event_2]
	\implies
	\eventdep(\event_1, \event_2, \trace, c, D) 
	\\ \quad 
	\implies 
	\left( \forall \event \in \trace' \st 
	\neg \eventdep(\event_1, \event, \trace[\event_1:\event], c, D)
	\lor  \pi_1(\event) \notin VAR(\pi_1(\event_2))
	\right) 
	\\ \quad 
	\implies 
	\pi_1(\event_1) \in VAR(\pi_1(\event_2)) \land {\pi_2(\event_1)} = \llabel(\trace)
\end{array}
\]
\end{itemize}
\end{lem}
%
\begin{proof}
Taking arbitrary $D \in \dbdom , c \in \cdom$, and an assignment events $\event_1 \in \eventset^{\asn}$ and another event 
$\event_2\in \eventset$.
\\
Without loss of generalization, 
taking arbitrary trace has the form $\trace = [\event_1; \cdots; \event_2]$ for arbitrary $\trace_2 \in \mathcal{T}$,
 then we know $\exists \trace' \in \mathcal{T} \st \trace = [\event_1] \tracecat \trace' \tracecat [\event_2]$, let $\trace_2$ be this $\trace'$.
%
\caseL{$\event_2 \in \eventset^{\asn}$}
%
By the definition of $\eventdep(\event_1, \event_2, \trace, c, D)$, 
taking $ \event_1', \event_2' \in \eventset^{\asn},
\trace_2' \in \mathcal{T}, c_1, c_2 \in \cdom$ as the events, traces and commands satisfying the definition,
 we have following two executions:
% \[
%   \exists \event_1', \event_2' \in \eventset^{\asn},
%   \trace_2' \in \mathcal{T}, c_1, c_2 \in \cdom \st
% \]
%
\[
\begin{array}{l}
\config{c, \trace_0} \rightarrow^{*}
\config{c_1, \trace_1 \tracecat [\event_1]} \rightarrow^{*} \config{c_2, \trace_1 \tracecat [\event_1] \tracecat \trace_2 \tracecat [\event_2]} 
\\ \quad
% \land
\config{c_1, \trace_1 \tracecat [\event_1']} \rightarrow^{*} \config{c_2, \trace_1 \tracecat [\event_1'] \tracecat \trace_2' \tracecat [\event_2']} 
\end{array}
\]
%
%
By inversion Lemma.~\ref{lem:inv_event} on $\event_2$ and $\event_2'$ in the two executions
and $\diff(\event_2, \event_2)$,
 we have the following two execution instances:
\[
\config{c_1, \trace_1 \tracecat [\event_1]} \rightarrow^{*} \config{[\assign{\pi_1(\event_2)}{\expr_2 / \query(\qexpr_2)}]{}^{\pi_2(\event_2)};c_2, \trace_1 \tracecat [\event_1] \tracecat \trace_2} 
\rightarrow^\rname{asn / query} \config{c_2, \trace_1 \tracecat [\event_1] \tracecat \trace_2 \tracecat [\event_2]}  
\]
%
\[
\config{c_1, \trace_1 \tracecat [\event_1']} \rightarrow^{*} \config{[\assign{\pi_1(\event_2)}{\expr_2 / \query(\qexpr_2)}]{}^{\pi_2(\event_2)};c_2, \trace_1 \tracecat [\event_1'] \tracecat \trace_2'} 
\rightarrow^\rname{asn / query} \config{c_2, \trace_1 \tracecat [\event_1'] \tracecat \trace_2' \tracecat [\event_2']}  
\]
, where $\expr_2 / \qexpr_2$ is the expression of the assignment command associated to the $\event_2$ and $\event_2'$ by the Inversion Lemma.~\ref{lem:inv_event}.
\\
Taking arbitrary $\event_z \in \trace_2$, we know 
$\neg \eventdep(\event_1, \event, \trace[\event_1:\event_z], c, D)
\lor  \pi_1(\event_z) \notin VAR(\expr_2 / \qexpr_2)$.
\\
In case of $\neg \eventdep(\event_1, \event, \trace[\event_1:\event_z], c, D)$,
by Definition~\ref{def:event_dep}, we know $\event_z \in \trace_2'$ and 
\[
	\env(\trace_1 \tracecat \trace[\event_1:\event_z]) \pi_1(\event_z) = \env(\trace_1 \tracecat \trace[\event_1':\event_z]) \pi_1(\event_z)
	\]
%
In case of $ \pi_1(\event_z) \notin VAR(\expr_2 / \qexpr_2)$, by Inversion Lemma~\ref{lem:inv_expr}
 of arithmetic and query expression cases, we know:
%
\[
	\forall x^i \in \lvar, \trace, \trace' \in \mathcal{T}, v, v' \st
	\Big( \forall z^j \in \lvar / \{\pi_1(\event_z)^{\pi_2(\event_z)} \} \st 
	\env(\trace) z = \env(\trace') z \Big) \land 
	\config{\trace, \expr_2 / \qexpr_2} \aarrow v \land \config{\trace', \expr_2} \aarrow  v' \implies v = v'
	\]
	\[
		\forall x^i \in \lvar, \trace, \trace' \in \mathcal{T}, \qval, \qval' \st
		\Big( \forall z^j \in \lvar / \{\pi_1(\event_z)^{\pi_2(\event_z)} \} \st 
		\env(\trace) z = \env(\trace') z \Big) \land 
		\config{\trace, \qexpr_2} \qarrow \qval \land \config{\trace', \qexpr_2} \qarrow \qval' \implies \qval =_{q} \qval'
		\]
for $\expr_2$ or $\qexpr_2$ respectively.
\\
Let $\kw{use}_{\trace_2}$ a subset of the events in $\trace_2$, satisfying: 
\[
	\begin{array}{l}
		\forall \event \in \eventset^{\asn} \st 
	\event \in \kw{use}_{\trace_2} \Longleftrightarrow 
	\event \in \trace_2 \land
	\pi_1(\event) \in VAR(\expr_2 / \qexpr_2)
\end{array}		
\]
Then we also know for every $\event_z \in \kw{use}_{\trace_2}$, 
$\neg \eventdep(\event_1, \event_z, \trace[\event_1:\event_z], c, D)$, i.e.,:
\[
	\forall z^l \in \lvar \setminus 
	\big( 
		( \lvar_{\trace_2} \setminus \lvar_{\kw{use}_{\trace_2}}) \cup \{\pi_1(\event_1)^{\pi_2(\event_1)}\} \big)
	\st
	\env(\trace_1 \tracecat [\event_1] \tracecat \trace_2) z = \env(\trace_1 \tracecat [\event_1'] \tracecat \trace_2') z
	~ (1)
\]
 and
 \\ 
$
	\forall z^l \in \lvar \setminus ( \lvar_{\trace_2} \setminus \lvar_{\kw{use}_{\trace_2}}), 
	\trace, \trace' \in \mathcal{T}, v, v' \st 
	\env(\trace) z = \env(\trace') z 
	\land 
	\config{\trace, \expr_2} \aarrow v 
	\land 
	\config{\trace', \expr_2} \aarrow v'
	\implies 
	v = v' 
	~ (2a)
$;
\\
$
	\forall z^l \in \lvar \setminus ( \lvar_{\trace_2} \setminus \lvar_{\kw{use}_{\trace_2}}), 
	\trace, \trace' \in \mathcal{T}, \qval, \qval' \st 
	\env(\trace) z = \env(\trace') z 
	\land 
	\config{\trace, \qexpr_2} \qarrow \qval 
	\land 
	\config{\trace', \qexpr_2} \qarrow \qval'
	\implies 
	\qval =_q \qval' 
	~ (2q)
$,
\\
where $\lvar_{\trace_2}$ and $\lvar_{\kw{use}_{\trace_2}}$ are
 the sets of labelled variables of every event in $\trace_2$ and $\kw{use}_{\trace_2}$ respectively .
% \\
% and 
% $
% 	\forall\event_z \in \lvar/\kw{diff}_{\eventset}, 
% 	\trace, \trace' \in \mathcal{T}, v, v' \st 
% 	\env(\trace) \pi_1(\event_z) = \env(\trace') \pi_1(\event_z) 
% 	\land 
% 	\config{\trace, \expr_2} \aarrow v 
% 	\land 
% 	\config{\trace', \expr_2} \aarrow v'
% 	\implies 
% 	v = v' 
% $
% \\
% By the label consistency, we know 
% \[
% 	\forall \event_z \in \kw{use}_{\trace_2} \st
% 	\env(\trace_1 \tracecat [\event_1] \tracecat \trace_2) \pi_1(\event_z) 
% 	= \env(\trace_1 \tracecat [\event_1'] \tracecat \trace_2']) \pi_1(\event_z)
% \]
% \[
% 	\forall\event_z \in \trace_2 \setminus \kw{use}_{\eventset}, 
% 	\trace, \trace' \in \mathcal{T}, v, v' \st 
% 	% \env(\trace) \pi_1(\event_z) = \env(\trace') \pi_1(\event_z) 
% 	% \land 
% 	\config{\trace, \expr_2} \aarrow v 
% 	\land 
% 	\config{\trace', \expr_2} \aarrow v'
% 	\implies 
% 	v = v' 
% \]
% Let $\diff_{\eventset}$ be a subset of the events in $\trace_2$, satisfying: 
% % \todo{refine the notation}
% \[
% 	\begin{array}{l}
% 		\forall \event_z \in \eventset^{\asn} \st 
% 	\event_z \in \diff_{\eventset} \Leftrightarrow 
% 	\exists \trace_2^h, \trace'^h_2, \trace_2^t, \trace'^t_2, \event_z' \in \trace_2' \st 
% 	\trace_2 = \trace_2^h \tracecat [\event_z] \tracecat \trace_2^t
% 	\\ \quad
% 	\land 
% 	\trace_2' = \trace'^h_2 \tracecat [\event_z'] \tracecat \trace'^t_2
% 	\land 
% 	\diff(\event_z, \event_z')
% 	\land 
% 	\vcounter(\trace_2^h) \pi_1(\event_z) = \vcounter(\trace'^h_2)(\event_z)
% \end{array}		
% \]
%
% Then we know for all $\event_z \in  \diff_{\eventset}$,
%  $\eventdep(\event_1, \event_z, \trace[\event_1:\event_z], c, D)$;
% \\
% and $\forall z^j \in (\lvar \setminus (\mathbb{LV}_{\diff_{\eventset}} \cup\{\pi_1(\event_1)^{\pi_2(\event_2)}\}) ) \st 
% \env(\trace_1 \tracecat [\event_1] \tracecat \trace_2) z = \env(\trace_1 \tracecat [\event_1'] \tracecat \trace_2') z $,
% \\
% where $\mathbb{LV}_{\diff_{\eventset}}$ is the set of labelled variables of every event in $\diff_{\eventset}$.
% We also know for all $\event_z \in  \diff_{\eventset}$, $\pi_1(\event_z) \notin VAR(\pi_1(\event_2))$ and follows:
% $\neg \eventdep(\event_z, \event_2, \trace[\event_z:\event_2], c, D) $.
% \\
% Then by $\eventdep$ definition, we know 
% $\forall \event_z', \event_2'' \in \eventset^{\asn}, \trace_z',\trace_z,  \trace_1' \in \mathcal{T}, c_z \in \cdom$ satisfying following two executions, we have $\event_2 \eventeq \event_2''$.
% \[
% 	\begin{array}{l}
% 		\config{c, \trace_0} \rightarrow^{*}
% 		\config{c_z, \trace_1' \tracecat [\event_z]} \rightarrow^{*} \config{c_2, \trace_1' \tracecat [\event_z] \tracecat \trace_z \tracecat [\event_2]} 
% 		\\ \quad
% 		\land
% 		\config{c_z, \trace_1' \tracecat [\event_z']} \rightarrow^{*} \config{c_2, \trace_1' \tracecat [\event_z'] \tracecat \trace_z' \tracecat [\event_2']'} 
% 		\end{array}		
% \]
%
% Then we know 
% \todo{type correctness}
% \[
% 	\forall z^j \in (\lvar \setminus \mathbb{LV}_{\diff_{\eventset}} ), \trace, \trace' \in \mathcal{T}, v, v' \st 
% 	\env(\trace) z = \env(\trace') z 
% 	\land 
% 	\config{\trace, \expr_2} \aarrow v 
% 	\land 
% 	\config{\trace', \expr_2} \aarrow v'
% 	\land 
% 	v = v' 
% \]
%
Since $\diff(\event_2, \event_2')$, we also know:
%
\[	
\config{\trace_1 \tracecat [\event_1] \tracecat \trace_2, \expr_2} \aarrow \pi_3(\event_2)
\land 
\config{\trace_1 \tracecat [\event_1'] \tracecat \trace_2', \expr_2} \aarrow \pi_3(\event_2') 
\land 
\pi_3(\event_2) \neq \pi_3(\event_2')
\]
%
% By construction of $\diff_{\eventset}$, we also have:
% \\
% $\forall z^j \in (\lvar \setminus \diff_{\eventset} \cup\{\event_1\} ) \st 
% \env(\trace_1 \tracecat [\event_1] \tracecat \trace_2) z = \env(\trace_1 \tracecat [\event_1'] \tracecat \trace_2') z $.
% \\
% By $\forall \event \in (\trace[\event_1:\event_y]\setminus \{\event_1, \event_y\}) \st
%   \neg \eventdep (\event_1, \event, c, D) \lor \neg \eventdep (\event, \event_y, c, D)$,
We know $\event_1$ is the only cause of the difference in $\event_y$ and $\event_y'$ when evaluating 
$[\assign{\pi_1(\event_2)}{\expr_2 / \query(\qexpr_2)}]{}^{\pi_2(\event_2)}$.
%
\\
By inversion Lemma.~\ref{lem:inv_expr_gnl} of arithmetic and query expression cases, 
given the two traces
$\trace_1 \tracecat [\event_1'] \tracecat \trace_2'$ and 
$\trace_1 \tracecat [\event_1'] \tracecat \trace_2'$ 
satisfying this lemma by $(1)$, $(2a)$ and $(2q)$, 
we know
\[
  \pi_1(\event_1) \in VAR(\expr_2 / \qexpr_2) \land {\pi_2(\event_1)} = \llabel(\trace_1 \tracecat [\event_1] \tracecat \trace_2) \pi_1(\event_1) 
\]
%
By $\flowsto$ definition:
% \todo{add the Liveness inversion}
\[
  \flowsto(\pi_1(\event_1)^{\pi_2(\event_1)}, \pi_1(\event_y)^{\pi_2(\event_y)}, c)
\]
This case is proved.
%
\caseL{$\event_2 \in \eventset^{\test}$}
\\
By the definition of $\eventdep(\event_1, \event_2, \trace, c, D)$, 
taking $ \event_1' \in \eventset^{\asn},
\trace_2' \in \mathcal{T}, c_1, c_2 \in \cdom$ and $\event_2' \in \eventset^{\test}$
 as the events, traces and commands satisfying the definition,
 we have following two executions:
% \[
%   \exists \event_1', \event_2' \in \eventset^{\asn},
%   \trace_2' \in \mathcal{T}, c_1, c_2 \in \cdom \st
% \]
%
\[
\begin{array}{l}
\config{c, \trace_0} \rightarrow^{*}
\config{c_1, \trace_1 \tracecat [\event_1]} \rightarrow^{*} \config{c_2, \trace_1 \tracecat [\event_1] \tracecat \trace_2 \tracecat [\event_2]} 
\\ \quad
% \land
\config{c_1, \trace_1 \tracecat [\event_1']} \rightarrow^{*} \config{c_2, \trace_1 \tracecat [\event_1'] \tracecat \trace_2' \tracecat [\event_2']} 
\end{array}
\]
%
Taking arbitrary $\event_z \in \trace_2$, we know 
$\neg \eventdep(\event_1, \event, \trace[\event_1:\event_z], c, D)
\lor  \pi_1(\event_z) \notin VAR(\expr_2 / \qexpr_2)$.
\\
Then by the same proof in \textbf{case: $\event_2 \in \eventset^{\asn}$}, and applying the Inversion Lemma~\ref{lem:inv_expr} and \ref{lem:inv_expr_gnl} of the boolean expression case,
we have:
\[ 
	\pi_1(\event_1) \in VAR(\pi_1(\event_2)) \land {\pi_2(\event_1)} = \llabel(\trace)
	\]
	This case is proved.
\end{proof}
%
% \begin{lem}[Arithmetic Inversion]
% \label{lem:inv_a}
% For all {$ x^i \in \lvar$, and $\trace, \trace' \in \mathcal{T}$,  and arithmetic expression $\aexpr$}, if
% $ \forall z^j \in \lvar / \{x^i\} \st 
% \env(\trace) z = \env(\trace') z $ and $\config{\trace, \aexpr} \aarrow v $ and 
% $\config{\trace', \aexpr} \aarrow v' $ with $v' \neq v$, then $ x $ is in the free variables of $\aexpr$, i.e., $x \in VAR(\aexpr)$.
% %
% % \[
% 	% \forall x^i \in \lvar, \trace, \trace' \in \mathcal{T}, \aexpr \st
% % 	\Big( \forall z^j \in \lvar / \{x^i\} \st 
% % 	\env(\trace) z = \env(\trace') z\Big) \land 
% % 	\config{\trace, \expr} \aarrow v \land \config{\trace', \expr} \aarrow v' \land v \neq v'
% % 	\implies x \in VAR(\expr)
% % \]
% \end{lem}
% %
% %
% \begin{lem}[Boolean Inversion]
% \label{lem:inv_b}
% For all {$ x^i \in \lvar$, and $\trace, \trace' \in \mathcal{T}$,  and boolean expression $\bexpr$}, if
% $ \forall z^j \in \lvar / \{x^i\} \st 
% \env(\trace) z = \env(\trace') z $ and $\config{\trace, \bexpr} \barrow v $ and 
% $\config{\trace', \bexpr} \barrow v' $ with $v' \neq v$, then $ x $ is in the free variables of $\bexpr$, i.e., $x \in VAR(\bexpr)$.
% % \[
% % 	text{(\forall x^i \in \lvar, \trace, \trace' \in \mathcal{T}, \bexpr \st )}
% % 	\Big(\forall z^j \in \lvar / \{x^i\} \st
% % 	\env(\trace) z = \env(\trace') z\Big) \land
% % 	\config{\trace, \bexpr} \barrow v \land \config{\trace', \bexpr} \barrow v' \land v \neq v'
% % 	\implies x \in VAR(\bexpr) 
% % \]
% \end{lem}
% %
% \begin{lem}[Query Inversion]
% \label{lem:inv_q}
% For all {$ x^i \in \lvar$, and $\trace, \trace' \in \mathcal{T}$,  and query expression $\qexpr$}, if
% $ \forall z^j \in \lvar / \{x^i\} \st 
% \env(\trace) z = \env(\trace') z $ and $\config{\trace, \qexpr} \qarrow \qval $ and 
% $\config{\trace', \qexpr} \qarrow \qval' $ with $\qval \neq_{q} \qval'$, then $ x $ is in the free variables of $\qexpr$, i.e., $x \in VAR(\qexpr)$.
% % \[
% % 	\forall x^i \in \lvar, \trace, \trace' \in \mathcal{T}, \qexpr \st 
% % 	\left(\forall z^j \in \lvar / \{x^i \} \st
% % 	\env(\trace) z = \env(\trace') z \right) \implies 
% % 	\config{\trace, \qexpr} \qarrow \qval \land \config{\trace', \qexpr} \qarrow \qval' 
% % 	\implies \qval \neq_{q} \qval'
% % 	\implies x \in VAR(\qexpr) 
% % \]
% \end{lem}
%
\begin{lem}[Expression Inversion]
	\label{lem:inv_expr}
	For all {$ x^i \in \lvar$, and $\trace, \trace' \in \mathcal{T}$, and an expression $\expr$} if
	$ \forall z^j \in \lvar / \{x^i\} \st 
	\env(\trace) z = \env(\trace') z $, and if
	\begin{itemize}
		\item $\expr$ is an arithmetic expression $\aexpr$,
		% \\ 
		and $\config{\trace, \aexpr} \aarrow v $ and 
	$\config{\trace', \aexpr} \aarrow v' $ with $v' \neq v$, 
	then $ x $ is in the free variables of $\aexpr$ and $i$ is the latest label for $x$ 
    in $\trace$, i.e., $x \in VAR(\aexpr)$ and $i = \llabel(\trace) x$.
%
	\item $\expr$ is a boolean expression $\bexpr$,
	% \\
  and $\config{\trace, \bexpr} \barrow v $ and 
 $\config{\trace', \bexpr} \barrow v' $ with $v' \neq v$, then $ x $ is in the free variables of $\bexpr$ and $i$ is the latest label for $x$ 
 in $\trace$, i.e., $x \in VAR(\bexpr)$ and $i = \llabel(\trace) x$.
% 
	\item $\expr$ is a query expression $\qexpr$,
	% \\
	and $\config{\trace, \qexpr} \qarrow \qval $ and 
	$\config{\trace', \qexpr} \qarrow \qval' $ with $\qval \neq_{q} \qval'$, then $ x $ is in the free variables of $\qexpr$ and $i$ is the latest label for $x$ 
    in $\trace$, i.e., $x \in VAR(\qexpr)$ and $i = \llabel(\trace) x$.
\end{itemize}	%
	\end{lem}
    Proof Summary:
    \\
    To show $x \in VAR(\aexpr)$, by showing contradiction ($\forall \trace, \trace'$ in second hypothesis  $v = v'$)
     if $x \notin VAR(\aexpr)$.
     \\
    To show $i = \llabel(\trace)$, by showing contradiction ($\forall \trace, \trace'$ in second hypothesis  $v = v'$ ) 
    if $j = \llabel(\trace) x$ and $i \neq j$.
    \begin{proof}
		Take two arbitrary traces $\trace, \trace' \in \mathcal{T}$, and an expression $\expr$ satisfying
		$ \forall z^j \in \lvar / \{x^i\} \st 
		\env(\trace) z = \env(\trace') z $, we have the following three cases.
    \caseL{$\expr$ is an arithmetic expression $\aexpr$}
	We have $\config{\trace, \bexpr} \barrow v $ and 
	$\config{\trace', \bexpr} \barrow v' $ with $v' \neq v$ from the lemma hypothesis.
	\\
	To show $x \in VAR(\qexpr)$ and $i = \llabel(\trace) x$: 
	\\
	Assuming $x \notin VAR(\aexpr)$,
	% by $\config{\trace, \aexpr} \aarrow v $	and $\config{\trace', \aexpr} \aarrow v'$, 
	since
	%  $x \notin VAR(\aexpr)$ and 
	$ \forall z^j \in \lvar / \{x^i\} \st 
		\env(\trace) z = \env(\trace') z $,
	we know $v = v'$, which is contradicted to $v' \neq v$.
	\\
	Then we know $x \in VAR(\qexpr)$.
	\\
	Assuming $j = \llabel(\trace) x \land i \neq j$,
	by 
	% $\config{\trace, \aexpr} \aarrow v $	and $\config{\trace', \aexpr} \aarrow v'$, 
	% since $x \notin VAR(\aexpr)$ and 
	$ \forall z^j \in \lvar / \{x^i\} \st 
		\env(\trace) z = \env(\trace') z $, we know 
		$\env(\trace) x = \env(\trace') x$, i.e., 
	\\
	$\forall z^j \in \lvar \st \env(\trace) z = \env(\trace') z$.
	\\
	Then by the determination of the evaluation, 
	% and 
	% $\config{\trace, \aexpr} \aarrow v $ and $\config{\trace', \aexpr} \aarrow v'$, 
	we know $v = v'$, which is contradicted to $v' \neq v$.
	\\
	Then we know $i = \llabel(\trace) x$.

    \caseL{$\expr$ is a boolean expression $\bexpr$}
	This case is proved trivially in the same way as the case of the arithmetic expression.
	\caseL{$\expr$ is a query expression $\qexpr$}
	This case is proved trivially in the same way as the case of the arithmetic expression.
\end{proof}
	%
	%
	% \begin{lem}[Boolean Inversion]
	% \label{lem:inv_b}
	% For all {$ x^i \in \lvar$, and $\trace, \trace' \in \mathcal{T}$,  and boolean expression $\bexpr$}, if
	% $ \forall z^j \in \lvar / \{x^i\} \st 
	% \env(\trace) z = \env(\trace') z $ and $\config{\trace, \bexpr} \barrow v $ and 
	% $\config{\trace', \bexpr} \barrow v' $ with $v' \neq v$, then $ x $ is in the free variables of $\bexpr$, i.e., $x \in VAR(\bexpr)$.
	% % \[
	% % 	text{(\forall x^i \in \lvar, \trace, \trace' \in \mathcal{T}, \bexpr \st )}
	% % 	\Big(\forall z^j \in \lvar / \{x^i\} \st
	% % 	\env(\trace) z = \env(\trace') z\Big) \land
	% % 	\config{\trace, \bexpr} \barrow v \land \config{\trace', \bexpr} \barrow v' \land v \neq v'
	% % 	\implies x \in VAR(\bexpr) 
	% % \]
	% \end{lem}
	% %
	% \begin{lem}[Query Inversion]
	% \label{lem:inv_q}
	% For all {$ x^i \in \lvar$, and $\trace, \trace' \in \mathcal{T}$,  and query expression $\qexpr$}, if
	% $ \forall z^j \in \lvar / \{x^i\} \st 
	% \env(\trace) z = \env(\trace') z $ and $\config{\trace, \qexpr} \qarrow \qval $ and 
	% $\config{\trace', \qexpr} \qarrow \qval' $ with $\qval \neq_{q} \qval'$, then $ x $ is in the free variables of $\qexpr$, i.e., $x \in VAR(\qexpr)$.
	% % \[
	% % 	\forall x^i \in \lvar, \trace, \trace' \in \mathcal{T}, \qexpr \st 
	% % 	\left(\forall z^j \in \lvar / \{x^i \} \st
	% % 	\env(\trace) z = \env(\trace') z \right) \implies 
	% % 	\config{\trace, \qexpr} \qarrow \qval \land \config{\trace', \qexpr} \qarrow \qval' 
	% % 	\implies \qval \neq_{q} \qval'
	% % 	\implies x \in VAR(\qexpr) 
	% % \]
	% \end{lem}
%
% \begin{lem}[Arithmetic Inversion Generalization]
% 	\label{lem:inv_a_gnl}
% 	% For all {$ x^i \in \lvar$, and $\trace, \trace' \in \mathcal{T}$,  and arithmetic expression $\aexpr$}, if
% 	% $ \forall z^j \in \lvar / \{x^i\} \st 
% 	% \env(\trace) z = \env(\trace') z $ and $\config{\trace, \aexpr} \aarrow v $ and 
% 	% $\config{\trace', \aexpr} \aarrow v' $ with $v' \neq v$, then $ x $ is in the free variables of $\aexpr$, i.e., $x \in VAR(\aexpr)$.
% 	%
% 	For all subset of the labelled variables $\diff \subset \lvar$, and $x^i \in (\lvar \setminus \diff)$,
% 	and an arithmetic expression $\aexpr$,
% 	if for all $z^j \in \lvar \setminus \diff, \trace, \trace' \in \mathcal{T}, v, v'$ such that 
% 	$\env(\trace) z = \env(\trace') z$, and 
% 	$
% 	\config{\trace, \aexpr} \aarrow v$, and $\config{\trace', \aexpr} \aarrow v'$ with $v = v'$;
% 	and for all $z^j \in \lvar / (\diff \cup \{x^i\} )$ 
% 	there exist $\trace, \trace' \in \mathcal{T}, v, v'$ such that 
% 	$\env(\trace) z = \env(\trace') z$, and 
% 	$
% 	\config{\trace, \aexpr} \aarrow v$, and $\config{\trace', \aexpr} \aarrow v'$ with $v \neq v'$,
% 	then $x \in VAR(\aexpr)$ and $i = \llabel(\trace) x$.
% 	\[
% 		\begin{array}{l}
% 		\forall \diff \subset \lvar,  x^i \in (\lvar \setminus \diff), \aexpr \st
% 		\\ \quad
% 		\forall z^j \in \lvar \setminus \diff, \trace, \trace' \in \mathcal{T}, v, v' \st 
% 		\env(\trace) z = \env(\trace') z \land 
% 		\config{\trace, \aexpr} \aarrow v \land \config{\trace', \aexpr} \aarrow v' \land v = v'
% 		\\ \quad
% 		\implies 
% 		\forall z^j \in \lvar / (\diff \cup \{x^i\} ) \st 
% 		 \exists \trace, \trace' \in \mathcal{T}, v, v'\st 
% 		\env(\trace) z = \env(\trace') z \land 
% 		\config{\trace, \aexpr} \aarrow v \land \config{\trace', \aexpr} \aarrow v' \land v \neq v'
% 		\\ \qquad
% 		\implies x \in VAR(\aexpr) \land i = \llabel(\trace) x
% 		\end{array}
% 	\]
% 	\end{lem}
% \begin{proof}.
% 	\\
% 	To show $x \in VAR(\aexpr)$, by showing contradiction ($\forall \trace, \trace'$ in second hypothesis  $v = v'$)
% 	 if $x \notin VAR(\aexpr)$.
% 	 \\
% 	To show $i = \llabel(\trace)$, by showing contradiction ($\forall \trace, \trace'$ in second hypothesis  $v = v'$ ) 
% 	if $j = \llabel(\trace) x$ and $i \neq j$.
% \end{proof}
% 	%
% \begin{lem}[Boolean Inversion Generalization]
% 		\label{lem:inv_b_gnl}
% 		% For all {$ x^i \in \lvar$, and $\trace, \trace' \in \mathcal{T}$,  and arithmetic expression $\aexpr$}, if
% 		% $ \forall z^j \in \lvar / \{x^i\} \st 
% 		% \env(\trace) z = \env(\trace') z $ and $\config{\trace, \aexpr} \aarrow v $ and 
% 		% $\config{\trace', \aexpr} \aarrow v' $ with $v' \neq v$, then $ x $ is in the free variables of $\aexpr$, i.e., $x \in VAR(\aexpr)$.
% 		%
% 		\[
% 			\begin{array}{l}
% 			\forall \diff \subset \lvar,  x^i \in (\lvar \setminus \diff), \bexpr \st
% 			\\ \quad
% 			\forall z^j \in \lvar \setminus \diff, \trace, \trace' \in \mathcal{T}, v, v' \st 
% 			\env(\trace) z = \env(\trace') z \land 
% 			\config{\trace, \bexpr} \barrow v \land \config{\trace', \bexpr} \barrow v' \land v = v'
% 			\\ \quad
% 			\implies 
% 			\forall z^j \in \lvar / (\diff \cup \{x^i\} ) \st 
% 			 \exists \trace, \trace' \in \mathcal{T}, v, v'\st 
% 			\env(\trace) z = \env(\trace') z \land 
% 			\config{\trace, \bexpr} \barrow v \land \config{\trace', \bexpr} \barrow v' \land v \neq v'
% 			\\ \qquad
% 			\implies x \in VAR(\bexpr) \land i = \llabel(\trace)
% 			\end{array}
% 		\]
% \end{lem}%
\begin{lem}[Expression Inversion Generalization]
	\label{lem:inv_expr_gnl}
	For all subset of the labelled variables $\diff \subset \lvar$, and $x^i \in (\lvar \setminus \diff)$,
	and an expression $\expr$, if 
	\begin{itemize}
		\item $\expr$ is an arithmetic expression $\aexpr$,
		% \\ 
		and for all $z^j \in \lvar \setminus \diff, \trace, \trace' \in \mathcal{T}, v, v'$ such that 
		$\env(\trace) z = \env(\trace') z$, and 
		$
		\config{\trace, \aexpr} \aarrow v$, and $\config{\trace', \aexpr} \aarrow v'$ with $v = v'$;
		and for all $z^j \in \lvar / (\diff \cup \{x^i\} )$ 
		there exist $\trace, \trace' \in \mathcal{T}, v, v'$ such that 
		$\env(\trace) z = \env(\trace') z$, and 
		$
		\config{\trace, \aexpr} \aarrow v$, and $\config{\trace', \aexpr} \aarrow v'$ with $v \neq v'$,
		then $x \in VAR(\aexpr)$ and $i = \llabel(\trace) x$.
		\[
			\begin{array}{l}
			\forall \diff \subset \lvar,  x^i \in (\lvar \setminus \diff), \aexpr \st
			\\ \quad
			\forall z^j \in \lvar \setminus \diff, \trace, \trace' \in \mathcal{T}, v, v' \st 
			\env(\trace) z = \env(\trace') z \land 
			\config{\trace, \aexpr} \aarrow v \land \config{\trace', \aexpr} \aarrow v' \land v = v'
			\\ \quad
			\implies 
			\forall z^j \in \lvar / (\diff \cup \{x^i\} ) \st 
			\exists \trace, \trace' \in \mathcal{T}, v, v'\st 
			\env(\trace) z = \env(\trace') z \land 
			\config{\trace, \aexpr} \aarrow v \land \config{\trace', \aexpr} \aarrow v' \land v \neq v'
			\\ \qquad
			\implies x \in VAR(\aexpr) \land i = \llabel(\trace) x
			\end{array}
		\]
	\item $\expr$ is a boolean expression $\bexpr$,
	and for all $ z^j \in \lvar \setminus \diff, \trace, \trace' \in \mathcal{T}, v, v'$ such that 
	$ \env(\trace) z = \env(\trace') z \land 
	\config{\trace, \bexpr} \barrow v \land \config{\trace', \bexpr} \barrow v' \land v = v'$;
	and for all
	$ z^j \in \lvar / (\diff \cup \{x^i\} ) \st 
	 \exists \trace, \trace' \in \mathcal{T}, v, v'\st 
	\env(\trace) z = \env(\trace') z \land 
	\config{\trace, \bexpr} \barrow v \land \config{\trace', \bexpr} \barrow v' \land v \neq v'$
	then 
	 $x \in VAR(\bexpr) \land i = \llabel(\trace) x$
	% \\
	\[
		\begin{array}{l}
		\forall \diff \subset \lvar,  x^i \in (\lvar \setminus \diff), \bexpr \st
		\\ \quad
		\forall z^j \in \lvar \setminus \diff, \trace, \trace' \in \mathcal{T}, v, v' \st 
		\env(\trace) z = \env(\trace') z \land 
		\config{\trace, \bexpr} \barrow v \land \config{\trace', \bexpr} \barrow v' \land v = v'
		\\ \quad
		\implies 
		\forall z^j \in \lvar / (\diff \cup \{x^i\} ) \st 
		 \exists \trace, \trace' \in \mathcal{T}, v, v'\st 
		\env(\trace) z = \env(\trace') z \land 
		\config{\trace, \bexpr} \barrow v \land \config{\trace', \bexpr} \barrow v' \land v \neq v'
		\\ \qquad
		\implies x \in VAR(\bexpr) \land i = \llabel(\trace) x
		\end{array}
	\]
% 
	\item $\expr$ is a query expression $\qexpr$,
	and for all $\diff \subset \lvar,  x^i \in (\lvar \setminus \diff), \qexpr$ such that 
	for all $ z^j \in \lvar \setminus \diff, \trace, \trace' \in \mathcal{T}, \qval, \qval' \st 
 \env(\trace) z = \env(\trace') z \land 
 \config{\trace, \qexpr} \qarrow \qval \land \config{\trace', \qexpr} \qarrow \qval' \land \qval =_q \qval'$;
 and for all 
	$ z^j \in \lvar / (\diff \cup \{x^i\} ) \st 
  \exists \trace, \trace' \in \mathcal{T}, \qval, \qval'\st 
 \env(\trace) z = \env(\trace') z \land 
 \config{\trace, \qexpr} \qarrow \qval \land \config{\trace', \qexpr} \qarrow \qval' \land \qval \neq_{q} \qval'$,
 then  $x \in VAR(\qexpr) \land i = \llabel(\trace) x$.
	% \\
	\[
		\begin{array}{l}
		\forall \diff \subset \lvar,  x^i \in (\lvar \setminus \diff), \qexpr \st
		\\ \quad
		\forall z^j \in \lvar \setminus \diff, \trace, \trace' \in \mathcal{T}, \qval, \qval' \st 
		\env(\trace) z = \env(\trace') z \land 
		\config{\trace, \qexpr} \qarrow \qval \land \config{\trace', \qexpr} \qarrow \qval' \land \qval =_q \qval'
		\\ \quad
		\implies 
		\forall z^j \in \lvar / (\diff \cup \{x^i\} ) \st 
		 \exists \trace, \trace' \in \mathcal{T}, \qval, \qval'\st 
		\env(\trace) z = \env(\trace') z \land 
		\config{\trace, \qexpr} \qarrow \qval \land \config{\trace', \qexpr} \qarrow \qval' \land \qval \neq_{q} \qval'
		\\ \qquad
		\implies x \in VAR(\qexpr) \land i = \llabel(\trace) x
		\end{array}
	\]
	\end{itemize}
	\end{lem}
	%
Proof Summary: 
\\
To show $x \in VAR(\aexpr)$, by showing contradiction ($\forall \trace, \trace'$ in second hypothesis  $v = v'$)
 if $x \notin VAR(\aexpr)$.
 \\
To show $i = \llabel(\trace)$, by showing contradiction ($\forall \trace, \trace'$ in second hypothesis  $v = v'$ ) 
if $j = \llabel(\trace) x$ and $i \neq j$.
\begin{proof}
	Taking an arbitrary expression $\expr$,
	 we have the following three cases.
\caseL{$\expr$ is an arithmetic expression $\aexpr$}
Taking an arbitrary set of labelled variables 
	$\diff \subset \lvar$, $x^i \in (\lvar \setminus \diff)$ satisfies:
	\\
	$\forall z^j \in \lvar \setminus \diff, \trace, \trace' \in \mathcal{T}, v, v' \st 
	\env(\trace) z = \env(\trace') z \land 
	\config{\trace, \aexpr} \aarrow v \land \config{\trace', \aexpr} \aarrow v' \land v = v' ~ (1)
	$
	\\
	and 
	$\forall z^j \in \lvar \setminus (\diff \cup \{x^i\} ) \st 
	\exists \trace, \trace' \in \mathcal{T}, v, v'\st 
	\env(\trace) z = \env(\trace') z \land 
	\config{\trace, \aexpr} \aarrow v \land \config{\trace', \aexpr} \aarrow v' \land v \neq v' ~ (2) 
	$,
	\\
	Let $\trace, \trace' \in \mathcal{T}, v, v'$ be the two traces and values satisfies hypothesis $(2)$.
	\\
	To show: $x \in VAR(\aexpr) \land i = \llabel(\trace) x$:
	\\
% \[
% 	\begin{array}{l}
% 	\forall \diff \subset \lvar,  x^i \in (\lvar \setminus \diff), \aexpr \st
% 	\\ \quad
% 	\forall z^j \in \lvar \setminus \diff, \trace, \trace' \in \mathcal{T}, v, v' \st 
% 	\env(\trace) z = \env(\trace') z \land 
% 	\config{\trace, \aexpr} \aarrow v \land \config{\trace', \aexpr} \aarrow v' \land v = v'
% 	\\ \quad
% 	\implies 
% 	\forall z^j \in \lvar / (\diff \cup \{x^i\} ) \st 
% 	\exists \trace, \trace' \in \mathcal{T}, v, v'\st 
% 	\env(\trace) z = \env(\trace') z \land 
% 	\config{\trace, \aexpr} \aarrow v \land \config{\trace', \aexpr} \aarrow v' \land v \neq v'
% 	\\ \qquad
% 	\implies x \in VAR(\aexpr) \land i = \llabel(\trace) x
% 	\end{array}
% \] 
Assuming $x \notin VAR(\aexpr)$, we know from the Inversion Lemma~\ref{lem:inv_expr} of the arithmetic expression case,
\\
$\forall z^j \in \lvar \setminus \{x^i\}, \trace, \trace' \in \mathcal{T}, v, v' \st 
\env(\trace) z = \env(\trace') z \land 
\config{\trace, \aexpr} \aarrow v \land \config{\trace', \aexpr} \aarrow v' \land v = v'$.
\\
Then with the hypothesis $(1)$, we know:
\\
$\forall z^j \in \lvar \setminus (\diff \cup \{x^i\} ), \trace, \trace' \in \mathcal{T}, v, v'\st 
\env(\trace) z = \env(\trace') z \land 
\config{\trace, \aexpr} \aarrow v \land \config{\trace', \aexpr} \aarrow v' \land v = v'$
\\
This is contradicted to the hypothesis $(2)$.
\\
Then we know $x \in VAR(\expr)$.
\\
Assuming $j = \llabel(\trace) x \land i \neq j$,
by hypothesis $(2)$ where 
% $\config{\trace, \aexpr} \aarrow v $	and $\config{\trace', \aexpr} \aarrow v'$, 
% since $x \notin VAR(\aexpr)$ and 
$ \forall z^j \in \lvar \setminus (\diff \cup \{x^i\} )  \st\env(\trace) z = \env(\trace') z $, 
we know $\env(\trace) x = \env(\trace') x$, i.e., 
\\
$\forall z^j \in  \lvar \setminus (\diff  ) \st \env(\trace) z = \env(\trace') z$.
\\
Then we have $v' = v$ by hypothesis $(1)$, which is contradicted to $v' \neq v$.
\\
Then we know $i = \llabel(\trace) x $.

\caseL{$\expr$ is a boolean expression $\bexpr$}
This case is proved trivially in the same way as the case of the arithmetic expression.
\caseL{$\expr$ is a query expression $\qexpr$}
This case is proved trivially in the same way as the case of the arithmetic expression.
\end{proof}
	%
% \begin{lem}[Boolean Inversion Generalization]
% 		\label{lem:inv_b_gnl}
% 		% For all {$ x^i \in \lvar$, and $\trace, \trace' \in \mathcal{T}$,  and arithmetic expression $\aexpr$}, if
% 		% $ \forall z^j \in \lvar / \{x^i\} \st 
% 		% \env(\trace) z = \env(\trace') z $ and $\config{\trace, \aexpr} \aarrow v $ and 
% 		% $\config{\trace', \aexpr} \aarrow v' $ with $v' \neq v$, then $ x $ is in the free variables of $\aexpr$, i.e., $x \in VAR(\aexpr)$.
% 		%
% 		\[
% 			\begin{array}{l}
% 			\forall \diff \subset \lvar,  x^i \in (\lvar \setminus \diff), \bexpr \st
% 			\\ \quad
% 			\forall z^j \in \lvar \setminus \diff, \trace, \trace' \in \mathcal{T}, v, v' \st 
% 			\env(\trace) z = \env(\trace') z \land 
% 			\config{\trace, \bexpr} \barrow v \land \config{\trace', \bexpr} \barrow v' \land v = v'
% 			\\ \quad
% 			\implies 
% 			\forall z^j \in \lvar / (\diff \cup \{x^i\} ) \st 
% 			 \exists \trace, \trace' \in \mathcal{T}, v, v'\st 
% 			\env(\trace) z = \env(\trace') z \land 
% 			\config{\trace, \bexpr} \barrow v \land \config{\trace', \bexpr} \barrow v' \land v \neq v'
% 			\\ \qquad
% 			\implies x \in VAR(\bexpr) \land i = \llabel(\trace)
% 			\end{array}
% 		\]
% \end{lem}%
%
% \begin{lem}[Assignment Inversion].
% \label{lem:inv_a}
% \[
% 	\forall x \in \lvar, \expr \st 
% 	\Big( \exists \trace, \trace' \st \forall z^i \in \lvar / \{x^l\} \st 
% 	\env(\trace) z = \env(\trace') z \st 
% 	\config{\trace, \expr} \aarrow v \land \config{\trace', \expr} \aarrow v' \land v  \neq v'
% 	\implies x \in VAR(\expr) \land x^l \in \Big)
% \]
% \end{lem}
%
% \begin{lem}[Assignment Event Inversion]
% \label{lem:inv_asn}
% \[
% \begin{array}{l}
% 	\forall \trace_0 \in \mathcal{T}, c \in \cdom,
% 	\event \in \eventset^{\asn} \st
% 	\config{c, \trace_0} \rightarrow^* \config{\eskip, \trace_0 \tracecat \trace_1} \implies
% 	(\event \eventin \trace_1 \land	x = \pi_1(\event) \land l = \pi_2(\event))
% 	\\ 
% 	\implies 
% 	\big( 
% 		\exists \trace_1' \in \mathcal{T}, \expr, c' \st
% 		\config{c, \trace_0} \rightarrow^* \config{ [\assign{x}{\expr}]^l;c', \trace_0  \tracecat  \trace'} \rightarrow^\rname{assn}
% 		\config{c', \trace_0 \tracecat \trace_1'\tracecat [\event]} \rightarrow^{*}
% 		\config{\eskip, \trace_0  \tracecat  \trace_1}
% 	\big)
% 	\\ \qquad \lor
% 	\big( 
% 		\exists \trace_1' \in \mathcal{T}, \qexpr, c' \st
% 		\config{c, \trace_0} \rightarrow^* \config{ [\assign{x}{\query(\qexpr)}]^l;c', \trace_0 \tracecat \trace_1'} \rightarrow^{query}
% 		\config{c', \trace_0  \tracecat  \trace_1' \tracecat [\event] } \rightarrow^{*}
% 		\config{ \eskip, \trace_0\trace_1}
% 	\big)
% \end{array}
% \]
% %
% \end{lem}
% %
% \begin{lem}[Testing Event Inversion]
% \label{lem:inv_test}
% \[
% \begin{array}{l}
% 	\forall c\in \cdom, \trace_0 \in \mathcal{T}, \event = (b, l, n, v) \in \eventset^{\test} \st
% 	 \config{c, \trace_0} \rightarrow^* \config{\eskip, \trace_0 \tracecat \trace_1}
% 	\implies  \event \eventin \trace_1 \\
% 	\implies 
% 	\big( 
% 		\exists \trace_1' \in \mathcal{T}, \bexpr, c', c_t, c_f, c'' \in \cdom \st
% 		\config{c, \trace_0} \rightarrow^* \config{\eif ([b]^l, c_t, c_f);c', \trace_0 \tracecat \trace_1'} \rightarrow^{if-b}
% 		\config{c'', \trace_0 \tracecat \trace_1'\tracecat [\event] } \rightarrow^{*}
% 		\config{\eskip, \trace_0 \tracecat \trace_1} 
% 	\big)
% 	\\ \qquad \lor
% 	\big( 
% 		\exists \trace_1' \in \mathcal{T}, \bexpr, c', c_w, c'' \in \cdom \st
% 		\config{c, \trace_0} \rightarrow^* \config{ \ewhile([b]^l, c_w);c', \trace_0 \tracecat  \trace_1'} \rightarrow^{while-b}
% 		\config{c'', \trace_0 \tracecat \trace_1'\tracecat [\event] } \rightarrow^{*}
% 		\config{\eskip, \trace_0  \tracecat \trace_1}
% 	\big)
% \end{array}
% \]
% \end{lem}
\begin{lem}[Event Inversion]
\label{lem:inv_event}
For all $c\in \cdom, \trace_0 \in \mathcal{T}, \event \in \eventset$such that 
$\config{c, \trace_0} \rightarrow^* \config{\eskip, \trace_0 \tracecat \trace_1}$, 
and $\event \eventin \trace_1$, if 
\begin{itemize}
	\item $\event \in \eventset^{\asn}$, then either
	\begin{itemize}
	 \item there exists $\trace_1' \in \mathcal{T}, c' \in \cdom, \expr$ such that
\[
\begin{array}{l}
	% \forall \trace_0 \in \mathcal{T}, c \in \cdom,
	% \event \in \eventset^{\asn} \st
	% \config{c, \trace_0} \rightarrow^* \config{\eskip, \trace_0 \tracecat \trace_1} \implies
	% (\event \eventin \trace_1 \land	x = \pi_1(\event) \land l = \pi_2(\event))
	% \\ 
	% \implies 
	% \big( 
		% \exists \trace_1' \in \mathcal{T}, \expr, c' \st
		\config{c, \trace_0} \rightarrow^* \config{ [\assign{x}{\expr}]^l;c', \trace_0  \tracecat  \trace'} \rightarrow^\rname{assn}
		\config{c', \trace_0 \tracecat \trace_1'\tracecat [\event]} \rightarrow^{*}
		\config{\eskip, \trace_0  \tracecat  \trace_1}
	% \big)
	% \\  \lor
	% \big( 
	% 	\exists \trace_1' \in \mathcal{T}, \qexpr, c' \st
	% 	\config{c, \trace_0} \rightarrow^* \config{ [\assign{x}{\query(\qexpr)}]^l;c', \trace_0 \tracecat \trace_1'} \rightarrow^{query}
	% 	\config{c', \trace_0  \tracecat  \trace_1' \tracecat [\event] } \rightarrow^{*}
	% 	\config{ \eskip, \trace_0\trace_1}
	% \big)
\end{array}
\]
\item or there exists $\trace_1' \in \mathcal{T}, c' \in \cdom, \qexpr$ such that 
\[
\begin{array}{l}
	% \forall \trace_0 \in \mathcal{T}, c \in \cdom,
	% \event \in \eventset^{\asn} \st
	% \config{c, \trace_0} \rightarrow^* \config{\eskip, \trace_0 \tracecat \trace_1} \implies
	% (\event \eventin \trace_1 \land	x = \pi_1(\event) \land l = \pi_2(\event))
	% \\ 
	% \implies 
	% \big( 
	% 	\exists \trace_1' \in \mathcal{T}, \expr, c' \st
	% 	\config{c, \trace_0} \rightarrow^* \config{ [\assign{x}{\expr}]^l;c', \trace_0  \tracecat  \trace'} \rightarrow^\rname{assn}
	% 	\config{c', \trace_0 \tracecat \trace_1'\tracecat [\event]} \rightarrow^{*}
	% 	\config{\eskip, \trace_0  \tracecat  \trace_1}
	% \big)
	% \\  \lor
	% \big( 
		% \exists \trace_1' \in \mathcal{T}, \qexpr, c' \st
		\config{c, \trace_0} \rightarrow^* \config{ [\assign{x}{\query(\qexpr)}]^l;c', \trace_0 \tracecat \trace_1'} \rightarrow^{query}
		\config{c', \trace_0  \tracecat  \trace_1' \tracecat [\event] } \rightarrow^{*}
		\config{ \eskip, \trace_0 \tracecat \trace_1}
	% \big)
\end{array}
\]
\end{itemize}

\item $\event\in \eventset^{\test}$ then either 
\begin{itemize}
\item there exists $\trace_1' \in \mathcal{T}, c', c_t, c_f, c'' \in \cdom, \bexpr$ such that
\[
\begin{array}{l}
	% \big( 
	% 	\exists \trace_1' \in \mathcal{T}, c', c_t, c_f, c'' \in \cdom, \bexpr \st
		\config{c, \trace_0} \rightarrow^* \config{\eif ([b]^l, c_t, c_f);c', \trace_0 \tracecat \trace_1'} \rightarrow^{if-b}
		\config{c'', \trace_0 \tracecat \trace_1'\tracecat [\event] } \rightarrow^{*}
		\config{\eskip, \trace_0 \tracecat \trace_1} 
	% \big)
	% \\  \lor
	% \big( 
	% 	\exists \trace_1' \in \mathcal{T}, \bexpr, c', c_w, c'' \in \cdom \st
	% 	\config{c, \trace_0} \rightarrow^* \config{ \ewhile([b]^l, c_w);c', \trace_0 \tracecat  \trace_1'} \rightarrow^{while-b}
	% 	\config{c'', \trace_0 \tracecat \trace_1'\tracecat [\event] } \rightarrow^{*}
	% 	\config{\eskip, \trace_0  \tracecat \trace_1}
	% \big)
\end{array}
\]
\item or there exists $ \trace_1' \in \mathcal{T}, c', c_w, c'' \in \cdom, \bexpr$ such that 
\[
% \begin{array}{l}
% 	\big( 
% 		\exists \trace_1' \in \mathcal{T}, \bexpr, c', c_t, c_f, c'' \in \cdom \st
% 		\config{c, \trace_0} \rightarrow^* \config{\eif ([b]^l, c_t, c_f);c', \trace_0 \tracecat \trace_1'} \rightarrow^{if-b}
% 		\config{c'', \trace_0 \tracecat \trace_1'\tracecat [\event] } \rightarrow^{*}
% 		\config{\eskip, \trace_0 \tracecat \trace_1} 
% 	\big)
% 	\\  \lor
% 	\big( 
% 		\exists \trace_1' \in \mathcal{T}, \bexpr, c', c_w, c'' \in \cdom \st
		\config{c, \trace_0} \rightarrow^* \config{ \ewhile([b]^l, c_w);c', \trace_0 \tracecat  \trace_1'} \rightarrow^{while-b}
		\config{c'', \trace_0 \tracecat \trace_1'\tracecat [\event] } \rightarrow^{*}
		\config{\eskip, \trace_0  \tracecat \trace_1}
% 	\big)
% \end{array}
\]
\end{itemize}
\end{itemize}
%
\end{lem}
Proof Summary: trivially by induction on $c$ and enumerate all operational semantic rules.
\begin{proof}
	Take arbitrary $\trace_0 \in \mathcal{T}$, by induction on $c$, we have following cases:
		\caseL{$c = [\assign{x}{\expr}]^l$}
		By the evaluation rule $\rname{assn}$, we have
		$
		{
		\config{[\assign{{x}}{\aexpr}]^{l},  \trace } 
		\xrightarrow{} 
		\config{\eskip, \trace \tracecat [({x}, l, v) ]}
		}$.
		\\
		Then we know $\trace_1 = [({x}, l, v)]$ and there is only 1 event $(x, l, v) \in \trace_1$.
		\\
		Then we have $\trace_1' = []$ and $c' = \eskip$ satisfying
		\\
		$\config{c, \trace_0} \rightarrow^* \config{ [\assign{x}{\expr}]^l;c', \trace_0  \tracecat  \trace'} \rightarrow^\rname{assn}
		\config{c', \trace_0 \tracecat \trace_1'\tracecat [\event]} \rightarrow^{*}
		\config{\eskip, \trace_0  \tracecat  \trace_1}$.
		\\
		This case is proved.
		\caseL{$c = [\assign{x}{\query(\qexpr)}]^l$}
		This case is proved trivially in the same way as \textbf{case: $c = [\assign{x}{\expr}]^l$}.
		\caseL{$c = c_{s1};c_{s2}$}
		This case is proved trivially by the induction hypothesis on $c_{s1}$ and $c_{s2}$ separately, we have this case proved.
		\caseL{$\ewhile [b]^{l} \edo c$}
		If the rule applied to is $\rname{while-t}$, we have:
		\\
		$\config{{\ewhile [b]^{l} \edo c_w, \trace}}
			\xrightarrow{} 
			\config{{
			c_w; \ewhile [b]^{l} \edo c_w,
			\trace \tracecat [(b, l, \etrue)]}}
			\xrightarrow{*} 
			\config{{
			\eskip,
			\trace \tracecat \trace_1}}
		$,
		\\
		%
		$(b, l, \etrue) \in \event^{\test}$ and $(b, l, \etrue) \in \trace_1$.
		\\
		Let $\trace' = []$, $c' = \eskip$ and $c'' = c_w; \ewhile [b]^{l} \edo c_w$, we know that they satisfy
		\\
		$\config{c, \trace_0} \rightarrow^* \config{ \ewhile([b]^l, c_w);c', \trace_0 \tracecat  \trace_1'} \rightarrow^{while-b}
		\config{c'', \trace_0 \tracecat \trace_1'\tracecat [\event] } \rightarrow^{*}
		\config{\eskip, \trace_0  \tracecat \trace_1}$
		\\
		% And we also have the existence of $l = l_b, b$ and $c_w$, and $\ewhile [b]^{l} \edo c_w \in_c c_2$ and  $c_1 \in c_w$.
		% \\
		% If $c_w$ isn't a sequence command, let $c_1 = c_w$, then we have $c_2 = \ewhile [b]^{l} \edo c_w,  \eskip)$ 
		% and $c_1 \in_c c_2$.
		% \\
		% And we also have the existence of $l = l_b, b$ and $c_w$, and $\ewhile [b]^{l} \edo c_w \in_c c_2$ and  $c_1 \in c_w$.
		% \\
		This case is proved.
		\\
		If the rule applied to is $\rname{while-f}$, we have
		\\
		$
		{
			\config{{\ewhile [b]^{l} \edo c_w, \trace}}
			\xrightarrow{}^\rname{while-f}
			\config{{
			\eskip,
			\trace \tracecat [((b, l, \efalse))]}}
		}$,
		$(b, l, \etrue) \in \event^{\test}$, and $(b, l, \etrue) \in \trace_1$.
		\\
		Let $\trace' = []$, $c' = \eskip$ and $c'' = \eskip$, we know that they satisfy
		\\
		$\config{c, \trace_0} \rightarrow^* \config{ \ewhile([b]^l, c_w);c', \trace_0 \tracecat  \trace_1'} 
		\rightarrow^\rname{while-f}
		\config{c'', \trace_0 \tracecat \trace_1'\tracecat [(b, l, \efalse)] } \rightarrow^{*}
		\config{\eskip, \trace_0  \tracecat \trace_1}$
		\\
		This case is proved.
		\caseL{$\eif([b]^l, c_t, c_f)$}
		This case is proved in the same way as \textbf{case: $c = [\assign{x}{\query(\qexpr)}]^l$}.
	\end{proof}
%
% \begin{lem}[Testing Event Inversion]
% \label{lem:inv_test}
% \[
% \begin{array}{l}
% 	\forall c\in \cdom, \trace_0 \in \mathcal{T}, \event = (b, l, n, v) \in \eventset^{\test} \st
% 	 \config{c, \trace_0} \rightarrow^* \config{\eskip, \trace_0 \tracecat \trace_1}
% 	\implies  \event \eventin \trace_1 \\
% 	\implies 
% 	\big( 
% 		\exists \trace_1' \in \mathcal{T}, \bexpr, c', c_t, c_f, c'' \in \cdom \st
% 		\config{c, \trace_0} \rightarrow^* \config{\eif ([b]^l, c_t, c_f);c', \trace_0 \tracecat \trace_1'} \rightarrow^{if-b}
% 		\config{c'', \trace_0 \tracecat \trace_1'\tracecat [\event] } \rightarrow^{*}
% 		\config{\eskip, \trace_0 \tracecat \trace_1} 
% 	\big)
% 	\\ \qquad \lor
% 	\big( 
% 		\exists \trace_1' \in \mathcal{T}, \bexpr, c', c_w, c'' \in \cdom \st
% 		\config{c, \trace_0} \rightarrow^* \config{ \ewhile([b]^l, c_w);c', \trace_0 \tracecat  \trace_1'} \rightarrow^{while-b}
% 		\config{c'', \trace_0 \tracecat \trace_1'\tracecat [\event] } \rightarrow^{*}
% 		\config{\eskip, \trace_0  \tracecat \trace_1}
% 	\big)
% \end{array}
% \]
% \end{lem}
%
% \begin{lem}[Control Dependency -> Exists Testing Event]
% \label{lem:inv_ctltotest}
% \[
% 	\forall \event_1, \event_2 \in \eventset, c \st 
% 	\eventdep^{\ctl}(\event_1, \event_2, c, D)
% 	\implies
% 	\exists \event_b \in \eventset^{\test}, \trace_2 \in \mathcal{T} \st \eventdep(\event_1, \event_b, \trace_2, c, D)
% \]
% \end{lem}

% \begin{lem}[Control Dependency -> Event 2 in the Body Command of the Testing Event]
% \label{lem:inv_ctltoevent2}
% \[
% \begin{array}{l}
% 	\forall \event_1, \event_2 = (x_2, l_2, n_2, v_2) \in \eventset, c \st 
% 	\eventdep^{\ctl}(\event_1, \event_2, c, D)\\
% 	\implies
% 	\exists \event_b = (b, l, n, v) \in \eventset^{\test}, \expr_2 \st \eventdep(\event_1, \event_b, c, D)\\
% 	\quad \land \Big(
% 	\exists c_t, c_f \st (\eif ([b]{}^l, c_t, c_f)) \in_{c} c \land ([\assign{x_2}{\expr_2}]^{l_2}) \in_c c_t;c_f \\
% 	\qquad \lor\exists c_w \st (\ewhile [b]{}^l \edo c_w) \in_{c} c \land ([\assign{x_2}{\expr_2}]^{l_2}) \in_c c_w
% 	\Big)
% \end{array}
% \]
% \end{lem}
%
\begin{lem}[Reachable Varibale Inversion]
\label{lem:inv_live}
For all $c \in \cdom \trace, \trace' \in \mathcal{T} $, if 
$\config{c, \trace} \xrightarrow{}^* \config{c', \trace'}$,
and for all $x^l \in \lvar_c$ such that 
% $\llabel(\trace') x = l $, then $(x^l \in \live^{\entry_{c'}}(c))$.
$\llabel(\trace') x = l $, then $x^l \in \live(\absinit(c), c)$.
%
\[
	\forall c \in \cdom , \trace, \trace' \in \mathcal{T} \st
	\config{c, \trace} \xrightarrow{}^* \config{c', \trace'}
	\implies
	% \forall x^l \in \lvar_c \st \llabel(\trace') x = l \implies (x^l \in \live^{\entry_{c'}}(c))
	\forall x^l \in \lvar_c \st \llabel(\trace') x = l \implies x^l \in \live(\absinit(c), c)
\]
\end{lem}
Proof Summary: 
If a variable with the label which is the latest one in the trace,
Then by the environment definition, the value associated to this labelled variable is read from the trace.
\\
Then this labelled variable must be reachable at the point of $\entry_{c'}$, i.e., 
% $x^l \in \live^{\entry_{c'}}(c)$.
$x^l \in \live(\absinit(c), c)$.
\begin{proof}
	Take arbitrary $c \in \cdom , \trace, \trace' \in \mathcal{T}$ satisfying 
	$\config{c, \trace} \xrightarrow{}^* \config{c', \trace'}$, 
	and an arbitrary $x^l \in \lvar_c$ satisfying $\llabel(\trace') x = l$.
	\\
	By definition of $\llabel$, we know $\trace'$ has the form $\trace'_{a} \tracecat [(x, l, v)] \tracecat \trace_{b}'$
	for some $\trace'_{a} , \trace_{b}' \in \mathcal{T}$ and $v$.
	\\
	And the variable $x$ doesn't show up in all the events in $\trace_b'$.
%
\\
	Then, by the environment definition, we know:
	$\env(\trace') x = v$, i.e., $x^l$ is 
	reachable at the point of 
	% $\entry_{c'}$.
	$\absinit(c)$.
	\\
	By the $in(l)$ operator define in Section~\ref{sec:alg_edgegen}, we know $x^l$ is in the $in(\absinit(c)$ for prpgram $c$.
	\\
	% By the $\live$ definition, 
	Since $\live(\absinit(c), c)$ is a stabilized closure of $in(l)$ for $c$,
	we know 
	% $x^l \in \live^{\entry_{c'}}(c)$.
	$x^l \in \live(\absinit(c), c)$.
	\\
	This lemma is proved.
\end{proof}
%
\begin{lem}[While Loop Inversion]
	\label{lem:inv_while}
	For every $\trace, \trace' \in \mathcal{T}, c, c_1, c_2 \in \cdom$ 
	if $ \config{c, \trace} \rightarrow^* \config{c_1; c_2, \trace'}$ and 
	$c_1 \in_c c_2$, 
	then there must exist a $\ewhile$ command in $c_2$ and $c_1$ must shows up in the body of that $\ewhile$ command,
	 i.e., $\exists l \in \mathbb{N}, b \in \mathcal{B}, c_w \in \cdom \st 
	(\ewhile [b]^l \edo c_w) \in_c c_2 \land c_1 \in_c c_w$.
	%
	\[
	\begin{array}{l}
	\forall \trace, \trace' \in \mathcal{T}, c, c_1, c_2 \in \cdom \st
		\\ \quad
		\config{c, \trace} \rightarrow^* \config{c_1; c_2, \trace'}
		\implies
		c_1 \in_c c_2
		\implies
		\exists l \in \mathbb{N}, b \in \mathcal{B}, c_w \in \cdom \st 
		(\ewhile [b]^l \edo c_w) \in_c c_2 \land c_1 \in_c c_w
	\end{array}
	\]
	\end{lem}	
	Proof Summary: trivially by induction on $c$ and enumerate all operational semantic rules.
\begin{proof}
	Take arbitrary $\trace \in \mathcal{T}$, by induction on $c$, we have following cases:
		\caseL{$c = [\assign{x}{\expr}]^l$}
		By the evaluation rule $\rname{assn}$, we have
		$
		{
		\config{[\assign{{x}}{\aexpr}]^{l},  \trace } 
		\xrightarrow{} 
		\config{\eskip, \trace \tracecat [({x}, l, v) ]}
		}$.
		\\
		Since there doesn't exist $c_1, c_2 \in \cdom$ satisfying $\eskip = c_1; c_2$, this theorem is vacuously true.
		\caseL{$c = [\assign{x}{\query(\qexpr)}]^l$}
		By the evaluation rule $\rname{query}$, we have
		$
		{
		\config{[\assign{{x}}{\query(\qexpr)}]^{l},  \trace } 
		\xrightarrow{} 
		\config{\eskip, \trace \tracecat [({x}, l, \qval, v) ]}
		}$.
		\\
		Since there doesn't exist $c_1, c_2 \in \cdom$ satisfying $\eskip = c_1; c_2$, this theorem is vacuously true.
		\caseL{$c = \eif([b]^{l}, c_1, c_2)$}
		By the evaluation rule $\rname{query}$ and $\rname{if-f}$, and the label consistency, we know:
		\\
		for all possible $c_{t1}$ and $c_{t2}$ 
		such that $c_t$ has the form $c_t = c_{t1};c_{t2}$;
		\\
		all possible $c_{f1}$ and $c_{f2}$ 
		such that $c_f$ has the form $c_f = c_{f1};c_{f2}$,
		\\
		$c_{t1} \notin c_{t1}$ and $c_{f1} \notin c_{f2}$.
		\\
		Then this theorem is vacuously true.
		\caseL{$c = c_{s1};c_{s2}$}
		By label consistency, we know for every $c_1' \in_c c_{s1}$, $c_1' \notin c_{s2}$.
		\\
		Then by the induction hypothesis on $c_{s1}$ and $c_{s2}$ separately, we have this case proved.
		\caseL{$\ewhile [b]^{l} \edo c$}
		By rule $\rname{while-t}$, we have:
		\[
			\config{{\ewhile [b]^{l} \edo c_w, \trace}}
			\xrightarrow{} 
			\config{{
			c_w; \ewhile [b]^{l} \edo c_w,  \eskip),
			\trace \tracecat [\event]}}
		\]
		%
		If $c_w$ is a sequence command,
		let $c_1 = c_{w1}$ be the any possible command in this sequence, for all possible $c_{w1}$ and $c_{w2}$ 
		such that $c_w$ has the form $c_w = c_{w1};c_{w2}$.
		\\
		Then we have $c_2 = c_{w2};\ewhile [b]^{l} \edo c_w,  \eskip)$ and $c_1 \in_c c_2$.
		\\
		And we also have the existence of $l = l_b, b$ and $c_w$, and $\ewhile [b]^{l} \edo c_w \in_c c_2$ and  $c_1 \in c_w$.
		\\
		If $c_w$ isn't a sequence command, let $c_1 = c_w$, then we have $c_2 = \ewhile [b]^{l} \edo c_w,  \eskip)$ 
		and $c_1 \in_c c_2$.
		\\
		And we also have the existence of $l = l_b, b$ and $c_w$, and $\ewhile [b]^{l} \edo c_w \in_c c_2$ and  $c_1 \in c_w$.
		\\
		This case is proved.
		\\
		By the evaluation rule $\rname{while-f}$, we have
		$
		{
			\config{{\ewhile [b]^{l}, \edo c_w, \trace}}
			\xrightarrow{} 
			\config{{
			[\eskip]^l ,
			\trace \tracecat [((b, l, \efalse))]}}
		}$.
		\\
		Since there doesn't exist $c_1, c_2 \in \cdom$ satisfying $\eskip = c_1; c_2$, this theorem is vacuously true.
	\end{proof}
%
%
\begin{lem}[Only $\eskip$ Command doesn't Produce Event].
	\label{lem:inv_skip}
	For all trace $\trace\in \mathcal{T}$, and $c, c' \in \cdom$,  
	$\config{c, \trace} \rightarrow \config{c', \trace}$ if and only if $c = [\eskip];c'$. 
	\[
		\forall \trace\in \mathcal{T}, c, c' \in \cdom \st
		\config{c, \trace} \rightarrow \config{c', \trace}
		\Leftrightarrow 
		c = [\eskip];c'
	% \footnote{$([\eskip];){}^*$ denotes a sequence command only composed of $[\eskip]$ commands.}
	\]
	\end{lem}
\begin{proof}
	Proved trivially by induction on $c$ and enumerate all operational semantic rules.
\end{proof}
% \begin{lem}[Independent Events Doesn't Block $\flowsto$ for Testing Event]
% 	\label{lem:inv_indepeventstest}
% 	%
% 	For every $D \in \dbdom , c \in \cdom, \trace \in \mathcal{T}$, an assignment event
% 	$\event_1 \in \eventset^{\asn}$ and a test event $\event_2 \in \eventset^{\test}$,
% 	if the trace $trace$ has the form $\trace = [\event_1] \tracecat \trace' \tracecat [\event_2]$ with $\trace' \in \mathcal{T}$, 
% 	and $\eventdep(\event_1, \event_2, \trace, c, D)$,
% 	and every $\event \in \trace'$ doesn't have the \emph{May-Dependency} relations both on $\event_1$ and to $\event_2$,
% 	then 
% 	$\pi_1(\event_1) \in VAR(\pi_1(\event_2))$, and $ {\pi_2(\event_1)} = \llabel(\trace)$
% 	%
% 	\[
% 	\begin{array}{l}
% 		\forall D \in \dbdom , c \in \cdom, \trace \in \mathcal{T} \st \forall \event_1,\in \eventset^{\asn}, \event_2 \in \eventset^{\test} \st
% 		 \exists \trace' \in \mathcal{T} \st \trace = [\event_1] \tracecat \trace' \tracecat [\event_2]
% 		\implies
% 		\eventdep(\event_1, \event_2, \trace, c, D) 
% 		\\ \quad 
% 		\implies 
% 		\left( \forall \event \in \trace' \st \neg \eventdep(\event_1, \event, \trace[\event_1:\event], c, D)
% 		\lor \neg \eventdep(\event, \event_2, \trace[\event:\event_2], c, D) 
% 		\right) 
% 		\\ \quad 
% 		\implies 
% 		\pi_1(\event_1) \in VAR(\pi_1(\event_2)) \land {\pi_2(\event_1)} = \llabel(\trace)
% 	\end{array}
% 	\]
% \end{lem}
%
% \begin{lem}[Flow Search Algorithm ($\mathcal{A}$) Inversion 1]
% \label{lem:inv_alg1}
% For all $D \in \dbdom , c \in \cdom, \trace \in \mathcal{T}, \event_1, \event_2 \in \eventset^{\asn}$, and a list $l$,
% if $l \in \mathcal{A}(\event_1, \event_2, \trace, c, D)$,
% then l must have the form $[\pi_1(\event_1)^{\pi_2(\event_1)},\ldots, \pi_1(\event_2)^{\pi_2(\event_2)}]$.
% \[
% \begin{array}{l}
%     \forall D \in \dbdom , c \in \cdom, \trace \in \mathcal{T}, l \st \forall \event_1, \event_2 \in \eventset^{\asn} \st
%   \\ \quad 
%  l\in \mathcal{A}(\event_1, \event_2, \trace, c, D)  \implies  l = [\pi_1(\event_1)^{\pi_2(\event_1)},\ldots, \pi_1(\event_2)^{\pi_2(\event_2)}]
% \end{array}
% \]
% \end{lem}
% %
% \begin{proof}
% % Let $l \in \mathcal{A}(\event_1, \event_2, \trace, c, D)$. By definition of $\mathcal{A}$ we have 
% % \[l\in \kw{setmap} 
% % 	% \bigcup\limits_{l \in \kw{dfs}(\trace, c, D) \land l = \event_1 :: l'}
% % 	\left(\emap 
% % 		(\efun  \event \to \pi_1(\event)^{\pi_2(\event)})	
% % 	(\efilter 
% % 		(\efun \event \to  \event \in \eventset^{\asn})) \right)
% % 	S
% % \]
% % for $S=\kw{setfilter}
% % 		(\efun l \to \exists l' \st l = \event_1 :: l' ++ [\event_2]) ~ \kw{dfs}(\trace, c, D)$.
% % So, in particular by definition of setmap there is a list $l_1\in S$ such that 
% % \[
% % 	% \bigcup\limits_{l \in \kw{dfs}(\trace, c, D) \land l = \event_1 :: l'}
% % \emap 
% % 		(\efun  \event \to \pi_1(\event)^{\pi_2(\event)})	
% % 	(\efilter 
% % 		(\efun \event \to  \event \in \eventset^{\asn}))
% % 	l_1 = l
% % \]
% % \[
% % l = [\event_1, \cdots, \event_2]
% % \]
% % \\
% Let  $l \in \mathcal{A}(\event_1, \event_2, \trace, c, D)$,
% by definition of $\mathcal{A}$, we know 
% %
% $$l\in \kw{setmap} 
% 	% \bigcup\limits_{l \in \kw{dfs}(\trace, c, D) \land l = \event_1 :: l'}
% 		\left(\efun l \to ( \emap 
% 		(\efun  \event \to \pi_1(\event)^{\pi_2(\event)})
% 	(\efilter 
% 		(\efun \event \to  \event \in \eventset^{\asn}) ~ l) \right)
% 	~ S,
% $$
% %
% where $S=(\kw{setfilter} ~(\efun l \to l = [\event_1, \cdots, \event_2]) ~ (\kw{dfs} \eapp \trace \eapp c \eapp  D))$.
% \\
% Then, by definition of $\kw{setmap}$, we know $l$ is an output of
% \\
% $\left(\efun l \to ( \emap 
% 		(\efun  \event \to \pi_1(\event)^{\pi_2(\event)})
% 	(\efilter 
% 		(\efun \event \to  \event \in \eventset^{\asn}) ~ l) \right)$.
% \\
% Then we know there exists a preimage
% $l_e \in S $
% for $l$ such that 
% $$
% \emap (\efun  \event \to \pi_1(\event)^{\pi_2(\event)}) 
% (\efilter (\efun \event \to  \event \in \eventset^{\asn}) ~ l_e) 
% = l.
% $$
%  %
% Since $l_e \in (\kw{setfilter} ~(\efun l \to l = [\event_1, \cdots, \event_2]) ~ (\kw{dfs} \eapp \trace \eapp c \eapp  D))$,
% \\
% by the $\kw{setfilter}$ function,
% we know only the lists of events in $(\kw{dfs} \eapp \trace \eapp c \eapp  D)$ having the form
% $ [\event_1, \cdots, \event_2] $ are preserved in $S$, i.e.,
% \[
% 	\forall l \in (\kw{setfilter} ~(\efun l \to l = [\event_1, \cdots, \event_2]) ~ \kw{dfs}(\trace, c, D))
% 	\st l = [\event_1, \cdots, \event_2]
% \]
% %
% Then we know $l_e$ also has the same form, 
% i.e., $l_e = [\event_1, \cdots, \event_2]$.
% %
% \\
% Let $l_{ef} = (\efilter (\efun \event \to  \event \in \eventset^{\asn})) ~ l_e$, 
% by $\event_1, \event_2 \in \eventset^{\asn}$, 
% we know $\event_1$ and $\event_2$ are preserved in $l_{ef}$, i.e.,:
% \[
% 	l_{ef} =[\event_1, \cdots, \event_2]
% \]
% %
% Then, by applying the function
% $\emap (\efun  \event \to \pi_1(\event)^{\pi_2(\event)})$ to 
% $l_{ef}$, we have $l$ as follows:
% \[
% 	[\pi_1(\event_1)^{\pi_2(\event_1)}, \cdots, \pi_1(\event_2)^{\pi_2(\event_2)}]
% \]
% %
% %
% This lemma is proved.
% \end{proof}
% %
% \begin{lem}[Flow Search Algorithm ($\mathcal{A}$) Inversion 2]
% \label{lem:inv_alg2}
% For every $\event_1, \event_2 \in \eventset^{\asn}, D \in \dbdom , c \in \cdom$, we have either one of the two following cases:
% \begin{enumerate}
%   \item $\mathcal{A}(\event_1, \event_2,  [\event_1; \event_2], c, D) = 
%   \left\{[\pi_1(\event_1)^{\pi_2(\event_1)}, \pi_1(\event_2)^{\pi_2(\event_2)}] \right \}$ 
%   and $\eventdep(\event_1, \event_2, [\event_1; \event_2], c, D)$.
%   \item  $\mathcal{A}(\event_1, \event_2, [\event_1; \event_2], c, D) = \{\}$ 
%   and $\neg \eventdep(\event_1, \event_2, [\event_1; \event_2] c, D)$;
% \end{enumerate}
% \end{lem}
% % \wq{ Good! I just realize this lemma is only used for case 4 of 5.3.}
% %\jl{Yes!}
% \begin{proof}
% By definition of $A$, we know:
% %
% \[
% 	\begin{array}{l}
% 	\mathcal{A}(\event_1, \event_2, [\event_1; \event_2], c, D)
% 	= 
% 	\kw{setmap} ~
% 	% \bigcup\limits_{l \in \kw{dfs}(\trace, c, D) \land l = \event_1 :: l'}
% 	\\ \qquad \qquad
% 	\left(\efun l \to ( \emap 
% 		(\efun  \event \to \pi_1(\event)^{\pi_2(\event)})
% 	(\efilter 
% 		(\efun \event \to  \event \in \eventset^{\asn}) ~ l) \right)
% 	\\ \qquad \qquad
% 	(\kw{setfilter} ~
% 		(\efun l \to l = [\event_1, \cdots, \event_2]) ~ 
% 		% \left(\left\{[\event_2]\right\} \cup \left\{ \event_1 \stackrel{[\event_1; \event_2]}{\uplus} [\event_2] \right\} \right))
% 		\left(\left\{[\event_2]\right\} \cup \left(  {\uplus} \eapp \event_1 \eapp {[\event_1; \event_2]} \eapp [\event_2] \right) \right))
% 	\end{array}
% \]
% by definition of $ {\uplus} \eapp \event_1 \eapp {[\event_1; \event_2]} \eapp [\event_2]  $, we know 
% \[
% 	\begin{array}{l}
% 	% \event_1 \stackrel{[\event_1; \event_2]}{\uplus} [\event_2]
% 	{\uplus} \eapp \event_1 \eapp {[\event_1; \event_2]} \eapp [\event_2] 
% 	=   
% 	\\ \quad \qquad 	
% 	\ecase \eventdep(\event_1, \event_2, [\event_1; \event_2], c, D)
% 	\to \left\{ [\event_1, \event_2] \right\}
% 	\\ \quad \qquad 	
% 	\ecase \_
% 	\to \left\{ \right\}
% \end{array}
% \]
% %
% By simplification of the $\kw{setfilter}$, $\emap$, $\efilter$ and $\kw{setmap}$ functions, we know
% \\
% in the case of $\eventdep(\event_1, \event_2, [\event_1; \event_2], c, D)$:
% % \\
% $\mathcal{A}(\event_1, \event_2, [\event_1; \event_2], c, D) = 
%   \left\{[\pi_1(\event_1)^{\pi_2(\event_1)}, \pi_1(\event_2)^{\pi_2(\event_2)}] \right \}$
% \\
% (1) is proved.
% \\
% And in the case of $\neg \eventdep(\event_1, \event_2, [\event_1; \event_2], c, D)$: 
% % \\
% $\mathcal{A}(\event_1, \event_2, \cdot  \event_1 \tracecat [\event_2], c, D) = 
%   \left\{ \right \}$
% \\
% (2) is proved.
% \end{proof}
%
%
% \begin{lem}[\todo{Assignment Evaluation Inversion}].
% 	\label{lem:inv_eval_asn}
% 	\[
% 	\begin{array}{l}
% 		\forall x \in \lvar_c, \kw{V_{ptl}} \in \subseteq \lvar_c \expr \st 
% 		\exists \trace, \trace' \in \mathcal{T} \st 
% 		\\ \quad
% 		\forall z \in \lvar_c \setminus (\{x\} \cup \kw{V_{ptl}}) \st 
% 		\env(\trace) z = \env(\trace') z 
% 		\\ \quad \land
% 		\forall \event \in \trace, \event' \in \trace' \st 
% 		\pi_1(\event) \in \kw{V_{ptl}} \land \diff(\event, \event') 
% 		\\ \quad
% 		\implies 
% 		\neg \eventdep(\event, \event_y, \trace[\event:\event_y] ) 
% 		\implies
% 		\config{\trace, [\assign{y}{\expr}]{}^l;c'} \rightarrow^{asn} \config{\trace\cdot \event_y, c'}
% 		\\ \quad
% 		\implies 
% 		\config{\trace', [\assign{y}{\expr}]{}^l;c'} \rightarrow^{asn} \config{\trace'\cdot \event_y',c'}
% 		\land \diff(\event_y, \event_y')
% 		\implies x \in VAR(\expr)
% 	\end{array}
% 	\]
% 	\end{lem}	
%
%
% \todo{Event Dependency Transitivity}
\begin{lem}
	\label{lem:valdep_trans}
(Value Dependency Transitivity)
For every $D \in \dbdom , c \in \cdom, \trace \in \mathcal{T}$, and $\event_1, \event_2, \event_3 \in \eventset^{\asn}, \trace_{12}, \trace_{23} \in \mathcal{T}$,
if $\eventdep(\event_1, \event_2, \trace_{12}, c, D)$
and $\eventdep(\event_2, \event_3, \trace_{23}, c, D) $,
then $\eventdep(\event_1, \event_3, \trace_{12}\tracecat\trace_{23}, c, D)$.
  % An event $\event_2 \in \eventset^{\asn}$ is in the \emph{may-dependency} relation with another
  % event $\event_1 \in \eventset^{\asn}$ in a program ${c}$ with a hidden database $D$, denoted as $\eventdep(\event_1, \event_2, c, D)$,
  % if and only if
  \[
	  \begin{array}{l}
  \forall D \in \dbdom , c \in \cdom, \event_1, \event_2, \event_3 \in \eventset^{\asn}, \trace_{12}, \trace_{23} \in \mathcal{T} \st 
  \eventdep(\event_1, \event_2, \trace_{12}, c, D) 
  \land
  \eventdep(\event_2, \event_3, \trace_{23}, c, D) 
  \\ \quad
  \implies
  \eventdep(\event_1, \event_3, \trace_{12}\tracecat\trace_{23}, c, D)
	  \end{array}
  \]
\end{lem}
%
%
\begin{lem}(Control Dependency Transitivity)
\label{lem:ctl_trans}
For every $D \in \dbdom , c \in \cdom, \event_1, \event_2 \in \eventset^{\test}, \event_3 \in \eventset$
if $\eventdep^{\ctl}(\event_1, \event_2, c, D)$ and $\eventdep^{\ctl}(\event_2, \event_3, c, D)$,
then $\eventdep^{\ctl}(\event_1, \event_3, c, D)$.
%
\[
  \forall D \in \dbdom , c \in \cdom, \event_1, \event_2 \in \eventset^{\test}, \event_3 \in \eventset \st
  \eventdep^{\ctl}(\event_1, \event_2, c, D) 
  \land \eventdep^{\ctl}(\event_2, \event_3, c, D)
  \implies \eventdep^{\ctl}(\event_1, \event_3, c, D)
\]
\end{lem}
%
% \begin{lem}
% 	\label{lem:eventdep_trans}
% (\emph{Variable May-Dependency} Transitivity)
% 	\[
% 	\forall c \in \cdom, D \in \dbdom , \event_1, \event_2, \event_3 \in \eventset^{\asn}\st 
% 	\eventdep(\event_1, \event_2, c, D) 
% 	\land
% 	\eventdep(\event_2, \event_3, c, D) 
% 	\implies
% 	\eventdep(\event_1, \event_3, c, D)
% 	\]
%   \end{lem}
%
\begin{lem}[\emph{Variable May-Dependency} Transitivity]
	\label{lem:vardep_trans}
For every $c \in \cdom, x^i, y^j, z^l \in \lvar_c$, 
if $\vardep(x^i, y^j, c)$ and 
$\vardep(y^j, z^l, c)$, then $\vardep(x^i, z^l, c)$.
	\[
	\forall c \in \cdom, x^i, y^j, z^l \in \lvar_c \st 
	\vardep(x^i, y^j, c) 
	\land
	\vardep(y^j, z^l, c) 
	\implies
	\vardep(x^i, z^l, c)
	\]
  \end{lem}
  %
%

\clearpage
% \section{Proof of the Correctness of $\flowsto$ Searching Algorithm}
% \begin{defn}[The $\flowsto$ Searching Algorithm]
\[
\begin{array}{l}
	\mathcal{A}:(\eventset \times \eventset \times \mathcal{T} \times \cdom \times \dbdom )\to \mathcal{P}(List(\eventset))
	\\
	\mathcal{A}(\event_1, \event_2, \trace, c, D) 
	= 
	\kw{setmap} ~
	% \bigcup\limits_{l \in \kw{dfs}(\trace, c, D) \land l = \event_1 :: l'}
	\\ \qquad \qquad
	\left(\efun l \to ( \emap 
		(\efun  \event \to \pi_1(\event)^{\pi_2(\event)})
	(\efilter 
		(\efun \event \to  \event \in \eventset^{\asn}) ~ l) \right)
	\\ \qquad \qquad
	(\kw{setfilter} ~
		(\efun l \to l = [\event_1, \cdots, \event_2]) ~ (\kw{dfs} \eapp \trace \eapp c \eapp  D ))
	\end{array}
\]
%
% %
% \[
% \begin{array}{l}
% 	\kw{dfs}(\trace, c, D)
% 	= 
% 	\ematch~  \trace ~ \ewith
% 	\\ \qquad
% 	| \event \to \{[\event]\}
% 	\\ \qquad
% 	|  \cdot \event \tracecat \trace' \to  
% 	% \\ \quad \qquad
% 	 \bigcup\limits_{l \in \kw{dfs}(\trace', c, D) }
% 	\left(  \eif \event \stackrel{\trace}{\uplus} l \neq [] 
% 	\ethen \left\{ \event \stackrel{\trace}{\uplus} l \right\} \eelse \left\{ l \right\}
% 	\right)
% \end{array}
% \]
%
\[
\begin{array}{l}
	\kw{dfs} : \mathcal{T} \to \cdom \to \dbdom \to \mathcal{P}(List(\eventset))
	\\
	\kw{dfs} \eapp \trace \eapp c \eapp  D
	= 
	\ematch~  \trace ~ \ewith
	\\ \qquad
	| \event \to \{[\event]\}
	\\ \qquad
	|  \cdot \event \tracecat \trace' \to  
	% \\ \quad \qquad
	(\kw{dfs} \eapp \trace'  \eapp c  \eapp D)
	\cup 
	\left(   \bigcup\limits_{l \in \kw{dfs} \eapp \trace' \eapp c \eapp  D }
	\left( \uplus \eapp  \event  \eapp \trace  \eapp l \right)
	\right)
	\\ \qquad
	| \cdot \to \{\}
\end{array}
\]
%
%
\[
\begin{array}{l}
	\uplus : \eventset \to \mathcal{T} \to List(\eventset) \to \mathcal{P}(List(\eventset))
	\\
	% \event \stackrel{\trace}{\uplus} 
	\uplus \eapp  \event  \eapp \trace  \eapp l
	= 
	\ematch~  l ~ \ewith
	\\ \qquad
	| [] \to \{\}
	\\ \qquad
	|  \event' :: l' \to  
	\\ \quad \qquad 	
	\ecase \event \in \eventset^{\asn}  \land \eventdep^{val}(\event, \event', \trace[\event:\event'], c, D)
	\to \left\{ \event::(\event' :: l') \right\}
	\\ \quad \qquad 	
	\ecase \event \in \eventset^{\test} \land \eventdep^{\ctl}(\event, \event', c, D)
	\to \left\{ \event::(\event' :: l') \right\}\cup 
	% \left( \event \stackrel{\trace}{\uplus} l' \right)
	\left( \uplus \eapp  \event  \eapp \trace  \eapp l' \right)
	\\ \quad \qquad 	
	% \ecase \_ \to \event \stackrel{\trace}{\uplus} l' 
	\ecase \_ \to \uplus \eapp  \event  \eapp \trace  \eapp l'  
\end{array}
\]
\end{defn}
%
\begin{thm}[Algorithm Soundness w.r.t. $\flowsto$ and $\eventdep$]
% (Algorithm Soundness)
\label{thm:alg_correct}
\[
\begin{array}{l}
  \forall D \in \dbdom , c \in \cdom, \trace \in \mathcal{T} \st \forall \event_1, \event_2 \in \eventset^{\asn} \st
  \\ \quad 
   \exists \trace' \in \mathcal{T} \st \trace = \cdot \event_1 \tracecat \trace' \cdot \event_2
   \implies    \forall  z^i, y^j \in \lvar_c, l_h, l_t \st 
  \\ \qquad 
   l_h ++ [z^i, y^j] ++ l_t \in \mathcal{A}(\event_1, \event_2, \trace, c, D)
   \implies \flowsto(z^i, y^j, c) \land \eventdep(\event_1, \event_2, \trace, {c}, D)
    % \land \vardep({x}_1^{l_1}, {x}_2^{l_2}, {c})
\end{array}
\]
\end{thm}
%
% \begin{thm}[Algorithm Completeness w.r.t. $\flowsto$]
% \label{thm:alg_complete}
% \[
% \begin{array}{l}
%   \forall c, z^i, y^j \in \lvar_c \st 
%    \flowsto(z^i, y^j)
%   \\ \quad 
%    \implies \exists \event_1, \event_2 \in \eventset^{\asn}, \trace \in \mathcal{T}, D \in \dbdom \st
%    l_h, l_t \st 
%    l_h ++ [z^i, y^j] ++ l_t \in \mathcal{A}(\event_1, \event_2, \trace, c, D)
% \end{array}
% \]
% \end{thm}
%
\begin{thm}[Completeness of Algorithm w.r.t. $\eventdep$].
\\
\label{thm:algeventdep_sound}.
\[
\begin{array}{l}
	\forall D \in \dbdom , c \in \cdom, \trace \in \mathcal{T} \st \forall \event_1, \event_2 \in \eventset^{\asn} \st
	\\ \quad 
   \exists \trace' \in \mathcal{T} \st \trace = \cdot \event_1 \tracecat \trace' \cdot \event_2
  \implies \eventdep(\event_1, \event_2, \trace, {c}, D)
   \implies
   \mathcal{A}(\event_1, \event_2, \trace, c, D) \neq \emptyset
\end{array}
\]
\end{thm}
%
\jl{the algorithm is (complete w.r.t. $\eventdep$ while incomplete w.r.t. $\flowsto$ )
}
% \clearpage
% We first Proved following Inversion Lemmas.
% \begin{lem}[Arithmetic Inversion]
% \label{lem:inv_a}
% For all {$ x^i \in \lvar$, and $\trace, \trace' \in \mathcal{T}$,  and arithmetic expression $\aexpr$}, if
% $ \forall z^j \in \lvar / \{x^i\} \st 
% \env(\trace) z = \env(\trace') z $ and $\config{\trace, \aexpr} \aarrow v $ and 
% $\config{\trace', \aexpr} \aarrow v' $ with $v' \neq v$, then $ x $ is in the free variables of $\aexpr$, i.e., $x \in VAR(\aexpr)$.
% %
% % \[
% 	% \forall x^i \in \lvar, \trace, \trace' \in \mathcal{T}, \aexpr \st
% % 	\Big( \forall z^j \in \lvar / \{x^i\} \st 
% % 	\env(\trace) z = \env(\trace') z\Big) \land 
% % 	\config{\trace, \expr} \aarrow v \land \config{\trace', \expr} \aarrow v' \land v \neq v'
% % 	\implies x \in VAR(\expr)
% % \]
% \end{lem}
% %
% %
% \begin{lem}[Boolean Inversion]
% \label{lem:inv_b}
% For all {$ x^i \in \lvar$, and $\trace, \trace' \in \mathcal{T}$,  and boolean expression $\bexpr$}, if
% $ \forall z^j \in \lvar / \{x^i\} \st 
% \env(\trace) z = \env(\trace') z $ and $\config{\trace, \bexpr} \barrow v $ and 
% $\config{\trace', \bexpr} \barrow v' $ with $v' \neq v$, then $ x $ is in the free variables of $\bexpr$, i.e., $x \in VAR(\bexpr)$.
% % \[
% % 	text{(\forall x^i \in \lvar, \trace, \trace' \in \mathcal{T}, \bexpr \st )}
% % 	\Big(\forall z^j \in \lvar / \{x^i\} \st
% % 	\env(\trace) z = \env(\trace') z\Big) \land
% % 	\config{\trace, \bexpr} \barrow v \land \config{\trace', \bexpr} \barrow v' \land v \neq v'
% % 	\implies x \in VAR(\bexpr) 
% % \]
% \end{lem}
% %
% \begin{lem}[Query Inversion]
% \label{lem:inv_q}
% For all {$ x^i \in \lvar$, and $\trace, \trace' \in \mathcal{T}$,  and query expression $\qexpr$}, if
% $ \forall z^j \in \lvar / \{x^i\} \st 
% \env(\trace) z = \env(\trace') z $ and $\config{\trace, \qexpr} \qarrow \qval $ and 
% $\config{\trace', \qexpr} \qarrow \qval' $ with $\qval \neq_{q} \qval'$, then $ x $ is in the free variables of $\qexpr$, i.e., $x \in VAR(\qexpr)$.
% % \[
% % 	\forall x^i \in \lvar, \trace, \trace' \in \mathcal{T}, \qexpr \st 
% % 	\left(\forall z^j \in \lvar / \{x^i \} \st
% % 	\env(\trace) z = \env(\trace') z \right) \implies 
% % 	\config{\trace, \qexpr} \qarrow \qval \land \config{\trace', \qexpr} \qarrow \qval' 
% % 	\implies \qval \neq_{q} \qval'
% % 	\implies x \in VAR(\qexpr) 
% % \]
% \end{lem}
%
\begin{lem}[Expression Inversion]
	\label{lem:inv_expr}
	For all {$ x^i \in \lvar$, and $\trace, \trace' \in \mathcal{T}$, and an expression $\expr$} if
	$ \forall z^j \in \lvar / \{x^i\} \st 
	\env(\trace) z = \env(\trace') z $, and if
	\begin{itemize}
		\item $\expr$ is an arithmetic expression $\aexpr$,
		% \\ 
		and $\config{\trace, \aexpr} \aarrow v $ and 
	$\config{\trace', \aexpr} \aarrow v' $ with $v' \neq v$, 
	then $ x $ is in the free variables of $\aexpr$, i.e., $x \in VAR(\aexpr)$.
%
	\item $\expr$ is a boolean expression $\bexpr$,
	% \\
  and $\config{\trace, \bexpr} \barrow v $ and 
 $\config{\trace', \bexpr} \barrow v' $ with $v' \neq v$, then $ x $ is in the free variables of $\bexpr$, i.e., $x \in VAR(\bexpr)$.
% 
	\item $\expr$ is a query expression $\qexpr$,
	% \\
	and $\config{\trace, \qexpr} \qarrow \qval $ and 
	$\config{\trace', \qexpr} \qarrow \qval' $ with $\qval \neq_{q} \qval'$, then $ x $ is in the free variables of $\qexpr$, i.e., $x \in VAR(\qexpr)$.
\end{itemize}	%
	\end{lem}
	%
	%
	% \begin{lem}[Boolean Inversion]
	% \label{lem:inv_b}
	% For all {$ x^i \in \lvar$, and $\trace, \trace' \in \mathcal{T}$,  and boolean expression $\bexpr$}, if
	% $ \forall z^j \in \lvar / \{x^i\} \st 
	% \env(\trace) z = \env(\trace') z $ and $\config{\trace, \bexpr} \barrow v $ and 
	% $\config{\trace', \bexpr} \barrow v' $ with $v' \neq v$, then $ x $ is in the free variables of $\bexpr$, i.e., $x \in VAR(\bexpr)$.
	% % \[
	% % 	text{(\forall x^i \in \lvar, \trace, \trace' \in \mathcal{T}, \bexpr \st )}
	% % 	\Big(\forall z^j \in \lvar / \{x^i\} \st
	% % 	\env(\trace) z = \env(\trace') z\Big) \land
	% % 	\config{\trace, \bexpr} \barrow v \land \config{\trace', \bexpr} \barrow v' \land v \neq v'
	% % 	\implies x \in VAR(\bexpr) 
	% % \]
	% \end{lem}
	% %
	% \begin{lem}[Query Inversion]
	% \label{lem:inv_q}
	% For all {$ x^i \in \lvar$, and $\trace, \trace' \in \mathcal{T}$,  and query expression $\qexpr$}, if
	% $ \forall z^j \in \lvar / \{x^i\} \st 
	% \env(\trace) z = \env(\trace') z $ and $\config{\trace, \qexpr} \qarrow \qval $ and 
	% $\config{\trace', \qexpr} \qarrow \qval' $ with $\qval \neq_{q} \qval'$, then $ x $ is in the free variables of $\qexpr$, i.e., $x \in VAR(\qexpr)$.
	% % \[
	% % 	\forall x^i \in \lvar, \trace, \trace' \in \mathcal{T}, \qexpr \st 
	% % 	\left(\forall z^j \in \lvar / \{x^i \} \st
	% % 	\env(\trace) z = \env(\trace') z \right) \implies 
	% % 	\config{\trace, \qexpr} \qarrow \qval \land \config{\trace', \qexpr} \qarrow \qval' 
	% % 	\implies \qval \neq_{q} \qval'
	% % 	\implies x \in VAR(\qexpr) 
	% % \]
	% \end{lem}
%
% \begin{lem}[Arithmetic Inversion Generalization]
% 	\label{lem:inv_a_gnl}
% 	% For all {$ x^i \in \lvar$, and $\trace, \trace' \in \mathcal{T}$,  and arithmetic expression $\aexpr$}, if
% 	% $ \forall z^j \in \lvar / \{x^i\} \st 
% 	% \env(\trace) z = \env(\trace') z $ and $\config{\trace, \aexpr} \aarrow v $ and 
% 	% $\config{\trace', \aexpr} \aarrow v' $ with $v' \neq v$, then $ x $ is in the free variables of $\aexpr$, i.e., $x \in VAR(\aexpr)$.
% 	%
% 	For all subset of the labeled variables $\diff \subset \lvar$, and $x^i \in (\lvar \setminus \diff)$,
% 	and an arithmetic expression $\aexpr$,
% 	if for all $z^j \in \lvar \setminus \diff, \trace, \trace' \in \mathcal{T}, v, v'$ such that 
% 	$\env(\trace) z = \env(\trace') z$, and 
% 	$
% 	\config{\trace, \aexpr} \aarrow v$, and $\config{\trace', \aexpr} \aarrow v'$ with $v = v'$;
% 	and for all $z^j \in \lvar / (\diff \cup \{x^i\} )$ 
% 	there exist $\trace, \trace' \in \mathcal{T}, v, v'$ such that 
% 	$\env(\trace) z = \env(\trace') z$, and 
% 	$
% 	\config{\trace, \aexpr} \aarrow v$, and $\config{\trace', \aexpr} \aarrow v'$ with $v \neq v'$,
% 	then $x \in VAR(\aexpr)$ and $i = \llabel(\trace) x$.
% 	\[
% 		\begin{array}{l}
% 		\forall \diff \subset \lvar,  x^i \in (\lvar \setminus \diff), \aexpr \st
% 		\\ \quad
% 		\forall z^j \in \lvar \setminus \diff, \trace, \trace' \in \mathcal{T}, v, v' \st 
% 		\env(\trace) z = \env(\trace') z \land 
% 		\config{\trace, \aexpr} \aarrow v \land \config{\trace', \aexpr} \aarrow v' \land v = v'
% 		\\ \quad
% 		\implies 
% 		\forall z^j \in \lvar / (\diff \cup \{x^i\} ) \st 
% 		 \exists \trace, \trace' \in \mathcal{T}, v, v'\st 
% 		\env(\trace) z = \env(\trace') z \land 
% 		\config{\trace, \aexpr} \aarrow v \land \config{\trace', \aexpr} \aarrow v' \land v \neq v'
% 		\\ \qquad
% 		\implies x \in VAR(\aexpr) \land i = \llabel(\trace) x
% 		\end{array}
% 	\]
% 	\end{lem}
% \begin{proof}.
% 	\\
% 	To show $x \in VAR(\aexpr)$, by showing contradiction ($\forall \trace, \trace'$ in second hypothesis  $v = v'$)
% 	 if $x \notin VAR(\aexpr)$.
% 	 \\
% 	To show $i = \llabel(\trace)$, by showing contradiction ($\forall \trace, \trace'$ in second hypothesis  $v = v'$ ) 
% 	if $j = \llabel(\trace) x$ and $i \neq j$.
% \end{proof}
% 	%
% \begin{lem}[Boolean Inversion Generalization]
% 		\label{lem:inv_b_gnl}
% 		% For all {$ x^i \in \lvar$, and $\trace, \trace' \in \mathcal{T}$,  and arithmetic expression $\aexpr$}, if
% 		% $ \forall z^j \in \lvar / \{x^i\} \st 
% 		% \env(\trace) z = \env(\trace') z $ and $\config{\trace, \aexpr} \aarrow v $ and 
% 		% $\config{\trace', \aexpr} \aarrow v' $ with $v' \neq v$, then $ x $ is in the free variables of $\aexpr$, i.e., $x \in VAR(\aexpr)$.
% 		%
% 		\[
% 			\begin{array}{l}
% 			\forall \diff \subset \lvar,  x^i \in (\lvar \setminus \diff), \bexpr \st
% 			\\ \quad
% 			\forall z^j \in \lvar \setminus \diff, \trace, \trace' \in \mathcal{T}, v, v' \st 
% 			\env(\trace) z = \env(\trace') z \land 
% 			\config{\trace, \bexpr} \barrow v \land \config{\trace', \bexpr} \barrow v' \land v = v'
% 			\\ \quad
% 			\implies 
% 			\forall z^j \in \lvar / (\diff \cup \{x^i\} ) \st 
% 			 \exists \trace, \trace' \in \mathcal{T}, v, v'\st 
% 			\env(\trace) z = \env(\trace') z \land 
% 			\config{\trace, \bexpr} \barrow v \land \config{\trace', \bexpr} \barrow v' \land v \neq v'
% 			\\ \qquad
% 			\implies x \in VAR(\bexpr) \land i = \llabel(\trace)
% 			\end{array}
% 		\]
% \end{lem}%
\begin{lem}[Expression Inversion Generalization]
	\label{lem:inv_expr_gnl}
	For all subset of the labeled variables $\diff \subset \lvar$, and $x^i \in (\lvar \setminus \diff)$,
	and an expression $\expr$, if 
	\begin{itemize}
		\item $\expr$ is an arithmetic expression $\aexpr$,
		% \\ 
		and for all $z^j \in \lvar \setminus \diff, \trace, \trace' \in \mathcal{T}, v, v'$ such that 
		$\env(\trace) z = \env(\trace') z$, and 
		$
		\config{\trace, \aexpr} \aarrow v$, and $\config{\trace', \aexpr} \aarrow v'$ with $v = v'$;
		and for all $z^j \in \lvar / (\diff \cup \{x^i\} )$ 
		there exist $\trace, \trace' \in \mathcal{T}, v, v'$ such that 
		$\env(\trace) z = \env(\trace') z$, and 
		$
		\config{\trace, \aexpr} \aarrow v$, and $\config{\trace', \aexpr} \aarrow v'$ with $v \neq v'$,
		then $x \in VAR(\aexpr)$ and $i = \llabel(\trace) x$.
		\[
			\begin{array}{l}
			\forall \diff \subset \lvar,  x^i \in (\lvar \setminus \diff), \aexpr \st
			\\ \quad
			\forall z^j \in \lvar \setminus \diff, \trace, \trace' \in \mathcal{T}, v, v' \st 
			\env(\trace) z = \env(\trace') z \land 
			\config{\trace, \aexpr} \aarrow v \land \config{\trace', \aexpr} \aarrow v' \land v = v'
			\\ \quad
			\implies 
			\forall z^j \in \lvar / (\diff \cup \{x^i\} ) \st 
			\exists \trace, \trace' \in \mathcal{T}, v, v'\st 
			\env(\trace) z = \env(\trace') z \land 
			\config{\trace, \aexpr} \aarrow v \land \config{\trace', \aexpr} \aarrow v' \land v \neq v'
			\\ \qquad
			\implies x \in VAR(\aexpr) \land i = \llabel(\trace) x
			\end{array}
		\]
	\item $\expr$ is a boolean expression $\bexpr$,
	and for all $ z^j \in \lvar \setminus \diff, \trace, \trace' \in \mathcal{T}, v, v'$ such that 
	$ \env(\trace) z = \env(\trace') z \land 
	\config{\trace, \bexpr} \barrow v \land \config{\trace', \bexpr} \barrow v' \land v = v'$;
	and for all
	$ z^j \in \lvar / (\diff \cup \{x^i\} ) \st 
	 \exists \trace, \trace' \in \mathcal{T}, v, v'\st 
	\env(\trace) z = \env(\trace') z \land 
	\config{\trace, \bexpr} \barrow v \land \config{\trace', \bexpr} \barrow v' \land v \neq v'$
	then 
	 $x \in VAR(\bexpr) \land i = \llabel(\trace)$
	% \\
	\[
		\begin{array}{l}
		\forall \diff \subset \lvar,  x^i \in (\lvar \setminus \diff), \bexpr \st
		\\ \quad
		\forall z^j \in \lvar \setminus \diff, \trace, \trace' \in \mathcal{T}, v, v' \st 
		\env(\trace) z = \env(\trace') z \land 
		\config{\trace, \bexpr} \barrow v \land \config{\trace', \bexpr} \barrow v' \land v = v'
		\\ \quad
		\implies 
		\forall z^j \in \lvar / (\diff \cup \{x^i\} ) \st 
		 \exists \trace, \trace' \in \mathcal{T}, v, v'\st 
		\env(\trace) z = \env(\trace') z \land 
		\config{\trace, \bexpr} \barrow v \land \config{\trace', \bexpr} \barrow v' \land v \neq v'
		\\ \qquad
		\implies x \in VAR(\bexpr) \land i = \llabel(\trace)
		\end{array}
	\]
% 
	\item $\expr$ is a query expression $\qexpr$,
	and for all $\diff \subset \lvar,  x^i \in (\lvar \setminus \diff), \qexpr$ such that 
	for all $ z^j \in \lvar \setminus \diff, \trace, \trace' \in \mathcal{T}, \qval, \qval' \st 
 \env(\trace) z = \env(\trace') z \land 
 \config{\trace, \qexpr} \qarrow \qval \land \config{\trace', \qexpr} \qarrow \qval' \land \qval =_q \qval'$;
 and for all 
	$ z^j \in \lvar / (\diff \cup \{x^i\} ) \st 
  \exists \trace, \trace' \in \mathcal{T}, \qval, \qval'\st 
 \env(\trace) z = \env(\trace') z \land 
 \config{\trace, \qexpr} \qarrow \qval \land \config{\trace', \qexpr} \qarrow \qval' \land \qval \neq_{q} \qval'$,
 then  $x \in VAR(\qexpr) \land i = \llabel(\trace)$.
	% \\
	\[
		\begin{array}{l}
		\forall \diff \subset \lvar,  x^i \in (\lvar \setminus \diff), \qexpr \st
		\\ \quad
		\forall z^j \in \lvar \setminus \diff, \trace, \trace' \in \mathcal{T}, \qval, \qval' \st 
		\env(\trace) z = \env(\trace') z \land 
		\config{\trace, \qexpr} \qarrow \qval \land \config{\trace', \qexpr} \qarrow \qval' \land \qval =_q \qval'
		\\ \quad
		\implies 
		\forall z^j \in \lvar / (\diff \cup \{x^i\} ) \st 
		 \exists \trace, \trace' \in \mathcal{T}, \qval, \qval'\st 
		\env(\trace) z = \env(\trace') z \land 
		\config{\trace, \qexpr} \qarrow \qval \land \config{\trace', \qexpr} \qarrow \qval' \land \qval \neq_{q} \qval'
		\\ \qquad
		\implies x \in VAR(\qexpr) \land i = \llabel(\trace)
		\end{array}
	\]
	\end{itemize}
	\end{lem}
	%
\begin{proof}.
	\\
	To show $x \in VAR(\aexpr)$, by showing contradiction ($\forall \trace, \trace'$ in second hypothesis  $v = v'$)
	 if $x \notin VAR(\aexpr)$.
	 \\
	To show $i = \llabel(\trace)$, by showing contradiction ($\forall \trace, \trace'$ in second hypothesis  $v = v'$ ) 
	if $j = \llabel(\trace) x$ and $i \neq j$.
\end{proof}
	%
% \begin{lem}[Boolean Inversion Generalization]
% 		\label{lem:inv_b_gnl}
% 		% For all {$ x^i \in \lvar$, and $\trace, \trace' \in \mathcal{T}$,  and arithmetic expression $\aexpr$}, if
% 		% $ \forall z^j \in \lvar / \{x^i\} \st 
% 		% \env(\trace) z = \env(\trace') z $ and $\config{\trace, \aexpr} \aarrow v $ and 
% 		% $\config{\trace', \aexpr} \aarrow v' $ with $v' \neq v$, then $ x $ is in the free variables of $\aexpr$, i.e., $x \in VAR(\aexpr)$.
% 		%
% 		\[
% 			\begin{array}{l}
% 			\forall \diff \subset \lvar,  x^i \in (\lvar \setminus \diff), \bexpr \st
% 			\\ \quad
% 			\forall z^j \in \lvar \setminus \diff, \trace, \trace' \in \mathcal{T}, v, v' \st 
% 			\env(\trace) z = \env(\trace') z \land 
% 			\config{\trace, \bexpr} \barrow v \land \config{\trace', \bexpr} \barrow v' \land v = v'
% 			\\ \quad
% 			\implies 
% 			\forall z^j \in \lvar / (\diff \cup \{x^i\} ) \st 
% 			 \exists \trace, \trace' \in \mathcal{T}, v, v'\st 
% 			\env(\trace) z = \env(\trace') z \land 
% 			\config{\trace, \bexpr} \barrow v \land \config{\trace', \bexpr} \barrow v' \land v \neq v'
% 			\\ \qquad
% 			\implies x \in VAR(\bexpr) \land i = \llabel(\trace)
% 			\end{array}
% 		\]
% \end{lem}%
%
% \begin{lem}[Assignment Inversion].
% \label{lem:inv_a}
% \[
% 	\forall x \in \lvar, \expr \st 
% 	\Big( \exists \trace, \trace' \st \forall z^i \in \lvar / \{x^l\} \st 
% 	\env(\trace) z = \env(\trace') z \st 
% 	\config{\trace, \expr} \aarrow v \land \config{\trace', \expr} \aarrow v' \land v  \neq v'
% 	\implies x \in VAR(\expr) \land x^l \in \Big)
% \]
% \end{lem}
%
% \begin{lem}[Assignment Event Inversion]
% \label{lem:inv_asn}
% \[
% \begin{array}{l}
% 	\forall \trace_0 \in \mathcal{T}, c \in \cdom,
% 	\event \in \eventset^{\asn} \st
% 	\config{c, \trace_0} \rightarrow^* \config{\eskip, \trace_0 \tracecat \trace_1} \implies
% 	(\event \eventin \trace_1 \land	x = \pi_1(\event) \land l = \pi_2(\event))
% 	\\ 
% 	\implies 
% 	\big( 
% 		\exists \trace_1' \in \mathcal{T}, \expr, c' \st
% 		\config{c, \trace_0} \rightarrow^* \config{ [\assign{x}{\expr}]^l;c', \trace_0  \tracecat  \trace'} \rightarrow^{assn}
% 		\config{c', \trace_0 \tracecat \trace_1'\cdot \event} \rightarrow^{*}
% 		\config{\eskip, \trace_0  \tracecat  \trace_1}
% 	\big)
% 	\\ \qquad \lor
% 	\big( 
% 		\exists \trace_1' \in \mathcal{T}, \qexpr, c' \st
% 		\config{c, \trace_0} \rightarrow^* \config{ [\assign{x}{\query(\qexpr)}]^l;c', \trace_0 \tracecat \trace_1'} \rightarrow^{query}
% 		\config{c', \trace_0  \tracecat  \trace_1' \cdot \event } \rightarrow^{*}
% 		\config{ \eskip, \trace_0\trace_1}
% 	\big)
% \end{array}
% \]
% %
% \end{lem}
% %
% \begin{lem}[Testing Event Inversion]
% \label{lem:inv_test}
% \[
% \begin{array}{l}
% 	\forall c\in \cdom, \trace_0 \in \mathcal{T}, \event = (b, l, n, v) \in \eventset^{\test} \st
% 	 \config{c, \trace_0} \rightarrow^* \config{\eskip, \trace_0 \tracecat \trace_1}
% 	\implies  \event \eventin \trace_1 \\
% 	\implies 
% 	\big( 
% 		\exists \trace_1' \in \mathcal{T}, \bexpr, c', c_t, c_f, c'' \in \cdom \st
% 		\config{c, \trace_0} \rightarrow^* \config{\eif ([b]^l, c_t, c_f);c', \trace_0 \tracecat \trace_1'} \rightarrow^{if-b}
% 		\config{c'', \trace_0 \tracecat \trace_1'\cdot \event } \rightarrow^{*}
% 		\config{\eskip, \trace_0 \tracecat \trace_1} 
% 	\big)
% 	\\ \qquad \lor
% 	\big( 
% 		\exists \trace_1' \in \mathcal{T}, \bexpr, c', c_w, c'' \in \cdom \st
% 		\config{c, \trace_0} \rightarrow^* \config{ \ewhile([b]^l, c_w);c', \trace_0 \tracecat  \trace_1'} \rightarrow^{while-b}
% 		\config{c'', \trace_0 \tracecat \trace_1'\cdot \event } \rightarrow^{*}
% 		\config{\eskip, \trace_0  \tracecat \trace_1}
% 	\big)
% \end{array}
% \]
% \end{lem}
\begin{lem}[Event Inversion]
\label{lem:inv_event}
For all $\forall c\in \cdom, \trace_0 \in \mathcal{T}, \event \in \eventset$such that 
$\config{c, \trace_0} \rightarrow^* \config{\eskip, \trace_0 \tracecat \trace_1}$, 
and $\event \eventin \trace_1$, if 
\begin{itemize}
	\item $\event \in \eventset^{\asn}$, then either
	\begin{itemize}

	 \item there exists $\trace_1' \in \mathcal{T}, c' \in \cdom, \expr$ such that
\[
\begin{array}{l}
	% \forall \trace_0 \in \mathcal{T}, c \in \cdom,
	% \event \in \eventset^{\asn} \st
	% \config{c, \trace_0} \rightarrow^* \config{\eskip, \trace_0 \tracecat \trace_1} \implies
	% (\event \eventin \trace_1 \land	x = \pi_1(\event) \land l = \pi_2(\event))
	% \\ 
	% \implies 
	% \big( 
		% \exists \trace_1' \in \mathcal{T}, \expr, c' \st
		\config{c, \trace_0} \rightarrow^* \config{ [\assign{x}{\expr}]^l;c', \trace_0  \tracecat  \trace'} \rightarrow^{assn}
		\config{c', \trace_0 \tracecat \trace_1'\cdot \event} \rightarrow^{*}
		\config{\eskip, \trace_0  \tracecat  \trace_1}
	% \big)
	% \\  \lor
	% \big( 
	% 	\exists \trace_1' \in \mathcal{T}, \qexpr, c' \st
	% 	\config{c, \trace_0} \rightarrow^* \config{ [\assign{x}{\query(\qexpr)}]^l;c', \trace_0 \tracecat \trace_1'} \rightarrow^{query}
	% 	\config{c', \trace_0  \tracecat  \trace_1' \cdot \event } \rightarrow^{*}
	% 	\config{ \eskip, \trace_0\trace_1}
	% \big)
\end{array}
\]
\item or there exists $\trace_1' \in \mathcal{T}, c' \in \cdom, \qexpr$ such that 
\[
\begin{array}{l}
	% \forall \trace_0 \in \mathcal{T}, c \in \cdom,
	% \event \in \eventset^{\asn} \st
	% \config{c, \trace_0} \rightarrow^* \config{\eskip, \trace_0 \tracecat \trace_1} \implies
	% (\event \eventin \trace_1 \land	x = \pi_1(\event) \land l = \pi_2(\event))
	% \\ 
	% \implies 
	% \big( 
	% 	\exists \trace_1' \in \mathcal{T}, \expr, c' \st
	% 	\config{c, \trace_0} \rightarrow^* \config{ [\assign{x}{\expr}]^l;c', \trace_0  \tracecat  \trace'} \rightarrow^{assn}
	% 	\config{c', \trace_0 \tracecat \trace_1'\cdot \event} \rightarrow^{*}
	% 	\config{\eskip, \trace_0  \tracecat  \trace_1}
	% \big)
	% \\  \lor
	% \big( 
		% \exists \trace_1' \in \mathcal{T}, \qexpr, c' \st
		\config{c, \trace_0} \rightarrow^* \config{ [\assign{x}{\query(\qexpr)}]^l;c', \trace_0 \tracecat \trace_1'} \rightarrow^{query}
		\config{c', \trace_0  \tracecat  \trace_1' \cdot \event } \rightarrow^{*}
		\config{ \eskip, \trace_0\trace_1}
	% \big)
\end{array}
\]
\end{itemize}

\item $\event\in \eventset^{\test}$ then either 
\begin{itemize}
\item there exists $\trace_1' \in \mathcal{T}, c', c_t, c_f, c'' \in \cdom, \bexpr$ such that
\[
\begin{array}{l}
	% \big( 
	% 	\exists \trace_1' \in \mathcal{T}, c', c_t, c_f, c'' \in \cdom, \bexpr \st
		\config{c, \trace_0} \rightarrow^* \config{\eif ([b]^l, c_t, c_f);c', \trace_0 \tracecat \trace_1'} \rightarrow^{if-b}
		\config{c'', \trace_0 \tracecat \trace_1'\cdot \event } \rightarrow^{*}
		\config{\eskip, \trace_0 \tracecat \trace_1} 
	% \big)
	% \\  \lor
	% \big( 
	% 	\exists \trace_1' \in \mathcal{T}, \bexpr, c', c_w, c'' \in \cdom \st
	% 	\config{c, \trace_0} \rightarrow^* \config{ \ewhile([b]^l, c_w);c', \trace_0 \tracecat  \trace_1'} \rightarrow^{while-b}
	% 	\config{c'', \trace_0 \tracecat \trace_1'\cdot \event } \rightarrow^{*}
	% 	\config{\eskip, \trace_0  \tracecat \trace_1}
	% \big)
\end{array}
\]
\item or there exists $ \trace_1' \in \mathcal{T}, c', c_w, c'' \in \cdom, \bexpr$ such that 
\[
% \begin{array}{l}
% 	\big( 
% 		\exists \trace_1' \in \mathcal{T}, \bexpr, c', c_t, c_f, c'' \in \cdom \st
% 		\config{c, \trace_0} \rightarrow^* \config{\eif ([b]^l, c_t, c_f);c', \trace_0 \tracecat \trace_1'} \rightarrow^{if-b}
% 		\config{c'', \trace_0 \tracecat \trace_1'\cdot \event } \rightarrow^{*}
% 		\config{\eskip, \trace_0 \tracecat \trace_1} 
% 	\big)
% 	\\  \lor
% 	\big( 
% 		\exists \trace_1' \in \mathcal{T}, \bexpr, c', c_w, c'' \in \cdom \st
		\config{c, \trace_0} \rightarrow^* \config{ \ewhile([b]^l, c_w);c', \trace_0 \tracecat  \trace_1'} \rightarrow^{while-b}
		\config{c'', \trace_0 \tracecat \trace_1'\cdot \event } \rightarrow^{*}
		\config{\eskip, \trace_0  \tracecat \trace_1}
% 	\big)
% \end{array}
\]
\end{itemize}
\end{itemize}
%
\end{lem}
%
% \begin{lem}[Testing Event Inversion]
% \label{lem:inv_test}
% \[
% \begin{array}{l}
% 	\forall c\in \cdom, \trace_0 \in \mathcal{T}, \event = (b, l, n, v) \in \eventset^{\test} \st
% 	 \config{c, \trace_0} \rightarrow^* \config{\eskip, \trace_0 \tracecat \trace_1}
% 	\implies  \event \eventin \trace_1 \\
% 	\implies 
% 	\big( 
% 		\exists \trace_1' \in \mathcal{T}, \bexpr, c', c_t, c_f, c'' \in \cdom \st
% 		\config{c, \trace_0} \rightarrow^* \config{\eif ([b]^l, c_t, c_f);c', \trace_0 \tracecat \trace_1'} \rightarrow^{if-b}
% 		\config{c'', \trace_0 \tracecat \trace_1'\cdot \event } \rightarrow^{*}
% 		\config{\eskip, \trace_0 \tracecat \trace_1} 
% 	\big)
% 	\\ \qquad \lor
% 	\big( 
% 		\exists \trace_1' \in \mathcal{T}, \bexpr, c', c_w, c'' \in \cdom \st
% 		\config{c, \trace_0} \rightarrow^* \config{ \ewhile([b]^l, c_w);c', \trace_0 \tracecat  \trace_1'} \rightarrow^{while-b}
% 		\config{c'', \trace_0 \tracecat \trace_1'\cdot \event } \rightarrow^{*}
% 		\config{\eskip, \trace_0  \tracecat \trace_1}
% 	\big)
% \end{array}
% \]
% \end{lem}
%
% \begin{lem}[Control Dependency -> Exists Testing Event]
% \label{lem:inv_ctltotest}
% \[
% 	\forall \event_1, \event_2 \in \eventset, c \st 
% 	\eventdep^{\ctl}(\event_1, \event_2, c, D)
% 	\implies
% 	\exists \event_b \in \eventset^{\test}, \trace_2 \in \mathcal{T} \st \eventdep^{val}(\event_1, \event_b, \trace_2, c, D)
% \]
% \end{lem}

% \begin{lem}[Control Dependency -> Event 2 in the Body Command of the Testing Event]
% \label{lem:inv_ctltoevent2}
% \[
% \begin{array}{l}
% 	\forall \event_1, \event_2 = (x_2, l_2, n_2, v_2) \in \eventset, c \st 
% 	\eventdep^{\ctl}(\event_1, \event_2, c, D)\\
% 	\implies
% 	\exists \event_b = (b, l, n, v) \in \eventset^{\test}, \expr_2 \st \eventdep^{val}(\event_1, \event_b, c, D)\\
% 	\quad \land \Big(
% 	\exists c_t, c_f \st (\eif ([b]{}^l, c_t, c_f)) \in_{c} c \land ([\assign{x_2}{\expr_2}]^{l_2}) \in_c c_t;c_f \\
% 	\qquad \lor\exists c_w \st (\ewhile [b]{}^l \edo c_w) \in_{c} c \land ([\assign{x_2}{\expr_2}]^{l_2}) \in_c c_w
% 	\Big)
% \end{array}
% \]
% \end{lem}
%
\begin{lem}[Liveness Inversion]
\label{lem:inv_live}
For all $c \in \cdom \trace, \trace' \in \mathcal{T} $, if 
$\config{c, \trace} \xrightarrow{}^* \config{c', \trace'}$,
and for all $x^l \in \lvar_c$ such that 
$\llabel(\trace') x = l $, then $(x^l \in \live^{\entry_{c'}}(c))$.
%
\[
	\forall c \in \cdom , \trace, \trace' \in \mathcal{T} \st
	\config{c, \trace} \xrightarrow{}^* \config{c', \trace'}
	\implies
	\forall x^l \in \lvar_c \st \llabel(\trace') x = l \implies (x^l \in \live^{\entry_{c'}}(c))
\]
\end{lem}
%
\begin{lem}[Only $\eskip$ Command doesn't Produce Event].
\label{lem:inv_skip}
For all trace $\trace\in \mathcal{T}$, and $c, c' \in \cdom$,  
$\config{c, \trace} \rightarrow \config{c', \trace}$ if and only if $c = [\eskip];c'$. 
\[
	\forall \trace\in \mathcal{T}, c, c' \in \cdom \st
	\config{c, \trace} \rightarrow \config{c', \trace}
	\Leftrightarrow 
	c = [\eskip];c'
% \footnote{$([\eskip];){}^*$ denotes a sequence command only composed of $[\eskip]$ commands.}
\]
\end{lem}
%
\begin{lem}[Control Dependency Inversion]
\label{lem:ctldep_inv}
For all $c \in \cdom$, $D \in \dbdom, \event_b \in \eventset^{\test}, \event \in \eventset^{\asn} $, if 
$\eventdep^{\ctl}(\event_b, \event, c, D)$, 
then for all  $z \in VAR(\pi_1(\event_b))$ there exists a label $i \in \mathbb{N}$ such that 
$\flowsto(z^i, \pi_1(\event)^{\pi_2(\event)}, c)$
\[		
	\begin{array}{l}
		\forall c \in \cdom, D \in \dbdom, \event_b \in \eventset^{\test}, \event \in \eventset^{\asn} \st 
		\eventdep^{\ctl}(\event_b, \event, c, D) 
		\\ \quad 
		\implies
		\forall z \in VAR(\pi_1(\event_b)) \st \exists i \in \mathbb{N} \st
		\flowsto(z^i, \pi_1(\event)^{\pi_2(\event)}, c)
\end{array}
\]
\end{lem}
%
\begin{lem}[General Dependency Inversion]
	\label{lem:dep_inv}
For all $ c \in \cdom, D \in \dbdom, x^i \in \lvar_c$, and $\event_y \in \eventset^{\asn}$, 
if $x^i \in VAR(\expr_y)$, 
or there exists $\event_b \in \eventset^{\test}$ such that 
$x^i \in VAR(\pi_1(\event_b)$ and 
$\eventdep^{\ctl}(\event_b, \event_y, c, D)$, then $\flowsto(x^i, \pi_1(\event_y)^{\pi_2(\event_y)}, c)$.
%
	\[
	\begin{array}{l}
		\forall c \in \cdom, D \in \dbdom, x^i \in \lvar_c, \event_y \in \eventset^{\asn}
		\st
		\\ \quad
		(x^i \in VAR(\expr_y)\lor 
		(\exists \event_b \in \eventset^{\test} \st x^i \in VAR(\pi_1(\event_b)) 
		\land \eventdep^{\ctl}(\event_b, \event_y, c, D)))
		\implies \flowsto(x^i, \pi_1(\event_y)^{\pi_2(\event_y)}, c)
	\end{array}
\]
\end{lem}
%
\begin{lem}[Flow Search Algorithm ($\mathcal{A}$) Inversion 1]
\label{lem:inv_alg1}
For all $D \in \dbdom , c \in \cdom, \trace \in \mathcal{T}, \event_1, \event_2 \in \eventset^{\asn}$, and a list $l$,
if $l \in \mathcal{A}(\event_1, \event_2, \trace, c, D)$,
then l must have the form $[\pi_1(\event_1)^{\pi_2(\event_1)},\ldots, \pi_1(\event_2)^{\pi_2(\event_2)}]$.
\[
\begin{array}{l}
    \forall D \in \dbdom , c \in \cdom, \trace \in \mathcal{T}, l \st \forall \event_1, \event_2 \in \eventset^{\asn} \st
  \\ \quad 
 l\in \mathcal{A}(\event_1, \event_2, \trace, c, D)  \implies  l = [\pi_1(\event_1)^{\pi_2(\event_1)},\ldots, \pi_1(\event_2)^{\pi_2(\event_2)}]
\end{array}
\]
\end{lem}
%
\begin{proof}
% Let $l \in \mathcal{A}(\event_1, \event_2, \trace, c, D)$. By definition of $\mathcal{A}$ we have 
% \[l\in \kw{setmap} 
% 	% \bigcup\limits_{l \in \kw{dfs}(\trace, c, D) \land l = \event_1 :: l'}
% 	\left(\emap 
% 		(\efun  \event \to \pi_1(\event)^{\pi_2(\event)})	
% 	(\efilter 
% 		(\efun \event \to  \event \in \eventset^{\asn})) \right)
% 	S
% \]
% for $S=\kw{setfilter}
% 		(\efun l \to \exists l' \st l = \event_1 :: l' ++ [\event_2]) ~ \kw{dfs}(\trace, c, D)$.
% So, in particular by definition of setmap there is a list $l_1\in S$ such that 
% \[
% 	% \bigcup\limits_{l \in \kw{dfs}(\trace, c, D) \land l = \event_1 :: l'}
% \emap 
% 		(\efun  \event \to \pi_1(\event)^{\pi_2(\event)})	
% 	(\efilter 
% 		(\efun \event \to  \event \in \eventset^{\asn}))
% 	l_1 = l
% \]
% \[
% l = [\event_1, \cdots, \event_2]
% \]
% \\
Let  $l \in \mathcal{A}(\event_1, \event_2, \trace, c, D)$,
by definition of $\mathcal{A}$, we know 
%
$$l\in \kw{setmap} 
	% \bigcup\limits_{l \in \kw{dfs}(\trace, c, D) \land l = \event_1 :: l'}
		\left(\efun l \to ( \emap 
		(\efun  \event \to \pi_1(\event)^{\pi_2(\event)})
	(\efilter 
		(\efun \event \to  \event \in \eventset^{\asn}) ~ l) \right)
	~ S,
$$
%
where $S=(\kw{setfilter} ~(\efun l \to l = [\event_1, \cdots, \event_2]) ~ (\kw{dfs} \eapp \trace \eapp c \eapp  D))$.
\\
Then, by definition of $\kw{setmap}$, we know $l$ is an output of
\\
$\left(\efun l \to ( \emap 
		(\efun  \event \to \pi_1(\event)^{\pi_2(\event)})
	(\efilter 
		(\efun \event \to  \event \in \eventset^{\asn}) ~ l) \right)$.
\\
Then we know there exists a preimage
$l_e \in S $
for $l$ such that 
$$
\emap (\efun  \event \to \pi_1(\event)^{\pi_2(\event)}) 
(\efilter (\efun \event \to  \event \in \eventset^{\asn}) ~ l_e) 
= l.
$$
 %
Since $l_e \in (\kw{setfilter} ~(\efun l \to l = [\event_1, \cdots, \event_2]) ~ (\kw{dfs} \eapp \trace \eapp c \eapp  D))$,
\\
by the $\kw{setfilter}$ function,
we know only the lists of events in $(\kw{dfs} \eapp \trace \eapp c \eapp  D)$ having the form
$ [\event_1, \cdots, \event_2] $ are preserved in $S$, i.e.,
\[
	\forall l \in (\kw{setfilter} ~(\efun l \to l = [\event_1, \cdots, \event_2]) ~ \kw{dfs}(\trace, c, D))
	\st l = [\event_1, \cdots, \event_2]
\]
%
Then we know $l_e$ also has the same form, 
i.e., $l_e = [\event_1, \cdots, \event_2]$.
%
\\
Let $l_{ef} = (\efilter (\efun \event \to  \event \in \eventset^{\asn})) ~ l_e$, 
by $\event_1, \event_2 \in \eventset^{\asn}$, 
we know $\event_1$ and $\event_2$ are preserved in $l_{ef}$, i.e.,:
\[
	l_{ef} =[\event_1, \cdots, \event_2]
\]
%
Then, by applying the function
$\emap (\efun  \event \to \pi_1(\event)^{\pi_2(\event)})$ to 
$l_{ef}$, we have $l$ as follows:
\[
	[\pi_1(\event_1)^{\pi_2(\event_1)}, \cdots, \pi_1(\event_2)^{\pi_2(\event_2)}]
\]
%
%
This lemma is proved.
\end{proof}
%
\begin{lem}[Flow Search Algorithm ($\mathcal{A}$) Inversion 2]
\label{lem:inv_alg2}
For every $\event_1, \event_2 \in \eventset^{\asn}, D \in \dbdom , c \in \cdom$, we have either one of the two following cases:
\begin{enumerate}
  \item $\mathcal{A}(\event_1, \event_2,  [\event_1; \event_2], c, D) = 
  \left\{[\pi_1(\event_1)^{\pi_2(\event_1)}, \pi_1(\event_2)^{\pi_2(\event_2)}] \right \}$ 
  and $\eventdep^{val}(\event_1, \event_2, [\event_1; \event_2], c, D)$.
  \item  $\mathcal{A}(\event_1, \event_2, [\event_1; \event_2], c, D) = \{\}$ 
  and $\neg \eventdep^{val}(\event_1, \event_2, [\event_1; \event_2] c, D)$;
\end{enumerate}
\end{lem}
% \wq{ Good! I just realize this lemma is only used for case 4 of 5.3.}
%\jl{Yes!}
\begin{proof}
By definition of $A$, we know:
%
\[
	\begin{array}{l}
	\mathcal{A}(\event_1, \event_2, [\event_1; \event_2], c, D)
	= 
	\kw{setmap} ~
	% \bigcup\limits_{l \in \kw{dfs}(\trace, c, D) \land l = \event_1 :: l'}
	\\ \qquad \qquad
	\left(\efun l \to ( \emap 
		(\efun  \event \to \pi_1(\event)^{\pi_2(\event)})
	(\efilter 
		(\efun \event \to  \event \in \eventset^{\asn}) ~ l) \right)
	\\ \qquad \qquad
	(\kw{setfilter} ~
		(\efun l \to l = [\event_1, \cdots, \event_2]) ~ 
		% \left(\left\{[\event_2]\right\} \cup \left\{ \event_1 \stackrel{[\event_1; \event_2]}{\uplus} [\event_2] \right\} \right))
		\left(\left\{[\event_2]\right\} \cup \left(  {\uplus} \eapp \event_1 \eapp {[\event_1; \event_2]} \eapp [\event_2] \right) \right))
	\end{array}
\]
by definition of $ {\uplus} \eapp \event_1 \eapp {[\event_1; \event_2]} \eapp [\event_2]  $, we know 
\[
	\begin{array}{l}
	% \event_1 \stackrel{[\event_1; \event_2]}{\uplus} [\event_2]
	{\uplus} \eapp \event_1 \eapp {[\event_1; \event_2]} \eapp [\event_2] 
	=   
	\\ \quad \qquad 	
	\ecase \eventdep^{val}(\event_1, \event_2, [\event_1; \event_2], c, D)
	\to \left\{ [\event_1, \event_2] \right\}
	\\ \quad \qquad 	
	\ecase \_
	\to \left\{ \right\}
\end{array}
\]
%
By simplification of the $\kw{setfilter}$, $\emap$, $\efilter$ and $\kw{setmap}$ functions, we know
\\
in the case of $\eventdep^{val}(\event_1, \event_2, [\event_1; \event_2], c, D)$:
% \\
$\mathcal{A}(\event_1, \event_2, [\event_1; \event_2], c, D) = 
  \left\{[\pi_1(\event_1)^{\pi_2(\event_1)}, \pi_1(\event_2)^{\pi_2(\event_2)}] \right \}$
\\
(1) is proved.
\\
And in the case of $\neg \eventdep^{val}(\event_1, \event_2, [\event_1; \event_2], c, D)$: 
% \\
$\mathcal{A}(\event_1, \event_2, \cdot  \event_1 \tracecat [\event_2], c, D) = 
  \left\{ \right \}$
\\
(2) is proved.
\end{proof}
%
%
% \begin{lem}[\todo{Assignment Evaluation Inversion}].
% 	\label{lem:inv_eval_asn}
% 	\[
% 	\begin{array}{l}
% 		\forall x \in \lvar_c, \kw{V_{ptl}} \in \subseteq \lvar_c \expr \st 
% 		\exists \trace, \trace' \in \mathcal{T} \st 
% 		\\ \quad
% 		\forall z \in \lvar_c \setminus (\{x\} \cup \kw{V_{ptl}}) \st 
% 		\env(\trace) z = \env(\trace') z 
% 		\\ \quad \land
% 		\forall \event \in \trace, \event' \in \trace' \st 
% 		\pi_1(\event) \in \kw{V_{ptl}} \land \diff(\event, \event') 
% 		\\ \quad
% 		\implies 
% 		\neg \eventdep^{val}(\event, \event_y, \trace[\event:\event_y] ) 
% 		\implies
% 		\config{\trace, [\assign{y}{\expr}]{}^l;c'} \rightarrow^{asn} \config{\trace\cdot \event_y, c'}
% 		\\ \quad
% 		\implies 
% 		\config{\trace', [\assign{y}{\expr}]{}^l;c'} \rightarrow^{asn} \config{\trace'\cdot \event_y',c'}
% 		\land \diff(\event_y, \event_y')
% 		\implies x \in VAR(\expr)
% 	\end{array}
% 	\]
% 	\end{lem}	
%
\begin{lem}[Generalized Dependency Implies $\flowsto$ ]
	\label{lem:indepevents_inv_gnl}
For every $D \in \dbdom , c \in \cdom, \trace \in \mathcal{T}$, and two assignment events 
$\event_1, \event_2 \in \eventset^{\asn}$,
if the trace $trace$ has the form $\trace = [\event_1] \tracecat \trace' \tracecat [\event_2]$ with $\trace' \in \mathcal{T}$, and $\eventdep(\event_1, \event_2, \trace, c, D)$
then $\flowsto(\pi_1(\event_1)^{\pi_2(\event_1)}, \pi_1(\event_2)^{\pi_2(\event_2)}, c) $,
or otherwise there exists
$\event \in \trace'$ such that
$\left( 		
   \eventdep(\event_1, \event, \trace[\event_1:\event], c, D)
\land 
\flowsto(\pi_1(\event)^{\pi_2(\event)}, \pi_1(\event_2)^{\pi_2(\event_2)}, c) 
\right)$.
%
	\[
	\begin{array}{l}
		\forall D \in \dbdom , c \in \cdom, \trace \in \mathcal{T} \st \forall \event_1, \event_2 \in \eventset^{\asn} \st
		 \exists \trace' \in \mathcal{T} \st \trace = [\event_1] \tracecat \trace' \tracecat [\event_2]
		\implies
		\eventdep(\event_1, \event_2, \trace, c, D) 
		\\ \quad 
		\implies 
		\flowsto(\pi_1(\event_1)^{\pi_2(\event_1)}, \pi_1(\event_2)^{\pi_2(\event_2)}, c) 
		\\ \qquad \quad \lor
		\exists \event \in \trace' \st 
		\left( 		
			\eventdep(\event_1, \event, \trace[\event_1:\event], c, D)
		\land 
		\flowsto(\pi_1(\event)^{\pi_2(\event)}, \pi_1(\event_2)^{\pi_2(\event_2)}, c) 
	\right) 
		% \\ \qquad \qquad \lor
		% \flowsto(\pi_1(\event_1)^{\pi_2(\event_1)}, \pi_1(\event_2)^{\pi_2(\event_2)}, c) 
	\end{array}
	\]
\end{lem}
%
\begin{lem}[Independent Events Doesn't Block $\flowsto$ ].
		\label{lem:inv_indepevents}
		\[
		\begin{array}{l}
			\forall D \in \dbdom , c \in \cdom, \trace \in \mathcal{T} \st \forall \event_1, \event_2 \in \eventset^{\asn} \st
			 \exists \trace' \in \mathcal{T} \st \trace = [\event_1] \tracecat \trace' \tracecat [\event_2]
			\implies
			\eventdep^{val}(\event_1, \event_2, \trace, c, D) 
			\\ \quad 
			\implies 
			\left( \forall \event \in \trace' \st \neg \eventdep^{val}(\event_1, \event, \trace[\event_1:\event], c, D)
			\lor \neg \eventdep^{val}(\event, \event_2, \trace[\event:\event_2], c, D) 
			\right) 
			\\ \quad 
			\implies 
			\flowsto(\pi_1(\event_1)^{\pi_2(\event_1)}, \pi_1(\event_2)^{\pi_2(\event_2)}, c)
		\end{array}
		\]
\end{lem}
%
\begin{proof}
Taking arbitrary $D \in \dbdom , c \in \cdom$, and two assignment events $\event_1, \event_2\in \eventset^{\asn}$.
\\
Picking a trace satisfying $\trace = [\event_1] \tracecat \trace_2 \tracecat [\event_2]$ for arbitrary $\trace_2 \in \mathcal{T}$,
 then we know 
\\
$\exists \trace' \in \mathcal{T} \st \trace = [\event_1] \tracecat \trace' \tracecat [\event_2]$ by picking $\trace' = \trace_2$.
\\
Unfolding $\eventdep^{val}(\event_1, \event_2, \trace, c, D)$, we know there exist $ \event_1', \event_2' \in \eventset^{\asn},
\trace_2' \in \mathcal{T}, c_1, c_2 \in \cdom$ such that 
$\diff(\event_2, \event_2')$,
 and following two executions:
% \[
%   \exists \event_1', \event_2' \in \eventset^{\asn},
%   \trace_2' \in \mathcal{T}, c_1, c_2 \in \cdom \st
% \]
%
\[
\begin{array}{l}
\config{c, \trace_0} \rightarrow^{*}
\config{c_1, \trace_1 \tracecat [\event_1]} \rightarrow^{*} \config{c_2, \trace_1 \tracecat [\event_1] \tracecat \trace_2 \tracecat [\event_2]} 
\\ \quad
\land
\config{c_1, \trace_1 \tracecat [\event_1']} \rightarrow^{*} \config{c_2, \trace_1 \tracecat [\event_1'] \tracecat \trace_2' \tracecat [\event_2']} 
\end{array}
\]
%
%
By inversion Lemma.~\ref{lem:inv_event} on $\event_2$ and $\event_2'$ in the two executions, 
since $\diff(\event_2, \event_2)$ we know there exists an expression $\expr_2$ or $\qexpr_2$ in the following two execution instances:
\[
\config{c_1, \trace_1 \tracecat [\event_1]} \rightarrow^{*} \config{[\assign{\pi_1(\event_2)}{\expr_2 / \query(\qexpr_2)}]{}^{\pi_2(\event_2)};c_2, \trace_1 \tracecat [\event_1] \tracecat \trace_2} 
\rightarrow^{asn / query} \config{c_2, \trace_1 \tracecat [\event_1] \tracecat \trace_2 \tracecat [\event_2]}  
\]
%
\[
\config{c_1, \trace_1 \tracecat [\event_1']} \rightarrow^{*} \config{[\assign{\pi_1(\event_2)}{\expr_2 / \query(\qexpr_2)}]{}^{\pi_2(\event_2)};c_2, \trace_1 \tracecat [\event_1'] \tracecat \trace_2'} 
\rightarrow^{asn / query} \config{c_2, \trace_1 \tracecat [\event_1'] \tracecat \trace_2' \tracecat [\event_2']}  
\]
%
Let $\diff_{\eventset}$ be a subset of the events in $\trace_2$, satisfying: 
% \todo{refine the notation}
\[
	\begin{array}{l}
		\forall \event_z \in \eventset^{\asn} \st 
	\event_z \in \diff_{\eventset} \Leftrightarrow 
	\exists \trace_2^h, \trace'^h_2, \trace_2^t, \trace'^t_2, \event_z' \in \trace_2' \st 
	\trace_2 = \trace_2^h \tracecat [\event_z] \tracecat \trace_2^t
	\\ \quad
	\land 
	\trace_2' = \trace'^h_2 \tracecat [\event_z'] \tracecat \trace'^t_2
	\land 
	\diff(\event_z, \event_z')
	\land 
	\vcounter(\trace_2^h) \pi_1(\event_z) = \vcounter(\trace'^h_2)(\event_z)
\end{array}		
\]
%
Then we know for all $\event_z \in  \diff_{\eventset}$,
 $\eventdep^{val}(\event_1, \event_z, \trace[\event_1:\event_z], c, D)$;
\\
and $\forall z^j \in (\lvar \setminus (\mathbb{LV}_{\diff_{\eventset}} \cup\{\pi_1(\event_1)^{\pi_2(\event_2)}\}) ) \st 
\env(\trace_1 \tracecat [\event_1] \tracecat \trace_2) z = \env(\trace_1 \tracecat [\event_1'] \tracecat \trace_2') z $,
\\
where $\mathbb{LV}_{\diff_{\eventset}}$ is the set of labelled variables of every event in $\diff_{\eventset}$.
\\
From the hypothesis, we know for all $\event_z \in  \diff_{\eventset}$, $\neg \eventdep^{val}(\event_z, \event_2, \trace[\event_z:\event_2], c, D) $.
\\
Then by $\eventdep^{val}$ definition, we know 
$\forall \event_z', \event_2'' \in \eventset^{\asn}, \trace_z',\trace_z,  \trace_1' \in \mathcal{T}, c_z \in \cdom$ satisfying following two executions, we have $\event_2 \eventeq \event_2''$.
\[
	\begin{array}{l}
		\config{c, \trace_0} \rightarrow^{*}
		\config{c_z, \trace_1' \tracecat [\event_z]} \rightarrow^{*} \config{c_2, \trace_1' \tracecat [\event_z] \tracecat \trace_z \tracecat [\event_2]} 
		\\ \quad
		\land
		\config{c_z, \trace_1' \tracecat [\event_z']} \rightarrow^{*} \config{c_2, \trace_1' \tracecat [\event_z'] \tracecat \trace_z' \tracecat [\event_2']'} 
		\end{array}		
\]
%
Then we know 
% \todo{type correctness}
\[
	\forall z^j \in (\lvar \setminus \mathbb{LV}_{\diff_{\eventset}} ), \trace, \trace' \in \mathcal{T}, v, v' \st 
	\env(\trace) z = \env(\trace') z 
	\land 
	\config{\trace, \expr_2} \aarrow v 
	\land 
	\config{\trace', \expr_2} \aarrow v'
	\land 
	v = v' 
\]
%
Since $\diff(\event_2, \event_2')$, we also know:
%
\[	
\config{\trace_1 \tracecat [\event_1] \tracecat \trace_2, \expr_2} \aarrow \pi_3(\event_2)
\land 
\config{\trace_1 \tracecat [\event_1'] \tracecat \trace_2', \expr_2} \aarrow \pi_3(\event_2') 
\land 
\pi_3(\event_2) \neq \pi_3(\event_2')
\]
%
% By construction of $\diff_{\eventset}$, we also have:
% \\
% $\forall z^j \in (\lvar \setminus \diff_{\eventset} \cup\{\event_1\} ) \st 
% \env(\trace_1 \tracecat [\event_1] \tracecat \trace_2) z = \env(\trace_1 \tracecat [\event_1'] \tracecat \trace_2') z $.
% \\
% By $\forall \event \in (\trace[\event_1:\event_y]\setminus \{\event_1, \event_y\}) \st
%   \neg \eventdep^{val} (\event_1, \event, c, D) \lor \neg \eventdep^{val} (\event, \event_y, c, D)$,
Then, we know $\event_1$ is the only cause of the difference in $\event_y$ and $\event_y'$ when evaluating 
$[\assign{\pi_1(\event_2)}{\expr_2 / \query(\qexpr_2)}]{}^{\pi_2(\event_2)}$.
%
\\
By inversion Lemma.~\ref{lem:inv_expr_gnl}, we know
\[
  \pi_1(\event_1) \in VAR(\expr) \land {\pi_2(\event_1)} = \llabel(\trace_1 \tracecat [\event_1] \tracecat \trace_2) \pi_1(\event_1) 
\]
%
By $\flowsto$ definition:
% \todo{add the Liveness inversion}
\[
  \flowsto(\pi_1(\event_1)^{\pi_2(\event_1)}, \pi_1(\event_y)^{\pi_2(\event_y)}, c)
\]
\end{proof}
%
\begin{lem}[Dependency with Empty Trace Implies $\flowsto$ ].
	\label{lem:emptytrace_dep}
	\[
	\begin{array}{l}
		\forall \event_1, \event_2 \in \eventset^{\asn}, c \in \cdom, D \in \dbdom 
		\st 
		\eventdep^{val}(\event_1, \event_2, [\event_1; \event_2],  c, D) 
		\\ \quad 
		\implies 
		\flowsto(\pi_1(\event_1)^{\pi_2(\event_1)}, \pi_1(\event_2)^{\pi_2(\event_2)}, c)
	\end{array}
	\]
\end{lem}
\begin{proof}
	This lemma is a special case of Lemma~\ref{lem:inv_indepevents} and proved in Subproof~\ref{pf:alg_correct_base}.
\end{proof}
%
\begin{lem}[Independent Events Doesn't Block $\flowsto$ for Testing Event].
	\label{lem:inv_indepeventstest}
	\[
	\begin{array}{l}
		\forall D \in \dbdom , c \in \cdom, \trace \in \mathcal{T} \st \forall \event_1,\in \eventset^{\asn}, \event_2 \in \eventset^{\test} \st
		 \exists \trace' \in \mathcal{T} \st \trace = [\event_1] \tracecat \trace' \tracecat [\event_2]
		\implies
		\eventdep^{val}(\event_1, \event_2, \trace, c, D) 
		\\ \quad 
		\implies 
		\left( \forall \event \in \trace' \st \neg \eventdep^{val}(\event_1, \event, \trace[\event_1:\event], c, D)
		\lor \neg \eventdep^{val}(\event, \event_2, \trace[\event:\event_2], c, D) 
		\right) 
		\\ \quad 
		\implies 
		\pi_1(\event_1) \in VAR(\pi_1(\event_2)) \land {\pi_2(\event_1)} = \llabel(\trace)
	\end{array}
	\]
\end{lem}
%
\todo{Event Dependency Transitivity}
\begin{thm}
	\label{lem:valdep_trans}
(Value Dependency Transitivity)
  % An event $\event_2 \in \eventset^{\asn}$ is in the \emph{may-dependency} relation with another
  % event $\event_1 \in \eventset^{\asn}$ in a program ${c}$ with a hidden database $D$, denoted as $\eventdep(\event_1, \event_2, c, D)$,
  % if and only if
  \[
	  \begin{array}{l}
  \forall \event_1, \event_2, \event_3 \in \eventset^{\asn}, \trace_{12}, \trace_{23} \in \mathcal{T} \st 
  \eventdep^{val}(\event_1, \event_2, \trace_{12}, c, D) 
  \land
  \eventdep^{val}(\event_2, \event_3, \trace_{23}, c, D) 
  \\ \quad
  \implies
  \eventdep^{val}(\event_1, \event_3, \trace_{12}\tracecat\trace_{23}, c, D)
	  \end{array}
  \]
\end{thm}
%
%
\begin{lem}(Control Dependency Transitivity)
\label{lem:ctl_trans}
\[
  \forall \event_1, \event_2 \in \eventset^{\test}, \event_3 \in \eventset \st
  \eventdep^{\ctl}(\event_1, \event_2, c, D) 
  \land \eventdep^{\ctl}(\event_2, \event_3, c, D)
  \implies \eventdep^{\ctl}(\event_1, \event_3, c, D)
\]
\end{lem}
%
\begin{lem}
	\label{lem:eventdep_trans}
(Event Dependency Transitivity)
	\[
	\forall \event_1, \event_2, \event_3 \in \eventset^{\asn}\st 
	\eventdep(\event_1, \event_2, c, D) 
	\land
	\eventdep(\event_2, \event_3, c, D) 
	\implies
	\eventdep(\event_1, \event_3, c, D)
	\]
  \end{lem}
%
\begin{lem}[While Loop Inversion].
\label{lem:inv_while}
\[
\begin{array}{l}
\forall \trace, \trace' \in \mathcal{T}, c, c_1, c_2 \in \cdom \st
	\\ \quad
	\config{c, \trace} \rightarrow^* \config{c_1; c_2, \trace'}
	\implies
	c_1 \in_c c_2
	\implies
	\exists l \in \mathbb{N}, b \in \mathcal{B}, c_w \in \cdom \st 
	(\ewhile [b]^l \edo c_w) \in_c c_2 \land c_1 \in_c c_w
\end{array}
\]
\end{lem}
%

\clearpage
\begin{thm}[Correctness]
\label{thm:alg_correct}
\[
\begin{array}{l}
  \forall \event_1, \event_2 \in \eventset^{\asn}, \trace \in \mathcal{T}, D \in \dbdom , c \in \cdom \st
  \\ \quad 
   \exists \trace' \in \mathcal{T} \st \trace = \cdot \event_1 \tracecat \trace' \cdot \event_2
   \implies    \forall  z^i, y^j \in \lvar_c, l_h, l_t \st 
  \\ \qquad 
   l_h ++ [z^i, y^j] ++ l_t \in \mathcal{A}(\event_1, \event_2, \trace, c, D)
   \implies \flowsto(z^i, y^j, c)
\end{array}
\]
\end{thm}

\begin{proof}[Proof of Theorem.~\ref{thm:alg_correct}].
\\
Taking arbitrary $\event_1, \event_2 \in \eventset^{\asn}, \trace \in \mathcal{T}, D \in \dbdom , c \in \cdom$,
by strong induction on $\trace$, we have following cases.
\begin{case}($\trace = \cdot$)
\\
Since $\not\exists \trace' \in \mathcal{T}$,satisfies $
\cdot  = \cdot {} \event_1 \tracecat \trace' \cdot \event_2$, the theorem is vacuously true.
\end{case}
%
\begin{case}($\event \in \eventset, \trace = \cdot \event$)
\\
Since $\not\exists \trace' \in \mathcal{T}$,satisfies $
\cdot  \event = \cdot \event_1 \tracecat \trace' \cdot \event_2$, the theorem is vacuously true.
\end{case}
%
\begin{case}($\event_1', \event_2' \in \eventset, \trace_{ih} \in \mathcal{T}, 
\trace = \cdot \event_1' \tracecat \trace_{ih} \cdot \event_2' \land 
(\event_1 \eventneq \event_1' \lor \event_2 \eventneq \event_2')$)
\\
By $(\event_1 \eventneq \event_1' \lor \event_2 \eventneq \event_2')$,
we know $\not\exists \trace' \in \mathcal{T}$ satisfies $
\cdot \event_1' \tracecat \trace_{ih} \cdot \event_2' = \cdot \event_1 \tracecat \trace' \cdot \event_2$.
\\
The theorem is vacuously true.
\end{case}
%
%
\begin{case}
\label{case:alg_correct_base}
($\trace = \cdot {} \event_1 \cdot {} \event_2$)
\\
To show:
% \wq{$\forall l \in \mathcal{A}(\event_1, \event_2, \trace, c, D)$ below?}
%\jl{yes, there is a typo}
\[
\begin{array}{l}
  \exists \trace' \in \mathcal{T} \st \trace = \cdot \event_1 \tracecat \trace' \cdot \event_2
  \implies    
  \forall  z^i, y^j \in \lvar_c, l_h, l_t \st 
    \\ \qquad 
    l_h ++ [z^i, y^j] ++ l_t \in \mathcal{A}(\event_1, \event_2, \trace, c, D)
   \implies \flowsto(z^i, y^j, c)
\end{array}
\]
%
Let $\trace' = \cdot$, we know $\exists \trace' \in \mathcal{T}$ satisfying 
$\cdot {} \event_1  {} \cdot {} \event_2 = \cdot {} \event_1 \tracecat \trace' \cdot \event_2$.
\\
By Inversion Lemma~\ref{lem:inv_alg2}, we have either one of the two following cases:
\begin{enumerate}
  \item $\mathcal{A}(\event_1, \event_2, \cdot \event_1 \cdot \event_2, c, D) = 
  \left\{[\pi_1(\event_1)^{\pi_2(\event_1)}, \pi_1(\event_2)^{\pi_2(\event_2)}] \right \}$ 
  and $\eventdep^{val}(\event_1, \event_2, \cdot  \event_1 \cdot \event_2, c)$.
  \item  $\mathcal{A}(\event_1, \event_2, \cdot \event_1 \cdot \event_2, c, D) = \{\}$ 
  and $\neg \eventdep^{val}(\event_1, \event_2, \cdot  \event_1 \cdot \event_2, c)$;
\end{enumerate}
%
In case of $\mathcal{A}(\event_1, \event_2, \cdot \event_1 \cdot \event_2, c, D) = \{\}$,
since $\not\exists  z^i, y^j \in \lvar_c, l_h, l_t \st l_h ++ [z^i, y^j] ++ l_t \in \{\}$, the theorem is vacuously true.
%
\\
Then it is sufficient to show: 
% \wq{Because $l \in A \implies DEP$ }
% \jl{yes}
\[
  \eventdep^{val}(\event_1, \event_2, \cdot \event_1 \cdot \event_2, c) \implies \flowsto(\pi_1(\event_1)^{\pi_2(\event_1)}, \pi_1(\event_2)^{\pi_2(\event_2)}, c)
\]
%
\begin{subproof}[Subproof of Case~\ref{case:alg_correct_base}].
\label{pf:alg_correct_base}
\\
Unfolding $\eventdep^{val}_{\trace}(\event_1, \event_2, \cdot \event_1 \cdot \event_2, c, D)$, we have:
\begin{equation}
\label{eq:eventdep_def_base}
\exists \vtrace_0,
\vtrace_1, \vtrace' \in \mathcal{T}, \event_2' \in \eventset, \event_1' \in \eventset^{\asn}, {c}_1, {c}_2,  {c}_2' \in \cdom \st
  \left(
  \begin{array}{ll}   
 & \config{{c}, \vtrace_0} \rightarrow^{*} 
\config{{c}_1, \vtrace_1 \cdot \event_1}  \rightarrow^{*} 
  \config{{c}_2,  \vtrace_1 \cdot \event_1 \cdot \event_2 } 
  % 
 \\ 
 \bigwedge &
  \config{{c}_1, \vtrace_1 \cdot \event_1'}  \rightarrow^{*} 
  \config{{c}_2,  \vtrace_1 \cdot \event_1' \tracecat \vtrace' \cdot \event_2' } 
\\
\bigwedge &  \pi_1(\event_1) = \pi_1(\event_1') \land \pi_2(\event_1) = \pi_2(\event_1') 
\\
\bigwedge & 
\diff(\event_2,\event_2' ) \land 
\vcounter(\vtrace) ~ \pi_2(\event_2)
= 
\vcounter(\vtrace') ~ \pi_2(\event_2)\\
\end{array}
\right)
\end{equation}
%
Let $\vtrace_0,
\vtrace_1, \vtrace' \in \mathcal{T}, \event_2' \in \eventset, \event_1' \in \eventset^{\asn}, {c}_1, {c}_2,  {c}_2'$ be the traces, events and commands satisfying the executions and restrictions in Eq.~\ref{eq:eventdep_def_base}.
\\
By Inversion Lemma~\ref{lem:inv_asn} on 
 $\event_1$, $\event_2$ and the first execution in Eq.~\ref{eq:eventdep_def_base},
 %
 we know $\exists \expr_1$ or $\qexpr_1$, $\exists \expr_2$ or $\qexpr_2$ and following instance of execution:
%
\footnote{
$\assign{x}{\expr / \query(\qexpr)}$ denotes variable $x$ is assigned by either an expression $\expr$ or query $\query(\qexpr)$
}
 %
\begin{equation}
\label{eq:valdep_inv1}
  \begin{array}{l}   
\config{{c}, \vtrace_0} \rightarrow^{*} 
\config{[\assign{{x}_1}{\expr_1 / \query(\qexpr_1)}]^{l_1} ; {c}_1, \vtrace_0 \cdot \vtrace_1}  
\rightarrow^{assn/query}
 \config{c_1, \vtrace_0 \cdot \vtrace_1 \cdot \event_1} \\
  \qquad \rightarrow^{*} 
  \config{[\assign{{x}_2}{\expr_2 / \query(\qexpr_2)}]^{l_2};{c}_2, 
  \vtrace_0 \cdot \vtrace_1 \cdot \event_1} 
  \rightarrow^{assn/query} 
  \config{{c}_2,  \vtrace_0 \cdot \vtrace_1 \cdot \event_1 \cdot \event_2} 
  % 
\end{array}
\end{equation}
%
%
By Inversion Lemma~\ref{lem:inv_asn} on 
$\event_2'$ and the second execution in Eq.~\ref{eq:eventdep_def_base},
% \[
% \config{{c}_1, \vtrace_1 \cdot \event_1'}  \rightarrow^{*} 
%   \config{{c}_2',  \vtrace_1 \cdot \event_1' \cdot \vtrace_2' \cdot \event_2' } 
%   \], 
we have $\exists \expr_2'$ or $\qexpr_2'$ and following execution instance,
 \[
  \begin{array}{l}   
  \config{c_1, \vtrace_0 \cdot \vtrace_1 \cdot \event_1'} 
  \rightarrow^{*} 
  \config{[\assign{{x}_2'}{\expr_2' / \query(\qexpr_2')}]^{l_2'} ; {c}_2', \vtrace_0 \cdot \vtrace_1 \cdot \event_1 \cdot \vtrace'} 
  \rightarrow^{assn/query} 
  \config{{c}_2',  \vtrace_0 \cdot \vtrace_1 \cdot \event_1' \cdot \vtrace' \cdot \event_2'} 
  % 
\end{array}
 \]
%
Unfolding $\diff(\event_2,\event_2')$, we have:
\[
  x_2 = x_2' \land l_2 = l_2' 
\] 
%
Since each command in $c$ have distinct label, we have $\expr_2' = \expr_2$, $\qexpr_2 = \qexpr_2'$, and following execution instance:
\begin{equation}
\label{eq:valdep_inv2}
  \config{c_1, \vtrace_0 \cdot \vtrace_1 \cdot \event_1'} 
  \rightarrow^{*} 
  \config{[\assign{{x}_2}{\expr_2 / \query(\qexpr_2)}]^{l_2} ; {c}_2, \vtrace_0 \cdot \vtrace_1 \cdot \event_1' \cdot \vtrace_2'} 
  \rightarrow^{assn/query} 
  \config{{c}_2,  \vtrace_0 \cdot \vtrace_1 \cdot \event_1' \cdot \vtrace_2' \cdot \event_2'} 
\end{equation}
%
From Eq.~\ref{eq:eventdep_def_base}, we also have
\begin{equation}
\label{eq:valdep_invn}
  \vcounter(\vtrace') l_2 = \vcounter( \cdot ) l_2 = 0
\end{equation}
%
%
By Induction on the operational semantics rules on following execution from Eq.~\ref{eq:valdep_inv1}:
 %
 \[\config{c_1, \vtrace_0 \cdot \vtrace_1 \cdot \event_1}
  \rightarrow^{*} 
  \config{[\assign{{x}_2}{\expr_2 / \query(\qexpr_2)}]^{l_2};{c}_2, 
  \vtrace_0 \cdot \vtrace_1 \cdot \event_1} 
\]
 %
By Inversion Lemma~\ref{lem:inv_skip}, we know:
 \[
 c_1 =_c 
 [\eskip]{}^*;[\assign{{x}_2}{\expr_2 / \query(\qexpr_2)}]^{l_2};{c}_2
 \]
 %
By substituting $c_1$ in Eq.~\ref{eq:valdep_inv2}, the following subproof shows there is only 1 possible execution instance of Eq.~\ref{eq:valdep_inv2}.
\begin{subproof}[Subproof]
\label{pf:noiteration_inv2}
There are two possible cases, 
where $[\assign{{x}_2}{\expr_2 / \query(\qexpr_2)}]^{l_2} \in_c c_2$ 
or $[\assign{{x}_2}{\expr_2 / \query(\qexpr_2)}]^{l_2} \notin_c c_2$.
%
\begin{enumerate}
\item{$[\assign{{x}_2}{\expr_2 / \query(\qexpr_2)}]^{l_2}\notin_c c_2$}
\\
In this case, we have the following execution instance:
%
\footnote{$\rightarrow^{\eskip^*}$ denotes every evaluation step in this execution is an evaluation of the $[\eskip]{}^*$ command}
 %
  \[
  \config{c_1, \vtrace_0 \cdot \vtrace_1 \cdot \event_1'} 
  \rightarrow^{\eskip^*} 
  \config{[\assign{{x}_2}{\expr_2 / \query(\qexpr_2)}]^{l_2} ; {c}_2, \vtrace_0 \cdot \vtrace_1 \cdot \event_1'} 
  \rightarrow^{assn/query} 
  \config{{c}_2,  \vtrace_0 \cdot \vtrace_1 \cdot \event_1' \cdot \event_2'} 
 \]
%
\item{$[\assign{{x}_2}{\expr_2 / \query(\qexpr_2)}]^{l_2} \in_c c_2$}
\\
By Inversion Lemma~\ref{lem:inv_while}, 
we pick $(\ewhile [b_w]^l_w \edo c_w)$ 
as the while command such that
$(\ewhile [b_w]^l_w \edo c_w) \in_c c_2$ and 
$[\assign{{x}_2}{\expr_2 / \query(\qexpr_2)}]^{l_2} \in_c c_w$.
\\
Then, we have the following possible execution instances:
 %
  \[
  \config{c_1, \vtrace_0 \cdot \vtrace_1 \cdot \event_1'} 
  \rightarrow^{\eskip^*} 
  \config{[\assign{{x}_2}{\expr_2 / \query(\qexpr_2)}]^{l_2} ; {c}_2, \vtrace_0 \cdot \vtrace_1 \cdot \event_1'} 
  \rightarrow^{assn/query} 
  \config{{c}_2,  \vtrace_0 \cdot \vtrace_1 \cdot \event_1' \cdot (x_2, l_2, n_2', v_2')} 
 \]
%
  \[
  \begin{array}{l}
  \config{c_1, \vtrace_0 \cdot \vtrace_1 \cdot \event_1'} 
  \rightarrow^{\eskip^*} 
  \config{[\assign{{x}_2}{\expr_2 / \query(\qexpr_2)}]^{l_2} ; {c}_2, \vtrace_0 \cdot \vtrace_1 \cdot \event_1'} 
  \rightarrow^{assn/query} 
  \config{{c}_2,  \vtrace_0 \cdot \vtrace_1 \cdot \event_1' \cdot (x_2, l_2, n_2', v_2')} 
  \\ \qquad
  \rightarrow^{*} 
  \config{[\assign{{x}_2}{\expr_2 / \query(\qexpr_2)}]^{l_2} ; {c}_2, 
  \vtrace_0 \cdot \vtrace_1 \cdot \event_1' \cdot (x_2, l_2, v_2') \cdot \trace_3} 
  \\ \qquad
  \rightarrow^{assn/query} 
  \config{{c}_2,  \vtrace_0 \cdot \vtrace_1 \cdot \event_1' \cdot (x_2, l_2, n_2', v_2') \cdot \trace_3 \cdot (x_2, l_2, n_2'', v_2'')} 
 \end{array}
 \]
\[
  \cdots
\] 
by iterating the $(\ewhile [b_w]^l_w \edo c_w)$ in $c_2$ $0$ or more times.
%
\\
%
For each execution instance, we have the corresponding instance of $\trace'$ as follows:
\\
$\trace'  = \cdot$
\\
$\trace' = \cdot (x_2, l_2, n_2', v_2') \cdot \trace_3 $
%
\\
$\cdots$
%
\\
%
By Eq.~\ref{eq:valdep_invn}, we know:
%
\[
 \vcounter(\trace') l_2 = 0
\]
%
Only the first execution with 0 iteration of while body in $c_2$ satisfy this restriction, i.e., $\trace' = \cdot$ and $\event_2' = (x_2, l_2, n_2', v_2')$
%
\end{enumerate}
In conclusion, we have the only qualified execution instance as follows where $\trace_2' = \cdot$.
  \[
  \config{c_1, \vtrace_0 \cdot \vtrace_1 \cdot \event_1'} 
  \rightarrow^{\eskip^*} 
  \config{[\assign{{x}_2}{\expr_2 / \query(\qexpr_2)}]^{l_2} ; {c}_2, \vtrace_0 \cdot \vtrace_1 \cdot \event_1'} 
  \rightarrow^{assn/query} 
  \config{{c}_2,  \vtrace_0 \cdot \vtrace_1 \cdot \event_1' \cdot (x_2, l_2, n_2', v_2')} 
 \]
\end{subproof}
%
By $\eventdep^{val}(\event_1, \event_2, c)$, and definition of environment, 
the environment only able to obtain different values for variable $x_1$ 
from trace $\vtrace_0 \cdot \vtrace_1 \cdot \event_1$ and 
$\vtrace_0 \cdot \vtrace_1 \cdot \event_1'$, i.e.,
\[
  \forall z^r \in \lvar_c \setminus \{x_1^{l_1}\} ,
  \env(\vtrace_0 \cdot \vtrace_1 \cdot \event_1) (z) =  
  \env(\vtrace_0 \cdot \vtrace_1 \cdot \event_1') (z)
\]
%
By {Inversion Lemma~\ref{lem:inv_a}} of arithmetic expression evaluation, we have
\[
  x_1 \in VAR(\expr_2 / \qexpr_2) 
\]
Since $\llabel() x_1 = l_1$, by Inversion Lemma~\ref{lem:inv_live} we know $x_1^{l_1} \in \live^{l_2}(c)$.
%
\\
%
By $\flowsto$ definition, we have:
%
\[
\flowsto(x_1^{l_1}, {x}_2^{l_2}, c)
\]
i.e.,
%
\[
\Big(\exists z_1^{r_1}, \cdots, z_n^{r_n} \in \lvar_{{c}} \st (0 \leq n )
 \land \flowsto(x_1^{l_1}, z_1^{r_1}, c) \land \cdots \land \flowsto(z_n^{r_n}, {x}_2^{l_2}, c) \Big)
 \]
%
This case is proved.
\end{subproof}
Proved in Subproof~\ref{pf:alg_correct_base}.
%
\end{case}
%
\begin{case}
($\trace_{ih} \in \mathcal{T}, \trace = \event_1 \tracecat \trace_{ih} \cdot \event_2 \land \trace_{ih} \neq \cdot$)
\\
The induction hypothesis are as follows:
%
\[
\begin{array}{l}
  \forall \event_{ih1}, \event_{ih2} \eventin \trace_{ih} \cdot \event_2 \st
   \\ \quad
   \exists \trace' \in \mathcal{T} \st 
   \trace[\event_{ih1}:\event_{ih2}] = \event_{ih1} \tracecat \trace' \cdot \event_{ih2}
   \implies 
   \forall z^i, y^j \in \lvar_c, l_h, l_t \st 
    \\ \qquad 
   l_h ++ [z^i, y^j] ++ l_t  \in \mathcal{A}(\event_{ih1}, \event_{ih2}, \trace[\event_{ih1}:\event_{ih2}], c, D)
   \implies \flowsto(z^i, y^j, c)
   % \\ 
\end{array}
\]
%
To show:
\[
\begin{array}{l}
  % \forall \event_{ih1}, \event_{ih2} \eventin \trace_{ih} \cdot \event_2 \st
  %  \\ \quad
  %  \exists \trace' \in \mathcal{T} \st 
  %  \trace[\event_{ih1}:\event_{ih2}] = \event_{ih1} \tracecat \trace' \cdot \event_{ih2}
  %  \implies 
  %  \forall z^i, y^j \in \lvar_c, l_h, l_t \st 
  %   \\ \qquad 
  %  l_h ++ [z^i, y^j] ++ l_t  \in \mathcal{A}(\event_{ih1}, \event_{ih2}, \trace[\event_{ih1}:\event_{ih2}], c, D)
  %  \implies \flowsto(z^i, y^j, c)
  %  \\
  %  \implies
    \exists \trace' \in \mathcal{T} \st \trace = \event_1 \tracecat \trace' \cdot \event_2
   \implies \forall  z^i, y^j \in \lvar_c, l_h, l_t \st
   \\ \quad \qquad 
   l_h ++ [z^i, y^j] ++ l_t \in \mathcal{A}(\event_1, \event_2, \trace, c, D)
   \implies \flowsto(z^i, y^j, c)  
\end{array}
\]
%
Let $\trace' = \trace_{ih}$, we know $\exists \trace' \in \mathcal{T}$ satisfying 
$\event_1  {}\tracecat \trace_{ih} \cdot {} \event_2 = \event_1 \tracecat \trace' \cdot \event_2$.
\\
Unfolding $\mathcal{A}(\event_1, \event_2, \trace, c, D)$, it is sufficient to show following 4 cases:
\begin{equation}
\begin{array}{l}
   \forall \event_z, \event_y \in \eventset^{\asn}, l_h, l_t \st 
   \\ \quad 
   l_h ++ [\event_z, \event_y] ++ l_t \in \kw{dfs}(\trace, c, D)
   \implies \flowsto(\pi_1(\event_z)^{\pi_2(\event_y)}, \pi_1(\event_y)^{\pi_2(\event_y)}, c)
\end{array}
\end{equation}
%
\begin{equation}
\begin{array}{l}
  \forall \event_z \in \eventset^{\asn}, \event_y \in \eventset^{\test}, l_h, l_t \st
   \\ \quad 
   l_h ++ [\event_z, \event_y] ++ l_t \in \kw{dfs}(\trace, c, D)
  \implies \pi_1(\event_z) \in VAR(\pi_1(\event_y))
\end{array}
\end{equation}
%
\begin{equation}
\begin{array}{l}
  \forall \event_z \in \eventset^{\test}, \event_y \in \eventset^{\asn}, l_h, l_t \st
   \\ \quad 
   l_h ++ [\event_z, \event_y] ++ l_t \in \kw{dfs}(\trace, c, D)
  \implies \forall z \in VAR(\pi_1(\event_z)) \st \exists i \in\mathbb{N} \st
  \flowsto(z^i, \pi_1(\event_y)^{\pi_2(\event_y)}, c)
\end{array}
\end{equation}
%
\begin{equation}
\begin{array}{l}
  \forall \event_z, \event_y \in \eventset^{\test}, l_h, l_t \st
   \\ \quad 
   l_h ++ [\event_z, \event_y] ++ l_t \in \kw{dfs}(\trace, c, D)
  \implies l_h ++ [\event_z] ++ l_t \in \kw{dfs}(\trace, c, D)
\end{array}
\end{equation}
%
\begin{subcase}($\event_z, \event_y \in \eventset^{\asn}$)
\\
Unfolding $\kw{dfs}(\trace, c, D)$, let $l$ be arbitrary $l \in  \kw{dfs}(\trace, c, D)$, it is sufficient to show 
\[
   \forall \event_z, \event_y \in \eventset^{\asn}, l_h, l_t \st 
   l = l_h ++ [\event_z, \event_y] ++ l_t 
   \implies \flowsto(\pi_1(\event_z)^{\pi_2(\event_y)}, \pi_1(\event_y)^{\pi_2(\event_y)}, c)
\] 
%
Proving by cases of $\event_z \eventeq \event_1 \lor \event_z \eventneq \event_1$.
\begin{subsubcase}[$\event_z \eventneq \event_1$]
Let $\event_z, \event_y \in \eventset^{\asn}, l_h, l_t$ be arbitrary events and lists such that:
\[
  l = l_h ++ [\event_z, \event_y] ++ l_t 
\]
%
Then it is sufficient to show $\flowsto(\pi_1(\event_z)^{\pi_2(\event_y)}, \pi_1(\event_y)^{\pi_2(\event_y)}, c)$.
\\
Unfolding $\kw{dfs}(\trace, c, D)$, we have:
%
% \[
%   l_h ++ [\event_z, \event_y] ++ l_t  \in 
%      \bigcup\limits_{l \in \kw{dfs}(\trace_{ih} \cdot \event_2, c, D) }
%   \left(  \eif \event_1 \stackrel{\trace}{\uplus} l \neq [] 
%   \ethen \left\{ \event_1 \stackrel{\trace}{\uplus} l \right\} \eelse \left\{ l \right\}
%   \right)
% \]
% %
% \todo{which is sound and correct and easy to prove}
\[
  l_h ++ [\event_z, \event_y] ++ l_t  \in 
  \left(  \kw{dfs}(\trace_{ih} \cdot \event_2, c, D) \bigcup\limits_{l \in \kw{dfs}(\trace_{ih} \cdot \event_2, c, D) } \event_1 \stackrel{\trace}{\uplus} l 
  \right)
\]
%
By $\event_z \neq \event_1$, we have:
\[
  [\event_z, \event_y] ++ l_t \in \kw{dfs}(\trace_{ih} \cdot \event_2, c, D)
\]
%
% Let $\trace_{tl}$ be a subtrace  of $\trace$, such that: 
% $\trace[\event_z:] = \cdot \event_z \tracecat \trace_{tl}$, we know $\trace_{tl}$ is also a subtrace  of $\trace_{ih} \cdot \event_2$.
% \\
% Unfolding $\kw{dfs}(\trace, c, D)$, by $l_h ++ [\event_z, \event_y] ++ l_t  \in\kw{dfs}(\trace, c, D)$, we have:
% \[
% \begin{array}{l}
%   \exists l_{h}' \st l_{h}' ++ [\event_y] ++ l_t \in \kw{dfs}( \trace_{tl}, c, D) \st
%   (\event_z \stackrel{\trace[\event_z:]}{\uplus} l_{h}' ++ [\event_y] ++ l_t) = \{[\event_z, \event_y] ++ l_t\} 
%   \end{array}
% \]
% %
% Then
% we have
% \[
%   [\event_z, \event_y] ++ l_t 
%   \in \kw{dfs}(\trace[\event_z:], c, D)
% \]
%
%
By induction hypothesis on subtrace $\trace_{ih} \cdot \event_2$ and $[\event_z, \event_y] ++ l_t 
  \in \kw{dfs}(\trace_{ih} \cdot \event_2, c, D)$ we know:
\[
  \flowsto(\pi_1(\event_z)^{\pi_2(\event_z)}, \pi_1(\event_y)^{\pi_2(\event_y)}, c)
\]
This case is proved.
%
\end{subsubcase}
%
\begin{subsubcase}[$\event_z \eventeq \event_1$].
\\
Let $\event_y \in \eventset^{\asn}, l_h, l_t$ be arbitrary events and lists such that:
\[                                                                                                                                                                                                                                                                                                                                                                                                          
  l = l_h ++ [\event_z, \event_y] ++ l_t 
\]
%
Unfolding $\kw{dfs}(\trace, c, D)$, we have:
%
\[
  l_h ++ [\event_z, \event_y] ++ l_t  \in 
     \bigcup\limits_{l \in \kw{dfs}(\trace_{ih} \cdot \event_2, c, D) }
  \left(  \eif \event_1 \stackrel{\trace}{\uplus} l \neq [] 
  \ethen \left\{ \event_1 \stackrel{\trace}{\uplus} l \right\} \eelse \left\{ l \right\}
  \right)
  \]
By $\event_z \eventeq \event_1$, we know $l_h = []$ and 
% \[
%   [\event_1, \event_y] ++ l_t  \in 
%   \left( \bigcup\limits_{l \in \kw{dfs}(\trace_{ih} \cdot \event_2, c, D)} 
%   \event_1 \stackrel{\trace}{\uplus} l \right)
% \]
% %
% Then, we also know
% \[
%   \left( \bigcup\limits_{l \in \kw{dfs}(\trace_{ih} \cdot \event_2, c, D)} 
%   \event_1 \stackrel{\trace}{\uplus} l \right) \neq \emptyset
% \]
%
\[
\begin{array}{l}
  \exists l' \in \kw{dfs}(\trace_{ih}\cdot \event_2, c, D) \st
   (\event_1 \stackrel{\trace}{\uplus} l') \neq []
\end{array}
\]
%
% Unfolding $\left( \bigcup\limits_{l \in \kw{dfs}(\trace_{ih} \cdot \event_2, c, D)} 
%   \event_1 \stackrel{\trace}{\uplus} l \right)$, by 
Since $[\event_1, \event_y] ++ l_t $ belongs to this set, let $l' \in \kw{dfs}(\trace_{ih}\cdot \event_2, c, D)$ be the list such that:
\[
\begin{array}{l}
   (\event_1 \stackrel{\trace}{\uplus} l') = \{[\event_1, \event_y] ++ l_t\}
\end{array}
\]
%
Unfolding $(\event_1 \stackrel{\trace}{\uplus} l') $, we have:
\[
\begin{array}{l}
  \exists l_{h}' \st l' = l_{h}' ++ [\event_y] ++ l_t \in \kw{dfs}(\trace_{ih}\cdot \event_2, c, D) \st
  \eventdep^{val}(\event_1, \event_y, \trace[\event_1:\event_y], c, D)
  \\ \quad \land 
  (\forall \event_i \in l_{h}' \st \neg \eventdep^{val} (\event_1, \event_i, \trace[\event_1: \event_i], c, D))
\end{array}
\]
%
%
To show $\flowsto(\pi_1(\event_1)^{\pi_2(\event_1)}, \pi_1(\event_y)^{\pi_2(\event_y)}, c)$, 
\todo{need to insert the detail of unfolding}
%
by unfolding $\kw{dfs}(\trace_{ih}\cdot \event_2, c, D) $, we have:
\[
  \forall \event \in (\trace[\event_1:\event_y]\setminus \{\event_1, \event_y\}) 
 \land \event \notin l_h' 
 \st 
 \neg \eventdep^{val} (\event, \event_y, c, D)
\]
%
Since $l_{h}'$ is subset of $\trace[\event_1:\event_y] \setminus \{\event_1, \event_y\}$, and $(\forall \event_i \in l_{h}' \st \neg \eventdep^{val} (\event_1, \event_i, \trace[\event_1: \event_i], c, D))$,
then we know:
\[
  \forall \event \in (\trace[\event_1:\event_y]\setminus \{\event_1, \event_y\}) \st
  \neg \eventdep^{val} (\event_1, \event, c, D) \lor \neg \eventdep^{val} (\event, \event_y, c, D)
\]
%
Unfolding $\eventdep^{val}(\event_1, \event_y, \trace[\event_1:\event_y], c, D)$, we have:
\[
  \exists \event_1', \event_y' \in \eventset^{\asn},
  \trace' \in \mathcal{T}, c_1, c_2, c_2' \st
\]
%
\[
\config{c_1, \trace_1 \cdot \event_1} \rightarrow^{*} \config{c_2, \trace_1 \tracecat \trace[\event_1:\event_y]} 
\land
\config{c_1, \trace_1 \cdot \event_1'} \rightarrow^{*} \config{c_2', \trace_1 \tracecat \trace'[\event_1':\event_y']} 
\]
%
%
By inversion Lemma.~\ref{lem:inv_asn}, we have exists an expression $\expr$ or $\qexpr$ such that:
\[
\config{c_1, \trace_1 \cdot \event_1} \rightarrow^{*} \config{[\assign{\pi_1(\event_y)}{\expr / \query(\qexpr)}]{}^{\pi_2(\event_y)};c_2, \trace_1 \tracecat \trace'} \rightarrow^{asn / query} \config{c_2, \trace_1 \tracecat \trace[\event_1:\event_y]} 
\]
%
By $\forall \event \in (\trace[\event_1:\event_y]\setminus \{\event_1, \event_y\}) \st
  \neg \eventdep^{val} (\event_1, \event, c, D) \lor \neg \eventdep^{val} (\event, \event_y, c, D)$,
we know $\event_1$ is the only cause of the difference in $\event_y$ and $\event_y'$ when evaluating $[\assign{\pi_1(\event_y)}{\expr / \query(\qexpr)}]{}^{\pi_2(\event_y)}$.
%
\\
By inversion Lemma.~\ref{lem:inv_a} and ~\ref{lem:inv_eval_asn}, 
\[
  \pi_1(\event_1) \in VAR(\expr)
\]
%
By $\flowsto$ definition:
\[
  \flowsto(\pi_1(\event_1)^{\pi_2(\event_1)}, \pi_1(\event_y)^{\pi_2(\event_y)}, c)
\]
%
\end{subsubcase}
%
\end{subcase}
%
%
\begin{subcase}[ $l_h ++ {[\event_z, \event_b]} ++ l_t$ ].
%
\end{subcase}
%
\begin{subcase}[ $l_h ++ {[\event_b, \event_y]} ++ l_t$ ].
%
\end{subcase}
%
\begin{subcase}[ $l_h ++ {[\event_{b_1}, \cdots, \event_{b_n}]} ++ l_t$ ].
%
\end{subcase}
%
\end{case}
%
\end{proof}
\clearpage
\section{Proof of the Soundness of Reachability Bounds Estimation}
\label{apdx:reachability_soundness}

{
  \begin{thm}[Soundness of the Reachability Bounds Estimation]
    \label{thm:addweight_soundness}
  Given a program ${c}$ with its program-based dependency graph $\progG = (\vertxs, \edges, \weights, \qflag)$, we have:
  %
  \[
  \forall x^l \in \lvar_c \st 
  \max \left\{ \vcounter(\vtrace') l ~ \middle\vert~
  \forall \vtrace \in \mathcal{T} \st \config{{c}, \trace} \to^{*} \config{\eskip, \trace\tracecat\vtrace'} \right\} 
  \leq 
  \weights(x^l)
  \]
  \end{thm}
}
\begin{proof}
  Taking an arbitrary a program ${c}$ with its program-based dependency graph $\progG = (\vertxs, \edges, \weights, \qflag)$, 
  and an arbitrary labelled variable $x^l \in \lvar_c$.
  \\
  By definition of $\progV$ in $\progG(c)$, we know $ x^l \in \vertxs$. 
  \\
  By Definition of $\progW$ in $\progG(c)$, we know $  \weights(x^l) = \absW(l) = \max \{ \absclr(\absevent) | \absevent = (l, \_, \_)\}$.
  \\
  By Lemma~\ref{lem:abscfg_sound}, there exists an abstract event in $\absflow(c)$ of form $(\absevent) = (l, \_, \_)$,
  corresponding to the assignment command associated to labeled variable $x^l$. 
  \\
  Let $(\absevent) = (l, dc, l') \in \absflow(c)$ be this event for some $dc$ and $l'$ such that  $(\absevent) = (l, dc, l') \in \absflow(c)$,
  by the last step of phase 2, we know
  $
  \progW(x^l) 
  \triangleq \absclr(\absevent)
  $.
   Then, it is sufficient to show:
  \[
    \max \left\{ \vcounter(\vtrace') l ~ \middle\vert~
  \forall \vtrace \in \mathcal{T} \st \config{{c}, \trace} \to^{*} \config{\eskip, \trace\tracecat\vtrace'} \right\} 
  \leq 
  \absclr(\absevent)
  \]
  % By line:2 of Algorithm~\ref{alg:add_weights}, there are 2 cases:
  By definition of $\absclr(\absevent)$:
  \[
 \begin{array}{ll}
  \locbound(\absevent) & \locbound(\absevent) \in \constdom \\
  Incr(\locbound(\absevent)) + 
  \sum\{\absclr(\absevent') \times \max(\varinvar(a) + c, 0) | (\absevent', a, c) \in \reset(\locbound(\absevent))\} 
  & \locbound(\absevent) \notin \constdom
\end{array}
\]
  \caseL{$\locbound(\absevent) \in \constdom$}
  \\
  Proved by the soundness of Local bound in Lemma~\ref{lem:local_bound_sound}.
  \caseL{$\locbound(\absevent) \notin \constdom$}
To show:
\[
  \begin{array}{l}
    \max \left\{ \vcounter(\vtrace') l ~ \middle\vert~
\forall \vtrace \in \mathcal{T} \st \config{{c}, \trace} \to^{*} \config{\eskip, \trace\tracecat\vtrace'} \right\} 
\\
\leq 
Incr(\locbound(\absevent)) + 
\sum\{\absclr(\absevent') \times \max(\varinvar(a) + c, 0) | (\absevent', a, c) \in \reset(\locbound(\absevent))\} 
\end{array}
\]
  % \caseL{$l \in prel$}
  % \\
  Taking an arbitrary initial trace
  $\trace_0 \in \mathcal{T}$, 
  executing $c$ with $\trace_0$, let $\trace$ be the trace after evaluation, i.e., $\config{{c}, \trace_0} \to^{*} \config{\eskip,\vtrace}$, it is sufficient to show:
  \[ 
    \begin{array}{l}
      \vcounter(\vtrace') l \leq 
    Incr(\locbound(\absevent)) + 
    \sum\{\absclr(\absevent') \times \max(\varinvar(a) + c, 0) | (\absevent', a, c) \in \reset(\locbound(\absevent))\}
  \end{array}
  \]
%
 By the soundness of the (1) Transition Bound and (2) Variable Bound Invariant 
 in \cite{sinn2017complexity} Theorem 1, 
This case is proved.
\end{proof}
%
\begin{lem}[Soundness of the Abstract Execution Trace]
  \label{lem:abscfg_sound}
Given a program ${c}$, we have:
%
\[
  \begin{array}{l}
    \forall \vtrace_0, \trace \in \mathcal{T} ,  \event = (\_, l, \_) \in \eventset \st
\config{{c}, \trace_0} \to^{*} \config{\eskip, \trace_0 \tracecat \vtrace} 
\land \event \in \trace 
\\
\qquad \implies \exists \absevent = (l, \_, \_) \in Label(c) \times Label(c) \times \absdom \st 
\absevent \in \absflow(c)
\end{array}
\]
\end{lem}
\begin{proof}
  Taking arbitrary $\trace_0 \in \mathcal{T}$, and an arbitrary event $\event = (\_, l, \_) \in \eventset$, it is sufficient to show:
  \[
  \begin{array}{l}
    \forall \trace \in \mathcal{T} \st
\config{{c}, \trace_0} \to^{*} \config{\eskip, \trace_0 \tracecat \vtrace} 
\land \event \in \trace 
\\
\qquad \implies \exists \absevent = (l, \_, \_) \in Label(c) \times Label(c) \times \absdom \st 
\absevent \in \absflow(c)
\end{array}
\]
  By induction on program $c$, we have the following cases:
  \caseL{$c = [\assign{x}{\expr}]^{l'}$}
  By the evaluation rule $\rname{assn}$, we have
  $
  {
  \config{[\assign{{x}}{\aexpr}]^{l'},  \trace } 
  \xrightarrow{} 
  \config{\eskip, \trace \tracecat [({x}, l', v) ]}
  }$, for some $v \in \mathbb{N}$ and $\trace = [({x}, l', v) ]$.
  \\
  There are 2 cases, where $l' = l$ and $l' \neq l$.
  \\
  In case of $l' \neq l$, we know $\event \not\eventin \trace$, then this Lemma is vacuously true.
    \\
    In case of $l' = l$, by the abstract Execution Trace computation, we know 
    $\absflow(c) = \absflow'([x := \expr]^{l}; \clabel{\eskip}^{l_e}) = \{(l, \absexpr(\expr), l_e)\}$  
    \\
  Then we have $\absevent = (l, \absexpr(\expr), l_e) $ and $\absevent \in \absflow(c)$.
  \\
  This case is proved.
  \caseL{$c = [\assign{x}{\query(\qexpr)}]^{l'}$}
  This case is proved in the same way as \textbf{case: $c = [\assign{x}{\expr}]^l$}.
  \caseL{$\ewhile [b]^{l_w} \edo c$}
  If the rule applied to is $\rname{while-t}$, we have
  \\
  $\config{{\ewhile [b]^{l_w} \edo c_w, \trace}}
    \xrightarrow{} 
    \config{{
    c_w; \ewhile [b]^{l_w} \edo c_w,
    \trace_0 \tracecat [(b, l, \etrue)]}}
  $.
  \\
  Let $\trace_w \in \mathcal{T}$ satisfying following execution:
  \\
  $
  \config{{
  c_w,
  \trace_0 \tracecat [(b, l_w, \etrue)]}}
  \xrightarrow{*} 
  \config{{
  \eskip,
  \trace_0 \tracecat [(b, l_w, \etrue)] \tracecat \trace_w}}
$
\\
Then we have the following execution:
\\
$\config{{\ewhile [b]^{l_w} \edo c_w, \trace}}
\xrightarrow{} 
\config{{
c_w; \ewhile [b]^{l_w} \edo c_w,
\trace_0 \tracecat [(b, l_w, \etrue)]}}
\xrightarrow{*} 
\config{{
  \ewhile [b]^{l_w} \edo c_w,
\trace_0 \tracecat [(b, l_w, \etrue)] \tracecat \trace_w}}
\xrightarrow{*} 
\config{{
\eskip,
\trace_0 \tracecat [(b, l_w, \etrue)] \tracecat \trace_w \tracecat \trace_1}}
$ for some $\trace_1 \in \mathcal{T}$ and $\trace = [(b, l_w, \etrue)] \tracecat \trace_w \tracecat \trace_1$.
\\
Then we have 3 cases: 
(1) $\event \eventeq (b, l_w, \etrue)$, 
(2) $\event \in \trace_w$ or 
(3) $\event \in \trace_1$.
  \\
In case of (1). $\event \eventeq (b, l_w, \etrue)$, since $\absflow(c) = \absflow'(c;\clabel{\eskip}^{l_e}) = \{(l, \top, \init(c_w))\} \cup \cdots $, we have $\absevent = (l, \top, \init(c_w))$ and this case is proved.
\\
In case of (2). $\event \in \trace_w$,
by induction hypothesis on 
$c_w$ with the execution 
  $\config{{
  c_w,
  \trace_0 \tracecat [(b, l_w, \etrue)]}}
  \xrightarrow{*} 
  \config{{
  \eskip,
  \trace_0 \tracecat [(b, l_w, \etrue)] \tracecat \trace_w}}$ and trace $\trace_w$, 
  we know there is an abstract event of the form 
  $\absevent' = (l, \_, \_ ) \in \absflow(c_w)$ where $\absflow(c_w) = \absflow'(c_w;\clabel{\eskip}^{l_e})$.
  \\
  Let $\absevent' = (l, dc, l')$ for some $dc$ and $l'$ such that $\absevent \in \absflow(c)$.
  \\
  By definition of $\absflow'$, we have 
  $ \absflow'(c_w;\clabel{\eskip}^{l_e}) = 
  \absflow'(c_w) \cup  \{ (l', dc, l_e) | (l', dc) \in \absfinal(c_w) \} $.
  \\
  There are 2 subcases: (2.1) $\absevent' \in \absflow'(c_w)$ or 
  $ (2.2) \absevent' \in \{ (l', dc, l_e) | (l', dc) \in \absfinal(c_w) \}$.
  \subcaseL{(2.1)}
  Since $\absflow(c) = \absflow'(c_w) \cup \{(l', dc, l_w)| (l', dc) \in \absfinal(c_w) \} \cup \cdots $, 
  we know the abstract event $\absevent' \in \absflow(c)$. 
  \\
  This case is proved.
  \subcaseL{(2.2) $\absevent' \in \{ (l', dc, l_e) | (l', dc) \in \absfinal(c_w) \}$ }
  In this case, we know $(l, dc) \in \absfinal(c_w)$.
  \\
  Since $\absflow(c) = \absflow'(c_w) \cup \{(l', dc, l_w)| (l', dc) \in \absfinal(c_w) \} \cup \cdots $, 
  we know $(l, dc, l_w) \in \{(l', dc, l_w)| (l', dc) \in \absfinal(c_w) \}$, 
   i.e., the abstract event $(l, dc, l_w) \in \absflow(c)$ and $(l, dc, l_w)$ has the form $(l, \_, \_)$.
  \\
  This case is proved.
  \\
  %
In case of (3). $\event \in \trace_1$, we know either $\event = (b, l_w, \_)$, or $\event \in \trace_w'$ where $\trace_w' \in \mathcal{T}$ is the trace of executing $c_w$ in an iteration.
\\
Then this case is proved by repeating the proof in case (1) and case (2).
  % And we also have the existence of $l = l_b, b$ and $c_w$, and $\ewhile [b]^{l} \edo c_w \in_c c_2$ and  $c_1 \in c_w$.
  % \\
  % If $c_w$ isn't a sequence command, let $c_1 = c_w$, then we have $c_2 = \ewhile [b]^{l} \edo c_w,  \eskip)$ 
  % and $c_1 \in_c c_2$.
  % \\
  % And we also have the existence of $l = l_b, b$ and $c_w$, and $\ewhile [b]^{l} \edo c_w \in_c c_2$ and  $c_1 \in c_w$.
  % \\
  \\
  If the rule applied to is $\rname{while-f}$, we have
  \\
  $
  {
    \config{{\ewhile [b]^{l_w} \edo c_w, \trace_0}}
    \xrightarrow{}^\rname{while-f}
    \config{{
    \eskip,
    \trace_0 \tracecat [(b, l_w, \efalse)]}}
  }$,
  In this case, we have $\trace = [(b, l_w, \efalse)]$ and $\event = (b, l_w, \efalse)$ (o.w., $\event \not\eventin \trace$ and this lemma is vacuously true) with $l = l_w$.
  \\
  By the abstract execution trace computation, $\absflow(c) = \{(l, \top, \init(c_w))\} \cup \cdots $, 
  we have $\absevent = (l, \top, \init(c_w))$  and $\absevent \in \absflow(c)$.
\\
  This case is proved.
  \caseL{$\eif([b]^l, c_t, c_f)$}
  This case is proved in the same way as \textbf{case: $c = \ewhile [b]^{l} \edo c$}.
  \caseL{$c = c_{s1};c_{s2}$}
 By the induction hypothesis on $c_{s1}$ and $c_{s2}$ separately, and the same step as case (2). of \textbf{case: $c = \ewhile [b]^{l} \edo c$},
 we have this case proved.
\end{proof}

\begin{lem}[Soundness of the Local Bound]
  \label{lem:local_bound_sound}
Given a program ${c}$, we have:
%
\[
\forall \absevent = (l, dc, l') \st 
\max \left\{ \vcounter(\vtrace') l ~ \middle\vert~
\forall \vtrace \in \mathcal{T} \st \config{{c}, \trace} \to^{*} \config{\eskip, \trace\tracecat\vtrace'} \right\} 
\leq 
\locbound(\absevent)
\]
\end{lem}
\begin{proof}
  \subcaseL{$l \notin SCC(\absG(c))$}
  In this case, we know variable $x^l$ isn't involved in the body of any $\ewhile$ command. 
  \\
  Taking an arbitrary $\vtrace_0 \in \mathcal{T}$, 
  let $\trace \in \mathcal{T}$ be of resulting trace of executing $c$ with $\trace$, 
  i.e., $\config{{c}, \trace_0} \to^{*} \config{\eskip, \trace}$,
  \\
  we know the
  assignment command at line $l$ associated with the abstract event $\absevent$ will be executed at most once, i.e.,:
  %
  $\vcounter(\vtrace) l \leq 1$
  \\
  By $\locbound$ definition, we know $\locbound(\absevent) = 1$.
  \\
  This case is proved.
  \subcaseL{$l \in SCC(\absG(c)) \land \absevent \in \dec(x) $}  in this case, we know $\locbound(\absevent) \triangleq x$.
  \subcaseL{$l \in SCC(\absG(c)) \land \absevent 
  \notin \bigcup_{x \in VAR} \dec(x)
  \land \absevent \notin SCC(\absG(c)/\dec(x)) $}  in this case, we know $\locbound(\absevent) \triangleq x$.
  \\
  In the two cases above, the soundness is discussed in \cite{sinn2017complexity} Section 4 of Paragraph \emph{Discussion on Soundness} in Page 25.
\end{proof}

% \begin{lem}[Soundness of the Variable Bound Invariant]
%   \label{lem:var_invariant_soundness}
% Given a program ${c}$, we have:
% %
% \[
% \forall x^l \in \lvar_c \st 
% \max \left\{ \vcounter(\vtrace') l ~ \middle\vert~
% \forall \vtrace \in \mathcal{T} \st \config{{c}, \trace} \to^{*} \config{\eskip, \trace\tracecat\vtrace'} \right\} 
% \leq 
% \rb(x^l, c)
% \]
% \end{lem}

% \begin{lem}[Soundness of the Transition Clousre ]
%   \label{lem:transition_closure_soundness}
% Given a program ${c}$, we have:
% %
% \[
% \forall x^l \in \lvar_c \st 
% \max \left\{ \vcounter(\vtrace') l ~ \middle\vert~
% \forall \vtrace \in \mathcal{T} \st \config{{c}, \trace} \to^{*} \config{\eskip, \trace\tracecat\vtrace'} \right\} 
% \leq 
% \rb(x^l, c)
% \]
% \end{lem}

%   {
%   \begin{lem}[Soundness of the Reachability Analysis]
%     \label{lem:reachability_soundness}
%   Given a program ${c}$, we have:
%   %
%   \[
%   \forall x^l \in \lvar_c \st 
%   \max \left\{ \vcounter(\vtrace') l ~ \middle\vert~
%   \forall \vtrace \in \mathcal{T} \st \config{{c}, \trace} \to^{*} \config{\eskip, \trace\tracecat\vtrace'} \right\} 
%   \leq 
%   \rb(x^l, c)
%   \]
%   \end{lem}
% }
% Proof Summary:
% \\
% 1. Translating of each command estimate the upper bound of the change of each variable showing up in the guard of the while command, in each iteration.
% \\
% 2. Composition of sequence either preserve the latest update of the variable, or compose it with variables flows to it.
% \\
% 3. Composition of if preserve the variable upper bound in both of the 2 branches.
% \\
% 4. Composition of a nested $\ewhile$ multiples the variable change upper bound by the bound of the nested while loop, which safely estimated the variable upper bound for the outside while loop.
% \\
% 5. Ranking function matches the pattern for every possibility and Give the max upper bound of changes for variable showing up inside the guard of the while.
% \\
% 6. By estimating the changes for all the variables in the boolean expression of the guard of the while in 1 iteration, computeBound divides the n by the changes of the boolean expression is the safe upper bound of how many times this while can looped. 
%
\begin{lem}[Uniqueness of the Abstract Event]
  \label{lem:absevent_unique}
Given a program ${c}$, we have:
%
\[
  \begin{array}{l}
    \forall x^l \in \lvar_c \st
\exists \absevent = (l, \_, \_) \in Label(c) \times Label(c) \times \absdom \st 
\absevent \in \absflow(c)
\end{array}
\]
\end{lem}
\begin{proof}
  This is proved trivially by induction on the program $c$.
\end{proof}
%
%
% \begin{lem}[Correspondance between $flow(c)$ and $\absflow(c)$]
%   \label{lem:flow_to_absflow}
% Given a program ${c}$, we have:
% %
% \[
% \forall \absevent = (l, dc, l') \st 
% \absevent \in \absflow(c) \land l' \neq l_e
% \implies (l, l') \in flow(c)
% \]
% \end{lem}
% \begin{proof}
%   This is proved trivially by induction on the program $c$.
% \end{proof}
\clearpage
\section{Proof of the Soundness of Data-Flow Refinement Analysis}
\label{apdx:rd_soundness}
\begin{thm}[Soundness of the Data-Flow Refinement Algorithm]
    \label{thm:rd_soundness}
    \[
    \forall c \in \cdom, x^i, y^j \in \lvar_c \st 
    \flowsto(x^i, y^j, c) \iff (x^i, y^j) \in dcdg(c)
    \]
    \end{thm}
\begin{proof}
    \label{pf:rd_soundness}
        By induction on the program $c$, we have the following cases:
        \caseL{$c = \clabel{\assign{x}{\expr}}^l$}
        We have $\lvar_c = \{x^l\}$, and the only choice for $x^i$ and $y^j$ is $x^l$.
        \subcaseL{$\implies$}
        By Definition~\ref{def:flowsto}, and $\flowsto(x^l, x^l, \clabel{\assign{x}{\expr}}^l)$, we know $x \in FV(\expr)$ and $x^l \in \live(l, c)$.
        \\
        Then, we have $dcdg(\clabel{\assign{x}{\expr}}^l) = \{(x^l, x^l)\}$ by definition~\ref{def:feasible_flowsto}
        \\
        This case is proved.
        \subcaseL{$\Longleftarrow$}
        by definition~\ref{def:feasible_flowsto} and $(x^l, x^l) \in dcdg(\clabel{\assign{x}{\expr}}^l)$, we have 
        $x \in FV(\expr)$ and $x^l \in \live(l, c)$.
        By Definition~\ref{def:flowsto}, we know $\flowsto(x^l, x^l, \clabel{\assign{x}{\expr}}^l)$.
        \\
        This case is proved.
        \caseL{$c = [\assign{x}{\query(\qexpr)}]^{l'}$}
        This case is proved in the same way as \textbf{case: $c = [\assign{x}{\expr}]^l$}.
        \caseL{$\ewhile \clabel{b}^{l_w} \edo c_w$}
        Take arbitrary $x^i, y^j \in \lvar_{\ewhile \clabel{b}^{l_w} \edo c_w}$, it is sufficient to show the following two directions.
        \subcaseL{$\implies$}
        By Definition~\ref{def:flowsto} and $\flowsto(x^i, y^j, \ewhile \clabel{b}^{l_w} \edo c_w)$ we have two sub-cases:
        \\
        (1) $\flowsto(x^i, y^j, c_w)$,
        By induction hypothesis, we have $(x^i, y^j) \in dcfg(c_w)$.
        \\
        By Definition~\ref{def:feasible_flowsto}, we know $dcfg(\ewhile \clabel{b}^{l_w} \edo c_w) = dcfg(c_w) \cup \{\cdots\}$.
        \\
        Then we have $(x^i, y^j) \in dcfg(\ewhile \clabel{b}^{l_w} \edo c_w)$, this case is proved.
        \\
        (2). $\exists \expr \lor \qexpr \st	\clabel{\assign{y}{\expr \lor \query(\qexpr)}}^{j} \in_{c}  {c_w}  
        \land {x} \in FV(\sbexpr) \land x^i \in \live(l, c)$.
        \\
        By $blocks$ definition, we have $ (\clabel{\assign{y}{\expr \lor \query(\qexpr)}}^{j}) \in blocks(c_w)$. 
        \\
        Then we know $(x^i, y^j) \in \{(x^i,y^j) | x \in FV(b) \land (x,i) \in \live(l, c) \land ([y = \_]^j) \in blocks(c_w)$, i.e., $(x^i, y^j) \in dcfg(\ewhile \clabel{b}^{l_w} \edo c_w)$.
        \\
        This case is proved.
        \subcaseL{$\Longleftarrow$}
        By Definition~\ref{def:feasible_flowsto} and $(x^i, y^j) \in dcfg(\ewhile \clabel{b}^{l_w} \edo c_w)$, we have two sub-cases:
        \\
        (1) $(x^i, y^j) \in dcfg(c_w)$.
        \\
        By induction hypothesis, we have $\flowsto(x^i, y^j, c_w)$. Then we have  $\flowsto(x^i, y^j, \ewhile \clabel{b}^{l_w} \edo c_w)$ by Definition~\ref{def:flowsto}.
        \\
        This case is proved.
        \\
        (2). $(x^i, y^j) \in \{(x^i,y^j) | x \in FV(b) \land (x,i) \in \live(l, c) \land ([y = \_]^j) \in blocks(c_w) \}$
        \\
        By $blocks$ definition, we know there exists $\expr$ or $\qexpr$ such that $ (\clabel{\assign{y}{\expr \lor \query(\qexpr)}}^{j}) \in_c c_w$. 
        \\
        Then, we know $\exists \expr \lor \qexpr \st	\clabel{\assign{y}{\expr \lor \query(\qexpr)}}^{j} \in_{c}  {c_w}  
        \land {x} \in FV(\sbexpr) \land x^i \in \live(l, c)$.
        \\
        Then we have $\flowsto(x^i, y^j, \ewhile \clabel{b}^{l_w} \edo c_w)$ by Definition~\ref{def:flowsto}.
        \\
        This case is proved.
        \caseL{$\eif([b]^l, c_t, c_f)$}
        This case is proved in the same way as \textbf{case: $c = \ewhile [b]^{l} \edo c$}.
        \caseL{$c = c_{s1};c_{s2}$}
       By the induction hypothesis on $c_{s1}$ and $c_{s2}$ separately, and the same step as case (2). of \textbf{case: $c = \ewhile [b]^{l} \edo c$},
       we have this case proved.   
\end{proof}
\clearpage
\section{Proof of the Soundness of Adaptivity Computation Algorithm}
\label{apdx:adaptalg_soundness}
\begin{thm}[Soundness of $\pathsearch$]
  \label{thm:sound_adaptalg}
  For every program $c$, given its \emph{Program-Based Dependency Graph} $\progG$,
   $$\pathsearch(\progG) \geq \progA(\progG).$$
\end{thm}
proof Summary:
1. for each SCC, a subgraph of $\progG$, $\pathsearch_{scc}(SCC) \geq \progA(SCC)$
2. for every two nodes with a path $x, \cdots, y$, let $\walks(k_{x,y})$ be all the walks from $x$ to $y$ on $\progG$,
then $adapt[SCC(x)] + \cdots + adapt[SCC(y)] \geq \max\{\qlen(k)\}$
\begin{proof}

\end{proof}
% %
\section{Proof of the Conditional Completeness of Adaptivity Computation Algorithm}
\label{apdx:adaptalg_completeness}
\begin{thm}[Partial Completeness of $\pathsearch$]
  \label{thm:adaptalg_pcomplete}
  For every program $c$, given its \emph{Program-Based Dependency Graph} $\progG$,
  if $\progG(c)$ is acyclic directed, then
   $$\pathsearch(\progG) = \progA(\progG).$$
\end{thm}
proof Summary:
\\
1. for every two vertices $x, y$ with a walk $k_{x,y}$ from $x$ to $y$ on $\progG$, 
\\
2 if they are on the same SCC, 
\\
2.1 Then this walk must also be in this SCC.
(By the property that each SCC are single direct connected, otherwise they are the same SCC)
\\
2.2 By Lemma~\ref{lem:sound_adaptalg_scc}, $\qlen$ of this walk is bound by the longest walk of this SCC.
\\
2.3 The output of $\pathsearch(\progG)$ is greater than longest walk of a single SCC.
\\
3. if they are on different SCC. 
\\
3.1 Then this walk can be split into $n, 2 \leq n$ sub-walks, and each sub-walk belongs to a different SCC. (Also by the property of SCC)
\\
3.2 By Lemma~\ref{lem:sound_adaptalg_scc}, $\qlen$ of each sub-walk is bound by the longest walk of the SCC it belongs to.
\\
% we have a path
%  $p_{x,y} = (x, v_1, \cdots, y)$ by the $\kw{bfs}$ algorithm,
% and $adapt[\sccgraph(x)] + adapt[\sccgraph(v_1)] + \cdots + adapt[\sccgraph(y)] \geq \{\qlen(k_{x,y})\}$.
3.3 By line: in algorithm, the output of $\pathsearch(\progG)$ is greater than sum up the $\qlen$ of longest walk in every SCC that each sub-walk belongs to.
\\
4. Then we have
$\pathsearch(\progG(c)) \geq \progA(c)$.
\begin{proof}
  Taking arbitrary program $c \in \cdom$, let $\progG(c) = (\progV, \progE, \progW, \progF)$ be its 
  program based dependency graph.
  \\
  Taking an arbitrary walk $k_{x,y} \in \walks{(\progG)}$, with vertices sequence
  $(x, s_1, \cdots, y)$, it is sufficient to show:
  \[
    \qlen(k_{x,y}) = \len(s | s \in (x, s_1, \cdots, y) \land \qflag(s) = 1) \leq \pathsearch(\progG(c))
  \]
  By line:3 of $\pathsearch(\progG)$ algorithm defined in Algorithm~\ref{alg:adpt_alg}, let $\kw{\sccgraph_1}, \cdots, \kw{\sccgraph_n}$ be all the strong connected components on $\progG$ with $0 \leq n \leq |\vertxs|$,
  where each $\kw{\sccgraph_i} = (\vertxs_i, \edges_i, \weights_i, \qflag_i)$,
  \\
  By line:5-6 in Algorithm~\ref{alg:adpt_alg}, let $\kw{adapt_{scc}[\sccgraph_i]}$ be the result of $\pathsearch_{\kw{scc}}(\sccgraph_i)$ for each $\sccgraph_i$.
    % i.e.,
  % \[
  %   \]
  \\
  There are 2 cases:
  \caseL{$x, y$ on the same SCC}
  Let  $\sccgraph$ be this SCC where vertices $x$ and $y$ on, by Lemma~\ref{lem:sound_adaptalg_scc}, we know
  \[
    \qlen(k_{x,y}) \leq \max\{\qlen(k) | k \in \walks(\sccgraph)\} \leq \pathsearch_{scc}(\sccgraph)
  \]
%
By line:15 and line:18 in $\pathsearch(\progG)$ algorithm in Algorithm~\ref{alg:adpt_alg}, 
let $\kw{adapt}$ be the output value,
we know $\pathsearch(\progG(c))  = \kw{adapt} \geq \kw{adapt_{tmp}} \geq  \kw{adapt_{scc}(SSC)} $.
\\
i.e., 
\[
  \qlen(k_{x,y}) \leq \pathsearch(\progG(c)) 
  \]
This case is proved.
%
%
\caseL{$x, y$ on different SSC}
Let $\sccgraph_x, \sccgraph_1, \cdots, \sccgraph_m, \sccgraph_y, 0 \leq m$ be all the SCC this walk pass by, where each vertex in 
$(x, s_1, \cdots, s_n, y) $ belongs to a single SCC number. 
\\
By the property of SCC, we know every 2 SCCs are single direct connected. Then we can divide this walk into $m+2$ sub-walks:
\\
$k_x = (x, s_1, \cdots, s_{scc_x})$;
\\
$k_1 = (s_{scc_x}, \cdots, s_{scc_1})$;
\\
$\cdots$
\\
$k_y = (s_{scc_m}, \cdots, s_y)$;
\\
where $k_x \in \walks(\sccgraph_x), \cdots, k_y \in \walks(\sccgraph_y)$.
\\
By Lemma~\ref{lem:sound_adaptalg_scc}, we know for each walk $k_i$:
\[ \qlen(k_i) \leq \max\{\qlen(k_i) | k_i \in \walks(\sccgraph_i)\} \leq \pathsearch_{scc}(\sccgraph_i) = \kw{adapt_{scc}[\sccgraph_i]} \]
%
Then we have:
\[ 
  \qlen(k_{x,y}) = \qlen(k_x) + \qlen(k_1) + \cdots + \qlen(k_y) \leq 
  \kw{adapt_{scc}[\sccgraph_1]} + \kw{adapt_{scc}[\sccgraph_1]}  + \cdots + \kw{adapt_{scc}[\sccgraph_y]}
  \leq \kw{adapt}
  \]
, where $\kw{adapt}$ is the output of $\pathsearch(\progG)$.
This case is proved.
\end{proof}

\begin{lem}[Soundness of $\pathsearch_{scc}$]
  \label{lem:sound_adaptalg_scc}
  For every program $c$, given its \emph{Program-Based Dependency Graph} $\progG$, if $\sccgraph$ is a strong connected sub-graph of $\progG$, then
  $\max\{\qlen(k) | k \in \walks(\sccgraph)\} \leq \pathsearch_{scc}(\sccgraph) $.
  %
  \[
    \forall c \in \cdom, \sccgraph \in \mathcal{Graph} \st \sccgraph \subseteq_{\kw{graph}} \progG(c)
    \implies 
    \max\{\qlen(k) | k \in \walks(\sccgraph)\} \leq \pathsearch_{scc}(\sccgraph) 
    \]
\end{lem}

ProofSummary:
\\
(1) for each node $x$ on SCC, by property of SCC, 
for every walk on SCC $k_{x, x} = (x, s_1, \cdots, x)$,
with set of unique vertex $\{v_1, \cdots, x\}$
there are $\paths(p_{x,x})$ on $\sccgraph$.
\\
(2) For every path $p_{x,x}^{i} = (x, v_1, \cdots, x) \in \paths(p_{x,x})$,  
$\kw{flowcapacity} (p_{x,x}^{i})$ is the maximum visiting times for every $v \in (x, v_1, \cdots, x)$, 
$\visit(s) (s_1, \cdots, x)) \leq \kw{flowcapacity}(p_{x,x}^{i})$;
\\
(3) $\kw{querynum}(p_{x,x}^{i})  * \kw{flowcapacity}(p_{x,x}^{i}$)  $\geq\len(s | s \in ( s_1, \cdots, x) \land \qflag(s) = 1) =  \qlen(k)$,
\\
(4) Then, the $\max\limits_{p_{x,x}^{i} \in \paths(p_{x,x})} \geq \max\{\qlen(k_{x, x}) | k_{x, x} \in \walks(k_{x, x})\}$
\\
(5) Then,  $\max\{\kw{querynum}(p_{x,x}^{i})  * \kw{flowcapacity}(p_{x,x}^{i}) | x \in \sccgraph \land {p_{x,x}^{i} \in \paths(p_{x,x})} \} 
\geq \max\{\qlen(k_{x, x}^i) |x \in \sccgraph \land  k_{x, x}^i \in \walks(k_{x, x})\}$
\\
(6) We also know by the property of SCC, $\forall x, y \in \sccgraph, $ let $k_{x, y}$ be arbitrary walk on $\sccgraph$,
 $\qlen(k_{x, y}) \leq \max\{\qlen(k_{x, x}^i) | k_{x, x}^i \in \walks(k_{x, x})\}$.
\\
(7) Then,$ \max\{\qlen(k_{x, x}^i) |x \in \sccgraph \land  k_{x, x}^i \in \walks(k_{x, x})\} \geq  \max\{\qlen(k_{x, y}^i) |x, y \in \sccgraph \land  k_{x, y}^i \in \walks(k_{x, y})\}$
\\
i.e., 
$ \max\{\qlen(k_{x, x}^i) |x \in \sccgraph \land  k_{x, x}^i \in \walks(k_{x, x})\} \geq  \max\{\qlen(k) | k\in \walks(\sccgraph)\} = \progA(\sccgraph)$.
\\
(8) We also know 
$\pathsearch_{scc}(\sccgraph) = \max\{\kw{querynum}(p_{x,x}^{i})  * \kw{flowcapacity}(p_{x,x}^{i}) | x \in \sccgraph \land {p_{x,x}^{i} \in \paths(p_{x,x})} \} $ by the $\pathsearch_{scc}$ algorithm.
\\
Then we have
$\pathsearch_{scc}(\sccgraph) \geq \progA(\sccgraph)$
\\
\begin{proof}
  Taking arbitrary program $c \in \cdom$, let $\progG(c) = (\vertxs, \edges, \weights, \qflag)$ be its 
  program based dependency graph and $\sccgraph = (\sccV, \sccE, \sccW, \sccF)$ be an arbitrary sub SCC graph of $\progG$.
  \\
There are 2 cases:
\caseL{$\sccgraph$ contains no edge and only 1 vertex $v$, i.e., $|\edges| = 0 \land |\vertxs| = 1$}
%
In this case there is no walk in this graph, i.e., $\walks(\sccgraph) = \emptyset$.
\\
The adaptivity is $\qflag(v)$.
\\
This case is proved.
  %
  \caseL{$\sccgraph$ contains at least 1 edge and at least 1 vertex $v$, i.e., $1 \leq |\edges| \land 1 \leq |\vertxs|$}
%
  Taking arbitrary walk $k_{x,y} \in \walks{(\sccgraph})$, with vertices sequence
  $(x, s_1, \cdots, y)$, it is sufficient to show:
  \[
    \qlen(k_{x,y}) = \len(s | s \in (x, s_1, \cdots, y) \land \qflag(s) = 1) \leq \pathsearch_{scc}(\sccgraph)
  \]
  By $\pathsearch_{scc}(\sccgraph)$ algorithm line 19, in the iteration where $x$ is the starting vertex,
  we know $\pathsearch_{scc}(\sccgraph) = \kw{r_{scc}} = \max(\kw{r_{scc}, \kw{dfs(\sccgraph, x, visited)}})$,
  then it is sufficient to show:
  $$
  \len(s | s \in (x, s_1, \cdots, y) \land \qflag(s) = 1) \leq \kw{dfs(\sccgraph, x, visited)}.
  $$
  %
  Let  $\{v_1, \cdots, x\}$ be the set of all the distinct vertices of $k_{x,y}$'s vertices sequence $(x, s_1, \cdots, y)$, and 
  $(v_1, \cdots, x)$ be a subsequence containing all the vertices in $\{x, v_1, \cdots, y\}$.
  \\
  By the definition of walk,
 there  is a path $p_{x,y} $ from $x$ to $y$ with this vertices sequence: $(x, v_1, \cdots, y)$.
  \\
  By line:13 of the $\kw{dfs(\sccgraph, x, visited)}$ in Algorithm~\ref{alg:adaptscc},
  \\
  we know $\kw{dfs(\sccgraph, x, visited)} = r[x]$ and
  $r[x] = \max\{\kw{flowcapacity}(p) \times \kw{querynum}(p) | p \in \paths_{x,x}(\sccgraph)\}$,
  where $\paths_{x,x}(\sccgraph)$ is a subset of $\paths_{x,x}(\sccgraph)$, in which every path starts from $x$ and goes back to $x$.
  \\
  By the property of strong connected graph, we know in this case  $\paths_{x,x}(\sccgraph) \neq \emptyset$ and there are 2 cases, $x = y$ and $x \neq y$.
  \caseL{$x = y$}
  In this case, we know $p_{x, y} \in p \in \paths_{x,x}(\sccgraph)$,  then it is sufficient to show: 
  % $\len(s | s \in (x, s_1, \cdots, y, v_1', \cdots, x) \land \qflag(s) = 1) \leq r[x]$.
  % \\
  % \\
  % Then we know 
  % \\
  $$ 
  \len(s | s \in (x, s_1, \cdots, y) \land \qflag(s) = 1) \leq \kw{flowcapacity}(p_{x, y}) \times \kw{querynum}(p_{x, y} ) 
  $$
  %
  % , where $(p_{x, y} + p_{y, x})$ is the path $p_{x, y}$ concatenated by path $p_{y, x}$ and we know $(p_{x, y} + p_{y, x}) \in \paths(\sccgraph)$.
  \\
By line:7 and line:13 in Algorithm~\ref{alg:adaptscc}, we know $\kw{flowcapacity}(p_{x, y})$ is the maximum visiting times for every $v \in (x, v_1, \cdots, y)$, 
\\
we know for every $s$ in the vertices sequence of walk $k_{x,y}$, 
$\visit(s) (x, s_1, \cdots, y)  \leq \kw{flowcapacity}(p_{x, y})$
  \\
  Also by line:8 and line:13 in Algorithm~\ref{alg:adaptscc}, we know $\kw{querynum}(p_{x, y})$ is the number of vertices with $\qflag$ equal to $1$,
  \\
  Then we know 
  \\
  $\len(s | s \in (x, s_1, \cdots, y) \land \qflag(s) = 1) \leq \kw{flowcapacity}(p_{x, y}) \times \kw{querynum}(p_{x, y}) $
  \\
  This case is proved.
  \caseL{$x \neq y$}
  we also have a path start from $y$ and go back to $x$.
  % \todo{pass through the same vertices sequence.}
  \\
  Let $p_{y, x}$ be this path with vertices sequence $(y, v_1', \cdots, x)$, we have a path $p_{x,x}$, which is the path $p_{x, y}$ concatenated by path $p_{y, x}$ with vertices sequence $ (x, v_1, \cdots, y, v_1', \cdots, v_{m}', x)$, where $m \geq 0$.
  \\
  % Then we have a potential walk $k_{x,x}$ with vertices sequence $(x, s_1, \cdots, y) + (v_1', \cdots, x)$, and
  % we know $\qlen(k_{x,y}) \leq \qlen(k_{x,x})$.
  % Then, it is sufficient to show 
  % $$
  % \qlen(k_{x,x}) = \len(s | s \in (x, s_1, \cdots, y, v_1', \cdots, x) \land \qflag(s) = 1) \leq \kw{dfs(\sccgraph, x, visited)}
  % $$
  %
  % By line:13 of the $\kw{dfs(\sccgraph, x, visited)}$ in Algorithm~\ref{alg:adaptscc},
  % \\
  % we know $\kw{dfs(\sccgraph, x, visited)} = r[x]$ and
  % $r[x] = \max\{\kw{flowcapacity}(p) \times \kw{querynum}(p) | p \in \paths_{x,x}(\sccgraph)\}$.
  % \\
  Then in this case, it is sufficient to show: 
  % $\len(s | s \in (x, s_1, \cdots, y, v_1', \cdots, x) \land \qflag(s) = 1) \leq r[x]$.
  % \\
  % \\
  % Then we know 
  % \\
  $$ 
  \len(s | s \in (x, s_1, \cdots, y) \land \qflag(s) = 1) \leq \kw{flowcapacity}(p_{x, x}) \times \kw{querynum}(p_{x, x}) 
  $$
  %
  % \\
Since $\kw{flowcapacity}(p_{x, y} + p_{y, x})$ is the maximum visiting times for every $v \in (x, v_1, \cdots, y, v_1', \cdots, x)$, 
\\
By line:7 in Algorithm~\ref{alg:adaptscc}, we know $\kw{flowcapacity}(p_{x, y})$ is the maximum visiting times for every $v \in (x, v_1, \cdots, y)$, 
\\
we know for every $s$ in the vertices sequence of walk $k_{x,y}$, 
$\visit(s) (x, s_1, \cdots, y)  \leq \kw{flowcapacity}(p_{x, y})$
  \\
  Also by line:8 in Algorithm~\ref{alg:adaptscc}, we know $\kw{querynum}(p_{x, y})$ is the number of vertices with $\qflag$ equal to $1$,
  \\
  Then we know 
  \\
  $\len(s | s \in (x, s_1, \cdots, y) \land \qflag(s) = 1) \leq \kw{flowcapacity}(p_{x, y}) \times \kw{querynum}(p_{x, y}) = r[y]$
  \\
  By line:13, we also know $r[x] = \max(r[x], r[v_{m}'] + \kw{flowcapacity}(p_{x, x}) \times \kw{querynum}(p_{x, x})$, and $r[y] \leq r[w_{m}']$
  then we know $r[y] \leq r[x]$, i.e., 
  $\len(s | s \in (x, s_1, \cdots, y) \land \qflag(s) = 1) \leq r[x]$
  \\
  This case is proved.
  %
%
%
\end{proof}
\bibliographystyle{plain}
\bibliography{adaptivity.bib}

\end{document}



