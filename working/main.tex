\documentclass[a4paper,11pt]{article}
\usepackage[table]{xcolor}



%%% Attempt 1: Linear 1



\newcommand{\diam}{{\color{red}\diamond}}
\newcommand{\dagg}{{\color{blue}\dagger}}
\let\oldstar\star
\renewcommand{\star}{\oldstar}

\newcommand{\im}[1]{\ensuremath{#1}}

\newcommand{\kw}[1]{\im{\mathtt{#1}}}


\newcommand{\set}[1]{\im{\{{#1}\}}}

\newcommand{\mmax}{\ensuremath{\mathsf{max}}}

%%%%%%%%%%%%%%%%%%%%%%%%%%%%%%%%%%%%%%%%%%%%%%%%%%%%%%%%
% Comments
\newcommand{\omitthis}[1]{}

% Misc.
\newcommand{\etal}{\textit{et al.}}
\newcommand{\bump}{\hspace{3.5pt}}

% Text fonts
\newcommand{\tbf}[1]{\textbf{#1}}
%\newcommand{\trm}[1]{\textrm{#1}}

% Math fonts
\newcommand{\mbb}[1]{\mathbb{#1}}
\newcommand{\mbf}[1]{\mathbf{#1}}
\newcommand{\mrm}[1]{\mathrm{#1}}
\newcommand{\mtt}[1]{\mathtt{#1}}
\newcommand{\mcal}[1]{\mathcal{#1}}
\newcommand{\mfrak}[1]{\mathfrak{#1}}
\newcommand{\msf}[1]{\mathsf{#1}}
\newcommand{\mscr}[1]{\mathscr{#1}}









\newcommand{\defeq}{\mathrel{\doteq}}
\newcommand{\conj}{\mathrel{\wedge}}
\newcommand{\disj}{\mathrel{\vee}}

\newcommand{\lzero}{0}


% context
\newcommand{\tctx}{\Gamma}
\newcommand{\ictx}{ }


% expression
\newcommand{\expr}{e}
\newcommand{\aexpr}{a}
\newcommand{\bexpr}{b}
\newcommand{\sexpr}{\textrm{e} }
\newcommand{\qexpr}{\psi}
\newcommand{\qval}{\alpha}
\newcommand{\query}{{\tt query}}
\newcommand{\saexpr}{\textrm{a} }
\newcommand{\sbexpr}{\textrm{b} }
\newcommand{\vall}{w}
\newcommand{\valr}{v}
\newcommand{\eif}{\kw{if}}
\newcommand{\eapp}{\;}
\newcommand{\eprojl}{\kw{fst}}
\newcommand{\eprojr}{\kw{snd}}
\newcommand{\eifvar}{\kw{ifvar}}
%expression and commands for WHILE language
\newcommand{\ewhile}{\kw{while}}
\newcommand{\bop}{*}
\newcommand{\uop}{\circ}
\newcommand{\eskip}{\kw{skip}}

\newcommand{\eloop}{\kw{loop}}
\newcommand{\edo}{\kw{do}}
\newcommand{\qdom}{\mathcal{QD}}

%configuration
\newcommand{\config}[1]{\langle #1 \rangle}
\newcommand{\ematch}{\kw{match}}
\newcommand{\clabel}[1]{\left[ #1 \right]}


%\newcommand{\eprov}[1]{\eta_{#1}}
\newcommand{\etrue}{\kw{true}}
\newcommand{\efalse}{\kw{false}}
\newcommand{\econst}{c}
\newcommand{\eop}{\delta}
\newcommand{\efix}{\mathop{\kw{fix}}}
\newcommand{\elet}{\mathop{\kw{let}}}
\newcommand{\ein}{\mathop{ \kw{in}} }
\newcommand{\eas}{\mathop{ \kw{as}} }
\newcommand{\enil}{\kw{nil}}
\newcommand{\econs}{\mathop{\kw{cons}}}
%\newcommand{\labelA}{\ell}
%monad expressions / terms
\newcommand{\term}{t}
\newcommand{\return}{\kw{return}}
\newcommand{\bernoulli}{\kw{bernoulli}}
\newcommand{\uniform}{\kw{uniform}}
 \newcommand{\epack}{\mbox{pack\;}}
\newcommand{\eunpack}{\mbox{unpack\;}}
\newcommand{\eilam}{\Lambda.}

\newcommand{\evec}{\kw{dict}}
\newcommand{\eget}{\kw{get}}

% trace
\newcommand{\triapp}[2]{\kw{IApp}(#1,#2)}
\newcommand{\trow}{\text{row}}
\newcommand{\tr}{T}
\newcommand{\trift}{\eif^{\kw{t}}}
\newcommand{\triff}{\eif^{\kw{f}}}
\newcommand{\trprojl}{\eprojl}
\newcommand{\trprojr}{\eprojr}
\newcommand{\trtrue}{\etrue}
\newcommand{\trfalse}{\efalse}
\newcommand{\trconst}{\econst}
\newcommand{\trop}{\eop}
\newcommand{\trfix}{\efix}
\newcommand{\trapp}[5]{#1 \; #2 \mathrel{\triangleright} {\efix
#3(#4).#5}}
\newcommand{\trnil}{\enil}
\newcommand{\trcons}{\econs}
\newcommand{\trlet}{\elet}
%types for monad
\newcommand{\treal}{\kw{real}}
\newcommand{\tint}{\kw{int}}
\newcommand{\tmonad}{\kw{M}}
\newcommand{\tunit}{\kw{unit}}
\newcommand{\tdb}{\kw{tdb}}

% adaptivity
\newcommand{\adap}{\kw{adap}}
\newcommand{\ddep}[1]{\kw{depth}_{#1}}
\newcommand{\nat}{\mathbb{N}}
\newcommand{\natb}{\nat_{\bot}}
\newcommand{\natbi}{\natb^\infty}
\newcommand{\nnatA}{Z}
\newcommand{\nnatB}{m}
\newcommand{\nnatbA}{s}
\newcommand{\nnatbB}{t}
\newcommand{\nnatbiA}{q}
\newcommand{\nnatbiB}{r}

%type
\newcommand{\type}{\tau}
\newcommand{\tbase}{\kw{b}}
\newcommand{\tbool}{\kw{bool}}
\newcommand{\tbox}[1]{ \kw{\square} \, (#1) }
\newcommand{\tarr}[5]{#1; #3 \xrightarrow{#4; \, #5} #2}
\newcommand{\tlist}[1]{\kw{list} \, #1 }
\newcommand{\env}{\theta}
\newcommand{\tforall}[3]{\forall#3 \overset{#1, #2}{::} S.\, }
\newcommand{\texists}[1]{\exists#1 {::} S.\, }
\newcommand{\lto}{\multimap}
\newcommand{\bang}[1]{ !_{#1}}
\newcommand{\whynot}[1]{ ?_{#1} }
\newcommand{\ltype}{A}
\newcommand{\adapt}{R}
% index
\newcommand{\idx}{I }
\newcommand{\smax}[2]{\kw{max}(#1,#2)}
\newcommand{\ienv}{\sigma}

%evaluation
\newcommand{\bigstep}[1]{\mathrel{\to^{#1}}}

\newcommand{\dmap}{\rho}
\newcommand{\dmapb}{\bot_\dmap}
\newcommand{\supp}{\kw{supp}}
\newcommand{\dom}{\kw{dom}}
\newcommand{\codom}{\kw{codom}}

\newcommand{\tvdash}[1]{\vdash_{#1}}

\newcommand{\lrv}[1]{[\![ #1 ]\!]_{\text{V}}}
\newcommand{\lre}[3]{[\![ #3 ]\!]_{\text{E}}^{#1, #2}}
\newcommand{\stepiA}{k}
\newcommand{\stepiB}{j}
\newcommand{\size}[1]{|#1|}

%logic relations
\newcommand{\lr}[1]{[\![ #1 ]\!]}
\newcommand{\lrt}[1]{\mathcal{T}[\![ #1 ]\!]}


\newcommand{\wf}[1]{\vdash #1 \, \kw{wf} }
\newcommand{\sub}[2]{ #1 \, <: \, #2 }
\newcommand{\eqv}[3]{ #1 \, \equiv \, #2 \Rightarrow \textcolor{red}
{#3}  }
\newcommand{\eqvt}[3]{ #1 \, \sqsubseteq \, #2 \Rightarrow \textcolor{red}
{#3}  }
\newcommand{\eqvc}[2]{ #1 \, \equiv^c \, #2   }


%core calculus
\newcommand{\ctyping}[3]{ \tvdash{ #1} {#2} :^c #3 }
\newcommand{\cbox}{\mathsf{box}}
\newcommand{\cder}{\mathsf{der}}
\newcommand{\elab}[4]{ \vdash_{ #1} #2 \rightsquigarrow #3 : #4}
\newcommand{\coerce}[2]{\mathsf{coerce}_{#1, #2}}

%algorithmic typing rules
\newcommand{\infr}[4]{{#1} ~ {\textcolor{red}\uparrow} ~ {\color{red} #2} \Rightarrow
{ } {\color{red} #3} }
\newcommand{\chec}[3]{{#1} ~ {\downarrow} ~ {#2} \Rightarrow {\color{red} #3} }
% \newcommand{\restriction}{\Phi}
\newcommand{\fresh}{ \mathsf{fresh}}
\newcommand{\red}[1]{ \textcolor{red} {#1} }
\newcommand{\fiv}[1]{ \mathsf{FIV} (#1)   }
\newcommand{\fv}[1]{ \mathsf{FV} (#1)   }

\newcommand{\todo}[1]{{\small \color{red}\textbf{[[ #1 ]]}}}
\newcommand{\todomath}[1]{{\scriptstyle \color{red}\mathbf{[[ #1 ]]}}}

\newcommand{\caseL}[1]{\item \textbf{#1}\newline}

\newcommand{\attr}{\mathsf{attr}}
\newcommand{\weight}{\mathsf{W}}
\newcommand{\num}{\mathsf{n}}

\usepackage{enumitem}
\setenumerate{listparindent=\parindent}

\newlist{enumih}{enumerate}{3}
\setlist[enumih]{label=\alph*),before=\raggedright, topsep=1ex, parsep=0pt,  itemsep=1pt }

\newlist{enumconc}{enumerate}{3}
\setlist[enumconc]{leftmargin=0.5cm, label*= \arabic*.  , topsep=1ex, parsep=0pt,  itemsep=3pt }


\newlist{enumsub}{enumerate}{3}
\setlist[enumsub]{ leftmargin=0.7cm, label*= \textbf{subcase} \bf \arabic*: }

\newlist{enumsubsub}{enumerate}{3}
\setlist[enumsubsub]{ leftmargin=0.5cm, label*= \textbf{subsubcase} \bf \arabic*: }

\newlist{mainitem}{itemize}{3}
\setlist[mainitem]{ leftmargin=0cm , label= {\bf Case} }

%%%%COLORS
\definecolor{periwinkle}{rgb}{0.8, 0.8, 1.0}
\definecolor{powderblue}{rgb}{0.69, 0.88, 0.9}
\definecolor{sandstorm}{rgb}{0.93, 0.84, 0.25}
\definecolor{trueblue}{rgb}{0.0, 0.45, 0.81}


\usepackage{array}

\newlength\Origarrayrulewidth

% horizontal rule equivalent to \cline but with 2pt width
\newcommand{\Cline}[1]{%
 \noalign{\global\setlength\Origarrayrulewidth{\arrayrulewidth}}%
 \noalign{\global\setlength\arrayrulewidth{2pt}}\cline{#1}%
 \noalign{\global\setlength\arrayrulewidth{\Origarrayrulewidth}}%
}

% draw a vertical rule of width 2pt on both sides of a cell
\newcommand\Thickvrule[1]{%
  \multicolumn{1}{!{\vrule width 2pt}c!{\vrule width 2pt}}{#1}%
}

% draw a vertical rule of width 2pt on the left side of a cell
\newcommand\Thickvrulel[1]{%
  \multicolumn{1}{!{\vrule width 2pt}c|}{#1}%
}

% draw a vertical rule of width 2pt on the right side of a cell
\newcommand\Thickvruler[1]{%
  \multicolumn{1}{|c!{\vrule width 2pt}}{#1}%
}

\newcommand{\command}{c}
\newcommand{\green}[1]{{ \color{green} #1 } }

\newcommand{\func}[2]{\mathsf{AD}(#1) \to (#2)}
\newcommand{\varEst}{\bf{VetxEst}}
\newcommand{\graphGen}{\bf{GraphGen}}

\newcommand{\ag}[2]{\mathsf{VetxEst}{(#1)}\to {(#2)}}
\newcommand{\ad}[2]{\mathsf{GraphGen}{(#1)}\to {(#2)}}
\newcommand{\rb}{\mathsf{RechBound}}
\newcommand{\pathsearch}{\mathsf{AdaptPathSearch}}

%Packages
\usepackage[T1]{fontenc}
\usepackage{fourier} 
\usepackage[english]{babel} 
\usepackage{amsmath,amsfonts,amsthm} 
\usepackage{lscape}
\usepackage{geometry}
\usepackage{amsmath}
\usepackage{algorithm}
\usepackage{algorithmic}
\usepackage{amssymb}
\usepackage{amsfonts}
\usepackage{times}
\usepackage{bm}
\usepackage{ stmaryrd }
\usepackage{ amssymb }
\usepackage{ textcomp }
\usepackage[normalem]{ulem}
% For derivation rules
\usepackage{mathpartir}
\usepackage{color}
\usepackage{a4wide}


\newcommand{\sctag}[1]{\tag{\textsc{#1}}\label{eq:#1}}
\newcommand{\ands}{~\wedge~}

%Variations
\newcommand{\astable}{\mathbb{S}}
\newcommand{\achange}{U}

%Index Terms
\newcommand{\scond}[3]{(\eif\im{#1}\ethen\im{#2}\eelse\im{#3})}
\newcommand{\keps}[2]{\epsilon(#1,#2)}
\newcommand{\spower}[2]{#1^{#2}}
\newcommand{\ssum}[4]{\sum\limits_{#1=#2}^{#3}#4}
\newcommand{\smin}[2]{\kw{min}({#1},{#2})}
\newcommand{\smax}[2]{\kw{max}(#1,#2)}
\newcommand{\smaxx}[3]{\kw{max}(#1,#2,#3)}
\newcommand{\smaxxx}[4]{\kw{max}(#1,#2,#3,#4)}

%Sizes
\newcommand{\szero}{0}
\newcommand{\sone}{1}
\newcommand{\splus}[2]{#1 + #2}
\newcommand{\ssucc}[1]{#1 {+} \sone}
\newcommand{\sminus}[2]{#1 - #2}
\newcommand{\sdiv}[2]{\frac{#1}{#2}}
\newcommand{\smult}[2]{#1\cdot#2}
\newcommand{\splusone}[1]{#1+1}
\newcommand{\sceil}[1]{\ceil*{#1}}
\newcommand{\sfloor}[1]{\floor*{#1}}
\newcommand{\size}[1]{|#1|}
\newcommand{\slog}[1]{\kw{log}_2(#1)}
\newcommand{\sinf}{\infty}

%Sorts
\newcommand{\ssize}{\mathbb{N}}
\newcommand{\svar}{\mathbb{V}}
\newcommand{\scost}{\mathbb{R}}
\newcommand{\sfun}[2]{#1\mbox{\ra} #2}
\newcommand{\sfunmon}[2]{#1\xrightarrow{\mbox{mon}} #2}
% \newcommand{\sort}{\varsigma}
\newcommand{\sort}{S}
\newcommand{\sorted}[1]{#1 \mathrel{::} \sort}
\newcommand{\sized}[1]{#1 \mathrel{::} \ssize}

% Types
\newcommand{\grt}{A}
\newcommand{\lbound}{\mathop{\uparrow}}
\newcommand{\tbool}{\mbox{bool}}
\newcommand{\trbool}{\mbox{bool}_r}
\newcommand{\tubool}{\mbox{bool}_u}
\newcommand{\trint}{\mbox{int}_r}
\newcommand{\tint}{\mbox{int}}
\newcommand{\tquery}{\mbox{query}}
\newcommand{\tunit}{\mbox{unit}}
\newcommand{\trunit}{\mbox{unit}_r}
\newcommand{\tlist}[3]{\mbox{list}[#1]^{#2}\,#3}
\newcommand{\tlists}[1]{ #1 \, \mbox{list} }
\newcommand{\ulist}[2]{\mbox{list}[#1]\,#2}
\newcommand{\tslist}[1]{\mbox{list}\,#1}
\newcommand{\ttree}[3]{\mbox{tree}[#1]^{#2}\,#3}
\newcommand{\utree}[2]{\mbox{tree}[#1]\,#2}
\newcommand{\tbase}{b}
\newcommand{\uarr}[2]{\mathrel{\xrightarrow[]{\wexec(#1,#2)}}} 
\newcommand{\uarrs}[1]{\mathrel{\xrightarrow[]{\mu(#1)}}} 
\newcommand{\uarrd}{\mathrel{\xrightarrow{\wdead}}}
\newcommand{\tarrd}[1]{\mathrel{\xrightarrow{\wdiff(#1)}}}



\newcommand{\tforall}[3]{\forall#1\overset{\wexec(#2,#3)}{::}S.\,}
\newcommand{\tforalld}[2]{\forall#1\overset{\wdiff(#2)}{::}S.\,}
\newcommand{\uforalls}[2]{\forall#1\overset{\mu(#2)}{::}S.\,}
\newcommand{\tforallS}[1]{\forall#1.\,}
\newcommand{\tsforall}[1]{\forall#1{::}S.\,}
\newcommand{\tforallN}[1]{\forall#1{::}\ssize.\,}
\newcommand{\texists}[1]{\exists#1{::}S.\,}
\newcommand{\texistsN}[1]{\exists#1{::}\ssize.\,}
\newcommand{\tcimpl}[2]{#1 \mathrel{\supset} #2}
\newcommand{\tcprod}[2]{#1 \mathrel{\&} #2}

\newcommand{\ttimes}{\mathrel{\times}}
\newcommand{\tsum}{\mathrel{+}}
\newcommand{\tinter}{\mathrel{\wedge}}
\newcommand{\tst}[1]{(#1)^{\astable}}
\newcommand{\tch}[2]{U\,(#1,#2)} 
\newcommand{\tchs}[1]{U\,(#1,#1)} 
\newcommand{\tcho}[1]{U\,#1} 
\newcommand{\tmu}[1]{(#1)^{\mu}}
\newcommand{\tno}[1]{(#1)^{\_}}
\newcommand{\trm}[2]{|#1|_{#2}}
\newcommand{\trmo}[1]{|#1|}
\newcommand{\tmup}[1]{(#1)^{\mu'}}
\newcommand{\tdual}[1]{d({#1})}
\newcommand{\tbox}[1]{\square\,#1}
\newcommand{\tdmu}[1]{#1^{\shortdownarrow {\mu}}}
\newcommand{\tmon}[1]{{\color{red}m(#1)}}
\newcommand{\tforce}[1]{#1^{\shortdownarrow \achange }}
\newcommand{\tlift}[2]{(#1,#2)^{\uparrow}}
\newcommand{\tpull}[1]{#1^{\nearrow}}
\newcommand{\tpushd}[1]{(#1)^{\downarrow\square}}

% Terms
\newcommand{\vbase}{r}
\newcommand{\vtrue}{\mbox{tt}}
\newcommand{\vfalse}{\mbox{ff}}

\newcommand{\la}{\langle} 
\newcommand{\ra}{\rangle}
\newcommand{\eapp}{\;} 
\newcommand{\eleft}{\pi_1}
\newcommand{\eright}{\pi_2} 
\newcommand{\econst}{\kw{n}}
\newcommand{\etrue}{\mbox{true}}
 \newcommand{\efalse}{\mbox{false}}
\newcommand{\eif}{\mbox{if\;}} 
\newcommand{\ethen}{\mbox{\;then\;}}
\newcommand{\eelse}{\mbox{\;else\;}} 
\newcommand{\einl}{\mbox{inl\;}}
\newcommand{\einr}{\mbox{inr\;}} 
\newcommand{\elet}{\mbox{let\;}}
\newcommand{\clet}{\mbox{clet}\;}
\newcommand{\ecimp}{\mbox{.}_c\;}
\newcommand{\eelimU}{\mbox{elim}_U\;} 
\newcommand{\ein}{\mbox{\;in\;}}
\newcommand{\ecase}{\mbox{\;case\;}} 
\newcommand{\eof}{\mbox{\;of\;}}
\newcommand{\eas}{\mbox{\;as\;}} 
\newcommand{\ecelim}{\mbox{celim\;}}
\newcommand{\enil}{\mbox{nil}} 
\newcommand{\epack}{\mbox{pack\;}}
\newcommand{\eunpack}{\mbox{unpack\;}}
\newcommand{\efix}{\mbox{fix\;}} 
\newcommand{\efixNC}{\mbox{fix$_{NC}$\;}} 
\newcommand{\eLam}{ \Lambda}
\newcommand{\elam}{ \lambda} 
\newcommand{\eApp}{ [\,]\,}
\newcommand{\eleaf}{\mbox{leaf}} 
\newcommand{\ewith}{\;\mbox{with}\;} 
\newcommand{\enode}{\mbox{node}}
\newcommand{\econs}{\mbox{cons}} 
\newcommand{\econsC}{\mbox{cons$_C$}} 
\newcommand{\econsNC}{\mbox{cons$_{NC}$}} 
\newcommand{\eunit}{()}
\newcommand{\eswitch}{\kw{switch}\;}
\newcommand{\enoch}{\kw{NC}\;}
\newcommand{\eder}{\kw{der}\;}
\newcommand{\esplit}{\kw{split}\;}
\newcommand{\ecoerce}[2]{\kw{coerce}_{#1,#2}\;}
\newcommand{\econtra}{\kw{contra}\;}

\newcommand{\ealloc}[2]{ \mathrel{ \mathsf{alloc}\, {#1} \, {#2} } }
\newcommand{\eallocB}[2]{ \mathrel{ \mathsf{alloc_b}\, {#1} \, {#2} } }
\newcommand{\eupdt}[3]{ \mathrel{ \mathsf{update} \ {#1} \ {#2} \ {#3} }  }
\newcommand{\ereadx}[2] { \mathrel{ \mathsf{read} \ {#1} \ {#2} }  }
\newcommand{\eupdtB}[3]{ \mathrel{ \mathsf{update_b} \ {#1} \ {#2} \ {#3} }  }
\newcommand{\ereadxB}[2] { \mathrel{ \mathsf{read_b} \ {#1} \ {#2} }  }
\newcommand{\eret}[1] {\mathrel{ \mathsf{return} \, {#1} }}
\newcommand{\eletx}[3]{  \mathrel{ \mathsf{let_m} \{ {#1} \} = {#2} \ \mathsf{in} \ {#3}  } }



\newcommand{\caseof}[1]{\mbox{case}~#1~\mbox{of}}
\newcommand{\tcaseof}[1]{\mathsf{case}~#1~\mathsf{of}}
\newcommand{\ofnil}[1]{~~\mbox{nil}~\to#1}
\newcommand{\ofzero}[1]{~~\kw{0}~\to#1}
\newcommand{\ofcons}[3]{|~#1::#2~\to~#3}

% Diff Rel
\newcommand{\udiff}{\gtrapprox}
\newcommand{\rdiff}{\ominus}
\newcommand{\rdiffs}{\lesssim}
\newcommand{\ldiff}{\lesssim}

% Evaluation
\newcommand{\red}[1]{\Downarrow^{#1}}

\newcommand{\wmax}{\mbox{\scriptsize max}}
\newcommand{\wmin}{\mbox{\scriptsize min}}
\newcommand{\wdiff}{\mbox{\scriptsize diff}}
\newcommand{\wexec}{\mbox{\scriptsize exec}}
\newcommand{\wdead}{\mbox{\scriptsize dead}}
%Logical relation
\newcommand{\step}{\text{Step index}}
\newcommand{\world}{\text{World}}
\newcommand{\values}{\text{Value}}
\newcommand{\expr}{\text{Expression}}
\newcommand{\ulr}[1]{\llbracket#1\rrbracket_{v}}
\newcommand{\ulrg}[1]{\llbracket#1\rrbracket_{\grt}}
\newcommand{\lr}[1]{\llparenthesis#1\rrparenthesis_{v}}
\newcommand{\lre}[2]{\llparenthesis#1\rrparenthesis_{\varepsilon}^{#2}}
\newcommand{\lrg}[1]{\llparenthesis#1\rrparenthesis_{\grt}}
\newcommand{\ulre}[3]{\llbracket#1\rrbracket_{\varepsilon}^{#2,#3}}
\newcommand{\ulrew}[1]{\llbracket#1\rrbracket_{\varepsilon}^{0,\sinf}}

\newcommand{\relwith}[2]{\{#1~|~#2\}}
\newcommand{\rel}[1]{\{#1\}}
\newcommand{\del}[1]{\mathcal{D}\llbracket#1\rrbracket}
\newcommand{\dd}[1]{\mathcal{D}\llbracket\Delta\rrbracket}
\newcommand{\ugsubst}[1]{\mathcal{G}\llbracket#1\rrbracket}
\newcommand{\gsubst}[1]{\mathcal{G}\llparenthesis#1\rrparenthesis}
\newcommand{\dsubst}[1]{\mathcal{D}\llbracket#1\rrbracket}
\newcommand{\s}{\sigma}
\newcommand{\peq}{\preceq}
\newcommand{\plt}{\prec}
\renewcommand{\d}{\delta}
\newcommand{\g}{\gamma}


% Typing judgments
\newcommand{\jiterm}[2]{\mathrel{\vdash {#1} :: #2}}
\newcommand{\jtype}[4]{\mathrel{\vdash_{#1}^{#2} {#3} : {#4}}}
\newcommand{\jtypeM}[4]{\mathrel{\vdash_{#1}^{#2} {#3} :^c {#4}}}

\newcommand{\jtypes}[3]{\mathrel{\vdash_{#1}^{\mu} {#2} : {#3}}}

\newcommand{\jstype}[3]{\mathrel{\vdash
    {#1} \backsim {#2} : {#3}}}


\newcommand{\jtypediff}[4]{\mathrel{\vdash% _{\wdiff}
    {#2} \ominus {#3} \ldiff #1 : {#4}}}

\newcommand{\jtypediffM}[4]{\mathrel{\vdash
    {#2} \ominus {#3} \ldiff #1 :^c {#4}}}
\newcommand{\jmintypesame}[3]{\mathrel{\vdash
    {#2} \ominus {#2} \ldiff #1 :^c {#3}}}

\newcommand{\jelab}[6]{\mathrel{\vdash
    {#2} \ominus {#3} \rightsquigarrow {#4} \ominus {#5} \ldiff #1 : {#6}}}
\newcommand{\jelabsame}[4]{\mathrel{\vdash
    {#2} \ominus {#3} \rightsquigarrow {#2} \ominus {#3} \ldiff #1 : {#4}}}

\newcommand{\jelabun}[5]{\mathrel{\vdash_{#1}^{#2}
    {#3} \rightsquigarrow {#4} : {#5}}}

\newcommand{\jelabc}[4]{\mathrel{\vdash
    {#2} \ominus {#3} \rightsquigarrow {#2}^* \ominus {#3}^* \ldiff #1 : {#4}}}


\newcommand{\jelabcu}[4]{\mathrel{\vdash_{#1}^{#2}
    {#3} \rightsquigarrow {#3}^* : {#4}}}

\newcommand{\jtypediffsym}[5]{\mathrel{\vdash
    #1 \ldiff {#3} \ominus {#4} \ldiff #2 : {#5}}}
\newcommand{\sty}[2]{\vdash#1 \mathrel{::} #2}


\newcommand{\rname}[1]{\mbox{\small{#1}}}

\newcommand{\vsem}[2]{\llbracket #1 \rrbracket_{V}^{#2}}
\newcommand{\esem}[2]{\llbracket #1 \rrbracket_{E}^{#2}}
\newcommand{\conj}{\mathrel{\wedge}}

\newcommand{\vusem}[1]{\llparenthesis #1 \rrparenthesis_{V}}
\newcommand{\eusem}[1]{\llparenthesis #1 \rrparenthesis_{E}}

\newcommand{\jsubtype}[2]{\sat#1\sqsubseteq#2}
\newcommand{\jasubtype}[2]{\sat^{\mathsf{\grt}}#1\sqsubseteq#2}
\newcommand{\jeqtype}[2]{\sat#1 \equiv#2}
\newcommand{\under}[2]{\sat #1 \trianglelefteq  #2}

\newcommand{\type}{\text{type}}
\newcommand{\rtype}{\text{relational type}}
\newcommand{\Type}{\text{Unary type}}
\newcommand{\Rtype}{\text{Binary type}}



% Cost Constants
\newcommand{\kvar}{c_{var}}
\newcommand{\kconst}{c_{n}}
\newcommand{\kinl}{c_{inl}}
\newcommand{\kinr}{c_{inl}}
\newcommand{\kcase}{c_{case}}
\newcommand{\kfix}{c_{fix}}
\newcommand{\kapp}{c_{app}}
\newcommand{\kLam}{c_{fix}}
\newcommand{\kiApp}{c_{iApp}}
\newcommand{\kpack}{c_{pack}}
\newcommand{\kunpack}{c_{unpack}}
\newcommand{\knil}{c_{nil}}
\newcommand{\kcons}{c_{cons}}
\newcommand{\kcaseL}{c_{caseL}}
\newcommand{\kleaf}{c_{leaf}}
\newcommand{\knode}{c_{node}}
\newcommand{\kcaseT}{c_{caseT}}
\newcommand{\kprod}{c_{prod}}
\newcommand{\kproj}{c_{proj}}
\newcommand{\klet}{c_{let}}


%Constraints
\newcommand{\creal}{\mathbb{R}}
\newcommand{\sat}[1]{\models#1}
\newcommand{\sata}[1]{\models_A#1}
\newcommand{\ceq}[2]{#1\mathrel{\doteq}#2}
\newcommand{\cleq}[2]{#1 \mathop{\leq} #2}
\newcommand{\cleqspec}[2]{#1 \overline{\mathop{\leq}} #2}
\newcommand{\clt}[2]{#1 \mathop{<} #2}
\newcommand{\cgt}[2]{#1 \mathop{>} #2}
\newcommand{\ceqz}[1]{#1 \mathrel{\doteq} 0}
\newcommand{\cneg}[1]{\mathop{\neg}#1}
\newcommand{\cand}[2]{#1 \wedge #2}
\newcommand{\cexists}[3]{\exists#1::#2.#3}
\newcommand{\cexistsK}[3]{\exists#1:#2.#3}
\newcommand{\cexistsS}[2]{\exists#1.#2}
\newcommand{\cexistsC}[2]{\exists#1::\scost.#2}
\newcommand{\cexistsall}[2]{\exists(#1).#2}
\newcommand{\cforall}[3]{\forall#1::#2.#3}
\newcommand{\cforallS}[3]{\forall#1:#2.#3}
\newcommand{\cimpl}[2]{#1\rightarrow#2}
\newcommand{\cor}[2]{#1 \vee #2}
\newcommand{\ctrue}{\top}
\newcommand{\cfalse}{\bottom}
\newcommand{\blank}[2][100]{\hfil\penalty#1\hfilneg }
\newcommand{\ccond}[3]{\im{#1}\mathrel{\mbox{?}}\im{#2}\mathrel{\colon}\im{#3}}


\newcommand{\wfty}[1]{\vdash   #1~\kw{wf}}
\newcommand{\awfty}[1]{\vdash^{\mathsf{\grt}}   #1~\kw{wf}}
\newcommand{\wfcs}[1]{\vdash   #1~\kw{wf}}
\newcommand{\wfctx}[1]{\vdash   #1~\kw{wf}}
\newcommand{\awfctx}[1]{\vdash^{\mathsf{\grt}}   #1~\kw{wf}}



\newcommand{\dc}{ downward closure (\lemref{lem:down-closure}) }
\newcommand{\ctx}{\Delta; \Phi_a; \Gamma}
\newcommand{\nctx}{\Delta; \Phi_a; \tbox{\Gamma}}
\newcommand{\primctx}{\Upsilon}
\newcommand{\octx}{\Delta; \Phi_a; \Omega}
\newcommand{\rctx}[1]{\Delta; \Phi_a; \trm{\Gamma}{#1}}

\newcommand{\shade}[1]{\colorbox{lightgray}{#1}}
\newcommand{\fv}[1]{\text{FV}(#1)}
\newcommand{\fcv}[1]{\text{dom}(#1)}
\newcommand{\fiv}[1]{\text{FIV}(#1)}
\newcommand{\fdv}[1]{\text{dom}(#1)}


\newcommand{\assC}[2]{\text{Assume that $\sat \s \Phi$ and there exists $\Gamma'$  s.t. $\fv{#2} \subseteq \fcv{\Gamma'} $ and $\Gamma' \subseteq \Gamma$ and $(m, \d) \in \ugsubst{\trm{\sigma \Gamma'}{#1}}$}}
\newcommand{\assCU}[1]{\text{Assume that $\sat \s \Phi$ and there exists $\Omega'$  s.t. $\fv{#1} \subseteq \fcv{\Omega'} $ and $\Omega' \subseteq \Omega$ and $(m, \d) \in \ugsubst{\s \Omega'}$}}
\newcommand{\IHassun}[1]{\text{$\fv{#1} \subseteq \fcv{\Omega'} $ and $\Omega' \subseteq \Omega$ and $(m, \d) \in \ugsubst{\s \Omega'}$}}


\newcommand{\IHassU}[2]{\text{$\fv{#2} \subseteq \fcv{\trm{\Gamma'}{#1}} $ and $\trm{\Gamma'}{#1} \subseteq \trm{\Gamma}{#1}$ and $(m, \d) \in \ugsubst{\trm{\sigma \Gamma'}{#1}}$}}


\newcommand{\IHass}[2]{\text{$\fv{#2} \subseteq \fcv{\Gamma'} $ and $\Gamma' \subseteq \Gamma$ and $(m, \d) \in \ugsubst{\trm{\sigma \Gamma'}{#1}}$}}
% Environment
\newcommand{\memory}{\Gamma}%\Delta \ | \ \Phi \ | \ \Gamma  \ | \
                            %\Sigma}
\newcommand{\senv}{\Delta}
\newcommand{\lenv}{\Sigma}
\newcommand{\uenv}{\Omega}
\newcommand{\renv}{\Gamma} 
\newcommand{\cenv}{\Phi} 
\newcommand{\sep}{ \ | \ }
\newcommand{\monad}[4]{\mathrel{ M( \overset{ \mathrel{\mathrm{exec}{#4 }}}{#2}  })}
\newcommand{\monadR}[4]{\mathrel{ \overset{\mathrm{diff}(#4)}{\{ {#1} \} \ {#2} \ \{ {#3} \} }}}
\newcommand{\depProd}[4]{ \mathrel{ \Pi {#1} \stackrel{\mathrm{exec} {#4}}{:}{#2} . \ {#3}}}
\newcommand{\depProdr}[4]{ \mathrel{ \Pi {#1} \stackrel{\mathrm{diff} {#4}}{:}{#2} . \ {#3}}}
\newcommand{\uarrow}[3]{ \mathrel{ \stackrel{\mathrm{exec} {#3}}{{#1} \longrightarrow#2}}}
\newcommand{\uforall}[4]{ \mathrel{ \stackrel{\mathrm{exec} {#4}}{\forall {#1} :#2 . \ #3}}}
\newcommand{\uexist}[3]{\mathrel{{\exists {#1} :: {#2} . \ {#3}}}}
\newcommand{\rarrow}[3]{ \mathrel{ \stackrel{\mathrm{diff}(#3)}{{#1} \longrightarrow {#2}}}}
\newcommand{\rarrowt}[3]{ \mathrel{ {#1} \stackrel{\mathrm{} {#3}}{\longrightarrow} {#2}}}
\newcommand{\rforall}[4]{ \mathrel{ \stackrel{\mathrm{diff}(#4)}{\forall {#1}{::}{#2} . \ {#3}}}}
\newcommand{\rexists}[3]{ \mathrel{ {\exists {#1} {::}{#2} . \ {#3}}}}
\newcommand{\rforallt}[4]{ \mathrel{ \forall {#1} \stackrel{\mathrm{} {#4}}{:}{#2} . \ {#3}}}
\newcommand{\arr}[3]{ \mathrel{ \mathsf{Array}_{#1}[{#2}] \ {#3}} }
\newcommand{\arrR}[3]{ \mathrel{ \mathsf{Array}_{#1}[{#2}] \ {#3}} }
\newcommand{\lst}[2]{ \mathrel{ \mathsf{list}[{#1}] \ {#2}} }
\newcommand{\lstR}[3]{ \mathrel{ \mathsf{list}^{#1}[{#2}] \ {#3}} }
\newcommand{\abs}[2]{\mathrel { \lambda {#1} . {#2} } }
\newcommand{\app}[2]{\mathrel{ {#1} \, {#2} }}
\newcommand{\ret}[1] {\mathrel{ \mathsf{return} \, {#1} }}

\newcommand{\letx}[3]{  \mathrel{ \mathsf{let}\   {#1} = {#2} \ \mathsf{in} \ {#3}  } }
\newcommand{\packx}[1]{  \mathrel{ \mathsf{pack} \, {#1}} }
\newcommand{\unpackx}[3]{  \mathrel{ \mathsf{unpack} \,  {#1} \, \mathsf{as} \, {#2} \, \mathsf{in} \ {#3}  } }
\newcommand{\alloc}[2]{ \mathrel{ \mathsf{alloc}\, {#1} \, {#2} } }
\newcommand{\updt}[3]{ \mathrel{ \mathsf{update} \ {#1} \ {#2} \ {#3} }  }
\newcommand{\readx}[2] { \mathrel{ \mathsf{read} \ {#1} \ {#2} }  }
\newcommand{\tTt}[3]{\mathrel{  {#1} \xrightarrow \ {#2} } }
\newcommand{\force}[1]{\mathrel{\mathsf{force} \ \ {#1}}}
\newcommand{\tfix}{\mathsf{Fix}}
\newcommand{\fix}[1]{\mathsf{fix} \, f(x). {#1}}

%Relational
\newcommand{\monadx}[3]{\mathrel{ \{ {#1} \} \ {#2} \ \{ {#3} \} }}
\newcommand{\monadu}[4]{\mathrel{ \overset{ \mathrel{\mathrm{exec}{#4 }}}{\{ {#1} \} \ #2 \ \{ {#3} \}} }}
\newcommand{\cmp}[4] {\mathrel{   \vdash  {#1} \ominus {#2} \ldiff {#4}  : {#3}  }}
\newcommand{\pair}[1]{\mathrel{ {#1}_{1}{#1}_{2}}}
\newcommand{\imp}[2]{\mathrel{  {#1} \Rightarrow {#2} }}
\newcommand{\eval}[3]{\mathrel{ {#1} \Downarrow^{#3} {#2}   }}
\newcommand{\evalf}[3]{\mathrel{ {#1} \Downarrow^{#3}_{f} {#2}   }}
\newcommand{\evalp}[3]{\mathrel{ {#1} \Downarrow^{#3}_{p} {#2}   }}
\newcommand{\heap}[1]{ ;  {#1}}
\newcommand {\spc} { \  \ }
\newcommand{\monadL}[3]{\mathrel{ \{ {#1} \}  \\  \ {#2} \ \\  \{ {#3} \} }}

\newcommand{\wfa}[1]{\mathrel{\vdash {#1} \quad wf}}
\newcommand{\wf}[1]{\mathrel{\vdash {#1} \quad wf}}
\newcommand{\subtypeA}[2]{\mathrel{ \models {#1} \sqsubseteq {#2} } }
\newcommand{\subtype}[2]{\mathrel{   \models {#1} \sqsubseteq {#2} }   }
\newcommand{\subcost}[3]{\mathrel{   \models {#1} {#3} {#2} }   }

\newcommand{\emptyhp}{\mathsf{empty}}
\newcommand{\llb}[1]{ \llbracket {#1} \rrbracket }
\newcommand{\llu}[2]{ \llb{#1}_{#2}}
\newcommand{\llp}[2]{ \llparenthesis {#1} \rrparenthesis_{#2} }
\newcommand{\llbe}[1]{ \llbracket {#1} \rrbracket^{E} }
\newcommand{\llpe}[2]{ \llparenthesis {#1} \rrparenthesis_{#2}^{E} }

\newcommand{\mg}[1]{\textcolor[rgb]{.90,0.00,0.00}{[MG: #1]}}
\newcommand{\dg}[1]{\textcolor[rgb]{0.00,0.5,0.5}{[DG: #1]}}
\newcommand{\wq}[1]{\textcolor[rgb]{.50,0.0,0.7}{[WQ: #1]}}

% Helpful shortcuts

\newcommand{\freshSize}[1]{#1\in\text{fresh}(\ssize)}
\newcommand{\freshCost}[1]{#1\in\text{fresh}(\scost)}
\newcommand{\freshVar}[1]{#1 \in \text{fresh}(S)}

\newcommand{\m}{M} 
%Bi-directional Typing Judgement
\newcommand{\chdiff}[5]{\vdash{#1}\rdiff{#2}~{\downarrow}~#3,#4 \Rightarrow
{\color{red}#5}}

\newcommand{\chsdiff}[3]{\vdash{#1}\backsim{#2}~{\downarrow}~#3}


\newcommand{\chdiffNC}[5]{\vdash^{\color{blue}NC}{#1}\rdiff{#2}~{\downarrow}~#3,#4 \Rightarrow
{{\color{red}#5}}}

\newcommand{\infdiff}[6]{\vdash{#1}\rdiff{#2}~{\uparrow}~{\color{red}{#3}}\Rightarrow[{\color{red}#4}],{\color{red}#5},{\color{red}#6}}

\newcommand{\infsdiff}[3]{\vdash{#1}\backsim{#2}~{\uparrow}~{\color{red}{#3}}}

\newcommand{\infdiffsimple}[5]{\vdash{#1}\rdiff{#2}~{\uparrow}~{\color{red}{#3}}\Rightarrow{\color{red}#4},{\color{red}#5}}

\newcommand{\chmax}[4]{\vdash^{\wmax}{#1}~{\downarrow}~#2, #3  \Rightarrow
{{\color{red}#4}}}
\newcommand{\chmin}[4]{\vdash^{\wmin}{#1}~{\downarrow}~#2, #3  \Rightarrow
{{\color{red}#4}}}

\newcommand{\chexec}[5]{\vdash{#1}~{\downarrow}~#2, #3, #4  \Rightarrow
{{\color{red}#5}}}

\newcommand{\infmax}[5]{\vdash^{\wmax}{#1}~{\uparrow}~{\color{red}{#2}}\Rightarrow[{\color{red}#3}], {\color{red}#4},{\color{red}#5}}
\newcommand{\infmin}[5]{\vdash^{\wmin}{#1}~{\uparrow}~{\color{red}{#2}}\Rightarrow[{\color{red}#3}], {\color{red}#4},{\color{red}#5}}

\newcommand{\infexec}[6]{\vdash{#1}~{\uparrow}~{\color{red}{#2}}\Rightarrow[{\color{red}#3}], {\color{red}#4},{\color{red}#5},{\color{red}#6}}

\newcommand{\infexecsimple}[5]{\vdash{#1}~{\uparrow}~{\color{red}{#2}}\Rightarrow{\color{red}#3}, {\color{red}#4},{\color{red}#5}}

\newcommand{\emptypsi}{.}

%Existential elimination
\newcommand{\elimExt}[3]{#1 \vdash \kw{elimExt}(#2)~\downarrow #3}
\newcommand{\solveVar}[6]{#1 \vdash \kw{solve}(#2;#3) \downarrow (#4;#5;#6)}



%Shortcuts
\newcommand{\al}{\alpha}
\newcommand{\algwf}[1]{\vdash  #1~\kw{wf}}
\newcommand{\algwfa}[1]{\vdash^{A} #1~\kw{wf}}
\newcommand{\jalgeqtype}[3]{\sat#1\equiv#2\Rightarrow {\color{red}#3}}
\newcommand{\jalgasubtype}[3]{\sat^{\mathsf{\grt}}#1\sqsubseteq#2\Rightarrow {\color{red}#3}}
\newcommand{\jalgsubtype}[3]{\sat#1\sqsubseteq#2\Rightarrow {\color{red}#3}}
\newcommand{\jalgssubtype}[2]{\sat#1\leq#2}


\newcommand{\fvars}[1]{\text{FV}(#1)}
\newcommand{\fivars}[1]{\text{FIV}(#1)}
\newcommand{\filtercost}[1]{\text{filterCost}(#1)}
\newcommand{\uctx}{\Delta; \psi_a; \Phi_a; \Omega}
\newcommand{\bctx}{\Delta; \psi_a; \Phi_a; \Gamma}


\newcommand{\suba}[1]{{#1}[\theta_a]}
\newcommand{\subaex}[2]{{#1}[\theta_a, #2]}
\newcommand{\subt}[1]{{#1}[\theta]}
\newcommand{\subta}[1]{{#1}[\theta\,\theta_a]}
\newcommand{\subsat}[3]{#1~ \rhd~ #2 : #3}

\newcommand{\erty}[1]{|#1|}
\newcommand{\eanno}[4]{(#1:#2,#3,#4)}
\newcommand{\eannobi}[3]{(#1:#2,#3)}
\newcommand{\e}{\overline{e}}
\newcommand{\trans}{\rightsquigarrow}
\newcommand{\tboxp}[1]{\square(#1)}
\newcommand{\tlr}[1]{\tlift{\trm{#1}{i}}}
\newcommand*\bang{!}



%%% Local Variables:
%%% mode: latex
%%% TeX-master: "main"
%%% End:

\newcommand{\nform}{\mathsf{F}}
\newcommand{\mechanism}{\mathsf{M}}
\newcommand{\depth}{\mathsf{depth}}
\newcommand{\query}{\text{Q}}

% \newcommand{\caseof}[2]{\mathsf{case} \ {#1} \ \mathsf{of}\ \{ {#2}\}}
\newtheorem{lemma}{Lemma}
\newtheorem{theorem}{Theorem}[section]
\newtheorem{corollary}{Corollary}[theorem]

\usepackage{tikz}
\usetikzlibrary{shapes,arrows}
\newcommand{\THESYSTEM}{\textsf{AdaptFun}}

% Define block styles
\tikzstyle{decision} = [diamond, draw, fill=blue!20, 
    text width=4.5em, text badly centered, node distance=3cm, inner sep=0pt]
\tikzstyle{block} = [rectangle, draw, fill=blue!20, 
    text width=5em, text centered, rounded corners, minimum height=4em]
\tikzstyle{line} = [draw, -latex']
\tikzstyle{cloud} = [draw, ellipse,fill=red!20, node distance=3cm,
    minimum height=2em]

\begin{document}
\title{Program Analysis for Adaptivity Analysis}

\author{}

\date{}

\maketitle


%
\section{Labeled SSA Language}
%
%
\subsection{SSA form Language}
\[
\begin{array}{llll}
 \mbox{Arithmetic Operators} 
& \oplus_a & ::= & + ~|~ - ~|~ \times 
%
~|~ \div \\  
\mbox{Boolean Operators} 
& \oplus_b & ::= & \lor ~|~ \land
\\
  %
\mbox{Relational Operators} 
& \sim & ::= & < ~|~ \leq ~|~ == 
\\  
%
\mbox{Label} 
& l & := & \mathbb{N} 
\\ 
%
\mbox{SSA Arithmetic Expression} 
& \saexpr & ::= & 
n ~|~ \ssa{x} ~|~ \saexpr \oplus_a \saexpr  
\\
%
\mbox{SSA Boolean Expression} & \sbexpr & ::= & 
	%
	\etrue ~|~ \efalse  ~|~ \neg \sbexpr
	 ~|~ \sbexpr \oplus_b \sbexpr
	%
	~|~ \saexpr \sim \saexpr 
	\\
%
\mbox{SSA Query Expression} 
& \ssa{\qexpr} & ::= 
& { \qval ~|~ \saexpr ~|~ \qexpr \oplus_a \qexpr} 
\\
%
\mbox{Query Value} & \qval & ::= 
& {n ~|~ \chi[n] ~|~ \chi[n] \oplus_a  \chi[n] ~|~ n \oplus_a  \chi[n]
~|~ \chi[n] \oplus_a  n}
\\
%
\mbox{Value} 
& v & ::= & { n \sep \etrue \sep \efalse ~|~ [] ~|~ [v, \dots, v]}  
\\
%
\mbox{SSA Expression} & \sexpr & ::= & v ~|~ \saexpr \sep \sbexpr ~|~ [\expr, \dots, \expr]
\\	
%
\mbox{Labeled SSA Command} 
& \ssa{c} & ::= &   [\assign {\ssa{x}}{ \ssa{\expr}}]^{l} ~|~  [\assign {\ssa{x} } {\ssa{\query(\qexpr)}}]^{l}
%
~|~  {{\eifvar(\bar{\ssa{x}}, \bar{\ssa{x}}')}} 
%
\\ 
&&& 
{\ewhile ~ [ \sbexpr ]^{l} , n,
~ 
[\bar{\ssa{x}}, \bar{\ssa{x_1}}, \bar{\ssa{x_2}}] 
~ \edo ~  \ssa{c} }
\\
&&&
~|~ \ssa{c};\ssa{c}  
~|~ [\eif(\sbexpr, [\bar{\ssa{x}}, \bar{\ssa{x_1}}, \bar{\ssa{x_2}}] , \ssa{c}, \ssa{c})]^l 
~|~ [\eskip]^{l} 
\\
%
\mbox{Assignment Event} 
& \asnevent & ::= & (\ssa{x}, l, n, v) | (\ssa{x}, l, n, \qval, v)
\\
%
\mbox{Testing Event} 
& \testevent & ::= & (\sbexpr, l, n, v)
\\
%
\mbox{Event} 
& \event & ::= & \asnevent | \testevent
\\
%
\mbox{Trace} & \vtrace
& ::= & | \vtrace \cdot \event
\\
%
\mbox{Event Signature} & \sig
& ::= & (x, l, n) | (x, l, n, \query) | (b, l, n)
\\
%
\mbox{Environment} 
& \env & ::= & \vtrace \to \mathcal{SVAR} \to v \cup \{\bot\}
\\
%
\mbox{Query Environment} 
& \qenv & ::= & \vtrace \to \mathcal{SVAR} \to \qval \cup \{\bot\}
\end{array}
\]
We use following notations to represent the set of corresponding terms:
\[
\begin{array}{lll}
\mathcal{SVAR} & : & \mbox{Set of Variables}  
\\ 
%
\mathcal{VAL} & : & \mbox{Set of Values} 
\\ 
%
%
\eventset  & : & \mbox{Set of Events}  
\\
%
%
\eventset^{\asn}  & : & \mbox{Set of Assignment Events}  
\\
%
\eventset^{\test}  & : & \mbox{Set of Testing Events}  
\\
%
%
\eventset  & : & \mbox{Set of Events}  
\\
%
\dbdom  & : & \mbox{{Set of Databases}} 
\\
%
\qdom = {[-1,1]} & : & \mbox{{Domain of Query Results}}
\end{array}
\]
%
%
%
Environment $ \env : \vtrace \to \mathcal{SVAR} \to \mathcal{VAL} \cup \{\bot\}$
\[
\begin{array}{ll}
\env(\vtrace \cdot (x, l, n, v)) x \triangleq v
&
\env(\vtrace \cdot (x, l, n, \qval, v)) x \triangleq v
\\
\env(\vtrace \cdot (y, l, n, v)) x \triangleq \env(\vtrace) x
&
\env() x \triangleq \bot
\end{array}
\]
%
Counter $\vcounter : \trace \to \mathbb{N} \to \mathbb{N}$
\[
\begin{array}{ll}
\vcounter(\vtrace \cdot (x, l, n, v)) l \triangleq n
&
\vcounter() l \triangleq 0
\end{array}
\]
%
\subsection{Trace-based Operational Semantics for SSA Language}
%
%
%
{
\begin{mathpar}
\boxed{ \config{\trace,\aexpr} \aarrow v \, : \, Trace  \times AExpr \Rightarrow Value }
\\
\boxed{ \config{\trace, \bexpr} \barrow v \, : \, Trace \times BExpr \Rightarrow Value }
\end{mathpar}
}
%

\begin{mathpar}
  \boxed{ \config{\trace, \qexpr} \qarrow \qval \, : \, Trace  \times QExpr \qarrow QValue }
  \\
  \inferrule{ 
    \config{\trace, \qexpr_1} \qarrow \qval_1
    \and
    \config{m, \qexpr_2} \qarrow \qval_2
  }{
   \config{\trace,  \qexpr_1 \oplus_a \qexpr_2} 
   \qarrow n'
  }
  \and
  \inferrule{ 
    \config{m, \aexpr} \aarrow v
  }{
   \config{m,  \chi[\aexpr]} \qarrow \chi[v]
  }
  \end{mathpar}
%
The trace based operational semantics are defined in Figure \ref{fig:os_ssa}.
%
\begin{figure}
\jl{
  \begin{mathpar}
  \boxed{
  Command \times Trace
  \xrightarrow{}
  Command \times Trace
  }
  \\
  \boxed{\config{\ssa{c, \vtrace, \vcounter}}
  \xrightarrow{} 
  \config{\ssa{c',  \vtrace', \vcounter'}}
  }
  \\
  %
  \inferrule
  {
  \event = (\ssa{x}, l, \vcounter(\vtrace) l + 1, v)
  }
  {
  \config{[\assign{\ssa{x}}{\saexpr}]^{l},  \vtrace } 
  \xrightarrow{} 
  \config{[\eskip]^{l}, \vtrace \cdot \event, \vcounter'}
  }
  ~\textbf{assn}
  %
  \and
  %
  \inferrule
  {
   \vtrace, b \barrow \etrue
   \and 
   \event = (b, l, \vcounter(\vtrace) l + 1, \etrue)
  }
  {
  \config{\ssa{\ewhile ~ [b]^{l}, n, [\bar{{x}}', \bar{{x_1}}, \bar{{x_2}}] ~ \edo ~ c, \vtrace, \vcounter}}
  \\
  \xrightarrow{} 
  \config{\ssa{
  c[\bar{x_i}/\bar{x'}]; \ewhile ~ [b]^{l}, (n + 1), [\bar{{x}}', \bar{{x_1}}, \bar{{x_2}}]  ~ \edo ~ c,  \eskip),
  \vtrace\cdot \event}}
  }
  ~\textbf{ssa-while-t}
  %
  %
    \and
  %
  \inferrule
  {
   \vtrace, b \barrow \efalse
   \and 
   \event = (b, l, \vcounter(\vtrace) l + 1, \efalse)
  }
  {
  \config{\ssa{\ewhile ~ [b]^{l}, n, [\bar{{x}}', \bar{{x_1}}, \bar{{x_2}}] ~ \edo ~ c, \vtrace}}
  \\
  \xrightarrow{} 
  \config{\ssa{
  [\eskip]^l ,
  \vtrace \cdot \event}}
  }
  ~\textbf{ssa-while-f}
  %
  \and
  %
  {
  \inferrule
  {
   \vtrace, \qexpr \qarrow \qval
   \and 
  \query(\qval) = v
  \and 
  \event = (\ssa{x}, l, \vcounter(\vtrace) l + 1, \qval, v)
  }
  {
  \config{\ssa{[\assign{x}{\query(\qexpr)}]^l, \vtrace}}
  \xrightarrow{} 
  \config{\ssa{\eskip,  \vtrace \cdot \event} }
  }
  ~\textbf{ssa-query}
  }
  %
  \and
  %
  %
  \inferrule
  {
  \config{\ssa{c_1, \vtrace}}
  \xrightarrow{}
  \config{\ssa{c_1',  \vtrace'}}
  }
  {
  \config{\ssa{c_1; c_2, \vtrace}} 
  \xrightarrow{} 
  \config{\ssa{c_1'; c_2, \vtrace'}}
  }
  ~\textbf{ssa-seq1}
  %
  ~~~~~~~
  %
  \inferrule
  {
    \config{\ssa{c_2, \vtrace}}
  \xrightarrow{}
  \config{\ssa{c_2',  \vtrace'}}
  }
  {
  \config{\ssa{[\eskip]^{l} ; c_2, \vtrace}} \xrightarrow{} \config{\ssa{ c_2', \vtrace'}}
  }
  ~\textbf{ssa-seq2}
  %
  \and
  %
  %
  \inferrule
  {
     \vtrace, b \barrow \etrue
   \and 
   \event = (b, l, \vcounter(\vtrace) l + 1, \etrue)
  }
  {
  \config{\ssa{
  \eif([\etrue]^{l}, [\bar{{x}}, \bar{{x_1}}, \bar{{x_2}}],
  [\bar{\ssa{y}}, \bar{\ssa{y_1}}, \bar{\ssa{y_2}}],
  [\bar{\ssa{z}},\bar{\ssa{z_1}}, \bar{\ssa{z_2}}], c_1, c_2), 
  \vtrace}}
  \\
  \xrightarrow{} 
  \config{\ssa{ 
  c_1; 
  \eifvar(\bar{\ssa{x}},\bar{\ssa{x_2}}); 
  \eifvar(\bar{\ssa{y}},\bar{\ssa{y_1}});
  \eifvar(\bar{\ssa{z}},\bar{\ssa{z_2}}), 
  \vtrace \cdot \event}}
  }
  ~\textbf{ssa-if-t}
  %
  \and
  %
  \inferrule
  {
   \vtrace, b \barrow \efalse
   \and 
   \event = (b, l, \vcounter(\vtrace) l + 1, \efalse)
  }
  {
  \config{\ssa{
  \eif([\efalse]^{l},[\bar{{x}}, \bar{{x_1}}, \bar{{x_2}}],
  [\bar{\ssa{y}}, \bar{\ssa{y_1}}, \bar{\ssa{y_2}}],
  [\bar{\ssa{z}},\bar{\ssa{z_1}}, \bar{\ssa{z_2}}], c_1, c_2), \vtrace}}
  \\
  \xrightarrow{} 
  \config{\ssa{
  c_2;
  \eifvar(\bar{\ssa{x}},\bar{\ssa{x_2}}); 
  \eifvar(\bar{\ssa{y}},\bar{\ssa{y_1}});
  \eifvar(\bar{\ssa{z}},\bar{\ssa{z_2}}), 
  \vtrace \cdot \event}}
  }
  ~\textbf{ssa-if-f}
  %
  \and
  %
  \inferrule
  {
  }
  {
   \config{\eifvar(\ssa{\bar{x}, \bar{x}'}), \ssa{\vtrace}} 
   \to 
   \config{\ssa{\eskip, \vtrace}}
  }~\textbf{ssa-\eifvar}
  % %
  %
  %
  \end{mathpar}
}
  % \end{subfigure}
      \caption{Trace-based Operational Semantics for SSA Language.}
      \label{fig:os_ssa}
  \end{figure}
  %
%
%
\subsection{Event and Trace}
%
%
Environment $ \env : \vtrace \to \mathcal{SVAR} \to \mathcal{VAL} \cup \{\bot\}$
\[
\begin{array}{ll}
\env(\vtrace \cdot (x, l, n, v)) x \triangleq v
&
\env(\vtrace \cdot (x, l, n, \qval, v)) x \triangleq v
\\
\env(\vtrace \cdot (y, l, n, v)) x \triangleq \env(\vtrace) x
&
\env() x \triangleq \bot
\end{array}
\]

%
Query Environment $\qenv: \vtrace \to \mathcal{SVAR} \to \qval \cup \{\bot\}$
\[
\begin{array}{ll}
\qenv(\vtrace \cdot (x, l, n, v)) x \triangleq \qenv(\vtrace) x
&
\qenv(\vtrace \cdot (x, l, n, \qval, v)) x \triangleq \qval
\\
\qenv(\vtrace \cdot (y, l, n, v)) x \triangleq \qenv(\vtrace) x
&
\qenv() x \triangleq \bot
\end{array}
\]
%
Event Value : $\pi_v : \eventset \to \mathcal{VAL}$
\[
\begin{array}{ll}
\pi_v((x, l, n, v)) \triangleq v
&
\pi_v (x, l, n, \qval, v) \triangleq v
\\
\pi_v (b, l, n, v)  \triangleq v
&
\end{array}
\]
%
Event Query Value : $\pi_{q} : \eventset \to \mathcal{VAL}$
\[
\begin{array}{ll}
\pi_{q} (x, l, n, v) \triangleq v
&
\pi_{q} (x, l, n, \qval, v) \triangleq \qval
\\
\pi_{q} (b, l, n, v)  \triangleq v
&
\end{array}
\]%
% 
Event Signature : $\pi_{\sig} : \eventset \to \mathcal{VAL}$
\[
\begin{array}{ll}
\pi_{\sig} (x, l, n, v) \triangleq (x, l, n)
&
\pi_{\sig} (x, l, n, \qval, v) \triangleq (x, l, n, \query)
\\
\pi_{\sig} (b, l, n, v)  \triangleq (b, l, n)
&
\end{array}
\]
%
Equivalence of 2 events : $\event_1 \eventeq \event_2$
\[
\event_1 \sigeq \event_2 \triangleq
\left\{
\begin{array}{ll}
\etrue & \event_1 = (x, l, n, v) \land \event_2 = (x, l, n, v) \\
\etrue & \event_1 = (x, l, n, \qval_1, v) \land \event_2 = (x, l, n, \qval', v)  
\land \qval_1 =_{q} \qval_2\\
\etrue & \event_1 = (b, l, n, v) \land \event_2 = (b, l, n, v) \\
\efalse & o.w.
\end{array}
\right.
\]

Signature Equivalence of 2 events : $\event_1 \sigeq \event_2$
\[
\event_1 \sigeq \event_2 \triangleq
\left\{
\begin{array}{ll}
\etrue & \event_1 = (x, l, n, v) \land \event_2 = (x, l, n, v') \\
\etrue & \event_1 = (x, l, n, \qval, v) \land \event_2 = (x, l, n, \qval', v') \\
\etrue & \event_1 = (b, l, n, v) \land \event_2 = (b, l, n, v') \\
\efalse & o.w.
\end{array}
\right.
\]
%
Events preserve the order of execution. The order relation is defined in Definition~\ref{def:query_dir}.
%
\begin{defn}[Order of Events].
\label{def:query_dir}
\\
Given 2 annotated queries 
$\event_1 = (x_1, v_1, l_1, n_1), 
\event_2 = (x_2, v_2, l_2, n_2)$
:
%
\[
\event_1 \eventlt \event_2
 \triangleq 
 \left\{
 \begin{array}{ll}
    l_1 < l_2 & n_1 = n_2
    \\
    n_1 < n_2  & o.w.
\end{array}  
\right.
\]
%
$\event_1 \eventgeq \event_2$  is defined vice versa.
\end{defn}
%
%
%
Given the evaluation rules for query expression, we define its equivalence relation in Definition~\ref{def:query_equal}.
%
\begin{defn}[Equivalence of Query].
%
\label{def:query_equal}
 Given a memory $m$ and 2 query expressions $\qexpr_1$, $\qexpr_2$ s.t., $FV(\qexpr_1) \in \dom(m)$ and $FV(\qexpr_2) \in \dom(m)$:
$$
\qexpr_1 =_{q}^{m} \qexpr_2 \triangleq
\left\{
    \begin{array}{ll} 
      \etrue      
      & 
    \exists \qval_1, \qval_2.
    \begin{array}{l} 
      (\config{m,  \qexpr_1} \qarrow \qval_1 \land \config{m,  \qexpr_2 } \qarrow \qval_2) 
      \\
      \land (\forall r \in \qdom. \exists v. ~ s.t., ~ 
            \config{m, \qval_1[r/\chi]} \aarrow v \land \config{m,  \qval_2[r/\chi] } \aarrow v)  
    \end{array}\\
      \efalse         
      & \text{o.w.} 
    \end{array}
    \right.
$$
%
, where $FV(\qexpr)$ is the set of free variables in the query expression $\qexpr$.
$\qexpr_1 \neq_{q}^{m} \qexpr_2$  is defined vice versa.
%
We use $=_{q}$  and $\neq_{q}$ as the shorthands for $=_{q}^{[]}$ and $\neq^{[]}_{q}$.
\end{defn}
%
Then, we have the corresponding equivalence relation between 2 annotated queries defined in Definition~\ref{def:av_equal}:
%
%
%
%
%
Given an annotated query $\event$, $\event$ belongs to a trace $\trace$, i.e., $\event \eventin \trace$ are defined as follows:
    %
\begin{equation}
    \event \eventin \trace  
    \triangleq \left\{
    \begin{array}{ll} 
      \etrue                  & \trace =  (\trace' \cdot \event') \land (\event \sigeq \event') \\
      (\trace' \cdot \event') & \trace =  (\trace' \cdot \event') \land (\event \signeq \event') \\ 
      \efalse                 & o.w.
    \end{array}
    \right.
  \end{equation}
  %
  %
%
\begin{defn}[Equivalence of Program]
%
\label{def:aq_prog}
Given 2 programs $c_1$ and $c_2$:
\[
c_1 =_{c} c_2
 \triangleq 
 \left\{
    \begin{array}{ll} 
      \etrue        
      & c_1 = \eskip \land c_2 = \eskip
      \\ 
      \forall m. \exists v. ~ \config{m, \expr_1} \aarrow^{*} v \land \config{m, \expr_1} \aarrow^{*} v     
      & c_1 = \assign{x}{\expr_1} \land c_2 = \assign{x}{\expr_2} 
      \\ 
      \qexpr_1 =_{q} \qexpr_2       
      & c_1 = \assign{x}{\query(\qexpr_1)} \land c_1 = \assign{x}{\query(\qexpr_2)} 
      \\
      c_1^f =_{c} c_2^f \land c_1^t =_{c} c_2^t
      & c_1 = \eif(b, c_1^t, c_1^f) \land c_2 = \eif(b, c_2^t, c_2^f)
      \\ 
      c_1' =_{c} c_2'         
      & c_1 = \ewhile b \edo c_1' \land c_2 = \ewhile b \edo c_2'
      \\ 
      c_1^h =_{c} c_2^h \land c_1^t =_{c} c_2^t
      & c_1 = c_1^h;c_1^t \land c_2 = c_2^h;c_2^t 
    \end{array}
    \right.
\]
%
$c_1 \neq_{c} c_2$  is defined vice versa.
%
\end{defn}
%
Given 2 programs $c$ and $c'$, $c'$ is a sub-program of$c$, i.e., $c' \in_{c} c$ is defined as:
\begin{equation}
c' \in_{c} c \triangleq \exists c_1, c_2, c''. ~ s.t.,~
c =_{c} c_1; c''; c_2 \land c' =_{c} c''
\end{equation} 
%

\begin{defn}[Labeled Variables ($\lvar : \ssa{c} \to \mathcal{VAR} \times \mathbb{N}$)]
$$
  \lvar({\ssa{c}}) \triangleq
  \left\{
  \begin{array}{ll}
      \{\ssa{x}^l\}                   
      & \ssa{c} = [\ssa{\assign x e}]^{l} 
      \\
      \{\ssa{x}^l\}                   
      & \ssa{c} = [\ssa{\assign x \query(\qexpr)}]^{l} 
      \\
      \lvar{(\ssa{c_1})} \cup \lvar{(\ssa{c_2})}  
      & \ssa{c} = \ssa{c_1};\ssa{c_2}
      \\
      \lvar({\ssa{c}}) \cup \lvar({\ssa{c_2}}) \cup \ssa{\{\bar{x}, \bar{y}, \bar{z}\}} 
      & \ssa{c} =\eif([\sbexpr]^{(l, w)} , \ssa{[\bar{x}, \bar{x_2}, \bar{x_2}], 
      [\bar{y}, \bar{y_2}, \bar{y_3}], 
      [\bar{z}, \bar{z_2}, \bar{z_3}], c_1, c_2}) 
      \\
      \lvar({\ssa{c}'}) \cup \{\ssa{\bar{x}}\}
      & \ssa{c}   = \ewhile ([\sbexpr]^{(l, w)}, [\ssa{\bar{x}, \bar{x_2}, \bar{x_2}}], \ssa{c}')
\end{array}
\right.
$$
\end{defn}
%
\begin{defn}[Query Variables ($\qvar$)].
\\
Given a program $c$, its query variables $\qvar$ is a vector containing all variables newly assigned by a query in the programm, $\qvar \subset \mathcal{VAR}$.
It is defined as follows:
$$
  \qvar_{\ssa{c}} \triangleq
  \left\{
  \begin{array}{ll}
      []                  
      & \ssa{c} = [\ssa{\assign x e}]^{(l, w)} 
      \\
      \left[ \ssa{x} \right]                  
      & \ssa{c} = [\ssa{\assign x \query(\qexpr)}]^{(l, w)} 
      \\
      \lvar_{\ssa{c_1}} ++ \lvar_{\ssa{c_2}}  
      & \ssa{c} = \ssa{c_1};\ssa{c_2}
      \\
      \lvar_{\ssa{c_1}} ++ \lvar_{\ssa{c_2}} ++ \ssa{[\bar{x}, \bar{y}, \bar{z}]} 
      & \ssa{c} =\eif([\sbexpr]^{(l, w)} , \ssa{[\bar{x}, \bar{x_2}, \bar{x_2}], 
      [\bar{y}, \bar{y_2}, \bar{y_3}], 
      [\bar{z}, \bar{z_2}, \bar{z_3}], c_1, c_2}) 
      \\
      \lvar_{\ssa{c}'} ++ [\ssa{\bar{x}}]
      & \ssa{c}   = \ewhile ([\sbexpr]^{(l, w)}, [\ssa{\bar{x}, \bar{x_2}, \bar{x_2}}], \ssa{c}')
\end{array}
\right.
$$
\end{defn}
%
%
%
\subsection{ Trace-based Adaptivity}
%
%
% 
%
%
%
\begin{defn}
[Value Dependency of Events]
\label{def:event_valdep}.
\\
One event $\event_2$ may have a value dependency on an assignment event $\event_1$ or 
in a program $\ssa{c}$
with hidden database $D$, denoted as 
%
$\eventdep^{val}(\event_1, \event_2, c, D)$, where $\event_1 = (\ssa{x}_1, l_1, n_1, v_1) $ or 
$\event_1 = (\ssa{x}_1, \qval_1, v_1, l_1, n_1)$
%
\[
\begin{array}{ll}
\begin{array}{l}
\forall \vtrace_0
\\
\exists 
\vtrace_1, \vtrace_2, \vtrace_2', \ssa{c}_1, \ssa{c}_2.
\\
  \left(
  \begin{array}{l}   
\config{\ssa{c}, \vtrace_0} \rightarrow^{*} 
\config{[\assign{\ssa{x}_1}{\expr_1}]^{l_1} ; \ssa{c}_1, \vtrace_0 \vtrace_1}  \rightarrow^{assn}
\\ 
 \config{c_1, \vtrace_0 \vtrace_1 \cdot \event_1} 
  \qquad \rightarrow^{*} 
  \config{[\assign{\ssa{x}_2}{\expr_2 ~ or ~ \query(\qexpr_2)}]^{l_2} 
  \\
  or
  \eif([\sbexpr]^l_2, \cdots) 
  or \ewhile [\sbexpr]^l_2 \cdots; \ssa{c}_2, 
  \vtrace_0 \vtrace_1 \cdot \event_1\vtrace_2} 
  \\
  \qquad \rightarrow^{assn/query/test} 
  \config{\ssa{c}_3,  \vtrace_0 \vtrace_1 \cdot \event_1 \vtrace_2 \cdot \event_2, } 
  % 
 \\ 
 \bigwedge
 \config{c_1, \vtrace_0 \vtrace_1 \cdot \event_1',} 
  \qquad \rightarrow^{*} 
  \config{[\action]^{l_2} ; \ssa{c}_2, \vtrace_0 \vtrace_1 \cdot \event_1 \vtrace_2'} 
  \\
  \qquad \rightarrow^{assn/query/test} 
  \config{\ssa{c}_2,  \vtrace_0 \vtrace_1 \cdot \event_1 \vtrace_2' \cdot \event_2'} 
\\
\bigwedge
\event_2 \neq_{v} \event_2'
\end{array}
\right)
\end{array} 
&
\end{array}
 \]
%
\end{defn}
%
\begin{defn}
[Testing Dependency of Events]
\label{def:event_testdep}.
\\
One event $\event_2$ may have a testing dependency on a testing event $\event_1 = (\ssa{b}_1, l_1, n_1, v_1)$
in a program $\ssa{c}$, with a hidden database $D$, 
denoted as 
%
$\eventdep^{test}(\event_1, \event_2, c, D)$, is defined as follows: 
%
\[
\begin{array}{l}
\exists \vtrace_0
\vtrace_1, \vtrace_2, \vtrace_2', \ssa{c}_1, \ssa{c}_2.
\\
  \left(
  \begin{array}{l}   
\config{\ssa{c}, \vtrace_0} \rightarrow^{*} 
\config{[\assign{\ssa{x}_1}{\expr_1}]^{l_1} ; \ssa{c}_1, \vtrace_0 \vtrace_1}  \rightarrow^{assn}
\\ 
 \config{c_1, \vtrace_0 \vtrace_1 \cdot \event_1} 
  \qquad \rightarrow^{*} 
  \config{[\action]^{l_2} ; \ssa{c}_2, \vtrace_0 \vtrace_1 \cdot \event_1\vtrace_2} 
  \\
  \qquad \rightarrow^{assn/query/test} 
  \config{\ssa{c}_2,  \vtrace_0 \vtrace_1 \cdot \event_1 \vtrace_2 \cdot \event_2} 
  % 
 \\ 
 \bigwedge
 \config{c_1, \vtrace_0 \vtrace_1 \cdot \event_1'} 
  \qquad \rightarrow^{*} 
  \config{\ssa{c}_2,  \vtrace_0 \vtrace_1 \cdot \event_1 \vtrace_2'} 
\\
\bigwedge
(\event_2) \notin_{\sig} \vtrace_2'
\end{array}
\right)
\end{array} 
 \]
%
\end{defn}
%
%
\begin{defn}[Event May Dependency].
\label{def:event_dep}
\\
One assignment event $\event_2$ may depend on another assignment event $\event_1$
in a program $\ssa{c}$, with a hidden database $D$, denoted as 
%
$\eventdep(\event_1, \event_2, c, D)$, is defined as follows: 
%
\[
\eventdep^{val}(\event_1, \event_2, c, D) 
\lor
\Big(
\exists \event_b \in \eventset^{test}. ~ \eventdep^{val}(\event_1, \event_b, c, D) 
\land \eventdep^{test}(\event_b, \event_2, c, D) 
\Big)
\] 
%
%
\end{defn}
%
\begin{defn}[Variable May Dependency].
\label{def:var_dep}
\\
Given a program $\ssa{c}$, 
one variable $\ssa{x}_2^{l_2} \in \lvar_{\ssa{c}}$ may depend on another variable 
$\ssa{x}_1^{l_1} \in \lvar_{\ssa{c}}$ in $\ssa{c}$ denoted as 
%
$\vardep(\ssa{x}_1^{l_1}, \ssa{x}_2^{l_2}, \ssa{c})$ is defined below.
%
\[
\exists \event_1, \event_2 \in \eventset^{\asn}, D \in \dbdom. ~
\projl{\event_1} = (\ssa{x}_1, l_1)
\land
\projl{\event_2} = (\ssa{x}_2, l_2)
\land 
\eventdep(\event_1, \event_2, c, D)
\] 
%
%
\end{defn}
%
%
\begin{defn}[Execution Based Dependency Graph].
\\
Given a program $\ssa{c}$ with its assigned variables $\lvar_c$ 
the dependency graph $\traceG(\ssa{c}, D) = (\vertxs, \edges, \weights, \qflag)$ is defined as:
%
\[
\begin{array}{rlcl}
  \text{Vertices} &
  \vertxs & := & \left\{ 
  x^l \in \mathcal{VAR} \times \mathbb{N}
  ~ \middle\vert ~
  x^l \in \lvar({\ssa{c}})
  \right\}
  \\
  \text{Directed Edges} &
  \edges & := & 
  \left\{ 
  (x^l, x'^{l'}) \in (\mathcal{VAR} \times \mathbb{N}) \times (\mathcal{VAR} \times \mathbb{N})
  ~ \middle\vert ~
  \vardep(x^l, x'^{l'}, c) \land
  x^l, x'^{l'} \in \lvar({\ssa{c}})
  \right\}
  \\
  \text{Weights} &
  \weights & := & 
  \left\{ 
  (x^l, n) \in \mathcal{VAR} \times \mathbb{N}
  ~ \middle\vert ~ 
  x^l \in \lvar({\ssa{c}}) 
  \forall \vtrace, s.t.,
  \config{\ssa{c}, } \to^{*} \config{\eskip, \vtrace},
  n = \max_{\vtrace}\vcounter(\vtrace)l; 
  \right\}
  \\
  \text{Query Flags} &
  \qflag & := & 
  \left\{(x^l, n)  \in \vertxs \times \{0, 1\} 
  ~ \middle\vert ~
  \left\{
  \begin{array}{ll}
  n = 1 & x^l \in \qvar({\ssa{c}}) \\ 
  n = 0 & o.w.
  \end{array}
  \right\};
  x^l \in \lvar({\ssa{c}})
  \right\}
\end{array}
\]
\end{defn}
%
%
\begin{defn}[Finite Walk ($k$)].
\label{def:finitewalk}
\\
Given a labeled weighted graph $G = (\vertxs, \edges, \weights, \qflag)$, a \emph{finite walk} $k$ in $G$ is a sequence of edges $(e_1 \ldots e_{n - 1})$ 
for which there is a sequence of vertices $(v_1, \ldots, v_{n})$ such that:
\begin{itemize}
    \item $e_i = (v_{i},v_{i + 1})$ for every $1 \leq i < n$.
    \item every vertex $v \in \vertxs$ appears in this vertices sequence $(v_1, \ldots, v_{n})$ of $k$ at most $W(v)$ times.  
\end{itemize}
$(v_1, \ldots, v_{n})$ is the vertex sequence of this walk.
\\
%
Length of this finite walk $k$ is the number of vertices in its vertex sequence, i.e., $\len(k) = n$.
\end{defn}
%
Given a labeled weighted graph $G = (\vertxs, \edges, \weights, \qflag)$, 
we use $\walks(G)$ to denote a set containing all finite walks $k$ in $G$;
and $k_{v_1 \to v_2} \in \walks(G)$where $v_1, v_2 \in \vertxs$ denotes the walk from vertex $v_1$ to $v_2$ .
%
%
\begin{defn}[Length of Finite Walk w.r.t. Query ($\qlen$)].
\label{def:qlen}
\\
Given a labeled weighted graph $G = (\vertxs, \edges, \weights, \qflag)$ and a \emph{finite walk} $k$ in $G$ with its vertex sequence $(v_1, \ldots, v_{n})$, the length of $k$ w.r.t query is defined as:
\[
  \qlen(k) = \len\big(
  v \mid v \in (v_1, \ldots, v_{n}) \land \flag(v) = 1 \big)
\]
, where $\big(v \mid v \in (v_1, \ldots, v_{n}) \land \flag(v) = 1 \big)$ is a subsequence of $k$'s vertex sequence.
\end{defn}
%
Given a program $c$ with a database $D$, we generate its program-based graph 
$\traceG(\ssa{c}, D) = (\vertxs, \edges, \weights, \qflag)$.
%
Then the adaptivity bound based on program analysis for $\ssa{c}$ is the number of query vertices on a finite walk in $\progG(\ssa{c})$. This finite walk satisfies:
%
\begin{itemize}
\item the number of query vertices on this walk is maximum
\item the visiting times of each vertex $v$ on this walk is bound by its weight $\weights(v)$.
\end{itemize}
%
It is formally defined in \ref{def:trace_adapt}.
%
\begin{defn}
[Adaptivity of A Program].
\label{def:trace_adapt}
\\
Given a program $\ssa{c}$ in SSA language, 
its adaptivity is defined for all possible starting SSA memory $\ssa{m}$ and database $D$ as follows:
%
$$
A(c) = \max \big 
\{ \qlen(k) \mid D \in \dbdom , k \in \walks(\traceG(c, D) \big \} 
$$
\end{defn}
%
%
%
%
\todo{
The following lemma describes a property of the trace-based dependency graph.
For any program $c$ with a database $D$ and a starting memory $m$,
the directed edges in its trace-based dependency graph can only be constructed from nodes representing 
smaller annotated queries to annotated queries of greater order.
There doesn't exist backward edges with direction from greater annotated queries to smaller ones.
}
\begin{lem}
\label{lem:edgeforwarding}
[Edges are Forwarding Only].
\\
%
Given a program $c$, a database $D$, a starting memory $m$ and the corresponding trace-based dependency graph $G(c,D) = (\vertxs, \edges)$, 
for any directed edge $(\event', \event) \in \edges$, 
this is not the case that:
%
$$\event' \eventgeq \event$$
%
\end{lem}
%
\begin{proof}
Proof in File: {\tt ``edge\_forward.tex''}.
% \begin{proof}
This lemma is proved by showing there is a contradiction. 

\jl{
Assume there exists an edge  $(\av', \av) \in \edges$ and $\av' \avgeq \av$, where $\av' = ({\qval}',l',w')$ and $\av = ({\qval},l,w)$.
%
According to the Definition~\ref{def:trace-based_graph}, we have:
%
$$
DEP(\av', \av, c, m, D) ~ (1)
$$
%
%
Unfolding the Definition \ref{def:query_dep} in $(1)$,
we know: there exists $t_1, t_3, m_1, m_3, c_2$ s.t.,
%
\[
\config{m, c, [], []} \rightarrow^{*} 
\config{m_1, [\assign{x}{\query({\qval'})}]^{l'} ; c_2,
  t_1, w'} 
\rightarrow^{\textbf{query-v}} 
\config{m_1[v_1/x], c_2,
(t_1 ++ [\av'], w_1} \rightarrow^{*} \config{m_3, \eskip, t_3,w_3} ~ (a)  
\]
%
and
%
\[
 \bigwedge
 \begin{array}{l}   
  % 
  \left( 
  \begin{array}{l}
  \av \avin (t_3 - (t_1 ++ [\av'])) 
  % 
  \\
  \implies 
  \exists v \in \qdom, v \neq v_1, m_3', t_3', w_3'. ~  
  \config{m_1[v/x], {c_2}, t_1 ++ [\av'], w_1} 
  \\ 
  \quad \quad 
  \rightarrow^{*}
  (\config{m_3', \eskip, t_3', w_3'} 
  \land 
  \av \not \avin (t_3'-(t_1 ++ [\av'])))
\end{array} \right ) ~(b)
\\
\left( 
  \begin{array}{l}
  \av \not\avin (t_3 - (t_1 ++ [\av']))
    % 
    \\
    \implies 
  \exists v \in \qdom, v \neq v_1, m_3', t_3', w_3'. 
  \config{m_1[v/x], {c_2}, t_1 ++ [\av'], w_1}
  \\ 
  \quad \quad 
  \rightarrow^{*} 
  (\config{m_3', \eskip, t_3', w_3'} 
  \land 
  \av  \avin (t_3' - (t_1 ++ [\av'])))
\end{array} \right ) ~ (c)
\end{array}
\]
%
%
According to the Theorem \ref{thm:os_wf_trace} and $(a)$, we know both $t_1$, $t_3$ are well-formed traces.
% }
% \\
% \jl{
\\
Consider 2 cases:
 \[\av \avin (t_3 - (t_1 ++[\av'])) ~(d) ~~ \lor ~~ \av \notin_{aq} (t_3 - (t_1 ++[\av'])) ~(e)\]
%
\begin{itemize}
%
  \caseL{ (d) \[\av \avin (t_3 - (t_1 ++[\av'])) ~ (d)\]}
  By unfolding the trace subtraction operations, we have;
  \[
    \exists t_2. ~ s.t., ~ t_1 ++[\av'] ++ t_2 = t_3 \land \av \avin t_2 ~ (3)
  \]
%
%
According to the Corollary~\ref{coro:aqintrace} and $\av \avin t_2$, we have:
%
\[
  \exists t_{21}, t_{22}, \av' ~ s.t.,~ (\av \aveq \av') \land t_{21} ++ [\av'] ++ t_{22} = t_2 ~ (4)
\]
%
By rewriting $(4)$ inside $(3)$, we have:
%
\[
  \exists t_{21}, t_{22}, \av'' ~ s.t.,~ (\av \aveq \av'') \land t_1 ++[\av'] ++ t_{21} ++ [\av''] ++ t_{22} = t_3 ~ \star
\]
%
By the \emph{ordering} property in definition \ref{def:wf_trace} and $(\star)$, we know
%
\[
  \av' <_{aq} \av
\]
%
This is contradict to our assumption, where $\av' \avgeq \av$. 
%
%
\caseL{(e), \[\av \notin_{aq} (t_3 - (t_1 ++[\av'])) ~(e)\]} 
%
According to the condition $(c)$, we know: $\exists v \in \qdom, v \neq v_1, m_3', t_3', w_3'.$
    % 
\[ 
  \config{m_1[v/x], {c_2}, t_1 ++ [\av'], w_1} 
  \rightarrow^{*} (\config{m_3', \eskip, t_3', w_3'} ~ (5)
  \land \av  \in_{q} (t_3' - (t_1 ++ [\av']))) ~ (6)
\] 
%
According to the Theorem \ref{thm:os_wf_trace}, Sub-Lemma-t and $(5)$,
we know $t_3'- (t_1 ++ [\av'])$ is a well-formed trace.
\\
From $(a)$, we know that the minimal line number of $c_2$ is greater than $l'$, so we know that : 
\[
  \forall \av'' \in t_3'- (t_1 ++ [\av']), \av''>_{aq} \av'
\]
%
Unfolding the trace subtraction operations in $(6)$, we have;
\[
  \exists t_2'. ~ s.t., ~ t_1 ++[\av'] ++ t_2' = t_3' \land \av \avin t_2' ~ (7)
\]
%
%
According to the sub-lemma and $(7)$, we have:
%
\[
  \exists t_{21}', t_{22}' ~ s.t.,~ t_{21}' ++ [\av] ++ t_{22}' = t_2' ~ (8)
\]
%
By rewriting $(8)$ inside $(7)$, we have:
%
\[
  t_1 ++[\av'] ++ t_{21}' ++ [\av] ++ t_{22}' = t_3' ~ \diamond
\]
%
By the \emph{ordering} property in definition \ref{def:wf_trace} and $(\diamond)$, we know
%
\[
  \av' <_{aq} \av
\]
%
This is contradict to the assumption,  $\av' \avgeq \av$.
%
\end{itemize}
%
From both cases, we derive $\av' \avgeq \av$, which is contradict to the hypothesis, i.e., $\av' \avgeq \av$.
Then, we can conclude that 
for any directed edge $(\av', \av) \in \edges$, 
this is not the case that:
%
$$\av' \avgeq \av$$
%
}
\end{proof}
%
%
%
\end{proof}
%
%
%
\begin{lem}
\label{lem:DAG}
[Trace-based Dependency Graph is Directed Acyclic].
\\
%
{
Every trace-based dependency graph is a directed acyclic graph.
}
\end{lem}
%
{
\begin{proof}
Proof is obvious based on the Lemma \ref{lem:edgeforwarding}.
\end{proof}
}
%
\begin{lem}
[Adaptivity is Bounded].
\\
{
Given the program $c$ with a certain database $D$ and starting memory $m$, the $A(c)$ w.r.t. the $D$ and $m$ is bounded, i.e.,:
%
\[
\config{m, c, [], []} 
\rightarrow^{*} 
\config{m', \eskip, t', w'} 
\implies
A_{D, m}(c) \leq |t'|
\]
}
\end{lem}
%
\begin{proof}
{
Proof is obvious based on the Lemma \ref{lem:DAG}.
}
\end{proof}
%
%
\clearpage
%
%
% \subsection{SSA Transformation and Soundness of Transformation}
% in File {\tt ``ssa\_transform\_sound.tex''}
% %
\subsection{SSA Transformation}
We use a translation environment $\delta$, to map variables $x$ in the {\tt While} language to those variables $\ssa{x}$ in the SSA language.
We use a name environment denoted as $\Sigma$ as a set of ssa variables, to get a fresh variable by $fresh(\Sigma)$. 
We define $\delta_1 \bowtie \delta_2 $ in a similar way as
\cite{vekris2016refinement}.
%
\[ 
\delta_1 \bowtie \delta_2 = \{ ( x, {\ssa{x_1}, \ssa{x_2}} ) \in 
\mathcal{VAR} \times \mathcal{SVAR} \times \mathcal{SVAR} \mid x \mapsto {\ssa{x_1}} \in \delta_1 , x \mapsto {\ssa{x_2} } \in \delta_2, {\ssa{x_1} \not= {\ssa{x_2} }  }  \} 
\]
%
\[ 
\delta_1 \bowtie \delta_2 / \bar{x} = \{ ( x, {\ssa{x_1}, \ssa{x_2}} ) \in 
\mathcal{VAR} \times \mathcal{SVAR} \times \mathcal{SVAR}
 \mid x \not\in \bar{x} \land x \mapsto {\ssa{x_1}} \in \delta_1 , x \mapsto {\ssa{x_2} } \in \delta_2, {\ssa{x_1} \not= {\ssa{x_2} }   }  \} 
 \]
 %
We call a list of variables $\bar{x}$.
\[
 [\bar{x}, \bar{\ssa{x_1}}, \bar{\ssa{x_2}}] = \{ (x, x_1,x_2)  | \forall 0 \leq i < |\bar{x}|, x = \bar{x }[i] \land x_1 = \bar{x_1}[i] \land x_2 = \bar{x_2 }[i] \land |\bar{x}| = |\bar{x_1}| = |\bar{x_2}|   \}
\]
%
\begin{mathpar}
\boxed{ \delta ; e \hookrightarrow \ssa{e} }
\and
\inferrule{
}{
 \delta ; x \hookrightarrow \delta(x)
}~{\textbf{S-VAR}}
\and
\boxed{ \Sigma; \delta ; c  \hookrightarrow \ssa{c} ; \delta' ; \Sigma' }
\and
\inferrule{
  { \delta ; \bexpr \hookrightarrow \ssa{\bexpr} }
  \and
  { \Sigma; \delta ; c_1 \hookrightarrow \ssa{c_1} ; \delta_1;\Sigma_1 }
  \and
  {\Sigma_1; \delta ; c_2 \hookrightarrow \ssa{c_2} ; \delta_2 ; \Sigma_2 }
  \\
  {[\bar{x}, \ssa{\bar{{x_1}}, \bar{{x_2}}}] = \delta_1 \bowtie \delta_2  }
  \and
   {[\bar{y}, \ssa{\bar{{y_1}}, \bar{{y_2}}}] = \delta \bowtie \delta_1 / \bar{x} }
  \and
   {[\bar{z}, \ssa{\bar{{z_1}}, \bar{{z_2}}}] = \delta \bowtie \delta_2 / \bar{x} }
  \\
  { \delta' =\delta[\bar{x} \mapsto \ssa{\bar{{x}}'} ][\bar{y} \mapsto \ssa{\bar{{y}}'} ][\bar{z} \mapsto \ssa{\bar{{z}}'} ]}
  \and 
  {\ssa{\bar{{x}}', \bar{y}', \bar{z}'} \ fresh(\Sigma_2)
  }
  \and{\Sigma' = \Sigma_2 \cup \{ \ssa{ \bar{x}', \bar{y}', \bar{z}' } \} }
}{
 \Sigma; \delta ; [\eif(\bexpr, c_1, c_2)]^l  \hookrightarrow [\ssa{ \eif(\bexpr, [\bar{{x}}', \bar{{x_1}}, \bar{{x_2}}] ,[\bar{{y}}', \bar{{y_1}}, \bar{{y_2}}] ,[\bar{{z}}', \bar{{z_1}}, \bar{{z_2}}] , {c_1}, {c_2})}]^l; \delta';\Sigma'
}~{\textbf{S-IF}}
%
\and
%
\inferrule{
 {\delta ; \expr \hookrightarrow \ssa{\expr} }
 \and
 {\delta' = \delta[x \mapsto \ssa{{x}} ]}
 \and{ \ssa{x} \ fresh(\Sigma) }
 \and { \Sigma' = \Sigma \cup \{ \ssa{x} \} }
}{
 \Sigma;\delta ; [\assign x \expr]^{l} \hookrightarrow [\ssa{\assign {{x}}{ \expr}}]^{l} ; \delta'; \Sigma'
}~{\textbf{S-ASSN}}
%
\and
%
\inferrule{
 {\delta ; \query \hookrightarrow \ssa{\query}}
 \and
 {\delta ; \qexpr \hookrightarrow \ssa{\qexpr}}
 \and
 {\delta' = \delta[x \mapsto \ssa{x} ]}
 \and{ \ssa{x} \ fresh(\Sigma) }
  \and { \Sigma' = \Sigma \cup \{ \ssa{x} \} }
}{
 \Sigma;\delta ; [\assign{x}{\query(\qexpr)}]^{l} \hookrightarrow 
 [\assign {\ssa{x}}{ \ssa{\query(\qexpr)}}]^{l} ; \delta';\Sigma'
}~{\textbf{S-QUERY}}
%
%%
\and
%
%
\and
%
\inferrule{
    { \Sigma; \delta ; c \hookrightarrow \ssa{c_1} ; \delta_1; \Sigma_1 }
     \and
    { [ \bar{x}, \ssa{\bar{{x_1}}}, \ssa{\bar{{x_2}}} ] = \delta \bowtie \delta_1 }
    \\
     {
     \ssa{\bar{{x}}'} \ fresh(\Sigma_1 )}
    \and {\delta' = \delta[\bar{x} \mapsto \ssa{\bar{{x}}'}]}
    \and 
     {\delta' ; \bexpr \hookrightarrow \ssa{\bexpr} }
     \and
    {\ssa{c' = c_1[\bar{x}'/ \bar{x_1}]   } }
  }{ 
  \Sigma; \delta ;  \ewhile ~ [\bexpr]^{l} ~ \edo ~ c 
  \hookrightarrow 
  \ssa{\ewhile ~ [\bexpr]^{l}, 0, [\bar{{x}}', \bar{{x_1}}, \bar{{x_2}}] ~ \edo ~ {c} } ; \delta'; \Sigma_1 \cup \{\ssa{\bar{x}'}  \}
}~{\textbf{S-WHILE}
}
\and
%
\inferrule{
 {\Sigma;\delta ; c_1 \hookrightarrow \ssa{c_1} ; \delta_1; \Sigma_1} 
 \and
 {\Sigma_1; \delta_1 ; c_2 \hookrightarrow \ssa{c_2} ; \delta'; \Sigma'} 
}{
\Sigma;\delta ; c_1 ; c_2 \hookrightarrow \ssa{c_1} ; \ssa{c_2} \ ; \delta';\Sigma'
}~{\textbf{S-SEQ}}
\end{mathpar}

\paragraph{Concrete examples.}
\[
c_1 \triangleq
\begin{array}{l}
     \left[x \leftarrow \query(1) \right]^1; \\
     \eif \; (x ==0)^{2} \; \\
    \ethen \; \left[y \leftarrow \query(2) \right]^3\; \\
    \eelse \; \left[y \leftarrow 0 \right]^4 ; \\
    \eif \; (x == 1 )^5\; \\
    \ethen \; \left[ y \leftarrow 0 \right]^6\; \\
    \eelse \; \left[y \leftarrow \query(2) \right]^7\\
\end{array}
%
%
\hspace{20pt} \hookrightarrow  \hspace{20pt}
%
\begin{array}{l}
     \left[ \ssa{x_1} \leftarrow \query(1) \right]^1; \\
     \eif \; (\ssa{x_1 ==0})^{2}, [\ssa{ y_3, y_1,y_2  }],[],[]  \; \\
    \ethen \; \left[ \ssa{y_1} \leftarrow \query(2) \right]^3\; \\
    \eelse \; \left[\ssa{y_2 \leftarrow 0 } \right]^4 ; \\
    \eif \; (\ssa{x_1 == 1} )^{5} , [\ssa{ y_6, y_4, y_5 } ] \; \\
    \ethen \; \left[ \ssa{y_4 \leftarrow 0} \right]^6\; \\
    \eelse \; \left[\ssa{y_5} \leftarrow \query(2) \right]^7\\
\end{array}
\]
\[
c_2 \triangleq
\begin{array}{l}
   \left[ x \leftarrow \query(1) \right]^1; \\
   \left[y \leftarrow \query(2) \right]^2 ; \\
    \eif \;( x + y == 5 )^3\; \\
    \ethen \;\left[ z \leftarrow \query(3)\right]^4 \; \\
    \eelse \;\left[ \ssa{\eskip}\right]^5 ; \\
   \left[ w \leftarrow q_4 \right]^6; \\
\end{array}
\hspace{20pt} \hookrightarrow \hspace{20pt}
%
\begin{array}{l}
   \left[ \ssa{ x_1 } \leftarrow \query(1) \right]^1; \\
   \left[\ssa{ y_1} \leftarrow \query(2) \right]^2 ; \\
    \eif \;( \ssa{ x_1 + y_1 == 5} )^3, [ ],[] ,[ ]\; \\
    \ethen \;\left[ \ssa{ z_1 }
    \leftarrow \query(3)\right]^4 \; \\
    \eelse \;\left[ \eskip\right]^5 ; \\
   \left[ \ssa{ w_1} \leftarrow \query(4) \right]^6; \\
\end{array}
\]

{
\[
c_3 \triangleq
\begin{array}{l}
     \left[x \leftarrow \query(1) \right]^1 ; \\
     \left[i \leftarrow 0 \right]^2 ; \\
    \ewhile ~  [i < 100]^3 ~ \edo
    \\
    ~ \Big( 
    \left[z \leftarrow \query(3) \right]^4; \\
    \left[x \leftarrow z + x \right]^5; \\
    \left[i \leftarrow i + 1 \right]^6
    \Big) ;
\end{array}
%
\hspace{20pt} \hookrightarrow \hspace{20pt} 
%
\begin{array}{l}
     \left[\ssa{x_1} \leftarrow \query(1) \right]^1 ; \\
     \left[\ssa{i_1} \leftarrow 0 \right]^2 ; \\
    \ewhile
    ~ [\ssa{i_1} < 100]^3, 0,
    ~\ssa{[ x_3,x_1 ,x_2 ], [i_3, i_1, i_2] }~
    \edo \\
    ~ \Big( 
    \left[\ssa{z_1} \leftarrow \query(3) \right]^4; \\
    \left[ \ssa{x_2} \leftarrow \ssa{z_1 + x_3} \right]^5; \\
    \left[\ssa{i_2} \leftarrow \ssa{i_3} + 1 \right]^6
    \Big) ;
\end{array}
\]
}
%
\begin{figure}
   \[
 \begin{array}{lll}
    | \ewhile ~ [ \sbexpr ]^{l}, n, [\bar{\ssa{x}}, \bar{\ssa{x_1}}, \bar{\ssa{x_2}}] 
    ~ \edo ~  \ssa{c}|  
    &=& \ewhile ~ [|\sbexpr|]^{l},  ~ \edo ~ |\ssa{c}| 
	\\
    |\ssa{c_1 ; c_2}|  &=& |\ssa{c_1}| ; |\ssa{c_2}| 
    \\
    |[\eif(\sbexpr,
    [ \bar{\ssa{x}}, \bar{\ssa{x_1}}, \bar{\ssa{x_2}}] ,
    [ \bar{\ssa{y}}, \bar{\ssa{y_1}}, \bar{\ssa{y_2}}] , 
    [\bar{\ssa{z}}, \bar{\ssa{z_1}}, \bar{\ssa{z_2}}] , 
    \ssa{ c_1, c_2)}]^{l}|  
    &=&
    [\eif(|\sbexpr|, |\ssa{ c_1}|, |\ssa{c_2}|)]^{l}
    \\
    | [\assign {\ssa{x}}{\ssa{\expr}}]^{l}| & = & [\assign {|\ssa{x}|}{|\ssa{\expr}|} ]^{l}
    \\
    | [\assign {\ssa{x}}{\query(\ssa{\qexpr})} ]^{l} | & = & [\assign {|\ssa{x}|}{|\query(\ssa{\qexpr})|}]^{l}
    \\
    |\ssa{x}_i| & = & x 
    \\
    |n | & = & n 
    \\
    | \saexpr_1 \oplus_{a} \saexpr_2 | & = &  |\ssa{\aexpr_1}| \oplus_a |\ssa{\aexpr_2}| \\
    | \sbexpr_1 \oplus_{b} \sbexpr_2 | & = &  |\sbexpr_1| \oplus_b |\sbexpr_2|
 \end{array}
\]
    \caption{The Erasure of SSA}
    \label{fig:ssa_erasure-while}
\end{figure}
%
%
%
% 
%
\subsection{The Soundness of the Transformation}
In this subsection, we show our transformation from the {\tt While} language to its SSA form is sound with respect to the adaptivity. 
To be specific, a transformed program $\ssa{c}$ starting with appropriate configuration, generates the same trace as the program before the transformation $c$, in its corresponding configuration.
%
%
\begin{defn}[\todo{Written Variables}].
\\
We defined the assigned variables in the while language program $c$ as $\avars{c}$,the assigned variables in the ssa-form program $\ssa{c}$ as $\avarssa{\ssa{c}}$ defined as follows.
\[
\begin{array}{lll}
    \avars{\assign{x}{\expr}} & =& \{ x \} \\
    \avars{\assign{x}{\query(\qexpr)}} & =& \{ x \} \\
    \avars{c_1; c_2}  & = & \avars{c_1} \cup \avars{c_2} \\
    \avars{\ewhile ~ \bexpr ~ \edo ~ c} &= &  \avars{c} \\
    \avars{\eif(\bexpr, c_1, c_2)} & =&  \avars{c_1} \cup \avars{c_2}\\
\end{array} 
\]
%
\[
\begin{array}{lll}
    \avarssa{\ssa{\assign{x}{\expr}}} & =& \{ \ssa{x} \}
    \\
    \avarssa{\ssa{\assign{x}{\query(\ssa{\qexpr})}}} & =& \{ \ssa{x} \}
    \\
    \avarssa{\ssa{c_1; c_2 } }  & = & \avarssa{\ssa{c}_1} \cup \avarssa{\ssa{c}_2}
    \\
    \avarssa{\ewhile ~ \ssa{\bexpr, n, [\bar{x}, \bar{x_1}, \bar{x_2}] ~ \edo ~ \ssa{c}}}
    & = &  
    \{\ssa{\bar{x}}\} \cup \avarssa{\ssa{c}} 
    \\
    \avarssa{\eif(\ssa{\bexpr,[\bar{x}, \bar{x_1}, \bar{x_2}],[\bar{y}, \bar{y_1}, \bar{y_2}],[\bar{z}, \bar{z_1}, \bar{z_2}], c_1, c_2} )} 
    & =&  \{ \ssa{\bar{x}},\ssa{\bar{y}} , \ssa{\bar{z}} \} 
    \cup \avarssa{\ssa{c_1}} \cup \avarssa{\ssa{c_2}}\\
\end{array}
\]
\end{defn}
\begin{defn}[\todo{Read Variables}].
\\
{
The variables read in the while language programs $c$ as $\vars{c}$, variables used in ssa-form program $\ssa{c}$ : 
}
\[
\begin{array}{lll}
    \vars{\assign{x}{\expr}} & =& \vars{\expr}  \\
    \vars{\assign{x}{\query(\qexpr)}} & =&\{  \} \\
    \vars{ c_1; c_2  }  & = & \vars{c_1} \cup \vars{c_2} \\
    \vars{  \eloop ~ \aexpr ~ \edo ~ c  } &= &\vars{\aexpr} \cup \vars{c} \\
    \vars{\eif(\bexpr, c_1, c_2)} & =& \vars{\bexpr} \cup \vars{c_1} \cup \vars{c_2}\\
\end{array}
\]
\[
\begin{array}{lll}
    \varssa{\ssa{\assign{x}{\expr}}} & =& \varssa{\ssa{\expr}}  \\
    \varssa{\ssa{\assign{x}{\query(\qexpr)}}} & =& \{  \} \\
    \varssa{ \ssa{c_1; c_2}  }  & = & \varssa{\ssa{c}_1} \cup \varssa{\ssa{c}_2} \\
    % \varssa{  \eloop ~ \ssa{\aexpr, n, [\bar{x}, \bar{x_1}, \bar{x_2}] ~ \edo ~ c} } &= &\varssa{\ssa{\aexpr}} \cup \varssa{\ssa{c}}  \cup \{ \ssa{\bar{x_1}} \} \cup \{ \ssa{\bar{x_2}} \}\\
    {\varssa{  \ewhile ~ \ssa{\bexpr, n, [\bar{x}, \bar{x_1}, \bar{x_2}] ~ \edo ~ c} }} 
    &= &
    \varssa{\ssa{\bexpr}} \cup \varssa{\ssa{c}}  \cup \{ \ssa{\bar{x_1}} \} \cup \{ \ssa{\bar{x_2}} \}\\
    \varssa{\eif(\ssa{\bexpr,[\bar{x}, \bar{x_1}, \bar{x_2}], [\bar{y}, \bar{y_1}, \bar{y_2}],[\bar{z}, \bar{z_1}, \bar{z_2}], c_1, c_2} )} & =& \varssa{\ssa{\bexpr}} \cup \varssa{\ssa{c_1}} \cup \varssa{\ssa{c_2}} \cup \{\ssa{\bar{x_1}, \bar{x_2},\bar{y_1}, \bar{y_2},\bar{z_1}, \bar{z_2} }\}  \\
\end{array}
\]
\end{defn}
%
\begin{defn}[\todo{Necessary Variables}].
\\
{
We call the variables needed to be assigned before executing the program $c$ as necessary variables $\fv{c}$. Its ssa form is : $\fvssa{\ssa{c}}$.
}  
 \[
 \begin{array}{lll}
     \fvars{\assign{x}{\expr} }  & = & \vars{\expr}  \\
     \fvars{\assign{x}{\query(\qexpr)} }  & = & \{ \}  \\
     {\fvars{  \ewhile ~ \bexpr ~ \edo ~ c  } }&= & \vars{\bexpr} \cup \fvars{c} \\
     \fvars{\eif(\bexpr, c_1, c_2)} & =& \vars{\bexpr} \cup \fvars{c_1} \cup \fvars{c_2}  \\
      \fvars{c_1 ; c_2} & = & \fvars{c_1} \cup ( \fvars{c_2} - \avars{c_1})
 \end{array}
 \]
 \[
 \begin{array}{lll}
     \fvssa{\ssa{\assign{x}{\expr}} }  & = & \varssa{\ssa{\expr}}  \\
     \fvssa{ \ssa{ \assign{x}{\query(\qexpr)}} }  & = & \{ \}  \\
     {\fvssa{  \ewhile ~ \ssa{\bexpr, n, [\bar{x}, \bar{x_1}, \bar{x_2}] ~ \edo ~ c} } }
     &= & 
     \varssa{\ssa{\bexpr}} \cup \fvssa{\ssa{c}}[\ssa{ \bar{x_1}} / \ssa{\bar{x}}]\\
     \fvssa{\eif(\ssa{\bexpr,[\bar{x}, \bar{x_1}, \bar{x_2}],[\bar{y}, \bar{y_1}, \bar{y_2}],[\bar{z}, \bar{z_1}, \bar{z_2}], c_1, c_2} )} & =& \varssa{\ssa{\bexpr}} \cup \fvssa{\ssa{c_1}} \cup \fvssa{\ssa{c_2}}  \\
      \fvssa{\ssa{c_1 ; c_2}} & = & \fvssa{\ssa{c_1}} \cup ( \fvssa{\ssa{c_2}} - \avarssa{\ssa{c_1}})
 \end{array}
 \]
%
\end{defn}
%
The Lemma~\ref{lem:fv} and \ref{lem:same_value} proved the preserving properties for variables and values during the transformation.
%
\begin{lem}[Variable Preserving]
\label{lem:fv}
If $\Sigma;\delta ; c \hookrightarrow \ssa{c} ; \delta';\Sigma' $, $\fvssa{\ssa{c}} = \delta(\fvars{c})$. 
\end{lem}
\begin{proof}
 By induction on the transformation.
 \begin{itemize}
    \caseL{Case $\inferrule{
  { \delta ; \bexpr \hookrightarrow \ssa{\bexpr} }
  \and
  { \delta ; c_1 \hookrightarrow \ssa{c_1} ; \delta_1 }
  \and
  {\delta ; c_2 \hookrightarrow \ssa{c_2} ; \delta_2 }
  \\
  {[\bar{x}, \ssa{\bar{{x_1}}, \bar{{x_2}}}] = \delta_1 \bowtie \delta_2  }
  \and
   {[\bar{y}, \ssa{\bar{{y_1}}, \bar{{y_2}}}] = \delta \bowtie \delta_1 / \bar{x} }
  \and
   {[\bar{z}, \ssa{\bar{{z_1}}, \bar{{z_2}}}] = \delta \bowtie \delta_2 / \bar{x} }
  \\
  { \delta' =\delta[\bar{x} \mapsto \ssa{\bar{{x}}'} ]}
  \and 
  {\ssa{\bar{{x}}', \bar{y}', \bar{z}'} \ fresh }
}{
 \delta ; [\eif(\bexpr, c_1, c_2)]^l  \hookrightarrow [\ssa{ \eif(\bexpr, [\bar{{x}}', \bar{{x_1}}, \bar{{x_2}}] ,[\bar{{y}}', \bar{{y_1}}, \bar{{y_2}}] ,[\bar{{z}}', \bar{{z_1}}, \bar{{z_2}}] , {c_1}, {c_2})}]^l; \delta'
}~{\textbf{S-IF}} $}
From the definition of $\fvssa{[\eif(\sbexpr, [\bar{\ssa{x'}}, \bar{\ssa{x_1}}, \bar{\ssa{x_2}}] , \ssa{c_1}, \ssa{c_2})]^l} = \varssa{\ssa{\bexpr}} \cup \fvssa{\ssa{c_1}} \cup \fvssa{\ssa{c_2}}$. We want to show: \[\varssa{\ssa{\bexpr}}) \cup \fvssa{\ssa{c_1}} \cup \fvssa{\ssa{c_2}} = \delta( \vars{\bexpr}) \cup \delta(\fv{c_1}) \cup \delta(\fv{c_2}  )\]
By induction hypothosis on the second and third premise, we know that : $\fvssa{\ssa{c_1}} = \delta(\fv{c_1}) $ and $\fvssa{\ssa{c_2}} = \delta(\fv{c_2}) $.  We still need to show that: 
\[
  \varssa{\ssa{\bexpr}} = \delta(\vars{\bexpr})
\] 
From the first premise, we know that $\vars{b} \subseteq \dom(\delta)$. This is goal is proved by the rule $\textbf{S-VAR}$ on all the variables in $\bexpr$.\\
{\caseL{Case
$\inferrule{
    { \Sigma; \delta ; c \hookrightarrow \ssa{c_1} ; \delta_1; \Sigma_1 }
     \and
    { [ \bar{x}, \ssa{\bar{{x_1}}}, \ssa{\bar{{x_2}}} ] = \delta \bowtie \delta_1 }
    \\
     {\ssa{\bar{{x}}'} \ fresh(\Sigma_1 )}
    \and {\delta' = \delta[\bar{x} \mapsto \ssa{\bar{{x}}'}]}
    \and 
     {\delta' ; \bexpr \hookrightarrow \ssa{\bexpr} }
     \and
    {\ssa{c' = c_1[\bar{x}'/ \bar{x_1}]   } }
    % \and{ \delta' ; c \hookrightarrow \ssa{c'} ; \delta'' }
  }{ 
  \Sigma; \delta ;  \ewhile ~ [\bexpr]^{l} ~ \edo ~ c 
  \hookrightarrow 
  \ssa{\ewhile ~ [\bexpr]^{l}, 0, [\bar{{x}}', \bar{{x_1}}, \bar{{x_2}}] ~ \edo ~ {c} } ; \delta'; \Sigma_1 \cup \{\ssa{\bar{x}'}  \}
}~{\textbf{S-WHILE}}
$}
}
{
Unfolding the definition, we need to show:
\[\varssa{\ssa{\bexpr}} \cup \fvssa{\ssa{c'}}[\ssa{ \bar{x_1}} / \ssa{\bar{x}}] = \delta (\vars{\bexpr}) \cup \delta(\fv{c} ) \]
We can similarly show that $\varssa{\ssa{\bexpr}} = \delta(\vars{b})$ as in the if case. We still need to show that: 
\[
 \fvssa{\ssa{c_1[\bar{x}' / \bar{x_1}]}}[ \ssa{ \bar{x_1} } / \ssa{\bar{x}'}] =  \delta(\fv{c} )
\]
It is proved by induction hypothesis on $  { \Sigma; \delta ; c \hookrightarrow \ssa{c_1} ; \delta_1; \Sigma_1 }$.\\
}
%
\caseL{Case $\inferrule{
 {\Sigma;\delta ; c_1 \hookrightarrow \ssa{c_1} ; \delta_1; \Sigma_1} 
 \and
 {\Sigma_1; \delta_1 ; c_2 \hookrightarrow \ssa{c_2} ; \delta'; \Sigma'} 
}{
\Sigma;\delta ; c_1 ; c_2 \hookrightarrow \ssa{c_1} ; \ssa{c_2} \ ; \delta';\Sigma'
}~{\textbf{S-SEQ}}$}
To show:
  \[ \fvssa{\ssa{c_1}} \cup ( \fvssa{\ssa{c_2}} - \avarssa{\ssa{c_1}}) = \delta(\fv{c_1} )\cup \delta( \fv{c_2} - \avars{c_1}) \]
  By induction hypothesis on the first premise, we know that : $ \fvssa{\ssa{c_1}} = \delta(\fv{c_1} ) $, still to show: 
    \[ ( \fvssa{\ssa{c_2}} - \avarssa{\ssa{c_1}}) = \delta( \fv{c_2} - \avars{c_1})
    \]
    We know that $\delta_1 = \delta [\avars{c_1} \mapsto \avarssa{\ssa{c_1}} ]$, so by induction hypothesis, we know: $ \fvssa{\ssa{c_2}} = \delta[\avars{c_1} \mapsto \avarssa{\ssa{c_1}} ]( \fv{c_2})  = \delta(\fv{c_2}) \cup \avarssa{\ssa{c_1}} - \delta(\avars{c_1}) $.
    
    This case is proved.
 \end{itemize}
 
\end{proof}

{
We first define a good memory in the {\tt While} language $m$ or in the ssa language $\ssa{m}$ with respect to a translation environment $\delta$, denoted as $m \vDash \delta$ and $\ssa{m} \vDash \delta$ respectively. 
%
\begin{defn}[Well Defined Memory].
\begin{enumerate}
    % \item $m \vDash c \triangleq \forall x \in \fv{c}, \exists v, (x, v) \in m$.
    \item $ m \vDash \delta  \triangleq \forall x \in \dom(\delta), \exists v, (x,v) \in m$.
    % \item $\ssa{m} \vDash_{ssa} \ssa{c} \triangleq \forall \ssa{x} \in \fvssa{\ssa{c}}, \exists v, (\ssa{x}, v) \in \ssa{m}$.
    \item $ \ssa{m} \vDash_{ssa} \delta  \triangleq \forall \ssa{x} \in \codom(\delta), \exists v, (\ssa{x},v) \in \ssa{m}$.
\end{enumerate}
\end{defn}
%
The part declared in the translation environment $\delta$ in a ssa memory $\ssa{m}$ can be reverted to corresponding part of the memory $m$ with an inverse of $\delta$ as follows.
%
\begin{defn}[Inverse of Transformed memory]
 $m = \delta^{-1}(\ssa{m}) \triangleq \forall x \in \dom(\delta), (\delta(x), m(x)) \in \ssa{m} $.
\end{defn}
}
%
\begin{lem}[Value Preserving].
\label{lem:same_value}
{
Given $\delta; e \hookrightarrow \ssa{e}$,  $\forall m. m \vDash \delta. \forall \ssa{m}, \ssa{m} \vDash_{ssa} \delta \land m = \delta^{-1}(\ssa{m})$, then $\config{m, e} \to v $ and $\config{
\ssa{m}, \ssa{e}} \to {v}$.
}
\end{lem}

\begin{thm}[Soundness of transformation]
Given $\Sigma; \delta ; c \hookrightarrow \ssa{c} ; \delta';\Sigma' $, $\forall m. m \vDash \delta. \forall \ssa{m}, \ssa{m} \vDash_{ssa} \delta \land m = \delta^{-1}(\ssa{m})$, if there exist an execution of $c$ in the while language, starting with a trace $t$ and loop maps $w$, $\config{m, c, t, w} \to^{*} \config{m', \eskip, t', w' } $,  then there also exists a corresponding execution of $\ssa{c}$ in the ssa language so that 
  $\config{  {\ssa{m}}, \ssa{c}, t, w} \to^{*} \config{{  \ssa{m'}}, \eskip, t', w' } $ and $ m' = \delta'^{-1}(\ssa{m'}) $.
\end{thm}

\begin{proof}
 We assume that $q$ is the same when transformed to $\ssa{q}$, as the primitive in both languages. And the value remains the same during the transformation.  
 It is proved by induction on the transformation rules.
 \begin{itemize}
   \caseL{Case $\inferrule{
 {\Sigma;\delta ; c_1 \hookrightarrow \ssa{c_1} ; \delta_1;\Sigma_1} 
 \and
 {\Sigma_1; \delta_1 ; c_2 \hookrightarrow \ssa{c_2} ; \delta'; \Sigma'} 
}{
\Sigma;\delta ; c_1 ; c_2 \hookrightarrow \ssa{c_1} ; \ssa{c_2} \ ; \delta';\Sigma'
}~{\textbf{S-SEQ}}$}
We choose an arbitrary memory $m$ so that $m \vDash \delta$, we choose a trace $t$ and a loop maps $w$.
\[
\inferrule
{
{\config{m, c_1,  t,w} \xrightarrow{}^{*} \config{m_1, \eskip,  t_1,w_1}}
\and
{\config{m_1, c_2,  t_1,w_1} \xrightarrow{}^{*} \config{m', \eskip,  t',w'}}
}
{
\config{m, c_1; c_2,  t,w} \xrightarrow{}^{*} \config{m', \eskip, t',w'}
}
\]
 We choose the transformed memory ${\ssa{m}} $ so that  $ m =\delta^{-1}(\ssa{m})$.
 
 To show: $ \config{\ssa{ m, c_1;c_2 }, t, w } \xrightarrow{}^{*} \config{\ssa{m', \eskip}, t'. w' }$ and $ m' = \delta'^{-1} (\ssa{m'}) $.
 
 By induction hypothesis on the first premise, we have:
 \[ \config{\ssa{m, c_1}, t,w} \xrightarrow{}^{*} \config{\ssa{m_1, \eskip},t_1,w_1 } \land m_1 = \delta_1^{-1}(\ssa{m_1}) \]
  By induction hypothesis on the second premise, using the conclusion $ m_1 = \delta_1^{-1}(\ssa{m_1}) $.
  We have:
  \[
   \config{\ssa{m_1, c_2}, t_1,w_1} \xrightarrow{}^{*} \config{\ssa{m', \eskip},t',w' } \land m' = \delta'^{-1}(\ssa{m'})
  \]
  So we know that 
  \[
  \inferrule{
  { \config{\ssa{m, c_1}, t,w} \xrightarrow{}^{*} \config{\ssa{m_1, \eskip},t_1,w_1 }  }
  \and
  { \config{\ssa{m_1, c_2}, t_1,w_1} \xrightarrow{}^{*} \config{\ssa{m', \eskip},t',w' } }
  }{
  \config{\ssa{m, c_1;c_2 }, t,w} \xrightarrow{}^{*} \config{\ssa{m', \eskip}, t' , w' }
  }
  \]
 \caseL{Case $\inferrule{
 { \delta ; \expr \hookrightarrow \sexpr}
 \and
 {\delta' = \delta[x \mapsto \ssa{x} ]}
 \and{ \ssa{x} \ fresh(\Sigma) }
 \and {\Sigma' = \Sigma \cup \{x\} }
}{
 \Sigma;\delta ; [\assign x \expr]^{l} \hookrightarrow [\assign {\ssa{x}}{ \ssa{\expr}}]^{l} ; \delta';\Sigma'
}~{\textbf{S-ASSN}} $ }

 We choose an arbitrary memory $m$ so that $m \vDash \delta$, we choose a trace $t$ and a loop maps $w$, we know that the resulting trace is still $t$ from its evaluation rule $\textbf{assn}$ when we suppose $m, \expr \to v$.
 \[
 \inferrule
{
}
{
\config{m, [\assign x v]^{l},  t,w} \xrightarrow{} \config{m[v/x], [\eskip]^{l}, t,w}
}
~\textbf{assn}
 \]
 We choose the transformed memory ${\ssa{m}} $ so that  $ m =\delta^{-1}(\ssa{m})$.
 
 To show: $\config{\ssa{m}, [\assign {\ssa{x}}{ \ssa{\expr}}]^{l} , t, w} \to^{*} \config{\ssa{m'}, \eskip, t, w} $ and $ m' = \delta'^{-1}(\ssa{m'}) $.
 
 From the rule \textbf{ssa-assn}, we assume $\ssa{m}, \ssa{\expr} \to \ssa{v}$, we know that 
 \[
 \inferrule
{
}
{
\config{\ssa{ m, [\assign x v]^{l}},  t,w } \xrightarrow{} \config{\ssa{m[x \mapsto v], [\eskip]^{l}}, t,w }
}
~\textbf{ssa-assn}
 \]
 We also know that $\delta'= \delta[x \mapsto \ssa{x}]$ and $m = \delta^{-1}(\ssa{m})$, $m'= m[v/x]$. We can show that $ m[v/x] = \delta[x \mapsto \ssa{x}]^{-1}(\ssa{m}[\ssa{x} \mapsto v]) $.
 
\caseL{Case $\inferrule{
 {\delta ; q \hookrightarrow \ssa{q}}
 \and
 {\delta ; \expr \hookrightarrow \ssa{\expr}}
 \and
 {\delta' = \delta[x \mapsto \ssa{x} ]}
 \and{ \ssa{x} \ fresh(\Sigma) }
 \and{ \Sigma' = \Sigma \cup \{x\} }
}{
 \Sigma;\delta ; [\assign{x}{\query(\qexpr)}]^{l} \hookrightarrow [\assign {\ssa{x}}{ \ssa{\query(\qexpr)}}]^{l} ; \delta'
}~{\textbf{S-QUERY}}$} 
We choose an arbitrary memory $m$ so that $m \vDash \delta$, we choose a trace $t$ and a loop maps $w$, we know that when we suppose $\config{m, \expr} \to v$.
 \[
\inferrule
{
\query(v)(D) = \qval 
}
{
\config{m, [\assign{x}{\query(v)}]^l, t, w} \xrightarrow{} \config{m[ \qval/ x], \eskip,  t \mathrel{++} [\query(v),l,w )],w }
}
~\textbf{query}
 \]
 We choose the transformed memory ${\ssa{m}} $ so that  $ m =\delta^{-1}(\ssa{m})$.
 
 To show: $\config{\ssa{m}, [\assign {\ssa{x}}{ \ssa{\query(\qexpr)}}]^{l} , t, w} \to^{*} \config{\ssa{m'}, \eskip, t, w} $ and $ m' = \delta'^{-1}(\ssa{m'}) $.
 
 From the rule \textbf{ssa-query}, we know that 
 \[
 \inferrule
{
\ssa{\query(v)(D) = \qval} 
}
{
\config{ \ssa{ m, [\assign{\ssa{x}}{\ssa{\query(\qexpr)}}]^l}, t, w} \xrightarrow{} \config{\ssa{  m[  x \mapsto v], \eskip,}  t \mathrel{++} [(q^{(l,w )},v)],w }
}
~\textbf{ssa-query}
 \]
 We also know that $\delta'= \delta[x \mapsto \ssa{x}]$ and $m = \delta^{-1}(\ssa{m})$, $m'= m[v/x]$. We can show that $ m[v/x] = \delta[x \mapsto \ssa{x}]^{-1}(\ssa{m}[\ssa{x} \mapsto v]) $.

  \caseL{Case $\inferrule{
  { \delta ; \bexpr \hookrightarrow \ssa{\bexpr} }
  \and
  { \Sigma; \delta ; c_1 \hookrightarrow \ssa{c_1} ; \delta_1;\Sigma_1 }
  \and
  {\Sigma_1; \delta ; c_2 \hookrightarrow \ssa{c_2} ; \delta_2 ; \Sigma_2 }
  \\
  {[\bar{x}, \ssa{\bar{{x_1}}, \bar{{x_2}}}] = \delta_1 \bowtie \delta_2  }
  \and
   {[\bar{y}, \ssa{\bar{{y_1}}, \bar{{y_2}}}] = \delta \bowtie \delta_1 / \bar{x} }
  \and
   {[\bar{z}, \ssa{\bar{{z_1}}, \bar{{z_2}}}] = \delta \bowtie \delta_2 / \bar{x} }
  \\
  { \delta' =\delta[\bar{x} \mapsto \ssa{\bar{{x}}'} ][\bar{y} \mapsto \ssa{\bar{{y}}'} ][\bar{z} \mapsto \ssa{\bar{{z}}'} ]}
  \and 
  {\ssa{\bar{{x}}', \bar{y}', \bar{z}'} \ fresh(\Sigma_2)
  }
  \and{\Sigma' = \Sigma_2 \cup \{ \ssa{ \bar{x}', \bar{y}', \bar{z}' } \} }
}{
 \Sigma; \delta ; [\eif(\bexpr, c_1, c_2)]^l  \hookrightarrow [\ssa{ \eif(\bexpr, [\bar{{x}}', \bar{{x_1}}, \bar{{x_2}}] ,[\bar{{y}}', \bar{{y_1}}, \bar{{y_2}}] ,[\bar{{z}}', \bar{{z_1}}, \bar{{z_2}}] , {c_1}, {c_2})}]^l; \delta';\Sigma'
}~{\textbf{S-IF}}$}
We choose an arbitrary memory $m$ so that $m \vDash \delta$, we choose a trace $t$ and a loop maps $w$.
There are two possible evaluation rules depending on the the condition $b$, we choose the case when $b = \etrue$, we know there is an execution in ssa language so that $\ssa{\bexpr} = \etrue$, we use the rule $\textbf{if-t}$.  
 \[\inferrule
{
}
{
\config{m, [\eif(\etrue, c_1, c_2)]^{l},t,w} 
\xrightarrow{} \config{m, c_1,  t,w} \to^{*} \config{m', \eskip, t', w'}
}
\]
 We choose the transformed memory ${\ssa{m}} $ so that  $ m =\delta^{-1}(\ssa{m})$.
 
 To show: $\config{\ssa{m}, [\eif(\etrue, [\bar{\ssa{x}}', \bar{\ssa{x_1}}, \bar{\ssa{x_2}}] ,[\bar{\ssa{y}}', \bar{\ssa{y_1}}, \bar{\ssa{y_2}}] ,[\bar{\ssa{z}}', \bar{\ssa{z_1}}, \bar{\ssa{z_2}}] , c_1, c_2)]^{l}, t, w} \to^{*} \config{\ssa{m'}, \eskip, t', w'} $ and $ m' = \delta'^{-1}(\ssa{m'}) $.

We use the corresponding rule $\textbf{SSA-IF-T}$.  
\[
\inferrule
{
}
{
\config{\ssa{ { m} , [\eif(\etrue, [\bar{\ssa{x}}', \bar{\ssa{x_1}}, \bar{\ssa{x_2}}] , [\bar{\ssa{y}}', \bar{\ssa{y_1}}, \bar{\ssa{y_2}}] ,[\bar{\ssa{z}}', \bar{\ssa{z_1}}, \bar{\ssa{z_2}}] , \ssa{c_1, c_2})]^{l}},t,w} 
\xrightarrow{} \\ \config{\ssa{ m, c_1}; \eifvar(\ssa{\bar{x}', \bar{x_1}});\eifvar(\ssa{\bar{y}', \bar{y_2}});\eifvar(\ssa{\bar{z}', \bar{z_1}}),  t,w } 
}~\textbf{ssa-if-t}
\]
By induction hypothesis on $ \Sigma;\delta ; c_1 \hookrightarrow  \ssa{c_1}; \delta_1;\Sigma_1$, and we know that $\config{m, c_1,  t,w} \to^{*} \config{m', \eskip, t', w'} $, from our assumption that $ m =\delta^{-1}(\ssa{m})$, we know that 
\[\config{\ssa{ { m}, c_1},  t,w} \to^{*} \config{ \ssa{ { m_1 }, \eskip,} t', w' } \land m' = \delta_1^{-1}(\ssa{m_1}) \]
and we then have:
\[
\inferrule
{
  \config{\ssa{ { m}, c_1},  t,w} \to^{*} \config{ \ssa{ { m_1 }, \eskip,} t', w' }
}
{
 \config{\ssa{  m, c_1;} \eifvar(\ssa{\bar{x}', \bar{x_1})};\eifvar(\ssa{\bar{y}', \bar{y_1})};\eifvar(\ssa{\bar{z}', \bar{z_1})},  t,w  }  \to^{*}
 \config{\ssa{ { m_1 [ \bar{x}' \mapsto {m_1}(\bar{x_1}),\bar{y}' \mapsto {m_1}(\bar{y_2}),\bar{z}' \mapsto {m_1}(\bar{z_1}) ] }, \eskip}, t', w'  }
}
\]
Now, we want to show that $ m' = \delta[\bar{x} \mapsto \ssa{\bar{x}'},\bar{y} \mapsto \ssa{\bar{y}'},\bar{z} \mapsto \ssa{\bar{z}'} ]^{-1}(\ssa{ m_1 [ \bar{x}' \mapsto {m_1}(\bar{x_1}),\bar{y}' \mapsto {m_1}(\bar{y_2}),\bar{z}' \mapsto {m_1}(\bar{z_1}) ] }) $.

Unfold the definition, we want to show that $$\forall x  \in ( \dom(\delta) \cup \bar{x} \cup \bar{y} \cup \bar{z} ), (\delta[\bar{x} \mapsto \ssa{\bar{x}'},\bar{y} \mapsto \ssa{\bar{y}'},\bar{z} \mapsto \ssa{\bar{z}'} ](x), m'(x)) \in \ssa{m_1 [ \bar{x}' \mapsto {m_1}(\bar{x_1}),\bar{y}' \mapsto {m_1}(\bar{y_2}),\bar{z}' \mapsto {m_1}(\bar{z_1}) ] } .$$
\begin{enumerate}
    \item For variable $x$ in $\bar{x}$, we can find a corresponding ssa variable $\ssa{x} \in \ssa{\bar{x}'}$, so that $( \ssa{x}, m'(x) ) \in \ssa{ m_1 [\bar{x}' \mapsto m_1(\bar{x_1})] } $. It is because we know $[x \mapsto \ssa{x_1}]$ for certain $\ssa{x_1} \in \ssa{\bar{x_1}}$ in $\delta_1$, then by unfolding  $m' = \delta_1^{-1}(\ssa{m_1})$ and $\ssa{\bar{x_1}} \in \codom(\delta_1)$, we know $(\ssa{x_1}, m'(x)) \in \ssa{m_1}$ so that $m'(x) = \ssa{m_1}(\ssa{x_1})$.
    \item For variable $y \in \bar{y}$, we know that $y \in \dom(\delta_1)$, then $[ y \mapsto \ssa{y_2} ]$ for certain $\ssa{y_2} \in \ssa{\bar{y_2}}$ in $\delta_1$.  So we know that $(\delta_1(y), m'(y) ) \in \ssa{m_1}$, and then $m'(y) = \ssa{m_1(y_2)}$. We can show $(\ssa{y}, m'(y)) \in \ssa{m_1[\bar{y}' \mapsto m_1(\bar{y_2})]}$.
    \item For variable $z \in \bar{z}$, we know that $z \not\in \dom(\delta_1)$ by the definition (otherwise $z$ will appear in $\bar{x}$), then $[ z \mapsto \ssa{z_1} ]$ for certain $ \ssa{z_1} \in \ssa{\bar{z_1}}$ in $\delta$.  We know $(\delta(z), m(z)) \in \ssa{m}$ from our assumption, so we have $ m(z) = \ssa{m(z_1)}$. Because $z$ is not modified in $c_1$, so that $m(z) = m'(z)$. Also $\ssa{m}$ will not shrink during execution and $\ssa{z_1}$ will not be written in $\ssa{c_1}$, so $(\ssa{z_1}, m'(z)) \in \ssa{m_1}$. Then we can show that $ (\ssa{z}, m'(z) ) \in \ssa{m_1[\bar{z}' \mapsto m_1(\bar{z_1})] }$.
    \item For variable $k \in \dom(\delta)- \bar{x} - \bar{y}-\bar{z}$. From our assumption $ m = \delta^{-1}(\ssa{m})$, we can show $(\delta(k), m(k) ) \in \ssa{m}$. We know that $k$ is not written in either branch from our definition, so $(\delta(k), m'(k) ) \in \ssa{m_1} $ .
\end{enumerate}

{
\caseL{
	Case
	$
	\inferrule{
    { \Sigma; \delta ; c \hookrightarrow \ssa{c_1} ; \delta_1; \Sigma_1 }
     \and
    { [ \bar{x}, \ssa{\bar{{x_1}}}, \ssa{\bar{{x_2}}} ] = \delta \bowtie \delta_1 }
    \\
     {\ssa{\bar{{x}}'} \ fresh(\Sigma_1 )}
    \and {\delta' = \delta[\bar{x} \mapsto \ssa{\bar{{x}}'}]}
    \and 
     {\delta' ; \bexpr \hookrightarrow \ssa{\bexpr} }
     \and
    {\ssa{c' = c_1[\bar{x}'/ \bar{x_1}]   } }
    % \and{ \delta' ; c \hookrightarrow \ssa{c'} ; \delta'' }
  }{ 
  \Sigma; \delta ;  \ewhile ~ [\bexpr]^{l} ~ \edo ~ c 
  \hookrightarrow 
  \ssa{\ewhile ~ [\bexpr]^{l}, 0, [\bar{{x}}', \bar{{x_1}}, \bar{{x_2}}] ~ \edo ~ {c} } ; \delta'; \Sigma_1 \cup \{\ssa{\bar{x}'}  \}
}~{\textbf{S-WHILE}}
$
}
}
\\
{
We choose an arbitrary memory $m$ so that $m \vDash \delta$, we choose a trace $t$ and a loop maps $w$. Suppose $ \config{m ,a} \to v_N $. There are two cases, when $v_N=0$, the loop body is not executed so we can easily show that the trace is not modified.
%
When the while loop is still running ($v_N > 0$), we have the following evaluation in the while language:
\[
\inferrule
{
 \empty
}
{
\config{
m, \ewhile ~ [b]^{l} ~ \edo [c]^{l + 1},  t, w 
}
\xrightarrow{} \config{m, c ; 
\eif_w (b, c ; 
\ewhile ~ [b]^{l} ~ \edo [c]^{l + 1},  \eskip),
t, w }
}
~\textbf{while-b}
\]
which follows by the following evaluation:
\[
	\inferrule
{
 m, b \xrightarrow{} b'
}
{
\config{m, \eif_w (b, c,  \eskip) ,  t, w }
\xrightarrow{} \config{m, 
 \eif_w (b', c,  \eskip), t, w }
}
~\textbf{ifw-b}
\]
In the corresponding ssa-form language, we have the corresponding evaluation in the same way by assuming 
$m = \delta^{-1}(\ssa{m})$.
%
\[
	\inferrule
{
 {n = 0 \rightarrow i = 1 }
 \and
 {n > 0 \rightarrow i = 2 }
}
{
\config{
\ssa{m},  
\ssa{\ewhile ~ [\bexpr]^{l}, n, 
[\bar{{x}}', \bar{{x_1}}, \bar{{x_2}}] 
~ \edo ~ {c} 
},  t, w 
}
\xrightarrow{} \\ 
\config{
\ssa{m},
\eif_w 
(\ssa{b[\bar{x_i}/\bar{x'}], [\bar{{x}}', \bar{{x_1}}, \bar{{x_2}}], n,  c[\bar{x_i}/\bar{x'}] }; 
\ssa{
\ewhile ~ [b]^{l}, n+1, 
[\bar{{x}}', \bar{{x_1}}, \bar{{x_2}}]  
~ \edo ~ c} ,  \eskip),
t, w
}
}
~\textbf{ssa-while-b}
\]
This evaluation is followed by the following evaluation:
\[
	\inferrule
{
 \ssa{m, b \xrightarrow{} b'}
}
{
\config{\ssa{m, \eif_w (b, [\bar{{x}}', \bar{{x_1}}, \bar{{x_2}}] , n,  c_1,  c_2)} ,  t, w }
\xrightarrow{} \config{\ssa{ m, 
 \eif_w (b', [\bar{{x}}', \bar{{x_1}}, \bar{{x_2}}] , n , c_1 , c_2 )}, t, w }
}
~\textbf{ssa-ifw-b}
\]
%
Depending on if the counter $n$ is equal to $0$ or not, there are two possible execution paths (the variables $\ssa{\bar{x}}$ is replaced by the $\ssa{\bar{x_1}}$ or $\ssa{\bar{x_2}}$). We start from the first iteration (when $n =0$) when $v_N >0$. 
}
{
By induction hypothsis on the premise $ { \Sigma; \delta ; c \hookrightarrow \ssa{c_1} ; \delta_1; \Sigma_1 }$, we know that 
\[ \config{\ssa{{m}, c'[ \bar{x_1}/\bar{x}'  ]}, t, (w+l)  } \to^{*} \config{\ssa{{m'}, \eskip}, t'_{i}, w'  } \land m' = \delta_1^{-1}(\ssa{m'})   \]
Hence we can conclude that:
\[
  \inferrule{
   \config{\ssa{{m}, c'[ \bar{x_1}/\bar{x}'  ]}, t, (w+l) }  \to^{*} \config{\ssa{{m'}, \eskip}, t'_{1}, w'  }
  }{
  \config{\ssa{ {m}, c'[ \bar{x_1}/\bar{x}'  ];  [\eloop ~ (\valr_N-1), n+1, [\bar{\ssa{x}}', \bar{\ssa{x_1}}, \bar{\ssa{x_2}}] ~  \edo ~ c' ]^{l} },  t, (w + l)  }  \to^{*} \\ \config{ \ssa{{m'}, [\eloop ~ (\valr_N-1), n+1, [\bar{\ssa{x}}', \bar{\ssa{x_1}}, \bar{\ssa{x_2}}] ~  \edo ~ c' ]^{l}}, t'_{1}, w'  } 
  }
\]
%
Then there are two cases, 
%
\begin{enumerate}
     \item  when guard in the $\eif_w$ expression evaluates to $\efalse$, the while loop terminates and exits.
     The execution in the while language is defined in the evaluation rule $\textbf{ifw-false}$ as follows.
     \[
		\inferrule
		{
		 \empty
		}
		{
		\config{\ssa{
		m, \eif_w (
		\efalse, [\bar{{x}}', \bar{{x_1}}, \bar{{x_2}}],   n, 
		c; {\ewhile ~ [b]^{l} ~ \edo ~ c},
		\eskip)
		)} ,  t, w }
		\\
		\xrightarrow{} 
		\config{\ssa{m, 
		{\eskip}; \eifvar(\bar{x'}, \bar{x_i}) }, t, (w - l) }
		}
		~\textbf{ifw-false}
	\]
%
	The corresponding ssa-form evaluation as follows:
	\[
		\inferrule
		{
		 { n = 0 \rightarrow i = 1 }
		 \and
		 {n > 0 \rightarrow i =2}
		}
		{
		\config{\ssa{
		m, \eif_w (
		\efalse, [\bar{{x}}', \bar{{x_1}}, \bar{{x_2}}],   n, 
		{  
		c; \ssa{\ewhile ~ [b]^{l}, n, [\bar{{x}}', \bar{{x_1}}, \bar{{x_2}}]  ~ \edo ~ c},
		\eskip)
		} 
		)} ,  t, w }
		\\
		\xrightarrow{} 
		\config{\ssa{m, 
		{\eskip}; \eifvar(\bar{x'}, \bar{x_i}) }, t, (w - l) }
		}
		~\textbf{ssa-ifw-false}
	\]
	We can see that both traces are not changed during the exit of the while. We need to show that $ m' = \delta^{-1} (\ssa{m'[\bar{x} \mapsto m'(\bar{x_2})]}) $. We know that $[ \bar{x} \mapsto \bar{x_2}]$ in $\delta_1$ from the definition, so we can show that for any variable $\ssa{x_2} \in \bar{x_2}$, $( \ssa{x_2}, m'(x) ) \in \ssa{m'}$. For variables $x \in {\dom(\delta) - \bar{x} } $, the variable is not modified during the execution of $c$ so that we know $m(x) = m'(x)$, and then we can show that $(\delta(x), m'(x)) \in \ssa{m'} $ because $\delta(x)$ is not written in $\ssa{c'[\bar{x_1}/ \bar{x}']}$ .
%
  	\item 
		when guard in the $\eif_w$ expression evaluates to $\etrue$, the while terminates and exits.
     The execution in the while language is defined in the evaluation rule $\textbf{ifw-true}$.
          %
     We want to show that : assuming in the $i-th$ $(i < \ssa{n})$ iteration, starting with $t_i$ and $w_i$ and $m_i = \delta_1^{-1}(\ssa{m_i})$,
     this command is evaluated according to the while language operation semantics as
     	$
		\config{m, \eif_w (\etrue, c ; \ewhile ~ [b]^{l} ~ \edo ~ c, ,  \eskip) ,  t, w }
		\xrightarrow{}^* \config{m, c 
		t, (w + l) }
 		$.
     %
     Then the corresponding ssa form evaluation as follows : 
     %
     \[ 
     \inferrule{}{
     	\config{
		\ssa{
			m, 
			{
			\eif_w (\etrue, [\bar{{x}}', \bar{{x_1}}, \bar{{x_2}}], n,  
			c; \ssa{\ewhile ~ [b]^{l}, n, [\bar{{x}}', \bar{{x_1}}, \bar{{x_2}}]  ~ \edo ~ c},
			\eskip)
			} 
		},  t, w 
		}
		\\
		\xrightarrow{} 
		\config{
		\ssa{m, 
		{
		\eif_w (\etrue, [\bar{{x}}', \bar{{x_1}}, \bar{{x_2}}], n,  
		c; \ssa{\ewhile ~ [b]^{l}, n, [\bar{{x}}', \bar{{x_1}}, \bar{{x_2}}]  ~ \edo ~ c},
		}
		}
		t, (w + l) }
		} 
     \]  
     and $m_i = \delta^{-1}(\ssa{m_i}) $.
     We then have the evaluation in the while language:
     \[
		\inferrule
		{
		 \empty
		}
		{
		\config{m, 
		\eif_w (b, 
		c ; \ewhile ~ [b]^{l} ~ \edo ~ c, 
		\eskip),
		t, w }
		\xrightarrow{} 
		\config{m, 
		c ; \ewhile ~ [b]^{l} ~ \edo ~ c,  
		t, (w + l) }
		}
		~\textbf{ifw-true}
	\]
	We then have the following evaluation:
	\[
		\inferrule
		{
		 \empty
		}
		{
		\config{
		\ssa{
		m, 
		{
		\eif_w (\etrue, [\bar{{x}}', \bar{{x_1}}, \bar{{x_2}}], n,  
		c; \ssa{\ewhile ~ [b]^{l}, n, [\bar{{x}}', \bar{{x_1}}, \bar{{x_2}}]  ~ \edo ~ c},
		\eskip)
		} 
		},  t, w 
		}
		\\
		\xrightarrow{} 
		\config{
		\ssa{m, 
		{
		\eif_w (\etrue, [\bar{{x}}', \bar{{x_1}}, \bar{{x_2}}], n,  
		c; \ssa{\ewhile ~ [b]^{l}, n, [\bar{{x}}', \bar{{x_1}}, \bar{{x_2}}]  ~ \edo ~ c},
		}
		}
		t, (w + l) }
		}
		~\textbf{ssa-ifw-true}
	\]
%
By induction hypothsis on the premise $  { \Sigma; \delta_1 ; c \hookrightarrow \ssa{c_2} ; \delta_1; \Sigma_1 }$, we know that
%
\[
\config{\ssa{{m_i}, c'[ \bar{x_2}/\bar{x}'  ]}, t_i, (w_i+l)  } \to^{*} \config{\ssa{{m_{i+1}}, \eskip}, t_{i+1}, w_{i+1}  } \land m_{i+1} = \delta_1^{-1}(\ssa{m_{i+1}})
\]
%
Hence we can conclude that:
\[
  \inferrule{
   \config{\ssa{{m_i}, c'[ \bar{x_2}/\bar{x}'  ]}, t_i, (w_i+l) }  \to^{*} \config{\ssa{{m_{i+1}}, \eskip}, t_{i+1}, w_{i+1}  }
  }{
  \config{\ssa{ {m_i}, c'[ \bar{x_2}/\bar{x}'  ];  [\eloop ~ (\valr_N-i-1), n+1, [\bar{\ssa{x}}', \bar{\ssa{x_1}}, \bar{\ssa{x_2}}] ~  \edo ~ c' ]^{l} },  t_i, (w_i + l)  }  \to^{*} \\ \config{ \ssa{{m_{i+1}}, [\eloop ~ (\valr_N-i-1), n+1, [\bar{\ssa{x}}', \bar{\ssa{x_1}}, \bar{\ssa{x_2}}] ~  \edo ~ c' ]^{l}}, t_{i+1}, w_{i+1}  } 
  }
\]
So we can show that before the exit of the loop after ($v_N= n $) iterations, we have $t_{n} = t_{n}$ and $m_{n} = \delta_1^{-1}(\ssa{m_{n}})$.
 \end{enumerate}
%
This proof is similar when it comes to the exit as in case 1. 
}
\end{itemize}
%
\end{proof}
%
\clearpage
%
\clearpage
%
%
% % In this section, we present our algorithm for computing the upper bound for a program $c$'s adaptivity
% $A(c)$ defined~\ref{def:trace_adapt} through static program analysis.
% This section presents the key definitions
% for the static analysis algorithm in Section~\ref{sec:algorithm-keys} before going into the detail of the algorithm,
% then shows the complete static analysis algorithm.
% \mg{
% In this section, we present our static program analysis for computing an upper bound on the adaptivity a program $c$
% }
In this section, we present our static program analysis for computing an upper bound on the adaptivity a program $c$.
%
\subsection{A guide to the algorithm}
In order to have the upper bound of adaptivity:
\\
1. $\THESYSTEM$  first build a program-based dependency graph to {over-}approximate the
trace-based dependency graph
through Section~\ref{sec:alg_vertexgen}, Section~\ref{sec:alg_weightedgegen} and~\ref{sec:alg_graphgen}:
% in the phases one to phases four of $\THESYSTEM$ in Section~\ref{sec:abscfg} to Section~\ref{sec:alg_graphgen}:
\\
1.1 approximate the vertices: 
% in the forth step of
in the first phase of $\THESYSTEM$ in
% the algorithm in Section~\ref{sec:alg_graphgen} 
Section~\ref{sec:alg_vertexgen}
without extra static analysis technique.
\\
1.2 approximate the vertices weights:
in the second phase of 
% the algorithm in  
of $\THESYSTEM$ specifically in Section~\ref{sec:alg_weightgen}
\\
1.3 approximate the edges between vertices:
also in the second phase of $\THESYSTEM$ specifically in 
% Section~\ref{sec:alg_weightgen}
Section~\ref{sec:alg_edgegen}
\\
1.4 generate the final approximated program-based dependency graph in Section~\ref{sec:alg_graphgen}
%  to {over-}approximate the
% approximate the query annotation: 
% in the forth step of
in the third phase of $\THESYSTEM$.
% the algorithm  without extra static analysis technique.
\\
2. Then in the last phase in Section~\ref{sec:alg_adaptcompute}, $\THESYSTEM$
% we compute the upper bound for adaptivity over this approximated graph:
% , as an upper bound for
% program's adaptivity
computes the upper bound for adaptivity over this approximated graph.
% in the last phase of this algorithm in Section~\ref{sec:alg_adaptcompute}.
\\

% \subsection{Adaptivity Based on Program Analysis in \THESYSTEM}
% In order to give a bound on the program's adaptivity, we first build a
% program-based data-dependency graph to {over-}approximate the
% trace-based dependency graph.  Then, we define a program-based
% adaptivity over this approximated graph, as an upper bound for
% $A(c)$.

This program-based graph has a similar topology structure as the one
of the trace-based (semantic) dependency graph. It has the same
vertices and query annotations, and approximated edges and weights.  
% An
% approximated edge correspond to a program-based data dependency
% relation ($\flowsto$ in Definition~\ref{def:flowsto}) and an approximated
% weight corresponds to a reachability bound analysis results from
% Definition~\ref{def:transition_closure}.

% %
% \subsubsection{Program-Based Variable Dependency}
% The program-based dependency relation over two labeled variables ($x^i, y^j)$ is defined as a $\flowsto$ relation with respect to the program $c$ as follows.
% %
% \begin{defn}[Data Flow Relation between Assigned Variables ($\flowsto$)].
% \label{def:flowsto}
% \\
% Given a program  ${c}$,
% a variable ${x^i}  \in \lvar_c $ is in the \emph{flows to} relation with another variable ${y^j} \in \lvar_c$, if and only if:
% \mg{I cannot even parse the next formula. Why there is a big disjunction on the left? Disjunction is a binary operation, or n-ary if given a set, what is this disjunction between?}\\
% \mg{please, remove the underscript $c$ in the exists. It just makes everything mroe difficult to parse.}\\
% \mg{The use of $\lor$ is odd. E.g. $\exists {(\expr \lor \qexpr)}$ or
%   $ [{\assign{y}{\expr \lor \query(\qexpr)}}]^{j} $. I suggest to write the whole formula instead of using weird shortenings.}\\
% \mg{Also, now it is too late to change this but instead of breaking down the definition using the subterm relation and then defining the flowto relation, it would have been better to give just one inductive definition of Flowto - I imagine that this makes also the proof more awkward.}
% %
% {\footnotesize 
% \[
% \begin{array}{l}
% \flowsto({x^i, y^j, c}) \triangleq 
% \\
% \left( \bigvee
% \begin{array}{l}
% (\exists \expr \st \clabel{\assign{y}{\expr}}^j \in_{c} {c} 
% \land {x} \in FV(\expr) \land (x^i \in \live(j, c)))
% \\
% (\exists {\qexpr} \st [\assign{y}{\query({\qexpr})}]^j \in_{c} {c} 
% \land x \in FV({\qexpr}) \land (x^i \in \live(j,c))))
% \\
% \Big(\exists {c_w} \in \cdom, l \in \mathcal{L}, \bexpr \st
% 	\ewhile [\bexpr]^l \edo {c_w} \in_{c} {c}
% 	\land \flowsto(x^i, y^j, c_w)
% 	\\ \qquad	
%      \lor 
% 	\big( \exists {(\expr \lor \qexpr)} \st
% 	[{\assign{y}{\expr \lor \query(\qexpr)}}]^{j} \in_{c}  {c_w}  \land {x} \in FV(\bexpr) \land x^i \in \live(l, c)
% 	\big)
% 	\Big)
% \\
% \Big(
% \exists {c_1}, {c_2} \in \cdom, l \in \mathcal{L}, \bexpr 
% \st 
% 	\eif([\bexpr]^l, {c_1}, {c_2}) \in_{c} {c} \land
% 	\flowsto(x^i, y^j, c_1) \lor \flowsto(x^i, y^j, c_2)
% 	\\ \qquad 
% 	\lor 
% 	\big( \exists {(\expr \lor \qexpr)} \st
% 	\land {x} \in FV(\bexpr) \land x^i \in \live(l, c) \land
% 	([{\assign{y}{\expr \lor \query(\qexpr)}}]^{j} \in_{c}  {c_1}  
% 	\lor [{\assign{y}{\expr \lor \query(\qexpr)}}]^{j} \in_{c}  {c_2})
% 	\big)
% \Big)
% % \\
% \end{array}
% \right).
% \end{array}
% \]
% }
% %
% \end{defn}
% %
% \mg{The next notation is inconsistent with the one used above. Also, this definition should be given before the definition of flowto. From the description I have no cluse what this notion of reachability means. Also, the definition is referred to does not define this notation.}\\
% \mg{the definition somehow seems to make sense but until when the or notation is fixed and I don't see the definition of RD, I cannot tell for sure.}
% $\live^l(c) \subseteq \lvar_c$,
% which is the set of all the reachable variables at location of label $l$ in the program $c$.
% For every labelled variable $x^l$ in this set, 
% the value assigned to that variable
% in the assignment command associated to that label is reachable at the entry point of  executing the command of label $l$.
% This is formally defined , formally computed in Definition~\ref{def:feasible_flowsto}
% \\
% \mg{This description seems inconsistent with the definition. I suggest to use the same variables and terms.}
% To understand the $\flowsto$ intuition, 
% given a program  ${c}$ with its labelled variables $\lvar_c$, and two variables ${x^i}, y^j  \in \lvar_c $ 
% % showing up as $i$-th, $j$-th elements in $\lvar$ 
% % (i.e., ${x} = \lvar(i)$ and ${y} = \lvar(j)$),
% we say $y^j$ flows to ${x^i}$ in ${c}$ if and only if 
% the value of $y^j$ directly or indirectly influence the evaluation of the value of ${x}$ as follows:
% %
% \begin{itemize}
% \item (Explicit Influence) The program ${c}$ contains either 
% a command $[\assign{{x}}{\aexpr}]^i$ or $[\assign{{x}}{\query({\qexpr})}]^i$,
% such that ${y}$ shows up as a free variable in $\expr$ or ${\qexpr}$.
% We use $\flowsto({x^i, y^j, c})$ to denote $y^j$ flows to $x^i$ in ${c}$.
% %
% \item (Implicit Influence) The program ${c}$ contains either a while loop
% command
% or if command, 
% such that $x$ shows up in the guard
% and $y$ shows up in the left hand of an assignment command and this assignment command showing up
%  in the body of the while loop, or branches of if command.
% \end{itemize}
% %
% % This is formally defined in \ref{def:flowsto}.
% % We use $FV(\expr)$, $FV(\sbexpr)$ and $FV(\qexpr)$ denote the set of free variables in 
% % expression $\expr$, boolean expression $\sbexpr$ and query expression $\qexpr$ respectively.
% %
% %
% \mg{I don't understand what this definition of equivalence means. It is not observational equivalence
% and it is not syntactic equivalence. What are we trying to capture here? Also, it is equivalence of programs, not of program.}
% \begin{defn}[Equivalence of Program]
% %
% \label{def:aq_prog}
% Given 2 programs $c_1$ and $c_2$:
% \[
% c_1 =_{c} c_2
% \triangleq 
% \left\{
%   \begin{array}{ll} 
%     \etrue        
%     & c_1 = \eskip \land c_2 = \eskip
%     \\ 
%     \forall \trace \in \mathcal{T} \st \exists v \in \mathcal{VAL}
%     \st \config{ \trace, \expr_1} \aarrow v \land \config{ \trace, \expr_1} \aarrow v     
%     & c_1 = \assign{x}{\expr_1} \land c_2 = \assign{x}{\expr_2} 
%     \\ 
%     \qexpr_1 =_{q} \qexpr_2       
%     & c_1 = \assign{x}{\query(\qexpr_1)} \land c_1 = \assign{x}{\query(\qexpr_2)} 
%     \\
%     c_1^f =_{c} c_2^f \land c_1^t =_{c} c_2^t
%     & c_1 = \eif(b, c_1^t, c_1^f) \land c_2 = \eif(b, c_2^t, c_2^f)
%     \\ 
%     c_1' =_{c} c_2'         
%     & c_1 = \ewhile b \edo c_1' \land c_2 = \ewhile b \edo c_2'
%     \\ 
%     c_1^h =_{c} c_2^h \land c_1^t =_{c} c_2^t
%     & c_1 = c_1^h;c_1^t \land c_2 = c_2^h;c_2^t 
%   \end{array}
%   \right.
% \]
% %
% As usual, we denote by $c_1 \neq_{c} c_2$ the negation of the equivalence.
% %
% \end{defn}
% %
% \mg{This definition needs to go before it is used. }
% Given 2 programs $c$ and $c'$, we denote by $c' \in_{c} c$  that $c'$ is a sub-program of $c$ defined as follows,
% \begin{equation}
% c' \in_{c} c \triangleq \exists c_1, c_2, c''. ~ s.t.,~
% c =_{c} c_1; c''; c_2 \land c' =_{c} c''
% \end{equation} 
% %

% \subsubsection{Program Analysis Based Dependency Graph}
% We give the formal definition for the program-based dependency graph for a program $c$, 
% $\progG({c}) = (\vertxs, \edges, \weights, \flag)$ as follows.
% \begin{defn}
%     [Program-Based Dependency Graph].
%     \label{def:prog_graph}
%     \\
% Given a program ${c}$
% its program-based graph 
% $\progG({c}) = (\vertxs, \edges, \weights, \qflag)$ is defined as:
% {\footnotesize
% \[
% \begin{array}{rlcl}
% \text{Vertices} &
% \vertxs & := & \left\{ 
% x^l \in \mathcal{LV} 
% ~ \middle\vert ~
% x^l \in \lvar_{c}
% \right\}
% \\
% \text{Directed Edges} &
% \edges & := & 
% \left\{ 
%   ({x}_1^{i}, {x}_2^{j}) \in \mathcal{LV} \times \mathcal{LV}
%   ~ \middle\vert ~
%   \begin{array}{l}
%     {x}_1^{i}, {x}_2^{j} \in \vertxs
% 	\land
%     % \\
%     \exists n \in \mathbb{N}, z_1^{r_1}, \cdots, z_n^{r_n} \in \lvar_{{c}} \st 
%     n \geq 0 \land
%     \\
%     \flowsto(x^i,  z_1^{r_1}, c) 
%     \land \cdots \land \flowsto(z_n^{r_n}, y^j, c) 
%   \end{array}
% \right\}
% \\
% \text{Weights} &
% \weights & := &
% % \bigcup
% % \begin{array}{l}
% 	\left\{ (x^l, w) \in  \mathcal{LV} \times EXPR(\constdom)
% 	\mid
% 	x^l \in \lvar_{{c}} \land w = \absW(l)
% 	\right\}
% % \end{array} 
% \\
% \text{Query Annotation} &
% \qflag & := & 
% \left\{(x^l, n)  \in  \mathcal{LV} \times \{0, 1\} 
% ~ \middle\vert ~
%  x^l \in \lvar_{c},
% n = 1 \iff x^l \in \qvar_{c} \land n = 0 \iff  x^l \in \qvar_{c} .
% \right\}
% \end{array}
% \] 
% }
% , where the $\absW(l)$ is the symbolic reachability bound in domain of $EXPR(\constdom)$,
% % for the assignment command of label $l$ to which  
% the labeled variable $x^l$, 
% % is associated, 
% computed from the $\THESYSTEM$ algorithm 
% in Definition~\ref{def:transition_closure}.
% The $EXPR(\constdom)$ is an expression over symbolic constants containing the
% input variables and natural number.
% \end{defn} 
% %
% \paragraph{Program-Based Adaptivity ($\progA(c)$)}
% %
% Given a program ${c}$, we generate its program-based graph 
% $\progG({c}) = (\vertxs, \edges, \weights, \qflag)$.
% %
% Then the adaptivity bound based on program analysis for ${c}$ 
% % is the number of query vertices on a finite walk in $\progG({c})$. This finite walk satisfies:
% % \begin{itemize}
% % \item the number of query vertices on this walk is maximum
% % \item the visiting times of each vertex $v$ on this walk is bound by its reachability bound $\weights(v)$.
% % \end{itemize}
% is computed as the maximum query length over all finite walks in $\walks(\progG({c}))$,
% %
% % It is formally defined in \ref{def:prog_adapt}.
% defined formally as follows.
% %
% %
% \begin{defn}
% [{Program-Based Adaptivity}].
% \label{def:prog_adapt}
% \\
% {
% Given a program ${c}$ and its program-based graph 
% $\progG({c}) = (\vertxs, \edges, \weights, \qflag)$,
% %
% the program-based adaptivity for $c$ is defined as%
% \[
% \progA({c}) 
% := \max
% \left\{ \qlen(k)\ \mid \  k\in \walks(\progG({c}))\right \}.
% \]
% }
% \end{defn}  
%
%
% {
% \begin{defn}[Variable Flags ($\flag$)].
% \\
% Given a program  ${c}$ with its labelled variables $\lvar$, the $\flag$ is a vector of the same length as $\lvar$, s.t. for each variable ${x}$ showing up as the $i$-th element in $\lvar$ (i.e., ${x} = \lvar(i)$), 
% $\flag(i) \in \{0, 1, 2\}$ is defined as follows:
% %
% %
% \[
% \flag(i) := 
% \left\{
% \begin{array}{ll}
% 2 & 
% {x^l} \in \lvar_{c} \land 
% (\exists {\qexpr}. ~ s.t., ~
% [\assign{{x}}{\query({\qexpr})}]^l \in_{c} {c})
% \\
% 1 &  
% \begin{array}{l}
% {x^l} \in \lvar_{c} \bigwedge \\
% \left(
% \begin{array}{l}
% \big(\exists  ~ {c'}, {\expr}, \sbexpr, l, l'. ~
% 	\ewhile [\sbexpr]^l \edo {c'} \in_{c} {c}
% 	\land 
% 	[{\assign{x}{\expr}}]^{l'} \in_{c}  {c'}
% \big) \bigvee
% \\
% \big(\exists ~ \sbexpr, l, l_1, l_2, {c_1}, {c_2}, {\expr}_1, {\expr}_2. ~
% 	\eif([\sbexpr]^l, {c_1}, {c_2}) \in_{c} {c} \land
% 	([{\assign{x}{\expr_1}}]^{l1} \in_{c} {c_1} \lor 
% 	[{\assign{x}{\expr_2}}]^{l2} \in_{c} {c_2})
% \big)
% \end{array}
% \right)
% \end{array}
% \\
% 0 & \text{o.w.}
% \end{array}
% \right\}. 
% \] 
% %
% \end{defn}
%
% Operations on $\flag$ are defined as follows:
% \begin{equation}
% \begin{array}{llll}
% {\flag_1 \uplus \flag_2}(i) & := &
% \left\{
% \begin{array}{ll}
% k & k = \max{\big\{\flag_1(i), \flag_2(i)\big\}} 
% \land |\flag_1| = |\flag_2|\\
% 0 & o.w.
% \end{array}\right.
% & i = 1, \cdots, |\flag_1|  
% \\
% {\flag \uplus n}(i) & := & 
% \max\big\{ \flag(i), n \big\} 
% & i = 1, \ldots, |\flag|    
% \\
% \left[ n \right]^k (i) & := &  n
% & i = 1, \ldots, k ~ \land ~ |\left[ n \right]^k| = k
% \end{array}
% \end{equation}
%
%
%
% \begin{defn}[Data Flow Matrix ($\Mtrix$)]
% The data flow matrix $\Mtrix$ of a program $c$ is a matrix of size $|\lvar_c| \times |\lvar_c|$ 
% s.t.,
% %
% \[
% \Mtrix(i, j) \triangleq
% \left\{
% \begin{array}{ll}
% 1	&	\flowsto({x^i, y^j, c}) \\
% 0	& o.w.
% \end{array}
% \right., {x^i}, y^j  \in \lvar_c.
% \]
% %
% \end{defn}
% %
% Operations on the data flow matrices are defined as follows:
% %
% \begin{equation}
% \Mtrix_1 ; \Mtrix_2 
% := \Mtrix_2 \cdot \Mtrix_1 + \Mtrix_1 + \Mtrix_2
% \end{equation}
% %
% Consider the same program $c$ as above, its data flow matrix $\Mtrix$ and $\flag$ for the program $c$ is as follows:
% $$
% {c} = 
% \begin{array}{l}
% \left[{\assign {x_1} {\query(0)}}	\right]^1;
% \\
% \left[{\assign {x_2} {x_1 + 1}}		\right]^2;
% \\
% \left[{\assign {x_3} {x_2 + 2}}		\right]^3
% \end{array}
% ~~~~~~~~~~~~
% \Mtrix
% =  \left[ 
% \begin{matrix}
% 0 & 0 & 0 \\
% 1 & 0 & 0 \\
% 1 & 1 & 0 \\
% \end{matrix} \right] ~ , 
% \flag = \left [ \begin{matrix}
% 1 \\
% 0 \\
% 0 \\
% \end{matrix} \right ]
% $$
% %
% % There are two special matrices used for generating the data flow matrix $\Mtrix$ in the analysis algorithm. They are the left matrix $\lMtrix_i$ and right matrix $\mathsf{R_{(e, i)}}$.

% % Given a program  ${c}$ with its labelled variables $\lvar$ of length $N$,
% % the left matrix $\lMtrix_i$ generates a matrix of $1$ column, $N$ rows, 
% % where the $i$-th row is $1$ and all the other rows are $0$.
% % %
% % \begin{defn}[Left Matrix ($\lMtrix_i$)].
% % \\
% % Given a program  ${c}$ with its labelled variables $\lvar$ of size $N$, 
% % the left matrix $\lMtrix_i$ is defined as follows:
% % \[
% % \lMtrix_i(j) : = 
% % \left
% % \{
% % \begin{array}{ll}
% % 1 & j = i \\
% % 0 & o.w.
% % \end{array}
% % \right.,
% % j = 1, \ldots, N.
% % \]
% % \end{defn}
% % %
% % Given a program  ${c}$ with its labelled variables $\lvar$ of length $N$,
% % the right matrix $\rMtrix_{\expr, i}$ generates a matrix of one row and $N$ columns, 
% % where the locations of free variables in $\expr$ is marked as $1$. 
% % %
% % %
% % \begin{defn}[Right Matrix ($\rMtrix_{\expr}$)].
% % \\
% % Given a program  ${c}$ with its labelled variables $\lvar$ of length $N$, 
% % the right matrix $\rMtrix_{\expr}$ is defined as follows:
% % \[
% % \rMtrix_{\expr}(j) : = 
% % \left\{
% % \begin{array}{ll}
% % 1 & {x} \in FV(\expr) 
% % \\
% % 0 & o.w.
% % \end{array}
% % \right.,
% % {x} = \lvar(j) ~ , ~ j = 1, \ldots, N.
% % \]
% % %
% % %
% % \end{defn}
% % %
% % Using the same example program ${c}$ as above with labelled variables $\lvar = [ {x_1 , x_2 , x_3} ] $,
% % the left and right matrices w.r.t. its $2$-nd command 
% % $\left[{\assign {x_2} {x_1 + 1}}\right]^2$  are as follows:
% % \[
% % \lMtrix_1 = \left[ \begin{matrix}
% % 0   \\
% % 1 	 \\
% % 0   \\
% % \end{matrix}   \right ] 
% % ~~~~~~~~~~~~~~
% % \rMtrix_{{x}_1 + 1}
% % = \left[ \begin{matrix} 
% % 1 & 0 & 0 \\
% % \end{matrix}  \right]
% % \]
% %
% %
% %
% \subsection{ $\THESYSTEM$ Analysis Algorithm}
% \subsection{Dependency Graph Estimation}
\subsection{Vertices Estimation}
\label{sec:alg_vertexgen}
The vertices and query annotations in the execution based graph is built on static information, 
we reuse this information and construct the vertices and query annotations on 
this  {over-}approximate graph identical to the execution-based graph, specifically
%  in the last step of
% the algorithm in Section~\ref{}
as follows,
  \highlight{
\[
    \progV(c) \triangleq \left\{ 
  x^l \in \mathcal{LV} 
  ~ \middle\vert ~
  x^l \in \lvar_{c}
  \right\}
  \]
  }
% \wq{To do: Add $\THESYSTEM$, a data flow analysis algorithm to scan the program and give a graph.}
% {\THESYSTEM} consists of three phases: 
% \begin{enumerate}
%     \item Generating an abstract control flow graph with each edge representing an abstract event transiting between two command labels. 
%     \item Computing the value bound invariant for each variable in the event and 
%     the event transition closure over the abstract control flow graph,
%     we get the reachability bound for each labeled command.
%     \item Refining the abstract control flow graph with data-flow, by performing the reaching definition analysis, we generate a weighted data control flow graph.
%     \item An algorithm to find the appropriate path in the weighted data control flow graph
% \end{enumerate}

% \begin{enumerate}
%     \item An algorithm to generate a precise data control flow graph
%     \item An algorithm to perform a Reachability number analysis to calculate the weight of each node in the graph generated in phase 1.
%     \item An algorithm to find the appropriate path in the weighted data control flow graph
% \end{enumerate}
\subsection{Weight and Edge Estimation}
\label{sec:alg_weightedgegen}
The edge and weight in execution-based graph is built on all possible execution traces,
in order to approximate them statically, we show how to analyze the program to 
get an accurate approximation.

% This analysis first 
%  generate an abstract control flow graph
%  over all program labels, 
% in order to analyzing the data flow relations through variables assigned in every labeled command,
% and the reaching time of each variable.
% Then, it refines this control flow graph 
% % into a weighted data-dependency graph, 
% and generate the Program-Based Dependency Graph,
% through the data flow and reaching bound analysis results.
% In the last step, it finds the longest finite walk in this weighted data control flow graph w.r.t. the query variables,
% and return the number of query vertices traversed alongside.
% % \wq{To do: Add $\THESYSTEM$, a data flow analysis algorithm to scan the program and give a graph.}
% To be more specific, {\THESYSTEM} consists of five phases as follows,
% \\
% % \jl{Better to have a graph or picture of overview of the algorithm}
% \todo{graph}
% \todo{pass again}
This analysis
\begin{enumerate}
    % \item Generating 
    \item first generate 
    an abstract control flow graph over all labels, 
    % which are used as program's control locations,
    %
    \item then compute the weight of every vertex in $\progV(c)$ by computing a symbolic reachability bound for each label,
    % \\
    \item then estimate the edges between every vertex in $\progV(c)$ by computing the feasible data flow relation between every labeled variables.
  
\end{enumerate}
\subsubsection{Abstract Execution Control Flow graph}
\label{sec:abscfg}
We first show how to generate the abstract control flow graph in this part, in order to estimate 
the weight for vertex in $\progV(c)$ and edges between vertices in $\progV(c)$.
\\
In this abstract control flow graph, every vertex is a label,
 corresponding to a label command in the program.
Each directed 
 edge represents an abstract transition 
 between two control locations, i.e., the labels of two commands (we call the labels also control location and they refer to the same thing), 
 where the second labeled command will be executed after execution of the command with first label.
 The abstract transition contains a set of difference constraints for variables, generated by abstracting the command of the first labeled.
 We present the detail of program abstraction and the graph generation in following steps.
%
\paragraph*{Expression Abstraction}
We introduce the following notations and operations first
% an expression abstraction method based on the expression abstraction in paper \cite{sinn2017complexity}.
\\
% is enriched into $\constdom \triangleq \mathbb{N} \cup \inpvar \cup \{\max{(\dbdom)}\} $.
The Symbolic Constant:  $\constdom \triangleq \mathbb{N} \cup \inpvar \cup \{\max{(\dbdom)}\} $.
%  is enriched into $\constdom \triangleq \mathbb{N} \cup \inpvar \cup \{\max{(\dbdom)}\} $.
It consists of 
natural numbers $\mathbb{N}$,
% the symbolic constants
the program's input variables $\inpvar$ 
% (i.e., the set of the program's input variables), 
and a constant value $\max(\dbdom)$ for estimating the upper bound of variables which are
assigned by queries.
\\
Difference constraint: $DC : \mathcal{VAR} \cup \constdom \to \mathcal{\mathcal{VAR} \times (\mathcal{VAR} \cup \constdom) } \times (\mathbb{Z} \cup \{\infty\})$
 \\
$DC(\mathcal{VAR}  \cup \constdom) \cup \{\top\}$ represents all the difference constraints over the 
variable and symbolic constants.
% The difference constraint $DC$ over $\mathcal{VAR} \cup \constdom$ 
It is a set of the inequality of form $x \leq y + v$ where $x \in \mathcal{VAR} $, 
$y \in \mathcal{VAR}  \cup \constdom$ and $v \in \mathbb{Z}$. 
\\
This difference constraint is defined in the same way as
\cite{sinn2017complexity}. 
% represents the set of inequality over all $\mathcal{VAR}  \cup \constdom$. 
\\
% The symbolic constant is enriched into $\constdom \triangleq \mathbb{N} \cup \inpvar \cup \{\max{(\dbdom)}\} $.
% It consists of 
% natural number $\mathbb{N}$,
% the symbolic constants $\inpvar$ (i.e., the set of the program's input variables), 
% and a constant value $\max(\dbdom)$ for estimating the upper bound of variables which are
% assigned by queries.
% \\
% The symbolic constant is enriched into $\constdom \triangleq \mathbb{N} \cup \inpvar \cup \{\max{(\dbdom)}\} $.
% \\
For concise, we use $\dcdom^{\top}$ to represent the $DC(\mathcal{VAR}  \cup \constdom) \cup \{\top\}$ .
\\
% % $ \absdom: \mathcal{P}(DC(\mathcal{VAR}  \cup \constdom) \cup \{\top \})$:
% \\
% $\constdom: \mathbb{N} \cup \inpvar \cup \{\max{(\dbdom)}\} $ 
% The  constant 
\\
% % $DC(\mathcal{VAR}  \cup \constdom)$ represents the set of inequality over all $\mathcal{VAR}  \cup \constdom$.
% \\
Expression Abstraction: (adopted from the expression abstraction method in paper \cite{sinn2017complexity})
% We simplify the expression abstraction method from paper \cite{sinn2017complexity}, where 
$\absexpr : \expr \to \mathcal{VAR} \to DC(\mathcal{VAR}  \cup \constdom) \cup \{\top\} $,
computes the abstract value for each variable 
according to the expression $\expr$ assigned to it in the command.
% \[
%   \begin{array}{ll} 
%     \absexpr(y + c, x)  = x' \leq y + c  & c \in \mathbb{N} \land y \in (VAR \cup \constdom) \\
%     \absexpr(y - c, x)  = x' \leq y - c  & c \in \mathbb{N} \land y \in (VAR \cup \constdom) \\
%     \absexpr(v, x)  = x' \leq v + 0  & v \in (VAR \cup \constdom) \\
%     \absexpr(\aexpr, x) = x' \leq 0 + \infty   & \aexpr \text{ doesn't have any of the forms as above} \\
%     \absexpr(\qexpr, x)  = x' \leq 0 + \max(\dbdom) & \qexpr \text{ is a query expression}  \\
%     \absexpr(\bexpr, x) = x' \leq 0 + 1   & \bexpr \text{ is a boolean expression} \\
%   \end{array}
%   \]
  \[
    \begin{array}{ll} 
      \absexpr(x - v, x)  = x' \leq x - v  & x \in \grdvar \land v \in \mathbb{N} \\
      \absexpr(y + v, x)  = x' \leq y + v  & x \in \grdvar \land v \in \mathbb{Z} \land y \in (\grdvar \cup \constdom) \\
      \absexpr(v, x)  = x' \leq v + 0  & x \in \grdvar \land v \in (\grdvar \cup \constdom) \\
      \absexpr(y + v, x)  = x' \leq y + v & \\
      \grdvar = \grdvar \cup \{y\} & x \in \grdvar \land v \in \mathbb{Z} \land y \notin (\grdvar \cup \constdom)  \\
      \absexpr(\qexpr, x)  = x' \leq 0 + \max(\dbdom) & x \in \grdvar \land \qexpr \text{ is a query expression}  \\
      \absexpr(\bexpr, x) = x' \leq 0 + 1   & x \in \grdvar \land \bexpr \text{ is a boolean expression} \\
      \absexpr(\expr, x) = x' \leq \infty  &  x \in \grdvar \land \expr \text{ doesn't have any of the forms as above} \\
      \absexpr(\expr, x) = \top  &  x \notin \grdvar \\
    \end{array}
    \]
  In line:4 of case where variable in the $\grdvar$ is updated by a variable $y$ not in this set, we add $y$ into the set $\grdvar$ and repeat 
  above procedure  until $\grdvar$ and $\absexpr(\expr, x)$ is stabilized for every variable.
  \\
More specifically in 
% understanding the intuition, 
we handle a simplified normalized guard expression ($ x > 0$ for $x^l \in \lvar_c$)
 in $\ewhile$ and the same forms of the counter modification as in paper \cite{sinn2017complexity}.
\\
The counter variables only increase, decrease or reset by expression in the form of arithmetic minus and plus (able to extend to max and min.)
\\
For more complex expression assignments, where the counter reset, or calculated from $\elog$, 
multiplication or division, and expressions involving multiple variables, the constraint is approximated as reset of $\infty$.
\\
This simplification doesn't affect our analysis results in our examples. It is easy to extend the normalized expression 
into more complex forms as in \cite{sinn2017complexity}, as well as the 
counter variable manipulation with more advanced expressions as in \cite{}.
% \\ 
% The boolean expression in the guard of $\ewhile$ command is normalized into form of $ x > 0$ where $x^l \in \lvar_c$ for some $l$.
\paragraph{Program Event Abstraction}
We first compute the abstract initial and final state for program and then generate the abstract event for program in 
computing the abstract execution trace.
\\
The abstract initial state is a label from $\ldom$.
\\
The abstract final state is a pair from $\ldom \times \dcdom^{\top}$,  
where first component is a label from $\ldom$ and the second component is a constraint from $\dcdom^{\top}$.
%
\begin{defn}[Abstract Event]
  \label{def:abs_event}
  Abstract Event: 
  $\absevent \in $
  $\ldom \times \dcdom^{\top} \times \ldom$
  is a pair of abstract initial state and final state.
  % computed from program's abstract final and initial state, $\absfinal(c)$ and $\absinit(c)$ with formal definition, and algorithm detail in Appendix.
  \end{defn}
  The abstract event for program is generated when computing its abstract execution trace.
\\
%
Given a program $c$, its abstract initial state,
and the set of its abstract final state is computed as follows,
%
\[
  \begin{array}{ll}
    \absinit(\clabel{\assign{x}{\expr}}{}^l)  & = l  \\
    \absinit(\clabel{\assign{x}{\expr}}{}^l)  & = l \\
    \absinit(\clabel{\eskip}^{l})  & = l \\
    \absinit(\eif [b]^l \ethen c_1 \eelse c_2)  & = l \\
    \absinit(\ewhile [b]^l \edo c)  & = l \\
    \absinit(c_1 ; c_2)  & = \absinit(c_1) \\
 \end{array}
 \]
%
Final State Abstraction: 
$\absfinal: \cdom \to \mathcal{P}(\ldom \times \dcdom^{\top})$,
computes the set of Abstract Final State for the command. 
 \[
  \begin{array}{ll}
    \absfinal(\clabel{\assign{x}{\expr}}{}^l)  & = \{(l, \absexpr\eapp (\expr, x))\}  \\
     \absfinal(\clabel{\assign{x}{\query(\qexpr)}}{}^l)  & = \{
      (l, x' \leq 0 + \max(\dbdom) )\}  \\
     \absfinal(\clabel{\eskip}^{l})  
     & = \{(l, \top)\} \\
     \absfinal(\eif [b]^l \ethen c_1 \eelse c_2)  & = \absfinal(c_1) \cup \absfinal(c_2) \\
     \absfinal(\ewhile [b]^l \edo c)  & = \{(l, \top)\} \\
     \absfinal(c_1 ; c_2)  & =  \absfinal(c_2) \\
 \end{array}
 \]
 %
 \paragraph{Abstract Execution Trace}
 Then, we  extract the abstract execution trace  $\absflow(c)$ for a program, which computes the Abstract Execution Trace for program $c$, as a set of the abstract events $\absevent$.
 %
 \begin{defn}[Abstract Execution Trace]
 \label{def:abs_trace}
  $\absflow \in \cdom \to \mathcal{P}($Label $\times DC(\mathcal{VAR}  \cup \constdom) \cup \{\top\}) \times $ Label )
  \end{defn}
 %
 Abstract Execution Trace Computation:
  \\
  For simplicity, we use $\mathcal{P}(\absevent)$ represent the power set of all abstract events, and we have $\absflow(c) \in \mathcal{P}(\absevent)$.
 \\
 We first append a skip command with a symbolic label $l_e$, i.e., $\clabel{\eskip}^{l_e}$ at the end of the program $c$, and compute the $\absflow(c) = \absflow'(c')$ for $c'$, where $c' = c;\clabel{\eskip}^{l_e}$ as follows,
 %
 {\footnotesize
 \[
   \begin{array}{ll}
      \absflow'(\clabel{\assign{x}{\expr}}{}^l)  & = \emptyset  \\
      \absflow'(\clabel{\assign{x}{\query(\qexpr)}}{}^l)  & = \emptyset  \\
      \absflow'([\eskip]^{l})  & = \emptyset \\
      \absflow'(\eif [b]^l \ethen c_t \eelse c_f)  & =  \absflow'(c_t) \cup \absflow'(c_f)
      %   \\ & \quad 
        \cup \{(l, \top,  \absinit(c_t) ) ,  (l, \top, \absinit(c_f)) \} \\
       \absflow'(\ewhile [b]^l \edo c_w)  & =  \absflow'(c_w) \cup \{(l, \top, \absinit(c_w)) \} 
      %  \\ & \quad 
       \cup \{(l', dc, l)| (l', dc) \in \absfinal(c_w) \} \\
       \absflow'(c_1 ; c_2)  & = \absflow'(c_1) \cup  \absflow'(c_2) 
      %  \\ & \quad 
       \cup \{ (l, dc, \absinit(c_2)) | (l, dc) \in \absfinal(c_1) \} \\
   \end{array}
   \]
   }

   The $\absflow'([x := \expr]^{l})$, $\absflow'([x := \query(\qexpr)]^{l})$ and $\absflow'([\eskip]^{l})$ are all empty set. 
   For every event $\event$ with label $l$ in an execution trace $\trace$ of program $c$, 
   there is an abstract event in program's abstract execution trace of form $(l, \_, \_)$. 
   \begin{lem}[Soundness of the Abstract Execution Trace]
     \label{lem:abscfg_sound}
   Given a program ${c}$, we have:
   %
   \[
     \begin{array}{l}
       \forall \vtrace_0, \trace \in \mathcal{T} ,  \event = (\_, l, \_) \in \eventset \st
   \config{{c}, \trace_0} \to^{*} \config{\eskip, \trace_0 \tracecat \vtrace} 
   \land \event \in \trace 
   \\
   \qquad \implies \exists \absevent = (l, \_, \_) \in (\ldom\times \dcdom^{\top} \times \ldom) \st 
   \absevent \in \absflow(c)
   \end{array}
   \]
   \end{lem}
%    This lemma is proved formally in Appendix~\ref{apdx:reachability_soundness}.
% For every event $\event$ with label $l$ in an execution trace $\trace$ of program $c$, 
% there is an abstract event in program's abstract execution trace of form $(l, \_, \_)$. 
This lemma is proved formally in Lemma~\ref{lem:abscfg_sound} in Appendix~\ref{apdx:reachability_soundness}.
\\
For every labeled variable in program $c$, $x^l \in \lvar_c$, there is a unique abstract event in program's abstract execution trace $\absevent \in \absflow(c)$ of form $(l, \_, \_)$. 
\begin{lem}[Uniqueness of the Abstract Execution Trace]
  \label{lem:abscfg_unique}
Given a program ${c}$, we have:
%
\[
  \begin{array}{l}
    \forall \vtrace_0, \trace \in \mathcal{T} ,  \event = (\_, l, \_, \_) \in \eventset^{\asn} \st
\config{{c}, \trace_0} \to^{*} \config{\eskip, \trace_0 \tracecat \vtrace} 
\land \event \in \trace 
\\
\qquad \implies \exists! \absevent = (l, \_, \_) \in (\ldom\times \dcdom^{\top} \times \ldom) \st 
\absevent \in \absflow(c)
\end{array}
\]
\end{lem}
This lemma and proof is also 
formalized in Lemma~\ref{lem:absevent_unique} in Appendix~\ref{apdx:reachability_soundness}.

% We have a pre-processing algorithm to go through the programs and returns the list of labels associating with a loop and whose visiting times need to be analyzed.
%
\paragraph{Abstract Control Flow Graph} 
Through a program $c$'s abstract execution trace, its abstract control flow graph is computed in 
Definition~\ref{def:abs_cfg}.
% Given program $c$ with its abstract control flow $\absflow(c)$, the Abstract Control Flow Graph:
% \\
\begin{defn}[Abstract Control Flow Graph]
\label{def:abs_cfg}
Given a program $c$, 
with its abstract control flow $\absflow(c)$
its abstract control flow graph $\absG(c) =(\absV, \absE, \absW)$ is defined as follows,
\\
% \highlight{
% :
%
% \\
$\absE = \{(l_1, dc, l_2) | (l_1, dc, l_2) \in \absflow(c)\}$,
\\
$\absV = labels(c)\cup\{l_e\}$
\\
 $\absW 
\triangleq \left\{ (l, w) \in \mathbb{L} \times EXPR(\constdom) \right\}$.
% }
\\
, where the weight of every label to be computed in the next step.
\end{defn}
% 
The vertices in this graph are program's labels with an exit label $l_{ex}$.
Each directed 
 edge $(l_1, dc, l_2)$ from $l_1$ to $l_2$,
 represents an abstract transition 
 between two control locations with a set of difference constraints on it.
%  , i.e., the labels of two commands (we call the labels also control location and they refer to the same thing), 
%  where 
In this transition, the  command labeled with the second location $l_2$, 
 will be executed after execution of the command with label $l_1$,
%  The abstract transition contains a set of difference constraints for variables, 
with the difference constraints generated by abstracting the command of the first label.
% \\
% It is easy to show for every $(l_1, dc, l_2) \in \absflow(c)$ such that $l_2 \neq l_e$, $(l_1, l_2) \in flow(c)$. The formal Lemma and proof can be found in Lemma~\ref{lem:flow_to_absflow} in Appendix~\ref{apdx:reachability_soundness}.
The weight for every vertex is a symbolic expression over the symbolic constant, 
which is the estimated upper bound on the number of visiting time for every control location
through the reachability bound analysis as follows.
%
% It is easy to show for every $(l_1, dc, l_2) \in \absflow(c)$ such that $l_2 \neq l_e$, $(l_1, l_2) \in flow(c)$. The formal Lemma and proof can be found in Lemma~\ref{lem:flow_to_absflow} in Appendix~\ref{apdx:reachability_soundness}.
%
\subsubsection{Weight Estimation}
\label{sec:alg_weightgen}
%
In order to estimate weight for every vertex in $\progV(c)$,
we first show how to compute the reachability bound for every label in $c$,
then show how to compute the weight for every vertex in $\progV(c)$.
\\
we infer the invariant for every variable, and compute the transition closure for every abstract transition. By solving the closure
with the invariants of variables involved in this closure for every transition, we compute
the symbolic reachability bound of every commands corresponding to this transition.
\\
We present the details of invariant, closure generation, and reachability bound computation as follows.
%
\paragraph*{Variable Modification Tracking}
Identify the abstract events where each variable is increased, decreased and reset:
\\
$\inc: \mathcal{VAR} \to \mathcal{P}(\absevent) $
the set of the abstract events where the variable increase.
\\
$\inc(x) = \{(\absevent, c) | \absevent = (l, l', x' \leq x + v)\}$
\\
$\reset: \mathcal{VAR} \to \mathcal{P}(\absevent) $
The set of the abstract events where the variable is reset.
\\
$\dec: \mathcal{VAR} \to \mathcal{P}(\absevent) $
The set of abstract events where the variable decrease.
% \\
% $\dec(x) = \{(\absevent, c) | \absevent = (l, l', x' \leq x - v)\}$
\\
$Incr(v) \triangleq \sum\limits_{(\absevent, c) \in \inc(v)}\{\absclr(\absevent) \times v\}$
%
\paragraph*{Local Bounds}
Given a program $c$ with its abstract control flow graph 
$\absG(c) = (\absV, \absE)$
\\
Local Bounds Computation:
$\locbound: \absevent \to \mathcal{VAR} \cup \constdom$.
%
\[ 
\begin{array}{ll}
  \locbound(\absevent) \triangleq 1 
  & \absevent \notin SCC(\absG(c))
  \\
  \locbound(\absevent) \triangleq (x, v) 
  & \absevent \in SCC(\absG(c)) \land \absevent \in \dec(x) \land  \absevent = (\_, \_ , x' \leq x - v) \\
  \locbound(\absevent) \triangleq (x, \max(\dec(x))) 
  & \absevent \in SCC(\absG(c)) \land 
  \absevent  \notin \bigcup_{x \in \mathcal{VAR}} \dec(x)
  \land \absevent \notin SCC(\absG(c) \setminus \dec(x)) 
\end{array}
  \]
  The first case is straightforward. Since variable's visiting time outside of any while loop is at most 1, we do not need to analyze the visiting times of every node in the graph from phase 1.
  The second and third step is guaranteed by the \emph{Discussion on Soundness} in Section 4 of \cite{sinn2017complexity}.
  Then soundness proof is in Lemma~\ref{lem:local_bound_sound} in Appendix~\ref{apdx:reachability_soundness}.
%
\paragraph*{Abstract Event Closure and Bound Invariant}
Then, computing the bound invariants for variables and the transition closures for abstract events:
\\ 
$ \varinvar: \mathcal{VAR} \cup \constdom \to EXPR(\constdom)$
\\
$\absclr: \absevent \to EXPR(\constdom)$
\\
Then, the symbolic invariant for each variable 
as well as the symbolic transition closure for each transition is calculated as follows:
\[ 
\begin{array}{lll}
  \varinvar(x) & \triangleq c & c \in \constdom \\
  \varinvar(x) & \triangleq Incr(v) + \max(\{\varinvar(a) + c | (t, a, c) \in \reset(x)\}) & c \notin \constdom
\end{array}
\]
%
\begin{defn}
  \label{def:transition_closure_base}
\[ 
\begin{array}{lll}
  \absclr(\absevent) 
  & \triangleq x / v & \\ 
  & \locbound(\absevent) = (x, v) \in \constdom \times \mathbb{N} & \\
  \absclr(\absevent) 
  & \triangleq (Incr(x) + 
  \sum\limits_{(\absevent', y, v') \in \reset(x)}
  \absclr(\absevent') \times \max(\varinvar(y) + v', 0) ) / v & \\
  & \locbound(\absevent) = (x, v) \land x \notin \constdom & 
\end{array}
  \]
\end{defn}
%
\paragraph*{Improved Variable Modification Tracking}
Instead of just identifying the abstract events where each variable is reset,
this improvement identifies the chain of the events where a given variable is reset by the 
variables of the abstract events through the chain.
\\
$\resetchain: \mathcal{VAR} \to \mathcal{P}(\mathcal{P}(\absevent)) $
The set of the chain of abstract events where the variable is reset through the chain.
% \\
% $Incr(v) \triangleq \sum\limits_{(\absevent, c) \in \inc(v)}\{\absclr(\absevent) \times v\}$
%
\paragraph*{Improved Bound Invariant Computation}
Then, computing the bound invariants for variables and the transition closures for abstract events:
\\ 
$ \varinvar: \mathcal{VAR} \cup \constdom \to EXPR(\constdom)$
\\
$\absclr: \absevent \to EXPR(\constdom)$
\\
Then, the symbolic invariant for each variable 
as well as the symbolic transition closure for each transition is calculated as follows:
\[ 
\begin{array}{lll}
  \varinvar(x) & \triangleq c & c \in \constdom \\
  \varinvar(x) & \triangleq Incr(v) + \max(\{\varinvar(a) + c | (t, a, c) \in \reset(x)\}) & c \notin \constdom
\end{array}
\]
%
\begin{defn}
  \label{def:transition_closure}
\[ 
\begin{array}{lll}
  \absclr(\absevent) 
  & \triangleq x / v & \\ 
  & \locbound(\absevent) = (x, v) \in \constdom \times \mathbb{N} & \\
  \absclr(\absevent) 
  & \triangleq \Big(
    \sum\limits_{y \in \{ y ~|~ 
    ch \in \resetchain(x), (l_1, x, y, v, l_2) \in ch \} } Incr(x) & \\
    & \quad + 
  \sum\limits_{ch \in \resetchain(x)}
  \big( \min\limits_{\absevent' \in ch}({\absclr(\absevent')}) \times 
  \max(\varinvar(y) + \sum\limits_{(l_1, x, y, v, l_2) \in ch } v, 0)\big) \Big) / v & \\
  & \locbound(\absevent) = (x, v) \land x \notin \constdom & 
\end{array}
  \]
\end{defn}
  %
% \paragraph*{Adding the Reachability Bounds for Every Vertex in the Data-Control Flow Graph}
% Updating the weight of every vertex in the $\progG(c) = (\progV, \progE, \progW, \progF)$ for program $c$ generated from phase 1. 
% For every $x^l \in \progV$, find the abstract event $\absevent \in \absflow(c)$ of the form $(l, \_, \_)$, updating the $\progW(x^l) $ by the transition closure of this event.
% \\
$
\progW(x^l) 
  \triangleq \absclr(\absevent)
$
\paragraph*{Reachability Bound for Every program Label}
Through the transition closure computed above, 
The weight of every label in 
% Then we update 
the program $c$'s abstract control flow graph,
$\absG(c) =(\absV, \absE, \absW)$
is 
computed as the maximum over all the abstract events $\absevent \in \absE$ heading out from this vertex, formally as follows.
% by annotating each vertex with a symbolic weight. 
% This weight corresponds to 
%reachability bounds of
\\
$\absW 
\triangleq \left\{ (l, w) \in \mathbb{N} \times EXPR(\constdom) | w = \max\limits_{\absevent = (l, \_, \_)} \{ \absclr(\absevent)\} \right\}$.
% \\
\paragraph{Vertex Weight Computation}
Then compute the weight for each vertex in $\progV$ as follows,
\highlight{
% :
% \\
 \[\progW(c) \triangleq
   \left\{ (x^l, w) \in  \mathcal{LV} \times EXPR(\constdom)
\mid
x^l \in \progV(c) \land w = \absW(l)
\right\}
\]
}
%
\subsubsection{Edge Estimation}
\label{sec:alg_edgegen}
We show how to estimation the edge in this part.
\\
In this step,
 $\THESYSTEM$ performs a feasible data-flow analysis 
 through the reachable definition algorithm,
%  and then Then we generated the set of feasible data-flow between labeled variables based on that.
and generates the set of feasible data-flow between labeled variables.
\\
%  By generating set of all the reachable variables at location of label $l$ in the program $c$.
% For every labelled variable $x^l$ in this set, 
% the value assigned to that variable
% in the assignment command associated to that label is reachable at the entry point of  executing the command of label $l$.
% \\
In the first step, 
it performs the standard reaching definition analysis given a program $c$, on its every label $l$.  This step generates set of all the reachable variables at location of label $l$ in the program $c$.
The $\live(l, c)$ represent the analysis result, which is the set of 
reachable labeled variables in program $c$ at the location of label $l$.
For every labelled variable $x^l$ in this set, 
the value assigned to that variable
in the assignment command associated to that label is reachable at the entry point of  executing the command of label $l$.
% \\
% it performs the standard reaching definition analysis given a program $c$, on its every label $l$.
% \\
% Another operator \mathsf{blocks} 
The block, 
is either the command of the form of assignment, skip, or a test of the form of $[b]^{l}$, 
% and $block$ of program $c$ is 
denoted by $\mathsf{blocks}(c)$
the set of all the blocks 
in program $c$, where  $\mathsf{blocks}: \cdom \to \mathcal{P}(\cdom \cup \clabel{\bexpr}^{l})$.
Then it generates the set of feasible data-flow between labeled variables with detail in Definition~\ref{def:feasible_flowsto}, 
based on $\live(l, c)$ for every label in a program $c$ and its blocks $\mathsf{blocks}$.
\\
The details are as follows.
%
% Performing a feasible data-flow analysis through the reachable definition algorithm. 
%  By generating set of all the reachable variables at location of label $l$ in the program $c$.
% For every labelled variable $x^l$ in this set, 
% the value assigned to that variable
% in the assignment command associated to that label is reachable at the entry point of  executing the command of label $l$.
% \paragraph{Generate CFG}
%  \begin{def}
%   \label{def:init_label}
%   Define $\mathsf{init}$: Command -> label, which returns the initial label of the statement. 
% \[
%  \begin{array}{ll}
%     init([x := e]^{l})  & = l  \\
%      init([x := q(e)]^{l})  & = l \\
%      init([skip]^{l})  & = l \\
%      init([if [b]^l then C_1 else C_2]^{l})  & = l \\
%      init([while [b]^l do C]^{l})  & = l \\
%      init(C_1 ; C_2)  & = init(C_1) \\
%  \end{array}
%  \]
% \end{def}
%   Define $\mathsf{final}$: Command -> Powerset(label), which returns the final labels of the statement. 
%  \[
%  \begin{array}{ll}
%     final([x := e]^{l})  & = \{l\}  \\
%      final([x := q(e)]^{l})  & = \{l\}  \\
%      final([skip]^{l})  & = \{l\} \\
%      final([if [b]^l then C_1 else C_2]^{l})  & = final(C_1) \cup final(C_2) \\
%      final([while [b]^l do C]^{l})  & = \{l\} \footnote{while terminates after b evaluates to false} \\
%      final(C_1 ; C_2)  & =  final(C_2) \\
%  \end{array}
%  \]
\paragraph*{Blocks and Defs}
 Define block B to be either the command of the form of assignment, skip, or test of the form of $[b]^{l}$.\\
 Define $\mathsf{blocks}$ : command -> Powerset(Block)
 \[
 \begin{array}{ll}
    blocks([x := e]^{l})  & = \{[x := e]^{l}\}  \\
     block([x := q(e)]^{l})  & = \{[x := q(e)]^{l}\}  \\
     blocks([skip]^{l})  & = \{[skip]^{l}\} \\
     blocks([if [b]^l then C_1 else C_2]^{l})  & = {[b]^{l}} \cup blocks(C_1) \cup blocks(C_2) \\
     blocks([while [b]^l do C]^{l})  & = \{[b]^{l}\} \cup blocks(C) \\
     blocks(C_1 ; C_2)  & = blocks(C_1) \cup  blocks(C_2) \\
 \end{array}
 \]
 Define $\mathsf{labels}$ to get the labels of blocks.
 \[
   labels(C) = \{l | [B]^{l} \in blocks(C) \}
 \]  

% The control flow graph is generated by edges between labels. Define $\mathsf{flow}$: command -> P (label $\times$ label ).

% \[
%  \begin{array}{ll}
%     flow([x := e]^{l})  & = \emptyset  \\
%      flow([x := q(e)]^{l})  & = \emptyset  \\
%      flow([skip]^{l})  & = \emptyset \\
%      flow([if [b]^l then C_1 else C_2)  & =  flow(C_1) \cup flow(C_2)\cup \{(l, init(C_1)) , (l, init(C_2)) \} \\
%      flow([while [b]^l do C)  & =  flow(C) \cup \{(l, init(C)) \} \cup \{(l', l)| l' \in final(C) \} \\
%      flow(C_1 ; C_2)  & = flow(C_1) \cup  flow(C_2) \cup \{ (l,init(C_2)) | l \in final(C_1) \} \\
%  \end{array}
%  \]
 
 \paragraph{Reaching definition analysis}
 Set $?$ to be undefined, $label^{?}$ is label $\cup \{?\}$.\\
 Define $\mathsf{kill}$: $blocks \to \mathcal{P}(\mathcal{VAR} \times LABEL \cup \{?\})$, which produces the set of labelled variables of assignment destroyed by the block.
 \\
  Define $\mathsf{gen}$: $blocks \to \mathcal{P}(\mathcal{VAR} \times LABEL \cup \{?\})$, which generates the set of labelled variables generated by the block.
  \\\
  Define $defs(x)(c): \mathcal{VAR} \to LABEL$, gives all the labels where assigns value to variable x in the target program $c$. \[
 \begin{array}{ll}
    kill([x := e]^{l})  & = \{ (x, ?) \} \cup \{ (x, l') | l' \in defs(x) \} \\
     kill([x := q(e)]^{l})  & = \{ (x, ?) \} \cup \{ (x, l') | l' \in defs(x) \}  \\
     kill([skip]^{l})  & = \emptyset \\
     kill([ [b]^l ]^{l})  & =  \emptyset
 \end{array}
 ~~
  \begin{array}{ll}
      gen([x := e]^{l})  & = \{ (x, l) \}  \\
     gen([x := q(e)]^{l})  & = \{ (x, l) \}  \\
     gen([skip]^{l})  & = \emptyset \\
     gen([ [b]^l ]^{l})  & =  \emptyset 
 \end{array}
 \]
 Define $in(l)$, $out(l)$: LABEL$ \to \mathcal{VAR} \times LABEL \cup \{?\}$ for every block in program $c$ is computed as follows,
 \[
 \begin{array}{lll}
    in(l)  & = \{ (x, ?) | x^l \in \lvar_c \land  l = \absinit(c) \}  
    \cup \{ out(l')|  | (l',\_, l) \in \absE \land  l \neq \absinit(c)\}  \\
     out(l)  & =  gen(B^{l}) \cup \{ in(l) \setminus kill(B^l)  \} & B^l \in blocks(c)   
 \end{array}
 \]
 computing $in(l)$ and $out(l)$ for every $B^l \in blocks(c) $, and repeating these two step
until the $in(l)$ and $out(l)$ are stabilized for every $B^l \in blocks(c) $
We use $\live_{in}(l,c)$ and $\live_{out}(l, c)$ denote the stabilized results for the command of label $l$ in program $c$. 
\\
The $\live_{in}(l,c)$ and $\live_{out}(l, c)$ is computed by the Standard worklist algorithm with detail as below.
\begin{enumerate}
    \item initial in[l]=out[l]=$\emptyset$
    \item initial in[entry label] = $\emptyset$
    \item initialize a work queue, contains all the blocks in C
    \item while |W| != 0 \\
         pop l in W\\
          old = out[l]\\
          in(l) =  out(l') where (l',l) in flow(C)\\
           out(l) = gen($b^l$) $\cup$ (in(l) - kill($b^l$) ) where $b^l$ in block(C)   \\
          if (old != out(l)) W= W $\cup$ \{l'| (l,l') in flow(C)\}\\
          end while
\end{enumerate}
%
computing $in(l)$ and $out(l)$ for every $B^l \in blocks(c) $, and repeating these two step
until the $in(l)$ and $out(l)$ are stabilized for every $B^l \in blocks(c) $
We use $\live_{in}(l,c)$ and $\live_{out}(l, c)$ denote the stabilized results for the command of label $l$ in program $c$. 
The $\live_{in}(l,c)$ and $\live_{out}(l, c)$ is computed by the Standard worklist algorithm. (For simplicity, we use $\live(l,c)$ to represent $\live_{in}(l,c)$ in the other part of the paper.
%%
\paragraph{Feasible Data-Flow Generation}
by using the results of Reaching definition analysis results, specifically $\live(l, c)$ for every label in a program $c$, we refine the vertices and edges in the $\absG$ graph 
by generating the set of feasible data-flow between labeled variables as follows,
%
%   \[
%  \begin{array}{ll}
%     dcdg([x := e]^{l})  & = \{ (y^i, x^l) | y \in VAR(e) \land (y,i) \in \live_{in}(l) \}  \\
%      dcdg([x := q(e)]^{l})  & = \{ (y^i, x^l) | y \in VAR(e) \land (y,i) \in \live_{in}(l) \}  \\
%      dcdg([skip]^{l})  & = \emptyset \\
%      dcdg([if [b]^l then C_1 else C_2)  & =  dcdg(c_1) \cup dcdg(c_2)\\ & \cup \{(x^i,y^j) | x \in VAR(b) \land (x,i) \in \live_{in}(l) \land ([y = \_]^j) \in blocks(c_1) \} \\
%      &\cup \{(x^i,y^j) | x \in VAR(b) \land (x,i) \in \live_{in}(l) \land ([y = \_]^j) \in blocks(c_2) \} \\
%      dcdg([while [b]^l do c)  & =  dcdg(c) \cup \{(x^i,y^j) | x \in VAR(b) \land (x,i) \in \live_{in}(l) \land ([y = \_]^j) \in blocks(C) \} \\
%      dcdg(c_1 ;c_2)  & = dcdg(c_1) \cup  dcdg(c_2) \\
%  \end{array}
%  \]
%
\begin{defn}[Feasible Data-Flow]
  \label{def:feasible_flowsto}
  Given a program $c$ and two labeled variables $x^i, y^j$  in this program, 
  $\flowsto(x^i, y^j, c)$ is 
    {\footnotesize
    \[
   \begin{array}{ll}
    \flowsto(x^i, y^j, \clabel{\assign{x}{\expr}}{}^l)  & \triangleq (x^i, y^j) \in \{ (y^i, x^l) | y \in FV(e) \land (y,i) \in \live_{in}(l, \clabel{\assign{x}{\expr}}^l) \}  \\
    \flowsto(x^i, y^j, \clabel{\assign{x}{\query(\qexpr)}}{}^l)  & \triangleq (x^i, y^j) \in \{ (y^i, x^l) | y \in FV(e) \land (y,i) \in \live_{in}(l,\clabel{\assign{x}{\query(\qexpr)}}^l) \}  \\
    \flowsto(x^i, y^j, [\eskip]^{l})  & = \emptyset \\
    \flowsto(x^i, y^j, \eif [b]^l \ethen c_1 \eelse c_2)  & \triangleq \flowsto(x^i, y^j, c_1) \lor \flowsto(x^i, y^j, c_2) \\ 
        & \lor (x^i, y^j) \in
       \{(x^i,y^j) | x \in FV(b) \land (x,i) \in \live_{in}(l) \land ([y = \_]^j) \in blocks(c_1) \} \\
       &\lor (x^i, y^j) \in \{(x^i,y^j) | x \in FV(b) \land (x,i) \in \live_{in}(l) \land ([y = \_]^j) \in blocks(c_2) \} \\
       \flowsto(x^i, y^j, \ewhile [b]^l \edo c_w)  & \triangleq  \flowsto(x^i, y^j, c_w) \lor (x^i, y^j) \in  \{(x^i,y^j) | x \in FV(b) \land (x,i) \in \live_{in}(l) \land ([y = \_]^j) \in blocks(c_w) \} \\
       \flowsto(x^i, y^j, c_1 ;c_2)  & \triangleq \flowsto(x^i, y^j, c_1) \lor \flowsto(x^i, y^j, c_2) \\
   \end{array}
   \]
   }
   \end{defn}
%
\paragraph*{Edges Generation}
\highlight{
  \[
    \progE(c) \triangleq 
    \left\{ 
    ({x}_1^{i}, {x}_2^{j}) \in \mathcal{LV} \times \mathcal{LV}
    ~ \middle\vert ~
    \begin{array}{l}
      {x}_1^{i}, {x}_2^{j} \in \progV(c)
    \land
      % \\
      \exists n \in \mathbb{N}, z_1^{r_1}, \cdots, z_n^{r_n} \in \lvar_{{c}} \st 
      n \geq 0 \land
      \\
      \flowsto(x^i,  z_1^{r_1}, c) 
      \land \cdots \land \flowsto(z_n^{r_n}, y^j, c) 
    \end{array}
    \right\}
    \]
}
%  \begin{defn}[Feasible Data-Flow]
%   \label{def:feasible_flowsto}
%     {\footnotesize
%     \[
%    \begin{array}{ll}
%       dcdg(\clabel{\assign{x}{\expr}}{}^l)  & = \{ (y^i, x^l) | y \in FV(e) \land (y,i) \in \live_{in}(l, \clabel{\assign{x}{\expr}}^l) \}  \\
%        dcdg(\clabel{\assign{x}{\query(\qexpr)}}{}^l)  & = \{ (y^i, x^l) | y \in FV(e) \land (y,i) \in \live_{in}(l,\clabel{\assign{x}{\query(\qexpr)}}^l) \}  \\
%        dcdg([\eskip]^{l})  & = \emptyset \\
%        dcdg([\eif [b]^l \ethen c_1 \eelse c_2)  & =  dcdg(c_1) \cup dcdg(c_2)\\ & \cup 
%        \{(x^i,y^j) | x \in FV(b) \land (x,i) \in \live_{in}(l) \land ([y = \_]^j) \in blocks(c_1) \} \\
%        &\cup \{(x^i,y^j) | x \in FV(b) \land (x,i) \in \live_{in}(l) \land ([y = \_]^j) \in blocks(c_2) \} \\
%        dcdg([\ewhile [b]^l \edo c)  & =  dcdg(c) \cup \{(x^i,y^j) | x \in FV(b) \land (x,i) \in \live_{in}(l) \land ([y = \_]^j) \in blocks(C) \} \\
%        dcdg(c_1 ;c_2)  & = dcdg(c_1) \cup  dcdg(c_2) \\
%    \end{array}
%    \]
%    }
%    \end{defn}
%    For any two labeled variables $x^i, y^j$ in a program $c$, 
%   %  it is easy to see that there is a one-on-one correspondence between 
%   %  $\flowsto$ relation of the two variables, and the $dcdg$ analysis result on $c$.
%   we use $\flowsto()$ denote if they have a feasible data-flow relation in Definition~\ref{def:flowsto}.
%    \begin{defn}[Feasible Data-Flow ($\flowsto$)]
%    \label{def:flowsto}
%    \[
%    \forall c \in \cdom, x^i, y^j \in \lvar_c \st 
%    \flowsto(x^i, y^j, c) \iff (x^i, y^j) \in dcdg(c)
%    \]
%    \end{defn}
  %  This soundness is proved in Proof~\ref{pf:rd_soundness} in Appendix~\ref{apdx:rd_soundness}.
  %  For any two labeled variables in a program $c$, it is easy to see that there is a one-on-one correspondence between 
  %  $\flowsto$ relation of the two variables, and the $dcdg$ analysis result on $c$.
  %  \begin{thm}[Soundness of the Feasible Data-Flow Analysis]
  %  \label{thm:rd_soundness}
  %  \[
  %  \forall c \in \cdom, x^i, y^j \in \lvar_c \st 
  %  \flowsto(x^i, y^j, c) \iff (x^i, y^j) \in dcdg(c)
  %  \]
  %  \end{thm}
  %  This soundness is proved in Proof~\ref{pf:rd_soundness} in Appendix~\ref{apdx:rd_soundness}.
   \subsection{Program-Based Data Dependency Graph Generation}
  %  Weighted Data Dependency Graph Generation}
   \label{sec:alg_graphgen}
   %
%    Each directed edge represents an abstract transition 
%    between two control locations, i.e., the labels of two commands (we call the labels also control location and they refer to the same thing in the follows), 
%    where the second labeled command will be executed after execution of the command with first label.
%    The abstract transition contains a set of difference constraints for variables, generated by abstracting the command of the first label.
%   \item Computing 
%   % we get the reachability bound for each command.
%   the symbolic reachability bound for each command,
%   % the value bound invariant for each variable in the event and 
%   by inferring the value bound invariant for each variable 
%   % the event transition closure over the abstract control flow graph,
%   and the transition closure for every abstract transition through the constraints over the abstract control flow graph.
%   % \\
%   % Through this graph and constraint for every transition, we infer the  invariant for every variable,
%   % and compute the transition closure for every abstract transition.
%   % By solving the closure with the invariants of variables involved in this closure for every transition, 
%   % we compute the symbolic reachability bound of every commands corresponding to 
% %     % this transition.
% %     \item Performing a feasible data-flow analysis from the reachable definition algorithm. 
% % %  By generating set of all the reachable variables at location of label $l$ in the program $c$.
% % and generating the set of all the reachable variables for every program location.
% % For every labelled variable $x^l$ in this set, 
% % the value assigned to that variable
% % in the assignment command associated to that label is reachable at the entry point of  executing the command of label $l$.
% % \item Refining the abstract control flow graph into a weighted-data dependency graph, 
% % by annotating each vertex with reachability bounds and 
% % removing unfeasible edges and redundant edges and vertices.
% % adding edges between
% %     variables having data-flow relations, and
% % removing the edges between locations where the variables associated to that labeled command isn't reachable from the second location.
% % \\
% % first annotate each vertex of label $l$ with the variable 
% % assigned in that labeled command, and remove the rest doesn't correspond to an assignment command.
% % Then 
% % add direct edge between two labeled variables,
% % where the first variable 
% % is directly used in the assignment expression to the second variable, by restricting 
% % the first labeled variable is reachable at the the second label.
% %
% \item Computing the adaptivity through this weighted data dependency graph,
%   by finding a finite walk on this weighted graph, 
% traversing the maximum times of query variables, by restricting the visiting time of every vertex on this walk to its weight.
% The maximum number of vertices corresponding to a query variables visited on this walk is the estimated upper bound, for program's adaptivity.

%    In this step, $\THESYSTEM$ refines the abstract control flow graph into the program-based weighted-data dependency graph, 
% by annotating each vertex with reachability bounds and 
% removing unfeasible edges and redundant edges and vertices,
% % This graph is used 
% for approximating the trace-based weight-data dependency graph.
% \\
% Specifically, we first annotate each vertex of label $l$ with the variable 
% assigned in that labeled command, and remove the rest doesn't correspond to an assignment command.
% Then 
% add direct edge between two labeled variables,
% where the first variable 
% is directly used in the assignment expression to the second variable, by restricting 
% the first labeled variable is reachable at the second label.
% % \\
% The formal definition is as follows.
Finally we build the estimated data dependency graph based on program static analysis as follows:
\\
\highlight{
  \[
    \progG(c) = (\progV(c), \progE(c), \progW(c), \progF(c))
    \]
}
with $\progV(c), \progE(c), \progW(c)$ as computed in each steps above,\\
and $\progF(c) =\left\{(x^l, n)  \in  \mathcal{LV} \times \{0, 1\} 
~ \middle\vert ~
x^l \in \lvar_{c},
n = 1 \iff x^l \in \qvar_{c} \land n = 0 \iff  x^l \in \qvar_{c} .
\right\} $,
% The algorithm computation is 
formally as follows,
% Through the reachable definition set on every label,
% we remove the edges between labels where the variables associated to that labeled command isn't reachable from the second location.
%\absG(c) =(\absV, \absE, \absW)
\begin{defn}
  [Program-Based Dependency Graph]
  \label{def:prog_graph}
  % [Program-Based Weighted Data Dependency Graph Generation Algorithm]
% \label{def:analyz_dcfg}
Given a program $c$, with its abstract weighted control flow graph $\absG(c) = (\absV, \absE, \absW)$ and 
feasible data flow relation $\flowsto(x^i, y^j, c)$ for every $x^i, y^j \in \lvar_c$, its Program-Based Weighted Data Dependency Graph
$\progG(c) = (\progV, \progE, \progW, \progF)$,
is generated as follows,
% \\
% \highlight{
% $\progV =\{x^l | x^l \in \lvar_c\} $
% \\
% $\progE = \{(y^i, x^l) | (y^i, x^l)  \in dcdg(c) \}$
% \\
% $\progW = \{(x^l, w ) | (l, w ) \in \absW \land x^l \in \lvar_c\}$
% \\
% $\progF = \{(l, q) \in \mathcal{L} \times \{0, 1\}| q = 1 \iff l \in \qvar_c, q = 0 \iff l \notin \qvar_c \}$.
% }
% \end{defn}
% \begin{defn}
  % [Program-Based Dependency Graph].
  % \label{def:prog_graph}
%   % \\
% Given a program ${c}$
% its program-based graph 
% $\progG({c}) = (\vertxs, \edges, \weights, \qflag)$ is defined as:
{\footnotesize
\[
\begin{array}{rlcl}
\text{Vertices} &
\progV & := & \left\{ 
x^l \in \mathcal{LV} 
~ \middle\vert ~
x^l \in \lvar_{c}
\right\}
\\
\text{Directed Edges} &
\progE & := & 
\left\{ 
({x}_1^{i}, {x}_2^{j}) \in \mathcal{LV} \times \mathcal{LV}
~ \middle\vert ~
\begin{array}{l}
  {x}_1^{i}, {x}_2^{j} \in \vertxs
\land
  % \\
  \exists n \in \mathbb{N}, z_1^{r_1}, \cdots, z_n^{r_n} \in \lvar_{{c}} \st 
  n \geq 0 \land
  \\
  \flowsto(x^i,  z_1^{r_1}, c) 
  \land \cdots \land \flowsto(z_n^{r_n}, y^j, c) 
\end{array}
\right\}
\\
\text{Weights} &
\progW & := &
% \bigcup
% \begin{array}{l}
\left\{ (x^l, w) \in  \mathcal{LV} \times EXPR(\constdom)
\mid
x^l \in \lvar_{{c}} \land w = \absW(l)
\right\}
% \end{array} 
\\
\text{Query Annotation} &
\progF & := & 
\left\{(x^l, n)  \in  \mathcal{LV} \times \{0, 1\} 
~ \middle\vert ~
x^l \in \lvar_{c},
n = 1 \iff x^l \in \qvar_{c} \land n = 0 \iff  x^l \in \qvar_{c} .
\right\}
\end{array}
\] }
\end{defn}
% In last phase, we get a dependency graph whose node is computation blocks uniquely decided by its label. In this phase, we want to add more information to every node in the graph, which is the approximated visiting times (how many times this block is exeucted). The algorithm is defined in Algorithm~\ref{alg:add_weights}, with 3 main functions, the Control Location Insertion, 
% Program Transition Abstraction, 
% Bound Calculation 
% and ADDWEIGHTs.
% \begin{algorithm}
% \caption{
% {Add weights to dependency graph (the main algorithm of phase 2)}
% \label{alg:add_weights}
% }
% \begin{algorithmic}[1]
% \REQUIRE the program $c$, the dependency graph $G = (\vertxs, \edges, \weights, \qflag)$ from phase 1
% \STATE  prel = PREPROCESSING(c) 
% \STATE {\bf for} $x^l \in \vertxs$: 
% \STATE \qquad {\bf if} $l \in prel$:
% \STATE \qquad \qquad $\weights(x^l) = \rb(c, l)$
% \STATE \qquad {\bf else}:
% \STATE \qquad \qquad $\weights(x^l) = 1$
% \RETURN $G$
% \end{algorithmic}
% \end{algorithm}

% \paragraph{Getting Reachability Bounds}
% This is defined in Algorithm~\ref{alg:rb}.
% To be precise, we use static analysis method from \cite{Sumit2010rechability}, which is able to "provide the symbolic worst case bound on the number of times a block is reached", let us call it reachability bound analysis. 

% This analysis only works to find the symbolic bound of one block (in our graph, it corresponds to one node). This algorithm is summarized as follows.
% \begin{enumerate}
%     \item Build a transition system which describes the relation between variables in this target block and these variables in the successive visit to this block. There are defined translate functions which translate the statement to transition system and the corresponding operations such as the composition of transition systems, merging two transition systems from two control branches and so on. It is worth to mention that it calculates the transitive closure of the transition system obtained from a loop body, which can be analogy to computing the invariant of a loop.  
%     \item Use Ranking function which takes the transition system and outputs the bound
% \end{enumerate}

% \begin{algorithm}
% \caption{
% {Reachability Bound Analysis ($\rb$)}
% \label{alg:rb}
% }
% \begin{algorithmic}[1]
% \REQUIRE the program $C$, the target while loop with label $l$.
% \STATE  T  = GenerateTransitionSystem(C,l) 
% \STATE B = 1 + ComputeBound(T)
% \RETURN B
% \end{algorithmic}
% \end{algorithm}

% The algorithm $GenerateTransitionSystem(C,l)$ can be described as follows. It uses the control flow graph generated from the program $C$, and splits the node marked by $l$ into two nodes $l_1$ and $l_2$ to generate a new control flow graph from the $l_1$ to $l_2$. It translates the node in the graph into a transition systems by the its translate function and replaces node with transition systems. For loops in the graph, the loop itself is replaced by the transitive closure of the transition systems of its body. Finally, the new generated control flow graph can be transformed to a transition system. The transition system is a disjunction of transitions, and every transition is expressed as a conjuction of formulas over program variables $x,y,z$ in the target block (l) and its successive visits $x',y',z'$ in the same block.
% \\
% The transition formula for each command are as follows:
% \\
% \todo{adding the naive transition formula for assignment and if}
% \\
% \jl{$translate(c)$  is defined as follows:}
% \\
% $translate(\assign{x}{e}) = \{x := e\} if x \in VAR(b)$
% \\
% % $translate(\assign{x}{e}) = \{\} if x \notin VAR(b)$
% % \\
% $translate(\assign{x}{\qexpr}) = \{x \in VAR(b) \implies x := \qexpr\} $
% \\
% % $translate(\assign{x}{\qexpr}) = \{\} if x \notin VAR(b)$
% % \\
% $translate(\eif(\bexpr, c_1, c_2)) = \{\bexpr \implies translate(c_1), \neg\bexpr \implies translate(c_2)\}$
% \\
% $translate(\ewhile(\bexpr, c_w)) = \{ ComputeBound(GenerateTransitionSystem(C,l)) \times translate(c_w) \}$
% \\
% $translate(c_1;c_2) = translate(c_1) + translate(c_2)$
% \\
% \todo{adding the compose method for composing 2 transition formulas}
% $compose\{x := e_1, \cdots, x := e_2 \} = \{x := e_2\}$
% \\
% $\{y := e_1, \cdots, x := e_2 \} \land \flowsto(y, x, c) = \{x := [y \to e_1] e_2\}$
% \\
% \todo{add the $ComputeBound$ function, specifically the ranking function}
% \\
% The function $computeBound$ takes into a transition system (a disjuction of transitions), and computes the bound. There are ranking functions which take a transition and return the bound, that is used by $computeBound$. There are some heuristics in compute the bound based on the transitions systems, if interested, please look at the paper for more details.

% \paragraph{Add Weight} We also need to take care about the situation when a bound can not be predicted by {$\rb$}, we need to use another loose analysis to get a loose bound.



\clearpage
\subsection{Adaptivity Upper Bound Computation}
%  from refined weighted-labeled data-flow graph}
\label{sec:alg_adaptcompute}
This phase compute the adaptivity upper bound for a program $c$,
% Given a program ${c}$, we generate
based on its program-based data dependency graph 
$\progG({c}) = (\vertxs, \edges, \weights, \qflag)$. 
%
Defined in Definition~\ref{def:prog_adapt} as 
%
% Then the adaptivity bound based on program analysis for ${c}$ 
% is the number of query vertices on a finite walk in $\progG({c})$. This finite walk satisfies:
% \begin{itemize}
% \item the number of query vertices on this walk is maximum
% \item the visiting times of each vertex $v$ on this walk is bound by its reachability bound $\weights(v)$.
% \end{itemize}
the maximum query length over all finite walks in $\walks(\progG({c}))$, and computed 
% is computed as the maximum query length over all finite walks in $\walks(\progG({c}))$, and computed 
in Algorithm~\ref{alg:adpt_alg}.
%
% It is formally defined in \ref{def:prog_adapt}.
% defined formally as follows.
%
%
\begin{defn}
[{Program-Based Adaptivity}].
\label{def:prog_adapt}
\\
{
Given a program ${c}$ and its program-based graph 
$\progG({c}) = (\vertxs, \edges, \weights, \qflag)$,
%
the program-based adaptivity for $c$ is defined as%
\[
\progA({c}) 
:= \max
\left\{ \qlen(k)\ \mid \  k\in \walks(\progG({c}))\right \}.
\]
}
\end{defn} 

The following algorithm finds the walk with the longest query length on a program $c$'s execution-based dependency graph 
$\progG(c) = (\vertxs, \edges, \weights, \qflag)$, through a combination of 
% DFS and BFS algorithm 
deep first search and breath first search strategy
% as defined 
in Algorithm~\ref{alg:adpt_alg} and Algorithm~\ref{alg:adaptscc}.

\paragraph*{Challenge}
Following is the challenge of computing the adaptivity on a program based dependency graph.
In order to search for the finite walk having the longest query length, which isn't a simple longest weighted path.
\\
The visiting times of every vertex on this walk should be no more than its weight, which is a symbolic expression.
So we cannot simply search for the longest weight path where the visiting times of the vertex on it could possibly exceed its weights.
We can neither simply traverse on this graph by decreasing the weight of every node by 1 after every visiting,
because the weight is symbolic and simply traversing leads to non-termination.
\\
We can simply adopt either a deep first strategy to estimate the adaptivity as the length of the longest weight path, as in Algorithm~\ref{alg:overadp_alg}.
However, this gives us over-approximation to a large extend.
% In Algorithm~\ref{alg:adpt_alg}, 
% we first find all the strong connected components of this graph, 

\begin{algorithm}
    \caption{
    {Longest Adaptivity Search Algorithm ($\pathsearch$)}
    \label{alg:adpt_alg}
    }
    \begin{algorithmic}[1]
    \REQUIRE $G = (\vertxs, \edges, \weights, \qflag)$ \#\{The program based dependency graph\}
    % with a start vertex $s$ and destination vertex $t$ .
    \STATE  {\bf {$\kw{\pathsearch(G)}$}:}  
    \STATE {\bf init} 
    % \\
    % current node: $c$, 
    \\
    $q$: empty queue.
    % \\
    % $\kw{visited}$: List of length $|\vertxs|$, initialize with $\efalse$.
    % \\
    % $\kw{SSCvisited}$: List of length $|\vertxs|$, initialize with $\efalse$.
    % \\ 
    % $\kw{adapt_{scc}(SCC_i) = \pathsearch_{scc}(SCC_i)}$.
    \\
    $\kw{adapt}$ : the adaptivity of this graph initialize with $0$.
    \\
    \STATE Find all Strong Connected Components (SCC) in $G$: $\kw{SCC_1}, \cdots, \kw{SCC_n}, 0 \leq n \leq |\vertxs|$, 
    % where $\kw{SCC_i} = (\vertxs_i, \edges_i, \weights_i, \qflag_i)$.
    % and assign each vertex $x^i$ with an SCC number $\kw{SCC}(x^i)$
    \STATE {\bf for} every SCC: $\kw{SCC_i}$, compute its Adaptivity $\kw{SCC_i}$:
    \STATE \quad $\kw{adapt_{scc}[SCC_i] = \pathsearch_{scc}(SCC_i)}$;
    \STATE {\bf for} every $\kw{SCC_i}$:
    \STATE \qquad $q.append(\kw{SCC_i})$;
    \STATE \qquad $\kw{adapt_{tmp}} = 0$;
    \STATE \qquad {\bf while} $q$ isn't empty:
    \STATE \qquad \qquad $\kw{s} = q.pop()$;  \#\{take the top SCC from head of queue\}
    \STATE \qquad \qquad  $\kw{adapt_{tmp}}_0= \kw{adapt_{tmp}}$; \#\{record the adaptivity of last level\}
    \STATE \qquad \qquad  $\kw{SCC_{max}}$;  \#\{record the SCC with longest walk in this level\}
    % initialize cycle-adapt = 0.
    \STATE \qquad \qquad {\bf for} every 
    % SCC having a directed edge from $s$ of $s$: $\kw{SCC'}$:
    % directed edge goes out of $\kw{s}$ and connects a 
    different SCC, $\kw{s'}$ connected by $\kw{s}$ by a directed edge from $\kw{s}$:
    % \STATE \qquad \qquad   cycle-adapt$ = \max($cycle-adapt, $\kw{dfs_{refine}(G, v, v)})$;
    % \STATE \qquad \qquad \qquad \#\{compute the adaptivity of vertex $v$  on $\kw{SCC}(v)$, and update r[v] with the SCC-adapt\}
    % \STATE \qquad \qquad \qquad $ r[v] = r[s] + \kw{dfs_{refine}(G, v, visited)})$; 
    \STATE \qquad \qquad \qquad {\bf if} $(\kw{adapt_{tmp}} < \kw{adapt_{tmp}}_0 + \kw{adapt_{scc}[s']})$:
    \STATE \qquad \qquad \qquad \qquad $\kw{adapt_{tmp}} = \kw{adapt_{tmp}}_0 + \kw{adapt_{scc}[s']}$; 
    \STATE \qquad \qquad \qquad \qquad $\kw{SCC_{max} = s'} $; \#\{update the SCC with longest walk in this level\} 
    % \STATE \qquad   $r[c] = r[c] + $cycle-adapt;
    % \STATE \qquad for all unvisited vertex $v$ having directed edge from c and $! \kw{cycle}(c)$:
    % \STATE \qquad \qquad $r[v] = r[c] + \flag(v)$; 
    % \STATE \qquad \qquad \qquad  \#\{mark all the nodes with the same $\kw{SCC}$ number as visited\} 
    % \STATE \qquad \qquad \qquad  \#\{append the unvisited vertex to the rear of the queue\}
    % \STATE \qquad \qquad \qquad  \#\{mark all the nodes with the same $\kw{SCC}$ number as visited\} 
    % \STATE \qquad \qquad for $v \in V$,   $\kw{visited}[s] = 1$;
    \STATE \qquad \qquad \qquad $q.append(\kw{SCC_{max}})$;
    \STATE \qquad $\kw{adapt} = \max(\kw{adapt}, \kw{adapt_{tmp}})$;    
    \RETURN $\kw{adapt}$.
    \end{algorithmic}
    \end{algorithm}
    %

    In Algorithm~\ref{alg:adpt_alg}, 
    it first finds all the strong connected components (SCC) of this graph using the Kosaraju’s algorithm in line:3.
    Every $\kw{SCC_1}, \cdots, \kw{SCC_n}, 0 \leq n \leq |\vertxs|$ is a sub-graph of $\progG(c)$, where $\kw{SCC_i} = (\vertxs_i, \edges_i, \weights_i, \qflag_i)$.
    % where $\kw{SCC_i} = (\vertxs_i, \edges_i, \weights_i, \qflag_i)$.
    Then, 
    % we compute the adaptivity on every SCC, which is a subgraph of the $\progG(c)$, in line:4-5 by Algorithm~\ref{alg:adaptscc}.
    it computes the adaptivity on every SCC
    % , which is a subgraph of the $\progG(c)$, 
    in line:4-5 by Algorithm~\ref{alg:adaptscc}.
    We guarantee the soundness of the adaptivity on SCC by Lemma~\ref{lem:sound_adaptalg_scc} with proof in Appendix~\ref{apdx:adaptalg_soundness}.
    The $\progG(c)$ is then shrank into an acyclic directed graph where 
    % vertices are all the SCCs and edges are between every SCCs with their adaptivities as weights.
    $\kw{SCC_1}, \cdots, \kw{SCC_n}$ are vertices with the adaptivities as weights.
    % , and directed edges are .
    For every $(v_i, v_j) \in \edges$ such that $v_1 \in \vertxs_i$, $v_j \in \vertxs_j$ and $i \neq j$,
    there is a edge $(s_i, s_j)$ in this shrank graph. \\ 
    Then, we use the standard breath first search strategy to find the longest weighted path
    %  w.r.t. all the SCCs and their adaptivities.
    on this shrank graph and return the length as adaptivity.
    \\
    We guarantee that 
    % this longest weighted path is a sound computation of the adaptivity on this,
    the length of this longest weighted path is a sound computation of the adaptivity for program $c$,
    % as well as 
    and this longest weighted path a sound computation of the finite walk having the longest query length 
    % on this graph, in Theorem~\ref{thm:sound_adaptalg}
    on $c$'s program based dependency graph, in Theorem~\ref{thm:sound_adaptalg}
    in Appendix~\ref{apdx:adaptalg_soundness}.
%    
    % for every vertex which isn't on any SCC, it is easy to know that it will be visited 
    % at most once given no edges going back to this vertex. We can know the adaptivity on the SCC 
     %
    % \begin{algorithm}
    % \caption{
    % {Longest Adaptivity Search Algorithm ($\pathsearch$)}
    % \label{alg:adpt_alg}
    % }
    % \begin{algorithmic}
    % \REQUIRE Weighted Directed Graph $G = (\vertxs, \edges, \weights, \flag)$ with a start vertex $s$ and destination vertex $t$ .
    % \STATE  {\bf {bfs $(G)$}:}  
    % \STATE {\bf init} 
    % \\
    % current node: $c$, 
    % \\
    % queue: $q$ : List, add into $a$ an arbitrary v from $\vertxs$. 
    % \\
    % visited: List of length $|\vertxs|$, initialize with $\efalse$.
    % \\
    % results: $r$ : List of length $|\vertxs|$, initialize with -1.
    % \\
    % curr$\kw{flowcapacity}$: INT, initialize MAXINT.
    % \\
    % querynum: INT, initialize 0. \#\{To count the query numbers when we are walking inside a cycle\}
    % \\
    % \STATE \qquad {\bf while} $q$ isn't empty:
    % \STATE \qquad \qquad take the vertex from head $c= q.pop()$
    % \STATE \qquad \qquad mark $c$ as visited, visited $[c] = 1$.
    % \STATE \qquad \qquad {\bf if} $\kw{cycle}(c)$  \#\{we are inside a cycle\}
    % \STATE \qquad \qquad \qquad curr$\kw{flowcapacity}$ = min($\weights$(c), curr$\kw{flowcapacity}$).
    % \STATE \qquad \qquad \qquad querynum += $\flag(c)$.
    % \STATE \qquad \qquad  \qquad for all unvisited vertex $v$ having directed edge from c:
    % \STATE \qquad \qquad \qquad \qquad r[v] = r[c]; q.add(v)
    % \STATE \qquad \qquad \qquad  {\bf if}  $v$ is visited, then the circle finished
    % \STATE \qquad \qquad \qquad \qquad update the result $r[v] =  \max(r[v], r[c] + $curr$\kw{flowcapacity}$*querynum)
    % \STATE \qquad \qquad \qquad \qquad curr$\kw{flowcapacity}$ = MAXINT
    % \STATE \qquad \qquad \qquad \qquad querynum = 0.  
    % \STATE \qquad \qquad {\bf else} 
    % \STATE \qquad \qquad \qquad for all unvisited vertex $v$ having directed edge from c:
    % \STATE \qquad \qquad \qquad  \qquad $r[v] = \max(r[v], r[c] + \flag(c))$; q.add(v)
    % \RETURN max($r$)
    % \end{algorithmic}
    % \end{algorithm}
    %
    \begin{algorithm}
      \caption{
      {Over-Approximated Adaptivity on SCC}
      \label{alg:overadp_alg}
      }
      \begin{algorithmic}[1]
      \REQUIRE $G = (\vertxs, \edges, \weights, \qflag)$ \#\{An Strong Connected Symbolic Weighted Directed Graph\}
      % with a start vertex $s$ and destination vertex $t$ .
      \STATE {\bf {$\kw{\pathsearch_{scc-naive}(G)}$}:}  
      \STATE {\bf init} 
      \\
      $\kw{r_{scc}}$: the Adaptivity of this SCC
      % \STATE  {\bf def} {$\kw{dfs_{naive}(G, c,visited)}$}: 
      % % \STATE {\bf init} 
      % % \\
      % % current node: $c$, 
      % % \\
      % % visited: List of length $|\vertxs|$, initialize with $\efalse$.
      % % \\
      % % \STATE {\bf if} $c = s$:
      % % \RETURN \qquad  $\weights(s)*\flag(s) $.
      % \STATE \qquad $r[c] = \weights(c)*\qflag(c) $
      % \STATE \qquad {\bf for}  all vertex $v$ having directed edge from $c$:
      % \STATE \qquad \qquad {\bf if}  $v$ is unvisited:
      % \STATE \qquad \qquad \qquad  \#\{mark $v$ as visited\} $\kw{visited}[v] = 1$;
      % \STATE \qquad \qquad \qquad $r[c] += \kw{dfs_{naive}(G, v, visited)}$;
      % \STATE \qquad {\bf else}: \#\{There is a cycle finished\}
      % \RETURN \qquad \qquad $\weights(v)*\flag(v) $.
      \STATE  {\bf for} every vertex $v$ in $\vertxs$:
      % \STATE  \qquad initialize \kw{visited} with $\efalse$.
      \STATE  \qquad $r_{scc} += \weights(v)*\qflag(v)$  
      \RETURN $r[c]$
      \end{algorithmic}
      \end{algorithm}
      %
%
    % \begin{algorithm}
    %     \caption{
    %     {Over-Approximated Adaptivity on SCC}
    %     \label{alg:overadp_alg}
    %     }
    %     \begin{algorithmic}
    %     \REQUIRE Weighted Directed Graph $G = (\vertxs, \edges, \weights, \qflag)$ with a start vertex $s$ and destination vertex $t$ .
    %     \STATE  {\bf {$\kw{dfs_{naive}(G, c,visited)}$}:}  
    %     % \STATE {\bf init} 
    %     % \\
    %     % current node: $c$, 
    %     % \\
    %     % visited: List of length $|\vertxs|$, initialize with $\efalse$.
    %     % \\
    %     % \STATE {\bf if} $c = s$:
    %     % \RETURN \qquad  $\weights(s)*\flag(s) $.
    %     \STATE $r[c] = \weights(c)*\qflag(c) $
    %     \STATE {\bf for}  all vertex $v$ having directed edge from $c$:
    %     \STATE \qquad {\bf if}  $v$ is unvisited:
    %     \STATE \qquad \qquad  \#\{mark $v$ as visited\} $\kw{visited}[v] = 1$;
    %     \STATE \qquad \qquad $r[c] += \kw{dfs_{naive}(G, v, visited)}$;
    %     % \STATE \qquad {\bf else}: \#\{There is a cycle finished\}
    %     % \RETURN \qquad \qquad $\weights(v)*\flag(v) $.
    %     \RETURN $r[c]$
    %     \end{algorithmic}
    %     \end{algorithm}%
        %
    \begin{algorithm}
            \caption{
            {Adaptivity on $\kw{SCC}$}
            \label{alg:adaptscc}
            }
            \begin{algorithmic}[1]
              \REQUIRE $G = (\vertxs, \edges, \weights, \qflag)$ \#\{An Strong Connected program based dependency Graph\}
            \STATE  {\bf {$\kw{\pathsearch_{scc}(G)}$}:}  
            \STATE {\bf init} 
            \\
            $\kw{r_{scc}}$: $EXPR(\constdom)$, initialized $0$, the Adaptivity of this SCC
            \STATE \qquad {\bf init} 
            % \STATE \qquad current node: $c$, 
            % \\
            % visited: List of length $|\vertxs|$, initialize with $\efalse$.
            % \\ \qquad  $\kw{r_{scc}}$ : initialize $0$, the adaptivity of this graph
            \\ \qquad  $\kw{visited}$ : $\{0, 1\}$ List, 
            \\ \qquad  \#\{length $|\vertxs|$, initialize with $0$ for every vertex, recording whether a vertex is visted.\}
            \\ \qquad  $\kw{r}$ : $EXPR(\constdom)$ List, 
            \\ \qquad  \#\{length $|\vertxs|$, initialize with $\qflag(v)$ for every vertex, recording the adaptivity reaching each vertex.\}
            \\ \qquad  $\kw{flowcapacity}$: $EXPR(\constdom)$ List, 
            % INT List of length $|\vertxs|$, initialize MAXINT. 
            \\ \qquad  \#\{length $|\vertxs|$, initialize with $\infty$ for every vertex,
            % \#\{For every vertex, 
            recording the minimum weight when the walk reaching 
            that vertex, inside a cycle\}
            \\ \qquad  $\kw{querynum[v]}$: INT List,
            %  of length $|\vertxs|$, initialize with $\qflag(v)$ for every vertex. 
            \\ \qquad  \#\{length $|\vertxs|$, initialize with $\qflag(v)$ for every vertex, 
            % \#\{For every vertex, 
            recording the query numbers when the path reaching 
            that vertex, inside a cycle\}
            \STATE {\bf if} $|\vertxs| = 1$ and $|\edges| = 0$:
            \STATE \qquad {\bf return}  $\qflag(v)$
            \STATE  {\bf def} {$\kw{dfs(G, c,visited)}$}:
            % \STATE \qquad update the length of the longest path reaching this vertex
            % $r[s] =  r[s] + $$\kw{flowcapacity}$[s] * querynum[s].
            % \RETURN  \qquad $r[s]$.      
            \STATE \qquad {\bf for} every vertex $v$ 
            % having directed edge from $c$:
            connected by a directed edge from $c$:
            \STATE \qquad \qquad {\bf if} $\kw{visited}[v] = \efalse$:
            \STATE \qquad \qquad \qquad $\kw{flowcapacity[v] = \min(\weights(v), {flowcapacity}[c])}$;
            \STATE \qquad \qquad \qquad $\kw{querynum[v] = querynum[c] + \qflag(v)}$;
            % \STATE \qquad \qquad \qquad \#\{do not update the length of the longest walk reaching $v$ until the cycle is finished\}
            % \STATE \qquad \qquad \qquad $\kw{r[v] =  r[c] + flowcapacity[v] \times querynum[v]} $; \#\{do not update the length of the longest walk reaching $v$ until the cycle is finished\}
            \STATE \qquad \qquad \qquad $\kw{r[v] =  \max(r[v], flowcapacity[v] \times querynum[v]}) $; 
            % \#\{do not update the length of the longest walk reaching $v$ until the cycle is finished\}
            \STATE \qquad \qquad \qquad  $\kw{visited}[v] = 1$; %\#\{mark $v$ as visited\}
            \STATE \qquad \qquad \qquad $\kw{dfs(G, v, visited)}$;
            \STATE \qquad \qquad {\bf else}: \#\{There is a cycle finished\}
            % \STATE \qquad \qquad \qquad \#\{update the length of the longest path reaching this vertex\}
            \STATE \qquad \qquad \qquad 
            $\kw{r[v] =  \max(r[v], r[c] +  \min(\weights(v), {flowcapacity}[c]) * (querynum[c] + \qflag(v)))}$; \#\{update the length of the longest walk reaching this vertex on this cycle\}
            %  $\kw{r[v] =  \max(r[v], r[c] + flowcapacity[v] * querynum[v])}$; \#\{update the length of the longest walk reaching this vertex on this cycle\}
            %  \STATE \qquad \qquad \qquad \#\{Recover the $\kw{flowcapacity}$ and querynumber to previous state, for different loops\}
            % \STATE \qquad \qquad \qquad $\kw{flowcapacity[v] = flowcapacity[c]}$; \#\{Recover the $\kw{flowcapacity}$\}
            % \STATE \qquad \qquad \qquad $\kw{querynum[v] = querynum[c]}$;\#\{Recover the $\kw{querynum}$\}
            \STATE \qquad {\bf return}  $\kw{r[c]}$
            \STATE  {\bf for} every vertex $v$ in $\vertxs$:
            \STATE  \qquad initialize the $\kw{visited, r, flowcapacity, querynum}$;
            \STATE  \qquad $\kw{r_{scc} = \max(r_{scc}, dfs(G, v, \kw{visited} ))}$ ; 
            \RETURN  $\kw{r_{scc}}$
            \end{algorithmic}
            \end{algorithm}
            \\
Following is the challenge of computing the adaptivity on a program based dependency graph.
In order to search for the finite walk having the longest query length, which isn't a simple longest weighted path.
\\
the visiting times of every vertex on this walk should be no more than its weight, which is a symbolic expression.
So we cannot simply search for the longest weight path where the visiting times of the vertex on it could possibly exceed its weights.
We can neither simply traverse on this graph by decreasing the weight of every node by 1 after every visiting,
because the weight is symbolic and simply traversing leads to non-termination.
\\
In Algorithm~\ref{alg:adaptscc}, if an SCC contains only one vertex without any edge, 
then it is easy to know that it will be visited 
at most once since there isn't edge going back to this vertex. 
So we can know that the adaptivity on this SCC is at most one if it is a query vertex,
and zero otherwise.
\\
For the SCC contains at least one edge, we are searching for the finite walk having the longest query length through a deep first search strategy.
The difficulty is, the visiting times of every vertex on this walk should be no more than its weight, which is a symbolic expression.
So we cannot simply search for the longest weight path where the visiting times of the vertex on it could possibly exceed its weights.
We can neither simply traverse on this graph by decreasing the weight of every node by 1 after every visiting,
because the weight is symbolic and simply traversing leads to non-termination.
\\
So we use the dfs to search for the finite path with a capacity limitation and use special parameter to compute the adaptivity
for every path.
We use a special parameter $\kw{flowcapacity}$  to track the minimum weight during the dfs process, and $\kw{querynum[v]}$
to track the total number of vertices which are query vertices.

we first initialize some parameters,
the $\kw{visited}$ a list of $\etrue$ or $\efalse$ for every vertex on this SCC, to guarantee the termination;
\\ 
$\kw{r}$ : INT List of length $|\vertxs|$, initialize with $\qflag(v)$ for every vertex. The adaptivity reaching each vertex.
\\ 
$\kw{flowcapacity}$ a list of symbolic expressions for every vertex, recording the minimum weight when the walk reaching 
that vertex, inside a cycle\}
\\ 
$\kw{querynum[v]}$: INT List of length $|\vertxs|$, initialize with $\qflag(v)$ for every vertex. 
\#\{For every vertex, recording the query numbers when the path reaching.
\\
Then from line5:11, we record the minimum weight and number of query vertices alone the path and update the adaptivity reaching 
this vertex, and then recursively dfs on all vertices heading out from this vertex.
\\
At line 12 where this vertex is visited the second time, 
we only update the adaptivity reaching this vertex and neither recursion nor update the $\kw{flowcapacity}$  and 
$\kw{querynum[v]}$.
\\
not recursion in order to guarantee the termination,
do not update the $\kw{querynum[v]}$ because a second visiting of the same vertex indicates there is a cycle goes back to this vertex, 
then, when we continuously search path heading out of this vertex, the minimum weight on this cycle will not affect the walks going out of this vertex that not pass this cycle.
However, if we keep recording the minimum weight, we are restricting the visiting times on a walk By
 using the minimum weight of vertices not on this walk, it is unsound anymore.
 %
 Then we compute the adaptivity of this SCC by taking the maximum adaptivity reaching every vertex on this SCC.
%
The soundness is formally guaranteed in Lemma~\ref{lem:sound_adaptalg_scc} in Appendix~\ref{apdx:adaptalg_soundness}.
            % \begin{algorithm}
        % \caption{
        % {Refined Adaptivity on $\kw{SCC}$}
        % \label{alg:dfscycle_alg}
        % }
        % \begin{algorithmic}
        % \REQUIRE Weighted Directed Graph $G = (\vertxs, \edges, \weights, \qflag)$ with a start vertex $s$ and destination vertex $t$ .
        % \STATE  {\bf {$\kw{dfs_{refine}(G, c, visited)}$}:}  
        % \STATE {\bf init} 
        % \\
        % current node: $c$, 
        % % \\
        % % visited: List of length $|\vertxs|$, initialize with $\efalse$.
        % \\
        % results: $r$ : INT List of length $|\vertxs|$, initialize with $\qflag(v)$ for every vertex.
        % \\
        % $\kw{flowcapacity}$: INT List of length $|\vertxs|$, initialize MAXINT. 
        % \#\{For every vertex, recording the minimum weight when the walk reaching 
        % that vertex, inside a cycle\}
        % \\
        % querynum: INT List of length $|\vertxs|$, initialize with $\qflag(v)$ for every vertex. 
        % \#\{For every vertex, recording the query numbers when the walk reaching 
        % that vertex, inside a cycle\}
        % \\
        % % \STATE {\bf if} $c = s$:
        % % \STATE \qquad update the length of the longest path reaching this vertex
        % % $r[s] =  r[s] + $$\kw{flowcapacity}$[s] * querynum[s].
        % % \RETURN  \qquad $r[s]$.      
        % \STATE {\bf for}  all vertex $v$ having directed edge from $c$:
        % \STATE \qquad \qquad $\kw{flowcapacity}$[v] = min($\weights(v)$, $\kw{flowcapacity}$[c]);
        % \STATE \qquad \qquad querynum[v] = querynum[c] + $\qflag(v)$;
        % \STATE \qquad \qquad \#\{do not update the length of the longest walk reaching $v$ until the cycle is finished\}
        % \STATE \qquad \qquad $r[v] =  r[c] $;
        % \STATE \qquad {\bf if}  $v$ is unvisited:
        % \STATE \qquad \qquad \#\{mark $v$ as visited\} $\kw{visited}[v] = 1$;
        % \STATE \qquad \qquad $\kw{dfs_{refine}(G, v, visited)}$;
        % \STATE \qquad {\bf else}: \#\{There is a cycle finished\}
        % \STATE \qquad \qquad \#\{update the length of the longest path reaching this vertex\}
        % \STATE \qquad \qquad 
        %  $r[v] =  \max(r[v], r[c] + $$\kw{flowcapacity}$[v] * querynum[v]);
        %  \STATE \qquad \qquad \#\{Recover the $\kw{flowcapacity}$ and querynumber to previous state, for different loops\}
        %  \STATE \qquad \qquad $\kw{flowcapacity}$[v] = $\kw{flowcapacity}$[c];
        %  \STATE \qquad \qquad querynum[v] = querynum[c];
        % \RETURN  $r[c]$
        % \end{algorithmic}
        % \end{algorithm}
        % %

\begin{thm}[Soundness of $\pathsearch$]
    \label{thm:sound_adaptalg}
    For every program $c$, given its \emph{Program-Based Dependency Graph} $\progG$,
     $$\pathsearch(\progG) \geq \progA(\progG).$$
\end{thm}
% \begin{thm}[Soundness of $\pathsearch$]
  \label{thm:sound_adaptalg}
  For every program $c$, given its \emph{Program-Based Dependency Graph} $\progG$,
   $$\pathsearch(\progG) \geq \progA(\progG).$$
\end{thm}
proof Summary:
1. for each SCC, a subgraph of $\progG$, $\pathsearch_{scc}(SCC) \geq \progA(SCC)$
2. for every two nodes with a path $x, \cdots, y$, let $\walks(k_{x,y})$ be all the walks from $x$ to $y$ on $\progG$,
then $adapt[SCC(x)] + \cdots + adapt[SCC(y)] \geq \max\{\qlen(k)\}$
\begin{proof}

\end{proof}
% % \paragraph{Variable Collection Algorithm, $\varCol$}
% % % The $\varCol$ algorithm shows how the labelled variables $\lvar$ are collected 
% % % (via the command ${\assign{x}{\expr}}$ or ${\assign{x}{\query(\qexpr)}}$) from the program ${c}$ in the first step.
% % % The algorithmic rules for $\varCol$ algorithm is defined in Figure~\ref{fig:var_col}. 
% % % It has the form: $\ag{\lvar; w; {c}}{ \lvar'; w'} $. 
% % % The input of $\varCol$ is the labelled variables $\lvar$ collected before the program ${c}$, a while map $w$ consistent with previous estimation, a program ${c}$. 
% % % The output of the algorithm is the updated labelled variables $\lvar'$, along with the updated while map $w$ for next steps' collecting.   
% % The $\varCol$ algorithm shows how the labelled variables $\lvar$ are collected 
% % (via the command ${\assign{x}{\expr}}$ or ${\assign{x}{\query(\qexpr)}}$) from the program ${c}$ in the first step, 
% % along with constructing the flag for each variable, i.e., $\flag$.
% % The algorithmic rules for $\varCol$ algorithm is defined in Figure~\ref{fig:var_col}. 
% % It has the form: 
% % {$\ag{\lvar; \flag; {c}}{ \lvar'; \flag'} $}. 
% % The input of $\varCol$ is a program ${c}$, 
% % the labelled variables $\lvar$ collected before the program ${c}$ 
% % as well as the flags $\flag$ for every corresponding variable .
% % The output of the algorithm is the updated labelled variables $\lvar'$ and flags $\flag'$ thorough the program ${c}$
% % %
% % % We have the algorithmic rules for $\varCol$ algorithm of the form: $\ag{\lvar; w; {c}}{\lvar';w'} $ as in Figure \ref{fig:var_col}. 
% % %
% % \begin{figure}
% % {
% % \begin{mathpar}
% % \inferrule
% % {
% % \empty
% % }
% % { \ag{\lvar ; \flag; {[\assign {x}{\expr}]^{l}}}
% % {\lvar ++ [{x}]; \flag++[0]}
% % }
% % ~\textbf{\varCol-asgn}
% % \and
% % \inferrule
% % {
% % }
% % { \ag{\lvar; \flag; [ \assign{{x}}{\query({\qexpr})}]^{l}}
% % {\lvar ++ [{x}]; \flag ++ [2]} 
% % }~\textbf{\varCol-query}
% % %
% % \and 
% % %
% % \inferrule
% % {
% % \ag{\lvar; [];  {c_1}}{\lvar_1; \flag_1}
% % \and 
% % \ag{\lvar_1; []; {c_2}}{ \lvar_2; \flag_2}
% % \and
% % \lvar_3 = \lvar_2 ++ \lvar'
% % \and
% % \flag_3 = \flag ++ ((\flag_1 ++ \flag_2) \uplus 1)
% % }
% % {
% % \ag{\lvar; \flag;
% % [\eif({\bexpr}, { c_1, c_2)}]^{l} }
% % {\lvar_3; \flag_3}
% % }~\textbf{\varCol-if}
% % %
% % %
% % %
% % \and 
% % %
% % \inferrule
% % {
% % \ag{\lvar; \flag {c_1}}{\lvar_1; \flag_1}
% % \and 
% % \ag{\lvar_1; \flag_1 ; {c_2}}{\lvar_2; \flag_2}
% % }
% % {
% % \ag{\lvar; \flag;
% % {(c_1 ; c_2)}}{\lvar_2 ; \flag_2}
% % }
% % ~\textbf{\varCol-seq}
% % \and 
% % %
% % %
% % {
% % \inferrule
% % {
% % { \ag{\lvar; [] ; {c}}
% % {\lvar'; \flag' }  }
% % \\
% % \lvar'' = \lvar'
% % \and 
% % \flag'' = \flag ++ (\flag' \uplus 1)
% % }
% % {
% % \ag{\lvar; \flag;  
% % \ewhile [{b}]^{l}
% % \edo  {c} }{\lvar''; \flag''}
% % }
% % ~\textbf{\varCol-while}
% % }
% % \end{mathpar}
% % }
% % \caption{The Algorithmic Rules of $\varCol$ }
% % \label{fig:var_col}
% % \end{figure}
% % %
% % %
% % The assignment commands are the source of variables $\varCol$ collecting, 
% % in the case $\textbf{\varCol-asgn}$ and $\textbf{\varCol-query}$, 
% % the output labelled variables are extended by ${x}$. 
% % \\
% % \todo{
% % When it comes to the $\eif \ldots \ethen \ldots \eelse$ command in the rule $\textbf{\varCol-if}$, variables assigned in the then branch ${c_1}$, as well as the variables assigned in the else branch ${c_2}$, and the new generated variables $\bar{{x}},\bar{{y}},\bar{{z}}$ in $ [ \bar{{x}}, \bar{{x_1}}, \bar{{x_2}}] ,[ \bar{{y}}, \bar{{y_1}}, \bar{{y_2}}],[ \bar{{z}}, \bar{{z_1}}, \bar{{z_2}}]$.
% % \\ 
% % The sequence command ${c_1;c_2}$ is standard by accumulating the predicted variables in the two commands ${c_1}$ and ${c_2}$ preserving their order. 
% % \\
% % The while command $\ewhile {\bexpr}, [{\bar{x}}] \ldots \edo {c}$ considers the newly generated variables by SSA transformation ${\bar{x}}$
% % as well and the newly labelled variables in its body ${c}$.
% % \\
% % %
% % Below we present the definition for a valid index, to have a clear understanding on the variable collecting algorithm:
% % }
% % %
% % %
% % \todo{
% % \begin{defn}[Valid Index (Remove?)]
% % Given an assigned variable list $\lvar$, $\lvar; \vDash ({c},i_1,i_2)$ iff 
% % $\lvar' = \lvar[0,\ldots, i_1-1], \lvar';{c} \to \lvar'' \land \lvar'' = \lvar[0, \ldots, i_2-1] $.  
% % \end{defn}}
% % %
% % %
% \todo{Data Dependency Analysis Algorithm Needed: (Possibly modify based on existing one, or a different one) get the more precise dependency information. 
% i.e., instead of dependency on all the over-approximated variables, 
% but dependency on only the variables assumed to be live.
% }
% \paragraph{Data Dependency Analysis Algorithm}
% %
% In this data flow matrix generating algorithm, we analyze the data flow information among all labelled variables $\lvar$ collected via the the $\varCol$ algorithm of length $N$.
% %
% We track the data flow relations between all these labelled variables. These informations are stored in a matrix $\Mtrix$, whose size is $N \times N$. 
% % We also track whether arbitrary variable is assigned with a query result in a vector $\flag$ with size $|\lvar|$. 
% %
% The algorithm to fill in the matrix is of the form: 
% {$\ad{\Gamma ; {c} ; \lvar}{\Mtrix}$}
% $\ad{\Gamma ; {c} ; i_1, i_2}{\Mtrix; \flag}$. 
% $\Gamma$ is a vector records the variables the current program ${c}$ depends on, the index $i_1$ is a pointer which refers to the position of the first new-generated variable in ${c}$ in the labelled variables $\lvar$, and $i_2$ points to the first new variable that is not in ${c}$ (if exists). 
% % %
% % %
% % {
% % \begin{defn}[Valid Gamma (Remove?)]
% % $\Gamma \vDash i_1$ iff $\forall i \geq i_1, \Gamma(i_1)=0 $.  
% % \end{defn}
% % }
% %%
% %
% % \framebox{$ {\Gamma} \vdash^{i_1, i_2}_{\Mtrix, \flag} ~ c $}
% % \begin{mathpar}
% % \inferrule
% % {\Mtrix = \lMtrix_i * ( \rMtrix_{{\expr},i} + \Gamma )
% % }
% % {
% %  \ad{\Gamma;[\assign {{x}}{{\expr}} ]^{l}; i }{\Mtrix; \flag_{0}; i+1 }
% % }
% % ~\textbf{\graphGen-asgn}
% % \and
% % {
% % \inferrule
% % {\Mtrix = \lMtrix_i * ( \rMtrix_{{\expr},i} + \Gamma )
% % \\
% % \flag = \lMtrix_i \and \flag(i) = 1
% % }
% % { 
% % \ad{\Gamma;[ \assign{{x}}{\query({\expr})} ]^{l} ; i }
% % {\Mtrix;\flag;i+1}
% % }~\textbf{\graphGen-query}}
% % %
% % \and 
% % %
% % {
% % \inferrule
% % {
% % {\ad{\Gamma + \rMtrix_{{\bexpr}, i_1}; {c_1} ; i_1 }{ \Mtrix_1;\flag_1;i_2 }}
% % \and 
% % {\ad{\Gamma + \rMtrix_{{\bexpr}, i_1};{c_2} ; i_2 }{ \Mtrix_2; \flag_2 ;i_3}}
% % \\
% % {\ad{\Gamma; [ \bar{{x}}, \bar{{x_1}}, \bar{{x_2}}]; i_3 }{ M_x; \flag_{\emptyset}; i_3+|\bar{{x}}| }}
% % %
% % \\
% % %
% % {\ad{\Gamma; [ \bar{{y}}, \bar{{y_1}}, \bar{{y_2}}]; i_3+|\bar{{x}}| }{ \Mtrix_y; \flag_{\emptyset}; i_3+|\bar{{x}}|+|\bar{{y}}| }}
% % %
% % \\
% % %
% % {\ad{\Gamma; [ \bar{{z}}, \bar{{z_1}}, \bar{{z_2}}]; i_3+|\bar{{x}}|+ |\bar{{y}}|}{ \Mtrix_y; \flag_{\emptyset}; i_3+|\bar{{x}}|+|\bar{{y}}| + |\bar{{z}}| }}
% % \\
% % {\Mtrix = (\Mtrix_1 + \Mtrix_2)+ \Mtrix_x+ \Mtrix_y + \Mtrix_z }
% % }
% % {
% % \ad{\Gamma ; \eif([{\bexpr}]^{l},[ \bar{{x}}, \bar{{x_1}},
% % \bar{{x_2}}] ,[ \bar{{y}}, \bar{{y_1}}, \bar{{y_2}}], 
% % [ \bar{{z}}, \bar{{z_1}}, \bar{{z_2}}],
% % { c_1, c_2)} ; i_1}{ \Mtrix ; \flag_1 \uplus \flag_2 \uplus 2  ; i_3+|\bar{x}|+|\bar{y}|+|\bar{z}| }
% % }
% % ~\textbf{\graphGen-if}
% % }
% % %
% % %
% % %
% % \and 
% % %
% % \inferrule
% % {
% % {\ad{\Gamma; {c_1} ; i_1 }{ \Mtrix_1 ; \flag_1; i_2 }  }
% % \and 
% % {
% % \ad{\Gamma;{c_2}; i_2}{ \Mtrix_2; \flag_2 ;i_3 }}
% % }
% % {
% % \ad{\Gamma ; ({c_1 ; c_2} ) ; i_1}{( \Mtrix_1 {;} \Mtrix_2) ; \flag_1 \uplus V_2 ; i_3  }
% % }
% % ~\textbf{\graphGen-seq}
% % %
% % \and 
% % %
% % \and 
% % %
% % { 
% % \inferrule
% % {
% % B= |{\bar{x}}| \and {A = |{c}|}
% % \\
% % {\ad{\Gamma;[\bar{{x}}, \bar{{x_1}}, \bar{{x_2}}]; i+ (B+A) }{ \Mtrix_{1};V_{1}; i+B+(B+A) }}
% % \\
% % {
% % \ad{\Gamma;{c} ; i+B+(B+A)  }{ \Mtrix_{2}; \flag_{2}; i+B+A+(B+A) }
% % }
% % \\
% % {
% % \ad{\Gamma ; [\bar{{x}}, \bar{{x_1}}, \bar{{x_2}}] ; i+(B+A) }{ \Mtrix; \flag ;i+(B+A)+B}
% % }
% % \\
% % { \Mtrix' = \Mtrix + ( \Mtrix_{1} + \Mtrix_{2}) }
% % \and
% % {
% % \flag' = \flag \uplus (( \flag_{1} \uplus \flag_{2}) \uplus 2)  }
% % }
% % {
% % \ad{\Gamma;
% % \ewhile ~ [ b ]^{l} ~ {n} ~
% % [\bar{{x}}, \bar{{x_1}}, \bar{{x_2}}] 
% % ~ \edo ~  c;
% % i }{ \Mtrix'; \flag' ;i+(B+A)+B }
% % }~\textbf{\graphGen-while}
% % }
% % \end{mathpar}
% {
% \framebox{$ \ad{\Gamma; c; \lvar_c}{\Mtrix}$}
% \begin{mathpar}
% \inferrule
% {
% {x}^l \in \lvar_c
% \and 
% \Mtrix = \lMtrix_i * ( \rMtrix_{{\expr}} + \Gamma )
% }
% {
% \ad{\Gamma; [\assign {{x}}{{\expr}} ]^{l}; \lvar_c}
% {\Mtrix}
% }
% ~\textbf{\graphGen-asgn}
% \and
% {
% \inferrule
% {
% {x}^l \in \lvar_c
% \and 
% \Mtrix = \lMtrix_i * ( \rMtrix_{{\expr}} + \Gamma )
% }
% { 
% \ad{\Gamma;[ \assign{{x}}{\query({\qexpr})} ]^{l} ; \lvar_c }
% {\Mtrix}
% }~\textbf{\graphGen-query}}
% %
% \and 
% %
% {
% \inferrule
% {
% {\ad{\Gamma + \rMtrix_{{\bexpr}}; {c_1} ; \lvar_c }{ \Mtrix_1}}
% \and 
% {\ad{\Gamma + \rMtrix_{{\bexpr}}; {c_2}; \lvar_c }{ \Mtrix_2}}
% \and
% {\Mtrix = (\Mtrix_1 + \Mtrix_2)}
% }
% {
% \ad{\Gamma ; \eif([{\bexpr}]^{l},{ c_1, c_2)}}
% { \Mtrix }
% }
% ~\textbf{\graphGen-if}
% }
% %
% %
% %
% \and 
% %
% \inferrule
% {
% {\ad{\Gamma; {c_1}; \lvar_c }{ \Mtrix_1}  }
% \and 
% {
% \ad{\Gamma;{c_2}; \lvar_c }{ \Mtrix_2}}
% }
% {
% \ad{\Gamma ; ({c_1 ; c_2} ); \lvar_c}
% {( \Mtrix_1 {;} \Mtrix_2) }
% }
% ~\textbf{\graphGen-seq}
% %
% \and 
% %
% \and 
% %
% { 
% \inferrule
% {
% {
% \ad{\Gamma + \rMtrix_{{\bexpr}};{c}; \lvar_c  }{ \Mtrix'}
% }
% }
% {
% \ad{\Gamma;
% \ewhile [ \sbexpr ]^{l} \edo  {c}; \lvar_c }{\Mtrix'}
% }~\textbf{\graphGen-while}
% }
% \end{mathpar}
% }
% %
% Below we define the valid data flow matrix, to have a clear understanding on the data flow generating algorithm:
% \begin{defn}[Valid Matrix]
% For a labelled variables $\lvar$, $\lvar \vDash (\Mtrix,\flag)$ iff the cardinality of $\lvar$ equals to the one of $\flag$, $|\lvar| = |\flag|$ 
% and the matrix $\Mtrix$ is of size $|\flag| \times |\flag|$.
% \end{defn}
% %
% \todo{Improvement if possible: Combining reachability bounds analysis into the static dependency analysis algorithm above, rather than adopting an external tool entirely.}
% %
% \paragraph{Reachability Bounds}
% Given a program $c$ with its labelled variables $\lvar$,
% we use the $\rb({x}, {c})$ algorithm, from paper \cite{10.1145/1806596.1806630}, to estimate the reachability bound for each variable ${x} \in \lvar$. 
% The input of $\rb$ is a program ${c}$ in SSA language and a variable ${x} $ from ${c}$.
% The output of $\rb({x}, {c})$ is an integer representing the reachability bound of ${x}$ in ${c}$.
% %

% %
% The following example programs ${c}2$ and ${c}3$ with while loop illustrate how the algorithm works.
% The collected labelled variables, $\lvar_{{c}2}$ and $\lvar_{{c}3}$,
% data flow matrix $\Mtrix_{{c}2}$ and  $\Mtrix_{{c}3}$
% and variable flags $\flag_{{c}2}$ and $\flag_{{c}3}$
% for program ${c}2$ and ${c}3$
% are presented in the right hand side.
%
% \[
% {{c}2 \triangleq
% \begin{array}{l}
% \left[{ x_1} \leftarrow \query(1)  \right]^1 ; 
% \\
% \left[{i_1} \leftarrow 0 \right]^2 ; 
% \\
% \ewhile
% ~ [{i_1} < 2]^3
% 	\\
% ~{[ x_3,x_1 ,x_2 ], [i_3, i_1, i_2] }
% ~ \edo 
% \\
% ~ \Big( 
% \left[{y}_1 \leftarrow \query(2) \right]^4;
% \\
% \left[{x_2 \leftarrow y_1  + x_3 } \right]^5;
% \\
% \left[{i_2 \leftarrow 1  + i_3 } \right]^6
% \Big) ; 
% \\
% \left[ {\assign{z_1}{x_3}} + 2  \right]^{7}
% \end{array}
% ,
% ~~~~
% \lvar_{{c}2} = \left [ \begin{matrix}
% {x}_1 \\
% {x}_3 \\
% {y}_1 \\
% {x}_2 \\
% {z}_1 \\
% {i}_1 \\
% {i}_2 \\
% {i}_3 
% \end{matrix} \right ]
% % \Mtrix =  \left[ \begin{matrix}
% %  & (x_1)  & (y_1) & (x_2) & (x_3) &  (z_1) & i_1 & i_2 & i_3\\
% % (x_1) & 0 & 0 & 0 & 0 & 0 & 0 & 0 & 0 \\
% % (y_1) & 0 & 0 & 0 & 0 & 0 & 1 & 1 & 1 \\
% % (x_2) & 0 & 1 & 0 & 1 & 0 & 1 & 1 & 1 \\
% % (x_3) & 1 & 0  & 1& 0 & 0 & 1 & 1 & 1 \\
% % (z_1) & 0 & 0 & 0 & 1 & 0 & 0 & 0 & 0 \\
% % (i_1) & 0 & 0 & 0 & 0 & 0 & 0 & 0 & 0 \\
% % (i_2) & 0 & 1 & 0 & 1 & 0 & 1 & 0 & 1 \\
% % (i_3) & 1 & 0  & 1& 0 & 0 & 1 & 1 & 1 \\
% % \end{matrix} \right]
% ,
% ~~~~~~
% \Mtrix_{{c}2} =  \left[ \begin{matrix}
% 0 & 0 & 0 & 0 & 0 & 0 & 0 & 0 \\
% 0 & 0 & 0 & 0 & 0 & 1 & 1 & 1 \\
% 0 & 1 & 0 & 1 & 0 & 1 & 1 & 1 \\
% 1 & 0  & 1& 0 & 0 & 1 & 1 & 1 \\
% 0 & 0 & 0 & 1 & 0 & 0 & 0 & 0 \\
% 0 & 0 & 0 & 0 & 0 & 0 & 0 & 0 \\
% 0 & 1 & 0 & 1 & 0 & 1 & 0 & 1 \\
% 1 & 0  & 1& 0 & 0 & 1 & 1 & 1 \\
% \end{matrix} \right]
% ,
% ~~~~
% \flag_{{c}2} = \left [ \begin{matrix}
% 1 \\
% 2 \\
% 1 \\
% 2 \\
% 0 \\
% 0 \\
% 2 \\
% 1 
% \end{matrix} \right ]
% }
% \]
% %
% %
% \[
% {{{c}3}  \triangleq
% \begin{array}{l}
% \left[{ x}_1 \leftarrow \query(1)  \right]^1 ;
% \\
% \left[{i_1} \leftarrow 1 \right]^2 ; 
% \\
% \ewhile ~ [i < 0]^{3} ,
% \\
% ~{[ x_3,x_1 ,x_2 ], [i_3, i_1, i_2] }
% ~ \edo
% \\
% ~ \Big( 
% \left[{ y_1} \leftarrow \query(2) \right]^3; \\
% \left[{x_2 \leftarrow y_1  + x_3 } \right]^5
% \Big) ; \\
% \left[ {\assign{z_1}{x_3}} + 2  \right]^{6}
% \end{array},
% ~~~~~~
% \lvar_{{c}3} = \left [ \begin{matrix}
% {x}_1 \\
% {i}_1 \\
% {x}_3 \\
% {i}_3 \\
% {z}_1 \\
% \end{matrix} \right ]
% ,~~~~~~
% \Mtrix_{{c}3}  =  \left[ \begin{matrix}
% 0 & 0 & 0 & 0 & 0 \\
% 0 & 0 & 0 & 0 & 0 \\
% 1 & 0 & 0 & 0 & 0 \\
% 0 & 1 & 0 & 0 & 0 \\
% 0 & 0 & 1 & 0 & 0 \\
% \end{matrix} \right]
% ,~~~~~~
% \flag_{{c}3} = \left [ \begin{matrix}
% 1 \\
% 0 \\
% 2 \\
% 2 \\
% 0 \\
% \end{matrix} \right ]
% }
% \]
% %
% We can now look at the if statement.
% \[ 
% %
% {c}4 \triangleq
% \begin{array}{l}
% 	\left[ {x}_1 \leftarrow \query(1) \right]^1; 
% 	\\
% 	\left[{y}_1 \leftarrow \query(2) \right]^2 ; 
% 	\\
% \eif \;( { x_1 + y_1 == 5} )^3,  \\
% {[ x_4,x_2,x_3 ],[] ,[y_3,y_1,y_2 ]} 
% \\
% \mathsf{then} ~ \left[ 
% {x}_2 \leftarrow \query(3) \right]^4 
% \\
% \mathsf{else} ~ \left[ 
% {x}_3 \leftarrow \query(4) \right]^5 ; 
% \\
% {y}_2 \leftarrow 2 ) \\
% \left[ { z_1 \leftarrow x_4 +y_3 }\right]^6
% \end{array},
% % \]
% % \[
% ~~~~~~
% \lvar_{{c}4} =  \left[ \begin{matrix}
% {x}_1 \\
% {y}_1 \\
% {x}_2 \\
% {x}_3 \\
% {y}_2 \\
% {x}_4 \\
% {y}_3 \\
% {z}_1 \\
% \end{matrix} \right], 
% ~~~~~ 
% \Mtrix_{{c}4} =  \left[ \begin{matrix}
% 0 & 0 & 0 & 0 & 0 & 0 & 0 & 0 \\
% 0 & 0 & 0 & 0 & 0 & 0 & 0 & 0 \\
% 0 & 0 & 0 & 0 & 0 & 0 & 0 & 0 \\
% 0 & 0 & 0 & 0 & 0 & 0 & 0 & 0 \\
% 0 & 0 & 0 & 0 & 0 & 0 & 0 & 0 \\
% 0 & 0 & 1 & 1 & 0 & 0 & 0 & 0 \\
% 0 & 1 & 0 & 0 & 1 & 0 & 0 & 0 \\
% 0 & 0 & 0 & 0 & 0 & 1 & 1 & 0 \\
% \end{matrix} \right], 
% ~~~~~ 
% \flag_{{c}4} = \left [ \begin{matrix}
% 1 \\
% 1 \\
% 1 \\
% 1 \\
% 0 \\
% 0 \\
% 0 \\
% 0 \\
% \end{matrix} \right ]
% \]
%
%
%
%


% By specifying the departure and destination vertices $s$ and $t$, the $\pathssearch(\progG, s, t)$ algorithm will 
% give the number of query vertices on a finite walk from $s$ to $t$, which contains the maximum number of query vertices.
% The pseudo-code of $\pathssearch(\progG, s, t)$ algorithm is defined in the Algorithm \ref{alg:adpt_alg}.
% %
% \begin{algorithm}
% \caption{
% {Walk Search Algorithm ($\pathssearch$)}
% \label{alg:adpt_alg}
% }
% \begin{algorithmic}
% \REQUIRE Weighted Directed Graph $G = (\vertxs, \edges, \weights, \flag)$ with a start vertex $s$ and destination vertex $t$ .
% \STATE  {\bf {bfs $(G, s, t)$}:}  
% \STATE \qquad {\bf init} 
% current node: $c = s$, 
% queue: $q = [c]$, 
% vector recoding if the vertex is visited: 
% visited$ = [0]*|\vertxs|$,
% result: $r$
% \STATE \qquad {\bf while} $q$ isn't empty:
% \STATE \qquad \qquad take the vertex from beginning $c= q.pop()$
% \STATE \qquad \qquad mark $c$ as visited, visited $[c] = 1$
% \STATE \qquad \qquad curr$\kw{flowcapacity}$ = min($\weights$(c), curr$\kw{flowcapacity}$).
% \STATE \qquad \qquad put all unvisited vertex $v$ having directed edge from c into $q$. 
% \STATE \qquad \qquad if $v$ is visited, then there is a circle in the graph, we update the result $r = r + $curr$\kw{flowcapacity}$
% \RETURN $r$
% \end{algorithmic}
% \end{algorithm}
%
%
% \subsection{\todo{Soundness of the \THESYSTEM}}

% {
% 	\begin{thm}[Soundness of the \THESYSTEM].
% 	Given a program ${c}$, we have:
% 	%
% 	\[
% 	\progA({c}) \geq A({c}).
% 	\]
% 	\end{thm}
% }
% {
% \begin{proof}
% Given a program ${c}$, 
% we construct its program-based graph $\progG({c}) = (\vertxs, \edges, \weights, \qflag)$
% by Definition~\ref{def:prog-based_graph}
% According to the Definition \ref{def:prog_adapt}, we have:
% %
% \[
% 	\progA({c}) 
% 	:= \max\left\{ \qlen(k)\ \mid \  k\in \walks(\progG({c}))\right \}.
% \]
% %
% According to the Definition \ref{def:trace-based_adapt}, we have the trace-based adaptivity as follows:
% $$
% A({c}) = \max \big 
% \{ \len(p) \mid {m} \in \mathcal{SM},D \in \dbdom ,p \in \paths(\traceG({c}, \text{D}, {m}) \big \} 
% $$
% %
% Then, we need to show:
% \[
% \max \big 
% \{ \len(p) \mid {m} \in \mathcal{SM},D \in \dbdom ,p \in \paths(\traceG({c}, \text{D}, {m}) \big \} 
% \leq
% \max\left\{ \qlen(k) \ \mid \  k\in \walks(\progG({c}))\right \}
% \]
% %
% It is sufficient to show that:
% \[
% 	\forall p, {m}, D, ~ s.t., ~ p \in \paths(\traceG({c}, \text{D}, {m}),
% 	\exists k \in \walks(\progG({c})) \land 
% 	\len(p) \leq \qlen(k)
% \]
% %
% Taking an arbitrary starting memory $m$ and an arbitrary underlying database $D$,
% we construct a trace-based graph $\traceG({c}, \text{D}, {m}) = (\vertxs, \edges)$ by the definition \ref{def:trace-based_graph}.
% %
% \\
% %
% Let $\midG({c},{m},\text{D}) = \{\midV, \midE, \midF\}$ be the intermediate graph by Definition~\ref{def:midgraph}.
% \\
% By Lemma~\ref{lem:bie_trace_to_mid}, we know:
% \[
% 	\forall p, {m}, D, ~ s.t., ~ p \in \paths(\traceG({c}, \text{D}, {m}),
% 	\exists p' \in \paths(\midG({c},{m},\text{D})) \land 
% 	\len(p) = \len_q(p')
% \]
% %
% Then it is sufficient to show that:
% %
% \[
% 	\forall p, {m}, D, ~ s.t., ~ p \in \paths(\midG({c}, \text{D}, {m}),
% 	\exists k \in \walks(\progG({c})) \land 
% 	\qlen(p) \leq \qlen(k)
% \]
% %
% We prove a stronger statement instead:
% \[
% 	\forall p, {m}, D, ~ s.t., ~ p \in \paths(\midG({c}, \text{D}, {m}),
% 	\exists k \in \walks(\progG({c})) \land 
% 	\qlen(p) = \qlen(k)	
% \]
% %
% %
% By Lemma~\ref{lem:sujv_mid_to_prog}, let $g$ be the surjective function $g: \progV \to \midV$ s.t.:
% %
% $$
% \forall \av \in \midV. ~ \progF(f(\av)) = \midF(\av) 
% \land |\kw{image}(f(\av))| \leq W(f(\av)).
% $$
% %
% %
% % \item(1) $\len(p_{\av_1 \to \av_2}) = \len(k_{f(\av_1) \to f(\av_2)})$
% % %
% % \item(2) $\forall \av \in p_{\av_1 \to \av_2}. ~ f(\av) \in k_{f(\av_1) \to f(\av_2)}$
% % %
% % \item(3) $\forall \av \in p_{\av_1 \to \av_2}. ~ 
% % \kw{image}(f(\av)) \cap {p_{\av_1 \to \av_2}}| = \# \{f(\av) \mid f(\av) \in k_{f(\av_1) \to f(\av_2)}\}$
% %
% Let ${m}$ and $D$ be an arbitrary memory and database $D$,
% taking an arbitrary path $p_{\av_1 \to \av_n} \in \paths(\midG({c}, \text{D}, {m})$ with:
% %
% \item Edge sequence: $(e, \ldots, e_{n-1})$
% %
% \item Vertices sequence: $(\av_1, \ldots, \av_n)$.
% \\
% By Lemma~\ref{lem:sujpathwalk_mid_to_prog}, let $h: \paths(\midG({c}, \text{D}, {m})) \to \walks(\progG({c}))$ be the surjective function satisfies:
% %
% \[
% 	\forall p_{\av_1 \to \av_n} \in \paths(\midG({c}, \text{D}, {m}))
% 	\text{ with }
% 	\left\{
% 	\begin{array}{ll}
% 	\mbox{edge sequence:} & (e, \ldots, e_{n-1})
% 	\\ 
% 	\mbox{vertices sequence:} & (\av_1, \ldots, \av_n)
% 	\end{array}
% 	\right.
% \]
% %
% \[
% 	\exists k_{f(\av_1) \to f(\av_n)} \in \walks(\progG({c}))
% 	\text{ with }
% 	\left\{
% 	\begin{array}{ll}
% 	\mbox{edge sequence:} & (g(e), \ldots, g(e_{n-1}) 
% 	\\ 
% 	\mbox{vertices sequence:} & (f(\av_1), \ldots, f(\av_{n}))
% 	\end{array}
% 	\right.
% \]
% %
% We have the walk:
% $k_{f(\av_1) \to f(\av_n)} \in \walks(\progG({c}))$ with:
% %
% \item Edges sequence: $(g(e), \ldots, g(e_{n-1}) $
% %
% \item Vertices sequence: $(f(\av_1), \ldots, f(\av_{n}))$.
% \\
% It is sufficient to show 
% %
% \[
% 	\qlen(p_{\av_1 \to \av_n}) = \qlen(k_{f(\av_1) \to f(\av_n)})
% \]
% %
% Unfold the definition of $\qlen$, it is suffice to show:
% \[
% \len \big( \av \mid \av \in (\av_1, \ldots, \av_n) \land \midF(\av) = 2 \big) 
% = \len \big(f(\av) \mid f(\av) \in (f(\av_1), \ldots, f(\av_{n})) \land \progF(f(\av)\big) = 2)	
% ~ (a)
% \]
% %
% By Lemma~\ref{lem:sujv_mid_to_prog}, we know:
% %
% \[
% 	\forall \av \in \midV. ~ \midF(\av) = \progF(f(\av)) ~(b)
% \]
% By rewriting $(b)$ in $(a)$, we have this case proved.
% %
% \\
% \todo{
% \begin{defn}[Intermediate Graph $\midG$].
% 	\label{def:midgraph}
% 	\\
% 	$\mathcal{AV}$ : Annotated Variables based on program execution
% 	\\
% 	Given a program ${c}$ with its labelled variables $\lvar$ of length $N$,
% 	a database $D$, a starting memory ${m}$,
% 	s.t., $\Gamma \vdash_{\Mtrix_c, \flag_c} {c}$,
% 	the intermediate graph 
% 	$\midG({c},{m},\text{D}) = (\vertxs, \edges, \flag)$ is defined as:%
% \[
% \begin{array}{rlcl}
% 	\text{Vertices} &
% 	\vertxs & := & \left\{ 
% 	\av \in \mathcal{AV} \middle\vert
% 	\exists {m'},  w', \qtrace, \vtrace.  ~ s.t., ~  
% 	\config{{m} ,{c}, [], [], []}  \to^{*}  \config{{m'} , \eskip, \qtrace, \vtrace, w' }
% 	\land \av \in \vtrace
% 	\right\}
% 	\\
% 	\text{Directed Edges} &
% 	\edges & := & 
% 	\left\{ 
% 	(\av, \av') \in \mathcal{AV} \times \mathcal{AV} 
% 	~ \middle\vert ~
% 	\flowsto(\av, \av', {c},{m},D) 
% 	\right\}
% 	\\
% 	\text{Flags} &
% 	\flag & := & 
% 	\big\{ (\av, n)  \in \vertxs \times \{0, 1, 2\} 
% 	\mid 
% 	(\pi_1(\av) = \lvar(i) \land n = \flag_c(i)); ~
% 	i = 1, \ldots, N
% 	\big\}
% \end{array}
% \]
% \end{defn}
% }
% %
% \\
% \todo{
% 	\begin{lem}[$\vardep$ is Transitive].
% 	\label{lem:vardep_trans}
% 	\\
% 	Given a program ${c}$, with a starting memory ${m}$ and a hidden database $D$, s.t., 
% 	$\config{{m}, {c}, [], [], []} \rightarrow^{*} \config{{m}', \eskip, \qtrace, \vtrace, w} $.
% 	Then, $\forall \av_1, \av_2, \av_3 \in \vtrace$:
% \[
% 	\Big(\vardep(\av_1, \av_2, {c}, {m}, D) \land 
% 	\vardep(\av_2, \av_3, {c}, {m}, D) \Big)
% 	\implies
% 	\vardep(\av_1, \av_3, {c}, {m}, D)
% \]
% 	\end{lem}
% 	\begin{subproof}[of Lemma~\ref{lem:vardep_trans}]
% 	Proof by unfolding and rewriting the Definition~\ref{def:var_dep}.
% 	\end{subproof}
% }
% \\
% %
% \todo{
% 	\begin{lem}[$\flowsto$ is Transitive ??].
% 	\label{lem:flowsto_trans}
% 	\\
% 	Given a program ${c}$ with its labelled variables $\lvar$ of length $N$. 
% 	Then $\forall x_1, x_2, x_3 \in \lvar$
% \[
% 	\Big(\flowsto(x_1, x_2) \land \flowsto(x_2, x_3) \Big)
% 	\implies
% 	\flowsto(x_1, x_3)
% \]
% 	\end{lem}
% 	\begin{subproof}[of Lemma~\ref{lem:flowsto_trans}]
% 	Proof by unfolding the Definition~\ref{def:flowsto}.
% 	\end{subproof}
% }
% \\
% %
% \todo{
% 	\begin{lem}[$\qdep$ Implies $\vardep$].
% 	\label{lem:querydep_vardep}
% 	\\
% 	Given a program ${c}$, with a starting memory ${m}$ and a hidden database $D$, s.t., 
% 	$\config{{m}, {c}, [], [], []} \rightarrow^{*} \config{{m}', \eskip, \qtrace, \vtrace, w} $.
% 	Then, $\forall \av_1, \av_2 \in \qtrace$
% \[
% 	\qdep(\av_1, \av_2, {c}, {m}, D) \implies 
% 	\vardep(\pi_2(\av_1), \pi_2(\av_2), {c}, {m}, D)
% \]
% 	\end{lem}
% 	\begin{subproof}[of Lemma~\ref{lem:querydep_vardep}]
% 	Proof by unfolding the Definition~\ref{def:var_dep} and Definition~\ref{def:query_dep}.
% 	\end{subproof}
% }
% \\
% %
% \todo{
% 	\begin{lem}[$\vardep$ Implies \flowsto].
% 	\label{lem:vardep_flows}
% 	\\
% 	Given a program ${c}$, with a starting memory ${m}$ and a hidden database $D$, s.t., 
% 	$\config{{m}, {c}, [], [], []} \rightarrow^{*} \config{{m}', \eskip, \qtrace, \vtrace, w} $.
% 	Then, $\forall \av_1, \av_2 \in \vtrace$
% \[
% 	\vardep(\av_1, \av_2, {c}, {m}, D) \implies 
% 	\flowsto(\pi_1(\av_1), \pi_1(\av_2))
% \]
% 	\end{lem}
% 	\begin{subproof}[of Lemma~\ref{lem:querydep_vardep}]
% 	Proof by showing contradiction based on the Definition~\ref{def:var_dep} and Definition~\ref{def:flowsto}.
% 	Let $\av_1, \av_2 \in \vtrace$ be 2 arbitrary annotated variables in the variable trace $\vtrace$,
% 	s.t., $\vardep(\av_1, \av_2, {c}, {m}, D)$.
% 	\\
% 	Unfolding the $\vardep$ definition, we have:	
% 	\end{subproof}
% }
% \\
% %
% \todo{
% 	\begin{lem}[Injective Mapping of vertices from $\traceG$ to $\midG$].
% 	\label{lem:injv_trace_to_mid}
% 	\\
% 	$\traceG({c}) = \{\traceV, \traceE\}$
% 	\\
% 	$\midG({c},{m},\text{D}) = \{\midV, \midE, \midF\}$
% \[
% 	\exists ~ \kw{injective} ~ f: \mathcal{AQ} \to \mathcal{AV}. 
% 	~ \forall \av \in \traceV. ~ 
% 	f(\av) \in \midV \land \midF(f(\av)) = 2
% \]
% 	\end{lem}
% \begin{subproof}
% Proving by Definition~\ref{def:midgraph} and Definition~\ref{def:prog_adapt}.
% \end{subproof}
% }
% \\
% \todo{
% 	\begin{lem}[One-on-One Mapping from $\edges$ of $\traceG$ to $\paths(\midG)$].
% 	\label{lem:bie_trace_to_mid}
% 	\\
% 	$\traceG({c}) = \{\traceV, \traceE\}$
% 	\\
% 	$\midG({c},{m},\text{D}) = \{\midV, \midE, \midF\}$
% 	\\
% 	An injective function $ f: \traceV \to \midV$ s.t.,
% 	$\forall \av \in \traceV. ~ \midF(f(\av)) = 2$ 
% \[
% 	\forall e = (\av_1, \av_2) \in \traceE. ~ 
% 	\exists p_{f(\av_1) \to f(\av_2)} \in \paths(\midG({c}, \text{D}, {m}))
% \]
% 	\end{lem}
% \begin{subproof}
% Proving by Lemma~\ref{lem:injv_trace_to_mid} and Definition~\ref{def:midgraph} and acyclic property of $\traceG$ and $\midG$.
% \end{subproof}
% }
% \\
% \todo{
% 	\begin{lem}[Surjective Mapping of Vertices from $\midG$ to $\progG$].
% 	\label{lem:sujv_mid_to_prog}
% 	\\
% 	$\midG({c},{m},\text{D}) = \{\midV, \midE, \midF\}$
% 	\\
% 	$\progG({c}) = \{\progV, \progE, \progF, \progW\}$
% 	\\
% 	$\exists ~ \kw{surjective} ~ f: \mathcal{AV} \to \mathcal{SVAR}.$
% 	%
% \[
% 	\forall \av \in \midV. ~ 
% 	f(\av) \in \progV \land \progF(f(\av)) = \midF(\av) \land
% 	|\kw{image}(f(\av))| \leq W(f(\av))
% \]
% \end{lem}
% \begin{subproof}
% Proving by Definition~\ref{def:midgraph}.
% \end{subproof}
% }
% \\
% \todo{
% 	\begin{lem}[Surjective Mapping from $\edges$ of $\midG)$ to $\edges$ of $\progG$].
% 	\label{lem:suje_mid_to_prog}
% 	\\
% 	$\midG({c},{m},\text{D}) = \{\midV, \midE, \midF\}$
% 	\\
% 	$\progG({c}) = \{\progV, \progE, \progF, \progW\}$
% 	\\
% 	A surjective function $f: \progV \to \midV$ s.t.,
% 	$\forall \av \in \midV. ~ \progF(f(\av)) = \midF(\av) \land |\kw{image}(f(\av))| \leq W(f(\av))$
% 	%
% \[
% 	\exists ~ \kw{surjective} ~ g: \midE \to \progE. ~
% 	\forall e_{mid} = (\av_1, \av_2) \in \midE. 
% 	\exists e_{prog} = ({f(\av_1), f(\av_2)}) \in \progE
% \]
% \end{lem}
% \begin{subproof}
% Proving by Lemma~\ref{lem:sujv_mid_to_prog}.
% \end{subproof}
% }
% \\
% \todo{
% 	\begin{lem}[Surjective Mapping from $\paths(\midG)$ to $\walks(\progG)$].
% 	\label{lem:sujpathwalk_mid_to_prog}
% 	\\
% 	$\midG({c},{m},\text{D}) = \{\midV, \midE, \midF\}$
% 	\\
% 	$\progG({c}) = \{\progV, \progE, \progF, \progW\}$
% 	\\
% 	A surjective function $f: \progV \to \midV$ s.t.,
% 	$\forall \av \in \midV. ~ \progF(f(\av)) = \midF(\av) \land |\kw{image}(f(\av))| \leq W(f(\av))$
% 	\\
% 	A surjective function $g: \midE \to \progE$ s.t.,
% 	$\forall e_{mid} = (\av_1, \av_2) \in \midE. 
% 	\exists e_{prog} = ({f(\av_1) \to f(\av_2)}) \in \progE$
% 	\\
% 	$\exists ~ \kw{surjective} ~ h: \paths(\midG({c},{m},\text{D})) \to \walks(\progG({c}))$ s.t.:
% 	%
% \[
% 	\forall p_{\av_1 \to \av_2} \in \paths(\midG({c},{m},\text{D}))
% 	\text{ with }
% 	\left\{
% 	\begin{array}{ll}
% 	\mbox{edge sequence:} & (e, \ldots, e_{n-1})
% 	\\ 
% 	\mbox{vertices sequence:} & (\av_1, \ldots, \av_n)
% 	\end{array}
% 	\right.
% \]
% \[
% 	\exists k_{f(\av_1) \to f(\av_2)} \in \walks(\progG({c}))
% 	\text{ with }
% 	\left\{
% 	\begin{array}{ll}
% 	\mbox{edge sequence:} & (g(e), \ldots, g(e_{n-1}) 
% 	\\ 
% 	\mbox{vertices sequence:} & (f(\av_1), \ldots, f(\av_{n}))
% 	\end{array}
% 	\right.
% \]
% % \item $(e, \ldots, e_{n-1})$, $(\av_1, \ldots, \av_n)$ are the edges sequence and vertices sequence of $p_{\av_1 \to \av_2}$.
% % then, 
% %  $\len(p_{\av_1 \to \av_2}) = \len(k_{f(\av_1) \to f(\av_2)})$
% % %
% % \item $\forall \av \in p_{\av_1 \to \av_2}. ~ f(\av) \in k_{f(\av_1) \to f(\av_2)}$
% % %
% % \item $\forall \av \in p_{\av_1 \to \av_2}. ~ 
% % \kw{image}(f(\av)) \cap {p_{\av_1 \to \av_2}}| = \# \{f(\av) \mid f(\av) \in k_{f(\av_1) \to f(\av_2)}\}
% % $
% \end{lem}
% %
% \begin{subproof}
% Proving by induction on the length of $l = p_{\av_1 \to \av_2} \in \paths(\midG({c},{m},\text{D}))$, and Lemma~\ref{lem:suje_mid_to_prog} and Lemma~\ref{lem:sujv_mid_to_prog}.
% \caseL{ $l = 1$: }
% \caseL{ $l = l' + 1$, $l' \geq 1$: }
% \end{subproof}
% }
% \end{proof}
% %

% %
% }

\begin{thm}[$\vardep$ implies $\flowsto$]
Given a program $\ssa{c}$, 
\[
	\forall \ssa{x}_1^{l_1}, \ssa{x}_2^{l_2} \in \lvar_{\ssa{c}}.
	\vardep(\ssa{x}_1^{l_1}, \ssa{x}_2^{l_2}, \ssa{c})
	\implies 
	\Big( \exists z_1, \cdots, z_n \in \lvar_{\ssa{c}}. ~ n \geq 0 \land
	\flowsto(x_1^{l_1}, z_1) 
	\land \cdots \land \flowsto(z_n, \ssa{x}_2^{l_2}) \Big)
\]
\end{thm}
\begin{proof}
Unfolding $\vardep(\ssa{x}_1^{l_1}, \ssa{x}_2^{l_2}, \ssa{c})$ by Definition~\ref{def:var_dep},
we get:
\[
\exists \event_1, \event_2 \in \eventset^{\asn}, D \in \dbdom. ~
\projl{\event_1} = (\ssa{x}_1, l_1)
\land
\projl{\event_2} = (\ssa{x}_2, l_2)
\land 
\eventdep(\event_1, \event_2, c, D)
\]
%
Unfolding $\eventdep(\event_1, \event_2, c, D)$ by Definition~\ref{def:event_dep}, we have:
\[
\eventdep^{val}(\event_1, \event_2, c, D) ~ (a) 
\lor
\Big(
\exists \event_b \in \eventset^{\test}. ~ \eventdep^{val}(\event_1, \event_b, c, D) 
\land \eventdep^{test}(\event_b, \event_2, c, D) ~ (b)
\Big)
\]
Prove by cases $(a)$ and $(b)$:
\caseL{$(a)$}
Unfolding $\eventdep^{val}(\event_1, \event_2, c, D)$ by Definition~\ref{def:event_valdep}, we have:
\[
\begin{array}{ll}
\begin{array}{l}
\forall \vtrace_0, \vcounter_0
\\
\exists \vcounter_1, \vcounter_2, \vcounter_3,
\vcounter_1', \vcounter_2', \vcounter_3', 
\vtrace_1, \vtrace_2, \vtrace_2', \ssa{c}_1, \ssa{c}_2.
\\
  \left(
  \begin{array}{l}   
\config{\ssa{c}, \vtrace_0, \vcounter_0} \rightarrow^{*} 
\config{[\assign{\ssa{x}_1}{\expr_1}]^{l_1} ; \ssa{c}_1, \vtrace_0 \vtrace_1, \vcounter_1}  \rightarrow^{assn}
\\ 
 \config{c_1, \vtrace_0 \vtrace_1 \cdot \event_1, \vcounter_1'} 
  \qquad \rightarrow^{*} 
  \config{[\assign{\ssa{x}_2}{\expr_2 ~ or ~ \query(\qexpr_2)}]^{l_2} 
  \\
  or
  \eif([\sbexpr]^l_2, \cdots) 
  or \ewhile [\sbexpr]^l_2 \cdots; \ssa{c}_2, 
  \vtrace_0 \vtrace_1 \cdot \event_1\vtrace_2, \vcounter_2} 
  \\
  \qquad \rightarrow^{assn/query/test} 
  \config{\ssa{c}_3,  \vtrace_0 \vtrace_1 \cdot \event_1 \vtrace_2 \cdot \event_2, \vcounter_3} 
  % 
 \\ 
 \bigwedge
 \config{c_1, \vtrace_0 \vtrace_1 \cdot \event_1', \vcounter_1} 
  \qquad \rightarrow^{*} 
  \config{[\action]^{l_2} ; \ssa{c}_2, \vtrace_0 \vtrace_1 \cdot \event_1 \vtrace_2', \vcounter_2'} 
  \\
  \qquad \rightarrow^{assn/query/test} 
  \config{\ssa{c}_2,  \vtrace_0 \vtrace_1 \cdot \event_1 \vtrace_2' \cdot \event_2', \vcounter_3'} 
\\
\bigwedge
\event_2 \neq_{v} \event_2'
\end{array}
\right)
\end{array} 
&
\end{array}
 \]
 %
 By inversion lemma, we know $\exists \expr_1 or \qexpr_1$, $\exists \expr_2 or \qexpr_2$
 \[
\begin{array}{ll}
\begin{array}{l}
\forall \vtrace_0, \vcounter_0
\\
\exists \vcounter_1, \vcounter_2, \vcounter_3,
\vcounter_1', \vcounter_2', \vcounter_3', 
\vtrace_1, \vtrace_2, \vtrace_2', \ssa{c}_1, \ssa{c}_2.
\\
  \left(
  \begin{array}{l}   
\config{\ssa{c}, \vtrace_0, \vcounter_0} \rightarrow^{*} 
\config{[\assign{\ssa{x}_1}{\expr_1~ or ~ \query(\qexpr_1)}]^{l_1} ; \ssa{c}_1, \vtrace_0 \vtrace_1}  \rightarrow^{assn}
\\ 
 \config{c_1, \vtrace_0 \vtrace_1 \cdot \event_1} 
  \qquad \rightarrow^{*} 
  \config{[\assign{\ssa{x}_2}{\expr_2 ~ or ~ \query(\qexpr_2)}]^{l_2};\ssa{c}_2, 
  \vtrace_0 \vtrace_1 \cdot \event_1\vtrace_2} 
  \\
  \qquad \rightarrow^{assn/query} 
  \config{\ssa{c}_3,  \vtrace_0 \vtrace_1 \cdot \event_1 \vtrace_2 \cdot \event_2} 
  % 
 \\ 
 \bigwedge
 \config{c_1, \vtrace_0 \vtrace_1 \cdot \event_1'} 
  \qquad \rightarrow^{*} 
  \config{[\action]^{l_2} ; \ssa{c}_2, \vtrace_0 \vtrace_1 \cdot \event_1 \vtrace_2'} 
  \\
  \qquad \rightarrow^{assn/query/test} 
  \config{\ssa{c}_2,  \vtrace_0 \vtrace_1 \cdot \event_1 \vtrace_2' \cdot \event_2'} 
\\
\bigwedge
\event_2 \neq_{v} \event_2'
\end{array}
\right)
\end{array} 
&
\end{array}
 \]
 By induction on length of $\vtrace_2$, $m = |\vtrace_2|$, it is sufficient to show:
 \[\Big( \exists z_1, \cdots, z_m \in \lvar_{\ssa{c}}. ~ m \geq 0 \land
	\flowsto(x_1^{l_1}, z_1) 
	\land \cdots \land \flowsto(z_m, \ssa{x}_2^{l_2}) \Big)
	\]
 \caseL{$m = 0$}
\[
	\config{[\assign{\ssa{x}_1}{\expr_1~ or ~ \query(\qexpr_1)}]^{l_1} ; \ssa{c}_1, \vtrace_0 \vtrace_1}  \rightarrow^{assn}
\\ 
 \config{c_1, \vtrace_0 \vtrace_1 \cdot \event_1} 
  \qquad \rightarrow^{\eskip^*} 
  \config{[\assign{\ssa{x}_2}{\expr_2 ~ or ~ \query(\qexpr_2)}]^{l_2};\ssa{c}_2, 
  \vtrace_0 \vtrace_1 \cdot \event_1} 
  \\
  \qquad \rightarrow^{assn/query} 
  \config{\ssa{c}_3,  \vtrace_0 \vtrace_1 \cdot \event_1\cdot \event_2} 
\]
and
\[
	 \config{c_1, \vtrace_0 \vtrace_1 \cdot \event_1'} 
  \qquad \rightarrow^{\eskip^*} 
  \config{[\action]^{l_2} ; \ssa{c}_2, \vtrace_0 \vtrace_1 \cdot \event_1} 
  \\
  \qquad \rightarrow^{assn/query} 
  \config{\ssa{c}_2,  \vtrace_0 \vtrace_1 \cdot \event_1' \cdot \event_2'} 
\]
By operational semantics rule \rname{ssa-assn} and \rname{ssa-query}, we have 
\[
	x_1 \in VAR(\expr_2) \lor x_1 \in VAR(\qexpr_2)
\]
By $\flowsto$ definition, we have:
\[
\flowsto(x_1, x_2)
\]
%
 \caseL{$m = 1$}
 let $\vtrace_2 = \cdot \event'$, there are 2 cases:
 \subcaseL{$\event' \in \eventset^{\test}$}
\[
\begin{array}{l}
	\config{[\assign{\ssa{x}_1}{\expr_1~ or ~ \query(\qexpr_1)}]^{l_1} ; \ssa{c}_1, \vtrace_0 \vtrace_1}  \rightarrow^{assn}
\\ 
 \config{c_1, \vtrace_0 \vtrace_1 \cdot \event_1} 
  \qquad \rightarrow^{\eskip^*} 
  \config{[\assign{\ssa{x}_2}{\expr_2 ~ or ~ \query(\qexpr_2)}]^{l_2};\ssa{c}_2, 
  \vtrace_0 \vtrace_1 \cdot \event_1} 
  \\
  \qquad \rightarrow^{assn/query} 
  \config{\ssa{c}_3,  \vtrace_0 \vtrace_1 \cdot \event_1\cdot \event_2} 
\end{array}
\]
and
\[
\begin{array}{l}
\config{c_1, \vtrace_0 \vtrace_1 \cdot \event_1'} 
  \qquad \rightarrow^{\eskip^*} 
  \config{[\assign{\ssa{x}_2}{\expr_2 ~ or ~ \query(\qexpr_2)}]^{l_2} ; \ssa{c}_2, \vtrace_0 \vtrace_1 \cdot \event_1} 
  \\
  \qquad \rightarrow^{assn/query} 
  \config{\ssa{c}_2,  \vtrace_0 \vtrace_1 \cdot \event_1' \cdot \event_2'}
\end{array} 
\]
By operational semantics rule \rname{ssa-assn} and \rname{ssa-query}, we have 
\[
	x_1 \in VAR(\expr_2) \lor x_1 \in VAR(\qexpr_2)
\]
By $\flowsto$ definition, we have:
\[
\flowsto(x_1, x_2)
\]
\subcaseL{$\event' \in \eventset^{\asn}$}
\[
	\flowsto(x_1, x_2) \lor \flowsto(x_1, z) \land \flowsto(z, x_2)
\]
this case is proved.

\caseL{$m = n + 1, n \geq 0$}
By induction hypothesis on $n$, we have $\exists y_1, \cdots, y_n$ s.t.:
\[
\begin{array}{ll}
			& \flowsto(x_1, x_2) \\
	\lor 	& \flowsto(x_1, y_1) \land \flowsto(y_1, x_2)\\
	\lor 	& \flowsto(x_1, y_1) \land \flowsto(y_1, z) \land \flowsto(z, x_2) \\
	\lor 	& \flowsto(x_1, z) \land \flowsto(z, y_1) \land \flowsto(y_1, x_2) \\
	\lor 	& \cdots \\
	\lor 	& \flowsto(x_1, y_1) \land \cdots \land \flowsto(y_n, z) \land \flowsto(z, x_2) \\
\end{array}
\]
This case is proved.

\caseL{$(b)$}
Unfolding $\eventdep^{val}(\event_1, \event_b, c, D)$ and $\eventdep^{test}(\event_b, \event_2, c, D)$, we have:
\[
	\ldots
\]
induction on length of $m = |\vtrace_2^v| + |\vtrace_2^b|$:
it is sufficient to show:
 \[\Big( \exists z_1, \cdots, z_m \in \lvar_{\ssa{c}}. ~ m \geq 0 \land
	\flowsto(x_1^{l_1}, z_1) 
	\land \cdots \land \flowsto(z_m, \ssa{x}_2^{l_2}) \Big)
	\]
 \caseL{$m = 0$}
\[
\begin{array}{l}
	\config{[\assign{\ssa{x}_1}{\expr_1~ or ~ \query(\qexpr_1)}]^{l_1} ; \ssa{c}_1, \vtrace_0 \vtrace_1}  \rightarrow^{assn}
\\ 
 \config{c_1, \vtrace_0 \vtrace_1 \cdot \event_1} 
  \qquad \rightarrow^{\eskip^*} 
  \config{\eif([b]^{l_2}, c_t, c_f) or \ewhile([b]^{l_2}, c');\ssa{c}_2, 
  \vtrace_0 \vtrace_1 \cdot \event_1} 
  \\
  \qquad \rightarrow^{test} 
  \config{\ssa{c}_3,  \vtrace_0 \vtrace_1 \cdot \event_1 \cdot \event_b} 
 \end{array}
\]
and
\[
	 \config{c_1, \vtrace_0 \vtrace_1 \cdot \event_1'} 
  \qquad \rightarrow^{\eskip^*} 
  \config{\eif([b]^{l_2}, c_t, c_f) or \ewhile([b]^{l_2}, c');\ssa{c}_2, 
  \vtrace_0 \vtrace_1 \cdot \event_1} 
  \\
  \qquad \rightarrow^{test} 
  \config{\ssa{c}_2,  \vtrace_0 \vtrace_1 \cdot \event_1' \cdot \event_b'} 
\]
By operational semantics rule \rname{ssa-if} and \rname{ssa-while}, we have 
\[
	x_1 \in VAR(b)
\]
By $\eventdep^{test}(\event_b, \event_2, c, D)$, we have:
\[
\begin{array}{l}
	\config{\eif([b]^{l_b}, c_t, c_f) or \ewhile([b]^{l_b}, c'); \ssa{c}_1, \vtrace_0 \vtrace_1}  \rightarrow^{\test}
\\ 
 \config{c_1, \vtrace_0 \vtrace_1 \cdot \event_b} 
  \qquad \rightarrow^{\eskip^*} 
  \config{[\assign{\ssa{x}_2}{\expr_2 ~ or ~ \query(\qexpr_2)}]^{l_2};\ssa{c}_2, 
  \vtrace_0 \vtrace_1 \cdot \event_1} 
  \\
  \qquad \rightarrow^{\asn} 
  \config{\ssa{c}_3,  \vtrace_0 \vtrace_1 \cdot \event_b \cdot \event_2} 
 \end{array}
\]
and
\[
\begin{array}{l}
	\config{c_1, \vtrace_0 \vtrace_1 \cdot \event_b'} 
  \qquad \rightarrow^{\eskip^*} 
  \config{[\assign{\ssa{x}_2}{\expr_2 ~ or ~ \query(\qexpr_2)}]^{l_2};\ssa{c}_2, 
  \vtrace_0 \vtrace_1 \cdot \event_1} 
  \\
  \qquad \rightarrow^{\asn} 
  \config{\ssa{c}_3,  \vtrace_0 \vtrace_1 \cdot \event_b'} 
 \end{array}\]
%
By $\flowsto$ definition, we have:
\[
\flowsto(x_1, x_2)
\]
%
 \caseL{$m = 1$}
 let $\vtrace_2 = \cdot \event'$, there are 2 cases:
 \subcaseL{$\event' \in \eventset^{\test}$}
\[
	\config{[\assign{\ssa{x}_1}{\expr_1~ or ~ \query(\qexpr_1)}]^{l_1} ; \ssa{c}_1, \vtrace_0 \vtrace_1}  \rightarrow^{assn}
\\ 
 \config{c_1, \vtrace_0 \vtrace_1 \cdot \event_1} 
  \qquad \rightarrow^{\eskip^*} 
  \config{[\assign{\ssa{x}_2}{\expr_2 ~ or ~ \query(\qexpr_2)}]^{l_2};\ssa{c}_2, 
  \vtrace_0 \vtrace_1 \cdot \event_1} 
  \\
  \qquad \rightarrow^{assn/query} 
  \config{\ssa{c}_3,  \vtrace_0 \vtrace_1 \cdot \event_1\cdot \event_2} 
\]
and
\[
	 \config{c_1, \vtrace_0 \vtrace_1 \cdot \event_1'} 
  \qquad \rightarrow^{\eskip^*} 
  \config{[\action]^{l_2} ; \ssa{c}_2, \vtrace_0 \vtrace_1 \cdot \event_1} 
  \\
  \qquad \rightarrow^{assn/query} 
  \config{\ssa{c}_2,  \vtrace_0 \vtrace_1 \cdot \event_1' \cdot \event_2'} 
\]
By operational semantics rule \rname{ssa-assn} and \rname{ssa-query}, we have 
\[
	x_1 \in VAR(\expr_2) \lor x_1 \in VAR(\qexpr_2)
\]
By $\flowsto$ definition, we have:
\[
\flowsto(x_1, x_2)
\]
\subcaseL{$\event' \in \eventset^{\asn}$}
\[
	\flowsto(x_1, x_2) \lor \flowsto(x_1, z) \land \flowsto(z, x_2)
\]
this case is proved.

\caseL{$m = n + 1, n \geq 0$}
By induction hypothesis on $n$, we have $\exists y_1, \cdots, y_n$ s.t.:
\[
\begin{array}{ll}
			& \flowsto(x_1, x_2) \\
	\lor 	& \flowsto(x_1, y_1) \land \flowsto(y_1, x_2)\\
	\lor 	& \flowsto(x_1, y_1) \land \flowsto(y_1, z) \land \flowsto(z, x_2) \\
	\lor 	& \flowsto(x_1, z) \land \flowsto(z, y_1) \land \flowsto(y_1, x_2) \\
	\lor 	& \cdots \\
	\lor 	& \flowsto(x_1, y_1) \land \cdots \land \flowsto(y_n, z) \land \flowsto(z, x_2) \\
\end{array}
\]
This case is proved.
\end{proof}

\subsection{\todo{Soundness of the \THESYSTEM}}
\jl{
	\begin{thm}[Soundness of the \THESYSTEM].
	Given a program $\ssa{c}$, we have:
	%
	\[
	\progA(\ssa{c}) \geq A(\ssa{c}).
	\]
	\end{thm}
}
\begin{proof}
Given a program $\ssa{c}$, 
we construct its program-based graph $\progG(\ssa{c}) = (\vertxs, \edges, \weights, \qflag)$
by Definition~\ref{def:prog-based_graph}
According to the Definition \ref{def:prog_adapt}, we have:
%
\[
	\progA(\ssa{c}) 
	:= \max\left\{ \qlen(k)\ \mid \  k\in \walks(\progG(\ssa{c}))\right \}.
\]
%
According to the Definition \ref{def:trace-based_adapt}, we have the trace-based adaptivity as follows:
$$
A(c) = \max \big 
\{ \qlen(k) \mid D \in \dbdom , k \in \walks(\traceG(c, D) \big \} 
$$
%
Then, we need to show:
\[
\max \big 
\{ \len(p) \mid \ssa{m} \in \mathcal{SM},D \in \dbdom ,p \in \paths(\traceG(\ssa{c}, \text{D}, \ssa{m}) \big \} 
\leq
\max\left\{ \qlen(k) \ \mid \  k\in \walks(\progG(\ssa{c}))\right \}
\]
%
It is sufficient to show that:
\[
	\forall p, \ssa{m}, D, ~ s.t., ~ p \in \paths(\traceG(\ssa{c}, \text{D}, \ssa{m}),
	\exists k \in \walks(\progG(\ssa{c})) \land 
	\len(p) \leq \qlen(k)
\]
%
Taking an arbitrary starting memory $m$ and an arbitrary underlying database $D$,
we construct a trace-based graph $\traceG(\ssa{c}, \text{D}, \ssa{m}) = (\vertxs, \edges)$ by the definition \ref{def:trace-based_graph}.
%
\\
%
Let $\midG(\ssa{c},\ssa{m},\text{D}) = \{\midV, \midE, \midF\}$ be the intermediate graph by Definition~\ref{def:midgraph}.
\\
By Lemma~\ref{lem:bie_trace_to_mid}, we know:
\[
	\forall p, \ssa{m}, D, ~ s.t., ~ p \in \paths(\traceG(\ssa{c}, \text{D}, \ssa{m}),
	\exists p' \in \paths(\midG(\ssa{c},\ssa{m},\text{D})) \land 
	\len(p) = \len_q(p')
\]
%
Then it is sufficient to show that:
%
\[
	\forall p, \ssa{m}, D, ~ s.t., ~ p \in \paths(\midG(\ssa{c}, \text{D}, \ssa{m}),
	\exists k \in \walks(\progG(\ssa{c})) \land 
	\qlen(p) \leq \qlen(k)
\]
%
We prove a stronger statement instead:
\[
	\forall p, \ssa{m}, D, ~ s.t., ~ p \in \paths(\midG(\ssa{c}, \text{D}, \ssa{m}),
	\exists k \in \walks(\progG(\ssa{c})) \land 
	\qlen(p) = \qlen(k)	
\]
%
%
By Lemma~\ref{lem:sujv_mid_to_prog}, let $g$ be the surjective function $g: \progV \to \midV$ s.t.:
%
$$
\forall \av \in \midV. ~ \progF(f(\av)) = \midF(\av) 
\land |\kw{image}(f(\av))| \leq W(f(\av)).
$$
%
%
% \item(1) $\len(p_{\av_1 \to \av_2}) = \len(k_{f(\av_1) \to f(\av_2)})$
% %
% \item(2) $\forall \av \in p_{\av_1 \to \av_2}. ~ f(\av) \in k_{f(\av_1) \to f(\av_2)}$
% %
% \item(3) $\forall \av \in p_{\av_1 \to \av_2}. ~ 
% \kw{image}(f(\av)) \cap {p_{\av_1 \to \av_2}}| = \# \{f(\av) \mid f(\av) \in k_{f(\av_1) \to f(\av_2)}\}$
%
Let $\ssa{m}$ and $D$ be an arbitrary memory and database $D$,
taking an arbitrary path $p_{\av_1 \to \av_n} \in \paths(\midG(\ssa{c}, \text{D}, \ssa{m})$ with:
%
\item Edge sequence: $(e, \ldots, e_{n-1})$
%
\item Vertices sequence: $(\av_1, \ldots, \av_n)$.
\\
By Lemma~\ref{lem:sujpathwalk_mid_to_prog}, let $h: \paths(\midG(\ssa{c}, \text{D}, \ssa{m})) \to \walks(\progG(\ssa{c}))$ be the surjective function satisfies:
%
\[
	\forall p_{\av_1 \to \av_n} \in \paths(\midG(\ssa{c}, \text{D}, \ssa{m}))
	\text{ with }
	\left\{
	\begin{array}{ll}
	\mbox{edge sequence:} & (e, \ldots, e_{n-1})
	\\ 
	\mbox{vertices sequence:} & (\av_1, \ldots, \av_n)
	\end{array}
	\right.
\]
%
\[
	\exists k_{f(\av_1) \to f(\av_n)} \in \walks(\progG(\ssa{c}))
	\text{ with }
	\left\{
	\begin{array}{ll}
	\mbox{edge sequence:} & (g(e), \ldots, g(e_{n-1}) 
	\\ 
	\mbox{vertices sequence:} & (f(\av_1), \ldots, f(\av_{n}))
	\end{array}
	\right.
\]
%
We have the walk:
$k_{f(\av_1) \to f(\av_n)} \in \walks(\progG(\ssa{c}))$ with:
%
\item Edges sequence: $(g(e), \ldots, g(e_{n-1}) $
%
\item Vertices sequence: $(f(\av_1), \ldots, f(\av_{n}))$.
\\
It is sufficient to show 
%
\[
	\qlen(p_{\av_1 \to \av_n}) = \qlen(k_{f(\av_1) \to f(\av_n)})
\]
%
Unfold the definition of $\qlen$, it is suffice to show:
\[
\len \big( \av \mid \av \in (\av_1, \ldots, \av_n) \land \midF(\av) = 2 \big) 
= \len \big(f(\av) \mid f(\av) \in (f(\av_1), \ldots, f(\av_{n})) \land \progF(f(\av)\big) = 2)	
~ (a)
\]
%
By Lemma~\ref{lem:sujv_mid_to_prog}, we know:
%
\[
	\forall \av \in \midV. ~ \midF(\av) = \progF(f(\av)) ~(b)
\]
By rewriting $(b)$ in $(a)$, we have this case proved.

\end{proof}

\newpage
\bibliographystyle{plain}
\bibliography{adaptivity.bib}

\end{document}



