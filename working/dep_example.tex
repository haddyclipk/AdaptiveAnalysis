\begin{example}[Execution Trace of a Program with While Command].
\\
\[
\ewhile [(x > 0)]{}^0 \edo [x = x - 1]{}^1;
\]
Let $\trace_0 \in \mathbb{T}$ be the initial trace, 
without loss of generalization, let $\env(\trace_0) x = 1$.
\\
By operational semantics rules, we have following evaluation:
  \begin{mathpar}
\inferrule
  {
   \vtrace_0, (x > 0) \barrow \etrue
   \and 
   \event = ((x > 0), 0, 1, \etrue)
  }
  {
  \config{\ssa{\ewhile [(x > 0)]{}^0, 0 \edo [x = x - 1]{}^1;, \vtrace_0}}
  \\
  \xrightarrow{} 
  \config{\ssa{[x = x - 1]{}^1; \ewhile [(x > 0)]{}^0, 0 \edo [x = x - 1]{}^1;, \vtrace_0 \cdot ((x > 0), 0, 1, \etrue)}}
  }
  ~\textbf{while-t}
\and
\inferrule
  {
  \inferrule
  {
	\config{x - 1, \vtrace_0 \cdot ((x > 0), 0, 1, \etrue)} \aarrow 0
  }
   {
   \config{\ssa{[x = x - 1]{}^1;, \vtrace_0 \cdot ((x > 0), 0, 1, \etrue)}}
  \xrightarrow{} 
  \config{\ssa{\eskip;, \vtrace_0 \cdot ((x > 0), 0, 1, \etrue) \cdot (x, 1, 1, 0)}}
   }~\textbf{asn}
  }
  {
  \config{\ssa{[x = x - 1]{}^1; \ewhile [(x > 0)]{}^0, 0 \edo [x = x - 1]{}^1;, \vtrace_0 \cdot \event}}
  \\
  \xrightarrow{} 
  \config{\ssa{\eskip; \ewhile [(x > 0)]{}^0, 0 \edo [x = x - 1]{}^1;, \vtrace_0 \cdot ((x > 0), 0, 1, \etrue) \cdot (x, 1, 1, 0)}}
  }
  ~\textbf{seq1}
\and
\inferrule
  {
  \inferrule
  {
   \vtrace_0, (x > 0) \barrow \efalse
   \and 
   \event = ((x > 0), 0, 2, \efalse)
  }
   {
   \config{\ssa{\ewhile [(x > 0)]{}^0, 1 \edo [x = x - 1]{}^1;, \vtrace_0 \cdot ((x > 0), 0, 1, \etrue)}}
  \\
  \xrightarrow{} 
  \config{\ssa{\eskip;, \vtrace_0 \cdot ((x > 0), 0, 1, \etrue) \cdot (x, 1, 1, 0)\cdot ((x > 0), 0, 2, \efalse)}}
   }~\textbf{while-f}
  }
  {
  \config{\ssa{\eskip; \ewhile [(x > 0)]{}^0, 0 \edo [x = x - 1]{}^1;, \vtrace_0 \cdot ((x > 0), 0, 1, \etrue) \cdot (x, 1, 1, 0)}}
  \\
  \xrightarrow{} 
  \config{\ssa{\eskip;, 
  \vtrace_0 \cdot ((x > 0), 0, 1, \etrue) \cdot (x, 1, 1, 0) \cdot ((x > 0), 0, 2, \efalse)}}
  }
  ~\textbf{seq2}
\end{mathpar}
%
Then we have following execution, where $\trace_0 \in \mathbb{T}$ and $\env(\trace_0) x = 1$.:
\[
\config{\ewhile [(x > 0)]{}^0 \edo [x = x - 1]{}^1;, \trace_0}
	\rightarrow^{*}
\config{\ssa{\eskip;, 
  \vtrace_0 \cdot ((x > 0), 0, 1, \etrue) \cdot (x, 1, 1, 0) \cdot ((x > 0), 0, 2, \efalse)}}
\]
\end{example}
%
%
\clearpage
\begin{example}[Example of Timing Channel under Trace Semantics].
\label{ex:timingdep}
\\
Using the same example from Example.~\ref{ex:excltiming}
\[
	\ewhile  [(x > 0)]{}^1 \edo [x = x - 1;]{}^2  [y = 1;]{}^3
\]
In this example, $y$'s execution times relies on value of $x$. Under the adaptivity scenario, $y$ depends on $x$ (control dependency).
%
This example shows we can derive by Definition.~\ref{def:var_dep}:
\[
	\vardep(x^2, y^3, {\ewhile  [(x > 0)]{}^1 \edo [x = x - 1;]{}^2  [y = 1;]{}^3})
\]
%
\begin{proof}
Let $\trace_0 = \cdot (x, 0, 1, 2) \in \mathbb{T}$, by semantics definition, we have:
%
\begin{equation}
\label{eq:os_timingdep}
\begin{array}{ll}
& \config{\ewhile  [(x > 0)]{}^1 \edo [x = x - 1;]{}^2  [y = 1;]{}^3 , \trace_0} \\
& \rightarrow^\rname{while-t}
\config{[x = x - 1;]{}^2  [y = 1;]{}^3; \ewhile  [(x > 0)]{}^1 \edo [x = x - 1;]{}^2  [y = 1;]{}^3, \trace_0 \cdot ((x > 0), 1, 1, \etrue)} \\
& \rightarrow^\rname{seq1}
\config{[\eskip]{}^2  [y = 1;]{}^3; \ewhile  [(x > 0)]{}^1 \edo [x = x - 1;]{}^2  [y = 1;]{}^3, \trace_0 \cdot ((x > 0), 1, 1, \etrue) \cdot(x, 2, 1, 1)} \\
& \rightarrow^\rname{seq2}
\config{[\eskip]{}^3; \ewhile  [(x > 0)]{}^1 \edo [x = x - 1;]{}^2  [y = 1;]{}^3, \trace_0 \cdot ((x > 0), 1, 1, \etrue) \cdot(x, 2, 1, 1) \cdot(y, 3, 1, 1)} \\
&\rightarrow^\rname{seq2}
\config{[x = x - 1;]{}^2  [y = 1;]{}^3; \ewhile  [(x > 0)]{}^1 \edo [x = x - 1;]{}^2  [y = 1;]{}^3, \trace_0 \cdot ((x > 0), 1, 1, \etrue) \cdot (x, 2, 1, 1) \\
& \quad \cdot(y, 3, 1, 1) \cdot ((x > 0), 1, 2, \etrue)} \\
& \rightarrow^\rname{seq1}
\config{[\eskip]{}^2  [y = 1;]{}^3; \ewhile  [(x > 0)]{}^1 \edo [x = x - 1;]{}^2  [y = 1;]{}^3, \trace_0 \cdot ((x > 0), 1, 1, \etrue) \cdot(x, 2, 1, 1) \\
& \quad \cdot(y, 3, 1, 1) \cdot ((x > 0), 1, 2, \etrue) \cdot(x, 2, 2, 0)} \\
& \rightarrow^\rname{seq2}
\config{[\eskip]{}^3; \ewhile  [(x > 0)]{}^1 \edo [x = x - 1;]{}^2  [y = 1;]{}^3, \trace_0 \cdot ((x > 0), 1, 1, \etrue) \cdot (x, 2, 1, 1) \\
& \quad \cdot(y, 3, 1, 1) \cdot ((x > 0), 1, 2, \etrue) \cdot(x, 2, 2, 0) \cdot(y, 3, 2, 1)} \\
& \rightarrow^\rname{seq2}
\config{[\eskip]{}^1;, \trace_0 \cdot ((x > 0), 1, 1, \etrue) \cdot(x, 2, 1, 1) \\
& \quad \cdot(y, 3, 1, 1) \cdot ((x > 0), 1, 1, \etrue) \cdot(x, 2, 2, 0) \cdot(y, 3, 2, 1) \cdot ((x > 0), 1, 3, \efalse)} \\
\end{array}
\end{equation}
%
Let $\event_1 = (x, 2, 1, 1)$,  $\event_1' = (x, 2, 1, 0)$ and $\event_b = ((x > 0), 1, 2, \etrue)$, then we have another execution as follows:
\[
\begin{array}{ll}
& \config{[\eskip]{}^2  [y = 1;]{}^3; \ewhile  [(x > 0)]{}^1 \edo [x = x - 1;]{}^2  [y = 1;]{}^3, \trace_0 \cdot ((x > 0), 1, 1, \etrue) \cdot \event_1'} \\
& \rightarrow^\rname{seq2}
\config{[\eskip]{}^3; \ewhile  [(x > 0)]{}^1 \edo [x = x - 1;]{}^2  [y = 1;]{}^3, \trace_0 \cdot ((x > 0), 1, 1, \etrue) \cdot \event_1' \cdot(y, 3, 1, 1)} \\
&\rightarrow^\rname{seq2}
\config{[\eskip;]{}^1, \trace_0 \cdot ((x > 0), 1, 1, \etrue) \cdot \event_1' \cdot(y, 3, 1, 1) \cdot ((x > 0), 1, 2, \efalse)} \\
\end{array}
\]
%
Then, we have $\event_b' = ((x > 0), 1, 2, \efalse)$ where $\event_b \sigeq \event_b'$ and $\event_b \eventneq \event_b'$.
\\
By Definition~\ref{def:event_valdep}, we have:
\\
\[
	\eventdep^{val}(\event_1, \event_b, \ewhile  [(x > 0)]{}^1 \edo [x = x - 1;]{}^2  [y = 1;]{}^3, D)
\]
\\
Let $\event_2 = (y, 3, 2, 1)$ then we have another execution as follows:
\[
\begin{array}{ll}
\config{[\eskip;]{}^1, \trace_0 \cdot ((x > 0), 1, 1, \etrue) \cdot \event_1' \cdot(y, 3, 1, 1) \cdot ((x > 0), 1, 2, \efalse)} \\
\rightarrow^{*} 
\config{[\eskip;]{}^1, \trace_0 \cdot ((x > 0), 1, 1, \etrue) \cdot \event_1' \cdot(y, 3, 1, 1) \cdot \event_b'}
\end{array}
\]
%
where $\event_2 \notsigin \trace_0 \cdot ((x > 0), 1, 1, \etrue) \cdot \event_1' \cdot(y, 3, 1, 1) \cdot \event_b'$
\\
By Definition~\ref{def:event_testdep}, we have:
\\
\[
	\eventdep^{\test}(\event_b, \event_2, \ewhile  [(x > 0)]{}^1 \edo [x = x - 1;]{}^2  [y = 1;]{}^3, D)
\]
\\

By Definition~\ref{def:event_dep}, we have: 
\[
	\eventdep(\event_1, \event_2, \ewhile  [(x > 0)]{}^1 \edo [x = x - 1;]{}^2  [y = 1;]{}^3, D)
\]
%
By Definition~\ref{def:event_dep}, we have:
\[
	\vardep(x^2, y^3, {\ewhile  [(x > 0)]{}^1 \edo [x = x - 1;]{}^2  [y = 1;]{}^3})
\]
%
%
\end{proof}
\end{example}
%
\clearpage
\begin{example}[Excluding the Over approximation in Example.~\ref{eq:sem_timingoverapp}].
\\
If $\mathsf{diff}(\omega, \omega')$ (in \cite{cousot2019abstract} Equation~(2)) simply includes timing channel,(i.e., $\omega$ is a strict prefix of $\omega'$) as follows:
\[
	\mathsf{diff}(\omega, \omega') \triangleq \exists \omega_0, \omega_1, \omega_1', v, v' 
	\st \bigvee \left\{
	\begin{array}{lr}
	(\omega = \omega_0 \cdot v \omega_1
		\land \omega' = \omega_0 \cdot v' \omega_1' \land v \neq v') & \mbox{original definition} \\
	(\omega = \omega' \cdot v \cdot \omega_1) & \mbox{including timing channel} \\
	\end{array}
	\right\}
\] 
then by Definition~2 (in \cite{cousot2019abstract}), there is a over approximation example:
\[
	\ewhile {}^2 (x > 0) {}^3 y = 1; 
\]
Let $z \in \mathbb{V}\setminus \{x\}$ be arbitrary variable different from $x$,
in this example, $y$ doesn't rely on $z$. 
However, according to value dependency defined in Definition~2 \cite{cousot2019abstract} we can derive 
\[
	z \rightsquigarrow^{2}_{\ewhile {}^2 (x > 0) {}^3 y = 1;} y
\]
%
\begin{proof}
By semantics definition, we have:
%
%
\end{proof}
\end{example}
%
