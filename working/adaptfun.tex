The adaptivity from the static program analysis result is defined firstly. 
Then the algorithms for {\THESYSTEM}.
{\THESYSTEM} consists of three phases: 
\begin{enumerate}
    \item An algorithm to generate a precise data control flow graph
    \item An algorithm to perform a Reachability number analysis to calculate the weight of each node in the graph generated in phase 1.
    \item An algorithm to find the appropriate path in the weighted data control flow graph
\end{enumerate}
%
\subsection{Adaptivity Based on Program Analysis in \THESYSTEM}
%
\subsubsection{$\flowsto$}
Given a program  ${c}$ with its labelled variables $\lvar_c$,
and two variables ${x^i}, y^j  \in \lvar_c $ 
% showing up as $i$-th, $j$-th elements in $\lvar$ 
% (i.e., ${x} = \lvar(i)$ and ${y} = \lvar(j)$),
we say $y^j$ flows to ${x^i}$ in ${c}$ if and only if 
the value of $y^j$ directly or indirectly influence the evaluation of the value of ${x}$ as follows:
%
\begin{itemize}
\item (Explicit Influence) The program ${c}$ contains either 
a command $[\assign{{x}}{\aexpr}]^i$ or $[\assign{{x}}{\query({\qexpr})}]^i$,
such that ${y}$ shows up as a free variable in $\sexpr$ or ${\qexpr}$.
We use $\flowsto({x^i, y^j, c})$ to denote $y^j$ flows to $x^i$ in ${c}$.
%
\item (Implicit Influence) The program ${c}$ contains either a while loop
command
or if condition command, 
such that $x$ shows up in the guard
and $y$ shows up in the left hand of an assignment command, and this assignment command shows up
 in the body associated to that condition command.
\end{itemize}
%
This is formally defined in \ref{def:flowsto}.
We use $FV(\expr)$, $FV(\sbexpr)$ and $FV(\qexpr)$ denote the set of free variables in 
expression $\expr$, boolean expression $\sbexpr$ and query expression $\qexpr$ respectively.
%
\\
$\live^l(c) \subseteq \lvar_c$ 
is the set of labelled variables.
For every labelled variable $x^l$ in this set, 
the value assigned to that variable
in the assignment command associated to that label is reachable at the entry point of  executing the command of label $l$.
This is formally defined and computed in the first step of the analysis algorithm.
%
\begin{defn}[Data Flows between Assigned Variables ($\flowsto$)].
\label{def:flowsto}
\\
In a program  ${c}$,
an variable ${x^i}  \in \lvar_c $ is in the \emph{flows to} relation with another variable ${y^j} \in \lvar_c$
in ${c}$, denoted as $\flowsto({x^i, y^j, c})$, is defined as follows:
%
\[
\begin{array}{l}
\flowsto({x^i, y^j, c}) \triangleq 
\\
\left( \bigvee
\begin{array}{l}
(\exists \sexpr . ~ [\assign{y}{\sexpr}]^j \in_{c} {c} 
\land {x} \in VAR(\sexpr) \land (x^i \in \live^j(c)))
\\
(\exists {\qexpr}. ~ [\assign{y}{\query({\qexpr})}]^j \in_{c} {c} 
\land x \in VAR({\qexpr}) \land (x^i \in \live^j(c))))
\\
\big(\exists  ~ {c_w}, {(\expr \lor \qexpr)}, \sbexpr, l \in \mathbb{N}. ~
	\ewhile [\sbexpr]^l \edo {c_w} \in_{c} {c}
	\land 
	[{\assign{y}{\expr \lor \query(\qexpr)}}]^{j} \in_{c}  {c_w}
\big) \land {x} \in VAR(\sbexpr) \land (x^i \in \live^l(c)))
\\
\big(
\exists ~ \sbexpr, l \in \mathbb{N}, {c_1}, {c_2}, {\expr}_1, {\expr}_2. ~
	\eif([\sbexpr]^l, {c_1}, {c_2}) \in_{c} {c} \land
	([{\assign{y}{\expr_1}}]^j \in_{c} {c_1} \lor 
	[{\assign{y}{\expr_2}}]^j \in_{c} {c_2})
\land {x} \in VAR(\sbexpr) \land (x^i \in \live^l(c)))
\big)
% \\
% (\exists z^r \in \lvar_c \st \flowsto(x^i, z^r, c) \land \flowsto(z^r, y^j, c))
\end{array}
\right).
\end{array}
\]
%
\end{defn}
%
\subsubsection{Program Analysis Based Dependency Graph}
\begin{defn}
    [Program-Based Dependency Graph].
    \label{def:prog_graph}
    \\
Given a program ${c}$ with its labelled variables $\lvar$ of length $N$, s.t.,
$\Gamma \vdash_{\Mtrix_c, \flag_c} {c}$, 
its program-based graph 
$G({c}) = (\vertxs, \edges, \weights, \flag)$ is. defined as:
\\
\[
\begin{array}{rlcl}
\text{Vertices} &
\vertxs & := & \left\{ 
x^l \in \mathcal{VAR} \times \mathbb{N}
~ \middle\vert ~
x^l \in \lvar({{c}})
\right\}
\\
\text{Directed Edges} &
\edges & := & 
\left\{ 
  ({x}_1^{i}, {x}_2^{j}) \in \mathcal{VAR} \times \mathbb{N} \times (\mathcal{VAR} \times \mathbb{N})
  ~ \middle\vert ~
  \begin{array}{l}
    {x}_1^{i}, {x}_2^{j} \in \vertxs
	\land
    \\
    \exists n \in \mathbb{N}, z_1^{r_1}, \cdots, z_n^{r_n} \in \lvar_{{c}} \st 
    n \geq 0 \land
    \\
    \flowsto(x^i,  z_1^{r_1}, c) 
    \land \cdots \land \flowsto(z_n^{r_n}, y^j, c) 
  \end{array}
\right\}
\\
\text{Weights} &
\weights & := &
% \bigcup
% \begin{array}{l}
	\big\{ (x^l, w) \in \mathcal{VAR} \times \mathbb{N} \times (\mathbb{N} \cup \aexpr)
	\mid
	x^l \in \lvar_{{c}} \land w = \rb(x^l, c)
	\big\} 
	% \\
	% \big\{(x^l, 1)  \in \mathcal{VAR} \times \mathbb{N} \times \{1\} 
	% \mid
	% x^l \in \lvar_{{c}} \land \flag(x^l) = 0
	\big\}
% \end{array} 
\\
\text{Query Flags} &
\qflag & := & 
\left\{(x^l, n)  \in \mathcal{VAR} \times \mathbb{N}  \times \{0, 1\} 
~ \middle\vert ~
 x^l \in \lvar_{c},
 \left\{
\begin{array}{ll}
n = 1 & x^l \in \qvar_{c} \\ 
n = 0 & o.w.
\end{array}
\right\}
\right\}
\end{array}
\]
%
\[
\begin{array}{rlcl}
\text{Vertices} &
\vertxs & := & \left\{ 
x^l 
~ \middle\vert ~
x^l \in \lvar_{{c}}
\right\}
\\
\text{Directed Edges} &
\edges & := & 
\left\{ 
  ({x}_1^{i}, {x}_2^{j}) 
  ~ \middle\vert ~
  \begin{array}{l}
    {x}_1^{i}, {x}_2^{j} \in \lvar_{{c}},
    \\
    \exists n \in \mathbb{N}, z_1^{r_1}, \cdots, z_n^{r_n} \in \lvar_{{c}} \st 
    n \geq 0 \land
    \\
    \flowsto(x^i,  z_1^{r_1}, c) 
    \land \cdots \land \flowsto(z_n^{r_n}, y^j, c) 
  \end{array}
\right\}
\\
\text{Weights} &
\weights & := &
% \bigcup
% \begin{array}{l}
	\big\{ (x^l, w)
	\mid
	x^l \in \lvar_{{c}}, w = \rb(x^l, c)
	\big\} 
	% \\
	% \big\{(x^l, 1)  \in \mathcal{VAR} \times \mathbb{N} \times \{1\} 
	% \mid
	% x^l \in \lvar_{{c}} \land \flag(x^l) = 0
	\big\}
% \end{array} 
\\
\text{Query Flags} &
\qflag & := & 
\left\{(x^l, n)   
~ \middle\vert ~
 x^l \in \lvar_{c},
 x^l \in \qvar_{c} \implies n = 1,
 x^l \notin \qvar_{c} \implies n = 0
%  \left\{
% \begin{array}{ll}
% n = 1 & x^l \in \qvar_{c} \\ 
% n = 0 & o.w.
% \end{array}
% \right\}
\right\}
\end{array}
\]
\end{defn} 
%
Given a program ${c}$, we generate its program-based graph 
$\progG({c}) = (\vertxs, \edges, \weights, \qflag)$.
%
Then the adaptivity bound based on program analysis for ${c}$ is the number of query vertices on a finite walk in $\progG({c})$. This finite walk satisfies:
\begin{itemize}
\item the number of query vertices on this walk is maximum
\item the visiting times of each vertex $v$ on this walk is bound by its reachability bound $\weights(v)$.
\end{itemize}
It is formally defined in \ref{def:prog_adapt}.
%
%
\begin{defn}
[{Program-Based Adaptivity}].
\label{def:prog_adapt}
\\
{
Given a program ${c}$ and its program-based graph 
$\progG({c}) = (\vertxs, \edges, \weights, \qflag)$,
%
the program-based adaptivity for $c$ is defined as%
\[
\progA({c}) 
:= \max
\left\{ \qlen(k)\ \mid \  k\in \walks(\progG({c}))\right \}.
\]
}
\end{defn}  
%
%
% {
% \begin{defn}[Variable Flags ($\flag$)].
% \\
% Given a program  ${c}$ with its labelled variables $\lvar$, the $\flag$ is a vector of the same length as $\lvar$, s.t. for each variable ${x}$ showing up as the $i$-th element in $\lvar$ (i.e., ${x} = \lvar(i)$), 
% $\flag(i) \in \{0, 1, 2\}$ is defined as follows:
% %
% %
% \[
% \flag(i) := 
% \left\{
% \begin{array}{ll}
% 2 & 
% {x^l} \in \lvar_{c} \land 
% (\exists {\qexpr}. ~ s.t., ~
% [\assign{{x}}{\query({\qexpr})}]^l \in_{c} {c})
% \\
% 1 &  
% \begin{array}{l}
% {x^l} \in \lvar_{c} \bigwedge \\
% \left(
% \begin{array}{l}
% \big(\exists  ~ {c'}, {\expr}, \sbexpr, l, l'. ~
% 	\ewhile [\sbexpr]^l \edo {c'} \in_{c} {c}
% 	\land 
% 	[{\assign{x}{\expr}}]^{l'} \in_{c}  {c'}
% \big) \bigvee
% \\
% \big(\exists ~ \sbexpr, l, l_1, l_2, {c_1}, {c_2}, {\expr}_1, {\expr}_2. ~
% 	\eif([\sbexpr]^l, {c_1}, {c_2}) \in_{c} {c} \land
% 	([{\assign{x}{\expr_1}}]^{l1} \in_{c} {c_1} \lor 
% 	[{\assign{x}{\expr_2}}]^{l2} \in_{c} {c_2})
% \big)
% \end{array}
% \right)
% \end{array}
% \\
% 0 & \text{o.w.}
% \end{array}
% \right\}. 
% \] 
% %
% \end{defn}
%
% Operations on $\flag$ are defined as follows:
% \begin{equation}
% \begin{array}{llll}
% {\flag_1 \uplus \flag_2}(i) & := &
% \left\{
% \begin{array}{ll}
% k & k = \max{\big\{\flag_1(i), \flag_2(i)\big\}} 
% \land |\flag_1| = |\flag_2|\\
% 0 & o.w.
% \end{array}\right.
% & i = 1, \cdots, |\flag_1|  
% \\
% {\flag \uplus n}(i) & := & 
% \max\big\{ \flag(i), n \big\} 
% & i = 1, \ldots, |\flag|    
% \\
% \left[ n \right]^k (i) & := &  n
% & i = 1, \ldots, k ~ \land ~ |\left[ n \right]^k| = k
% \end{array}
% \end{equation}
%
%
%
% \begin{defn}[Data Flow Matrix ($\Mtrix$)]
% The data flow matrix $\Mtrix$ of a program $c$ is a matrix of size $|\lvar_c| \times |\lvar_c|$ 
% s.t.,
% %
% \[
% \Mtrix(i, j) \triangleq
% \left\{
% \begin{array}{ll}
% 1	&	\flowsto({x^i, y^j, c}) \\
% 0	& o.w.
% \end{array}
% \right., {x^i}, y^j  \in \lvar_c.
% \]
% %
% \end{defn}
% %
% Operations on the data flow matrices are defined as follows:
% %
% \begin{equation}
% \Mtrix_1 ; \Mtrix_2 
% := \Mtrix_2 \cdot \Mtrix_1 + \Mtrix_1 + \Mtrix_2
% \end{equation}
% %
% Consider the same program $c$ as above, its data flow matrix $\Mtrix$ and $\flag$ for the program $c$ is as follows:
% $$
% {c} = 
% \begin{array}{l}
% \left[{\assign {x_1} {\query(0)}}	\right]^1;
% \\
% \left[{\assign {x_2} {x_1 + 1}}		\right]^2;
% \\
% \left[{\assign {x_3} {x_2 + 2}}		\right]^3
% \end{array}
% ~~~~~~~~~~~~
% \Mtrix
% =  \left[ 
% \begin{matrix}
% 0 & 0 & 0 \\
% 1 & 0 & 0 \\
% 1 & 1 & 0 \\
% \end{matrix} \right] ~ , 
% \flag = \left [ \begin{matrix}
% 1 \\
% 0 \\
% 0 \\
% \end{matrix} \right ]
% $$
% %
% % There are two special matrices used for generating the data flow matrix $\Mtrix$ in the analysis algorithm. They are the left matrix $\lMtrix_i$ and right matrix $\mathsf{R_{(e, i)}}$.

% % Given a program  ${c}$ with its labelled variables $\lvar$ of length $N$,
% % the left matrix $\lMtrix_i$ generates a matrix of $1$ column, $N$ rows, 
% % where the $i$-th row is $1$ and all the other rows are $0$.
% % %
% % \begin{defn}[Left Matrix ($\lMtrix_i$)].
% % \\
% % Given a program  ${c}$ with its labelled variables $\lvar$ of size $N$, 
% % the left matrix $\lMtrix_i$ is defined as follows:
% % \[
% % \lMtrix_i(j) : = 
% % \left
% % \{
% % \begin{array}{ll}
% % 1 & j = i \\
% % 0 & o.w.
% % \end{array}
% % \right.,
% % j = 1, \ldots, N.
% % \]
% % \end{defn}
% % %
% % Given a program  ${c}$ with its labelled variables $\lvar$ of length $N$,
% % the right matrix $\rMtrix_{\expr, i}$ generates a matrix of one row and $N$ columns, 
% % where the locations of free variables in $\expr$ is marked as $1$. 
% % %
% % %
% % \begin{defn}[Right Matrix ($\rMtrix_{\expr}$)].
% % \\
% % Given a program  ${c}$ with its labelled variables $\lvar$ of length $N$, 
% % the right matrix $\rMtrix_{\expr}$ is defined as follows:
% % \[
% % \rMtrix_{\expr}(j) : = 
% % \left\{
% % \begin{array}{ll}
% % 1 & {x} \in FV(\expr) 
% % \\
% % 0 & o.w.
% % \end{array}
% % \right.,
% % {x} = \lvar(j) ~ , ~ j = 1, \ldots, N.
% % \]
% % %
% % %
% % \end{defn}
% % %
% % Using the same example program ${c}$ as above with labelled variables $\lvar = [ {x_1 , x_2 , x_3} ] $,
% % the left and right matrices w.r.t. its $2$-nd command 
% % $\left[{\assign {x_2} {x_1 + 1}}\right]^2$  are as follows:
% % \[
% % \lMtrix_1 = \left[ \begin{matrix}
% % 0   \\
% % 1 	 \\
% % 0   \\
% % \end{matrix}   \right ] 
% % ~~~~~~~~~~~~~~
% % \rMtrix_{{x}_1 + 1}
% % = \left[ \begin{matrix} 
% % 1 & 0 & 0 \\
% % \end{matrix}  \right]
% % \]
% %
% %
% %
\subsection{ $\THESYSTEM$ Analysis Algorithm}
\wq{To do: Add $\THESYSTEM$, a data flow analysis algorithm to scan the program and give a graph.}
{\THESYSTEM} consists of three phases: 
\begin{enumerate}
    \item An algorithm to generate a precise data control flow graph
    \item An algorithm to perform a Reachability number analysis to calculate the weight of each node in the graph generated in phase 1.
    \item An algorithm to find the appropriate path in the weighted data control flow graph
\end{enumerate}

To be precise, we show the details. 
\subsubsection{Phase 1, precise data control flow graph}
There are 3 steps to generate the graph in this phase.
\begin{enumerate}
\item Generation of control flow graph
    \item Reaching definition analysis
   \item  data-control flow analysis
\end{enumerate}

\paragraph{Generate CFG}
 Define $\mathsf{init}$: Command -> label, which returns the initial label of the statement. 
 \[
 \begin{array}{ll}
    init([x := e]^{l})  & = l  \\
     init([x := q(e)]^{l})  & = l \\
     init([skip]^{l})  & = l \\
     init([if [b]^l then C_1 else C_2]^{l})  & = l \\
     init([while [b]^l do C]^{l})  & = l \\
     init(C_1 ; C_2)  & = init(C_1) \\
 \end{array}
 \]
  Define $\mathsf{final}$: Command -> Powerset(label), which returns the final labels of the statement. 
 \[
 \begin{array}{ll}
    final([x := e]^{l})  & = \{l\}  \\
     final([x := q(e)]^{l})  & = \{l\}  \\
     final([skip]^{l})  & = \{l\} \\
     final([if [b]^l then C_1 else C_2]^{l})  & = final(C_1) \cup final(C_2) \\
     final([while [b]^l do C]^{l})  & = \{l\} \footnote{while terminates after b evaluates to false} \\
     final(C_1 ; C_2)  & =  final(C_2) \\
 \end{array}
 \]
 Define block B to be either the command of the form of assignment, skip, or test of the form of $[b]^{l}$.\\
 Define $\mathsf{blocks}$ : command -> Powerset(Block)
 \[
 \begin{array}{ll}
    blocks([x := e]^{l})  & = \{[x := e]^{l}\}  \\
     block([x := q(e)]^{l})  & = \{[x := q(e)]^{l}\}  \\
     blocks([skip]^{l})  & = \{[skip]^{l}\} \\
     blocks([if [b]^l then C_1 else C_2]^{l})  & = {[b]^{l}} \cup blocks(C_1) \cup blocks(C_2) \\
     blocks([while [b]^l do C]^{l})  & = \{[b]^{l}\} \cup blocks(C) \\
     blocks(C_1 ; C_2)  & = blocks(C_1) \cup  blocks(C_2) \\
 \end{array}
 \]
 Define $\mathsf{labels}$ to get the labels of blocks.
 \[
   labels(C) = \{l | [B]^{l} \in blocks(C) \}
 \]  

The control flow graph is generated by edges between labels. Define $\mathsf{flow}$: command -> P (label $\times$ label ).

\[
 \begin{array}{ll}
    flow([x := e]^{l})  & = \emptyset  \\
     flow([x := q(e)]^{l})  & = \emptyset  \\
     flow([skip]^{l})  & = \emptyset \\
     flow([if [b]^l then C_1 else C_2)  & =  flow(C_1) \cup flow(C_2)\cup \{(l, init(C_1)) , (l, init(C_2)) \} \\
     flow([while [b]^l do C)  & =  flow(C) \cup \{(l, init(C)) \} \cup \{(l', l)| l' \in final(C) \} \\
     flow(C_1 ; C_2)  & = flow(C_1) \cup  flow(C_2) \cup \{ (l,init(C_2)) | l \in final(C_1) \} \\
 \end{array}
 \]
 
 \paragraph{Reaching definition analysis}
 Set $?$ to be undefined, $label^{?}$ is label $\cup \{?\}$.\\
 Define $\mathsf{kill}$: blocks -> Powerset(Var $\times$ $label^l$, which produces the set of labelled variables of assignment destroyed by the block.\\
 Define $\mathsf{gen}$: blocks->Powerset(Var $\times$ $label^l$, which generates the set of labelled variables generated by the block.\\
 Define $defs(x)(C)$: Var -> Set of labels, gives all the labels where assigns value to variable x in the target program C.
  \[
 \begin{array}{ll}
    kill([x := e]^{l})  & = \{ (x, ?) \} \cup \{ (x, l') | l' \in defs(x) \} \\
     kill([x := q(e)]^{l})  & = \{ (x, ?) \} \cup \{ (x, l') | l' \in defs(x) \}  \\
     kill([skip]^{l})  & = \emptyset \\
     kill([ [b]^l ]^{l})  & =  \emptyset \\
      gen([x := e]^{l})  & = \{ (x, l) \}  \\
     gen([x := q(e)]^{l})  & = \{ (x, l) \}  \\
     gen([skip]^{l})  & = \emptyset \\
     gen([ [b]^l ]^{l})  & =  \emptyset 
 \end{array}
 \]
 Define $in(l)$, $out(l)$: label -> (var $\times$ $label^l$) for the entry point and exit point of the node $l$ in the control flow graph.
 \[
 \begin{array}{lll}
    in(l)  & = \{ (x, ?) | x is assigned in C  \} & l = init(C)\\
    & \cup \{ out(l')|  | (l',l) \in flow(C) \} & ow \\
     out(l)  & =  gen(B^{l}) \cup \{ in(l) \setminus kill(B^l)  \} & B^l \in blocks(C)   
 \end{array}
 \]
 
The Reaching definition is calculated by the Worklist algorithm as below.
\begin{enumerate}
    \item initial in[l]=out[l]=$\emptyset$
    \item initial in[entry label] = $\emptyset$
    \item initialize a work queue, contains all the blocks in C
    \item while |W| != 0 \\
         pop l in W\\
          old = out[l]\\
          in(l) =  out(l') where (l',l) in flow(C)\\
           out(l) = gen($b^l$) $\cup$ (in(l) - kill($b^l$) ) where $b^l$ in block(C)   \\
          if (old != out(l)) W= W $\cup$ \{l'| (l,l') in flow(C)\}\\
          end while
\end{enumerate}

 \paragraph{control-data flow graph}
 We build this graph by using the results of Reaching definition analysis, to be specific, in(l) for every label. 
 
 Define dcgd: Command -> Labelled VAR $\times$ 
 Labelled VAR. dcgd is short for data control
 dependency graph. $RD_{in}$ is the result of reaching definition in last step. 
 
  \[
 \begin{array}{ll}
    dcdg([x := e]^{l})  & = \{ (y^i, x^l) | y \in VAR(e) \land (y,i) \in RD_{in}(l) \}  \\
     dcdg([x := q(e)]^{l})  & = \{ (y^i, x^l) | y \in VAR(e) \land (y,i) \in RD_{in}(l) \}  \\
     dcdg([skip]^{l})  & = \emptyset \\
     dcdg([if [b]^l then C_1 else C_2)  & =  dcdg(C_1) \cup dcdg(C_2)\\ & \cup \{(x^i,y^j) | x \in VAR(b) \land (x,i) \in RD_{in}(l) \land ([y = \_]^j) \in blocks(C_1) \} \\
     &\cup \{(x^i,y^j) | x \in VAR(b) \land (x,i) \in RD_{in}(l) \land ([y = \_]^j) \in blocks(C_2) \} \\
     dcdg([while [b]^l do C)  & =  dcdg(C) \cup \{(x^i,y^j) | x \in VAR(b) \land (x,i) \in RD_{in}(l) \land ([y = \_]^j) \in blocks(C) \} \\
     dcdg(C_1 ; C_2)  & = dcdg(C_1) \cup  dcdg(C_2) \\
 \end{array}
 \]
 
 

\subsubsection{Phase 2, reachability number analysis}
In last phase, we get a dependency graph whose node is computation blocks uniquely decided by its label. In this phase, we want to add more information to every node in the graph, which is the visit times (how many times this block is exeucted). To be precise, we use static analysis method from \wq{cite}, which is able to "provide the symbolic worst case bound on the number of times a block is reached", let us call it reachability bound analysis. 

This analysis only works to find the symbolic bound of one block (in our graph, it corresponds to one node). The high level idea as follows.
\begin{enumerate}
    \item Build a transition system which describes the relation between variables in this target block and these variables in the successive visit to this block. There are defined translate functions which translate the statement to transition system and the corresponding operations such as the composition of transition systems, merging two transition systems from two control branches and so on. It is worth to mention that it calculates the transitive closure of the transition system obtained from a loop body, which can be analogy to computing the invariant of a loop.  
    \item Use Ranking function which takes the transition system and outputs the bound
\end{enumerate}

\begin{algorithm}
\caption{
{Reachability Bound Analysis ($\rb$)}
\label{alg:rb}
}
\begin{algorithmic}
\REQUIRE the program $C$, the target block label $l$.
\STATE  T  = GenerateTransitionSystem(C,l) 
\STATE B = 1 +ComputeBound(T)
\RETURN B
\end{algorithmic}
\end{algorithm}

The algorithm $GenerateTransitionSystem(C,l)$ can be described as follows. It uses the control flow graph generated from the program $C$, and splits the node marked by $l$ into two nodes $l_1$ and $l_2$ to generate a new control flow graph from the $l_1$ to $l_2$. It translates the node in the graph into a transition systems by the its translate function and replaces node with transition systems. For loops in the graph, the loop itself is replaced by the transitive closure of the transition systems of its body. Finally, the new generated control flow graph can be transformed to a transition system. The transition system is a disjunction of transitions, and every transition is expressed as a conjuction of formulas over program variables $x,y,z$ in the target block (l) and its successive visits $x',y',z'$ in the same block.

The function $computeBound$ takes into a transition system (a disjuction of transitions), and computes the bound. There are ranking functions which take a trasition and return the bound, that is used by $computeBound$. There are some heuristics in compute the bound based on the transitions systems, if interested, please look at the paper for more details.

\paragraph{Preprocessing} Actually, we do not need to do $\rb$ on every node of the graph from phase 1. Only the node associated with a loop is necessary. We will have an preprocessing algorithms to go through the programs and returns the list of labels that needs {$\rb$}. 

\begin{algorithm}
\caption{
{Add weights to dependency graph (the main algorithm of phase 2)}
\label{alg:rb}
}
\begin{algorithmic}
\REQUIRE the program $C$, the dependency graph $G$ from phase 1
\STATE  RBLABELS = PREPROCESSING(C) 
\STATE FOR l $\in$ RBLABELS
\STATE w = $\rb$ (C, l)
\STATE Add weight (G, l, w)
\RETURN G
\end{algorithmic}
\end{algorithm}

\paragraph{Add Weight} we also need to take care about the situation when a bound can not be predicted by {$\rb$}, we need to use another loose analysis to get a loose bound.

\subsubsection{Phase 3, path finding algorithm in weighted graph (graph in phase 1 with weights predicted in phase 2) }
A combination of DFS and BFS Algorithm:
\begin{algorithm}
    \caption{
    {Longest Adaptivity Search Algorithm ($\pathsearch$)}
    \label{alg:adpt_alg}
    }
    \begin{algorithmic}
    \REQUIRE Weighted Directed Graph $G = (\vertxs, \edges, \weights, \flag)$ with a start vertex $s$ and destination vertex $t$ .
    \STATE {\bf init} 
    \\
    current node: $c$, 
    \\
    queue: $q$ : List, add into $a$ an arbitrary v from $\vertxs$. 
    \\
    visited: List of length $|\vertxs|$, initialize with $\efalse$.
    \\
    results: $r$ : List of length $|\vertxs|$, initialize with -1.
    \\
    \STATE {\bf while} $q$ isn't empty:
    \STATE \qquad take the vertex from head $c= q.pop()$
    \STATE \qquad mark $c$ as visited, visited $[c] = 1$. 
    % initialize cycle-adapt = 0.
    \STATE \qquad for all unvisited vertex $v$ having directed edge from c and $\kw{cycle}(v)$:
    % \STATE \qquad \qquad   cycle-adapt$ = \max($cycle-adapt, $\kw{dfs_{refine}(G, v, v)})$;
    \STATE \qquad \qquad $ r[v] = r[c] + \kw{dfs_{refine}(G, v, v)})$; \#\{update r[v] with the cycle-adapt\}
    % \STATE \qquad   $r[c] = r[c] + $cycle-adapt;
    \STATE \qquad for all unvisited vertex $v$ having directed edge from c and $! \kw{cycle}(c)$:
    \STATE \qquad \qquad $r[v] = r[c] + \flag(v)$; 
    $q.add(v)$
    \RETURN max($r$)
    \end{algorithmic}
    \end{algorithm}
    %
    %
    % \begin{algorithm}
    % \caption{
    % {Longest Adaptivity Search Algorithm ($\pathsearch$)}
    % \label{alg:adpt_alg}
    % }
    % \begin{algorithmic}
    % \REQUIRE Weighted Directed Graph $G = (\vertxs, \edges, \weights, \flag)$ with a start vertex $s$ and destination vertex $t$ .
    % \STATE  {\bf {bfs $(G)$}:}  
    % \STATE {\bf init} 
    % \\
    % current node: $c$, 
    % \\
    % queue: $q$ : List, add into $a$ an arbitrary v from $\vertxs$. 
    % \\
    % visited: List of length $|\vertxs|$, initialize with $\efalse$.
    % \\
    % results: $r$ : List of length $|\vertxs|$, initialize with -1.
    % \\
    % currMinFlow: INT, initialize MAXINT.
    % \\
    % queries: INT, initialize 0. \#\{To count the query numbers when we are walking inside a cycle\}
    % \\
    % \STATE \qquad {\bf while} $q$ isn't empty:
    % \STATE \qquad \qquad take the vertex from head $c= q.pop()$
    % \STATE \qquad \qquad mark $c$ as visited, visited $[c] = 1$.
    % \STATE \qquad \qquad {\bf if} $\kw{cycle}(c)$  \#\{we are inside a cycle\}
    % \STATE \qquad \qquad \qquad currMinFlow = min($\weights$(c), currMinFlow).
    % \STATE \qquad \qquad \qquad queries += $\flag(c)$.
    % \STATE \qquad \qquad  \qquad for all unvisited vertex $v$ having directed edge from c:
    % \STATE \qquad \qquad \qquad \qquad r[v] = r[c]; q.add(v)
    % \STATE \qquad \qquad \qquad  {\bf if}  $v$ is visited, then the circle finished
    % \STATE \qquad \qquad \qquad \qquad update the result $r[v] =  \max(r[v], r[c] + $currMinFlow*queries)
    % \STATE \qquad \qquad \qquad \qquad currMinFlow = MAXINT
    % \STATE \qquad \qquad \qquad \qquad queries = 0.  
    % \STATE \qquad \qquad {\bf else} 
    % \STATE \qquad \qquad \qquad for all unvisited vertex $v$ having directed edge from c:
    % \STATE \qquad \qquad \qquad  \qquad $r[v] = \max(r[v], r[c] + \flag(c))$; q.add(v)
    % \RETURN max($r$)
    % \end{algorithmic}
    % \end{algorithm}
    %
    \begin{algorithm}
        \caption{
        {Over-Approximated Adaptivity on Cycle}
        \label{alg:overadp_alg}
        }
        \begin{algorithmic}
        \REQUIRE Weighted Directed Graph $G = (\vertxs, \edges, \weights, \flag)$ with a start vertex $s$ and destination vertex $t$ .
        \STATE  {\bf {$\kw{dfs_{naive}(G, c, s)}$}:}  
        \STATE {\bf init} 
        \\
        current node: $c$, 
        \\
        visited: List of length $|\vertxs|$, initialize with $\efalse$.
        \\
        \STATE {\bf if} $c = s$:
        \RETURN \qquad  $\weights(s)*\flag(s) $.
        \STATE {\bf for}  all vertex $v$ having directed edge from $c$:
        \STATE \qquad {\bf if}  $v$ is unvisited:
        \STATE \qquad \qquad mark $v$ as visited, visited $[v] = 1$.
        \RETURN \qquad \qquad $\kw{dfs_{naive}(G, v, s)} +\weights(v)*\flag(v) $.
        \STATE \qquad {\bf else}: \#\{There is a cycle finished\}
        \RETURN \qquad \qquad $\weights(v)*\flag(v) $.
        \end{algorithmic}
        \end{algorithm}%
        %
    \begin{algorithm}
        \caption{
        {Refined Adaptivity on Cycle}
        \label{alg:dfscycle_alg}
        }
        \begin{algorithmic}
        \REQUIRE Weighted Directed Graph $G = (\vertxs, \edges, \weights, \flag)$ with a start vertex $s$ and destination vertex $t$ .
        \STATE  {\bf {$\kw{dfs_{refine}(G, c, s)}$}:}  
        \STATE {\bf init} 
        \\
        current node: $c$, 
        \\
        visited: List of length $|\vertxs|$, initialize with $\efalse$.
        \\
        results: $r$ : INT List of length $|\vertxs|$, initialize with -1.
        \\
        MinFlow: INT List of length $|\vertxs|$, initialize MAXINT. 
        \#\{For every vertex, recording the minimum weight when the walk reaching 
        that vertex, inside a cycle\}
        \\
        queries: INT List of length $|\vertxs|$, initialize 0. \#\{For every vertex, recording the query numbers when the walk reaching 
        that vertex, inside a cycle\}
        \\
        \STATE {\bf if} $c = s$:
        \STATE \qquad update the length of the longest path reaching this vertex
        $r[s] =  r[s] + $MinFlow[s] * queries[s].
        \RETURN  \qquad $r[s]$.      
        \STATE {\bf for}  all vertex $v$ having directed edge from s:
        \STATE \qquad {\bf if}  $v$ is unvisited:
        \STATE \qquad \qquad mark $v$ as visited, visited $[v] = 1$.
        \STATE \qquad \qquad MinFlow[v] = min($\weights(v)$, MinFlow[c]).
        \STATE \qquad \qquad queries[v] += $\flag(v)$.
        \STATE \qquad \qquad do not update the length of the longest path reaching this vertex before the cycle is finished
        $r[v] =  r[c] $
        \RETURN \qquad \qquad $\kw{dfs_{refine}(G, v, s)}$.
        \STATE \qquad {\bf else}: \#\{There is a cycle finished\}
        \STATE \qquad \qquad update the length of the longest path reaching this vertex
         $r[v] =  \max(r[v], r[c] + $MinFlow[c] * queries[c]).
        \RETURN  \qquad \qquad  $r[v]$
        \end{algorithmic}
        \end{algorithm}% %
% % \paragraph{Variable Collection Algorithm, $\varCol$}
% % % The $\varCol$ algorithm shows how the labelled variables $\lvar$ are collected 
% % % (via the command ${\assign{x}{\expr}}$ or ${\assign{x}{\query(\qexpr)}}$) from the program ${c}$ in the first step.
% % % The algorithmic rules for $\varCol$ algorithm is defined in Figure~\ref{fig:var_col}. 
% % % It has the form: $\ag{\lvar; w; {c}}{ \lvar'; w'} $. 
% % % The input of $\varCol$ is the labelled variables $\lvar$ collected before the program ${c}$, a while map $w$ consistent with previous estimation, a program ${c}$. 
% % % The output of the algorithm is the updated labelled variables $\lvar'$, along with the updated while map $w$ for next steps' collecting.   
% % The $\varCol$ algorithm shows how the labelled variables $\lvar$ are collected 
% % (via the command ${\assign{x}{\expr}}$ or ${\assign{x}{\query(\qexpr)}}$) from the program ${c}$ in the first step, 
% % along with constructing the flag for each variable, i.e., $\flag$.
% % The algorithmic rules for $\varCol$ algorithm is defined in Figure~\ref{fig:var_col}. 
% % It has the form: 
% % {$\ag{\lvar; \flag; {c}}{ \lvar'; \flag'} $}. 
% % The input of $\varCol$ is a program ${c}$, 
% % the labelled variables $\lvar$ collected before the program ${c}$ 
% % as well as the flags $\flag$ for every corresponding variable .
% % The output of the algorithm is the updated labelled variables $\lvar'$ and flags $\flag'$ thorough the program ${c}$
% % %
% % % We have the algorithmic rules for $\varCol$ algorithm of the form: $\ag{\lvar; w; {c}}{\lvar';w'} $ as in Figure \ref{fig:var_col}. 
% % %
% % \begin{figure}
% % {
% % \begin{mathpar}
% % \inferrule
% % {
% % \empty
% % }
% % { \ag{\lvar ; \flag; {[\assign {x}{\expr}]^{l}}}
% % {\lvar ++ [{x}]; \flag++[0]}
% % }
% % ~\textbf{\varCol-asgn}
% % \and
% % \inferrule
% % {
% % }
% % { \ag{\lvar; \flag; [ \assign{{x}}{\query({\qexpr})}]^{l}}
% % {\lvar ++ [{x}]; \flag ++ [2]} 
% % }~\textbf{\varCol-query}
% % %
% % \and 
% % %
% % \inferrule
% % {
% % \ag{\lvar; [];  {c_1}}{\lvar_1; \flag_1}
% % \and 
% % \ag{\lvar_1; []; {c_2}}{ \lvar_2; \flag_2}
% % \and
% % \lvar_3 = \lvar_2 ++ \lvar'
% % \and
% % \flag_3 = \flag ++ ((\flag_1 ++ \flag_2) \uplus 1)
% % }
% % {
% % \ag{\lvar; \flag;
% % [\eif({\bexpr}, { c_1, c_2)}]^{l} }
% % {\lvar_3; \flag_3}
% % }~\textbf{\varCol-if}
% % %
% % %
% % %
% % \and 
% % %
% % \inferrule
% % {
% % \ag{\lvar; \flag {c_1}}{\lvar_1; \flag_1}
% % \and 
% % \ag{\lvar_1; \flag_1 ; {c_2}}{\lvar_2; \flag_2}
% % }
% % {
% % \ag{\lvar; \flag;
% % {(c_1 ; c_2)}}{\lvar_2 ; \flag_2}
% % }
% % ~\textbf{\varCol-seq}
% % \and 
% % %
% % %
% % {
% % \inferrule
% % {
% % { \ag{\lvar; [] ; {c}}
% % {\lvar'; \flag' }  }
% % \\
% % \lvar'' = \lvar'
% % \and 
% % \flag'' = \flag ++ (\flag' \uplus 1)
% % }
% % {
% % \ag{\lvar; \flag;  
% % \ewhile [{b}]^{l}
% % \edo  {c} }{\lvar''; \flag''}
% % }
% % ~\textbf{\varCol-while}
% % }
% % \end{mathpar}
% % }
% % \caption{The Algorithmic Rules of $\varCol$ }
% % \label{fig:var_col}
% % \end{figure}
% % %
% % %
% % The assignment commands are the source of variables $\varCol$ collecting, 
% % in the case $\textbf{\varCol-asgn}$ and $\textbf{\varCol-query}$, 
% % the output labelled variables are extended by ${x}$. 
% % \\
% % \todo{
% % When it comes to the $\eif \ldots \ethen \ldots \eelse$ command in the rule $\textbf{\varCol-if}$, variables assigned in the then branch ${c_1}$, as well as the variables assigned in the else branch ${c_2}$, and the new generated variables $\bar{{x}},\bar{{y}},\bar{{z}}$ in $ [ \bar{{x}}, \bar{{x_1}}, \bar{{x_2}}] ,[ \bar{{y}}, \bar{{y_1}}, \bar{{y_2}}],[ \bar{{z}}, \bar{{z_1}}, \bar{{z_2}}]$.
% % \\ 
% % The sequence command ${c_1;c_2}$ is standard by accumulating the predicted variables in the two commands ${c_1}$ and ${c_2}$ preserving their order. 
% % \\
% % The while command $\ewhile {\bexpr}, [{\bar{x}}] \ldots \edo {c}$ considers the newly generated variables by SSA transformation ${\bar{x}}$
% % as well and the newly labelled variables in its body ${c}$.
% % \\
% % %
% % Below we present the definition for a valid index, to have a clear understanding on the variable collecting algorithm:
% % }
% % %
% % %
% % \todo{
% % \begin{defn}[Valid Index (Remove?)]
% % Given an assigned variable list $\lvar$, $\lvar; \vDash ({c},i_1,i_2)$ iff 
% % $\lvar' = \lvar[0,\ldots, i_1-1], \lvar';{c} \to \lvar'' \land \lvar'' = \lvar[0, \ldots, i_2-1] $.  
% % \end{defn}}
% % %
% % %
% \todo{Data Dependency Analysis Algorithm Needed: (Possibly modify based on existing one, or a different one) get the more precise dependency information. 
% i.e., instead of dependency on all the over-approximated variables, 
% but dependency on only the variables assumed to be live.
% }
% \paragraph{Data Dependency Analysis Algorithm}
% %
% In this data flow matrix generating algorithm, we analyze the data flow information among all labelled variables $\lvar$ collected via the the $\varCol$ algorithm of length $N$.
% %
% We track the data flow relations between all these labelled variables. These informations are stored in a matrix $\Mtrix$, whose size is $N \times N$. 
% % We also track whether arbitrary variable is assigned with a query result in a vector $\flag$ with size $|\lvar|$. 
% %
% The algorithm to fill in the matrix is of the form: 
% {$\ad{\Gamma ; {c} ; \lvar}{\Mtrix}$}
% $\ad{\Gamma ; {c} ; i_1, i_2}{\Mtrix; \flag}$. 
% $\Gamma$ is a vector records the variables the current program ${c}$ depends on, the index $i_1$ is a pointer which refers to the position of the first new-generated variable in ${c}$ in the labelled variables $\lvar$, and $i_2$ points to the first new variable that is not in ${c}$ (if exists). 
% % %
% % %
% % {
% % \begin{defn}[Valid Gamma (Remove?)]
% % $\Gamma \vDash i_1$ iff $\forall i \geq i_1, \Gamma(i_1)=0 $.  
% % \end{defn}
% % }
% %%
% %
% % \framebox{$ {\Gamma} \vdash^{i_1, i_2}_{\Mtrix, \flag} ~ c $}
% % \begin{mathpar}
% % \inferrule
% % {\Mtrix = \lMtrix_i * ( \rMtrix_{{\expr},i} + \Gamma )
% % }
% % {
% %  \ad{\Gamma;[\assign {{x}}{{\expr}} ]^{l}; i }{\Mtrix; \flag_{0}; i+1 }
% % }
% % ~\textbf{\graphGen-asgn}
% % \and
% % {
% % \inferrule
% % {\Mtrix = \lMtrix_i * ( \rMtrix_{{\expr},i} + \Gamma )
% % \\
% % \flag = \lMtrix_i \and \flag(i) = 1
% % }
% % { 
% % \ad{\Gamma;[ \assign{{x}}{\query({\expr})} ]^{l} ; i }
% % {\Mtrix;\flag;i+1}
% % }~\textbf{\graphGen-query}}
% % %
% % \and 
% % %
% % {
% % \inferrule
% % {
% % {\ad{\Gamma + \rMtrix_{{\bexpr}, i_1}; {c_1} ; i_1 }{ \Mtrix_1;\flag_1;i_2 }}
% % \and 
% % {\ad{\Gamma + \rMtrix_{{\bexpr}, i_1};{c_2} ; i_2 }{ \Mtrix_2; \flag_2 ;i_3}}
% % \\
% % {\ad{\Gamma; [ \bar{{x}}, \bar{{x_1}}, \bar{{x_2}}]; i_3 }{ M_x; \flag_{\emptyset}; i_3+|\bar{{x}}| }}
% % %
% % \\
% % %
% % {\ad{\Gamma; [ \bar{{y}}, \bar{{y_1}}, \bar{{y_2}}]; i_3+|\bar{{x}}| }{ \Mtrix_y; \flag_{\emptyset}; i_3+|\bar{{x}}|+|\bar{{y}}| }}
% % %
% % \\
% % %
% % {\ad{\Gamma; [ \bar{{z}}, \bar{{z_1}}, \bar{{z_2}}]; i_3+|\bar{{x}}|+ |\bar{{y}}|}{ \Mtrix_y; \flag_{\emptyset}; i_3+|\bar{{x}}|+|\bar{{y}}| + |\bar{{z}}| }}
% % \\
% % {\Mtrix = (\Mtrix_1 + \Mtrix_2)+ \Mtrix_x+ \Mtrix_y + \Mtrix_z }
% % }
% % {
% % \ad{\Gamma ; \eif([{\bexpr}]^{l},[ \bar{{x}}, \bar{{x_1}},
% % \bar{{x_2}}] ,[ \bar{{y}}, \bar{{y_1}}, \bar{{y_2}}], 
% % [ \bar{{z}}, \bar{{z_1}}, \bar{{z_2}}],
% % { c_1, c_2)} ; i_1}{ \Mtrix ; \flag_1 \uplus \flag_2 \uplus 2  ; i_3+|\bar{x}|+|\bar{y}|+|\bar{z}| }
% % }
% % ~\textbf{\graphGen-if}
% % }
% % %
% % %
% % %
% % \and 
% % %
% % \inferrule
% % {
% % {\ad{\Gamma; {c_1} ; i_1 }{ \Mtrix_1 ; \flag_1; i_2 }  }
% % \and 
% % {
% % \ad{\Gamma;{c_2}; i_2}{ \Mtrix_2; \flag_2 ;i_3 }}
% % }
% % {
% % \ad{\Gamma ; ({c_1 ; c_2} ) ; i_1}{( \Mtrix_1 {;} \Mtrix_2) ; \flag_1 \uplus V_2 ; i_3  }
% % }
% % ~\textbf{\graphGen-seq}
% % %
% % \and 
% % %
% % \and 
% % %
% % { 
% % \inferrule
% % {
% % B= |{\bar{x}}| \and {A = |{c}|}
% % \\
% % {\ad{\Gamma;[\bar{{x}}, \bar{{x_1}}, \bar{{x_2}}]; i+ (B+A) }{ \Mtrix_{1};V_{1}; i+B+(B+A) }}
% % \\
% % {
% % \ad{\Gamma;{c} ; i+B+(B+A)  }{ \Mtrix_{2}; \flag_{2}; i+B+A+(B+A) }
% % }
% % \\
% % {
% % \ad{\Gamma ; [\bar{{x}}, \bar{{x_1}}, \bar{{x_2}}] ; i+(B+A) }{ \Mtrix; \flag ;i+(B+A)+B}
% % }
% % \\
% % { \Mtrix' = \Mtrix + ( \Mtrix_{1} + \Mtrix_{2}) }
% % \and
% % {
% % \flag' = \flag \uplus (( \flag_{1} \uplus \flag_{2}) \uplus 2)  }
% % }
% % {
% % \ad{\Gamma;
% % \ewhile ~ [ b ]^{l} ~ {n} ~
% % [\bar{{x}}, \bar{{x_1}}, \bar{{x_2}}] 
% % ~ \edo ~  c;
% % i }{ \Mtrix'; \flag' ;i+(B+A)+B }
% % }~\textbf{\graphGen-while}
% % }
% % \end{mathpar}
% {
% \framebox{$ \ad{\Gamma; c; \lvar_c}{\Mtrix}$}
% \begin{mathpar}
% \inferrule
% {
% {x}^l \in \lvar_c
% \and 
% \Mtrix = \lMtrix_i * ( \rMtrix_{{\expr}} + \Gamma )
% }
% {
% \ad{\Gamma; [\assign {{x}}{{\expr}} ]^{l}; \lvar_c}
% {\Mtrix}
% }
% ~\textbf{\graphGen-asgn}
% \and
% {
% \inferrule
% {
% {x}^l \in \lvar_c
% \and 
% \Mtrix = \lMtrix_i * ( \rMtrix_{{\expr}} + \Gamma )
% }
% { 
% \ad{\Gamma;[ \assign{{x}}{\query({\qexpr})} ]^{l} ; \lvar_c }
% {\Mtrix}
% }~\textbf{\graphGen-query}}
% %
% \and 
% %
% {
% \inferrule
% {
% {\ad{\Gamma + \rMtrix_{{\bexpr}}; {c_1} ; \lvar_c }{ \Mtrix_1}}
% \and 
% {\ad{\Gamma + \rMtrix_{{\bexpr}}; {c_2}; \lvar_c }{ \Mtrix_2}}
% \and
% {\Mtrix = (\Mtrix_1 + \Mtrix_2)}
% }
% {
% \ad{\Gamma ; \eif([{\bexpr}]^{l},{ c_1, c_2)}}
% { \Mtrix }
% }
% ~\textbf{\graphGen-if}
% }
% %
% %
% %
% \and 
% %
% \inferrule
% {
% {\ad{\Gamma; {c_1}; \lvar_c }{ \Mtrix_1}  }
% \and 
% {
% \ad{\Gamma;{c_2}; \lvar_c }{ \Mtrix_2}}
% }
% {
% \ad{\Gamma ; ({c_1 ; c_2} ); \lvar_c}
% {( \Mtrix_1 {;} \Mtrix_2) }
% }
% ~\textbf{\graphGen-seq}
% %
% \and 
% %
% \and 
% %
% { 
% \inferrule
% {
% {
% \ad{\Gamma + \rMtrix_{{\bexpr}};{c}; \lvar_c  }{ \Mtrix'}
% }
% }
% {
% \ad{\Gamma;
% \ewhile [ \sbexpr ]^{l} \edo  {c}; \lvar_c }{\Mtrix'}
% }~\textbf{\graphGen-while}
% }
% \end{mathpar}
% }
% %
% Below we define the valid data flow matrix, to have a clear understanding on the data flow generating algorithm:
% \begin{defn}[Valid Matrix]
% For a labelled variables $\lvar$, $\lvar \vDash (\Mtrix,\flag)$ iff the cardinality of $\lvar$ equals to the one of $\flag$, $|\lvar| = |\flag|$ 
% and the matrix $\Mtrix$ is of size $|\flag| \times |\flag|$.
% \end{defn}
% %
% \todo{Improvement if possible: Combining reachability bounds analysis into the static dependency analysis algorithm above, rather than adopting an external tool entirely.}
% %
% \paragraph{Reachability Bounds}
% Given a program $c$ with its labelled variables $\lvar$,
% we use the $\rb({x}, {c})$ algorithm, from paper \cite{10.1145/1806596.1806630}, to estimate the reachability bound for each variable ${x} \in \lvar$. 
% The input of $\rb$ is a program ${c}$ in SSA language and a variable ${x} $ from ${c}$.
% The output of $\rb({x}, {c})$ is an integer representing the reachability bound of ${x}$ in ${c}$.
% %

% %
% The following example programs ${c}2$ and ${c}3$ with while loop illustrate how the algorithm works.
% The collected labelled variables, $\lvar_{{c}2}$ and $\lvar_{{c}3}$,
% data flow matrix $\Mtrix_{{c}2}$ and  $\Mtrix_{{c}3}$
% and variable flags $\flag_{{c}2}$ and $\flag_{{c}3}$
% for program ${c}2$ and ${c}3$
% are presented in the right hand side.
%
% \[
% {{c}2 \triangleq
% \begin{array}{l}
% \left[{ x_1} \leftarrow \query(1)  \right]^1 ; 
% \\
% \left[{i_1} \leftarrow 0 \right]^2 ; 
% \\
% \ewhile
% ~ [{i_1} < 2]^3
% 	\\
% ~{[ x_3,x_1 ,x_2 ], [i_3, i_1, i_2] }
% ~ \edo 
% \\
% ~ \Big( 
% \left[{y}_1 \leftarrow \query(2) \right]^4;
% \\
% \left[{x_2 \leftarrow y_1  + x_3 } \right]^5;
% \\
% \left[{i_2 \leftarrow 1  + i_3 } \right]^6
% \Big) ; 
% \\
% \left[ {\assign{z_1}{x_3}} + 2  \right]^{7}
% \end{array}
% ,
% ~~~~
% \lvar_{{c}2} = \left [ \begin{matrix}
% {x}_1 \\
% {x}_3 \\
% {y}_1 \\
% {x}_2 \\
% {z}_1 \\
% {i}_1 \\
% {i}_2 \\
% {i}_3 
% \end{matrix} \right ]
% % \Mtrix =  \left[ \begin{matrix}
% %  & (x_1)  & (y_1) & (x_2) & (x_3) &  (z_1) & i_1 & i_2 & i_3\\
% % (x_1) & 0 & 0 & 0 & 0 & 0 & 0 & 0 & 0 \\
% % (y_1) & 0 & 0 & 0 & 0 & 0 & 1 & 1 & 1 \\
% % (x_2) & 0 & 1 & 0 & 1 & 0 & 1 & 1 & 1 \\
% % (x_3) & 1 & 0  & 1& 0 & 0 & 1 & 1 & 1 \\
% % (z_1) & 0 & 0 & 0 & 1 & 0 & 0 & 0 & 0 \\
% % (i_1) & 0 & 0 & 0 & 0 & 0 & 0 & 0 & 0 \\
% % (i_2) & 0 & 1 & 0 & 1 & 0 & 1 & 0 & 1 \\
% % (i_3) & 1 & 0  & 1& 0 & 0 & 1 & 1 & 1 \\
% % \end{matrix} \right]
% ,
% ~~~~~~
% \Mtrix_{{c}2} =  \left[ \begin{matrix}
% 0 & 0 & 0 & 0 & 0 & 0 & 0 & 0 \\
% 0 & 0 & 0 & 0 & 0 & 1 & 1 & 1 \\
% 0 & 1 & 0 & 1 & 0 & 1 & 1 & 1 \\
% 1 & 0  & 1& 0 & 0 & 1 & 1 & 1 \\
% 0 & 0 & 0 & 1 & 0 & 0 & 0 & 0 \\
% 0 & 0 & 0 & 0 & 0 & 0 & 0 & 0 \\
% 0 & 1 & 0 & 1 & 0 & 1 & 0 & 1 \\
% 1 & 0  & 1& 0 & 0 & 1 & 1 & 1 \\
% \end{matrix} \right]
% ,
% ~~~~
% \flag_{{c}2} = \left [ \begin{matrix}
% 1 \\
% 2 \\
% 1 \\
% 2 \\
% 0 \\
% 0 \\
% 2 \\
% 1 
% \end{matrix} \right ]
% }
% \]
% %
% %
% \[
% {{{c}3}  \triangleq
% \begin{array}{l}
% \left[{ x}_1 \leftarrow \query(1)  \right]^1 ;
% \\
% \left[{i_1} \leftarrow 1 \right]^2 ; 
% \\
% \ewhile ~ [i < 0]^{3} ,
% \\
% ~{[ x_3,x_1 ,x_2 ], [i_3, i_1, i_2] }
% ~ \edo
% \\
% ~ \Big( 
% \left[{ y_1} \leftarrow \query(2) \right]^3; \\
% \left[{x_2 \leftarrow y_1  + x_3 } \right]^5
% \Big) ; \\
% \left[ {\assign{z_1}{x_3}} + 2  \right]^{6}
% \end{array},
% ~~~~~~
% \lvar_{{c}3} = \left [ \begin{matrix}
% {x}_1 \\
% {i}_1 \\
% {x}_3 \\
% {i}_3 \\
% {z}_1 \\
% \end{matrix} \right ]
% ,~~~~~~
% \Mtrix_{{c}3}  =  \left[ \begin{matrix}
% 0 & 0 & 0 & 0 & 0 \\
% 0 & 0 & 0 & 0 & 0 \\
% 1 & 0 & 0 & 0 & 0 \\
% 0 & 1 & 0 & 0 & 0 \\
% 0 & 0 & 1 & 0 & 0 \\
% \end{matrix} \right]
% ,~~~~~~
% \flag_{{c}3} = \left [ \begin{matrix}
% 1 \\
% 0 \\
% 2 \\
% 2 \\
% 0 \\
% \end{matrix} \right ]
% }
% \]
% %
% We can now look at the if statement.
% \[ 
% %
% {c}4 \triangleq
% \begin{array}{l}
% 	\left[ {x}_1 \leftarrow \query(1) \right]^1; 
% 	\\
% 	\left[{y}_1 \leftarrow \query(2) \right]^2 ; 
% 	\\
% \eif \;( { x_1 + y_1 == 5} )^3,  \\
% {[ x_4,x_2,x_3 ],[] ,[y_3,y_1,y_2 ]} 
% \\
% \mathsf{then} ~ \left[ 
% {x}_2 \leftarrow \query(3) \right]^4 
% \\
% \mathsf{else} ~ \left[ 
% {x}_3 \leftarrow \query(4) \right]^5 ; 
% \\
% {y}_2 \leftarrow 2 ) \\
% \left[ { z_1 \leftarrow x_4 +y_3 }\right]^6
% \end{array},
% % \]
% % \[
% ~~~~~~
% \lvar_{{c}4} =  \left[ \begin{matrix}
% {x}_1 \\
% {y}_1 \\
% {x}_2 \\
% {x}_3 \\
% {y}_2 \\
% {x}_4 \\
% {y}_3 \\
% {z}_1 \\
% \end{matrix} \right], 
% ~~~~~ 
% \Mtrix_{{c}4} =  \left[ \begin{matrix}
% 0 & 0 & 0 & 0 & 0 & 0 & 0 & 0 \\
% 0 & 0 & 0 & 0 & 0 & 0 & 0 & 0 \\
% 0 & 0 & 0 & 0 & 0 & 0 & 0 & 0 \\
% 0 & 0 & 0 & 0 & 0 & 0 & 0 & 0 \\
% 0 & 0 & 0 & 0 & 0 & 0 & 0 & 0 \\
% 0 & 0 & 1 & 1 & 0 & 0 & 0 & 0 \\
% 0 & 1 & 0 & 0 & 1 & 0 & 0 & 0 \\
% 0 & 0 & 0 & 0 & 0 & 1 & 1 & 0 \\
% \end{matrix} \right], 
% ~~~~~ 
% \flag_{{c}4} = \left [ \begin{matrix}
% 1 \\
% 1 \\
% 1 \\
% 1 \\
% 0 \\
% 0 \\
% 0 \\
% 0 \\
% \end{matrix} \right ]
% \]
%
%
%
%


% By specifying the departure and destination vertices $s$ and $t$, the $\pathssearch(\progG, s, t)$ algorithm will 
% give the number of query vertices on a finite walk from $s$ to $t$, which contains the maximum number of query vertices.
% The pseudo-code of $\pathssearch(\progG, s, t)$ algorithm is defined in the Algorithm \ref{alg:adpt_alg}.
% %
% \begin{algorithm}
% \caption{
% {Walk Search Algorithm ($\pathssearch$)}
% \label{alg:adpt_alg}
% }
% \begin{algorithmic}
% \REQUIRE Weighted Directed Graph $G = (\vertxs, \edges, \weights, \flag)$ with a start vertex $s$ and destination vertex $t$ .
% \STATE  {\bf {bfs $(G, s, t)$}:}  
% \STATE \qquad {\bf init} 
% current node: $c = s$, 
% queue: $q = [c]$, 
% vector recoding if the vertex is visited: 
% visited$ = [0]*|\vertxs|$,
% result: $r$
% \STATE \qquad {\bf while} $q$ isn't empty:
% \STATE \qquad \qquad take the vertex from beginning $c= q.pop()$
% \STATE \qquad \qquad mark $c$ as visited, visited $[c] = 1$
% \STATE \qquad \qquad currMinFlow = min($\weights$(c), currMinFlow).
% \STATE \qquad \qquad put all unvisited vertex $v$ having directed edge from c into $q$. 
% \STATE \qquad \qquad if $v$ is visited, then there is a circle in the graph, we update the result $r = r + $currMinFlow
% \RETURN $r$
% \end{algorithmic}
% \end{algorithm}
%
%
% \subsection{\todo{Soundness of the \THESYSTEM}}

% {
% 	\begin{thm}[Soundness of the \THESYSTEM].
% 	Given a program ${c}$, we have:
% 	%
% 	\[
% 	\progA({c}) \geq A({c}).
% 	\]
% 	\end{thm}
% }
% {
% \begin{proof}
% Given a program ${c}$, 
% we construct its program-based graph $\progG({c}) = (\vertxs, \edges, \weights, \qflag)$
% by Definition~\ref{def:prog-based_graph}
% According to the Definition \ref{def:prog_adapt}, we have:
% %
% \[
% 	\progA({c}) 
% 	:= \max\left\{ \qlen(k)\ \mid \  k\in \walks(\progG({c}))\right \}.
% \]
% %
% According to the Definition \ref{def:trace-based_adapt}, we have the trace-based adaptivity as follows:
% $$
% A({c}) = \max \big 
% \{ \len(p) \mid {m} \in \mathcal{SM},D \in \dbdom ,p \in \paths(\traceG({c}, \text{D}, {m}) \big \} 
% $$
% %
% Then, we need to show:
% \[
% \max \big 
% \{ \len(p) \mid {m} \in \mathcal{SM},D \in \dbdom ,p \in \paths(\traceG({c}, \text{D}, {m}) \big \} 
% \leq
% \max\left\{ \qlen(k) \ \mid \  k\in \walks(\progG({c}))\right \}
% \]
% %
% It is sufficient to show that:
% \[
% 	\forall p, {m}, D, ~ s.t., ~ p \in \paths(\traceG({c}, \text{D}, {m}),
% 	\exists k \in \walks(\progG({c})) \land 
% 	\len(p) \leq \qlen(k)
% \]
% %
% Taking an arbitrary starting memory $m$ and an arbitrary underlying database $D$,
% we construct a trace-based graph $\traceG({c}, \text{D}, {m}) = (\vertxs, \edges)$ by the definition \ref{def:trace-based_graph}.
% %
% \\
% %
% Let $\midG({c},{m},\text{D}) = \{\midV, \midE, \midF\}$ be the intermediate graph by Definition~\ref{def:midgraph}.
% \\
% By Lemma~\ref{lem:bie_trace_to_mid}, we know:
% \[
% 	\forall p, {m}, D, ~ s.t., ~ p \in \paths(\traceG({c}, \text{D}, {m}),
% 	\exists p' \in \paths(\midG({c},{m},\text{D})) \land 
% 	\len(p) = \len_q(p')
% \]
% %
% Then it is sufficient to show that:
% %
% \[
% 	\forall p, {m}, D, ~ s.t., ~ p \in \paths(\midG({c}, \text{D}, {m}),
% 	\exists k \in \walks(\progG({c})) \land 
% 	\qlen(p) \leq \qlen(k)
% \]
% %
% We prove a stronger statement instead:
% \[
% 	\forall p, {m}, D, ~ s.t., ~ p \in \paths(\midG({c}, \text{D}, {m}),
% 	\exists k \in \walks(\progG({c})) \land 
% 	\qlen(p) = \qlen(k)	
% \]
% %
% %
% By Lemma~\ref{lem:sujv_mid_to_prog}, let $g$ be the surjective function $g: \progV \to \midV$ s.t.:
% %
% $$
% \forall \av \in \midV. ~ \progF(f(\av)) = \midF(\av) 
% \land |\kw{image}(f(\av))| \leq W(f(\av)).
% $$
% %
% %
% % \item(1) $\len(p_{\av_1 \to \av_2}) = \len(k_{f(\av_1) \to f(\av_2)})$
% % %
% % \item(2) $\forall \av \in p_{\av_1 \to \av_2}. ~ f(\av) \in k_{f(\av_1) \to f(\av_2)}$
% % %
% % \item(3) $\forall \av \in p_{\av_1 \to \av_2}. ~ 
% % \kw{image}(f(\av)) \cap {p_{\av_1 \to \av_2}}| = \# \{f(\av) \mid f(\av) \in k_{f(\av_1) \to f(\av_2)}\}$
% %
% Let ${m}$ and $D$ be an arbitrary memory and database $D$,
% taking an arbitrary path $p_{\av_1 \to \av_n} \in \paths(\midG({c}, \text{D}, {m})$ with:
% %
% \item Edge sequence: $(e, \ldots, e_{n-1})$
% %
% \item Vertices sequence: $(\av_1, \ldots, \av_n)$.
% \\
% By Lemma~\ref{lem:sujpathwalk_mid_to_prog}, let $h: \paths(\midG({c}, \text{D}, {m})) \to \walks(\progG({c}))$ be the surjective function satisfies:
% %
% \[
% 	\forall p_{\av_1 \to \av_n} \in \paths(\midG({c}, \text{D}, {m}))
% 	\text{ with }
% 	\left\{
% 	\begin{array}{ll}
% 	\mbox{edge sequence:} & (e, \ldots, e_{n-1})
% 	\\ 
% 	\mbox{vertices sequence:} & (\av_1, \ldots, \av_n)
% 	\end{array}
% 	\right.
% \]
% %
% \[
% 	\exists k_{f(\av_1) \to f(\av_n)} \in \walks(\progG({c}))
% 	\text{ with }
% 	\left\{
% 	\begin{array}{ll}
% 	\mbox{edge sequence:} & (g(e), \ldots, g(e_{n-1}) 
% 	\\ 
% 	\mbox{vertices sequence:} & (f(\av_1), \ldots, f(\av_{n}))
% 	\end{array}
% 	\right.
% \]
% %
% We have the walk:
% $k_{f(\av_1) \to f(\av_n)} \in \walks(\progG({c}))$ with:
% %
% \item Edges sequence: $(g(e), \ldots, g(e_{n-1}) $
% %
% \item Vertices sequence: $(f(\av_1), \ldots, f(\av_{n}))$.
% \\
% It is sufficient to show 
% %
% \[
% 	\qlen(p_{\av_1 \to \av_n}) = \qlen(k_{f(\av_1) \to f(\av_n)})
% \]
% %
% Unfold the definition of $\qlen$, it is suffice to show:
% \[
% \len \big( \av \mid \av \in (\av_1, \ldots, \av_n) \land \midF(\av) = 2 \big) 
% = \len \big(f(\av) \mid f(\av) \in (f(\av_1), \ldots, f(\av_{n})) \land \progF(f(\av)\big) = 2)	
% ~ (a)
% \]
% %
% By Lemma~\ref{lem:sujv_mid_to_prog}, we know:
% %
% \[
% 	\forall \av \in \midV. ~ \midF(\av) = \progF(f(\av)) ~(b)
% \]
% By rewriting $(b)$ in $(a)$, we have this case proved.
% %
% \\
% \todo{
% \begin{defn}[Intermediate Graph $\midG$].
% 	\label{def:midgraph}
% 	\\
% 	$\mathcal{AV}$ : Annotated Variables based on program execution
% 	\\
% 	Given a program ${c}$ with its labelled variables $\lvar$ of length $N$,
% 	a database $D$, a starting memory ${m}$,
% 	s.t., $\Gamma \vdash_{\Mtrix_c, \flag_c} {c}$,
% 	the intermediate graph 
% 	$\midG({c},{m},\text{D}) = (\vertxs, \edges, \flag)$ is defined as:%
% \[
% \begin{array}{rlcl}
% 	\text{Vertices} &
% 	\vertxs & := & \left\{ 
% 	\av \in \mathcal{AV} \middle\vert
% 	\exists {m'},  w', \qtrace, \vtrace.  ~ s.t., ~  
% 	\config{{m} ,{c}, [], [], []}  \to^{*}  \config{{m'} , \eskip, \qtrace, \vtrace, w' }
% 	\land \av \in \vtrace
% 	\right\}
% 	\\
% 	\text{Directed Edges} &
% 	\edges & := & 
% 	\left\{ 
% 	(\av, \av') \in \mathcal{AV} \times \mathcal{AV} 
% 	~ \middle\vert ~
% 	\flowsto(\av, \av', {c},{m},D) 
% 	\right\}
% 	\\
% 	\text{Flags} &
% 	\flag & := & 
% 	\big\{ (\av, n)  \in \vertxs \times \{0, 1, 2\} 
% 	\mid 
% 	(\pi_1(\av) = \lvar(i) \land n = \flag_c(i)); ~
% 	i = 1, \ldots, N
% 	\big\}
% \end{array}
% \]
% \end{defn}
% }
% %
% \\
% \todo{
% 	\begin{lem}[$\vardep$ is Transitive].
% 	\label{lem:vardep_trans}
% 	\\
% 	Given a program ${c}$, with a starting memory ${m}$ and a hidden database $D$, s.t., 
% 	$\config{{m}, {c}, [], [], []} \rightarrow^{*} \config{{m}', \eskip, \qtrace, \vtrace, w} $.
% 	Then, $\forall \av_1, \av_2, \av_3 \in \vtrace$:
% \[
% 	\Big(\vardep(\av_1, \av_2, {c}, {m}, D) \land 
% 	\vardep(\av_2, \av_3, {c}, {m}, D) \Big)
% 	\implies
% 	\vardep(\av_1, \av_3, {c}, {m}, D)
% \]
% 	\end{lem}
% 	\begin{subproof}[of Lemma~\ref{lem:vardep_trans}]
% 	Proof by unfolding and rewriting the Definition~\ref{def:var_dep}.
% 	\end{subproof}
% }
% \\
% %
% \todo{
% 	\begin{lem}[$\flowsto$ is Transitive ??].
% 	\label{lem:flowsto_trans}
% 	\\
% 	Given a program ${c}$ with its labelled variables $\lvar$ of length $N$. 
% 	Then $\forall x_1, x_2, x_3 \in \lvar$
% \[
% 	\Big(\flowsto(x_1, x_2) \land \flowsto(x_2, x_3) \Big)
% 	\implies
% 	\flowsto(x_1, x_3)
% \]
% 	\end{lem}
% 	\begin{subproof}[of Lemma~\ref{lem:flowsto_trans}]
% 	Proof by unfolding the Definition~\ref{def:flowsto}.
% 	\end{subproof}
% }
% \\
% %
% \todo{
% 	\begin{lem}[$\qdep$ Implies $\vardep$].
% 	\label{lem:querydep_vardep}
% 	\\
% 	Given a program ${c}$, with a starting memory ${m}$ and a hidden database $D$, s.t., 
% 	$\config{{m}, {c}, [], [], []} \rightarrow^{*} \config{{m}', \eskip, \qtrace, \vtrace, w} $.
% 	Then, $\forall \av_1, \av_2 \in \qtrace$
% \[
% 	\qdep(\av_1, \av_2, {c}, {m}, D) \implies 
% 	\vardep(\pi_2(\av_1), \pi_2(\av_2), {c}, {m}, D)
% \]
% 	\end{lem}
% 	\begin{subproof}[of Lemma~\ref{lem:querydep_vardep}]
% 	Proof by unfolding the Definition~\ref{def:var_dep} and Definition~\ref{def:query_dep}.
% 	\end{subproof}
% }
% \\
% %
% \todo{
% 	\begin{lem}[$\vardep$ Implies \flowsto].
% 	\label{lem:vardep_flows}
% 	\\
% 	Given a program ${c}$, with a starting memory ${m}$ and a hidden database $D$, s.t., 
% 	$\config{{m}, {c}, [], [], []} \rightarrow^{*} \config{{m}', \eskip, \qtrace, \vtrace, w} $.
% 	Then, $\forall \av_1, \av_2 \in \vtrace$
% \[
% 	\vardep(\av_1, \av_2, {c}, {m}, D) \implies 
% 	\flowsto(\pi_1(\av_1), \pi_1(\av_2))
% \]
% 	\end{lem}
% 	\begin{subproof}[of Lemma~\ref{lem:querydep_vardep}]
% 	Proof by showing contradiction based on the Definition~\ref{def:var_dep} and Definition~\ref{def:flowsto}.
% 	Let $\av_1, \av_2 \in \vtrace$ be 2 arbitrary annotated variables in the variable trace $\vtrace$,
% 	s.t., $\vardep(\av_1, \av_2, {c}, {m}, D)$.
% 	\\
% 	Unfolding the $\vardep$ definition, we have:	
% 	\end{subproof}
% }
% \\
% %
% \todo{
% 	\begin{lem}[Injective Mapping of vertices from $\traceG$ to $\midG$].
% 	\label{lem:injv_trace_to_mid}
% 	\\
% 	$\traceG({c}) = \{\traceV, \traceE\}$
% 	\\
% 	$\midG({c},{m},\text{D}) = \{\midV, \midE, \midF\}$
% \[
% 	\exists ~ \kw{injective} ~ f: \mathcal{AQ} \to \mathcal{AV}. 
% 	~ \forall \av \in \traceV. ~ 
% 	f(\av) \in \midV \land \midF(f(\av)) = 2
% \]
% 	\end{lem}
% \begin{subproof}
% Proving by Definition~\ref{def:midgraph} and Definition~\ref{def:prog_adapt}.
% \end{subproof}
% }
% \\
% \todo{
% 	\begin{lem}[One-on-One Mapping from $\edges$ of $\traceG$ to $\paths(\midG)$].
% 	\label{lem:bie_trace_to_mid}
% 	\\
% 	$\traceG({c}) = \{\traceV, \traceE\}$
% 	\\
% 	$\midG({c},{m},\text{D}) = \{\midV, \midE, \midF\}$
% 	\\
% 	An injective function $ f: \traceV \to \midV$ s.t.,
% 	$\forall \av \in \traceV. ~ \midF(f(\av)) = 2$ 
% \[
% 	\forall e = (\av_1, \av_2) \in \traceE. ~ 
% 	\exists p_{f(\av_1) \to f(\av_2)} \in \paths(\midG({c}, \text{D}, {m}))
% \]
% 	\end{lem}
% \begin{subproof}
% Proving by Lemma~\ref{lem:injv_trace_to_mid} and Definition~\ref{def:midgraph} and acyclic property of $\traceG$ and $\midG$.
% \end{subproof}
% }
% \\
% \todo{
% 	\begin{lem}[Surjective Mapping of Vertices from $\midG$ to $\progG$].
% 	\label{lem:sujv_mid_to_prog}
% 	\\
% 	$\midG({c},{m},\text{D}) = \{\midV, \midE, \midF\}$
% 	\\
% 	$\progG({c}) = \{\progV, \progE, \progF, \progW\}$
% 	\\
% 	$\exists ~ \kw{surjective} ~ f: \mathcal{AV} \to \mathcal{SVAR}.$
% 	%
% \[
% 	\forall \av \in \midV. ~ 
% 	f(\av) \in \progV \land \progF(f(\av)) = \midF(\av) \land
% 	|\kw{image}(f(\av))| \leq W(f(\av))
% \]
% \end{lem}
% \begin{subproof}
% Proving by Definition~\ref{def:midgraph}.
% \end{subproof}
% }
% \\
% \todo{
% 	\begin{lem}[Surjective Mapping from $\edges$ of $\midG)$ to $\edges$ of $\progG$].
% 	\label{lem:suje_mid_to_prog}
% 	\\
% 	$\midG({c},{m},\text{D}) = \{\midV, \midE, \midF\}$
% 	\\
% 	$\progG({c}) = \{\progV, \progE, \progF, \progW\}$
% 	\\
% 	A surjective function $f: \progV \to \midV$ s.t.,
% 	$\forall \av \in \midV. ~ \progF(f(\av)) = \midF(\av) \land |\kw{image}(f(\av))| \leq W(f(\av))$
% 	%
% \[
% 	\exists ~ \kw{surjective} ~ g: \midE \to \progE. ~
% 	\forall e_{mid} = (\av_1, \av_2) \in \midE. 
% 	\exists e_{prog} = ({f(\av_1), f(\av_2)}) \in \progE
% \]
% \end{lem}
% \begin{subproof}
% Proving by Lemma~\ref{lem:sujv_mid_to_prog}.
% \end{subproof}
% }
% \\
% \todo{
% 	\begin{lem}[Surjective Mapping from $\paths(\midG)$ to $\walks(\progG)$].
% 	\label{lem:sujpathwalk_mid_to_prog}
% 	\\
% 	$\midG({c},{m},\text{D}) = \{\midV, \midE, \midF\}$
% 	\\
% 	$\progG({c}) = \{\progV, \progE, \progF, \progW\}$
% 	\\
% 	A surjective function $f: \progV \to \midV$ s.t.,
% 	$\forall \av \in \midV. ~ \progF(f(\av)) = \midF(\av) \land |\kw{image}(f(\av))| \leq W(f(\av))$
% 	\\
% 	A surjective function $g: \midE \to \progE$ s.t.,
% 	$\forall e_{mid} = (\av_1, \av_2) \in \midE. 
% 	\exists e_{prog} = ({f(\av_1) \to f(\av_2)}) \in \progE$
% 	\\
% 	$\exists ~ \kw{surjective} ~ h: \paths(\midG({c},{m},\text{D})) \to \walks(\progG({c}))$ s.t.:
% 	%
% \[
% 	\forall p_{\av_1 \to \av_2} \in \paths(\midG({c},{m},\text{D}))
% 	\text{ with }
% 	\left\{
% 	\begin{array}{ll}
% 	\mbox{edge sequence:} & (e, \ldots, e_{n-1})
% 	\\ 
% 	\mbox{vertices sequence:} & (\av_1, \ldots, \av_n)
% 	\end{array}
% 	\right.
% \]
% \[
% 	\exists k_{f(\av_1) \to f(\av_2)} \in \walks(\progG({c}))
% 	\text{ with }
% 	\left\{
% 	\begin{array}{ll}
% 	\mbox{edge sequence:} & (g(e), \ldots, g(e_{n-1}) 
% 	\\ 
% 	\mbox{vertices sequence:} & (f(\av_1), \ldots, f(\av_{n}))
% 	\end{array}
% 	\right.
% \]
% % \item $(e, \ldots, e_{n-1})$, $(\av_1, \ldots, \av_n)$ are the edges sequence and vertices sequence of $p_{\av_1 \to \av_2}$.
% % then, 
% %  $\len(p_{\av_1 \to \av_2}) = \len(k_{f(\av_1) \to f(\av_2)})$
% % %
% % \item $\forall \av \in p_{\av_1 \to \av_2}. ~ f(\av) \in k_{f(\av_1) \to f(\av_2)}$
% % %
% % \item $\forall \av \in p_{\av_1 \to \av_2}. ~ 
% % \kw{image}(f(\av)) \cap {p_{\av_1 \to \av_2}}| = \# \{f(\av) \mid f(\av) \in k_{f(\av_1) \to f(\av_2)}\}
% % $
% \end{lem}
% %
% \begin{subproof}
% Proving by induction on the length of $l = p_{\av_1 \to \av_2} \in \paths(\midG({c},{m},\text{D}))$, and Lemma~\ref{lem:suje_mid_to_prog} and Lemma~\ref{lem:sujv_mid_to_prog}.
% \caseL{ $l = 1$: }
% \caseL{ $l = l' + 1$, $l' \geq 1$: }
% \end{subproof}
% }
% \end{proof}
% %

% %
% }