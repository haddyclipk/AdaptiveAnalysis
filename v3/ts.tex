


\[
\begin{array}{llll}
 % \mbox{Index Term} & \idx, \nnatA & ::= &     i ~|~ n ~|~ \idx_1 + \idx_2 ~|~  \idx_1
 %                                  - \idx_2 ~|~ \smax{\idx_1}{\idx_2}\\
%                                  \mbox{Sort} & S & ::= & \nat \\
  \mbox{Linear type} & \ltype &::=  &  \type \lto \type ~|~ \tbase ~|~
  \tbool\\
  \mbox{Nonlinear Type} & \type & ::= & \bang{\idx} \ltype   \\
\end{array}
\]


\begin{figure}
  \begin{mathpar}
    \inferrule{
    }{
      \ictx \tctx , x: \bang{\nnatA}\ltype \tvdash{\nnatA} x: \bang{\nnatA}\ltype
    }~\textbf{Ax}
    %
    \and
    %
    \inferrule{
    }{
      \ictx \Gamma \tvdash{\nnatA} c : \bang{\nnatA}\tbase 
    }~\textbf{b}
    %
    \and
    %
    \inferrule{
    }{
      \ictx \Gamma \tvdash{\nnatA} \evec : \bang{\nnatA}\tbase 
    }~\textbf{Dict}
    %
    \and
    %
    \inferrule{
    }{
      \ictx \Gamma \tvdash{\nnatA} \etrue : \bang{\nnatA}\tbool
    }~\textbf{true}
     %
    \and
    %
    \inferrule{
    }{
      \ictx \Gamma \tvdash{\nnatA} \efalse : \bang{\nnatA}\tbool
    }~\textbf{false}
    %
    \and
    %
    \inferrule{
      \ictx \Gamma, x: \type_1
      \tvdash{r}
      \expr: \type_2
    }{
      \ictx r+\Gamma \tvdash{k+r} \lambda x. \expr : \bang{k}  ( \type_1
      \lto \type_2)
    }~\textbf{lambda}
    % %
    %     \and
    % %
    % \inferrule{
    %   \Gamma \tvdash{\nnatA} (\lambda x. \expr) : \bang{r}  ( \type_1
    %   \lto \type_2)
    %   \\
    %   \env \vDash \Gamma
    % }{
    %    \cdot \tvdash{\nnatA} (\lambda x. \expr, \env) : \bang{r}  ( \type_1
    %   \lto \type_2)\whynot{0}
    % }~\textbf{closure}
    %
    \and
    %
    \inferrule{
      \ictx \Gamma_1  \tvdash{\nnatA_1} \expr_1:  \bang{0} ( \type_1
      \lto \type_2      ) \\
      \ictx \Gamma_2 \tvdash{\nnatA_2} \expr_2: \type_1 
    }{
      \ictx \max (\Gamma_1, \Gamma_2 ) \tvdash{\max( \nnatA_1,\nnatA_2) } \expr_1 \eapp \expr_2 : \type_2
    }~\textbf{app}
    %
    \and
    %
    \inferrule{
      \ictx \Gamma \tvdash{\nnatA} \expr: \bang{k} \ltype 
    }{
      \ictx \Gamma' ,1+\Gamma  \tvdash{1+\nnatA} \delta(\expr): \bang{k} \ltype 
    }~\textbf{delta}
    %
    \and
    %
    \inferrule{
      \ictx \Gamma_1 \tvdash{\nnatA} \expr: \type_1 \\
      \ictx \Gamma_2, x: \type_1 \tvdash{\nnatA'} \expr' : \type
    }{
      \ictx \max(\Gamma_1,\Gamma_2) \tvdash{\max(\nnatA, \nnatA') } \elet x =  \expr \ein \expr': \type
    }~\textbf{let}
    %
    \and
    %
    \inferrule{
      \ictx \Gamma_1 \tvdash{\nnatA_1} \expr_1 : \bang{k} \tbool\\
      \ictx \Gamma_2 \tvdash{\nnatA_2} \expr_2 : \type\\
      \ictx \Gamma_2 \tvdash{\nnatA_2} \expr_3 : \type
    }{
      \ictx \boxed{\max(\Gamma_1, \nnatA_1+ \Gamma_2) } \tvdash{\nnatA_1 + \nnatA_2} \eif \eapp \expr_1 \eapp \expr_2
      \eapp \expr_3 : \type
    }~\textbf{if}
  \end{mathpar}
  \caption{Typing rules, first version}
  \label{fig:type-rules1}
\end{figure}


\begin{figure}
  \begin{mathpar}
    \inferrule{
    }{
      \ictx \tctx , x: \bang{\nnatA}\ltype \whynot{r}  \tvdash{\nnatA} x: \bang{\nnatA}\ltype\whynot{r}
    }~\textbf{Ax}
    %
    \and
    %
    \inferrule{
    }{
      \ictx \Gamma \tvdash{\nnatA} c : \bang{\nnatA}\tbase \whynot{0}
    }~\textbf{b}
    %
    \and
    %
    \inferrule{
    }{
      \ictx \Gamma \tvdash{\nnatA} \etrue : \bang{\nnatA}\tbool\whynot{0}
    }~\textbf{true}
    \and
    %
    \inferrule{
      \ictx \Gamma, x: \type_1
      \tvdash{k}
      \expr: \type_2
    }{
      \ictx r+\Gamma \tvdash{k+r} \lambda x. \expr : \bang{r}  ( \type_1
      \lto \type_2)\whynot{0}
    }~\textbf{lambda}
    %
        \and
    %
    \inferrule{
      \Gamma \tvdash{\nnatA} (\lambda x. \expr) : \bang{r}  ( \type_1
      \lto \type_2)\whynot{0}
      \\
      \env \vDash \Gamma
    }{
       \cdot \tvdash{\nnatA} (\lambda x. \expr, \env) : \bang{r}  ( \type_1
      \lto \type_2)\whynot{0}
    }~\textbf{closure}
    %
    \and
    %
    \inferrule{
      \ictx \Gamma_1  \tvdash{\nnatA_1} \expr_1:  \bang{0} ( \type_1
      \lto \type_2      ) \whynot{r} \\
      \ictx \Gamma_2 \tvdash{\nnatA_2} \expr_2: \type_1 
    }{
      \ictx \max (\Gamma_1, \Gamma_2 ) \tvdash{\max( \nnatA_1,r+\nnatA_2) } \expr_1 \eapp \expr_2 : \type_2
    }~\textbf{app}
    %
    \and
    %
    \inferrule{
      \ictx \Gamma \tvdash{\nnatA} \expr: \bang{k} \ltype \whynot{r}
    }{
      \ictx \Gamma' ,1+\Gamma  \tvdash{1+\nnatA} \delta(\expr): \bang{k} \ltype \whynot{r+1}
    }~\textbf{delta}
    %
    \and
    %
    \inferrule{
      \ictx \Gamma_1 \tvdash{\nnatA} \expr: \type_1 \\
      \ictx \Gamma_2, x: \type_1 \tvdash{\nnatA'} \expr' : \type
    }{
      \ictx \max(\Gamma_1,\Gamma_2) \tvdash{\max(\nnatA, \nnatA') } \elet x =  \expr \ein \expr': \type
    }~\textbf{let}
    %
    \and
    %
    \inferrule{
      \ictx \Gamma_1 \tvdash{\nnatA_1} \expr_1 : \bang{k} \tbool\whynot{r}\\
      \ictx \Gamma_2 \tvdash{\nnatA_2} \expr_2 : \type\\
      \ictx \Gamma_2 \tvdash{\nnatA_2} \expr_3 : \type
    }{
      \ictx \boxed{\max(\Gamma_1, \nnatA_1+ \Gamma_2) } \tvdash{\nnatA_1 + \nnatA_2} \eif \eapp \expr_1 \eapp \expr_2
      \eapp \expr_3 : \type
    }~\textbf{if}
  \end{mathpar}
  \caption{Typing rules, part 1}
  \label{fig:type-rules}
\end{figure}

\begin{thm}[Context Raising]
  If $ \Gamma \tvdash{ \nnatA} \expr : \type $ and $r \in \mathbb{N}$,
  then $ r+\Gamma \tvdash{ r+ \nnatA} \expr : r+ \type $.
\end{thm}

\begin{thm}[Context Weaking]
  If $ \Gamma \tvdash{ \nnatA} \expr : \type $,  $ \Gamma,x : \type' \tvdash{ \nnatA} \expr : \type $.
\end{thm}

\begin{thm}[Context Strengthening]
  If $ \Gamma,x : \type' \tvdash{ \nnatA} \expr : \type $ and $x \not
 \in \fv{\expr} $, then $ \Gamma  \tvdash{ \nnatA} \expr : \type $.
\end{thm}

\begin{thm}[Context Max]
  If $ \Gamma_1 \tvdash{ \nnatA_1} \expr : \type $ and $ \Gamma_2 \tvdash{ \nnatA_2} \expr : \type $,
  then $\max(\Gamma_1,\Gamma_2) \tvdash{\max( \nnatA_1 , \nnatA_2 )} \expr : \type $.
\end{thm}

\begin{thm}[Adaptivity Monotonicity]
   If $ \Gamma \tvdash{ \nnatA} \expr : \type $ and $\nnatA' >
   \nnatA$, then $ \Gamma \tvdash{ \nnatA'} \expr : \type $.
\end{thm}

\begin{thm}[Substitution]
  \begin{enumerate} 
   \item If $ \Gamma,x : \type' \tvdash{ \nnatA} \expr : \type $ and $
  \empty \tvdash{\nnatA'} \valr : \type'  $ and $ x \in \fv{\expr} $,  then  $\Gamma
  \tvdash{\max(\nnatA,\nnatA' )} \expr[\valr/x]  : \type$. 
  \item If $ \Gamma,x : \type' \tvdash{ \nnatA} \expr : \type $ and $
  \empty \tvdash{\nnatA'} \valr : \type'  $ and $ x \not \in \fv{\expr} $,  then  $\Gamma
  \tvdash{\nnatA } \expr[\valr/x]  : \type$.
  \end{enumerate}
\end{thm}

\begin{proof}
  By simultaneous induction on the typing derivation.\\
  Proof of Statement (2).\\
  Since we know $x \not \in \fv{\expr}$ so that $\expr[\valr/x]  =
  \expr$. \\
  From the assumption $ \Gamma,x : \type' \tvdash{ \nnatA} \expr :
  \type $, by context strengthening,  we know that $   \Gamma  \tvdash{
    \nnatA} \expr : \type $.\\

  Proof of Statement (1).\\
  
\caseL{
  $  \inferrule{
    }{
      \ictx \tctx , x: \bang{\nnatA} \ltype \whynot{r} \tvdash{\nnatA} x : \bang{\nnatA}\ltype\whynot{r}
    }~\textbf{Ax}  $
  }
  Assume  $\empty \tvdash{\nnatA'} \valr : \bang{\nnatA} \ltype \whynot{r}  ~(\star) $.\\
  TS:   $\Gamma \tvdash{\max(\nnatA, \nnatA')} x [\valr/x]  :
  \bang{\nnatA} \ltype \whynot{r} $.\\
  It is proved from the assumption $(\star)$ by weaking from $\nnatA'$
  to $\max(\nnatA,\nnatA')$.\\

% \caseL{
%   $  \inferrule{
%     }{
%       \ictx \tctx , x: \bang{\nnatA} \ltype, y : \type' \tvdash{\nnatA} x: \bang{\nnatA}\ltype
%     }~\textbf{Ax}  $
%   }
%   Assume  $\Gamma \tvdash{\nnatA'} \valr : \type' ~(\star) $.\\
%   TS:   $\Gamma \tvdash{\max(\nnatA, \nnatA')} x [\valr/y]  :
%   \bang{\nnatA} \ltype$.\\
%   It is proved from the assumption by weaking.\\ 
 

  \caseL{
  $  \inferrule{
      \ictx \Gamma, y: \type',  x: \type_1
      \tvdash{k}
      \expr: \type_2 (\diamond)
    }{
      \ictx r+\Gamma, y: (r+\type') \tvdash{k+r} \lambda x. \expr : \bang{r}  ( \type_1
      \lto \type_2) \whynot{0}
    }~\textbf{lambda} $
  }
  Assume  $\empty \tvdash{\nnatA'} \valr : \type' ~(\star) $.\\
  By Theorem 2 context raisig, we know: $\empty \tvdash{\nnatA'+r} \valr : r+\type' ~(\star\star) $\\
  TS:   $r+\Gamma \tvdash{\max(k+r, \nnatA'+r)} \lambda x. \expr [\valr/y]  :
  \bang{r}  ( \type_1 \lto \type_2)\whynot{0}$.\\
  We know that $y \in \fv{\expr}$ from the assumption $ y \in \fv{\lambda x.\expr} $.\\
  By IH on $\diamond$ along with $(\star)$, we get:  $\Gamma,x:\type_1 \tvdash{\max(k, \nnatA')}  \expr [\valr/y]  :
  \type_2 ~(1)$.\\
  Using the lambda rule, we conclude: \\
  $r+\Gamma \tvdash{\max(k+r, \nnatA'+r)} \lambda x. \expr [\valr/y]  :
  \bang{r}  ( \type_1 \lto \type_2)\whynot{0} $.\\

  
  \caseL{
     $  \inferrule{
      \ictx \Gamma_1 ,x:\type'  \tvdash{\nnatA_1} \expr_1:  \bang{0} ( \type_1
      \lto \type_2   )\whynot{r} ~(\diamond)\\
      \ictx \Gamma_2 ,x:\type' \tvdash{\nnatA_2} \expr_2: \type_1 ~(\clubsuit)
    }{
      \ictx \max (\Gamma_1, \Gamma_2 ),x:\type' \tvdash{\max( \nnatA_1,r+\nnatA_2) } \expr_1 \eapp \expr_2 : \type_2
    }~\textbf{app} $
  }
  Assume  $\empty \tvdash{\nnatA'} \valr : \type' ~(1) $.\\
  So we also know: $\empty \tvdash{\nnatA'+r} \valr : \type' ~(2)$. \\
   TS:   $\max (\Gamma_1, \Gamma_2 ) \tvdash{ \max(\max(
     \nnatA_1,r+\nnatA_2) , r+\nnatA') } (\expr_1
   \eapp \expr_2)[\valr/x] : \type_2$.\\
   There are three situations:
   \begin{enumerate}
   \item $x \in \fv{\expr_1}, x \in \fv{\expr_2}$, \\
     By IH1 on $(\diamond)$ with $(1)$, we get: $ \Gamma_1
\tvdash{\max(\nnatA_1, \nnatA' )} \expr_1 [\valr/x]:  \bang{0} ( \type_1
\lto \type_2   ) \whynot{r} $.\\
By IH1 on $(\clubsuit)$ and $(1)$, we get: $ \Gamma_2
\tvdash{\max(\nnatA_2, \nnatA')}
\expr_2 [\valr/x]: \type_1  $.\\
\item $x \not\in \fv{\expr_1}, x \in \fv{\expr_2}$\\
  By IH2 on $(\diamond)$ with $(1)$, we get: $ \Gamma_1
\tvdash{\nnatA_1} \expr_1 [\valr/x]:  \bang{0} ( \type_1
\lto \type_2 )\whynot{r} $, by Lemma Adaptivity Monotonicity, we know: $\Gamma_1
\tvdash{\max(\nnatA_1, \nnatA' )} \expr_1 [\valr/x]:  \bang{0} ( \type_1
\lto \type_2 )\whynot{r} $.\\
By IH1 on $(\clubsuit)$ and $(2)$, we get: $ \Gamma_2
\tvdash{\max(\nnatA_2, \nnatA')}
\expr_2 [\valr/x]: \type_1  $.\\
\item $x \in \fv{\expr_1}, x \not\in \fv{\expr_2}$\\
  By IH1 on $(\diamond)$ with $(1)$, we get: $ \Gamma_1
\tvdash{\max(\nnatA_1, \nnatA' )} \expr_1 [\valr/x]:  \bang{0} ( \type_1
\lto \type_2   )\whynot{r}  $.\\
By IH2 on $(\clubsuit)$ and $(2)$, by Lemma Adaptivity Monotonicty, we get: $ \Gamma_2
\tvdash{\max(\nnatA_2, \nnatA')}
\expr_2 [\valr/x]: \type_1  $.\\
     \end{enumerate}
   
By using the rule app, we conclude:\\
  $\max (\Gamma_1, \Gamma_2 ) \tvdash{ \max(\max(
     \nnatA_1,\nnatA') , r+\max(\nnatA_2, \nnatA') ) } (\expr_1 \eapp \expr_2)[\valr/x] : \type_2$.\\

  \caseL{
   $   \inferrule{
      \ictx \Gamma , x: \type' \tvdash{\nnatA} \expr: \bang{k} \ltype 
    }{
      \ictx \Gamma' ,1+\Gamma ,x : 1+\type' \tvdash{1+\nnatA} \delta(\expr): \bang{k} \ltype 
    }~\textbf{delta} $
  }
  Assume  $\empty  \tvdash{\nnatA'} \valr : \type' ~(1) $.\\
   By Theorem 2 context raisig, we know: $ \empty \tvdash{\nnatA'+1} \valr : 1+\type' ~(2) $\\
   TS:
   $\Gamma' ,1+\Gamma \tvdash{ \max( 1+\nnatA, \nnatA'+1)  }
   \delta(\expr): \bang{k} \ltype $.\\
   We know :$ x \in \fv{\expr}$ from the assumption $ x \in \fv{\delta{(\expr)}}$.\\
   By IH2 on the premise, we know: $ \Gamma
   \tvdash{\max(\nnatA,\nnatA')} \expr[\valr/x]: \bang{k} \ltype $.\\
   By the rule delta, we get :
      $\Gamma' ,1+\Gamma \tvdash{1+ \max( \nnatA, \nnatA')  } \delta(\expr[\valr/x]): \bang{k} \ltype $
.\\
  
  \caseL{
      $   \inferrule{
      \ictx \Gamma_1, y: \type'  \tvdash{\nnatA} \expr: \type_1 \\
      \ictx \Gamma_2, y: \type',  x: \type_1 \tvdash{\nnatA'} \expr' : \type
    }{
      \ictx \max(\Gamma_1,\Gamma_2), y: \type' \tvdash{\max(\nnatA, \nnatA') } \elet x =  \expr \ein \expr': \type
    }~\textbf{let}  $
  }
Assume  $\empty \tvdash{\nnatA_1'} \valr : \type' ~(1) $.\\
   TS:   $\max (\Gamma_1, \Gamma_2 ) \tvdash{ \max(
     \nnatA_1' , \max(\nnatA', \nnatA) ) } ( \elet x =  \expr \ein \expr' )[\valr/x] : \type_2$.\\
   Similar to the application rule, we have 3 situations.\\
   By IH on the first premise, we know :$\Gamma_1
   \tvdash{\max(\nnatA_1', \nnatA )}  \expr[\valr/y] : \type_1  $. \\
   By IH on the second premise, we know : $ \Gamma_2, x: \type_1
   \tvdash{\max(\nnatA_1', \nnatA'  )} $.\\
   By the rule let, we conclude that: \\
    $\max (\Gamma_1, \Gamma_2 ) \tvdash{ \max(
     \nnatA_1' , \max(\nnatA, \nnatA') ) } (\elet x =  \expr \ein \expr')[\valr/y] : \type_2$.\\

  % \caseL{
  % $  \inferrule{
  %     \ictx \Gamma_1 \tvdash{\nnatA_1} \expr_1 : \bang{k} \tbool\\
  %     \ictx \Gamma_2 \tvdash{\nnatA_2} \expr_2 : \type\\
  %     \ictx \Gamma_2 \tvdash{\nnatA_2} \expr_3 : \type
  %   }{
  %     \ictx \max(\Gamma_1, \nnatA_1+ \Gamma_2)  \tvdash{\nnatA_1 + \nnatA_2} \eif \eapp \expr_1 \eapp \expr_2
  %     \eapp \expr_3 : \type
  %   }~\textbf{if} $
  % }

  


  \end{proof} 

\begin{thm}[Adaptivity Monotonicity ]
  If $ \empty  \tvdash{\nnatA} \valr : \bang{k} \ltype \whynot{r}$ and
  $r' \geq r$, 
  then $ \empty  \tvdash{\nnatA} \valr : \bang{k} \ltype \whynot{r}$.
\end{thm}%



% \begin{thm}[Adaptivity Soundness theorem]
%   If $ \Gamma  \tvdash{\nnatA} \expr : \type$
%   and  exists $\env $ satisfies $\Gamma$  and  $\env, \expr \bigstep \valr, \tr $,
%   then $ A(\tr) \leq \nnatA  $.\\(
%   $\env$ satisfies $\Gamma$ means $\dom(\env) = \dom(\Gamma) \land
%   \forall x_i \in \dom(\Gamma).  \exists r_i.$ so that $\empty
%   \tvdash{r_i} \env(x_i) : \Gamma(x) $  )
% \end{thm}%

\begin{thm}[Adaptivity Soundness theorem]
  If $ \Gamma  \tvdash{\nnatA} \expr : \type$
  and  exists $\env $ satisfies $\Gamma$  and  $\env, \expr \bigstep{\num} \valr, \tr $,
  then $ \empty  \tvdash{\nnatA- A(\tr)} \valr : \type[?0]$ and
  $\nnatA- A(\tr) \geq 0 $. \\(
  $\env$ satisfies $\Gamma$ means $\dom(\env) = \dom(\Gamma) \land
  \forall x_i \in \dom(\Gamma).  $ so that $\empty
  \tvdash{ r_i } \env_1(x_i) : \Gamma(x)[?0] \land r_i \geq \adap(\env_2(x_i))$ , and $\type[?0] $ means that $
  \type = \bang{k}\ltype \whynot{r} \land \type[?0]= \bang{k}\ltype \whynot{0}$ )
\end{thm}%

\begin{proof}
  By indution on the typing derivation.\\


  \caseL{
    \inferrule{
    }{
      \ictx \tctx , x: \bang{\nnatA} \ltype \tvdash{\nnatA} x: \bang{\nnatA}\ltype\whynot{r}
    }~\textbf{Ax}
  }
  Assume the enviroment $\env \vDash \Gamma$,  and
  $\empty \tvdash{\nnatA  } \valr: \bang{\nnatA}\ltype \whynot{0}
  $. We know that $\env_2(x) = \tr$,
  so that
  $\inferrule{ }{\env, x \bigstep \valr, (x, \env ) }  $. \\
  So we conclude that: $ \nnatA- A(  (x, \env)  ) = \nnatA -  $ and
  $\empty \tvdash{ \nnatA- A(  (x, \env)  ) } \valr: \bang{\nnatA}\ltype\whynot{0}$.\\
   
  
  \caseL{
    \inferrule{
      \ictx \Gamma, x: \type_1
      \tvdash{k}
      \expr: \type_2
    }{
      \ictx r+\Gamma \tvdash{k+r} \lambda x. \expr : \bang{r}  ( \type_1
      \lto \type_2) \whynot{0}
    }~\textbf{lambda}
  }
 Assume exists the enviroment $\env$ so that\\
  $\inferrule{ }{\env, \lambda x. \expr \bigstep (\lambda
    x.\expr,\env), (\lambda x.\expr, \env ) }  $. \\
  So by the rule closure we conclude that: \\
  $ \nnatA- A(  (\lambda x.\expr,\env)  ) = \nnatA $ and
  $\empty \tvdash{ \nnatA- A(  (\lambda x.\expr, \env)  ) }   (\lambda
    x.\expr,\env) : \bang{r}  ( \type_1
      \lto \type_2)\whynot{0}$.\\

  


  
  \caseL{
     \inferrule{
      \ictx \Gamma_1  \tvdash{\nnatA_1} \expr_1:  \bang{0} ( \type_1
      \lto \type_2  ) \whynot{r} ~(\star) \\
      \ictx \Gamma_2 \tvdash{\nnatA_2} \expr_2: \type_1 ~(\diamond)
    }{
      \ictx \max (\Gamma_1, \Gamma_2 ) \tvdash{\max( \nnatA_1,r+\nnatA_2) } \expr_1 \eapp \expr_2 : \type_2
    }~\textbf{app}
  }

   For all the variables $x_i $ in $\dom(\max(\Gamma_1, \Gamma_2  )) $ and
   $x_i \not\in \dom(\Gamma_1)$. 
   We extend $\Gamma_1 $ to $\Gamma_1, x_i : \type_i \land \type_i =\Gamma_2(x_i) $. With the
   Theorem 11, from $(\star)$, we know :\\
   $\Gamma_1, x_i : \type_i \tvdash{\nnatA_1} \expr_1:  \bang{0} ( \type_1
   \lto \type_2  )\whynot{r}  ~(\star \star)   $.\\
   
   For all the variables $x_i' $ in $\dom(\max(\Gamma_1, \Gamma_2  )) $ and
   $x_i \not\in \dom(\Gamma_2)$. 
   We extend $\Gamma_2 $ to $\Gamma_2, x_i' : \type_i' \land \type'=\Gamma_1(x_i')$. With the
   Theorem 11, from $(\diamond)$, we know :\\
   $  \Gamma_2, x_i': \type_i' \tvdash{\nnatA_2} \expr_2: \type_1 ~(\diamond\diamond)$.   \\

   Assume $\env$ satisfies $\min(\Gamma_1, \Gamma_2)$. \\
   $ \min(\Gamma_1,\Gamma_2) = \{  \Gamma | \dom(\Gamma) = \dom(\Gamma_1)
   \cup \dom(\Gamma_2) \land    \Gamma(x) = \min(\Gamma_1(x),
   \Gamma_2(x) )   \} $. \\

  From Theorem 10, we conclude that $ \env $ satisfies $ \Gamma_1, x_i
  : \type_i $, as well as  $ \env $ satisfies $ \Gamma_2, x_i'
  : \type_i' $, also  $\env$ satisfies $ \max(\Gamma_1, \Gamma_2)$.\\

  By IH on $(\star\star)$, assume $ \env, \expr_1 \bigstep \valr_1,
  \tr_1  $,  we know: $ \empty  \tvdash{\nnatA_1- A(\tr_1)} \valr_1:  \bang{0} ( \type_1
  \lto \type_2  )\whynot{0}  \land A(\tr_1) \leq r ~(e) $.\\
  Because $\valr_1$ is a function, we assume: $\valr_1 = ( \lambda
  x.\expr, \env')$. So we have:  $ \empty \tvdash{\nnatA_1- A(\tr_1)} ( \lambda
  x.\expr, \env'):  \bang{0} ( \type_1
  \lto \type_2  )\whynot{0}  ~(\clubsuit)$, from the closure rule, we
  know : $\env' $ satisfies $\Gamma_1' \land  \Gamma_1'  \tvdash{\nnatA_1- A(\tr_1)}  \lambda
  x.\expr:  \bang{0} ( \type_1
  \lto \type_2  )\whynot{0}   ~(\clubsuit\clubsuit)$. \\

  By IH on $(\diamond\diamond)$, assume $\env, \expr_2 \bigstep
  \valr_2, \tr_2  $, we know $\empty
  \tvdash{\nnatA_2- A(\tr_2)} \valr_2: \type_1[?0]~(a)$. \\

  From $(\clubsuit \clubsuit )$ and the rule lambda:
  \[
    \inferrule{
        \Gamma_1' , x: \type_1
      \tvdash{ \nnatA_1- A(\tr_1) }
      \expr: \type_2 (\heartsuit)
    }{
       \Gamma_1' \tvdash{\nnatA_1- A(\tr_1) + 0} \lambda x. \expr : \bang{0} ( \type_1
  \lto \type_2  )
    }~\textbf{lambda}
  \]
  
 We know that : $ \env'[x \to \valr_2] $satisfies $ \Gamma_1', x: \type_1$ and
 assume $\env'[ x \mapsto \valr_2], \expr \bigstep \valr, \tr$.\\
By IH on $(\heartsuit)$, we know:  $ \empty \tvdash{\nnatA_1- A(\tr_1)
  - A(\tr)} \valr : \type_2[?0]~(d) $ and $ \nnatA_1- A(\tr_1)
  - A(\tr) \geq 0 \implies \nnatA_1 \geq A(\tr_1)
  + A(\tr)~ (b)$.\\

  From the theorem Depth Bound on $\heartsuit $, we assume that
  $\type_1 = \bang{k_1} \ltype_1 \whynot{r_1}$.\\
So we know that $ k_1 \geq \ddep{x}(\tr) $. \\

By the Theorem 13 on $(a)$, we know that:
 $ \nnatA_2- A(\tr_2) \geq k_1 \implies  \nnatA_2 - A(\tr_2) \geq \ddep{x}(\tr) $. \\
  So we conclude that  $\nnatA_2 \geq A(\tr_2) + \ddep{x}(\tr) ~(c)$.\\     

From the application evaluation rule, we have:
  \[
\inferrule{
    \env, \expr_1 \bigstep \valr_1, \tr_1 \\
    \wq{ \valr_1 = ( \lambda x.\expr, \env')} \\\\
    \env, \expr_2 \bigstep \valr_2, \tr_2 \\
    \env'[ x \mapsto \valr_2], \expr \bigstep \valr, \tr
  }{
    \env, \expr_1 \eapp \expr_2 \bigstep \valr, \trapp{\tr_1}{\tr_2}{f}{x}{\tr}
  }
  \]

TS1:  $ \empty \tvdash{ \max( \nnatA_1,r+\nnatA_2)  -  A(\trapp{\tr_1}{\tr_2}{f}{x}{\tr}  )}
\valr : \type_2[?0] $.\\
        It is proved from (d) and Theorem Aaptivity Monotonicity.\\
TS2: $ \max( \nnatA_1,r+\nnatA_2) \geq A(\trapp{\tr_1}{\tr_2}{f}{x}{\tr} $.\\
      It is proved by (b),(c), (e).\\
  
  \end{proof}


  \begin{thm}
    $\empty \tvdash{\nnatA }   \valr : \bang{k} \ltype \whynot{r}  $ and $k \leq
    k' $ and $r \leq r'$, then exists $\nnatA'$ so that $\empty
    \tvdash{\nnatA'}  \valr : \bang{k'} \ltype \whynot{r'} $.
  \end{thm}
%   \begin{proof}
%     By induction on the value $\valr$.\\
%     \caseL{
%         \inferrule{
%     }{
%      \empty \tvdash{\nnatA} c : \bang{\nnatA}\tbase
%     }~\textbf{b}
%   }

%   Assume $\nnatA \leq k'$. From the const rule, we know:\\
%    \[   \inferrule{
%     }{
%      \empty \tvdash{k'} c : \bang{k'}\tbase
%     }~\textbf{b}
%   \]
  
%   \caseL{
% \inferrule{
%       x: \type_1
%       \tvdash{k}
%       \expr: \type_2
%     }{
%       \empty \tvdash{k+r} \lambda x. \expr : \bang{r}  ( \type_1
%       \lto \type_2)
%     }~\textbf{lambda}
%   }
%   Assume $ k' = r + r'  \geq r$. Choose $\nnatA' = k+r +r'  $ so
%   that by the rule lambda:\\
%   \[
% \inferrule{
%       x: \type_1
%       \tvdash{k}
%       \expr: \type_2
%     }{
%       \empty \tvdash{k+r+r'} \lambda x. \expr : \bang{k'}  ( \type_1
%       \lto \type_2)
%     }~\textbf{lambda}
%   \]
    
%     \end{proof}


  
  \begin{thm}
   $\Gamma \tvdash{\nnatA} \expr: \type$ and $ x \not \in \fv{\expr}
   $, then  $\forall \type'. \Gamma, x: \type' \tvdash{\nnatA} \expr: \type  $. 
   \end{thm}

\begin{thm}
  If  $\Gamma \tvdash{\nnatA} \expr: \bang{k} \ltype \whynot{r} $ and
  $\env |= \Gamma$ and  $\env, \expr \bigstep
  \valr, \tr$, then $\adap(\tr) \leq r $. 
\end{thm}

\begin{thm}
  If  $\Gamma \tvdash{\nnatA} \expr: \bang{k} \ltype \whynot{r} $,
  then $\nnatA \geq k$.
   \end{thm}

 \begin{thm}[Depth Bound]
  If  $\Gamma, x: \bang{k} \ltype \whynot{r} \tvdash{\nnatA} \expr: \type$ and
  $\env |= \Gamma, x: \bang{k} \ltype \whynot{r}$ and  $\env, \expr \bigstep
  \valr, \tr$, then $\ddep{x}(\tr) \leq k$. 
\end{thm}

