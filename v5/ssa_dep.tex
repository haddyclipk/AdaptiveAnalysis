\subsection{SSA form Language with Different DEPENDENCY Graph}
\[
\begin{array}{llll}
%
\mbox{While Map} 
& w & ::= & [] ~|~  w[l \to n]
\\
%
\mbox{Annotated Query} 
& \aq & ::= & (\qval, l , w )
\\
%
\mbox{Annotated Variable} 
& \av & ::= & (\ssa{x}, v, l , w )
\\
%
\mbox{Query Trace} & \qtrace
& ::= & [] | \aq :: \qtrace
\\
%
\mbox{Variable Trace} & \vtrace
& ::= & [] | \av :: \qtrace
\end{array}
\]
%
We use $\mathcal{SVAR}$ and $\mathcal{SM}$ to denote the set of SSA Variables and SSA Memories respectively.
%
\[
\begin{array}{llll}
%
\mbox{Annotated Variable} 
& \av & ::= & (\ssa{x}, v, n)
\\
%
\mbox{Variable Trace} & \vtrace
& ::= & [] | \av :: \vtrace
\\
%
\mbox{Variable Visits} & t
& ::= & \mathcal{SVAR} \to \mathbb{N}
\end{array}
\]
%
We use $\mathcal{SVAR}$ and $\mathcal{SM}$ to denote the set of SSA Variables and SSA Memories respectively.

\todo{
\begin{defn}[Assigned Variables ($\avar$)]
Given a program $\ssa{c}$, its assigned variables $\avar_{\ssa{c}}$ is a vector containing all variables newly assigned in the program preserving the order, $\forall \ssa{x} \in \avar, \ssa{x} \in \mathcal{SVAR}$.
It is defined as follows:
$$
  \avar_{\ssa{c}} \triangleq
  \left\{
  \begin{array}{ll}
   		[\ssa{x}] 									
   		& \ssa{c} = [\ssa{\assign x e}]^{(l, w)} 
   		\\
     	\left[ \ssa{x} \right] 									
     	& \ssa{c} = [\ssa{\assign x \query(\qexpr)}]^{(l, w)} 
     	\\
     	\avar_{\ssa{c_1}} ++ \avar_{\ssa{c_2}} 	
     	& \ssa{c} = \ssa{c_1};\ssa{c_2}
     	\\
     	\avar_{\ssa{c_1}} ++ \avar_{\ssa{c_2}} ++ \ssa{[\bar{x}, \bar{y}, \bar{z}]} 
     	& \ssa{c} =\eif([\sbexpr]^{(l, w)} , \ssa{[\bar{x}, \bar{x_2}, \bar{x_2}], 
     	[\bar{y}, \bar{y_2}, \bar{y_3}], 
     	[\bar{z}, \bar{z_2}, \bar{z_3}], c_1, c_2}) 
     	\\
     	\avar_{\ssa{c}'} ++ [\ssa{\bar{x}}]
     	& \ssa{c} 	= \ewhile ([\sbexpr]^{(l, w)}, [\ssa{\bar{x}, \bar{x_2}, \bar{x_2}}], \ssa{c}')
\end{array}
\right.
$$
\end{defn}
%
\begin{defn}[Query Variables ($\qvar$)].
\\
Given a program $\ssa{c}$, its query variables $\qvar$ is a vector containing all variables newly assigned by a query in the programm, $\qvar \subset \mathcal{SVAR}$.
It is defined as follows:
$$
  \qvar_{\ssa{c}} \triangleq
  \left\{
  \begin{array}{ll}
   		[] 									
   		& \ssa{c} = [\ssa{\assign x e}]^{(l, w)} 
   		\\
     	\left[ \ssa{x} \right] 									
     	& \ssa{c} = [\ssa{\assign x \query(\qexpr)}]^{(l, w)} 
     	\\
     	\qvar_{\ssa{c_1}} ++ \qvar_{\ssa{c_2}} 	
     	& \ssa{c} = \ssa{c_1};\ssa{c_2}
     	\\
     	\qvar_{\ssa{c_1}} ++ \qvar_{\ssa{c_2}} ++ \ssa{[\bar{x}, \bar{y}, \bar{z}]} 
     	& \ssa{c} =\eif([\sbexpr]^{(l, w)} , \ssa{[\bar{x}, \bar{x_2}, \bar{x_2}], 
     	[\bar{y}, \bar{y_2}, \bar{y_3}], 
     	[\bar{z}, \bar{z_2}, \bar{z_3}], c_1, c_2}) 
     	\\
     	\qvar_{\ssa{c}'} ++ [\ssa{\bar{x}}]
     	& \ssa{c} 	= \ewhile ([\sbexpr]^{(l, w)}, [\ssa{\bar{x}, \bar{x_2}, \bar{x_2}}], \ssa{c}')
\end{array}
\right.
$$
\end{defn}
%
We are abusing the notations and operators from list here. 
The notation $[]$ represents an empty vector
and $x::A$ represents add an element $x$ to the head of the vector $A$.
The concatenation operation between 2 vectors $A_1$ and $A_2$, i.e., $A_1 ++ A_2$ is mimic the standard list concatenation operations as follows:
%
\begin{equation}
		A_1 ++ A_2  
		\triangleq \left\{
		\begin{array}{ll} 
			A_2 				& A_1 = []\\
			x::(A_1' ++ A_2)	& A_1 = x::A_1'
		\end{array}
		\right.
\end{equation}
%
We use index within parenthesis to denote the access to the element of corresponding location,
$A(i)$ denotes the element at location $i$ in the vector $A$ and 
$M(i, j)$ denotes the element at location $i$-th raw, $i$-th column in the matrix $M$. 
}
}
%
%
%
%
{
\begin{figure}
\todo{
\begin{mathpar}
\boxed{\config{\ssa{m, a}} \xrightarrow{} \config{\ssa{a}} \;} 
\and
\boxed{ \config{\ssa{m, c}, t} \xrightarrow{} \config{\ssa{ m', c'},  t'}}
\and
{Memory \times Command  \times QTrace \times VTrace \times While Map 
\Rightarrow^{}  Memory \times Command  \times QTrace \times VTrace \times While Map}
\\
{
	\inferrule
	{
		\ssa{\config{m,\qexpr} \qarrow \qexpr'}
	}
	{
	\config{\ssa{m, [\assign{x}{\query(\qexpr)}]^l, \qtrace, \vtrace, w } }
	\xrightarrow{} 
	\config{\ssa{m, [\assign{x}{\query(\qexpr')}]^l, \qtrace, \vtrace, w }}
	}
	~\textbf{ssa-query-e}}
%
\and
{
\inferrule
{
{
\ssa{\query(\qval)(D) = v}}
}
{
\config{\ssa{m, [\assign{\ssa{x}}{\ssa{\query(\qval)}}]^l, \qtrace, \vtrace, w}} 
\xrightarrow{} 
\config{\ssa{ (x \mapsto v)::m, \eskip, \qtrace ++ [(\qval, l, w)], \vtrace ++ [(\ssa{x}, v, l, w)], w}}}
~\textbf{ssa-query-v}
}
%
\and
%
{
\inferrule
{
t'(\ssa{x}) = t(\ssa{x}) + 1
}
{
\config{\ssa{ m, [\assign x v]^{l}, \qtrace, \vtrace, w} } 
\xrightarrow{} 
\config{\ssa{(x \mapsto v)::m, [\eskip]^{l}, \qtrace, \vtrace ++ [(\ssa{x}, v, l, w)], w} }
}
~\textbf{ssa-assn}
}
%
\end{mathpar}
}
\todo{
\begin{mathpar}
\boxed{\config{\ssa{m, a}} \xrightarrow{} \config{\ssa{a}} \;} 
\and
\boxed{ \config{\ssa{m, c}, t} \xrightarrow{} \config{\ssa{ m', c'},  t'}}
\and
{Memory \times Command  \times VTrace \times Visits
\Rightarrow^{}  
Memory \times Command  \times VTrace \times Visits}
\\
{
  \inferrule
  {
    \ssa{\config{m,\qexpr} \qarrow \qexpr'}
  }
  {
  \config{\ssa{m, [\assign{x}{\query(\qexpr)}]^l, \vtrace, t } }
  \xrightarrow{} 
  \config{\ssa{m, [\assign{x}{\query(\qexpr')}]^l, \vtrace, t }}
  }
  ~\textbf{ssa-query-e}}
%
\and
{
\inferrule
{
{
\ssa{\query(\qval)(D) = v}}
\and 
t'[\ssa{x}]  = t[\ssa{x}] + 1
}
{
\config{\ssa{m, [\assign{\ssa{x}}{\ssa{\query(\qval)}}]^l, \vtrace, t}} 
\xrightarrow{} 
\config{\ssa{ (x \mapsto v)::m, \eskip, \vtrace ++ [(\ssa{x}, \qval, t'[\ssa{x}])], t'}}}
~\textbf{ssa-query-v}
}
%
\and
%
{
\inferrule
{
t'(\ssa{x}) = t(\ssa{x}) + 1
}
{
\config{\ssa{ m, [\assign x v]^{l}, t, w} } 
\xrightarrow{} 
\config{\ssa{(x \mapsto v)::m, [\eskip]^{l}, \vtrace ++ [(\ssa{x}, v, t'[\ssa{x}])], t'} }
}
~\textbf{ssa-assn}
}
%
\end{mathpar}
}
    \caption{Operational Semantics for the SSA Language}
\end{figure}
%
%
%
%
\subsection{Trace-based Adaptivity in SSA Language}
%
\begin{defn}
[remove :?: Query May Dependency in SSA Language].
\\
{
One annotated query $\aq_1 = ({\qval}_1, l_1, w_1)$ may depend on another query $\aq_2 = ({\qval}_2,l_2, w_2)$ in a program $\ssa{c}$,
with a starting memory $\ssa{m}$ and  hidden database $D$, denoted as 
%
$\qdep^{ssa}(\aq_1, \aq_2, \ssa{c}, \ssa{m}, D)$ is defined below. 
%
%
\[
\begin{array}{l}
\exists \ssa{m}_1,\ssa{m}_3,t_1,t_3,\ssa{c}_2,v_1.\\
  \left (\begin{array}{l}   
\config{\ssa{m}, \ssa{c}, [], []} \rightarrow^{*} 
\config{\ssa{m}_1, [\assign{\ssa{x}}{\query({\qval}_1)}]^{l_1} ; c_2,
  t_1, w_1} 
\rightarrow^{\textbf{{ssa-query-v}}} 
\\ 
\config{\ssa{m}_1[v_1/\ssa{x}], c_2, t_1 ++ [\aq_1], w_1} 
\rightarrow^{*} \config{\ssa{m}_3, \eskip, t_3,w_3}
  % 
 \\ \bigwedge
  \left( 
  \begin{array}{l}
  \aq_2 \aqin (t_3 - (t_1 ++ [\aq_1])) 
  % 
  \\
  \implies 
  \exists v \in \qdom, v \neq v_1, \ssa{m}_3', t_3', w_3'.  
  \config{\ssa{m}_1[v/\ssa{x}], {\ssa{c}_2}, t_1 ++ [\aq_1], w_1} 
  \\ 
  \quad \quad 
  \rightarrow^{*}
  (\config{\ssa{m}_3', \eskip, t_3', w_3'} 
  \land 
  \aq_2 \not \aqin (t_3'-(t_1 ++ [\aq_1])))
\end{array} \right )
\\\bigwedge
\left( 
  \begin{array}{l}
  	\aq_2 \not\aqin (t_3 - (t_1 ++ [\aq_1]))
  	% 
  	\\
  	\implies 
	\exists v \in \qdom, v \neq v_1, \ssa{m}_3', t_3', w_3'. 
	\config{\ssa{m}_1[v/\ssa{x}], {\ssa{c}_2}, t_1 ++ [\aq_1], w_1}
	\\ 
	\quad \quad 
	\rightarrow^{*} 
	(\config{\ssa{m}_3', \eskip, t_3', w_3'} 
	\land 
	\aq_2  \aqin (t_3' - (t_1 ++ [\aq_1])))
\end{array} \right )
\end{array} \right )
\end{array}
\]
}
\end{defn}
%
%
\begin{defn}
[remove :?: Query Variable May Dependency].
\label{def:qvar_dep}
\\
{
One annotated ssa variable $\av_2 = (\ssa{x}_2,l_2, w_2)$ may depend on another one 
$\av_1 = (\ssa{x}_1, l_1, w_1)$ in a program $c$,
with a starting memory $m$ and a hidden database $D$, denoted as 
%
$\vardep^{ssa}(\av_1, \av_2, c, m, D)$ is defined below. 
%
\[
\exists \qval_1, \qval_2. ~
\aq_1 = (\qval_1, l_1, w_1)
\land
\aq_2 = (\qval_2, l_2, w_2)
\land 
\qdep^{ssa}(\aq_1, \aq_2, c, m, D)
\]
}
\end{defn}
%
\begin{defn}
[Annotated Variables May Dependency in SSA Language]
\label{def:avar_dep}.
\\
{
One annotated variable $\av_2 = (\ssa{x}_2, v_2, l_2, w_2)$ may depend on another one  $\av_1 = (\ssa{x}_1, v_1, l_1, w_1)$in a program $\ssa{c}$,
with a starting memory $\ssa{m}$ and  hidden database $D$, denoted as 
%
$\avdep(\av_1, \av_2, \ssa{c}, \ssa{m}, D)$ is defined below. 
%
%
\[
\begin{array}{l}
\exists \ssa{m}_1, \ssa{m}_3, \vtrace_1, \vtrace_3, \ssa{c}_2, v_1, ({\qval}_1 \lor \sexpr_1).\\
  \left (\begin{array}{l}   
\config{\ssa{m}, \ssa{c}, [], []} \rightarrow^{*} 
\config{\ssa{m}_1, [\assign{\ssa{x}_1}{\query({\qval}_1) (/ \sexpr_1)}]^{l_1} ; \ssa{c}_1, \qtrace_1,  \vtrace_1, w_1} 
\rightarrow^{\textbf{ssa-query-v (/ assn-v)}} 
\\ 
\config{\ssa{m}_1[v_1/\ssa{x}], c_2, \qtrace_1', \vtrace_1 ++ [\av_1], w_1} 
\rightarrow^{*} \config{\ssa{m}_3, \eskip, \qtrace_3, \vtrace_3, w_3}
  % 
 \\ \bigwedge
  \left( 
  \begin{array}{l}
  \av_2 \in (\vtrace_3'-(\vtrace_1 ++ [\av_1])) 
  % 
  \\
  \implies 
  \exists v \in \qdom, v \neq v_1, \ssa{m}_3', \qtrace_3', \vtrace_3', w_3'.  
  \config{\ssa{m}_1[v/\ssa{x}], {\ssa{c}_2}, \qtrace_1', \vtrace_1 ++ [\av_1], w_1} 
  \\ 
  \quad \quad 
  \rightarrow^{*}
  (\config{\ssa{m}_3', \eskip, \qtrace_3', \vtrace_3', w_3'} 
		\\ 
		\quad \quad 
  \land 
  \av_2 \not\in (\vtrace_3'-(\vtrace_1 ++ [\av_1])))
\end{array} \right )
\\\bigwedge
\left( 
  \begin{array}{l}
  	\av_2 \notin (\vtrace_3 - (\vtrace_1 ++ [\av_1]))
  	% 
  	\\
  	\implies 
	\exists v \in \qdom, v \neq v_1, \ssa{m}_3', \qtrace_3', \vtrace_3', w_3'. 
	\config{\ssa{m}_1[v /\ssa{x}], {\ssa{c}_2}, \qtrace_1', \vtrace_1 ++ [\av_1], w_1}
	\\ 
	\quad \quad 
	\rightarrow^{*} 
	(\config{\ssa{m}_3', \eskip, \qtrace_3', \vtrace_3', w_3'} 
		\\ 
		\quad \quad 
	\land 
	\av_2  \in (\vtrace_3' - (\vtrace_1 ++ [\av_1])))
\end{array} \right )
\end{array} \right )
\end{array}
\]
%
}
\end{defn}

\begin{defn}
[Annotated Variables May Dependency in SSA Language -- Version 2]
\label{def:avar_dep2}.
\\
{
One annotated variable $\av_2 = (\ssa{x}_2, v_2, n_2)$ may depend on another one  $\av_1 = (\ssa{x}_1, v_1, n_1)$in a program $\ssa{c}$,
with a starting memory $\ssa{m}$ and hidden database $D$, denoted as 
%
$\avdep(\av_1, \av_2, \ssa{c}, \ssa{m}, D)$ is defined below. 
%
\[
\begin{array}{l}
\exists \ssa{m}, \ssa{m}_1, \ssa{m}_2, \ssa{m}_3, \ssa{m}_2', \ssa{m}_3', 
\vtrace_1, \vtrace_2, \vtrace_2', t_1, t_2, t_2', \ssa{c}_1, \ssa{c}_2, v_1'.\\
  \left (\begin{array}{l}   
\config{\ssa{m}, \ssa{c}, []} \rightarrow^{*} 
\config{\ssa{m}_1, [\assign{\ssa{x}_1}{\query({\qval}_1) (/ \sexpr_1)}]^{l_1} ; \ssa{c}_1, \vtrace_1, t_1} 
\\ 
\config{\ssa{m}_1[v_1/\ssa{x}_1], c_1, \vtrace_1 ++ [\av_1], t_1[\ssa{x}_1]++} 
\rightarrow^{*} 
\config{\ssa{m}_2, [\assign{\ssa{x}_2}{\query({\qval}_2) (/ \sexpr_2)}]^{l_2} ; \ssa{c}_2, \vtrace_2, t_2} 
\rightarrow^{\textbf{{ssa-query-v} (/ assn-v)}} 
\config{\ssa{m}_3, \ssa{c}_2,  \vtrace_2 ++ [\av_2], t_2[\ssa{x}_2]++} 
  % 
 \\ 
 \bigwedge
 \config{\ssa{m}_1[v_1'/\ssa{x}_1], \ssa{c}_1, \vtrace_1, t_1} 
\rightarrow^{*} 
\config{\ssa{m}_2', \ssa{c}_2,  \vtrace_2', t_2'}
\\
\bigwedge
\av_2 \notin \vtrace_2'
\end{array}
\right)
\end{array}
 \]
where $\ssa{m}(x) = \kw{null}$ if $x \notin \kw{dom}(\ssa{m})$.
}
\end{defn}
%
\todo{
\begin{defn}[Variable May Dependency]
\label{def:var_dep}.
\\
Given a program $\ssa{c}$ with its assigned variables $\avar_{\ssa{c}}$, 
one variable $\ssa{x}_2 \in \avar_{\ssa{c}}$ may depend on another variable 
$\ssa{x}_1 \in \avar_{\ssa{c}}$ in $\ssa{c}$ denoted as 
%
$\vardep(\ssa{x}_1, \ssa{x}_2, \ssa{c})$ is defined below.
%
\[
\exists v_1, v_2, n_1, n_2, m, D. ~
\av_1 = (\ssa{x}_1, v_1, n_1)
\land
\av_2 = (\ssa{x}_2, v_2, n_2)
\land 
\avdep(\av_1, \av_2, c, m, D)
\] 
%
%
\end{defn}
%
%
\begin{defn}[Execution Based Dependency Graph].
\\
Given a program $\ssa{c}$, a database $D$, a starting memory $\ssa{m}$ with its corresponding execution:
$\config{\ssa{m}, \ssa{c}, [], [], []} 
\to^{*}
\config{\ssa{m'}, \eskip, \qtrace, \vtrace, []}$,
the dependency graph $\traceG(\ssa{c},\text{D},\ssa{m}) = (\vertxs, \edges)$ is defined as: \\
\[
\begin{array}{rlcl}
	\text{Vertices} &
	\vertxs & := & \left\{ 
	x \in \mathcal{SVAR}
	~ \middle\vert ~
	x \in \avar_{\ssa{c}}
	\right\}
	\\
	\text{Directed Edges} &
	\edges & := & 
	\left\{ 
	(x, x') \in \mathcal{SVAR} \times \mathcal{SVAR}
	~ \middle\vert ~
	\vardep(x, x', \ssa{c}) 
	\right\}
	\\
	\text{Query Vertices} &
	\qvertxs & := & 
	\left\{ 
	x \in \mathcal{SVAR}
	~ \middle\vert ~
	x \in \qvar_{\ssa{c}}
	\right\}
	\\
	\text{Weights} &
	\weights & := & 
	\left\{ 
	(x, n) \in \mathcal{SVAR} \times \mathbb{N}
	~ \middle\vert ~
	\len(\{\av \mid \av \in \vtrace \land \projl{\av} = x\}) = n, \forall x \in \qvar_{\ssa{c}}
	\right\}
\end{array}
\]
\end{defn}
}
%
%
%
\begin{defn}
[Adaptivity of A Program in SSA Language].
\\
Given a program $\ssa{c}$ in SSA language, 
its adaptivity is defined as the length of the longest path in its trace-based dependency graph in SSA language 
$\traceG(\ssa{c}, \text{D},\ssa{m}) = (\vertxs, \edges)$, 
for all possible starting SSA memory $\ssa{m}$ and database $D$.
%
%
$$
A(\ssa{c}) = \max \big 
\{ \qlen(k) \mid \ssa{m} \in \mathcal{SM},D \in \dbdom , k \in \walks(\traceG(\ssa{c}, \text{D}, \ssa{m}) \big \} 
$$
\end{defn}
%


