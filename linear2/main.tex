\documentclass[a4paper,11pt]{article}

\usepackage{mathpartir}
\usepackage{amsmath,amsthm,amsfonts}
\usepackage{ amssymb }
\usepackage{color}
\usepackage{algorithm}
\usepackage{algorithmic}
\usepackage{microtype}


%%% Attempt 1: Linear 1



\newcommand{\diam}{{\color{red}\diamond}}
\newcommand{\dagg}{{\color{blue}\dagger}}
\let\oldstar\star
\renewcommand{\star}{\oldstar}

\newcommand{\im}[1]{\ensuremath{#1}}

\newcommand{\kw}[1]{\im{\mathtt{#1}}}


\newcommand{\set}[1]{\im{\{{#1}\}}}

\newcommand{\mmax}{\ensuremath{\mathsf{max}}}

%%%%%%%%%%%%%%%%%%%%%%%%%%%%%%%%%%%%%%%%%%%%%%%%%%%%%%%%
% Comments
\newcommand{\omitthis}[1]{}

% Misc.
\newcommand{\etal}{\textit{et al.}}
\newcommand{\bump}{\hspace{3.5pt}}

% Text fonts
\newcommand{\tbf}[1]{\textbf{#1}}
%\newcommand{\trm}[1]{\textrm{#1}}

% Math fonts
\newcommand{\mbb}[1]{\mathbb{#1}}
\newcommand{\mbf}[1]{\mathbf{#1}}
\newcommand{\mrm}[1]{\mathrm{#1}}
\newcommand{\mtt}[1]{\mathtt{#1}}
\newcommand{\mcal}[1]{\mathcal{#1}}
\newcommand{\mfrak}[1]{\mathfrak{#1}}
\newcommand{\msf}[1]{\mathsf{#1}}
\newcommand{\mscr}[1]{\mathscr{#1}}









\newcommand{\defeq}{\mathrel{\doteq}}
\newcommand{\conj}{\mathrel{\wedge}}
\newcommand{\disj}{\mathrel{\vee}}

\newcommand{\lzero}{0}


% context
\newcommand{\tctx}{\Gamma}
\newcommand{\ictx}{ }


% expression
\newcommand{\expr}{e}
\newcommand{\aexpr}{a}
\newcommand{\bexpr}{b}
\newcommand{\sexpr}{\textrm{e} }
\newcommand{\qexpr}{\psi}
\newcommand{\qval}{\alpha}
\newcommand{\query}{{\tt query}}
\newcommand{\saexpr}{\textrm{a} }
\newcommand{\sbexpr}{\textrm{b} }
\newcommand{\vall}{w}
\newcommand{\valr}{v}
\newcommand{\eif}{\kw{if}}
\newcommand{\eapp}{\;}
\newcommand{\eprojl}{\kw{fst}}
\newcommand{\eprojr}{\kw{snd}}
\newcommand{\eifvar}{\kw{ifvar}}
%expression and commands for WHILE language
\newcommand{\ewhile}{\kw{while}}
\newcommand{\bop}{*}
\newcommand{\uop}{\circ}
\newcommand{\eskip}{\kw{skip}}

\newcommand{\eloop}{\kw{loop}}
\newcommand{\edo}{\kw{do}}
\newcommand{\qdom}{\mathcal{QD}}

%configuration
\newcommand{\config}[1]{\langle #1 \rangle}
\newcommand{\ematch}{\kw{match}}
\newcommand{\clabel}[1]{\left[ #1 \right]}


%\newcommand{\eprov}[1]{\eta_{#1}}
\newcommand{\etrue}{\kw{true}}
\newcommand{\efalse}{\kw{false}}
\newcommand{\econst}{c}
\newcommand{\eop}{\delta}
\newcommand{\efix}{\mathop{\kw{fix}}}
\newcommand{\elet}{\mathop{\kw{let}}}
\newcommand{\ein}{\mathop{ \kw{in}} }
\newcommand{\eas}{\mathop{ \kw{as}} }
\newcommand{\enil}{\kw{nil}}
\newcommand{\econs}{\mathop{\kw{cons}}}
%\newcommand{\labelA}{\ell}
%monad expressions / terms
\newcommand{\term}{t}
\newcommand{\return}{\kw{return}}
\newcommand{\bernoulli}{\kw{bernoulli}}
\newcommand{\uniform}{\kw{uniform}}
 \newcommand{\epack}{\mbox{pack\;}}
\newcommand{\eunpack}{\mbox{unpack\;}}
\newcommand{\eilam}{\Lambda.}

\newcommand{\evec}{\kw{dict}}
\newcommand{\eget}{\kw{get}}

% trace
\newcommand{\triapp}[2]{\kw{IApp}(#1,#2)}
\newcommand{\trow}{\text{row}}
\newcommand{\tr}{T}
\newcommand{\trift}{\eif^{\kw{t}}}
\newcommand{\triff}{\eif^{\kw{f}}}
\newcommand{\trprojl}{\eprojl}
\newcommand{\trprojr}{\eprojr}
\newcommand{\trtrue}{\etrue}
\newcommand{\trfalse}{\efalse}
\newcommand{\trconst}{\econst}
\newcommand{\trop}{\eop}
\newcommand{\trfix}{\efix}
\newcommand{\trapp}[5]{#1 \; #2 \mathrel{\triangleright} {\efix
#3(#4).#5}}
\newcommand{\trnil}{\enil}
\newcommand{\trcons}{\econs}
\newcommand{\trlet}{\elet}
%types for monad
\newcommand{\treal}{\kw{real}}
\newcommand{\tint}{\kw{int}}
\newcommand{\tmonad}{\kw{M}}
\newcommand{\tunit}{\kw{unit}}
\newcommand{\tdb}{\kw{tdb}}

% adaptivity
\newcommand{\adap}{\kw{adap}}
\newcommand{\ddep}[1]{\kw{depth}_{#1}}
\newcommand{\nat}{\mathbb{N}}
\newcommand{\natb}{\nat_{\bot}}
\newcommand{\natbi}{\natb^\infty}
\newcommand{\nnatA}{Z}
\newcommand{\nnatB}{m}
\newcommand{\nnatbA}{s}
\newcommand{\nnatbB}{t}
\newcommand{\nnatbiA}{q}
\newcommand{\nnatbiB}{r}

%type
\newcommand{\type}{\tau}
\newcommand{\tbase}{\kw{b}}
\newcommand{\tbool}{\kw{bool}}
\newcommand{\tbox}[1]{ \kw{\square} \, (#1) }
\newcommand{\tarr}[5]{#1; #3 \xrightarrow{#4; \, #5} #2}
\newcommand{\tlist}[1]{\kw{list} \, #1 }
\newcommand{\env}{\theta}
\newcommand{\tforall}[3]{\forall#3 \overset{#1, #2}{::} S.\, }
\newcommand{\texists}[1]{\exists#1 {::} S.\, }
\newcommand{\lto}{\multimap}
\newcommand{\bang}[1]{ !_{#1}}
\newcommand{\whynot}[1]{ ?_{#1} }
\newcommand{\ltype}{A}
\newcommand{\adapt}{R}
% index
\newcommand{\idx}{I }
\newcommand{\smax}[2]{\kw{max}(#1,#2)}
\newcommand{\ienv}{\sigma}

%evaluation
\newcommand{\bigstep}[1]{\mathrel{\to^{#1}}}

\newcommand{\dmap}{\rho}
\newcommand{\dmapb}{\bot_\dmap}
\newcommand{\supp}{\kw{supp}}
\newcommand{\dom}{\kw{dom}}
\newcommand{\codom}{\kw{codom}}

\newcommand{\tvdash}[1]{\vdash_{#1}}

\newcommand{\lrv}[1]{[\![ #1 ]\!]_{\text{V}}}
\newcommand{\lre}[3]{[\![ #3 ]\!]_{\text{E}}^{#1, #2}}
\newcommand{\stepiA}{k}
\newcommand{\stepiB}{j}
\newcommand{\size}[1]{|#1|}

%logic relations
\newcommand{\lr}[1]{[\![ #1 ]\!]}
\newcommand{\lrt}[1]{\mathcal{T}[\![ #1 ]\!]}


\newcommand{\wf}[1]{\vdash #1 \, \kw{wf} }
\newcommand{\sub}[2]{ #1 \, <: \, #2 }
\newcommand{\eqv}[3]{ #1 \, \equiv \, #2 \Rightarrow \textcolor{red}
{#3}  }
\newcommand{\eqvt}[3]{ #1 \, \sqsubseteq \, #2 \Rightarrow \textcolor{red}
{#3}  }
\newcommand{\eqvc}[2]{ #1 \, \equiv^c \, #2   }


%core calculus
\newcommand{\ctyping}[3]{ \tvdash{ #1} {#2} :^c #3 }
\newcommand{\cbox}{\mathsf{box}}
\newcommand{\cder}{\mathsf{der}}
\newcommand{\elab}[4]{ \vdash_{ #1} #2 \rightsquigarrow #3 : #4}
\newcommand{\coerce}[2]{\mathsf{coerce}_{#1, #2}}

%algorithmic typing rules
\newcommand{\infr}[4]{{#1} ~ {\textcolor{red}\uparrow} ~ {\color{red} #2} \Rightarrow
{ } {\color{red} #3} }
\newcommand{\chec}[3]{{#1} ~ {\downarrow} ~ {#2} \Rightarrow {\color{red} #3} }
% \newcommand{\restriction}{\Phi}
\newcommand{\fresh}{ \mathsf{fresh}}
\newcommand{\red}[1]{ \textcolor{red} {#1} }
\newcommand{\fiv}[1]{ \mathsf{FIV} (#1)   }
\newcommand{\fv}[1]{ \mathsf{FV} (#1)   }

\newcommand{\todo}[1]{{\small \color{red}\textbf{[[ #1 ]]}}}
\newcommand{\todomath}[1]{{\scriptstyle \color{red}\mathbf{[[ #1 ]]}}}

\newcommand{\caseL}[1]{\item \textbf{#1}\newline}

\newcommand{\attr}{\mathsf{attr}}
\newcommand{\weight}{\mathsf{W}}
\newcommand{\num}{\mathsf{n}}

\usepackage{enumitem}
\setenumerate{listparindent=\parindent}

\newlist{enumih}{enumerate}{3}
\setlist[enumih]{label=\alph*),before=\raggedright, topsep=1ex, parsep=0pt,  itemsep=1pt }

\newlist{enumconc}{enumerate}{3}
\setlist[enumconc]{leftmargin=0.5cm, label*= \arabic*.  , topsep=1ex, parsep=0pt,  itemsep=3pt }


\newlist{enumsub}{enumerate}{3}
\setlist[enumsub]{ leftmargin=0.7cm, label*= \textbf{subcase} \bf \arabic*: }

\newlist{enumsubsub}{enumerate}{3}
\setlist[enumsubsub]{ leftmargin=0.5cm, label*= \textbf{subsubcase} \bf \arabic*: }

\newlist{mainitem}{itemize}{3}
\setlist[mainitem]{ leftmargin=0cm , label= {\bf Case} }

%%%%COLORS
\definecolor{periwinkle}{rgb}{0.8, 0.8, 1.0}
\definecolor{powderblue}{rgb}{0.69, 0.88, 0.9}
\definecolor{sandstorm}{rgb}{0.93, 0.84, 0.25}
\definecolor{trueblue}{rgb}{0.0, 0.45, 0.81}


\usepackage{array}

\newlength\Origarrayrulewidth

% horizontal rule equivalent to \cline but with 2pt width
\newcommand{\Cline}[1]{%
 \noalign{\global\setlength\Origarrayrulewidth{\arrayrulewidth}}%
 \noalign{\global\setlength\arrayrulewidth{2pt}}\cline{#1}%
 \noalign{\global\setlength\arrayrulewidth{\Origarrayrulewidth}}%
}

% draw a vertical rule of width 2pt on both sides of a cell
\newcommand\Thickvrule[1]{%
  \multicolumn{1}{!{\vrule width 2pt}c!{\vrule width 2pt}}{#1}%
}

% draw a vertical rule of width 2pt on the left side of a cell
\newcommand\Thickvrulel[1]{%
  \multicolumn{1}{!{\vrule width 2pt}c|}{#1}%
}

% draw a vertical rule of width 2pt on the right side of a cell
\newcommand\Thickvruler[1]{%
  \multicolumn{1}{|c!{\vrule width 2pt}}{#1}%
}

\newcommand{\command}{c}
\newcommand{\green}[1]{{ \color{green} #1 } }

\newcommand{\func}[2]{\mathsf{AD}(#1) \to (#2)}
\newcommand{\varEst}{\bf{VetxEst}}
\newcommand{\graphGen}{\bf{GraphGen}}

\newcommand{\ag}[2]{\mathsf{VetxEst}{(#1)}\to {(#2)}}
\newcommand{\ad}[2]{\mathsf{GraphGen}{(#1)}\to {(#2)}}
\newcommand{\rb}{\mathsf{RechBound}}
\newcommand{\pathsearch}{\mathsf{AdaptPathSearch}}

\newcommand{\mg}[1]{\textcolor[rgb]{.90,0.00,0.00}{[MG: #1]}}
\newcommand{\dg}[1]{\textcolor[rgb]{0.00,0.5,0.5}{[DG: #1]}}
\newcommand{\wq}[1]{\textcolor[rgb]{.50,0.0,0.7}{ #1}}
\newcommand{\jl}[1]{\textcolor[rgb]{0.0, 0.5, 0.0}{[#1]}}

\let\originalleft\left
\let\originalright\right
\renewcommand{\left}{\mathopen{}\mathclose\bgroup\originalleft}
\renewcommand{\right}{\aftergroup\egroup\originalright}

\theoremstyle{definition}
\newtheorem{thm}{Theorem}
\newtheorem{lem}[thm]{Lemma}
\newtheorem{cor}[thm]{Corollary}
\newtheorem{prop}[thm]{Proposition}
\newtheorem{defn}[thm]{Definition}

\title{Adaptivity analysis}

\author{}

\date{}

\begin{document}

\maketitle

% \begin{abstract}
% An adaptive data analysis is based on multiple queries over a data set, in which some queries rely on the results of some other queries. The error of each query is usually controllable and bound independently, but the error can propagate through the chain of different queries and bring to high generalization error. To address this issue, data analysts are adopting different mechanisms in their algorithms, such as Gaussian mechanism, etc. To utilize these mechanisms in the best way one needs to understand the depth of chain of queries that one can generate in a data analysis. In this work, we define a programming language which can provide, through its type system, an upper bound on the adaptivity  depth (the length of the longest chain of queries) of a program implementing an adaptive data analysis. We show how this language can help to analyze the generalization error of two data analyses with different adaptivity structures.
% \end{abstract}


% \section{Everything Else}

% \paragraph{Adaptivity}
% Adaptivity is a measure of the nesting depth of a mechanism. To
% represent this depth, we use extended natural numbers. Define $\natb =
% \nat \cup \{\bot\}$, where $\bot$ is a special symbol and $\natbi =
% \natb \cup \{\infty\}$. We use $\nnatA, \nnatB$ to range over $\nat$,
% $\nnatbA, \nnatbB$ to range over $\natb$, and $\nnatbiA, \nnatbiB$ to
% range over $\natbi$.

% The functions $\max$ and $+$, and the order $\leq$ on natural numbers
% extend to $\natbi$ in the natural way:
% \[\begin{array}{lcl}
% \max(\bot, \nnatbiA) & = & \nnatbiA \\
% \max(\nnatbiA, \bot) & = & \nnatbiA \\
% \max(\infty, \nnatbiA) & = & \infty \\
% \max(\nnatbiA, \infty) & = & \infty \\
% \\
% %
% \bot + \nnatbiA & = & \bot \\
% \nnatbiA + \bot & = & \bot \\
% \infty + \nnatbiA & = & \infty ~~~~ \mbox{if } \nnatbiA \neq \bot \\
% \nnatbiA + \infty & = & \infty ~~~~ \mbox{if } \nnatbiA \neq \bot \\
% \\
% %
% \bot \leq \nnatbiA \\
% \nnatbiA \leq \infty
% \end{array}
% \]
% One can think of $\bot$ as $-\infty$, with the special proviso that,
% here, $-\infty + \infty$ is specifically defined to be $-\infty$.

% \paragraph{Language}
% Expressions are shown below. $\econst$ denotes constants (of some base
% type $\tbase$, which may, for example, be reals or rational
% numbers). $\eop$ represents a primitive operation (such as a
% mechanism), which determines adaptivity. For simplicity, we assume
% that $\eop$ can only have type $\tbase \to \tbool$. We make
% environments explicit in closures. This is needed for the tracing
% semantics later.
\[\begin{array}{llll}
\mbox{Expr.} & \expr & ::= & x ~|~ \expr_1 \eapp \expr_2 
 ~|~ \lambda x. \expr 
    \\
             & & &  \etrue ~|~ \efalse ~|~
  \eif  \expr \ethen \expr_2 \eelse \expr_3 ~|~
\econst ~|~ \eop(\expr)  % ~|~ \wq {\eilam \expr ~|~ \expr \eapp [] }
    \\
% & & & ~|~ \wq {\elet  x = \expr_1 \ein \expr_2 } ~|~ \enil ~|~  \econs (
%       \expr_1, \expr_2) \\
% & & & ~|~ \wq{ ~~~~~~~
%  \bernoulli \eapp \expr ~|~ \uniform \eapp \expr_1 \eapp
%       \expr_2 } ~|~  \wq{ \evec({\attr_i \to \expr_i'}^{ i \in 1\dots n})    }  \\
%
\mbox{Value} & \valr & ::= & \econst ~|~ \lambda x. \expr
% (\efix f(x:\type).\expr, \env) ~|~ (\valr_1, \valr_2) 
%     ~|~ \enil ~|~ \econs (\valr_1, \valr_2) |
    \\
% & & & \wq {(\eilam \expr , \env) } ~|~  \wq{ \evec({\attr_i \to \valr_i'}^{ i \in 1\dots n})    } \ \\ 
%
 % \mbox{Adaptivity} & \adapt& ::= & n\\
\mbox{Environment} & \env & ::= & x_1 \mapsto (\valr_1, \adapt_1), \ldots, x_n \mapsto (\valr_n,\adapt_n)
\end{array}\]





%%%%%%%%%%%%%%%%%%%%%%%%%%%%%%%%%%%%%%%%%%%%%%%%%%%% sementics

%%%%%%%%%%%%%%%%%%%%%%%%%%%%%%%%%%%%%%%%%%%%%%%%%%%%% 
\mg{I renamed var2 into var.}
\begin{figure}
  \begin{mathpar}
    \inferrule{
    }{
     \valr, \env \bigstep{0} \valr, \env} ~\textsf{val}
   %  \and
   % \inferrule{  \mathsf{fetch} (\env,x)  =  (\valr, \adapt)  }{x, \env
   %   \bigstep{\adapt} \valr, \env }~\textsf{var}
   \and
   %
     \inferrule{  \env(x)  =  (\valr, \env_1,  \adapt)  }{x,
       \env  \bigstep{\adapt} \valr, \env_1 }~\textsf{var}
     %
  %
  % \and
  % %
  % \inferrule{ }{\env, \etrue \bigstep{0} \etrue}
  % %
  % \and
  % %
  % \inferrule{ }{\env, \efalse \bigstep{0} \efalse}
  % %
 %  \and
 % \inferrule{  \env, \expr \bigstep{K} \econst }{\env, \bernoulli \eapp \expr \bigstep{K} \econst
 %    }~\textsf{bernoulli} 
 %  \and
 % \inferrule{ \env, \expr_1 \bigstep{R} \econst \\ \env, \expr_2 \bigstep{S} \econst  }{\env, \uniform \eapp \expr_1 \eapp
 %      \expr_2\bigstep{R+S} \econst  } ~\textsf{uniform}
 %  \and
 %
   \and
  %
   \inferrule{ }{\econst , \env \bigstep{0} \econst, \env}~\textsf{const}
   %
   \and
   %
 \inferrule{
  }{
    \lambda x. \expr, \env
    \bigstep{0} \lambda x.\expr, \env
  }~\textsf{lambda}
  %
  \and
  %
  \inferrule{
    \expr_1, \env_1 \bigstep{\adapt_1} \lambda x.\expr , \env_1' \\
    %\forall x_i \in \dom(\env_1 \cap \env_2).  \fresh \eapp x_i' \\
     \expr_2, \env_2 \bigstep{\adapt_2} \valr_2 , \env_2' \\
    \fresh \eapp x' \\
    \expr[x'/x], \env_1'[ x'  \to (\valr_2, \env_2', \adapt_2  ) ] 
    \bigstep{\adapt_3} \valr, \env_3
  }{
     \expr_1 \eapp \expr_2 , (\env_1 \uplus \env_2)\bigstep{\adapt_1+\adapt_3} \valr, \env_3
  }~\textsf{app}
 %
  \and
  % %
  % \wq{
  % \inferrule{
  %   \env, \expr_1 \bigstep{R} \valr_1 \\
  %   \env, \expr_2 \bigstep{S} \valr_2  }
  % {
  %   \env, (\expr_1, \expr_2) \bigstep{(R,S)} (\valr_1, \valr_2)
  % }~\textsf{pair}
  % }
  % %
  % \and
  % %
  % \wq{
  % \inferrule{
  %   \env, \expr \bigstep{(R_1,R_2)} (\valr_1, \valr_2)
  % }{
  %   \env, \eprojl(\expr) \bigstep{R_1} \valr_1
  % }~\textsf{fst}
  % }
  % %
  % \and
  % %
  % \inferrule{
  %   \env, \expr \bigstep{(R_1,R_2)} (\valr_1, \valr_2), \tr
  % }{
  %   \env, \eprojr(\expr) \bigstep{(R_2)} \valr_2, \trprojr(\tr)
  % }~\textsf{snd}
  % %
  % \and
  % %
  % \inferrule{
  %   \env, \expr \bigstep{R} \etrue\\
  %   \env, \expr_1 \bigstep{S} \valr, \tr_1
  % }{
  %   \env, \eif(\expr, \expr_1, \expr_2) \bigstep{R+S} \valr
  % }~\textsf{if-true}
  % %
  % \and
  % %
  % \inferrule{
  %   \env, \expr \bigstep{R} \efalse \\
  %   \env, \expr_2 \bigstep{S} \valr
  % }{
  %   \env, \eif(\expr, \expr_1, \expr_2) \bigstep{R+S } \valr
  % }~\textsf{if-false}
  % %
  \and
  %
  \inferrule{
    \expr , \env \bigstep{\adapt} \valr , \env_1 \\
    \eop{}(\valr\env) = \valr' \\
    FV(\valr')=\emptyset
  }{
    \eop(\expr), \env \bigstep{\adapt +1} \valr,  \env_1
  }~\textsf{delta}
  %
   % \and
% %
%   \inferrule{
% }
% { \env, \enil \bigstep{0} \enil, \trnil }~\textsf{nil}
% %
% \and
% %
% \inferrule{
% \env, \expr_1 \bigstep{R} \valr_1 \\
% \env, \expr_2 \bigstep{S} \valr_2
% }
% { \env, \econs (\expr_1, \expr_2)  \bigstep{  \max(R,S)} \econs (\valr_1, \valr_2)
% }~\textsf{cons}
% %
% \and
% %
% \inferrule{
%   \env, \expr_1 \bigstep{R} \valr_1 \\
%   \env[x \mapsto (\valr_1, R  )] , \expr_2 \bigstep{S} \valr
% }
% {\env, \elet x = \expr_1 \ein \expr_2 \bigstep{S} \valr }~\textsf{let}
% %
% \\\\
% %
% \inferrule
% {
%   \env, \expr \bigstep{R} \valr
% }
% {
%   \env, \eilam \expr \bigstep{0} \eilam \valr,
% }~\textsf{eilam}
% %
% \and
% %
% \inferrule{
%   \env, \expr \bigstep{K} (\eilam \expr') \\
%   \env, \expr' \bigstep{S} \valr
% }
% {\env, \expr [] \bigstep{K+S} \valr, \triapp{\tr_1}{\tr_2}
% }~\textsf{eiapp1}
% %
% \and
% %
% \inferrule{
%   \env, \expr \bigstep{K} \valr \not\equiv (\eilam \expr') \\
% }
% {\env, \expr [] \bigstep{K} \valr []
% }~\textsf{eiapp2}
% %
\and
 %
\wq{
 \inferrule{
   \expr, \env \bigstep{\adapt} \efalse, \env'
   \\
   \expr_2 , \env \bigstep{\adapt_2} \valr_2, \env_2
  }{
    \eif \expr \ethen \expr_1 \eelse \expr_2 , \env \bigstep{\adapt +
      \adapt_2 } \valr_2, \env_2
  }~\textsf{if-f}
}
\and
\wq{
 \inferrule{
   \expr , \env \bigstep{\adapt} \etrue, \env'
   \\
    \expr_1, \env \bigstep{\adapt_1} \valr_1, \env_1
  }{
    \eif \expr \ethen \expr_1 \eelse \expr_2 , \env \bigstep{\adapt +
      \adapt_1 } \valr_1, \env_1
  }~\textsf{if-t}
}
% %
% \and
% %
% \wq{
%  \inferrule{
%     \env, \expr \bigstep{R} \valr
%   }{
%     \env, \eprojl(\expr) \bigstep{R} \eprojl(\valr)
%   }~\textsf{fst1}
% }
% %
% \and
% %
% \wq{
%  \inferrule{
%     \env, \expr \bigstep{R} \valr
%   }{
%     \env, \eprojr(\expr) \bigstep{R} \eprojr(\valr)
%   }~\textsf{snd1}
%   }
  \\\\
  \begin{array}{llll}
    \env_1 \uplus \emptyset & \triangleq & \env_1 &\\
     \emptyset \uplus \env_2 & \triangleq & \env_2 &\\
    % (\env_1,[x \to (\valr, \adapt_1)] )\uplus (\env_2, [x \to (\valr,
    % \adapt_2)] )  &  \triangleq & (\env_1 \uplus \env_2),[x \to
    %                               (\valr, \max(\adapt_1, \adapt_2))] & \\       
    
  \end{array}
  % \\\\
  % \begin{array}{lllll}
  %   \mathsf{fetch} (\env, x) & ::=  &  \mathsf{Some} ( \Pi_1
  %                                     (\env(x)), \Pi_3(\env(x)) ) &   &  x \in  \dom(\env) \\
  %           &  & \mathsf{match} \eapp \env  \eapp \mathsf{with} & & x \not\in  \dom(\env) \\
  %            &    &     | (y \to (\valr', \env', R)) :: \env_t \to & 
  %                   \mathsf{match} \eapp \mathsf{fetch} (\env', x) \eapp
  %                   \mathsf{with} & \\                             
  %             &  &   & \mathsf{None}  \to \mathsf{fetch}(\env_t, x ) &
  %   \\
  %            &   &   & \mathsf{Some} (\valr, r) \to \mathsf{Some} (\valr,
  %                  r)  & \\ 
  %            & & | [ ] \to  \mathsf{None} & &                     
  % \end{array}    
\end{mathpar}
  \caption{Big-step semantics}
  \label{fig:semantics1}
\end{figure}



\begin{figure}
  \begin{mathpar}
    \inferrule{
    }{
     \valr, \env \bigstep{0} \valr, \env} ~\textsf{val}
   %  \and
   % \inferrule{  \mathsf{fetch} (\env,x)  =  (\valr, \adapt)  }{x, \env
   %   \bigstep{\adapt} \valr, \env }~\textsf{var}
   \and
   %
     \inferrule{  \env(x)  =  (\valr, \env_1,  \adapt)  }{x,
       \env  \bigstep{\adapt} \valr, \env_1 }~\textsf{var}
     %
   \and
  %
   \inferrule{ }{\econst , \env \bigstep{0} \econst, \env}~\textsf{const}
   %
   \and
   %
 \inferrule{
  }{
    \lambda x. \expr, \env
    \bigstep{0} \lambda x.\expr, \env
  }~\textsf{lambda}
  %
  \and
  %
  \inferrule{
    \expr_1, \env_1 \bigstep{\adapt_1} \lambda x.\expr , \env_1' \\
    %\forall x_i \in \dom(\env_1 \cap \env_2).  \fresh \eapp x_i' \\
     \expr_2, \env_2 \bigstep{\adapt_2} \valr_2 , \env_2' \\
    \fresh \eapp x' \\
    \expr[x'/x], \env_1'[ x'  \to (\valr_2, \env_2', \adapt_2  ) ] 
    \bigstep{\adapt_3} \valr, \env_3
  }{
     \expr_1 \eapp \expr_2 , (\env_1 \uplus \env_2)\bigstep{\jl{\max(\adapt_1, \adapt_3)}} \valr, \env_3
  }~\textsf{app}
 %
  \and
  %
 \jl{
  \inferrule{
     \expr , \env \bigstep{\adapt} \valr , \env_1 
   }{
     \eop(\expr), \env \bigstep{\adapt +1} \eop(\valr),  \env_1
   }~\textsf{delta}
 }%
\and
 %
 \inferrule{
   \expr, \env \bigstep{\adapt} \efalse, \env'
   \\
   \expr_2 , \env \bigstep{\adapt_2} \valr_2, \env_2
  }{
    \eif \expr \ethen \expr_1 \eelse \expr_2 , \env \bigstep{\adapt +
      \adapt_2 } \valr_2, \env_2
  }~\textsf{if-f}
%
\and
%
\inferrule{
   \expr , \env \bigstep{\adapt} \etrue, \env'
   \\
    \expr_1, \env \bigstep{\adapt_1} \valr_1, \env_1
  }{
    \eif \expr \ethen \expr_1 \eelse \expr_2 , \env \bigstep{\adapt +
      \adapt_1 } \valr_1, \env_1
  }~\textsf{if-t}
  %
  \and
  %
\jl{  
\inferrule{
    \expr, \env \bigstep{\adapt} (\expr_1, \expr_2) , \env_1 \\
    %\forall x_i \in \dom(\env_1 \cap \env_2).  \fresh \eapp x_i' \\
     \expr_1, \env_1 \bigstep{\adapt_1} \valr_1 , \env_1' \\
     \expr_2, \env_1 \bigstep{\adapt_2} \valr_2 , \env_2' \\
    \fresh \eapp x', y' \\
    \expr'[x'/x][y'/y], 
    \env_1[ x'  \to (\valr_1, \env_1', \adapt_1), 
    y' \to (\valr_2, \env_2', \adapt_2) ] 
    \bigstep{\adapt'} \valr, \env'
  }{
     \elet \eapp (x, y) = \expr \ein \expr' , \env \bigstep{{\max(\adapt, \adapt')}} \valr, \env'
  }~\textsf{bind}
  }
  \\\\
  \begin{array}{llll}
    \env_1 \uplus \emptyset & \triangleq & \env_1 &\\
     \emptyset \uplus \env_2 & \triangleq & \env_2 &\\
  \end{array}
\end{mathpar}
  \caption{Big-step semantics - Jan.28}
  \label{fig:semantics1}
\end{figure}



%%%%%%%%%%%%%%%%%%%%%%%%%%%%%%%%%%%%%%%%%%%%%%%%%%%%%

%%%%%%%%%%%%%%%%%%%%%%%%%%%%%%%%%%%%%%%%%%%%%%%%%%%%%


\[
\begin{array}{llll}
  \mbox{Index Term} & \idx, \nnatA & ::= &     i ~|~ n \\
 %                                  - \idx_2 ~|~ \smax{\idx_1}{\idx_2}\\
%                                  \mbox{Sort} & S & ::= & \nat \\
  \mbox{Linear type} & \type &::=  &  \ltype \lto{\nnatA} \type ~|~
                                     \tbase ~|~ \tbool \\
  \mbox{Nonlinear Type} & \ltype & ::= & \bang{\idx} \type   \\
  \mbox{Typing context } & \Gamma & ::= & x_1 : \ltype_1, \ldots,
                                          x_n : \ltype_n
\end{array}
\]

\begin{figure}
  \begin{mathpar}
    \inferrule{
    }{
      \ictx \tctx , x: \bang{1} \type \tvdash{0} x: \type
    }~\textbf{Ax}
    %
    \and
    %
    \inferrule{
    }{
      \ictx \Gamma \tvdash{0} c : \tbase 
    }~\textbf{const}
    %
    % \and
    % %
    % \inferrule{
    % }{
    %   \ictx \Gamma \tvdash{\nnatA} \evec : \bang{\nnatA}\tbase 
    % }~\textbf{Dict}
    %
    \and
    %
    \inferrule{
      \ictx \Gamma, x: \ltype
      \tvdash{\nnatA }
      \expr: \type
    }{
      \ictx \Gamma \tvdash{0} \lambda x. \expr : \ltype
      \lto{\nnatA} \type
    }~\textbf{lambda}
    \and
    %
    \inferrule{
      \ictx \Gamma_1  \tvdash{\nnatA_1} \expr_1:  \bang{\idx} \type_1
      \lto{\nnatA} \type_2      \\
      \ictx \Gamma_2 \tvdash{\nnatA_2} \expr_2: \type_1 
    }{
      \ictx   \Gamma_1 + \idx \times \Gamma_2  \tvdash{    \nnatA_1 +
        \idx \times \nnatA_2 + \nnatA    } \expr_1 \eapp \expr_2 : \type_2
    }~\textbf{app}
    %
    \and
    %
    \inferrule{
      \ictx \Gamma \tvdash{\nnatA} \expr:  \tbase 
    }{
      \ictx \Gamma  \tvdash{1+\nnatA} \delta(\expr): \tbase
    }~\textbf{delta}
     %
    \and
    %
    \inferrule{
      \ictx \Gamma'  \tvdash{\nnatA'} \expr: \type' \\
      \Gamma' \leqslant \Gamma \\
      \nnatA' \leq \nnatA\\
      \sub{\type'}{\type} 
    }{
      \ictx \Gamma  \tvdash{\nnatA} \expr: \type 
    }~\textbf{subtype}
      %
    \and
    %
    \inferrule{
      \ictx \Gamma, y: \type', x: \type ,\Gamma'  \tvdash{\nnatA} \expr: \type 
    }{
      \ictx \Gamma, x: \type, y: \type' ,\Gamma'  \tvdash{\nnatA} \expr: \type 
    }~\textbf{exchange}
    %
    \and
    %
    \wq{
    \inferrule{
      \ictx \Gamma \tvdash{\nnatA} \expr: \tbool
      \\
      \ictx \Gamma_1 \tvdash{\nnatA_1} \expr_1: \type
      \\
      \ictx \Gamma_2 \tvdash{\nnatA_2} \expr_2: \type
      \\
      \Gamma' = \Gamma + \Gamma_1 + \Gamma_2
      \\
      \nnatA' = \nnatA + \max(\nnatA_1, \nnatA_2)
    }{
      \ictx \Gamma' \tvdash{\nnatA'}  \eif \expr  \ethen \expr_1 \eelse \expr_2 : \type 
    }~\textbf{if}
    }
     \\\\
\begin{array}{llll}
  \idx \times \Gamma &\triangleq  &  \Gamma  & \idx =1  \\
                     &\triangleq  &  \bang{0} \Gamma & \idx =0 \\
  \bang{\idx_1} \type + \bang{\idx_2} \type  &\triangleq  & \bang{
                                                          \max(\idx_1,\idx_2)
                                                          } \type &  \\
  \Gamma + \emptyset & \triangleq & \Gamma & \\
  \emptyset+ \Gamma  & \triangleq & \Gamma & \\
  ( [x : \ltype ],\Gamma) +  ([x: \ltype'],\Delta)  & \triangleq
                            & [x: \ltype + \ltype' ], \Gamma +
                              \Delta &   \\
   \sub{\Gamma}{\Delta} & \triangleq &  \dom(\Gamma) = \dom(\Delta) & \\
    & &                                    \land \forall x \in
                                      \dom(\Gamma),
        \sub{\Delta(x)}{\Gamma(x)} &  
\end{array}
  \end{mathpar}
  \caption{Typing rules, first version}
  \label{fig:type-rules1}
\end{figure}

\begin{figure}
  \begin{mathpar}
    \inferrule{
      \idx_1 \leq \idx \\
      \sub{\ltype}{\ltype_1}
    }{
      \sub{\bang{\idx} \ltype}{\bang{\idx_1} \ltype_1}
    }~\textsf{bang}
    %
    \and
    %
    \inferrule{
        \nnatA \leq \nnatA' \\
        \sub{\type_1}{\type}   \\
      \sub{\type'}{\type_1'}
    }{
      \sub{\type \lto{\nnatA} \type' }{\type_1 \lto{\nnatA'} \ltype_1'}
    }~\textsf{arrow}
    %
    \and
    %
    \inferrule{
    }{
    \sub{\tbase}{\tbase}
    }~\textsf{base}
  \end{mathpar}
  \caption{subtyping}
 \end{figure}

 \clearpage

 \begin{figure}
  \begin{mathpar}
    \inferrule{
     \env ( x ) = (\valr, \env', \adapt)
      \\
      \tvdash{\nnatA} ( \valr, \env') : \type
          }{
     \tvdash{\adapt + \nnatA}   ( x, \env):  \type
    }~\textbf{C-Ax}
    %
    \and
    %
    \inferrule{
    }{
      \tvdash{0} (  c, \env) : \tbase
    }~\textbf{C-const}
   
    \and
    %
    \inferrule{
      \tvdash{\nnatA' } ( \valr', \theta') : \type_1
      \\
      \fresh\eapp  x' ~~ \forall \adapt'
      \\
      \tvdash{ S+ \idx \times (\adapt' + \nnatA' ) +\nnatA }
     ( \expr[x'/x], \env[x' \to (\valr', \theta', R')]      ) :
     \type_2
    }{
     \tvdash{S} (  \lambda x. \expr, \env )  : \bang{\idx} \type_1
      \lto{\nnatA} \type_2
    }~\textbf{C-lambda}
    \and
    %
    \inferrule{
       \tvdash{\nnatA_1} ( \expr_1, \env_1) :  \bang{\idx} \type_1
      \lto{\nnatA} \type_2      \\
      \tvdash{\nnatA_2} ( \expr_2, \env_2 ): \type_1
    }{
       \tvdash{    \nnatA_1 +
        \idx \times \nnatA_2 + \nnatA    } (  \expr_1 \eapp \expr_2, \env_1 \uplus \env_2   ) : \type_2
    }~\textbf{C-app}
    %
    \and
    %
    \inferrule{
      \tvdash{\nnatA} (\expr, \env) :  \tbase
   }{  \tvdash{1+\nnatA} (\delta(\expr) , \env ) : \tbase
    }~\textbf{C-delta}
    \\\\
    \begin{array}{lll}
       \theta  & \triangleq (x_i \to (\valr_i, \env_i, R_i)) & i \in
                                                               \mathbb{N}\\
      (x_i : \bang{ \idx }\type_i), \Gamma \vDash (x_i \to (\valr_i, \env_i, R_i))
      \uplus \theta & \triangleq ~~~\tvdash { \_ } (\valr_i, \env_i)
                                          :  \type_i  &\conj
                                   \Gamma \vDash \theta
      \end{array}
  \end{mathpar}
  \caption{Typing rules, configure}
  \label{fig:configure-rules}
\end{figure}

\begin{figure}
  \begin{mathpar}
    \begin{array}{lll}
      \lrv{\tbase} & = & \{  ( \econst, \env,  \nnatA)  \} \\
      %
      % \lrv{\type_1 \times \type_2} & = & \{(\valr_1, \valr_2) ~|~ \valr_1 \in \lrv{\type_1} \conj \valr_2 \in \lrv{\type_2} \}\\
      %
      \lrv{\bang{k} \type } & = & \{  ( \valr, \env,   \nnatA) |  (\valr, \env,
                                   \nnatA ) \in \lrv{\type}  \} \\
      %
      \lrv{ \bang{k} \type_1 \lto{\nnatA} \type_2    } & = &
      \{( \lambda x.\expr, \env,  \nnatA_1) ~|~ \forall \valr', \env',
                                                             \nnatA'. (
                                                             \valr',\env',                                   
                                                             \nnatA') \in
                                                             \lrv{
                                                             \bang{k} \type_1}.\\
      & & 
          \implies   \fresh \eapp x' \land \\
      & & \forall \adapt. ( \expr[x'/x], \env[x' \mapsto (\valr', \env', \adapt )] ) \in
          \lre{    }{ \nnatA_1+\nnatA+ \idx \times (\adapt + \nnatA') }{\type_2}     \} \\
      %
      \\
      %
      \lre{}{\nnatA}{\type} & = & \{  ( \expr, \env) ~|~  ( \expr , \env
                                  \bigstep{\adapt}  \valr, \env' ) \\
      & & ~~~~~~~~~~~~~\implies \adapt \leq \nnatA \conj 
     ( \valr, \env', \nnatA- \adapt) \in \lrv{\type})
      \}
    \end{array}
  \end{mathpar}
  \caption{Logical relation without step-indexing}
  \label{fig:lr:non-step}
\end{figure}


\clearpage

 \begin{thm}[Monotonicity]
  \label{mono}
  \begin{enumerate} 
   \item If  $(
     \expr, \env) \in  \lre{}{\nnatA}{ \type} $ and $\nnatA' \geq \nnatA$,  then  $  (
     \expr, \env) \in  \lre{}{\nnatA'}{ \type} $.
   \item   If  $(
     \valr,\env,  \nnatA) \in  \lrv{\type} $ and $\nnatA' \geq \nnatA$,  then  $ (
     \valr,\env, \nnatA') \in  \lrv{\type} $.
     \item If $ \tvdash{\nnatA} (\expr, \env) : \type$ and $\nnatA
       \leq \nnatA'$, then $\tvdash{\nnatA'} (\expr, \env) : \type $.
     \item If $\Gamma \tvdash{\nnatA} \expr: \type$ and $\nnatA
       \leq \nnatA'$,   then  $\Gamma \tvdash{\nnatA'} \expr: \type$.
  \end{enumerate}
\end{thm}

\clearpage
\[
  \begin{array}{ll}
    F_{c2t} (\expr, x_i) =   1 & x_i \in \fv{\expr} \\
        ~~~~~~~~~~~~~~~~~ 0 & x_i \not\in \fv{\expr}   
    \end{array}
  \]
  
\begin{thm}[ConfigurationToTyping]
  \label{c2t}
  \begin{enumerate} 
   \item If $ \tvdash{\nnatA}  ( \expr, \env) : \type $ and $\forall
     x_i \in \dom(\env). \env(x_i) = (\valr_i, \env_i, \adapt_i) \land
     \exists \nnatA_i . \tvdash{\nnatA_i} (\valr_i, \env_i) : \type_i
     $ ,  then $ x_i : \bang{ F_{c2t}(\expr, x_i) } \type_i \tvdash{
       \nnatA - F_{c2t}(\expr, x_i) \times (\adapt_i + \nnatA_i) } \expr: \type $.
  \end{enumerate}
\end{thm}
 %\begin{proof}
  By induction on the configuration derivation.\\
   
 \caseL{ Case$     \inferrule{
     \env ( x ) = (\valr, \env', R) ~ (\star)
      \\
      \tvdash{0} ( \valr, \env') : \type ~(\diamond)
    }{
     \tvdash{R}   ( x, \env):  \type
    }~\textbf{C-Ax}
    $
  }

  Asssume $\Gamma \vDash \env$,  TS: $\Gamma \tvdash{  \max(0, \nnatA-
    F_{c2t}(\env) ) } x: \type $.
  
  From $(\star), (\diamond)$ and the definition of $\Gamma \vDash
  \env$, we know $\Gamma(x) = \bang{1} \type$. In this sense, $
  F_{c2t}(\env)  \geq R $ so that $\max(0, \nnatA- F_{c2t}(\env) ) =0$.
  
 It is proved by typing rule Ax. \\

  

  

    \caseL{ Case
    $
       \inferrule{
      \tvdash{\_} ( \valr', \theta') : \type_1 ~(\star)
      \\
      \fresh\eapp  x'
      \\
      \tvdash{ S+  \idx \times R' +\nnatA }
     ( \expr[x'/x], \env[x' \to (\valr', \theta', R')]      ) :
     \type_2~(\diamond)
    }{
     \tvdash{S} (  \lambda x. \expr, \env )  : \bang{\idx} \type_1
      \lto{\nnatA} \type_2
    }~\textbf{C-lambda}
    $
  }

  Asssume $\Gamma \vDash \env$,  TS: $\Gamma \tvdash{ \max(0,  S-
    F_{c2t}(\env) ) } \lambda
  x. \expr: \bang{\idx} \type_1 \lto{\nnatA} \type_2 $.

  By induction hypothesis on $(\star)$, we know : $ \tvdash{ \_ }
  (\valr',\env') : \type_1 $.

  From the above statement, we conclude : $ \Gamma, x': \bang{\idx} \type_1
  \vDash ( \env[x' \to (\valr', \theta', R')] )~(1) $.

  By induction hypothesis on $(1)$, we have : $ \ictx \Gamma, x': \bang{\idx} \type_1
      \tvdash{\nnatA }
      \expr[x'/x]: \type_2 ~(a) $. 
  
  \[
 \inferrule{
      \ictx \Gamma, x': \bang{\idx} \type_1
      \tvdash{\nnatA }
      \expr[x'/x]: \type
    }{
      \ictx \Gamma \tvdash{0} \lambda x'. \expr[x'/x] : \bang{\idx} \type_1
      \lto{\nnatA} \type
    }~\textbf{lambda}
  \]


  \caseL{Case
    $
        \inferrule{
       \tvdash{\nnatA_1} ( \expr_1, \env_1) :  \bang{\idx} \type_1
      \lto{\nnatA} \type_2      \\
      \tvdash{\nnatA_2} ( \expr_2, \env_2 ): \type_1
    }{
       \tvdash{    \nnatA_1 +
        \idx \times \nnatA_2 + \nnatA    } (  \expr_1 \eapp \expr_2, \env_1 \uplus \env_2   ) : \type_2
    }~\textbf{C-app}
    $
  }

  \[
   \inferrule{
      \ictx \Gamma_1  \tvdash{\nnatA_1} \expr_1:  \bang{\idx} \type_1
      \lto{\nnatA} \type_2      \\
      \ictx \Gamma_2 \tvdash{\nnatA_2} \expr_2: \type_1 
    }{
      \ictx   \Gamma_1 + \idx \times \Gamma_2  \tvdash{    \nnatA_1 +
        \idx \times \nnatA_2 + \nnatA    } \expr_1 \eapp \expr_2 : \type_2
    }~\textbf{app}
  \]
  

\end{proof}


 \clearpage

 \[
\begin{array}{ll}
 F(\expr, \phi ) & where \eapp ~~ \phi(x_i) = (\idx_i, \adapt_i, \nnatA_i ) \\
   F(x,\phi) & = \sum_{x_i \in \fv{x}  } \idx_i \times (\adapt_i+ \nnatA_i)  \\
F(\lambda x. \expr ,  \phi  ) & = \sum_{x_i \in \fv{\lambda x.\expr}  } \idx_i \times (\adapt_i+ \nnatA_i)   \\ 
F(\delta(\expr) , \phi ) & = \sum_{x_i \in \fv{\delta(\expr)} } \idx_i \times (\adapt_i+ \nnatA_i)  \\
F(c, \phi ) & = 0  \\
F(\expr_1 \eapp \expr_2, \phi ) & = F(\expr_1, \phi ) +
                                        F(\expr_2,\phi )
\end{array} 
\]

\begin{thm}[TypingtoConfiguration]
  \label{t2c}
  \begin{enumerate} 
   \item If $ x_1: \bang{\idx_1} \type_1, \ldots ,  x_i : \bang{\idx_i} \type_i
     \tvdash{\nnatA} \expr: \type$,  and $ \tvdash{\nnatA_I } (\valr_i, \env_i
    ) : \type_i    $,  and $\env = [ x_1
     \to (\valr_1, \env_1, R_1), \ldots,  x_i \to (\valr_i, \env_i,
     R_i)   ]$, $ \phi = [x_1
     \to (\idx_1, \adapt_1, \nnatA_1), \ldots,  x_i \to (\idx_i, \adapt_i,
     \nnatA_i)  ] $ , then   $
     \tvdash{\nnatA +  F( \expr, \phi )  }
     ( \expr, \env ) : \type $ .
  \end{enumerate}
\end{thm}
\begin{proof}
  By induction on the typing derivation.\\

  \caseL{
    Case $
      \inferrule{
    }{
      \ictx \tctx , x: \bang{1} \type \tvdash{0} x: \type
    }~\textbf{Ax}
    $
  }

  assume $ \tvdash{\nnatA} (\valr, \env') :  \type$ and $ \tvdash{\nnatA_i} (\valr_i,
  \env_i) :  \type_i  $,
 

  assume $\env =  [x \to (\valr, \env', R)] \uplus [ x_1 \to (\valr_1,
  \env_1, R_1), \ldots,   x_i \to (\valr_i,
  \env_i, R_i)]   $.

  So $\phi = [x \to (1, \adapt, \nnatA) ] \uplus [ x_1 \to (\idx_1,
  \adapt_1, \nnatA_1), \ldots,   x_i \to (\idx_i,  \adapt_i, \nnatA_i)] $

  $F\big(x, \phi \big) = 1 \times (\adapt + \nnatA)  $.

  TS:$\tvdash{0 +1 \times (\adapt+ \nnatA ) } ( x,   \env  ) : \type $.

   
  
  We conclude from the configuratio rule C-Ax.
  
  \[ \inferrule{
     \env ( x ) = (\valr, \env', R)
      \\
      \tvdash{\nnatA} ( \valr, \env') : \type
    }{
     \tvdash{\adapt+\nnatA}   ( x, \env):  \type
    }~\textbf{C-Ax}
  \]

  \caseL{
    Case $
       \inferrule{
      \ictx \Gamma, x: \ltype
      \tvdash{\nnatA }
      \expr: \type
    }{
      \ictx \Gamma \tvdash{0} \lambda x. \expr : \ltype
      \lto{\nnatA} \type
    }~\textbf{lambda}
    $
  }

  let $\Gamma =x_1: \bang{k_1} \type_1 , \ldots,  x_i : \bang{k_i} \type_i$ and  $\ltype = \bang{k} \type_1$.

  Assume $\tvdash{\nnatA'}  (\valr', \env')  :  \type_1 ~(1)$ and $
  \tvdash{\nnatA_i} (\valr_i, \env_i) :  \type_i  $.

Assume  $ \env =  [x_1 \to (\valr_1, \env_1, \adapt_1), \ldots,   x_i \to (\valr_i, \env_i, \adapt_i)]  $.

So $\phi =  [ x_1 \to (\idx_1,
  \adapt_1, \nnatA_1), \ldots,   x_i \to (\idx_i,  \adapt_i, \nnatA_i)] $

  TS: $ \tvdash{0 + F(\lambda x. \expr,  \phi )  }   (\lambda
  x.\expr,  \env) :  \bang{\idx} \type_1  \lto{\nnatA} \type_2 $. 

  Let $S = \sum_{x_i \in \fv{\lambda x. \expr} } \idx_i \times
  (\adapt_i + \nnatA_i)    $ .

  From assumption $(1)$, we know :  $\tvdash{\nnatA' } ( \valr', \env) : \type_1~(\star)$.

  Take a fresh variable x', doing alpha renaming on the premise, pick
  $\adapt'$ so that $\env' = [x' \to (\valr', \env', \adapt' )] \uplus
  \env$ and $\phi'= [x' \to (\idx, \adapt', \nnatA')] \uplus \phi$.
  
  By induction hypothesis on the premise, we know: $\tvdash{ \nnatA+
 F(\expr[x'/x], \phi' ) }  ( \expr[x'/x],  [x' \to (\valr', \env', \adapt')] \uplus \env
) : \type~(\diamond) $.

 $F(\expr[x'/x], \phi' )  = \sum_{x_i \in \fv{\lambda x. \expr} } \idx_i
 \times (\adapt_i +\nnatA_i)  + \idx
 \times (\adapt'+\nnatA')  =  S + \idx \times (\adapt'+ \nnatA' )$.


 We can conclude the following by the configuration rule. 
    
  
  \[
       \inferrule{
      \tvdash{\nnatA'} ( \valr', \theta') : \type_1 ~(\star)
      \\
      \fresh\eapp  x'
      \\
      \tvdash{ S+  \idx \times (\adapt'+\nnatA') +\nnatA }
     ( \expr[x'/x], \env[x' \to (\valr', \env', \adapt')]      ) :
     \type~(\diamond)
    }{
     \tvdash{S} (  \lambda x. \expr, \env )  : \bang{\idx} \type_1
      \lto{\nnatA} \type
    }~\textbf{C-lambda}
  \]


  \caseL{Case
    $
        \inferrule{
      \ictx \Gamma_1  \tvdash{\nnatA_1} \expr_1:  \bang{\idx} \type_1
      \lto{\nnatA} \type_2      \\
      \ictx \Gamma_2 \tvdash{\nnatA_2} \expr_2: \type_1 
    }{
      \ictx   \Gamma_1 + \idx \times \Gamma_2  \tvdash{    \nnatA_1 +
        \idx \times \nnatA_2 + \nnatA    } \expr_1 \eapp \expr_2 : \type_2
    }~\textbf{app}
    $
  }
  
  Let us assume $\Gamma_1 = x_i : \bang{k_i} \type_i $ and $\Gamma_2 = x_i' : \bang{k_i'} \type_i'$,($\Gamma_1$ and $\Gamma_2$ may overleap.).
  
  Forall the variables $x_i''$ in $\dom (\Gamma_1 + k \times \Gamma_2
  )$, we assume $ (\Gamma_1 + k \times
  \Gamma_2 )(x_i'') = \bang{k_i''} \type_i'' $ so that
  $ \tvdash{\nnatA_i''}  (\valr_i'', \env_i'')  : \type_i'' $

  assume $\env = [x_1'' \to (\valr_1'', \env_1'', \adapt_1''), \ldots,
  x_i'' \to (\valr_i'', \env_i'', \adapt_i'')]$.

  So $\phi =  [x_1'' \to (\idx_1'', \adapt_1'', \nnatA_1''), \ldots,
  x_i'' \to (\idx_i'', \adapt_i'', \nnatA_i'')]$

  TS: $ \tvdash{\nnatA_1 + \idx \times \nnatA_2 + \nnatA   +
    F(\expr_1 \eapp \expr_2, \phi ) }   (\expr_1 \eapp \expr_2 , \env)
  : \type_2 $.\\

  let $\env_1 = \{  [x_i'' \to (\valr_i'', \env_i'', R_i'') ]   |     x_i'' \in \dom(\Gamma_1)  \}   $.

  let $\env_2 = \{  [x_i'' \to (\valr_i'', \env_i'', R_i'') ]   |
  x_i'' \in \dom(\Gamma_2)  \}   $.

  We know  $ \dom(\env) = \dom(\env_1) \uplus \dom(\env_2) \implies   \env = \env_1 \uplus \env_2 $

  We set $\phi_1 =\{  [x_i \to (\idx_i'', \adapt_i'', \nnatA_i'')] |
  x_i'' \in \dom(\Gamma_1)  \} $. and $\phi_2 =\{  [x_i \to (\idx_i'', \adapt_i'', \nnatA_i'')] |    x_i'' \in \dom(\Gamma_2)  \} $
  
  By induction hypothesis on the first premise, we have:
  $ \tvdash{ \nnatA_1 + F(\expr_1, \phi_1) } (\expr_1, \env_1) : \bang{\idx} \type_1 \lto{\nnatA} \type_2 ~(\star)$.

  By induction hypothesis on the second premise, we have:
  $ \tvdash{ \nnatA_2 + F(\expr_2, \phi_2 )  } (\expr_2, \env_2) : \type_1 $ .

 From the definition, we know: $ F(\expr_1, \phi_1) = F (\expr_1, \phi) $ and $ F(\expr_2, \phi_2) = F (\expr_2, \phi) $ .
  
  From the configuration rule C-app, we get:

  
  \[
 \inferrule{
       \tvdash{\nnatA_1 + F(\expr_1, \env_1, \Gamma_1) } ( \expr_1, \env_1) :  \bang{\idx} \type_1
      \lto{\nnatA} \type_2    ~(\star)  \\
      \tvdash{\nnatA_2 +  F(\expr_2, \env_2, \Gamma_2)} ( \expr_2, \env_2 ): \type_1 ~(\diamond)
        }{
       \tvdash{    \nnatA_1 + F(\expr_1, \env_1, \Gamma_1) + \idx
         \times (\nnatA_2 +  F(\expr_2, \env_2, \Gamma_2)) + \nnatA
       } (  \expr_1 \eapp \expr_2, \env_1 \uplus \env_2   ) : \type_2 ~(\clubsuit)
    }~\textbf{C-app} 
  \]

  Because $ F(\expr_1,\phi_1)  + \idx \times F(\expr_2,
  \phi_2 ) \leq F(\expr_1 \eapp \expr_2,  \phi) $, so we conclude :$  \nnatA_1 + F(\expr_1,\phi_1) + \idx \times
  (\nnatA_2 +  F(\expr_2, \phi_2)) + \nnatA  \leq \nnatA_1 + \idx
  \times \nnatA_2 + \nnatA   +   F(\expr_1 \eapp \expr_2, \phi )  $ .

  The TS can be shown using Lemma~\ref{mono} on the $\clubsuit$.
  
  
 \end{proof} 


\clearpage

\begin{thm}[ConfigurationSoundness]
  \label{sound}
  \begin{enumerate} 
   \item $ \tvdash{\nnatA}  ( \expr, \env) : \type $, then $(
     \expr, \env) \in  \lre{}{\nnatA}{ \type} $
   \item $ \tvdash{\nnatA} (\valr, \env) : \type  $, then $ (\valr,
     \env, \nnatA) \in \lrv{\type} $  .
  \end{enumerate}
\end{thm}

\begin{proof}
  Statement (1) is proved by induction on the configuration derivation.\\
   
 \caseL{ Case$     \inferrule{
     \env ( x ) = (\valr, \env', R)
      \\
      \tvdash{\nnatA} ( \valr, \env') : \type ~(\star)
    }{
     \tvdash{\adapt+\nnatA}   ( x, \env):  \type
    }~\textbf{C-Ax}
    $
  }

  TS: $(x, \env) \in  \lre{}{\adapt + \nnatA}{ \type}$.
  
  Let us first assume: \[   \inferrule{  \env(x)  =  (\valr, \env',  \adapt)  }{x,
      \env  \bigstep{\adapt} \valr, \env' }~\textsf{var2}  \]
  
  By unfolding the definition, STS: $\adapt \leq \adapt + \nnatA $  and $ (\valr, \env', \nnatA  ) \in \lrv{\type} $.

  By induction hypothesis on $(\star)$, we know $ (\valr, \env') \in
  \lre{}{\nnatA}{\type} $, by unfolding its definition and $ \valr,
  \env \bigstep{0} \valr, \env $,  we know that : $ (\valr, \env' ,
  \nnatA ) \in \lrv{\type}$.

 % This case is proved by applying Theorem~\ref{mono} on $(1)$.\\

  \caseL{ Case
    $
       \inferrule{
         \tvdash{\nnatA' } ( \valr', \theta') : \type_1 ~(\star)
      \\
      \fresh\eapp  x'
      \\
      \tvdash{  S+\idx \times (\adapt'+ \nnatA' ) +\nnatA }
     ( \expr[x'/x], \env[x' \to (\valr', \theta', R')]      ) :
     \type_2~(\diamond)
    }{
     \tvdash{S} (  \lambda x. \expr, \env )  : \bang{\idx} \type_1
      \lto{\nnatA} \type_2
    }~\textbf{C-lambda}
    $
  }

   TS: $(\lambda x. \expr , \env) \in  \lre{}{S}{  \bang{\idx} \type_1\lto{\nnatA} \type_2 }$.
  
  By unfoling the definition, as well as $\lambda x. \expr$ is value, we know :$ \valr, \env \bigstep{0} \valr, \env$.
  
  STS: $0 \leq S $  and $ (\lambda x.\expr, \env, S-0  ) \in \lrv{ \bang{\idx} \type_1\lto{\nnatA} \type_2 } $.

  By induction hypothesis on $(\star)$,  we know : $ (\valr', \env'
  ,\nnatA' -0) \in \lrv{\bang{k} \type_1} ~(1) $.

  Unfolding the definition of $\lrv{ \bang{\idx} \type_1\lto{\nnatA}
    \type_2 }$,  pick $ (\valr', \env' ,\nnatA') \in \lrv{\bang{k}
    \type_1} $.  

  STS: $ \fresh \eapp  x' \land \forall \adapt.  (\expr[x'/x] ,
  \env[x' \to (\valr', \theta', \adapt)] )  \in \lre{}{S+\nnatA+\idx
    \times (\adapt + \nnatA' )}{\type_2} $.

  Pick $\adapt = \adapt'$.
  STS: $  (\expr[x'/x] ,
  \env[x' \to (\valr', \theta', \adapt')] )  \in \lre{}{S+\nnatA+\idx
    \times (\adapt' +\nnatA' )}{\type_2} $

  It is proved by Induction hypothesis on $(\diamond)$. \\

  \caseL{Case
  $
     \inferrule{
       \tvdash{\nnatA_1} ( \expr_1, \env_1) :  \bang{\idx} \type_1
      \lto{\nnatA} \type_2    ~(\star)  \\
      \tvdash{\nnatA_2} ( \expr_2, \env_2 ): \type_1 ~(\diamond)
    }{
       \tvdash{    \nnatA_1 +\idx \times \nnatA_2 + \nnatA    } (  \expr_1 \eapp \expr_2, \env_1 \uplus \env_2   ) : \type_2
    }~\textbf{C-app}
  $
  }

     TS: $(\expr_1 \eapp  \expr_2, \env_1 \uplus \env_2) \in  \lre{}{ \nnatA_1 +\idx \times \nnatA_2 + \nnatA  }{ \type_2 }$.

  Let us first assume: \[    \inferrule{
     \expr_1, \env_1 \bigstep{\adapt_1} \lambda x.\expr , \env_1' ~(a)\\
     \expr_2, \env_2 \bigstep{\adapt_2} \valr_2 , \env_2' ~(b) \\
    \fresh \eapp x' \\
    \expr[x'/x], \env_1'[ x'  \to (\valr_2, \env_2', \adapt_2  ) ] 
    \bigstep{\adapt_3} \valr, \env_3~(c)
  }{
     \expr_1 \eapp \expr_2 , (\env_1 \uplus \env_2)\bigstep{\adapt_1+\adapt_3} \valr, \env_3
  }~\textsf{app}.
 \]

 By unfolding the definition:

 STS1: $ \adapt_1+\adapt_3 \leq \nnatA_1 +\idx \times \nnatA_2 + \nnatA   $ .

 STS2:  $ (\valr, \env_3, \nnatA_1 +\idx \times \nnatA_2 + \nnatA  -
 (\adapt_1  + \adapt_3) ) \in \lrv{\type_2} $.

By Induction hypothesis on $(\star)$,  we get: $ (\expr_1, \env_1) \in \lre{}{ \nnatA_1 }{ \bang{\idx} \type_1
  \lto{\nnatA} \type_2   } ~(1)$.

Unfolding $(1)$, from the assumption $(a)$, we know: $\adapt_1 \leq
\nnatA_1 \land (\lambda x.\expr, \env_1', \nnatA_1 - \adapt_1 ) \in \lrv{\bang{\idx} \type_1
  \lto{\nnatA} \type_2} ~(2)$.

By Induction hypothesis on $(\diamond)$,  we get: $ (\expr_2, \env_2) \in \lre{}{ \nnatA_2 }{ \type_1 } ~(3)$.

Unfolding $(3)$, from the assumption $(b)$, we know: $\adapt_2 \leq \nnatA_2 \land (\valr_2, \env_2', \nnatA_2-\adapt_2) \in \lrv{\type_1} ~(4)$.

Unfolding $(2)$,  pick $(\valr_2, \env_2', \nnatA_2 - \adapt_2 ) \in
\lrv{\type_1} $ from $(4)$.

We know: $ \fresh \eapp x' \land \forall \adapt.  (
\expr[x'/x], \env_1'[ x'  \to (\valr_2, \env_2', \adapt  ) ]  ) \in
\lre{}{ (\nnatA_1- \adapt_1) + \nnatA+\idx\times (\adapt+ \nnatA_2 -\adapt_2  ) }{\type_2} $.

Pick $\adapt = \adapt_2$.
We have: $ \fresh \eapp x' \land  (
\expr[x'/x], \env_1'[ x'  \to (\valr_2, \env_2', \adapt_2  ) ]  ) \in
\lre{}{ (\nnatA_1- \adapt_1) + \nnatA+\idx\times ( \adapt_2 + \nnatA_2
  - \adapt_2)   }{\type_2}
~(5) $.

Unfolding $(5)$,we conclude that: $ \adapt_3 \leq (\nnatA_1 -
\adapt_1) + \nnatA +\idx\times
(\nnatA_2  - \adapt_2 +\adapt_2)~(6)$.  and $(\valr, \env_3,   (\nnatA_1- \adapt_1) +
\nnatA+\idx\times \adapt_2 - \adapt_3 ) \in \lrv{\type_2} ~(7) $ .

STS1 is proved by using both $(6)$ .  STS2 is proved by $(7)$. \\

\caseL{Case
  $
 \inferrule{
      \tvdash{\nnatA} (\expr, \env) :  \tbase ~(\star)
    }{  \tvdash{1+\nnatA} (\delta(\expr) , \env ) : \tbase
    }~\textbf{C-delta}
  $
}

 TS: $(\delta(\expr), \env ) \in  \lre{}{ 1+ \nnatA}{ \tbase }$.
 We first assume:
 \[ \inferrule{
    \expr , \env \bigstep{\adapt} \valr' , \env_1 ~(a) \\
    \eop{}(\valr') = \valr
  }{
    \eop(\expr), \env \bigstep{\adapt +1} \valr,  \env_1
  }~\textsf{delta}.
  \]

  By unfolding the definition:

  STS1: $\adapt+1 \leq \nnatA +1$.
  
STS2: $ (\valr, \env_1, \nnatA- \adapt) \in \lrv{\tbase} $.

By induction hypothesis on $(\star)$, we get: $ (\expr, \env) \in \lre{}{\nnatA}{\tbase} $.
Unfold this statement, from the assumption $(a)$, we get: $ \adapt \leq \nnatA~(1) $ and $ (\valr', \env_1, \nnatA-\adapt)  \in \lrv{\tbase}~(2)$.

STS1 is proved by $(1)$,  STS2 is proved by $(2)$ and the
interpretation of $\delta(\valr')$. \\


Statement (2) is proved by induction on the value $\valr$.


\caseL{ Case
    $
       \inferrule{
         \tvdash{\nnatA' } ( \valr', \theta') : \type_1 ~(\star)
      \\
      \fresh\eapp  x'
      \\
      \tvdash{  S+\idx \times R' +\nnatA }
     ( \expr[x'/x], \env[x' \to (\valr', \theta', R')]      ) :
     \type_2~(\diamond)
    }{
     \tvdash{S} (  \lambda x. \expr, \env )  : \bang{\idx} \type_1
      \lto{\nnatA} \type_2
    }~\textbf{C-lambda}
    $
  }

   TS: $(\lambda x. \expr , \env, S) \in  \lrv{  \bang{\idx} \type_1\lto{\nnatA} \type_2 }$.

  By induction hypothesis on $(\star)$,  we know : $ (\valr', \env'
  ,\nnatA' -0) \in \lrv{\bang{k} \type_1} ~(1) $.

  Unfolding the definition of $\lrv{ \bang{\idx} \type_1\lto{\nnatA}
    \type_2 }$,  pick $ (\valr', \env' ,\nnatA') \in \lrv{\bang{k}
    \type_1} $.  

  STS: $ \fresh \eapp  x' \land \forall \adapt.  (\expr[x'/x] ,
  \env[x' \to (\valr', \theta', \adapt)] )  \in \lre{}{S+\nnatA+\idx
    \times \adapt}{\type_2} $.

  Pick $\adapt = \adapt'$.
  STS: $  (\expr[x'/x] ,
  \env[x' \to (\valr', \theta', \adapt')] )  \in \lre{}{S+\nnatA+\idx
    \times \adapt'}{\type_2} $

  It is proved by Induction hypothesis on $(\diamond)$. \\





 \end{proof} 

% \begin{thm}[Substitution]
%   \label{sub}
%   \begin{enumerate} 
%    \item If $ \Gamma,x : \type' \tvdash{ \nnatA} \expr : \type $ and $
%   \empty \tvdash{\nnatA'} \valr : \type'  $ , then  $\Gamma
%   \tvdash{\max(\nnatA,\nnatA' )} \expr[\valr/x]  : \type$. 
%   \end{enumerate}
% \end{thm}

% \begin{proof}
%   By induction on the typing derivation.\\
% \caseL{
%   $   \inferrule{
%     }{
%       \ictx \tctx , x: \bang{\nnatA}\ltype \tvdash{\nnatA} x: \bang{\nnatA}\ltype
%     }~\textbf{Ax}  $
%   }
% Assume $\empty \tvdash{\nnatA'} \valr : \bang{\nnatA}\ltype $, TS:  $\Gamma
%   \tvdash{\max(\nnatA,\nnatA' )} x[\valr/x]  : \type$. proved by
%   subtype rule on the assumption.
% \caseL{
%  $   \inferrule{
%     }{
%       \ictx \tctx ,y:\type', x: \bang{\nnatA}\ltype \tvdash{\nnatA} x: \bang{\nnatA}\ltype
%     }~\textbf{Ax2}  $
%   }
%   Assume $\empty \tvdash{\nnatA'} \valr : \bang{\nnatA}\ltype $, TS:
%   $\Gamma,   x: \bang{\nnatA}\ltype
%   \tvdash{\max(\nnatA,\nnatA' )} x[\valr/y]  : \type$. proved by rule
%   AX and then subtype.
%   \caseL{
%    \inferrule{
%       \ictx \Gamma, x: \type_1, y:\type'
%       \tvdash{\nnatA }
%       \expr: \type_2
%     }{
%       \ictx k+\Gamma, y: k + \type' \tvdash{k+\nnatA} \lambda x. \expr : \bang{k}  ( \type_1
%       \lto \type_2)
%     }~\textbf{lambda}
%   }
%    Assume $\empty \tvdash{k+\nnatA'} \valr : k+\type' $, TS:
%   $k+\Gamma
%   \tvdash{\max(k+\nnatA,k+\nnatA' )} (\lambda x. \expr)[\valr/y]  : \type$. From the
%   Lemma~\ref{para-dec} on the assumption, we know: $\empty
%   \tvdash{\nnatA'} \valr : \type' ~(1)$.\\
%   By Induction hypothesis on the premise, we get: $ \Gamma, x:\type_1
%   \tvdash{\max( \nnatA, \nnatA' )}
%       \expr[\valr/y]: \type_2 ~(2)$. By rule lambda, we conclude that
%       $k+\Gamma \tvdash{ k+ ( \max(\nnatA,\nnatA ) }
%       \lambda x.\expr[\valr/y]: \type_2 $.
%       \caseL{
%       \inferrule{
%       \ictx \Gamma_1,x:\type'  \tvdash{\nnatA_1} \expr_1:  \bang{0} ( \type_1
%       \lto \type_2      ) \\
%       \ictx \Gamma_2 ,x: \type'', \tvdash{\nnatA_2} \expr_2: \type_1 
%     }{
%       \ictx \max (\Gamma_1, \Gamma_2 ), x:\max(\type',\type'') \tvdash{\max( \nnatA_1,\nnatA_2) } \expr_1 \eapp \expr_2 : \type_2
%     }~\textbf{app}
%   }
%   Assume $\empty \tvdash{\nnatA'} \valr : \max(\type',\type'')$, TS: $\max (\Gamma_1, \Gamma_2 )
%   \tvdash{\max(\nnatA_1,\nnatA_2, \nnatA' )} (\expr_1 \eapp
%   \expr_2)[\valr/x]  : \type_2$. From the definition of $\max(\type',
%   \type'')$, we know that $\type'$ and $\type''$ have similar
%   form. Let us assume $\type'= \bang{k_1} \ltype$ and $\type'' =
%   \bang{k_2} \ltype$ so that $\max(\type',\type'') = \bang{\max(k_1,k_2)}
%   \ltype$.\\
%   From the Lemma~\ref{para-dec} on the assumption, we have $\empty
%   \tvdash{\nnatA' - (\max(k_1,k_2)-k_1) } \valr : \bang{k_1}
%   \ltype~(1)$ and $\empty
%   \tvdash{\nnatA' - (\max(k_1,k_2)-k_2) } \valr : \bang{k_2}
%   \ltype~(2)$.\\ By induction hypothesis on $(1)$ and $(2)$ respctively,
%   we know that:  $ \Gamma_1  \tvdash{ \max( \nnatA_1, \nnatA' - (\max(k_1,k_2)-k_1) ) } \expr_1[\valr/x]:  \bang{0} ( \type_1
%   \lto \type_2   ) ~(3)$  and $ \Gamma_2  \tvdash{\max(\nnatA_2 ,
%     \nnatA' - (\max(k_1,k_2)-k_2)   )} \expr_2[\valr/x]: \type_1 ~(4)$.  By the
%   rule app and $(3)$, $(4)$, we conclude that $$\max (\Gamma_1, \Gamma_2 )
%   \tvdash{\max(  \max( \nnatA_1, \nnatA' - (\max(k_1,k_2)-k_1) )  , \max(\nnatA_2 ,
%     \nnatA' - (\max(k_1,k_2)-k_2)   )  )} \expr_1[\valr/x] \eapp
%   \expr_2[\valr/x]  : \type_2 ~(5).$$
%   Because $\max(\nnatA' - (\max(k_1,k_2)-k_1) ) , \nnatA' -
%   (\max(k_1,k_2)-k_2)   ) \leq \nnatA' $, by subtype, we raise the
%   adaptivity to  $\max(\nnatA_1,\nnatA_2, \nnatA' ) $ from $(5)$.
%    \caseL{
%       \inferrule{
%       \ictx \Gamma_1,x:\type'  \tvdash{\nnatA_1} \expr_1:  \bang{0} ( \type_1
%       \lto \type_2      ) \\
%       \ictx \Gamma_2  \tvdash{\nnatA_2} \expr_2: \type_1 
%     }{
%       \ictx \max (\Gamma_1, \Gamma_2 ), x:\type' \tvdash{\max( \nnatA_1,\nnatA_2) } \expr_1 \eapp \expr_2 : \type_2
%     }~\textbf{app2}
%   }
%   It is another case for application when x only appear in the first
%   premise. In this case, $\expr_2[\valr/x] = \expr_2$. Another case
%   when variable x only appears in the second premise can be proved in
%   a similar way.\\
%   Assume $\empty \tvdash{\nnatA'} \valr :\type'$. TS:$\max (\Gamma_1, \Gamma_2 )
%   \tvdash{\max(\nnatA_1,\nnatA_2, \nnatA' )} (\expr_1 \eapp
%   \expr_2)[\valr/x]  : \type_2$.  By Induction Hypothesis on the first
%   premise using the assumption, we get: $\Gamma_1
%   \tvdash{\max(\nnatA_1, \nnatA')} \expr_1[\valr/x]:  \bang{0} ( \type_1
%       \lto \type_2  )  ~(1)$. By the rule app using (1) and the second
%       premise, we conclude that $$ \max (\Gamma_1, \Gamma_2 )
%       \tvdash{\max( \max(\nnatA_1,\nnatA'),\nnatA_2) }
%       \expr_1[\valr/x] \eapp \expr_2 : \type_2$$
%       \caseL{
%  \inferrule{
%       \ictx \Gamma, x:\type' \tvdash{\nnatA} \expr: \bang{k} \ltype 
%     }{
%       \ictx \Gamma' ,1+\Gamma, x:1+\type'  \tvdash{1+\nnatA} \delta(\expr): \bang{k} \ltype 
%     }~\textbf{delta}
%   }
%   Assume $\empty \tvdash{\nnatA'+1} \valr : 1+\type' $, TS: $ \Gamma'
%   ,1+\Gamma \tvdash{\max(1+\nnatA, 1+\nnatA')} \delta(\expr)
%   [\valr/x]: \bang{k} \ltype $.
%   By Lemma~\ref{para-dec} on the assumption, we have $\empty
%   \tvdash{\nnatA'} \valr : \type'~(1) $. By IH on the first premise
%   along with (1), we have: $\Gamma \tvdash{\max(\nnatA, \nnatA')}
%   \expr[\valr/x]: \bang{k} \ltype~ (2)$.
%    By the rule delta using (2), we conclude that $\Gamma' ,1+\Gamma  \tvdash{1+(\nnatA,\nnatA')} \delta(\expr[\valr/x]): \bang{k} \ltype$.
% \end{proof}

% \begin{thm}[Substitution]
%   \label{sub}
%   \begin{enumerate} 
%    \item If $ \Gamma,x : \bang{k} \ltype \tvdash{ \nnatA} \expr : \type $ and $
%   \empty \tvdash{k} \valr : \bang{k} \ltype  $ , then  $\Gamma
%   \tvdash{ \nnatA } \expr[\valr/x]  : \type$. 
%   \end{enumerate}
% \end{thm}

% \begin{proof}
%   By induction on the typing derivation.\\
%%%%%%%%%%%%%%%%%%%%%
% \caseL{
%   $   \inferrule{
%     }{
%       \ictx \tctx , x: \bang{\nnatA}\ltype \tvdash{\nnatA} x: \bang{\nnatA}\ltype
%     }~\textbf{Ax}  $
%   }
  
% Assume $\empty \tvdash{\nnatA} \valr : \bang{\nnatA}\ltype $ TS:  $\Gamma
% \tvdash{\nnatA } x[\valr/x]  : \type$. proved by the assumption.\\

% \caseL{
%  $   \inferrule{
%     }{
%       \ictx \tctx ,y:\type', x: \bang{\nnatA}\ltype \tvdash{\nnatA} x: \bang{\nnatA}\ltype
%     }~\textbf{Ax2}  $
%   }
  
% Assume $\empty \tvdash{\nnatA} \valr : \bang{\nnatA}\ltype $, TS:
%    $\Gamma,   x: \bang{\nnatA}\ltype
%    \tvdash{\nnatA } x[\valr/y]  : \type$. proved by the assumption.\\

% %%%%%%%%%%%%%%%%%%%%%%%%%%%%%%%
%   \caseL{
%    \inferrule{
%       \ictx \Gamma, x: \type_1, y: \bang{k_1} \ltype
%       \tvdash{\nnatA }
%       \expr: \type_2
%     }{
%       \ictx k+\Gamma, y: k + \bang{k_1} \ltype \tvdash{k} \lambda x. \expr : \bang{k}  ( \type_1
%       \lto^{\nnatA} \type_2)
%     }~\textbf{lambda}
%   }

%    Assume $\empty \tvdash{k+k_1 } \valr : k+ \bang{k_1} \ltype $, TS:
%   $k+\Gamma
%   \tvdash{ k } (\lambda x. \expr)[\valr/y]  : \bang{k}  ( \type_1
%       \lto^{\nnatA} \type_2)$.

%  From the  Lemma~\ref{para-dec} on the assumption, we know: $\empty
%   \tvdash{ k_1} \valr : \bang{k_1} \ltype ~(1)$.

%   By Induction hypothesis on the premise: $ \Gamma, x:\type_1
%   \tvdash{\nnatA }
%       \expr[\valr/y]: \type_2 ~(2)$. 

% By rule lambda, we conclude that
%       $k+\Gamma \tvdash{ k }
%       \lambda x.\expr[\valr/y]: \bang{k}  ( \type_1
%       \lto^{\nnatA} \type_2) $.\\
      
%       %%%%%%%%%%%%%%%%%%%%%%%%%%%%%%%%%
      
%       \caseL{
%       \inferrule{
%       \ictx \Gamma ,x:\bang{k} \ltype  \tvdash{\nnatA_1} \expr_1:  \bang{0} ( \type_1
%       \lto^{\nnatA} \type_2      ) \\
%       \ictx \Gamma ,x:\bang{k} \ltype  \tvdash{\nnatA_2} \expr_2: \type_1 
%     }{
%       \ictx \Gamma, x:\bang{k} \ltype \tvdash{ \nnatA+ \max( \nnatA,\nnatA_2) } \expr_1 \eapp \expr_2 : \type_2
%     }~\textbf{app}
%   }
  
%   Assume $\empty \tvdash{k} \valr : \bang{k} \ltype$, TS: $\Gamma
%   \tvdash{ \nnatA_1+ \max(\nnatA,\nnatA_2)  } (\expr_1 \eapp
%   \expr_2)[\valr/x]  : \type_2$.
  
% By induction hypothesis on the first and second premise, we conclude
% : $\ictx \Gamma \tvdash{\nnatA_1}  \expr_1[\valr/x]  :  \bang{0}
% ( \type_1 \lto^{\nnatA} \type_2 )~(1)$ and $ \ictx \Gamma \tvdash{\nnatA_2} \expr_2[\valr/x]: \type_1(2)$.

%  By the rule app and $(1)$, $(2)$,  we prove that $$ \Gamma
%   \tvdash{  \nnatA_1 + \max(\nnatA, \nnatA_2)  }  \expr_1[\valr/x] \eapp
%   \expr_2[\valr/x]  : \type_2$$

% \end{proof}

% \begin{lem}[Parameter Decreasing]
%   \label{para-dec}
%   if  $k+\Gamma \tvdash{\nnatA} \valr : k+ \type  $, then exists 
%   $\nnatA'$ so
%   that   $\Gamma \tvdash{\nnatA' } \valr :
%   \type$ and  $\nnatA' \leq \nnatA-k $.
% \end{lem}
% \begin{proof}
  
%   If $\valr$ is a constant, then it is trivial, assume $\type =
%   \bang{r} \tbase$, then $\nnatA = r+k$,  choose
%   $\nnatA' = r$, proved from the rule $const$.
  
%   If $\valr$ is an abstraction term, assuming $\valr = \lambda
%   x. \expr$. Correspondingly, the type of $\valr$ is an arrow type,
%   assuming $\Gamma = r+ \Gamma_1 $ and   $\type =\bang{r} (\type_1
%   \lto^{\nnatA} \ltype_2)$, from the typing derivation, we know : 
%   \[
%   \inferrule{
%       \ictx \Gamma_1, x: \type_1
%       \tvdash{\nnatA }
%       \expr: \type_2  ~(1)
%     }{
%       \ictx k+r+\Gamma_1  \tvdash{k+r} \lambda x. \expr : k + \bang{r}  ( \type_1
%       \lto^{\nnatA} \type_2)
%     }~\textbf{lambda}
% \]

% Use $(1)$ as premise, we use lambda rule again.

%  \[
%   \inferrule{
%       \ictx \Gamma_1, x: \type_1
%       \tvdash{\nnatA }
%       \expr: \type_2  ~(1)
%     }{
%       \ictx r+\Gamma_1  \tvdash{r} \lambda x. \expr :  \bang{r}  ( \type_1
%       \lto^{\nnatA} \type_2)
%     }~\textbf{lambda}
% \]

% \end{proof}

% \[
%   \begin{array}{lll}
%      \env \vDash \Gamma &\triangleq  & 
%                                         \forall x_i \in \dom( \Gamma) . \env(x_i) =
%                                        (\valr_i, \adapt_i) \land
%                                        \empty \tvdash{\adapt_i} \valr_i
%                                        : \Gamma(x_i)   \\
%     F(\env, \expr) & ::= &  \max(\max(\adapt_i) , 0 )    \\
%     & where &   \forall x_i \in \fv{\expr} . \env(x_i) = (\valr_i, \adapt_i).
%                                           \\
%     \end{array}
% \]

% \begin{thm}[Soundness- one attempt]
% \label{soundness}
% If $\Gamma \tvdash{\nnatA} \expr : \bang{k} \ltype$, $ \forall \env$ that $\env
% \vDash \Gamma$, exists $\env'$ and $\valr$ so that $\env , \expr \bigstep{\adapt} \valr,
% \env'  $, then  $ \adapt + k \leq  \nnatA + F(\env, \expr)$.  
% \end{thm}
% \begin{proof}
%   By Induction on the typing derivation.
%   \caseL{
%      $   \inferrule{
%     }{
%       \ictx \tctx , x: \bang{\nnatA}\ltype \tvdash{\nnatA} x: \bang{\nnatA}\ltype
%     }~\textbf{Ax}  $
%   }
  
%   Assume $\env= \Big( \env_1, [x \to (\valr,\adapt
%   )]  \Big) \vDash (\tctx , x: \bang{\nnatA}\ltype  )$ where $\env_1
%   \vDash \Gamma$.
  
%   We know from the evaluation rule var: $\Big( \env_1, [x \to (\valr,\adapt
%   )]  \Big) , x \bigstep{\adapt} \valr,
%   \env  $.
  
%   TS:  $ \adapt +\nnatA  \leq  \nnatA +
%   F(\env,x) \implies \adapt + \nnatA \leq \nnatA +  R
%   $. It is trivially true.\\

% \caseL{
%  $
% \inferrule{
%       \ictx \Gamma, x: \type_1
%       \tvdash{\nnatA }
%       \expr: \type_2
%     }{
%       \ictx k+\Gamma \tvdash{k} \lambda x. \expr : \bang{k}  ( \type_1
%       \lto^{\nnatA} \type_2)
%     }~\textbf{lambda}
%     \and
%     %
%  $
% }

% Assume $\env \vDash k+\Gamma $,  from the evaluation rule lambda:
% $\env , \lambda x. \expr \bigstep{0}   \lambda x. \expr ,
%   \env  $.

% TS: $ 0 + k  \leq  k  +
%   F(\env, \lambda x. \expr)  
%   $, which is trivially true. \\
  
% \caseL{
% $
%     \inferrule{
%       \ictx \Gamma  \tvdash{\nnatA_1} \expr_1:  \bang{0} ( \type_1
%       \lto^{\nnatA} \type_2      ) \\
%       \ictx \Gamma \tvdash{\nnatA_2} \expr_2: \type_1 
%     }{
%       \ictx \Gamma  \tvdash{ \nnatA_1 + \max( \nnatA,\nnatA_2) } \expr_1 \eapp \expr_2 : \type_2
%     }~\textbf{app}
% $
% }

% \end{proof}



% \begin{thm}[Soundness-original]
% \label{soundness}
% If $\Gamma \tvdash{\nnatA} \expr : \type$, $ \forall \env$ that $\env
% \vDash \Gamma$, exists $\env'$ and $\valr$ so that $\env , \expr \bigstep{\adapt} \valr,
% \env'  $, then  $ \adapt + adap(\valr, \env')  \leq  \nnatA + F(\env, \expr)$.  
% \end{thm}


% \begin{proof}
%   By Induction on the typing derivation.
%   \caseL{
%      $   \inferrule{
%     }{
%       \ictx \tctx , x: \bang{\nnatA}\ltype \tvdash{\nnatA} x: \bang{\nnatA}\ltype
%     }~\textbf{Ax}  $
%   }
%   Assume $\env= \Big( \env_1, [x \to (\valr,\adapt
%   )] , \Big) \vDash (\tctx , x: \bang{\nnatA}\ltype  )$ where $\env_1 \vDash \Gamma$. We know that $
%   \empty \tvdash{\adapt} \valr : \bang{\nnatA}\ltype $.
%   From the evaluation rule var, we know $\env , x \bigstep{\adapt} \valr,
%   \env  $.
%   TS:  $ \adapt + adap(\valr, \env)  \leq  \nnatA +
%   F(\env) \implies \adapt + 0 \leq \nnatA + \max( \adapt, F(\env_1))
%   $.It is trivially true.
% \caseL{
%   $
%     \inferrule{
%       \ictx \Gamma, x: \type_1
%       \tvdash{\nnatA }
%       \expr: \type_2
%     }{
%       \ictx k+\Gamma \tvdash{k+\nnatA} \lambda x. \expr : \bang{k}  ( \type_1
%       \lto \type_2)
%     }~\textbf{lambda}
%   $
% }
% Choose $\env \vDash  (k+\Gamma)$ so that $\forall x_i \in
% (\Gamma). \env(x_1) =(\valr_i, \adapt_i ) \land \empty
% \tvdash{\adapt_i } \valr_i: k+\Gamma(x_i) $.  By the evaluation rule
% we know $\env, \lambda x. \expr \bigstep{0}
%                                        \lambda x.\expr, \env $, TS: $0
%                                        + \adap(\lambda x.\expr, \env)
%                                        \leq  k+\nnatA + F(\env)$, which is trivially
%                                        true because $ \adap(\lambda
%                                        x.\expr, \env) \leq F(\env) $.
                                       
% \caseL{
%     $  \inferrule{
%       \ictx \Gamma_1  \tvdash{\nnatA_1} \expr_1:  \bang{0} ( \type_1
%       \lto \type_2      ) \\
%       \ictx \Gamma_2 \tvdash{\nnatA_2} \expr_2: \type_1 
%     }{
%       \ictx \max (\Gamma_1, \Gamma_2 ) \tvdash{\max( \nnatA_1,\nnatA_2) } \expr_1 \eapp \expr_2 : \type_2
%     }~\textbf{app}  $
%   }
%   Choose $\env = [x_i \to (\valr_i,0)] $ for all $x_i$ in
%   $\dom(\max(\Gamma_1,\Gamma_2))$
%   so that  $\empty \tvdash{\nnatA_i} \valr_i  : (\max(\Gamma_1,
%   \Gamma_2)(x_i) $.
%   From the definition, we know that $\env \vDash \Gamma_1$ and $\env
%   \vDash \Gamma_2$. Because $\expr_1$ has the arrow type and will be
%   evaluated to a function, assume exists $\env_1$ so that $\env,
%   \expr_1 \bigstep{\adapt_1} \lambda x.\expr , \env_1 $.  By induction
%   hypothesis on the first premise, we know that: $\adapt_1 +
%   \adap(\lambda x. \expr, \env_1) \leq \nnatA_1 + F(\env,
%   \Gamma_1)~(1)$.Assume exists $\env_2$ so that $\expr_2$ is evaluated
%   to an arbitrary value $\valr_2$ : $ \env, \expr_2 \bigstep{\adapt_2}
%   \valr_2 , \env_2$, by induction hypothesis, we conclude that :  $\adapt_2 +
%   \adap(\valr , \env_2) \leq \nnatA_2 + F(\env,
%   \Gamma_2)~(2)$.
                            


% \[
% \inferrule{
%     \env, \expr_1 \bigstep{\adapt_1} \lambda x.\expr , \env_1 \\
%     \env, \expr_2 \bigstep{\adapt_2} \valr_2 , \env_2 \\
%     (\env_1 \uplus \env_2)[ x  \to (\valr_2,   \adapt_2  ) ], \expr
%     \bigstep{\adapt_3} \valr, \env_3
%   }{
%     \env, \expr_1 \eapp \expr_2 \bigstep{\adapt_1+\adapt_3} \valr, \env_3
%   }~\textsf{app}
% \]
%  \end{proof} 


% \begin{thm}[Subject Reduction]
% \label{sub-red}
% If $\Gamma \tvdash{\nnatA} \expr : \bang{k} \ltype$, $\forall \env . \env
% \vDash \Gamma$,   exists $\env'$ and $\valr$, $\env , \expr \bigstep{\adapt} \valr,
% \env'  $, then $ \Gamma  \tvdash{ k} \valr :\bang{k} \ltype $.  
% \end{thm}
% By induction on the typing derivation.

\clearpage
\section{Typable Approach}
\[
\begin{array}{ll}
 F(\expr, \phi ) & where \eapp ~~ \phi(x_i) = (\idx_i, \adapt_i, \nnatA_i ) \\
   F(x,\phi) & = \sum_{x_i \in \fv{x}  } \idx_i \times (\adapt_i+ \nnatA_i)  \\
F(\lambda x. \expr ,  \phi  ) & =  \sum_{x_i \in \fv{\lambda x.\expr}  } \idx_i \times (\adapt_i+ \nnatA_i)   \\  %\sum_{x_i \in \fv{\lambda x.\expr} } k_i \times R_i  \\
F(\delta(\expr) , \phi ) & = \sum_{x_i \in \fv{\delta(\expr)} } \idx_i \times (\adapt_i+ \nnatA_i)  \\
F(c, \phi ) & = 0  \\
F(\expr_1 \eapp \expr_2, \phi ) & = F(\expr_1, \phi ) +
                                        F(\expr_2,\phi )
\end{array} 
\]

\begin{defn}[Typable]
  \label{typable}
  A closure $( \expr, [ x_1 \to (\valr_1 ,  \env_1 , \adapt_1 ) , \ldots, x_i \to (\valr_i, \env_i, \adapt_i )] )$ is typable with type $\type$ and adaptivity $J$ if exists $k_i$\\
  \[
     x_1 : \bang{\idx_1} \type_1, \ldots, \bang{\idx_i} \type_i 
     \tvdash{\nnatA}  \expr : \type  \]
   and each closure $(\valr_i, \env_i)$  is also typable with type $\bang{\idx_i} \type_i$ and adaptivity $\nnatA_i$, $ \phi = [x_1
     \to (\idx_1, \adapt_1, \nnatA_1), \ldots,  x_i \to (\idx_i, \adapt_i,
     \nnatA_i)  ] $,  $J = \nnatA + F( \expr, \phi ) $.
 \end{defn}

 \begin{defn}[ClosedClosure]
  \label{closure}
   A closure $(\expr, \env)$ is closed if $\fv{\expr} \subseteq \dom(\env)$. 
 \end{defn}

% \begin{lem}[ClosureTypable ]
%   \label{ct}
%    If a closure $(\expr, \env)$ is closed, then there exists $\type$ and $J$ so that $(\expr, \env)$ is typable with $\type$ and $J$.
%    \end{lem}
 

\begin{lem}[programTypable]
  \label{proglemma}
   If $ \tvdash{\nnatA}   \expr : \type $, then $(
     \expr, \emptyset ) $ is typable with $\type$ and adaptivity $\nnatA$. 
   \end{lem}

   \begin{lem}[TypableMono]
     \label{tmono}
     If a closure is $D$ is typable with $\type$ and $\nnatA$, and $\nnatA \leq \nnatA'$, then
     D is typable with $\type$ and $\nnatA'$.
    \end{lem} 

   
\begin{lem}[TypableSoundness]
  \label{tsound}
  If a closure $D$ is typable with $\type$ and $J$, and $D \bigstep{\adapt} E$, then
    closure $E$ is typable with $\type$ and adaptivity $J - \adapt$. 
   \end{lem}

   \begin{proof}
     By induction on the evaluation semantics.\\

     \caseL{Case
       \[
         \inferrule{  \env(x)  =  (\valr, \env',  \adapt)  }
         {x,\env  \bigstep{\adapt} \valr, \env' }~\textsf{var2}
       \]
     }

     Suppose $(x,\env)$ is typable with $\type$ and $J$ and $\env = [x \to (\valr, \env', \adapt), x_1 \to (\valr_1, \env_1, \adapt_1) , \ldots ,  x_i \to (\valr_i, \env_i, \adapt_i) ]  $\\
     
     We know that: exists $k_i$ so that $ x_1 : \bang{\idx_1} \type_1, \ldots, \bang{\idx_i} \type_i,  x: \bang{1} \type \tvdash{0} x : \type $,  and the each closure $ (\valr_i, \env_i) $ is typable with $\type_i$ and $\nnatA_i$ as well as $(\valr, \env')$ is typable with $\type$ and $\nnatA$.  and $J = 0 + (\adapt + \nnatA) $.\\

     TS: $(\valr, \env') $ is typable with $\type$ and $J - \adapt$,  it is proved by Lemma~\ref{tmono} on the assumption.\\

     \caseL{Case
       \[
         \inferrule{
    \expr_1, \env_1 \bigstep{\adapt_1} \lambda x.\expr , \env_1' \\
    %\forall x_i \in \dom(\env_1 \cap \env_2).  \fresh \eapp x_i' \\
     \expr_2, \env_2 \bigstep{\adapt_2} \valr_2 , \env_2' \\
    \fresh \eapp x' \\
    \expr[x'/x], \env_1'[ x'  \to (\valr_2, \env_2', \adapt_2  ) ] 
    \bigstep{\adapt_3} \valr, \env_3 ~(\clubsuit)
  }{
     \expr_1 \eapp \expr_2 , (\env_1 \uplus \env_2)\bigstep{\adapt_1+\adapt_3} \valr, \env_3
  }~\textsf{app}
     \]
   }
   
   Suppose for each variable $x_i \in \dom(\env_1)$, $(\valr_i, \env_i)$ is tyable with type $\type_i$ and $I_i$. For each variable $x_j \in \dom(\env_2)$, $(\valr_j, \env_j)$ is tyable with type $\type_j$ and $I_j$.
   
   We assume exists $\idx_i$ and $\Gamma_1 = x_1 : \bang{\idx_1} \type_1 , \ldots, x_i : \bang{\idx_i} \type_i$ for $x_i \in \dom(\env_1)$, so that $  \Gamma_1  \tvdash{\nnatA_1} \expr_1:  \bang{\idx} \type_1 \lto{\nnatA} \type_2 $.

   Similarly, we assume exists $\idx_j$ and $\Gamma_2 = x_1 : \bang{\idx_1} \type_1 , \ldots, x_j : \bang{\idx_j} \type_j$ for $x_j \in \dom(\env_j)$, so that $\Gamma_2 \tvdash{\nnatA_2} \expr_2: \type_1$.
  
   By the typing rule app, we have:  
   \[
     \inferrule{
      \ictx \Gamma_1  \tvdash{\nnatA_1} \expr_1:  \bang{\idx} \type_1
      \lto{\nnatA} \type_2      \\
      \ictx \Gamma_2 \tvdash{\nnatA_2} \expr_2: \type_1 
    }{
      \ictx   \Gamma_1 + \idx \times \Gamma_2  \tvdash{    \nnatA_1 +
        \idx \times \nnatA_2 + \nnatA    } \expr_1 \eapp \expr_2 : \type_2
    }~\textbf{app}
  \]

  We know that $\dom(\Gamma_1 + \idx \times \Gamma_2) = \dom(\Gamma_1) \cup \dom(\Gamma_2) = \dom( \env_1 \uplus \env_2 )$. So forall all the closure $(\valr_i, \env_i)$ assigned by variable $x_i \in \dom(\Gamma_1 + \idx \times \Gamma_2)$,  $(\valr_i, \env_i)$ is typable with $\type_i$ and $I_i$ from our assumption.

  $\phi = [x_1 \to (\idx_1, \adapt_1, I_1) , \ldots, x_i \to (\idx_i, \adapt_i, I_i)]$ where $x_i \in \dom(\env_1 \uplus \env_2)$.

  So, we know that : $( \expr_1 \eapp \expr_2,  \env_1 \uplus \env_2 )$ is typable with $\type_2$ and $J= \nnatA_1 +\idx \times \nnatA_2 + \nnatA   +F(\expr_1 \eapp \expr_2, \phi) $. \\
   
   TS: $(\valr, \env_3) $ is typable with  $\type_2$ and $J- (\adapt_1+\adapt_3)$.

   Set $\phi_1 =[x_1 \to (\idx_1, \adapt_1, I_1) , \ldots, x_i \to (\idx_i, \adapt_i, I_i)]$ where $x_i \in \dom(\env_1)$.

    Set $\phi_2 =[x_1 \to (\idx_1, \adapt_1, I_1) , \ldots, x_i \to (\idx_i, \adapt_i, I_i)]$ where $x_i \in \dom(\env_2)$.
   
   From our assumption, we also know that $(\expr_1, \env_1)$ is typable with $\bang{\idx} \type_1 \lto{\nnatA} \type_2$ and $J_1 = \nnatA_1 +   F(\expr_1, \phi_1) $.

   Similarly, we have: $(\expr_2, \env_2)$ is  typable with $\type_1$ and $J_2 = \nnatA_2 + F(\expr_2, \phi_2)$ for $x_j \in \dom(\env_2)~(2)$ .

   By induction hypothesis on $(1)$, $(2)$ respectively, we know that:
   
   $ ( \lambda x.\expr , \env_1') $ is typable with $\bang{\idx} \type_1 \lto{\nnatA} \type_2$ and $J_1 - \adapt_1$ ~(3). 
   
   $(\valr_2 , \env_2')  $ typable with $ \type_1 $ and $J_2 -\adapt_2$ (4). 

   From (3), we know the following if we assume exists $\idx_i''$ and $\Gamma_1'' = x_1'' : \bang{\idx_1''} \type_1'', \ldots, x_i'' : \bang{\idx_i''} \type_i'' $, and $\env_1'(x_i'') = (\valr_i'', \env_i'', \adapt_i'')$ and each closure $(\valr_i'', \env_i'' )  $ is typable with $\type_i''$ and $I_i''$.

 Set $\phi_1'' =[x_1'' \to (\idx_1'', \adapt_1'', I_1'') , \ldots, x_i'' \to (\idx_i'', \adapt_i'', I_i'')]$ where $x_i'' \in \dom(\env_1')$.

   
   $$ \inferrule{
      \ictx \Gamma_1', x: \bang{\idx} \type_1
      \tvdash{\nnatA }
      \expr: \type_2 ~(\star)
    }{
      \ictx \Gamma_1' \tvdash{0} \lambda x. \expr : \bang{\idx} \type_1
      \lto{\nnatA} \type_2
    }~\textbf{lambda} $$

    and we know $J_1 - \adapt_1 = 0 + F(  \lambda x. \expr , \phi_1'' )$.

    Take a fresh variable $x'$, from $(\star)$, we know: $\ictx \Gamma_1, x': \bang{\idx} \type_1 \tvdash{\nnatA }  \expr[x/x']: \type_2 ~(\star\star) $.
    
   Set $\phi_1''' = \phi_1''[x' \to (\idx, \adapt_2,  J_2-\adapt_2 )]$.
    
   From $(\star\star)$ and $(4)$, we conclude that $( \expr[x'/x], \env_1'[ x'  \to (\valr_2, \env_2', \adapt_2  ) ] )$ is typable with $\type_2$ and $ \nnatA+ F( \expr[x'/x], \phi_1''' )$.

Because  $ F( \expr[x'/x], \phi_1''' ) =   F(\lambda x.\expr, \phi_1'') +\idx \times (\adapt_2 + J_2 -\adapt_2 ) = J_1 - \adapt_1 + \idx \times  J_2$.

    By induction hypothesis using $(\clubsuit)$ and the above statement, we know:
    $\valr, \env_3$ is typable with $\type_2$ and $\nnatA + J_1 - \adapt_1 +\idx \times  J_2 - \adapt_3$.

    Unfold $J_1, J_2$, we know 
    $ \nnatA + J_1 - \adapt_1 +\idx \times  J_2 - \adapt_3  = \nnatA + \nnatA_1 + F(\expr_1, \phi_1)  + \idx \times \nnatA_2 + \idx \times F(\expr_2, \phi_2)  - (\adapt_1 + \adapt_3) \leq  J- (\adapt_1+\adapt_3) $.

    The TS is proved by lemma~\ref{tmono} on the above statement.
    
   
   \end{proof}  

\clearpage
\section{Typable Approach by Marco}
\begin{defn}[Typable Closures]
  \label{def:typable}
Let $\env=[ x_1 \to (\valr_1 ,  \env_1 , R_1 ) ,
  \ldots, x_n \to (\valr_n, \env_n, R_n )]$. 
  The closure $( \expr, \env)$ is typable with
  type $\type$ and adaptivity $J$ if:
\begin{enumerate}\item  
     $x_1 : \bang{k_1} \type_1, \ldots, \bang{k_i} \type_i 
     \tvdash{Z}  \expr : \type$, for some types $\bang{k_i}
   \type_i$ for $(1\leq i\leq n)$, 
\item each closure $(\valr_i, \env_i)$ for $(1\leq i\leq n)$ is typable with type
  $\bang{k_i} \type_i$ and adaptivity $Z_i$,
\item $J = Z + \sum_{(v_i,\theta_i,S_i)\in\theta} k_i \times (R_i
  +Z_i)$.
\end{enumerate}
 \end{defn}
To justify why we chose $\sum$ in the third clause above it is worth
to consider the following configuration:
$$
[x\mapsto (\lambda u.\lambda w.\delta(u)+\delta(w),[\,],0)
,y:\mapsto (v,[\,],2) ], x\, y\, y
$$
\mg{To elaborate the example above}
   
\begin{lem}[Soundness]
  \label{tsound}
  If a closure $D$ is typable with type $\type$ and adaptivity $J$, and $D \bigstep{R} E$, then
    the closure $E$ is typable with type $\type$ and adaptivity $I$,
    where $I+R\leq J$. 
   \end{lem}

   \begin{proof}
     By induction on the derivation showing $D \bigstep{R} E$. The
     rules \textsf{lambda},\textsf{val}, and \textsf{const} dollow
     from the hypothesis. We will discuss the other cases. \\

     \caseL{Case
       \[
         \inferrule{  \env(x)  =  (\valr, \env',  R)  }
         {x,\env  \bigstep{R} \valr, \env' }~\textsf{var}
       \]
     }

     By assumption, the closure $(x,\env)$ is typable with type
     $\type$ and adaptivity $J$. Let us assume without loss of
     generality that
     $\env=[x\to (\valr , \env' , R ), x_1 \to (\valr_1 , \env_1 , R_1 ) , \ldots, x_n \to
     (\valr_n, \env_n, R_n )]$.  By Definition~\ref{def:typable} we have that
     there exist $k$ and $\bang{k_i} \type_i$ for $(1\leq i\leq n)$ such
     that:
     \[ x: \bang{k}\type, x_1 : \bang{k_1} \type_1, \ldots, x_n:\bang{k_n} \type_n \tvdash{Z} x : \type \]
    Moreover, we have that the closure $(\valr,\env')$ is typable with
    type $\type$ and adaptivity $I$, and that each closure $ (\valr_i, \env_i) $ is
    typable with type $\type_i$ and adaptivity $Z_i$, and that 
$$J = Z + k\times (R+I)+ \sum_{1\leq
  i\leq n, i\neq j} k_i \times (R_i
  +Z_i).$$
By inversion on the typing rules, we have that $k= 1$, hence
we can rewrite $J$ as 
$$J = Z + R+I+ \sum_{1\leq
  i\leq n, i\neq j} k_i \times (R_i
  +Z_i)).$$
Hence, it is easy to see that $ I+R\leq J$ which is
what we need to show.\\


     \caseL{Case
       \[
         \inferrule{
    \expr_1, \env \bigstep{R_1} \lambda x.\expr , \env^1 \\
    %\forall x_i \in \dom(\env_1 \cap \env_2).  \fresh \eapp x_i' \\
     \expr_2, \env \bigstep{R_2} \valr_2 , \env^2 \\
    \fresh \eapp x' \\
    \expr[x'/x], \env^1[ x'  \to (\valr_2, \env^2, R_2  ) ] 
    \bigstep{R_3} \valr, \env^3 ~(\clubsuit)
  }{
     \expr_1 \eapp \expr_2 , \env\bigstep{R_1+R_3} \valr, \env^3
  }~\textsf{app}
     \]
   }
   By assumption, the closure $(\expr_1 \eapp \expr_2 , \env)$ is
   typable with type $\tau$ and adaptivity $J$.  Let us assume without
   loss of generality that
   $\env=[x_1 \to (\valr_1 , \env_1 ,
   S_1 ) , \ldots, x_n \to (\valr_n, \env_n, S_n )]$.  By
   Definition~\ref{def:typable} we have that there exist 
   $\bang{k_i} \type_i$ for $(1\leq i\leq n)$ such that:
     \begin{equation}
\label{fact:app}
       x_1 : \bang{k_1} \type_1, \ldots, x_n:\bang{k_n} \type_n
       \tvdash{Z} \expr_1\expr_2 : \type 
\end{equation}
    Moreover, we have that each closure $ (\valr_i, \env_i)\in\theta $ is
    typable with type $\type_i$ and adaptivity $Z_i$, and that 
$$J = Z + \sum_{(v_i,\theta_i,S_i)\in\theta} k_i \times (S_i
  +Z_i).$$
By inversion on Fact~\ref{fact:app} we have that there exist
$\sigma,Z',Z^1, k$ and $k_i^1$ for $(1\leq i\leq n)$ such that:
\begin{equation}
\label{fact:head}
 x_1 : \bang{k_1^1} \type_1, \ldots, x_n:\bang{k_n^1} \type_n
       \tvdash{Z^1} \expr_1 : \bang{k} \sigma
      \lto{Z'} \type
\end{equation}
Similarly, 
we have that there exist
$Z^2$ and $k_i^2$ for $(1\leq i\leq n)$ such that:
\begin{equation}
\label{fact:arg}
 x_1 : \bang{k_1^2} \type_1, \ldots, x_n:\bang{k_n^2} \type_n
       \tvdash{Z^2} \expr_2 : \sigma
\end{equation}
Moreover, we have that $Z=Z^1+k\times Z^2+Z'$, and that
$k_i=k_i^1+k\times k_i^2$.  

Using  Fact~\ref{fact:head} and the assumption
that each closure $ (\valr_i, \env_i) $ in $\theta$ is typable with
type $\type_i$ and adaptivity $Z_i$ we have that the closure
$(e_1,\theta)$ is typable with type $\bang{k} \sigma
      \lto{Z'} \type$ and adaptivity:
$$
Z^1+\sum_{(v_i,\theta_i,S_i)\in\theta}k_i^1\times(S_i+Z_i)
$$
By induction hypothesis we then have that the closure 
$(\lambda x. e,\theta^1)$ is typable with type $\bang{k} \sigma
      \lto{Z'} \type$ and adaptivity $I_1$ where 
$$I_1+R_1\leq Z^1+\sum_{(v_i,\theta_i,S_i)\in\theta }k_i^1\times(S_i+Z_i)$$
It is worth to stress here that by definition, using the fact that 
$$
 x_1 : \bang{k_1^1} \type_1, \ldots, x_n:\bang{k_n^1} \type_n
       \tvdash{0} \lambda x. \expr : \bang{k} \sigma
      \lto{Z'} \type
$$
which follows from Fact~\ref{fact:head} and inversion on the rule {\bf
  lambda} \mg{here we actually want a lemma that says that values are typable
  with $\vdash_0$, and that takes care of the free variables that we may
  have added} we have:
$$
I_1=\sum_{(v_i,\theta_i,S_i)\in\theta^1}k_i^1\times(S_i+Z_i)
$$
%
Using  Fact~\ref{fact:arg} and again the assumption
that each closure $ (\valr_i, \env_i) $ in $\theta$ is typable with
type $\type_i$ and adaptivity $Z_i$ we have that the closure
$(e_2,\theta)$ is typable with type $\type$ and adaptivity:
$$
Z^2+\sum_{(v_i,\theta_i,S_i)\in\theta}k_i^2\times(S_i+Z_i)
$$

By induction hypothesis we than have that the closure 
$(v_2,\theta_2)$ is typable with type $\type$ and adaptivity $I_2$ where 
$$I_2+R_2\leq Z^2+\sum_{(v_i,\theta_i,S_i)\in\theta}k_i^2\times(S_i+Z_i)$$
%
It is again worth to stress here that by definition, using the fact
that $v_2$ is a value we have:
$$
I_2=\sum_{(v_i,\theta_i,S_i)\in\theta^2}k_i^2\times(S_i+Z_i)
$$
By Fact~\ref{fact:head}, and type preservation and some other Lemma
\mg{TODO: we need to be more clear and handle the fact that the adaptivity can be 0 in the
lambda rule} we have 
\begin{equation}
\label{fact:reduct}
 x_1 : \bang{k_1^1} \type_1, \ldots, x_n:\bang{k_n^1} \type_n,
       x: \bang{k} \sigma\tvdash{{Z'}} \expr : 
       \type
\end{equation}
By composing this with the previous assumptions, we have that the closure 
$(e[x'/x],\theta^1[x'\mapsto (v_2,\theta_2,R_2)])$ is typable with
type $\type$ and adaptivity $I_3$ where 
$$I_3=Z'+k\times(R_2+I_2)+\sum_{(v_i,\theta_i,S_i)\in\theta^1}k_i^1\times(S_i+Z_i)$$
By induction hypothesis we then get that the closure $(v,\theta^3)$ is
typable with type $\type$ and adaptivity $I$ where $I+R_3\leq
I_3$. Notice also that we have a $\theta^4$ such that
$\theta^3=\theta^4[x' \mapsto (v_2,\theta_2,R_2) ]$.
Moreover, once again, note that by Lemma~\mg{Lemma showing that we can
  type values with adaptivity 0} we have:
$$
I=\sum_{(v_i,\theta_i,S_i)\in\theta^3}k_i\times(S_i+Z_i)= k\times(R_2+I_2)+\sum_{(v_i,\theta_i,S_i)\in\theta^4}k_i\times(S_i+Z_i)
$$
We want to prove $I+R_1+R_3\leq J$.  

Starting from the left hand side we have 
$$
\begin{array}{rcl}
I+R_1+R_3&=&
             \sum_{(v_i,\theta_i,S_i)\in\theta^3}(k_i\times(S_i+Z_i)) +R_1+R_3\\
&=&
             k\times(R_2+I_2)+\sum_{(v_i,\theta_i,S_i)\in\theta^4}k_i\times(S_i+Z_i) +R_3+R_1\\
&\leq&
    k\times(R_2+I_2)+Z'+\sum_{(v_i,\theta_i,S_i)\in\theta^1}k_i^1\times(S_i+Z_i)
    +R_1\\
&\leq&
    k\times(R_2+I_2)+Z'+Z^1+\sum_{(v_i,\theta_i,S_i)\in\theta}k_i^1\times(S_i+Z_i)\\
&=&
    k\times(R_2+\sum_{(v_i,\theta_i,S_i)\in\theta^2}k_i^2\times(S_i+Z_i))\\
&&+Z'+Z^1+\sum_{(v_i,\theta_i,S_i)\in\theta}k_i^1\times(S_i+Z_i)\\
&\leq&
    k\times(Z^2+\sum_{(v_i,\theta_i,S_i)\in\theta}k_i^2\times(S_i+Z_i))\\
&&+Z'+Z^1+\sum_{(v_i,\theta_i,S_i)\in\theta}k_i^1\times(S_i+Z_i)\\
&=&
    k\times Z^2 +Z'+Z^1+\\
&&\sum_{(v_i,\theta_i,S_i)\in\theta}k\times k_i^2\times(S_i+Z_i)) +\sum_{(v_i,\theta_i,S_i)\in\theta}k_i^1\times(S_i+Z_i)\\
&=&
    k\times Z^2 +Z'+Z^1+\sum_{(v_i,\theta_i,S_i)\in\theta}(k_i^1
    +k\times k_i^2)\times(S_i+Z_i)) \\
&=&J\\
\end{array}
$$
This concludes this case.
     \caseL{Case
       \[
           \inferrule{
    \expr , \env \bigstep{\adapt} \valr , \env_1 \\
    \eop{}(\valr\env_1) = \valr'\\
    FV(\valr')=\emptyset
  }{
    \eop(\expr), \env \bigstep{\adapt +1} \valr',  \env_1
  }~\textsf{delta}
       \]
     }
   By assumption, the closure $(\eop(\expr) , \env)$ is
   typable with type $\tau$ and adaptivity $J$.  
Let us assume without
   loss of generality that
   $\env=[x_1 \to (\valr_1 , \env_1 ,
   S_1 ) , \ldots, x_n \to (\valr_n, \env_n, S_n )]$.  By
   Definition~\ref{def:typable} we have that there exist $Z$ and 
   $\bang{k_i} \type_i$ for $(1\leq i\leq n)$ such that:
     \begin{equation}
\label{fact:delta}
       x_1 : \bang{k_1} \type_1, \ldots, x_n:\bang{k_n} \type_n
\tvdash{Z} \eop(\expr) : \type 
\end{equation}
    Moreover, we have that each closure $ (\valr_i, \env_i)\in\theta $ is
    typable with type $\type_i$ and adaptivity $Z_i$, and that 
$$J = Z + \sum_{(v_i,\theta_i,S_i)\in\theta} k_i \times (S_i
  +Z_i).$$
By inversion on Fact~\ref{fact:delta} we have that $Z=Z'+1$ for some
$Z'$, that $\type=\tbase$, that $k_i= k_i'+1$ for $(1\leq i\leq n)$,
and that:
     \begin{equation}
\label{fact:delta}
       x_1 : \bang{k_1'} \type_1, \ldots, x_n:\bang{k_n'} \type_n
\tvdash{Z'} \expr : \tbase.
\end{equation}
Hence, it is easy to see that the closure
$(\expr, \env)$
is typable with type $\tbase$ and adaptivity 
$$Z' + \sum_{(v_i,\theta_i,S_i)\in\theta} k_i' \times (S_i
  +Z_i)$$
By applying the induction hypothesis we have that also 
the closure 
$(v, \env')$ is typable with type $\tbase$ and adaptivity $I'$ such
that:
$$
I'+R\leq Z' + \sum_{(v_i,\theta_i,S_i)\in\theta} k_i' \times (S_i
  +Z_i)
$$
From this and the fact that $\vdash_0 v' :\tbase$, we have:
$$
I'+R+1\leq Z'+1 + \sum_{(v_i,\theta_i,S_i)\in\theta} (k_i'+1) \times (S_i
  +Z_i)
$$
This councludes this case. 
   \end{proof}  

\end{document}



