
\section{Examples}


\begin{algorithm}
\caption{A two-round analyst strategy for random data (Algorithm 4 in ...)}
\label{alg:BitGOF}
\begin{algorithmic}
\REQUIRE Mechanism $\mathcal{M}$ with a hidden state $X\in \{-1,+1\}^{n\times (k+1)}$.
\STATE  {\bf for}\ $j\in [k]$\ {\bf do}.  
\STATE \qquad {\bf define} $q_j(x)=x(j)\cdot x(k)$ where $x\in \{-1,+1\}^{k+1}$.
\STATE \qquad {\bf let} $a_j=\mathcal{M}(q_j)$ 
\STATE \qquad \COMMENT{In the line above, $\mathcal{M}$ computes approx. the exp. value  of $q_j$ over $X$. So, $a_j\in [-1,+1]$.}
\STATE {\bf define} $q_{k+1}(x)=\mathrm{sign}\big (\sum_{i\in [k]} x(i)\times\ln\frac{1+a_i}{1-a_i} \big )$ where $x\in \{-1,+1\}^{k+1}$.
\STATE\COMMENT{In the line above,  $\mathrm{sign}(y)=\left \{ \begin{array}{lr} +1 & \mathrm{if}\ y\geq 0\\ -1 &\mathrm{otherwise} \end{array} \right . $.}
\STATE {\bf let} $a_{k+1}=\mathcal{M}(q_{k+1})$
\STATE\COMMENT{In the line above,  $\mathcal{M}$ computes approx. the exp. value  of $q_{k+1}$ over $X$. So, $a_{k+1}\in [-1,+1]$.}
\RETURN $a_{k+1}$.
\ENSURE $a_{k+1}\in [-1,+1]$
\end{algorithmic}
\end{algorithm}





\newpage

%%%%%%EXAMPLES---TWO ROUNDS%%%%%%%%%%%%%%%%%%%%%%%%%%%%%%%%%%%%%%%%%%%%%%
\newpage
\begin{figure}
\small
Two-rounds:

\[
\begin{array}{l}
  \elet \eapp  g: \bot = 
  \efix \eapp f(j: \tint). \lambda k: \tint. \lambda db: \tdb.\\
  \hspace{.2cm}  \eif \big (  (j < k)  ,  \\
  \hspace{.8cm}  \elet \eapp  a: 0 = 
                 \eop \eapp  
                 (\lambda x: \tlist {\tint}. 
                 (\mathsf{get} \eapp x \eapp j) \cdot (\mathsf{get} \eapp x \eapp k) ) 
                  \eapp \ein \\
  \hspace{1.2cm} (a, j) :: (f  \eapp (j+1) \eapp  k) \\
  \hspace{1.2cm} ,\eapp  [] \big) \eapp \ein\\
  \hspace{.2cm}  \efix \eapp \mathsf{twoRound}(k : \tint). 
                 \lambda db:\tlist {\tlist {\tint}}.\\
  \hspace{.8cm}  \elet \eapp  l: 1 = g \eapp  0 \eapp  k \eapp bd \eapp  \ein \\
  \hspace{.8cm}  \elet \eapp  q: 1 =  \lambda x: \tlist {\tint}. \mathsf{sign} \eapp \\ 
  \hspace{.8cm}  (\mathsf{foldl} \eapp  (\lambda acc: \treal. 
                 \lambda ai: \treal \cdot \tint. 
                 \big(acc\eapp + (\mathsf{get} \eapp x \eapp  (\mathsf{snd} \eapp ai)) 
                 \cdot \mathsf{log}(\frac{1+ (\mathsf{fst} \eapp ai) }{1 - (\mathsf{fst} \eapp ai)}) \big)
                 \eapp  0.0 \eapp  l )) \eapp  \ein \\
  \hspace{.8cm}  \eop ( q )
\end{array}
\]
\end{figure}


\begin{tabbing}
    $ x: \trow \equiv \tlist{\tint}$\\
    $ \tdb \equiv \tlist{\tlist{\tint}} $\\
    $ \cdot: \{ \treal | \tint \} * \{ \treal | \tint \} \to \{ \treal | \tint \}$\\
    $ g: \tint \to \tarr{\tint}{
    \tarr{\tdb}{\tlist{\treal * \tint}}{\bot}{}{1}  
    }{\bot}{}{0}
    $\\      
    $ q: \tbox{  (\tarr{ \trow }{ \treal }{0}{}{0})     } $\\
    $ \mathsf{foldl} : (\treal \to \tlist{(\treal * \tint)} \to \treal) \to \treal \to \tlist{\tint * \treal} \to \treal$\\
    $ \mathsf{sign} : \treal \to \tint $\\
    $ \mathsf{log} : \treal \to \treal $\\
    $ \mathsf{twoRound} : \tarr{\tint}{
    \tarr{\tdb}{\treal}{1}{}{2}
    }
    {\bot}{}{0} $
\end{tabbing}

\begin{figure}
\tiny
Type derivation:\\
    % $A = \tarr{ \tint }{ \tarr{\tint}{\tlist{\treal * \tint}}{0}{[j:0]}{1} }{\bot}{[]}{0} $, \\
    % $FL1 = \treal \to \tlist{\tint * \treal} \to \treal$, $FL2 = \treal \to \tlist{\treal * \tint} \to \treal$\\
    % $Q_2 = \tbox{  (\tarr{ \trow }{ \treal }{0}{[l:0]}{0})} $,\\
    % $Q_1 = \tbox{  (\tarr{ \trow }{ \treal }{0}{[j:0, k:0]}{0})}$\\
    % $\Gamma = f: A,  j: \tint, k: \tint$; $\Gamma' = \Gamma, a: \treal$; $\Gamma'' = \Gamma, x: \trow$,\\
    % $\Delta = g: A, l: \tlist{\treal * \tint}$; $\Delta' = \Delta, q:Q_2$; $\Delta'' = \Delta, x: \trow$\\

\[
  \inferrule*[ right = let ]
   {
     \inferrule*[ right = fix ]
     {
        \Pi_L \vartriangleright
     }
     {
        \tvdash{0} \efix \eapp  f(j). \cdots 
        : \tint \to 
        \tarr{\cdots}{\tlist{\treal * \tint}}{1}{}{1}
      }
     \and
     \inferrule*[ right = fix ]
     {
        \Pi_R \vartriangleright
     }
     {
      \cdot ; g ; [g : \bot] 
      \tvdash{0} \efix \mathsf{twoRound} \cdots 
      : \cdots \tarr{\tdb}{\treal}{1}{}{2}
    }
   }
   { 
      \tvdash{0} \elet g : \bot = \cdots \ein \efix \mathsf{twoRound} \cdots 
      :   
      \tarr{\tint}
      {
      \tarr{\tdb}{\treal}{1}{}{2}
      }
      {\bot}{}{0} }
\]

Derivation $\Pi_L$ and $\Pi_R$ are shown as follows:\\
$\Pi_L$:
\begin{mathpar}
      \inferrule*[right = fix]
      {
        \inferrule*[right = 2fix]
        {
          \inferrule*[right = if]
          {
            \inferrule[bool]
            {
              \empty
            }
            {
              \cdot; \Gamma_1; \dmap_1 \tvdash{0} {j<k : \tbool}
            }
            \and
            \inferrule*[right = let]
            {
              \dots
            }
            {
              \cdot; \Gamma_1; \dmap_1 \tvdash{1} \elet a = \cdots \ein \cdots : \tlist{\treal * \tint}
            }
            \and
            \inferrule[nil]
            {
              \empty
            }
            {
              \cdot; \Gamma_1; \dmap_1 \tvdash{1} [] : \tlist{\treal * \tint}
            }
          }
          {
            \cdot; \Gamma_1 = f, j, k, db ;  \dmap_1 = [f: \infty, j: \bot , k:\bot, db : \bot] 
            \tvdash{1} \eif \cdots : \tlist{\treal * \tint}
          }
        }
        {
          f, j : \tint; [f: \infty, j: \bot ] 
          \tvdash{0} \lambda k. \lambda db. \eif \cdots 
          : \cdots \tarr{\tdb}{\tlist{\treal * \tint}}{\bot}{}{1}
        }
      }
      {
        \tvdash{0} \efix \eapp  f(j). \cdots 
        : \cdots \tarr{\tdb}{\tlist{\treal * \tint}}{\bot}{}{1}  
      }    
    
    \inferrule*[right = let]
    {
    \inferrule
        {
         \inferrule
            {
                \Pi_{L1} \vartriangleright
            }
            {
              \cdot; \Gamma_1; \dmap_1 \tvdash{0} \lambda x. (x \eapp j)\cdot (x \eapp k) : \tbox{\trow \to \treal}
            }
        }
        {
            \cdot; \Gamma_1; \dmap_1 \tvdash{1} \eop ( \lambda x. (x \eapp j)\cdot (x \eapp k)) : \treal
        }
    \and 
    \inferrule
        {
        \inferrule
            {
              \inferrule[var]
              {
                \empty
              }
              {
                \cdot; \Gamma_2; \dmap_2 \tvdash{0} a: \treal
              }
              \and
              \inferrule[var]
              {
                \empty
              }
              {
                \cdot; \Gamma_2; \dmap_2 \tvdash{0} j : \tint
              }
            }
            {
              \cdot; \Gamma_2; \dmap_2 \tvdash{0} (a, j) : \treal * \tint
            }
            \and
        \inferrule[app]
            {
                \Pi_{L2} \vartriangleright
            }
            {
              \cdot; \Gamma_2; \dmap_2
              \tvdash{1} f \eapp  j+1 \eapp  k \eapp db : \tlist{\treal * \tint}
            }
        }
        {
          \cdot; \Gamma_1, a : \treal; \dmap_1[a: 0] \tvdash{1} 
          (a, j) ::(\cdots) : \tlist{\treal * \tint}
        }
    }
    {
      \cdot; \Gamma_1; \dmap_1  \tvdash{1} 
      \elet a = \eop (\lambda x. (x \eapp j)\cdot (x \eapp k)) \ein (a, j) ::(f \eapp  j+1 \eapp  k \eapp db): \tlist{\treal * \tint}
    }
    \end{mathpar}
\end{figure}


\begin{figure}
\tiny
$\Pi_{L2}$:
\begin{mathpar}
\inferrule*[right = app]
{
   \inferrule*[right = app]
    {
      \inferrule*[right = app]
      {
        \inferrule[ var]
        {
          \empty
        }
        {
          \cdot; \Gamma'; \dmap_5' \tvdash{0} f: \tint \to \cdots
        }
        \and
        \inferrule[var]
        {
          \empty
        }
        {
          \cdot; \Gamma_2; \dmap_2 \tvdash{0} j + 1: \tint
        }
      }
      {
        \Gamma_2; \dmap_2 
        \tvdash{0} f \eapp j+1 
        : \tarr{\tint}{\tdb \cdots}{\bot}{[j:0, f : 0]}{0}
      }
        \and
        \inferrule[var]
      {
        \empty
      }
      {
        \cdot; \Gamma_2; \dmap_2 \tvdash{0} k : \tint
      }
    }
    {
      \cdot;\Gamma_2; \dmap_2 
      \tvdash{0} f \eapp  j+1 \eapp  k 
      : \tarr{\tdb}{\tlist{\treal * \tint}}{\bot}{}{1}
    }
    \and
    \inferrule[var]
    {
      \empty
    }
    {
      \cdot; \Gamma_2; \dmap_2
      \tvdash{0} db : \tdb
    }
}
{
  \cdot; \Gamma_2; \dmap_2
  \tvdash{1} f \eapp  j+1 \eapp  k \eapp db : \tlist{\treal * \tint}
}
    \end{mathpar}
\end{figure}



\begin{figure}
\tiny
$\Pi_{L1}$:
\begin{mathpar}
\inferrule*[right = box]
{
  \inferrule*[right = fix]
  {
    \inferrule
    {
      \inferrule
      {
        \inferrule[var]
        {
          \empty
        }
        {
          \cdot; \Gamma_3; \dmap_3 \tvdash{0} x: \trow
        }
        \and
        \inferrule[var]
        {
          \empty
        }
        {
          \cdot; \Gamma_3; \dmap_3 \tvdash{0} j : \tint
        }
      }
      {
        \cdot; \Gamma_3; \dmap_3 \tvdash{0} (x \eapp j) : \treal
      }
      \and
       \inferrule
      {
        \inferrule[var]
        {
        \empty
        }
        {
          \cdot; \Gamma_3; \dmap_3 \tvdash{0} x: \trow
        }
        \and
        \inferrule[var]
        {
          \empty
        }
        {
          \cdot; \Gamma_3; \dmap_3 \tvdash{0} k : \tint
        }
      }
      {
        \cdot; \Gamma_3; \dmap_3 \tvdash{0} (x \eapp k) : \treal
      }
    }
    {    
      \cdot; \Gamma_1, x; \dmap_1[x : 0] 
      \tvdash{0} (x \eapp j) \cdot (x \eapp k) : \treal
    }  
  }
  {
    \cdot; \Gamma_1; \dmap_1 \tvdash{0} \lambda x. (x \eapp j)\cdot (x \eapp k) : \tarr{\trow}{\treal}{0}{[j:0,k:0]}{0}
  }
}
{
  \cdot; \Gamma_1; \dmap_1 \tvdash{0} \lambda x. (x \eapp j)\cdot (x \eapp k) : \tbox{\trow \to \treal}
}
\end{mathpar}
\end{figure}


\begin{figure}
\tiny
$\Pi_R$:
\begin{mathpar}
\inferrule*[right = 2 fix]
{
  \inferrule*[right = let]
  {
    \inferrule
    {
      \inferrule
      {
        \inferrule
        {
          \inferrule[var]
          {
            \empty
          }
          {
              \cdot; \Gamma_1'' ; \dmap_1'' \tvdash{0} {g : \tint \to \cdots }
          }
          \and
          \inferrule[base]
          {
            \empty
          }
          {
            \cdot; \Gamma_1'' ; \dmap_1'' \tvdash{0} 0 : \tint
          }
        }
        {
          \cdot; \Gamma_1'' ; \dmap_1'' \tvdash{0} {g \eapp  0  : \tarr{\tint}{ \cdots}{\bot}{}{0} }
        }
        \and
        \inferrule[var]
        {
          \empty
        }
        {
          \cdot; \Gamma_1'' ; \dmap_1'' \tvdash{0} k : \tint
        }
      }
      {
        \cdot; \Gamma_1'' ; \dmap_1'' \tvdash{0}  g \eapp  0 \eapp  k 
        : \tarr{\tdb}{\tlist{\treal * \tint}}{\bot}{}{1}
      }
      \and
      \inferrule[var]
      {
        \empty
      }
      {
        \cdot; \Gamma_1''; \dmap_1'' \tvdash{0} db : \tdb
      }
    }
    {
      \cdot; \Gamma_1''; \dmap_1'' \tvdash{1}  g \eapp  0 \eapp  k \eapp db : \tlist{\treal * \tint}
    }
    \and
    \cdots
  }
  {
    \cdot; \Gamma_1'' ; \dmap_1'' 
    \tvdash{2} \elet l = \cdots \ein \elet q = \cdots \ein \eop ( q ) : \treal 
  }
  \\
  \Gamma_1'' =  \mathsf{twoRound}, k, db, g
  \\
  \dmap_1'' = [\mathsf{twoRound} : \infty , g:\bot, k: \bot, db : 1]
}
{
  \cdot ; g; [g:\bot] 
  \tvdash{0} \efix \mathsf{twoRound} \cdots : \cdots \tarr{\tdb}{\treal}{1}{}{2}
}


\inferrule*[right = let]
{
  \inferrule*[right = {\color{red} box}]
    {
    \inferrule*[right = fix]
      {
        \inferrule
          {
            \cdots
          }
          {
              \cdot; \Gamma_2'', x; \dmap_2''[x: 0]
              \tvdash{0} {\mathsf{sign}(\dots) :  \treal}
          }          
      }
      {
        {\color{red}
        \cdot; \Gamma_2''; \dmap_2''
        \tvdash{0} {\lambda x. \dots : \tarr{\trow}{\treal}{0}{[l:0]}{0} }
        }
      }
    }
    {
      \cdot; \Gamma_2''; \dmap_2''
      \tvdash{0} {\lambda x. \dots : \tbox{\trow \to \treal}}
    }
  \and
  \inferrule
    {
      \inferrule[var]
        {
          \empty
        }
        {
          \cdot; \Gamma_3''; \dmap_3'' - 1 \tvdash{0} { q : \tbox{\trow \to \treal}}
        }
    }
    {
      \cdot; \Gamma_2'', q; \dmap_2''[q: 1] = \dmap_1''[q:1] \tvdash{1} {\eop (q) : \treal}
    }
}
{
  \cdot; \Gamma_1'', l; \dmap_1''[l: 1] \tvdash{1} \elet q: 1 = \cdots \ein \cdots : \treal
}

\inferrule
{
  \inferrule[var]
  {
    \empty
  }
  {
    \cdot; \Gamma_4''; \dmap_4'' \tvdash{0} \mathsf{sign} :  \treal \to \treal
  }
  \and
  \inferrule
  {
    \inferrule
    {
    \cdots
    }
    {
      \cdot; \Gamma_4''; \dmap_4''
      \tvdash{0} \mathsf{foldl}(\cdots) \eapp 0 : \tlist{\tint * \treal} \to \treal
    }
    \\
    \inferrule[var]
    {
      \empty
    }
    {
      \cdot; \Gamma_4''; \dmap_4'' \tvdash{0} l : \tlist{\tint * \treal}
    }
  }
  {
    \cdot; \Gamma_4''; \dmap_4'' \tvdash{0} \mathsf{foldl}(\cdots) \eapp 0 \eapp l : \treal 
  }
}
{
  \cdot; \Gamma_4''; \dmap_4'' \tvdash{0} \mathsf{sign}(\dots) :  \treal
}

\inferrule
{
  \inferrule
  {
    \inferrule[var]
    {
      \empty
    }
    {
      \cdot; \Gamma_4''; \dmap_4''\tvdash{0} \mathsf{foldl} : \cdots
    }
    \and
    \inferrule
    {
      \cdots
    }
    {
      \cdot; \Gamma_4''; \dmap_4'' \tvdash{0} \lambda acc. \cdots : \cdots
    }
  }
  {
    \cdot; \Gamma_4''; \dmap_4'' \tvdash{0} \mathsf{foldl}(\cdots) : \cdots 
  }~\textbf{app}
  \and
  \inferrule[base]
  {
    \empty
  }
  {
    \cdot; \Gamma_4''; \dmap_4'' \tvdash{0} 0 : \tint
  }
}
{
  \cdot; \Gamma_4''; \dmap_4'' \tvdash{0} \mathsf{foldl}(\cdots) \eapp 0 : \tlist{\tint * \treal} \to \treal
}~\textbf{app}

\inferrule
{
  \inferrule
  {
    \inferrule
    {
      \inferrule[var]
      {\empty
      }
      {
        \cdot; \Gamma_6''; \dmap_6'' \tvdash{0} acc: \treal
      }
      \and
      \inferrule
      {
        \inferrule
        {
          \inferrule[var]
          {
          \empty
          }
          {
            \cdot;\Gamma_6''; \dmap_6'' \tvdash{0} x : \trow
          }
          \\
          \inferrule[var]
          {
            \empty
          }
          {
            \cdot; \Gamma_6''; \dmap_6'' \tvdash{0} i : \tint
          }
        }
        {
          \cdot; \Gamma_6''; \dmap_6'' \tvdash{0} x \eapp i: \treal
        }
        \and
        \inferrule[var]
        {
          \empty
        }
        {
          \cdot; \Gamma_6''; \dmap_6'' \tvdash{0} \mathsf{lg}(\cdot): \treal
        }
      }
      {
        \cdot; \Gamma_6''; \dmap_6'' \tvdash{0} (x\eapp i) * \cdots: \treal
      }
    }
    {
      \cdot; \Gamma_5'', ai; \dmap_5''[ai:0] \tvdash{0} acc + \cdots : \treal
    }
  }
  {
    \cdot; \Gamma_4'', acc; \dmap_4''[acc : 0] \tvdash{0} \lambda ai. \cdots :  \to \treal
  }~\textbf{fix}
}
{
  \cdot; \Gamma_4''; \dmap_4'' \tvdash{0} \lambda acc. \cdots : \cdots
}~\textbf{fix}
\end{mathpar}
\vspace{-0.5cm}
\caption{Type Derivation of Two Round Example}
\end{figure}


\clearpage

\begin{algorithm}
\footnotesize
\caption{A multi-round analyst strategy for random data (Algorithm 5 in ...)}
\label{alg:multiRound}
\begin{algorithmic}
\REQUIRE Mechanism $\mathcal{M}$ with a hidden state $X\in [N]^{n}$ sampled u.a.r., control set size $c$
\STATE Define control dataset $C = \{0,1, \cdots, c - 1\}$
\STATE Initialize $Nscore(i) = 0$ for $i \in [N]$, $I = \emptyset$ and $Cscore(C(i)) = 0$ for $i \in [c]$
\STATE  {\bf for}\ $j\in [k]$\ {\bf do} 
\STATE \qquad {\bf let} $p=\uniform(0,1)$ 
\STATE \qquad {\bf define} $q (x) = \bernoulli ( p )$ .
\STATE \qquad {\bf define} $qc (x) = \bernoulli ( p )$ .
\STATE \qquad {\bf let} $a = \mathcal{M}(q)$ 
\STATE \qquad {\bf for}\ $i \in [N]$\ {\bf do}
\STATE \qquad \qquad $Nscore(i) = Nscore(i) + (a - p)*(q (i) - p)$ if $i \notin I$
\STATE \qquad {\bf for}\ $i \in [c]$\ {\bf do}
\STATE \qquad \qquad $Cscore(C(i)) = Cscore(C(i)) + (a - p)*(qc (i) - p)$
\STATE \qquad {\bf let} $I = \{i | i\in [N] \land Nscore(i) > \max(Cscore)\}$
\STATE \qquad {\bf let} $X = X \setminus I$ 
\RETURN $X$.
% \ENSURE 
\end{algorithmic}
\end{algorithm}


\clearpage

\begin{figure}
\small
\[
\begin{array}{l}
\elet \mathsf{updtSC} =\\
                 \efix \eapp  \mathsf{f}(z: \tunit). \lambda sc: \tlist{\treal}. 
                 \lambda a: \treal. \lambda p: \treal. \lambda q: \tbox{\tint \to \tint}.\\
                 \lambda I: \tlist{\int}. \lambda i: \tint. \lambda n: \tint. \\
 \hspace{.2cm}   \eif \Big ( (i < n)  ,  \\
 \hspace{0.4cm}  \eif \big( ( \mathsf{in} \eapp i \eapp I  ) ,       \\
 \hspace{0.8cm}  \elet \eapp x: 0 =( \mathsf{get} \eapp sc \eapp i) 
                 + (a-p)*(q \eapp i - p)  \ein \\
 \hspace{0.8cm}  \elet \eapp sc': 0 =  \mathsf{updt} \eapp sc \eapp i
                 \eapp x \ein \\
 \hspace{1.2cm}  \mathsf{f}  \eapp () \eapp sc' \eapp a \eapp p
                 \eapp q \eapp I \eapp  \eapp (i+1) \eapp n  \\ 
 \hspace{0.4cm}  , \mathsf{f}  \eapp () \eapp sc \eapp a \eapp p
                 \eapp q \eapp I \eapp  \eapp (i+1) \eapp n \big )  \\ 
 \hspace{0.2cm}  , sc  \Big ) \eapp \ein
\end{array}
\]
%
%
\[
\begin{array}{l}
\elet \mathsf{updtSCC} = \\
                \efix \eapp  \mathsf{f}(z: \tunit). \lambda scc: \tlist{\treal}. \lambda a: \treal. 
                \lambda p: \treal. \lambda qc: \tbox{\tint \to \tint}.\\ 
                \lambda i: \tint. \lambda cr: \tint. \\
 \hspace{.2cm}  \eif \Big ( (i < cr) ,  \\
 \hspace{0.8cm} \elet \eapp x: 0 =( \mathsf{nth} \eapp scc \eapp i) 
                + (a-p)*(qc \eapp i - p)  \ein \\
 \hspace{0.8cm} \elet \eapp scc': 0 =  \mathsf{updt} \eapp scc \eapp i
                \eapp x \ein \\
 \hspace{1.2cm} \mathsf{f}  \eapp () \eapp scc' \eapp a \eapp p \eapp qc
                \eapp (i+1) \eapp  cr  \\ 
 \hspace{0.2cm} , scc  \Big ) \eapp \ein
\end{array}
\]
%
%
\[
\begin{array}{l}
\elet \mathsf{updtI} = \\
                 \efix \eapp  \mathsf{f}(z : \tunit). \lambda maxScc: \treal. 
                 \lambda sc: \tlist{\treal}. \lambda i: \tint. \lambda n: \tint. \\
 \hspace{.2cm}   \eif \Big (   (i < n)  ,  \\
 \hspace{0.4cm}  \eif \big( ( ( \mathsf{nth} \eapp scc \eapp i)  >  maxScc  ) ,       \\
 \hspace{0.8cm}  i :: ( \mathsf{f}  \eapp () \eapp maxScc \eapp sc
                 \eapp (i+1) \eapp n  )\\
 \hspace{0.8cm}  \mathsf{f}  \eapp () \eapp maxScc \eapp sc
                 \eapp (i+1) \eapp n  \big )  \\
 \hspace{0.2cm}  , [] \Big ) \eapp \ein
\end{array}
\]
%
%
\[
\begin{array}{l}
 \efix \eapp  \mathsf{multiRound}(z : \tunit). \Lambda k. \Lambda j. 
        \lambda k: \tint[k]. \lambda j: \tint[j]. \lambda sc: \tlist{\treal}.\\
        \lambda scc: \tlist{\treal}. \lambda il: \tlist{\tint}. \lambda n: \tint.
        \lambda cr: \tint. \lambda db: \tlist{\tint}.\\
 \hspace{.2cm}  \eif   \big (   (j < k)  ,  \\
 \hspace{.2cm}  \elet \eapp p: k - j = \uniform \eapp 0 \eapp 1 \ein \\
 \hspace{0.4cm} \elet \eapp q: k - j = \lambda x. \bernoulli \eapp p \ein \\
 \hspace{0.4cm} \elet \eapp qc: k - j = \lambda c. \bernoulli \eapp p \ein \\
 \hspace{0.4cm} \elet \eapp qj : k - j = \mathsf{restrict} \eapp q \eapp db\\
 \hspace{0.4cm} \elet \eapp a: k - j - 1 = \eop (q)  \ein \\
 \hspace{0.8cm} \elet \eapp sc': k - j - 1 =  \mathsf{updtSC} \eapp () \eapp sc  \eapp a \eapp p
                \eapp q \eapp il \eapp  \eapp 0 \eapp  n \eapp  \ein \\
 \hspace{0.8cm} \elet \eapp scc': k - j - 1 =  \mathsf{updtSCC} \eapp () \eapp scc \eapp a \eapp p
                \eapp qc \eapp  \eapp 0 \eapp  cr \ein \\
 \hspace{0.8cm} \elet \eapp maxScc: k - j - 1 =  \mathsf{foldl} \eapp 
                (\lambda acc : \treal. \lambda a: \treal. 
                \eif ( acc < a, a, acc)) \eapp 0 \eapp scc' \ein \\
 \hspace{0.8cm} \elet \eapp il': k - j - 1 =  \mathsf{updtI}  \eapp () \eapp maxScc \eapp sc
                \eapp 0 \eapp n  \ein \\
 \hspace{0.8cm} \elet \eapp db': k - j - 1 =  db \setminus il' \ein \\
 \hspace{1.2cm} a :: (\mathsf{multiRound} \eapp () \eapp [k] \eapp [j + 1]  
                 \eapp k \eapp (j+1) \eapp sc' \eapp scc' \eapp il'
                 \eapp n \eapp cr \eapp db')\\ 
 \hspace{0.2cm} , []  \big)
\end{array}
\]

\end{figure}


Type Declaration for Multi-rounds example:
\[
\begin{array}{ll} 
  \mathsf{multiRound} 
  :
    &
    \tarr{\tunit}
    {
      \forall j:: S. \forall k:: S.
      \tarr{\tint[k]}
      {
        \tarr{\tint[j]}
        {
          \empty
        }{\bot}{}{0}
      }{\bot}{}{0}
    }{\bot}{}{0} \\
    &
    \tarr{\tlist{\treal}}
    {
      \tarr{\tlist{\treal}}
      {
        \tarr{\tlist{\tint}}
        {
          \tarr{\tint}
          {
            \empty
          }{\bot}{}{0}
        }{\bot}{}{0} 
      }{\bot}{}{0}
    }{\bot}{}{0}\\
    &
    \tarr{\tint}
    {
      \tarr{\tlist{\treal}}
      {
        \treal
      }{k - j}{}{k - j}
    }{\bot}{}{0}\\

  \mathsf{updtSC} : 
    & \tunit \to \tlist{\treal} \to \treal \to \treal \to \tbox{ \tint \to \tint } \\       
    & \to \tlist{\tint} \to \tint[j] \to \tarr{\tint}{\tlist{\treal}}{0}{}{0}\\
  \mathsf{updtSCC} : 
    & \tunit \to \tlist{\treal} \to \treal \to \treal \to \tbox{ \tint \to \tint } \\
    & \to \tint[j] \to \tarr{\tint}{\tlist{\treal}}{0}{}{0}\\
  \mathsf{updtI} : & \tunit \to \treal \to \tlist{\treal} \to \tint[j] \to \tarr{\tint}{\tlist{\tint}}{0}{}{0}\\
  \mathsf{foldl} : & (\treal \to \treal \to \treal) \to \treal \to \tlist{\treal}\\
  q :              & \tbox{\tint \to \tint}\\
  qc :             & \tbox{\tint \to \tint}\\
\end{array}
\]
$\to$ without annotations is equivalent to $\tarr{}{}{\bot}{\dmap_{\bot}}{0}$ in types of $\mathsf{multiRound}, \mathsf{updtSC}, \mathsf{updtSCC}$ and $ \mathsf{updtI} $ functions, where $\forall x, \dmap_{\bot}(x) = \bot$.

$\dmap^2 = $

The \emph{depth map} in the type derivation of Multi-rounds example:
\[
\begin{array}{lll}
\dmap^1     & = & [\mathsf{updtSC}: \bot, \mathsf{updtSCC}: \bot, \mathsf{updtI}: \bot] \\
%
\dmap_0   & = & [\mathsf{updtSC}: \infty, \mathsf{updtSC}: \bot, 
                  \mathsf{updtSCC}: \bot, \mathsf{updtI}: \bot, 
                  z: \bot, k : \bot, j : \bot, sc: \bot,\\
            &   & scc: \bot, il : \bot, n : \bot, cr: \bot] \\
%
\dmap_2   & = & \dmap_1[p : k - j]\\
\dmap_3   & = & \dmap_2[q : k - j]\\
\dmap_4   & = & \dmap_3[qc : 0]\\
\dmap_5   & = & \dmap_4[a : k - j - 1]\\
\dmap_6   & = & \dmap_5\\
\dmap_7   & = & \dmap_5[sc' : k - j - 1, scc' : k - j - 1, il' ; k - j - 1]\\
\dmap_8   & = & \dmap_7[db' : k - j - 1]\\
\end{array}
\]

$\Delta = [J : S, K : S]$


\begin{figure}
\tiny
Type Derivation:\\
UpdtSC:
\begin{mathpar}
      \inferrule*[right = fix]
      {
        \inferrule*[right = fix * 4]
        {
          \inferrule*[right = if]
          {
            \inferrule*[right = bool]
            {
              \empty
            }
            {
             i:\tint , il : \tlist{\tint} , \Gamma \tvdash{0} {i
               <  N : \tbool}
            }
            \\
            \inferrule*[right = if]
            {
              \dots
            }
            {
              f:., \Gamma; \tvdash{0}  \eif (i \in il)\cdots : \tlist{\treal }
            }
            \and
            \inferrule[var]
            {
            }
            {
              f, sc ,\Gamma \tvdash{0} {sc : \tlist{\treal}}
            }
            \\
            \and
          }
          {
          f: ., sc: \tlist{\treal}, a:\tint, i:\tint \dots \Gamma
            \tvdash{0} \eif (i < n)  \cdots : \tlist{\treal}
          }
        }
        {
          f: \tunit \rightarrow \dots
          \tlist{\treal}, \Gamma \tvdash{0} {\lambda
            sc. \dots \lambda n.
            \eif \cdots :  \tlist{\treal} \rightarrow \dots \tlist{\treal}   }
        }
      }
      {
       \Gamma \tvdash{0} \efix \eapp  f( \_ ). \lambda sc. \lambda
        a. \dots \lambda n. \eif \Big ( (i < n), \dots, sc \Big ) : \tunit
        \rightarrow \tlist{\treal} \rightarrow \dots \rightarrow \tlist{\treal} 
      }

   \inferrule*[ right = if ]
   {
     \inferrule
     {
     }
     {
       \Gamma \tvdash{0}. f \eapp () \eapp sc' \dots \eapp N : \tlist{\treal}
      }
     \and
     \inferrule
     {
     \dots
     }
     {
      \Gamma\tvdash{0}  \elet x = \dots \ein \elet sc' = \dots
      \ein f \eapp () \eapp sc' \dots : \tlist{\treal}
    }
     \\
     i :\tint , I : \tlist{\tint},\Gamma \tvdash{0} i \in I : \tbool
   }
   { \Gamma \tvdash{0}  \eif \big(  (i \in il), \elet x = \dots  ,  f \eapp ()
     \eapp sc \dots N \big) : \tlist{\treal }    }

    
    \inferrule*[right = let]
    {
    \inferrule*[right = let]
        {
        \inferrule*[right = app]
            {
            }
            {
                \Gamma \tvdash{0} \mathsf{updt} \eapp sc \eapp
                i \eapp x : \tlist{\treal}
            }
            \and
        \inferrule*[right = app]
            {
                \dots
            }
            {
                \Gamma \tvdash{0} f () \eapp sc \dots \eapp  (i+1) \eapp  N : \tlist{\treal }
            }
            \\
            {  }
            \and
            {}
        }
        {
            x: \treal ,\Gamma \tvdash{0} 
             \elet sc' = \mathsf{updt} \eapp sc \eapp i \eapp x \ein \dots : \tlist{\treal}
        }
      \\
      \inferrule
        {
          \dots
        }
        {
            \Gamma \tvdash{0} ( \mathsf{nth} \eapp sc \eapp i  )+ (a-p)*(q \eapp i -
      p) : \treal
        }
    }
    {
      \tvdash{0} 
      \elet x =( \mathsf{nth} \eapp sc \eapp i  )+ (a-p)*(q \eapp i -
      p) \ein \dots: \tlist{\treal }
    }
    \end{mathpar}
\end{figure}



\begin{figure}
\tiny
Multi-rounds:
\begin{mathpar}
\inferrule*[right = fix]
{
  \inferrule
  {
    \inferrule*[right = fix*7]
    {
      \inferrule*[right = if]
      {
        \inferrule[bool]
        {
          \empty
        }
        {
          \Delta;  \Gamma_0; \dmap_0 \tvdash{0} j<k : \tbool
        }
        \\
        \inferrule[]
        {
          \cdots
        }
        {
          \Delta;  \Gamma_0; \dmap_0  \tvdash{k - j} \elet p = \cdots \ein \cdots : \tlist{\treal}
        }
        \and
        \inferrule[nil]
        {
          \empty
        }
        {
          \Delta;  \Gamma_0; \dmap_0  \tvdash{0} [] : \tlist{\treal}
        }
      }
      {
        \Delta; \Gamma_0; \dmap_0 
        \tvdash{k - j} \eif(j<k, \elet p = \cdots \ein \cdots, []) 
        : \tlist{\treal}
      }
    }
    {
    \Delta; \Gamma^2; \dmap^2 \tvdash{0} \lambda k. \cdots : \tint[k] \cdots
    }
  }
  {
  \Delta; \Gamma^1 ,\mathsf{multiRound}, z; 
  \dmap^1[\mathsf{multiRound}: \infty, z : \bot]
  \tvdash{0} \Lambda. \Lambda. \cdots : \forall. \forall. \cdots  
  }
}
{
  \Delta; \Gamma^1; \dmap^1 
  \tvdash{0} \efix \, \mathsf{multiRound} (z : \tunit) \cdots : \tarr{\tunit}{\cdots}{\bot}{\dmap^1}{0}
}

\inferrule*[right = let]
{
  \inferrule
  {
    \inferrule[base]
    {
    \empty
    }
    {
    \tvdash{0} 0: \treal    
    }
    \and
    \inferrule[base]
    {
      \empty
    }
    {
      \tvdash{0} 1: \treal
    }
  }
  {  
     \Delta; \Gamma_0; \dmap_0  \tvdash{0} \uniform \eapp 0 \eapp 1 : \treal
  }~\textbf{uniform}  
  \and
  \inferrule*[right = let]
  {
    \cdots
  }
  {
    \Delta; \Gamma_0, p : \treal; \dmap_0[p: k - j]
    \tvdash{k - j} \elet q = \lambda x. \cdots \ein \cdots : \tlist{\tint}
  }
}
{
  \Delta; \Gamma_0; \dmap_0  \tvdash{k - j} \elet p = \cdots \ein \cdots : \tlist{\treal}
}

\inferrule*[right = LET]
{
  \inferrule
  {
    \inferrule
    {
      \inferrule
      {
        \inferrule[var]
        {
          \empty
        }
        {
          \cdot; \Gamma_1, x: \tint; \dmap_1[x: 0]
          \tvdash{0} p : \treal        
        }
      }
      {
        \cdot; \Gamma_1, x: \tint; \dmap_1[x: 0]
        \tvdash{0} \bernoulli \eapp p : \tint
      }
    }
    { 
     \cdot;  \Gamma_1; \dmap_1 \tvdash{0} \lambda x. \bernoulli \eapp p : \tint \to \tint
    }
  }
  {
    \cdot;  \Gamma_1; \dmap_1 \tvdash{0} \lambda x. \bernoulli \eapp p : \tbox{\cdots}
  }
  \and
  \inferrule*[right = let]
  {
    \inferrule*[right = let * 4]
    {
      \inferrule*[right = let]
      {
        \inferrule
        {
          \inferrule[var]
          {
            \empty
          }
          {
            \Gamma_{3}; \dmap_3  
            \tvdash{0} q : \tbox{\tint \to \tint}
          }
        }
        { 
          \Gamma_{3}; \dmap_3 
          \tvdash{1} \eop (q): \treal
        }~\textbf{$\delta$}
        \and
        \Pi_1  \vartriangleright
      }
      {
        \Delta; \Gamma_3; \dmap_3 \tvdash{k - j} \elet a = \eop ( q ) \ein \cdots : \tlist{\tint}
      }
    }
    {
      \cdots
    }
  }
  {
    \Delta; \Gamma_1, q: \tbox{\cdots}; \dmap_1[q: k - j] \tvdash{k - j} \elet qc = \lambda x. \cdots : \tlist{\tint}
  }
}
{
  \Delta;  \Gamma_1; \dmap_1 \tvdash{k - j} \elet q = \lambda x. \bernoulli \eapp p \ein \cdots : \tlist{\treal}
}
\end{mathpar}

$\Pi_1:$
\begin{mathpar}
\inferrule*[right = let]
{ 
  \inferrule*[right = app]
  {
    \inferrule*[right = app * 4]
    {
      \inferrule[var]
      {
        \empty
      }
      {
        \cdot; \Gamma_4;\dmap_4 \tvdash{0} \mathsf{updtSC} : \cdots
      }
      \\
      \inferrule[base]
      {
        \empty
      }
      {
        \cdot; \Gamma_4; \dmap_4 \tvdash{0} () : \tunit
      }
    }
    {
      \cdots
    }
  }
  {
    \Gamma_4; \dmap_4 \tvdash{0} \mathsf{updtSC} \eapp () \eapp sc \cdots : \tlist{\treal}
  }
  \and
  \inferrule*[right = let]
  {
    \inferrule*[right = let * 4]
    {
      \Pi_2 \vartriangleright
    }
    {
      \Delta; \Gamma_5; \dmap_5 \tvdash{k - j - 1} \elet db' = \cdots \ein \cdots : \tlist{\tint}
    }
  }
  {
    \cdots
  }
}
{
  \Delta; \Gamma_3, a: \treal; \dmap_3[a: k - j - 1] \tvdash{k - j - 1} \elet sc' = \mathsf{updtSC} \cdots \ein \cdots : \tlist{\treal}
}
\end{mathpar}


$\Pi_2$:
\begin{mathpar}
\inferrule*[right = let]
  {
    \inferrule
    {
      \inferrule[var]
      {
        \empty
      }
      {
        \cdot; \Gamma_5; \dmap_5 \tvdash{0} db: \tlist{\tint}
      }
      \and
      \inferrule[var]
      {
        \empty
      }
      {
        \cdot; \Gamma_5; \dmap_5 \tvdash{0} il': \tlist{\tint}
      }
    }
    {
      \cdot; \Gamma_5; \dmap_5 \tvdash{0} d \setminus il' : \tlist{\tint}
    }
    \and
    \inferrule*[right = cons]
    {
      \inferrule[var]
      {
        \empty
      }
      {
        \Delta; \Gamma_6; \dmap_6 
        \tvdash{0} a : \treal
      }
      \inferrule
      {
        \cdots
      }
      {
        \Delta; \Gamma_6; \dmap_6
        \tvdash{k - j - 1} \mathsf{multiRound} \cdots : \tlist{\treal}
      }
    }
    {
      \Delta; \Gamma_5,db; \dmap_5[db: k - j - 1]
      \tvdash{k - j - 1} a::(\mathsf{multiRound} \cdots) : \tlist{\treal}
    }
  }
  {
    \Delta; \Gamma_5; \dmap_5
    \tvdash{k - j - 1} \elet db' = \cdots \ein \cdots : \tlist{\treal}
  }
\end{mathpar}
\end{figure}

\begin{figure}
\tiny
\begin{mathpar}
    \inferrule
{
  \inferrule
  {  
    \inferrule
    {
      \inferrule
      {
        \Pi_3 \vartriangleright
      }
      {
        \Delta; \Gamma_6; \dmap_6
        \tvdash{0} \mathsf{multiRound} \eapp () \eapp [k] \eapp [j + 1] : \tint[k] \to \tint[j + 1] \to \cdots
      }
    }
    {
      \cdots
    }
  }
  {
    \Delta; \Gamma_6; \dmap_6 
    \tvdash{0} \mathsf{multiRound} \eapp () \eapp [k] [j + 1] \cdot : \tarr{\cdot}{\cdot}{0}{\dmap}{k - (j + 1)}
  }~{\textbf{app}}
  \and
  \inferrule[var]
  {
  \empty
  }
  {
    \Delta; \Gamma_6; \dmap_6
    \tvdash{0} db' : \tlist{\tint}
  }
}
{
  \Delta; \Gamma_6; \dmap_6
  \tvdash{k - j - 1} \mathsf{multiRound} \eapp () \cdots : \tlist{\tint}
}~{\textbf{app}}
\end{mathpar}

$\Pi_3$:
\begin{mathpar}
\inferrule
{
  \inferrule
  {
    \inferrule
    {
      \inferrule[var]
      {
        \empty
      }
      {
        \Delta; \Gamma_6; \dmap_6 \tvdash{0} \mathsf{multiRound} : \cdots
      }
      \and
      \inferrule[base]
      {
        \empty
      }
      {
        \Delta; \Gamma_6; \dmap_6 \tvdash{0} () : \tunit
      }
    }
    {
      \Delta;\Gamma_6; \dmap_6
      \tvdash{0} \mathsf{multiRound} \eapp (): \eilam  \eilam  \tint[k] \to \tarr{\cdots}{\cdots}{0}{\dmap}{k - j}
    }
    \and
    \inferrule
    {
      \Delta(k) = S
    }
    {
      \Delta \tvdash{} k :: S
    }
  }
  {
    \Delta; \Gamma_6; \dmap_6
    \tvdash{0} \mathsf{multiRound} \eapp () \eapp [k] : \eilam \tint[k] \to \tint[j] \to \tarr{\cdots}{\cdots}{0}{\dmap}{k - j}
  }
  \and
  \inferrule
  {
    \Delta(j) = S
  }
  {
    \Delta \tvdash{} j :: S
  }
}
{
  \Delta; \Gamma_6; \dmap_6
  \tvdash{0} \mathsf{multiRound} \eapp () \eapp [k ] \eapp [j + 1] : \tint[k] \to \tint[j + 1] \to \cdots \tarr{\cdots}{\cdots}{0}{\dmap}{k - (j + 1)}
}

\end{mathpar}

\end{figure}








