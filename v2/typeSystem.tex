We build a type system that statically approximates both $\adap$ and
$\ddep{x}$ for every $x$ in the context.

A \emph{depth map}, denoted $\dmap$, is a map from variables to
$\natbi$.
%
%% We define the support of a depth map, $\supp(\dmap) \defeq \{x ~|~
%% \dmap(x) \neq \bot\}$. We are only interested in depth maps with
%% finite support. Such a depth map can be written $x_1: \nnatbiA_1,
%% \ldots, x_k: \nnatbiA_k$.
%
Our typing judgment takes the form $\Gamma; \dmap \tvdash{\nnatA}
\expr: \type$, where $\Gamma$ is a typing context, $\dmap$ is a depth
map, and $\nnatA \in \nat$. The idea is that $\dmap$ gives an upper
bound on the depth of each free variable of $\expr$. Obviously, we
only care about the values of $\dmap$ on the domain of $\Gamma$ (at
the remaining points, $\dmap$ can be $\bot$; such a $\dmap$ can always
be finitely represented as $x_1: \nnatbiA_1, \ldots, x_k: \nnatbiA_k$,
where $x_1,\ldots,x_k$ are bound by $\Gamma$). $\nnatA$ is an upper
bound on the adaptivity of $\expr$.

\paragraph{Types}
Types $\type$ are simple, except that the function type is annotated
with the adaptivity $\nnatA$ of the function body, a depth-map $\dmap$
which gives the depths of all free variables of the body (i.e.,
variables bound in outer scopes), and a depth $\nnatbiA \in \natbi$ of
the argument variable in the body. The function name, although free in
the body, always has depth $\infty$, so we don't write its depth
explicitly.\footnote{The function name is never substituted by a value
  with a nontrivial trace, so we can choose any depth for it. $\infty$
  is the most permissive depth, as will become clear from the typing
  rules shortly.}

\[
\begin{array}{llll}
\mbox{Index Term} & \idx, \nnatA & ::= &     i ~|~ n ~|~ \idx_1 + \idx_2 ~|~  \idx_1
                                 - \idx_2 ~|~ \smax{\idx_1}{\idx_2}\\
  \mbox{Sort} & S & ::= & \nat \\
  \mbox{Type} & \type & ::= & \tbase ~|~ \tbool ~|~ \type_1 \times
  \type_2 ~|~ \tarr{\type_1}{\type_2}{\nnatbiA}{\dmap}{\nnatA} ~|~
                              \tlist{\type} ~|~ \tbox{\type} ~|~ \\
    & & & \wq{   \treal ~|~ \tint ~|~   \tint[I]  ~|~      \tforall{\dmap}{\nnatA}{i} \type  }
\end{array}
\]

%% Note that the type $\tarr{\type_1}{\type_2}{\nnatbiA}{\dmap}{\nnatA}$
%% is well-formed only in a context $\Gamma$ that binds all variables in
%% the domain of $\dmap$. 

\paragraph{Typing rules}
The typing rules are shown in Figure~\ref{fig:type-rules}. In these
rules, we write $\nnatbiA + \dmap$ or $\dmap + \nnatbiA$ for the depth
map $\lambda x.\eapp  (\nnatbiA + \dmap(x))$ defined on the same domain as
$\dmap$. We also lift $\max$ and $+$ pointwise to depth maps.
%% and define the constant map $\dmapb$ that maps everything to
%% $\bot$.
These are also formally defined in Figure~\ref{fig:type-rules}.

\begin{figure}
  \begin{mathpar}
    \inferrule{
      \Gamma(x) = \type \\ 0 \leq \dmap(x) \mbox{ or equiv.\ } \dmap(x) \neq \bot
    }{
      \Delta; \Gamma; \dmap \tvdash{\nnatA} x: \type
    }~\textbf{var}
    %
    \and
    %
    \inferrule{
      \Delta; \Gamma; \dmap_1 \tvdash{\nnatA_1} \expr_1: (\tarr{\type_1}{\type_2}{\nnatbiA}{\dmap}{\nnatA}) \\
      \Delta; \Gamma; \dmap_2 \tvdash{\nnatA_2} \expr_2: \type_1 \\\\
      \nnatA' = \nnatA_1 + \max(\nnatA, \nnatA_2 + \nnatbiA) \\
      \dmap' = \max(\dmap_1, \nnatA_1 + \max(\dmap, \dmap_2 + \nnatbiA))
    }{
      \Delta; \Gamma; \dmap' \tvdash{\nnatA'} \expr_1 \eapp \expr_2 : \type_2
    }~\textbf{app}
    %
    \and
    %
    \inferrule{
      \Delta; \Gamma, f: (\tarr{\type_1}{\type_2}{\nnatbiA}{\dmap}{\nnatA}), x: \type_1;
      \dmap[f: \infty, x: \nnatbiA]
      \tvdash{\nnatA}
      \expr: \type_2
    }{
      \Delta; \Gamma; \dmap' \tvdash{\nnatA'} \efix f(x:
      \type_1).\expr: \tarr{\type_1}{( \boxed{ \type_2 \setminus x } ) }{\nnatbiA}{\dmap}{\nnatA}
    }~\textbf{fix}
    %
    \and
    %
    \inferrule{
      \Delta; \Gamma; \dmap_1 \tvdash{\nnatA_1} \expr_1: \type_1 \\
      \Delta; \Gamma; \dmap_2 \tvdash{\nnatA_2} \expr_2: \type_2 \\\\
      \dmap' = \max(\dmap_1,\dmap_2) \\
      \nnatA' = \max(\nnatA_1,\nnatA_2)
    }{
      \Delta; \Gamma; \dmap' \tvdash{\nnatA'} (\expr_1, \expr_2): \type_1 \times \type_2
    }~\textbf{pair}
    %
    \and
    %
    \inferrule{
      \Delta; \Gamma; \dmap \tvdash{\nnatA} \expr: \type_1 \times \type_2
    }{
      \Delta; \Gamma; \dmap \tvdash{\nnatA} \eprojl(\expr): \type_1
    }~\textbf{fst}
    %
    \and
    %
    \inferrule{
      \Delta; \Gamma; \dmap \tvdash{\nnatA} \expr: \type_1 \times \type_2
    }{
      \Delta; \Gamma; \dmap \tvdash{\nnatA} \eprojr(\expr): \type_2
    }~\textbf{snd}
    %
    \and
    %
    \inferrule{
    }{
      \Delta; \Gamma; \dmap \tvdash{\nnatA} \etrue: \tbool
    }~\textbf{true}
    %
    \and
    %
    \inferrule{
    }{
       \Delta; \Gamma; \dmap \tvdash{\nnatA} \efalse: \tbool
    }~\textbf{false}
    %
    \and
      \inferrule{
        \Delta; \Gamma; \dmap' \tvdash{\nnatA'} \expr: \type'
        \\
        \dmap' < \dmap 
        \\
        \nnatA' < \nnatA
        \\
        \Delta \models \sub{\type' }{ \type  } 
    }{
       \Delta; \Gamma; \dmap \tvdash{\nnatA} \expr: \type
    }~\textbf{subtype}
    %
    \and
     \inferrule{
      \Delta; \Gamma ; \dmap \tvdash{\nnatA} \expr : \treal
    }{
       \Delta; \Gamma; \dmap \tvdash{\nnatA} \bernoulli \eapp \expr: \treal
    }~\textbf{bernoulli}
    %
     \and
     \inferrule{
      \Delta; \Gamma ; \dmap_1 \tvdash{\nnatA_1} \expr_1 : \treal
      \\
      \Delta ; \Gamma ; \dmap_2 \tvdash{\nnatA_2} \expr_2 : \treal
      \\
      \nnatA = \max(\nnatA_1, \nnatA_2)
      \\
      \dmap' = \max(\dmap_1, \dmap_2)
    }{
       \Delta; \Gamma; \dmap' \tvdash{\nnatA'} \uniform \eapp \expr_1 \eapp \expr_2: \treal
    }~\textbf{uniform}
    
  \end{mathpar}
  \wq{    $( \tarr{\type_1}{ \type_2  }{\nnatbiA}{\dmap}{\nnatA} )  \setminus x $ $==$ $\tarr{(\type_1 \setminus x)}{ (\type_2 \setminus x) }{\nnatbiA}{(\dmap \setminus x) }{\nnatA} $  \\
   $\type \setminus x$ means removing x in all the $\dmap$ of arrow types containing in the type $\type$.  
  }
  \caption{Typing rules, part 1}
  \label{fig:type-rules}
\end{figure}

\begin{figure}
\begin{mathpar}
    %
    \inferrule{
       \Delta;\Gamma; \dmap_1 \tvdash{\nnatA_1} \expr_1: \tbool \\
       \Delta;\Gamma; \dmap \tvdash{\nnatA} \expr_2: \type \\
       \Delta;\Gamma; \dmap \tvdash{\nnatA} \expr_3: \type \\\\
      \nnatA' = \nnatA_1 + \nnatA \\
      \dmap' = \max(\dmap_1, \nnatA_1 + \dmap)
    }{
       \Delta;\Gamma; \dmap' \tvdash{\nnatA'} \eif(\expr_1, \expr_2, \expr_3):  \type
    }~\textbf{if}
    %
    \and
    %
    \inferrule{
    }{
       \Delta;\Gamma; \dmap \tvdash{\nnatA} \econst: \tbase
    }~\textbf{const}
    \and
    %
    \inferrule{
    }{
       \Delta;\Gamma; \dmap \tvdash{\nnatA} n: \tint[n]
    }~\textbf{intI} 
    \and
    %
    \inferrule{
    }{
       \Delta;\Gamma; \dmap \tvdash{\nnatA} n: \tint
    }~\textbf{int} 
   \and
    %
    \inferrule{
      \Delta; \dmap \wf{\type} \\
    }{
      \Delta; \Gamma; \dmap \tvdash{\nnatA} \enil: \tlist{\type}
    }~\textbf{nil}
    % 
 	\and
    %
    \inferrule
    {
      \Delta; \Gamma; \dmap \tvdash{\nnatA} \expr:  \tbox{  (\tarr{ \type_1
        }{ \type_2 }{0}{\dmap''}{0})     }   \\
      \nnatA' = 1 + \nnatA \\
      \wq{ \dmap' = 1 + \max(\dmap, \dmap''+\nnatA) } 
    }
    {
       \Delta; \Gamma; \dmap' \tvdash{\nnatA'} \eop(\expr): \treal
    }~\textbf{$\delta$}
%
  \and
%
     {
     \inferrule
     {
        \Delta; \Gamma; \dmap \tvdash{\nnatA} \expr: \type 
           \\
        \forall x \in \dom(\Gamma), \sub{\Gamma(x)}{  \tbox{\Gamma(x)}
         }
           \\
        \delta \not\in \expr
        \\
        \dom(\Gamma') = \dom(\dmap')
    }
    {
        \Delta; \Gamma, \Gamma'; \dmap, \boxed{\dmap'} \tvdash{\nnatA} \expr: \tbox{\type}
    } ~\textbf{box}
    }
    %
    \and
    %
   \inferrule{
   \Delta; \Gamma; \dmap_1 \tvdash{\nnatA_1} \expr_1 : \type \\
   \Delta; \Gamma; \dmap_2 \tvdash{\nnatA_2} \expr_2 : \tlist{\type}\\
   \dmap' = \max(\dmap_1, \dmap_2) \\
   \nnatA' = \max ( \nnatA_1, \nnatA_2 )
   }
   { 
   \Delta; \Gamma; \dmap' \tvdash{\nnatA'} \econs(\expr_1, \expr_2) :
     \tlist{\type}  } ~\textbf{cons}
   %
   \and
   %
   \inferrule{
     \Delta; \Gamma; \dmap_1 \tvdash{\nnatA_1} \expr_1 : \type_1 \\
     \Delta; \Gamma, x:\type_1 ; \dmap_2[x:q] \tvdash{\nnatA_2} \expr_2 :
     \type \\
     \dmap' = \max( \dmap_2, \dmap_1 + q ) \\
     \nnatA' = \max ( \nnatA_2, \nnatA_1 + q )
   }
   {  \Delta; \Gamma; \dmap' \tvdash{\nnatA'}  \elet x;q = \expr_1 \ein \expr_2 : \type}~\textbf{let}
   %
   \and
   %
  \inferrule{
      i,\Delta; \Gamma ;\dmap \tvdash{\nnatA} \expr: \type
      \\
      i \not\in  \fiv{\Gamma}
    }{
     \Delta;  \Gamma; \dmap' \tvdash{\nnatA'}    \eilam \expr    :  \tforall{\dmap}{\nnatA}{i} \type 
    }~\textbf{ilam}
    %
    \and
    %
  \inferrule{
        \Delta; \Gamma ;\dmap \tvdash{\nnatA} \expr: \tforall{\red{\dmap_1}}{\nnatA_1}{i} \type 
      \and
       \Delta \tvdash{}  I ::  S
       \\
       \dmap' = \max(\dmap, \nnatA + \dmap_1)
       \\
       \nnatA' = \nnatA_1[I/i] + \nnatA
    }{
     \Delta;  \Gamma; \dmap' \tvdash{\nnatA'}    \expr \eapp []   :
     \type[I/i]
    }
	 ~\textbf{iapp} \wq{domain ? }        
\end{mathpar}

   \framebox{
  \begin{mathpar}
    \mbox{\textbf{where: }} \\
%    \dmapb \defeq \lambda x.\eapp  \bot \\
    \nnatbiA + \dmap \defeq \dmap + \nnatbiA \defeq \lambda x.\eapp  (\nnatbiA + \dmap(x)) \\
    \dmap_1 + \dmap_2 \defeq \lambda x.\eapp  (\dmap_1(x) + \dmap_2(x)) \\
    \max(\dmap_1, \dmap_2) \defeq \lambda x.\eapp  \max(\dmap_1(x), \dmap_2(x))
  \end{mathpar}}

  \caption{Typing rules, part 2}
  \label{fig:type-rules2}
\end{figure}



\clearpage

\begin{figure}
% \[
% \begin{array}{llll}
% \mbox{Index Term} & \idx, \nnatA & ::= &     i ~|~ n ~|~ \idx_1 + \idx_2 ~|~  \idx_1
%                                  - \idx_2 ~|~ \smax{\idx_1}{\idx_2}\\
%   \mbox{Sort} & S & ::= & \nat \\
% \mbox{Expr.} & \expr & ::= & x ~|~ \expr_1 \eapp \expr_2 ~|~ \efix f(x).\expr
%             ~|~ (\expr_1, \expr_2) ~|~ \eprojl(\expr) ~|~ \eprojr(\expr) ~| \\
% %
% & & &           \etrue ~|~ \efalse ~|~ \eif(\expr_1, \expr_2, \expr_3) 
%             ~|~ \econst ~|~ \eop(\expr) \\
% & & &           ~|~ \elet  = \expr_1 \ein \expr_2 ~|~ \enil ~|~  \econs (
%                 \expr_1, \expr_2)  \\
% & & &  \eilam \expr  ~|~  \expr \eapp []  ~|~
%                             \epack \expr ~|~ \eunpack \expr \eas x
%                             \ein \expr \\
% & & & ~|~ \bernoulli \eapp \valr ~|~ \uniform \eapp \valr_1 \eapp \valr_2\\      
% %
%   \mbox{Type} & \type & ::= & \tbool ~|~ \type_1 \times
%   \type_2 ~|~ \tarr{\type_1}{\type_2}{\nnatbiA}{\dmap}{\nnatA} ~|~
%                               \tlist{\type} ~|~ \tbox{\type}\\
% & & &                 ~|~  \treal ~|~  \tint ~|~   \tint[I]  ~|~      \tforall{\dmap}{\nnatA}{i} \type  ~|~ \texists{i} \type
% %
% \end{array}
% \]

% \framebox{
%   \begin{mathpar}
%      {\color{blue}
%      \inferrule
%      {
%         \Delta; \Gamma; \dmap \tvdash{\nnatA} \expr: \type 
%            \\
%         \forall x \in \dom(\Gamma), \sub{\Gamma(x)}{  \tbox{\Gamma(x)}
%          }
%            \\
%         \delta \not\in \expr
%     }
%     {
%         \Delta; \Gamma, \Gamma'; \dmap \tvdash{\nnatA} \expr: \tbox{\type}
%     }
%     }
%     %
%     \and
%     %
%     \inferrule{
%       \dmap \wf{\type} \\
%     }{
%       \Delta; \Gamma; \dmap \tvdash{\nnatA} \enil: \tlist{\type}
%     }
%     % 
%     \and
%     %
%    \inferrule{
%    \Delta; \Gamma; \dmap_1 \tvdash{\nnatA_1} \expr_1 : \type \\
%    \Delta; \Gamma; \dmap_2 \tvdash{\nnatA_2} \expr_2 : \tlist{\type}\\
%    \dmap' = \max(\dmap_1, \dmap_2) \\
%    \nnatA' = \max ( \nnatA_1, \nnatA_2 )
%    }
%    { 
%    \Delta; \Gamma; \dmap' \tvdash{\nnatA'} \econs(\expr_1, \expr_2) :
%      \tlist{\type}  }
%    %
%    \and
%    %
%    \inferrule{
%      \Delta; \Gamma; \dmap_1 \tvdash{\nnatA_1} \expr_1 : \type_1 \\
%      \Delta; \Gamma, x:\type_1 ; \dmap_2[x:q] \tvdash{\nnatA_2} \expr_2 :
%      \type_2 \\
%      \dmap' = \max( \dmap_2, \dmap_1 + q ) \\
%      \nnatA' = \max ( \nnatA_2, \nnatA_1 + q )
%    }
%    {  \Delta; \Gamma; \dmap' \tvdash{\nnatA'}  \elet x = \expr_1 \ein \expr_2 : \type}
%    %
%    \and
%    %
%   {\color{red}
%   \inferrule{
%       \Delta, i; \Gamma ;\dmap \tvdash{\nnatA} \expr: \type
%     }{
%      \Delta;  \Gamma; \dmap \tvdash{\nnatA}    \eilam \expr    :  \tforall{\dmap}{\nnatA}{i} \type 
%     }
%     %
%     \and
%     %
%   \inferrule{
%       \Delta; \Gamma ;\dmap \tvdash{\nnatA} \expr: \tforall{\dmap_1}{\nnatA_1}{i} \type
%       \and
%        \Delta \tvdash{}  I ::  S
%        \\
%        \dmap' = \max(\dmap, \nnatA + \dmap_1)
%        \\
%        \nnatA' = \nnatA_1[I/i] + \nnatA
%     }{
%      \Delta;  \Gamma; \dmap' \tvdash{\nnatA'}    \expr \eapp []   :
%      \type[I/i]
%     }
% 	    }
%     \end{mathpar}
% }          
\boxed{  \Delta; \dmap \wf{\type} }
\begin{mathpar}
\inferrule{
      \Delta; \dmap \wf{\type} \\
    }{
      \Delta; \dmap \wf{ \tlist{\type} }
    }
   \and
     \inferrule{
    }{
      \Delta; \dmap \wf{ \tbool}
    }
  \and 
  \inferrule{
      \Delta; \dmap \wf{\type_1} \\
      \Delta; \dmap \wf{\type_2} 
    }{
      \Delta; \dmap \wf{ \type_1 \times
  \type_2 }
    }
 \and
    \wq{ \inferrule{
        \Delta; \dmap \wf{ \type_1 } \\
        \Delta; \dmap' \wf{\type_2}  \\
        \Delta \vdash \nnatA :: S \\
        \Delta \vdash \nnatbiA :: S 
    }{
      \Delta; \dmap \wf{ \tarr{\type_1}{\type_2}{\nnatbiA}{\dmap'}{\nnatA}   }
    } }
 \and
     \inferrule{
      \Delta; \dmap \wf{  \type }
    }{
      \Delta; \dmap \wf{  \tbox{\type}   }
    }
\and
     \inferrule{
    }{
      \Delta; \dmap \wf{  \treal  }
    }
\and
     \inferrule{
    }{
      \Delta; \dmap \wf{  \tint   }
    }
\and
     \inferrule{
       \Delta \vdash I :: S \\
    }{
      \Delta; \dmap \wf{  \tint[I]   }
    }
\and
  \inferrule{
   i::S, \Delta; \dmap \wf{ \type }  \\
    i::S, \Delta \vdash \nnatA :: S 
    }{ 
      \Delta; \dmap \wf{   \tforall{\dmap}{\nnatA}{i} \type   }
    }


\end{mathpar}  

  \caption{Well-formedness rules}
  \label{fig:well-formedness-rules}
\end{figure}

\begin{figure}
\boxed{  \Delta \tvdash{}  I :: S  }
    \begin{mathpar}
  \inferrule
  {
       \Delta (i) = S
  }
  {
     \Delta \tvdash{}  i :: S
  }
  %
  \and
  %
  \inferrule
  {
    \empty
  }
  {
    \Delta \tvdash{}  n :: \nat
  }
  %
  \and
  %
  \inferrule
  {
    \Delta \tvdash{}  \idx_1 :: S
    \and
    \Delta \tvdash{}  \idx_2 :: S
    \and
    \Delta \tvdash{}  \diamond \in \{ +, -, \max \}
  }
  {
    \Delta \tvdash{}  \idx_1 \diamond \idx_2 :: S
  }
  \end{mathpar}
\caption{Sorting rules}
\end{figure}




\begin{figure}
   \begin{mathpar}
     \inferrule{
    }{
       \Delta \models \sub{\tbase}{\tbox{\tbase} }
    }~\textbf{sb-box-base}
    %
    \and
    %
     \inferrule{
    }{
      \Delta \models \sub{\tbool}{\tbox{\tbool} }
    }~\textbf{sb-box-bool}
    %
    \and
    %
    \inferrule{
     \Delta \models \sub{\type_1 }{ \type_1'  } \\
         \Delta \models \sub{\type_2 }{ \type_2'  }
    }{
       \Delta \models \sub{\type_1 \times \type_2 }{ \type_1' \times \type_2'  }
    }~\textbf{sb-pair}
    %
    \and
    %
   \inferrule{
       % \forall x \in \dmap'. \dmap'(x) =0
    }{
       \Delta \models \sub{\tbox{(
          \tarr{\type_1}{\type_2}{\nnatbiA}{\dmap'}{\nnatA} )} 
      }{\tarr{(\tbox{\type_1}) }{(\tbox{\type_2})}{0}{\dmap'}{0} }
    } ~\textbf{sb-box-arrow}
    %
    % \and
    % %
    %  \inferrule{
    %    \dmap' \leq \dmap
    % }{
    %    \Delta \models \sub{\tarr{\type_1}{\type_2}{0}{\dmap'}{0} }{  \tbox{(
    %       \tarr{\type_1}{\type_2}{\nnatbiA}{\dmap}{\nnatA} )  }
    % } }
    %
    \and
    %
     \inferrule{
    }
    {
       \Delta \models \sub{ \tint[I] }{ \tint }
    }~\textbf{sb-intI}
    %
    \and
    %
     \inferrule
    {
      \Delta  \models \sub{ \type_1' }{ \type_1 }
      \and
      \Delta  \models \sub{ \type_2 }{ \type_2' }
      \and
      \dmap \leq \dmap'
      \and
      \nnatA \leq \nnatA'
      \and 
      q \leq q'
    }
    {
      \Delta  \models \sub{ \tarr{\type_1}{\type_2}{ \nnatbiA }{\dmap}{\nnatA} }
      { \tarr{\type_1'}{\type_2'}{\nnatbiA'}{\dmap'}{\nnatA'} }
    }~\textbf{sb-arrow}
  \and
    %
     \inferrule{
    }{
      \Delta  \models \sub{ \tbox{\type}  }{ \type }
    }~\textbf{sb-T}
  \and
    %
     \inferrule{
    }{
      \Delta  \models \sub{ \tbox{\type}  }{ \tbox{\tbox{\type}} }
    }~\textbf{sb-D}
  \and
    %
     \inferrule{
      \Delta \models \sub{ \type_1  }{ \type_2 }
    }{
      \Delta \models \sub{ \tbox{\type_1}  }{ \tbox{\type_2} }
    }~\textbf{sb-box}
  \and
    %
     \inferrule{
    }{
      \Delta \models \sub{ \type  }{ \type }
    }~\textbf{sb-refl}
 \and
    %
     \inferrule{
      \Delta \models \sub{ \type_1  }{ \type_2 } 
        \\
      \Delta \models \sub{ \type_2  }{ \type_3 }
    }{
      \Delta \models \sub{ \type_1  }{ \type_3 }
    }~\textbf{sb-tran}
\and
%
     \inferrule{
      \Delta \models \sub{ \type_1  }{ \type_2 } 
    }{
      \Delta \models \sub{ \tlist{\type_1}  }{ \tlist{\type_2} }
    }~\textbf{sb-list}
\and
%
    \inferrule
    {
      \empty
    }
    {
      \Delta \models \sub{ \tlist{\tbox{\type }}  }{ \tbox{\tlist{\type}} } 
    }~\textbf{sb-list-box}
 %
    \and
% \tforall{\dmap}{\nnatA}{i} \type
 \inferrule
    {
      i::S,\Delta \models \sub{ \type }{ \type' }
      \\
      \nnatA \leq \nnatA'
      \\
      \dmap \leq \dmap'
    }
    {
      \Delta \models \sub{   \tforall{\dmap}{\nnatA}{i} \type  }{ \tforall{\dmap'}{\nnatA'}{i} \type'} 
    }~\textbf{sb-$\forall$}

    \end{mathpar}
 \caption{Subtyping rules}
  \label{fig:sub-type-rules}
\end{figure}


The rules follow \emph{exactly} the definitions of $\adap()$ and
$\ddep{x}()$ to compute $\nnatA$ and $\dmap$ in the conclusion of the
typing rule for every construct. For instance, the typing rule for
$\expr_1 \eapp \expr_2$ does exactly what $\adap()$ and $\ddep{x}()$
do for the trace $\trapp{\tr_1}{\tr_2}{f}{x}{\tr_3}$.

\paragraph{Remark}
The astute reader might note that our type system looks very much like
a coeffect+effect system where the coeffect $\dmap$ estimates the
depth of each free variable and the effect $\nnatA$ estimates the
adaptivity. While this is true at a high-level, note there are two
essential differences between our type system and a coeffect+effect
system.
\begin{itemize}
\item[-] In some of our rules (those for $\expr_1 \eapp \expr_2$ and
  $\eif(\expr_1, \expr_2, \expr_3)$), the conclusion's coeffect
  depends on the effects of the premises. For instance, in the rule
  for $\expr_1 \eapp \expr_2$, the final coeffect $\dmap'$ depends on
  the effect $\nnatA_1$. This does not happen in standard
  coeffect+effect systems.
\item[-] Our type for functions internalizes the effect. In a standard
  coeffect + effect system, the effect is always on the typing  judgment, and at the point of function construction ($\efix$), one
  anticipatively adds the effects of future function applications to
  the effect in the conclusion.
\end{itemize}
It may be possible to do away with both these differences by tracking
more information in the coeffects (e.g., what is the maximum
adaptivity of a function's argument?), but the details need to be
worked out and verified.
