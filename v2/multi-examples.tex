
\documentclass[a4paper,11pt]{article}

\usepackage{mathpartir}
\usepackage{amsmath,amsthm,amsfonts}
\usepackage{color}
\usepackage{algorithm}
\usepackage{algorithmic}

%%% Attempt 1: Linear 1



\newcommand{\diam}{{\color{red}\diamond}}
\newcommand{\dagg}{{\color{blue}\dagger}}
\let\oldstar\star
\renewcommand{\star}{\oldstar}

\newcommand{\im}[1]{\ensuremath{#1}}

\newcommand{\kw}[1]{\im{\mathtt{#1}}}


\newcommand{\set}[1]{\im{\{{#1}\}}}

\newcommand{\mmax}{\ensuremath{\mathsf{max}}}

%%%%%%%%%%%%%%%%%%%%%%%%%%%%%%%%%%%%%%%%%%%%%%%%%%%%%%%%
% Comments
\newcommand{\omitthis}[1]{}

% Misc.
\newcommand{\etal}{\textit{et al.}}
\newcommand{\bump}{\hspace{3.5pt}}

% Text fonts
\newcommand{\tbf}[1]{\textbf{#1}}
%\newcommand{\trm}[1]{\textrm{#1}}

% Math fonts
\newcommand{\mbb}[1]{\mathbb{#1}}
\newcommand{\mbf}[1]{\mathbf{#1}}
\newcommand{\mrm}[1]{\mathrm{#1}}
\newcommand{\mtt}[1]{\mathtt{#1}}
\newcommand{\mcal}[1]{\mathcal{#1}}
\newcommand{\mfrak}[1]{\mathfrak{#1}}
\newcommand{\msf}[1]{\mathsf{#1}}
\newcommand{\mscr}[1]{\mathscr{#1}}









\newcommand{\defeq}{\mathrel{\doteq}}
\newcommand{\conj}{\mathrel{\wedge}}
\newcommand{\disj}{\mathrel{\vee}}

\newcommand{\lzero}{0}


% context
\newcommand{\tctx}{\Gamma}
\newcommand{\ictx}{ }


% expression
\newcommand{\expr}{e}
\newcommand{\aexpr}{a}
\newcommand{\bexpr}{b}
\newcommand{\sexpr}{\textrm{e} }
\newcommand{\qexpr}{\psi}
\newcommand{\qval}{\alpha}
\newcommand{\query}{{\tt query}}
\newcommand{\saexpr}{\textrm{a} }
\newcommand{\sbexpr}{\textrm{b} }
\newcommand{\vall}{w}
\newcommand{\valr}{v}
\newcommand{\eif}{\kw{if}}
\newcommand{\eapp}{\;}
\newcommand{\eprojl}{\kw{fst}}
\newcommand{\eprojr}{\kw{snd}}
\newcommand{\eifvar}{\kw{ifvar}}
%expression and commands for WHILE language
\newcommand{\ewhile}{\kw{while}}
\newcommand{\bop}{*}
\newcommand{\uop}{\circ}
\newcommand{\eskip}{\kw{skip}}

\newcommand{\eloop}{\kw{loop}}
\newcommand{\edo}{\kw{do}}
\newcommand{\qdom}{\mathcal{QD}}

%configuration
\newcommand{\config}[1]{\langle #1 \rangle}
\newcommand{\ematch}{\kw{match}}
\newcommand{\clabel}[1]{\left[ #1 \right]}


%\newcommand{\eprov}[1]{\eta_{#1}}
\newcommand{\etrue}{\kw{true}}
\newcommand{\efalse}{\kw{false}}
\newcommand{\econst}{c}
\newcommand{\eop}{\delta}
\newcommand{\efix}{\mathop{\kw{fix}}}
\newcommand{\elet}{\mathop{\kw{let}}}
\newcommand{\ein}{\mathop{ \kw{in}} }
\newcommand{\eas}{\mathop{ \kw{as}} }
\newcommand{\enil}{\kw{nil}}
\newcommand{\econs}{\mathop{\kw{cons}}}
%\newcommand{\labelA}{\ell}
%monad expressions / terms
\newcommand{\term}{t}
\newcommand{\return}{\kw{return}}
\newcommand{\bernoulli}{\kw{bernoulli}}
\newcommand{\uniform}{\kw{uniform}}
 \newcommand{\epack}{\mbox{pack\;}}
\newcommand{\eunpack}{\mbox{unpack\;}}
\newcommand{\eilam}{\Lambda.}

\newcommand{\evec}{\kw{dict}}
\newcommand{\eget}{\kw{get}}

% trace
\newcommand{\triapp}[2]{\kw{IApp}(#1,#2)}
\newcommand{\trow}{\text{row}}
\newcommand{\tr}{T}
\newcommand{\trift}{\eif^{\kw{t}}}
\newcommand{\triff}{\eif^{\kw{f}}}
\newcommand{\trprojl}{\eprojl}
\newcommand{\trprojr}{\eprojr}
\newcommand{\trtrue}{\etrue}
\newcommand{\trfalse}{\efalse}
\newcommand{\trconst}{\econst}
\newcommand{\trop}{\eop}
\newcommand{\trfix}{\efix}
\newcommand{\trapp}[5]{#1 \; #2 \mathrel{\triangleright} {\efix
#3(#4).#5}}
\newcommand{\trnil}{\enil}
\newcommand{\trcons}{\econs}
\newcommand{\trlet}{\elet}
%types for monad
\newcommand{\treal}{\kw{real}}
\newcommand{\tint}{\kw{int}}
\newcommand{\tmonad}{\kw{M}}
\newcommand{\tunit}{\kw{unit}}
\newcommand{\tdb}{\kw{tdb}}

% adaptivity
\newcommand{\adap}{\kw{adap}}
\newcommand{\ddep}[1]{\kw{depth}_{#1}}
\newcommand{\nat}{\mathbb{N}}
\newcommand{\natb}{\nat_{\bot}}
\newcommand{\natbi}{\natb^\infty}
\newcommand{\nnatA}{Z}
\newcommand{\nnatB}{m}
\newcommand{\nnatbA}{s}
\newcommand{\nnatbB}{t}
\newcommand{\nnatbiA}{q}
\newcommand{\nnatbiB}{r}

%type
\newcommand{\type}{\tau}
\newcommand{\tbase}{\kw{b}}
\newcommand{\tbool}{\kw{bool}}
\newcommand{\tbox}[1]{ \kw{\square} \, (#1) }
\newcommand{\tarr}[5]{#1; #3 \xrightarrow{#4; \, #5} #2}
\newcommand{\tlist}[1]{\kw{list} \, #1 }
\newcommand{\env}{\theta}
\newcommand{\tforall}[3]{\forall#3 \overset{#1, #2}{::} S.\, }
\newcommand{\texists}[1]{\exists#1 {::} S.\, }
\newcommand{\lto}{\multimap}
\newcommand{\bang}[1]{ !_{#1}}
\newcommand{\whynot}[1]{ ?_{#1} }
\newcommand{\ltype}{A}
\newcommand{\adapt}{R}
% index
\newcommand{\idx}{I }
\newcommand{\smax}[2]{\kw{max}(#1,#2)}
\newcommand{\ienv}{\sigma}

%evaluation
\newcommand{\bigstep}[1]{\mathrel{\to^{#1}}}

\newcommand{\dmap}{\rho}
\newcommand{\dmapb}{\bot_\dmap}
\newcommand{\supp}{\kw{supp}}
\newcommand{\dom}{\kw{dom}}
\newcommand{\codom}{\kw{codom}}

\newcommand{\tvdash}[1]{\vdash_{#1}}

\newcommand{\lrv}[1]{[\![ #1 ]\!]_{\text{V}}}
\newcommand{\lre}[3]{[\![ #3 ]\!]_{\text{E}}^{#1, #2}}
\newcommand{\stepiA}{k}
\newcommand{\stepiB}{j}
\newcommand{\size}[1]{|#1|}

%logic relations
\newcommand{\lr}[1]{[\![ #1 ]\!]}
\newcommand{\lrt}[1]{\mathcal{T}[\![ #1 ]\!]}


\newcommand{\wf}[1]{\vdash #1 \, \kw{wf} }
\newcommand{\sub}[2]{ #1 \, <: \, #2 }
\newcommand{\eqv}[3]{ #1 \, \equiv \, #2 \Rightarrow \textcolor{red}
{#3}  }
\newcommand{\eqvt}[3]{ #1 \, \sqsubseteq \, #2 \Rightarrow \textcolor{red}
{#3}  }
\newcommand{\eqvc}[2]{ #1 \, \equiv^c \, #2   }


%core calculus
\newcommand{\ctyping}[3]{ \tvdash{ #1} {#2} :^c #3 }
\newcommand{\cbox}{\mathsf{box}}
\newcommand{\cder}{\mathsf{der}}
\newcommand{\elab}[4]{ \vdash_{ #1} #2 \rightsquigarrow #3 : #4}
\newcommand{\coerce}[2]{\mathsf{coerce}_{#1, #2}}

%algorithmic typing rules
\newcommand{\infr}[4]{{#1} ~ {\textcolor{red}\uparrow} ~ {\color{red} #2} \Rightarrow
{ } {\color{red} #3} }
\newcommand{\chec}[3]{{#1} ~ {\downarrow} ~ {#2} \Rightarrow {\color{red} #3} }
% \newcommand{\restriction}{\Phi}
\newcommand{\fresh}{ \mathsf{fresh}}
\newcommand{\red}[1]{ \textcolor{red} {#1} }
\newcommand{\fiv}[1]{ \mathsf{FIV} (#1)   }
\newcommand{\fv}[1]{ \mathsf{FV} (#1)   }

\newcommand{\todo}[1]{{\small \color{red}\textbf{[[ #1 ]]}}}
\newcommand{\todomath}[1]{{\scriptstyle \color{red}\mathbf{[[ #1 ]]}}}

\newcommand{\caseL}[1]{\item \textbf{#1}\newline}

\newcommand{\attr}{\mathsf{attr}}
\newcommand{\weight}{\mathsf{W}}
\newcommand{\num}{\mathsf{n}}

\usepackage{enumitem}
\setenumerate{listparindent=\parindent}

\newlist{enumih}{enumerate}{3}
\setlist[enumih]{label=\alph*),before=\raggedright, topsep=1ex, parsep=0pt,  itemsep=1pt }

\newlist{enumconc}{enumerate}{3}
\setlist[enumconc]{leftmargin=0.5cm, label*= \arabic*.  , topsep=1ex, parsep=0pt,  itemsep=3pt }


\newlist{enumsub}{enumerate}{3}
\setlist[enumsub]{ leftmargin=0.7cm, label*= \textbf{subcase} \bf \arabic*: }

\newlist{enumsubsub}{enumerate}{3}
\setlist[enumsubsub]{ leftmargin=0.5cm, label*= \textbf{subsubcase} \bf \arabic*: }

\newlist{mainitem}{itemize}{3}
\setlist[mainitem]{ leftmargin=0cm , label= {\bf Case} }

%%%%COLORS
\definecolor{periwinkle}{rgb}{0.8, 0.8, 1.0}
\definecolor{powderblue}{rgb}{0.69, 0.88, 0.9}
\definecolor{sandstorm}{rgb}{0.93, 0.84, 0.25}
\definecolor{trueblue}{rgb}{0.0, 0.45, 0.81}


\usepackage{array}

\newlength\Origarrayrulewidth

% horizontal rule equivalent to \cline but with 2pt width
\newcommand{\Cline}[1]{%
 \noalign{\global\setlength\Origarrayrulewidth{\arrayrulewidth}}%
 \noalign{\global\setlength\arrayrulewidth{2pt}}\cline{#1}%
 \noalign{\global\setlength\arrayrulewidth{\Origarrayrulewidth}}%
}

% draw a vertical rule of width 2pt on both sides of a cell
\newcommand\Thickvrule[1]{%
  \multicolumn{1}{!{\vrule width 2pt}c!{\vrule width 2pt}}{#1}%
}

% draw a vertical rule of width 2pt on the left side of a cell
\newcommand\Thickvrulel[1]{%
  \multicolumn{1}{!{\vrule width 2pt}c|}{#1}%
}

% draw a vertical rule of width 2pt on the right side of a cell
\newcommand\Thickvruler[1]{%
  \multicolumn{1}{|c!{\vrule width 2pt}}{#1}%
}

\newcommand{\command}{c}
\newcommand{\green}[1]{{ \color{green} #1 } }

\newcommand{\func}[2]{\mathsf{AD}(#1) \to (#2)}
\newcommand{\varEst}{\bf{VetxEst}}
\newcommand{\graphGen}{\bf{GraphGen}}

\newcommand{\ag}[2]{\mathsf{VetxEst}{(#1)}\to {(#2)}}
\newcommand{\ad}[2]{\mathsf{GraphGen}{(#1)}\to {(#2)}}
\newcommand{\rb}{\mathsf{RechBound}}
\newcommand{\pathsearch}{\mathsf{AdaptPathSearch}}


\theoremstyle{definition}
\newtheorem{thm}{Theorem}
\newtheorem{lem}[thm]{Lemma}
\newtheorem{cor}[thm]{Corollary}
\newtheorem{prop}[thm]{Proposition}
\newtheorem{defn}[thm]{Definition}

\title{Adaptivity analysis}

\author{}

\date{January 19, 2019}

\begin{document}

\maketitle


\[\begin{array}{llll}
\mbox{Term} & \term & ::= &     \eilam \expr  ~|~  \expr \eapp []  ~|~
                            \epack \expr ~|~ \eunpack \expr \eas x
                            \ein \expr \\
\mbox{Index Term} & \idx, \nnatA & ::= &     i ~|~ n ~|~ \idx_1 + \idx_2 ~|~  \idx_1
                                 - \idx_2 ~|~ \smax{\idx_1}{\idx_2}\\
  \mbox{Sort} & S & ::= & \natbi \\
  \mbox{Type} & \type & ::= & \tforall{i} \type  ~|~ \texists{i} \type 
\end{array}\]


\begin{figure}
  \begin{mathpar}
    \inferrule{
      \Delta, i; \Gamma ;\dmap \tvdash{\nnatA} \expr: \type
    }{
     \Delta;  \Gamma; \dmap \tvdash{\nnatA}    \eilam \expr    :  \tforall{i} \type 
    }
    %
    \and
    %
  \inferrule{
      \Delta; \Gamma ;\dmap \tvdash{\nnatA} \expr: \tforall{i} \type
      \\
       \Delta \tvdash{}  I ::  S 
    }{
     \Delta;  \Gamma; \dmap \tvdash{\nnatA[I/i]}    \expr \eapp []   :
     \type[ I/i]
    }
  \end{mathpar}
  \caption{typing rules - monad}
  \label{fig:type-rules-monad}
\end{figure}


\newpage
\begin{figure}

Multi-rounds:
\[
\begin{array}{l}
 \efix \eapp  \mathsf{multiRound}(\_). \lambda sc. \lambda scc. \lambda
  I. \lambda j. \lambda k. \lambda D. \lambda N. \lambda C. \\
 \hspace{.2cm} \eif   \big (   (j < k)  ,  \\
  \hspace{.2cm} \elet \eapp p = \uniform \eapp 0 \eapp 1 \ein \\
  \hspace{0.4cm} \elet \eapp q = \lambda x. \bernoulli \eapp p \ein \\
 \hspace{0.4cm} \elet \eapp qc = \lambda c. \bernoulli \eapp p \ein \\
 \hspace{0.4cm} \elet \eapp a = \eop (q)  \ein \\
 \hspace{0.8cm} \elet \eapp sc' =  \mathsf{updtSC} \eapp () \eapp sc  \eapp a \eapp p
 \eapp q \eapp I \eapp  \eapp 0 \eapp  N
  \eapp  \ein \\
\hspace{0.8cm} \elet \eapp scc' =  \mathsf{updtSCC} \eapp () \eapp scc' \eapp a \eapp p
 \eapp qc \eapp  \eapp 0 \eapp  C \ein \\
\hspace{0.8cm} \elet \eapp maxScc =  \mathsf{foldl} \eapp (\lambda acc. \lambda a. \eif ( acc < a, a, acc)) \eapp 0 \eapp scc' \ein \\
\hspace{0.8cm} \elet \eapp I' =  \mathsf{updtI}  \eapp () \eapp maxScc \eapp sc
  \eapp 0 \eapp N  \ein \\
  \hspace{0.8cm} \elet \eapp D' =  D \setminus I' \ein \\
  \hspace{1.2cm} \mathsf{multiRound} ()  \eapp sc' \eapp scc' \eapp I'
  \eapp (j+1) \eapp  k \eapp D' \eapp N \eapp C\\ 
\hspace{0.2cm}   ,     D  \big ) \\
 
\end{array}
\]

UpdtSC
\[
\begin{array}{l}
 \mathsf{updtSC} = \efix \eapp  \mathsf{f}(\_). \lambda sc. \lambda a. \lambda
  p. \lambda q.  \lambda I. \lambda i. \lambda N. \\
 \hspace{.2cm} \eif   \Big (   (i < N)  ,  \\
 \hspace{0.4cm}  \eif \big( ( i \eapp \mathsf{in} I  ) ,       \\
 \hspace{0.8cm} \elet \eapp x =( \mathsf{nth} \eapp sc \eapp i) + (a-p)*(q
  \eapp i - p)  \ein \\
 \hspace{0.8cm} \elet \eapp sc' =  \mathsf{updt} \eapp sc \eapp i
  \eapp x \ein \\
  \hspace{1.2cm} \mathsf{f}  \eapp () \eapp sc' \eapp a \eapp p
 \eapp q \eapp I \eapp  \eapp (i+1) \eapp  N  \\ 
\hspace{0.4cm}  ,  \mathsf{f}  \eapp () \eapp sc \eapp a \eapp p
 \eapp q \eapp I \eapp  \eapp (i+1) \eapp  N \big )  \\ 
\hspace{0.2cm}   ,  sc  \Big ) \\
 
\end{array}
\]

UpdtSCC
\[
\begin{array}{l}
 \mathsf{updtSCC} = \efix \eapp  \mathsf{f}(\_). \lambda scc. \lambda a. \lambda
  p. \lambda qc.  \lambda i. \lambda C. \\
 \hspace{.2cm} \eif   \Big (   (i < C)  ,  \\
 \hspace{0.8cm} \elet \eapp x =( \mathsf{nth} \eapp scc \eapp i) + (a-p)*(qc
  \eapp i - p)  \ein \\
 \hspace{0.8cm} \elet \eapp scc' =  \mathsf{updt} \eapp scc \eapp i
  \eapp x \ein \\
  \hspace{1.2cm} \mathsf{f}  \eapp () \eapp scc' \eapp a \eapp p
 \eapp qc   \eapp (i+1) \eapp  C  \\ 
\hspace{0.2cm}   ,  scc  \Big ) \\
 
\end{array}
\]


UpdtI
\[
\begin{array}{l}
 \mathsf{updtI} = \efix \eapp  \mathsf{f}(\_). \lambda maxScc. \lambda sc. \lambda
  i. \lambda N. \\
 \hspace{.2cm} \eif   \Big (   (i < N)  ,  \\
\hspace{0.4cm}  \eif \big( ( ( \mathsf{nth} \eapp scc \eapp i)  >  maxScc  ) ,       \\
 \hspace{0.8cm}   i :: ( \mathsf{f}  \eapp () \eapp maxScc \eapp sc
  \eapp (i+1) \eapp N  )\\
 \hspace{0.8cm} , \mathsf{f}  \eapp () \eapp maxScc \eapp sc
  \eapp (i+1) \eapp N  \big )  \\
\hspace{0.2cm}   ,  [] \Big ) \\
\end{array}
\]
\end{figure}

\begin{figure}


UpdtSC:
\begin{mathpar}
      \inferrule*[right = FIX]
      {
        \inferrule*[right = FIX...]
        {
          \inferrule*[right = IF]
          {
            \inferrule*[right = BOOL]
            {
              \empty
            }
            {
             i:\tint , I : \tlist{\tint} , \Gamma \tvdash{} {i
               <  N : \tbool}
            }
            \\
            \inferrule*[right = IF]
            {
              \dots
            }
            {
              f:., \Gamma; \tvdash{}  \eif (i \in I)\cdots : \tlist{\treal }
            }
            \and
            \inferrule*[right = VAR]
            {
            }
            {
              f:., sc:. ,\Gamma \tvdash{} {sc : \tlist{\treal}}
            }
            \\
            \and
          }
          {
          f: ., sc: \tlist{\treal}, a:\tint, i:\tint \dots \Gamma
            \tvdash{ } \eif (i < N)  \cdots : \tlist{\treal}
          }
        }
        {
          f: \tunit \rightarrow \dots
          \tlist{\treal}, \Gamma \tvdash{} {\lambda
            sc. \dots \lambda N.
            \eif \cdots :  \tlist{\treal} \rightarrow \dots \tlist{\treal}   }
        }
      }
      {
       \Gamma \tvdash{} \efix \eapp  f( \_ ). \lambda sc. \lambda
        a. \dots \lambda N. \eif \Big ( (i <N), \dots, sc \Big ) : \tunit
        \rightarrow \tlist{\treal} \rightarrow \dots \rightarrow \tlist{\treal} 
      }

   \inferrule*[ right = IF ]
   {
     \inferrule
     {
     }
     {
       \Gamma \tvdash{ }. f \eapp () \eapp sc' \dots \eapp N : \tlist{\treal}
      }
     \and
     \inferrule
     {
     \dots
     }
     {
      \Gamma\tvdash{}  \elet x = \dots \ein \elet sc' = \dots
      \ein f \eapp () \eapp sc' \dots : \tlist{\treal}
    }
     \\
     i :\tint , I : \tlist{\tint},\Gamma \tvdash{} i \in I : \tbool
   }
   { \Gamma \tvdash{}  \eif \big(  (i \in I), \elet x = \dots  ,  f \eapp ()
     \eapp sc \dots N \big) : \tlist{\treal }    }

    
    \inferrule*[right = LET]
    {
    \inferrule*[right = LET]
        {
        \inferrule*[right = APP]
            {
            }
            {
                \Gamma \tvdash{ } \mathsf{updt} \eapp sc \eapp
                i \eapp x : \tlist{\treal}
            }
            \and
        \inferrule*[right = APP]
            {
                \dots
            }
            {
                \Gamma \tvdash{} f () \eapp sc \dots \eapp  (i+1) \eapp  N : \tlist{\treal }
            }
            \\
            {  }
            \and
            {}
        }
        {
            x: \treal ,\Gamma \tvdash{} 
             \elet sc' = \mathsf{updt} \eapp sc \eapp i \eapp x \ein \dots : \tlist{\treal}
        }
      \\
      \inferrule
        {
          \dots
        }
        {
            \Gamma \tvdash{} ( \mathsf{nth} \eapp sc \eapp i  )+ (a-p)*(q \eapp i -
      p) : \treal
        }
    }
    {
      \tvdash{} 
      \elet x =( \mathsf{nth} \eapp sc \eapp i  )+ (a-p)*(q \eapp i -
      p) \ein \dots: \tlist{\treal }
    }
    \end{mathpar}
\end{figure}



\end{document}
