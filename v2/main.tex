\documentclass[a4paper,11pt]{article}

\usepackage{mathpartir}
\usepackage{amsmath,amsthm,amsfonts}
\usepackage{color}

%%% Attempt 1: Linear 1



\newcommand{\diam}{{\color{red}\diamond}}
\newcommand{\dagg}{{\color{blue}\dagger}}
\let\oldstar\star
\renewcommand{\star}{\oldstar}

\newcommand{\im}[1]{\ensuremath{#1}}

\newcommand{\kw}[1]{\im{\mathtt{#1}}}


\newcommand{\set}[1]{\im{\{{#1}\}}}

\newcommand{\mmax}{\ensuremath{\mathsf{max}}}

%%%%%%%%%%%%%%%%%%%%%%%%%%%%%%%%%%%%%%%%%%%%%%%%%%%%%%%%
% Comments
\newcommand{\omitthis}[1]{}

% Misc.
\newcommand{\etal}{\textit{et al.}}
\newcommand{\bump}{\hspace{3.5pt}}

% Text fonts
\newcommand{\tbf}[1]{\textbf{#1}}
%\newcommand{\trm}[1]{\textrm{#1}}

% Math fonts
\newcommand{\mbb}[1]{\mathbb{#1}}
\newcommand{\mbf}[1]{\mathbf{#1}}
\newcommand{\mrm}[1]{\mathrm{#1}}
\newcommand{\mtt}[1]{\mathtt{#1}}
\newcommand{\mcal}[1]{\mathcal{#1}}
\newcommand{\mfrak}[1]{\mathfrak{#1}}
\newcommand{\msf}[1]{\mathsf{#1}}
\newcommand{\mscr}[1]{\mathscr{#1}}









\newcommand{\defeq}{\mathrel{\doteq}}
\newcommand{\conj}{\mathrel{\wedge}}
\newcommand{\disj}{\mathrel{\vee}}

\newcommand{\lzero}{0}


% context
\newcommand{\tctx}{\Gamma}
\newcommand{\ictx}{ }


% expression
\newcommand{\expr}{e}
\newcommand{\aexpr}{a}
\newcommand{\bexpr}{b}
\newcommand{\sexpr}{\textrm{e} }
\newcommand{\qexpr}{\psi}
\newcommand{\qval}{\alpha}
\newcommand{\query}{{\tt query}}
\newcommand{\saexpr}{\textrm{a} }
\newcommand{\sbexpr}{\textrm{b} }
\newcommand{\vall}{w}
\newcommand{\valr}{v}
\newcommand{\eif}{\kw{if}}
\newcommand{\eapp}{\;}
\newcommand{\eprojl}{\kw{fst}}
\newcommand{\eprojr}{\kw{snd}}
\newcommand{\eifvar}{\kw{ifvar}}
%expression and commands for WHILE language
\newcommand{\ewhile}{\kw{while}}
\newcommand{\bop}{*}
\newcommand{\uop}{\circ}
\newcommand{\eskip}{\kw{skip}}

\newcommand{\eloop}{\kw{loop}}
\newcommand{\edo}{\kw{do}}
\newcommand{\qdom}{\mathcal{QD}}

%configuration
\newcommand{\config}[1]{\langle #1 \rangle}
\newcommand{\ematch}{\kw{match}}
\newcommand{\clabel}[1]{\left[ #1 \right]}


%\newcommand{\eprov}[1]{\eta_{#1}}
\newcommand{\etrue}{\kw{true}}
\newcommand{\efalse}{\kw{false}}
\newcommand{\econst}{c}
\newcommand{\eop}{\delta}
\newcommand{\efix}{\mathop{\kw{fix}}}
\newcommand{\elet}{\mathop{\kw{let}}}
\newcommand{\ein}{\mathop{ \kw{in}} }
\newcommand{\eas}{\mathop{ \kw{as}} }
\newcommand{\enil}{\kw{nil}}
\newcommand{\econs}{\mathop{\kw{cons}}}
%\newcommand{\labelA}{\ell}
%monad expressions / terms
\newcommand{\term}{t}
\newcommand{\return}{\kw{return}}
\newcommand{\bernoulli}{\kw{bernoulli}}
\newcommand{\uniform}{\kw{uniform}}
 \newcommand{\epack}{\mbox{pack\;}}
\newcommand{\eunpack}{\mbox{unpack\;}}
\newcommand{\eilam}{\Lambda.}

\newcommand{\evec}{\kw{dict}}
\newcommand{\eget}{\kw{get}}

% trace
\newcommand{\triapp}[2]{\kw{IApp}(#1,#2)}
\newcommand{\trow}{\text{row}}
\newcommand{\tr}{T}
\newcommand{\trift}{\eif^{\kw{t}}}
\newcommand{\triff}{\eif^{\kw{f}}}
\newcommand{\trprojl}{\eprojl}
\newcommand{\trprojr}{\eprojr}
\newcommand{\trtrue}{\etrue}
\newcommand{\trfalse}{\efalse}
\newcommand{\trconst}{\econst}
\newcommand{\trop}{\eop}
\newcommand{\trfix}{\efix}
\newcommand{\trapp}[5]{#1 \; #2 \mathrel{\triangleright} {\efix
#3(#4).#5}}
\newcommand{\trnil}{\enil}
\newcommand{\trcons}{\econs}
\newcommand{\trlet}{\elet}
%types for monad
\newcommand{\treal}{\kw{real}}
\newcommand{\tint}{\kw{int}}
\newcommand{\tmonad}{\kw{M}}
\newcommand{\tunit}{\kw{unit}}
\newcommand{\tdb}{\kw{tdb}}

% adaptivity
\newcommand{\adap}{\kw{adap}}
\newcommand{\ddep}[1]{\kw{depth}_{#1}}
\newcommand{\nat}{\mathbb{N}}
\newcommand{\natb}{\nat_{\bot}}
\newcommand{\natbi}{\natb^\infty}
\newcommand{\nnatA}{Z}
\newcommand{\nnatB}{m}
\newcommand{\nnatbA}{s}
\newcommand{\nnatbB}{t}
\newcommand{\nnatbiA}{q}
\newcommand{\nnatbiB}{r}

%type
\newcommand{\type}{\tau}
\newcommand{\tbase}{\kw{b}}
\newcommand{\tbool}{\kw{bool}}
\newcommand{\tbox}[1]{ \kw{\square} \, (#1) }
\newcommand{\tarr}[5]{#1; #3 \xrightarrow{#4; \, #5} #2}
\newcommand{\tlist}[1]{\kw{list} \, #1 }
\newcommand{\env}{\theta}
\newcommand{\tforall}[3]{\forall#3 \overset{#1, #2}{::} S.\, }
\newcommand{\texists}[1]{\exists#1 {::} S.\, }
\newcommand{\lto}{\multimap}
\newcommand{\bang}[1]{ !_{#1}}
\newcommand{\whynot}[1]{ ?_{#1} }
\newcommand{\ltype}{A}
\newcommand{\adapt}{R}
% index
\newcommand{\idx}{I }
\newcommand{\smax}[2]{\kw{max}(#1,#2)}
\newcommand{\ienv}{\sigma}

%evaluation
\newcommand{\bigstep}[1]{\mathrel{\to^{#1}}}

\newcommand{\dmap}{\rho}
\newcommand{\dmapb}{\bot_\dmap}
\newcommand{\supp}{\kw{supp}}
\newcommand{\dom}{\kw{dom}}
\newcommand{\codom}{\kw{codom}}

\newcommand{\tvdash}[1]{\vdash_{#1}}

\newcommand{\lrv}[1]{[\![ #1 ]\!]_{\text{V}}}
\newcommand{\lre}[3]{[\![ #3 ]\!]_{\text{E}}^{#1, #2}}
\newcommand{\stepiA}{k}
\newcommand{\stepiB}{j}
\newcommand{\size}[1]{|#1|}

%logic relations
\newcommand{\lr}[1]{[\![ #1 ]\!]}
\newcommand{\lrt}[1]{\mathcal{T}[\![ #1 ]\!]}


\newcommand{\wf}[1]{\vdash #1 \, \kw{wf} }
\newcommand{\sub}[2]{ #1 \, <: \, #2 }
\newcommand{\eqv}[3]{ #1 \, \equiv \, #2 \Rightarrow \textcolor{red}
{#3}  }
\newcommand{\eqvt}[3]{ #1 \, \sqsubseteq \, #2 \Rightarrow \textcolor{red}
{#3}  }
\newcommand{\eqvc}[2]{ #1 \, \equiv^c \, #2   }


%core calculus
\newcommand{\ctyping}[3]{ \tvdash{ #1} {#2} :^c #3 }
\newcommand{\cbox}{\mathsf{box}}
\newcommand{\cder}{\mathsf{der}}
\newcommand{\elab}[4]{ \vdash_{ #1} #2 \rightsquigarrow #3 : #4}
\newcommand{\coerce}[2]{\mathsf{coerce}_{#1, #2}}

%algorithmic typing rules
\newcommand{\infr}[4]{{#1} ~ {\textcolor{red}\uparrow} ~ {\color{red} #2} \Rightarrow
{ } {\color{red} #3} }
\newcommand{\chec}[3]{{#1} ~ {\downarrow} ~ {#2} \Rightarrow {\color{red} #3} }
% \newcommand{\restriction}{\Phi}
\newcommand{\fresh}{ \mathsf{fresh}}
\newcommand{\red}[1]{ \textcolor{red} {#1} }
\newcommand{\fiv}[1]{ \mathsf{FIV} (#1)   }
\newcommand{\fv}[1]{ \mathsf{FV} (#1)   }

\newcommand{\todo}[1]{{\small \color{red}\textbf{[[ #1 ]]}}}
\newcommand{\todomath}[1]{{\scriptstyle \color{red}\mathbf{[[ #1 ]]}}}

\newcommand{\caseL}[1]{\item \textbf{#1}\newline}

\newcommand{\attr}{\mathsf{attr}}
\newcommand{\weight}{\mathsf{W}}
\newcommand{\num}{\mathsf{n}}

\usepackage{enumitem}
\setenumerate{listparindent=\parindent}

\newlist{enumih}{enumerate}{3}
\setlist[enumih]{label=\alph*),before=\raggedright, topsep=1ex, parsep=0pt,  itemsep=1pt }

\newlist{enumconc}{enumerate}{3}
\setlist[enumconc]{leftmargin=0.5cm, label*= \arabic*.  , topsep=1ex, parsep=0pt,  itemsep=3pt }


\newlist{enumsub}{enumerate}{3}
\setlist[enumsub]{ leftmargin=0.7cm, label*= \textbf{subcase} \bf \arabic*: }

\newlist{enumsubsub}{enumerate}{3}
\setlist[enumsubsub]{ leftmargin=0.5cm, label*= \textbf{subsubcase} \bf \arabic*: }

\newlist{mainitem}{itemize}{3}
\setlist[mainitem]{ leftmargin=0cm , label= {\bf Case} }

%%%%COLORS
\definecolor{periwinkle}{rgb}{0.8, 0.8, 1.0}
\definecolor{powderblue}{rgb}{0.69, 0.88, 0.9}
\definecolor{sandstorm}{rgb}{0.93, 0.84, 0.25}
\definecolor{trueblue}{rgb}{0.0, 0.45, 0.81}


\usepackage{array}

\newlength\Origarrayrulewidth

% horizontal rule equivalent to \cline but with 2pt width
\newcommand{\Cline}[1]{%
 \noalign{\global\setlength\Origarrayrulewidth{\arrayrulewidth}}%
 \noalign{\global\setlength\arrayrulewidth{2pt}}\cline{#1}%
 \noalign{\global\setlength\arrayrulewidth{\Origarrayrulewidth}}%
}

% draw a vertical rule of width 2pt on both sides of a cell
\newcommand\Thickvrule[1]{%
  \multicolumn{1}{!{\vrule width 2pt}c!{\vrule width 2pt}}{#1}%
}

% draw a vertical rule of width 2pt on the left side of a cell
\newcommand\Thickvrulel[1]{%
  \multicolumn{1}{!{\vrule width 2pt}c|}{#1}%
}

% draw a vertical rule of width 2pt on the right side of a cell
\newcommand\Thickvruler[1]{%
  \multicolumn{1}{|c!{\vrule width 2pt}}{#1}%
}

\newcommand{\command}{c}
\newcommand{\green}[1]{{ \color{green} #1 } }

\newcommand{\func}[2]{\mathsf{AD}(#1) \to (#2)}
\newcommand{\varEst}{\bf{VetxEst}}
\newcommand{\graphGen}{\bf{GraphGen}}

\newcommand{\ag}[2]{\mathsf{VetxEst}{(#1)}\to {(#2)}}
\newcommand{\ad}[2]{\mathsf{GraphGen}{(#1)}\to {(#2)}}
\newcommand{\rb}{\mathsf{RechBound}}
\newcommand{\pathsearch}{\mathsf{AdaptPathSearch}}


\theoremstyle{definition}
\newtheorem{thm}{Theorem}
\newtheorem{lem}[thm]{Lemma}
\newtheorem{cor}[thm]{Corollary}
\newtheorem{prop}[thm]{Proposition}


\title{Adaptivity analysis}

\author{}

\date{January 19, 2019}

\begin{document}

\maketitle


\paragraph{Language}
Expressions are shown below. $\econst$ denotes constants (of some base
type $\tbase$, which may, for example, be reals or rational
numbers). $\eop$ represents a primitive operation (such as a
mechanism). We restrict $\eop$ to base types ($\tbase$, $\tbool$)
only. We make environments explicit in closures. This is needed for
the tracing semantics later.
\[\begin{array}{llll}
\mbox{Expr.} & \expr & ::= & x ~|~ \expr_1 \eapp \expr_2 ~|~ \efix f(x).\expr
 ~|~ (\expr_1, \expr_2) ~|~ \eprojl(\expr) ~|~ \eprojr(\expr) ~| \\
%
& & & \etrue ~|~ \efalse ~|~ \eif(\expr_1, \expr_2, \expr_3) ~|~
\econst ~|~ \eop(\expr) \\
%
\mbox{Value} & \valr & ::= & \etrue ~|~ \efalse ~|~ \econst ~|~
(\efix f(x).\expr, \env) ~|~ (\valr_1, \valr_2) \\
%
\mbox{Environment} & \env & ::= & x_1 \mapsto \valr_1, \ldots, x_n \mapsto \valr_n
\end{array}\]

\paragraph{Traces}
A trace is a representation of the big-step derivation of an
expression's evaluation. Our big-step semantics output a trace. We use
traces to define the adaptivity of a run. Our notion of traces and the
tracing semantics is taken from~\cite[Section 4]{perera:dep}, but we
omit their ``holes'' for which we have no need. The construct
$\trapp{\tr_1}{\tr_2}{f}{x}{\tr_3}$ records a trace of function
application. $\tr_1$ is the trace of the head, $\tr_2$ the trace that
evaluates the argument and $\tr_3$ is the trace of the function
body. $f$ and $x$ are bound in $\tr_3$.
%
\[\begin{array}{llll}
\mbox{Trace} & \tr & ::= & x ~|~ \trapp{\tr_1}{\tr_2}{f}{x}{\tr_3} ~|~
\trfix f(x).e ~|~ (\tr_1, \tr_2) ~|~ \trprojl(\tr) ~|\\ 
%
& & & \trprojr(\tr) ~|~ \trtrue ~|~ \trfalse ~|~ \trift(\tr_b, \tr_t)
~|~ \triff(\tr_b, \tr_f) ~|~ \trconst ~|~ \trop(\tr)
\end{array}\]


\paragraph{Big-step tracing semantics}
The big-step, tracing semantics $\env, \expr \bigstep \valr, \tr$
computes a value $\valr$ and a trace $\tr$ from an expression $\expr$
and an enviroment $\env$ which maps the free variables of $\expr$ to
values. The rules, taken from~\cite{perera:dep}, are shown in
Figure~\ref{fig:big-step}. Some salient points:
\begin{itemize}
\item[-] Erasing the traces from the semantics yields a standard
  big-step semantics.
\item[-] The trace of a primitive application $\eop(\expr)$
  records that $\eop$ was applied to the trace of
  $\expr$. This enables us to define adaptivity from a trace later.
\item[-] The trace of a variable $x$ is $x$. Hence, traces record
  where substitutions occur and, hence, variable dependencies. This is
  also needed for defining adaptivity.
\end{itemize}

\begin{figure}
\begin{mathpar}
  \inferrule{ }{\env, x \bigstep \env(x), x}
  %
  \and
  %
  \inferrule{ }{\env, \econst \bigstep \econst, \trconst}
  %
  \and
  %
  \inferrule{ }{\env, \etrue \bigstep \etrue, \trtrue}
  %
  \and
  %
  \inferrule{ }{\env, \efalse \bigstep \efalse, \trfalse}
  %
  \and
  %
  \inferrule{
  }{
    \env, \efix f(x). \expr \bigstep (\efix f(x).\expr, \env), \trfix f(x).\expr
  }
  %
  \and
  %
  \inferrule{
    \env, \expr_1 \bigstep \valr_1, \tr_1 \\
    \valr_1 = (\efix f(x).\expr, \env') \\\\
    \env, \expr_2 \bigstep \valr_2, \tr_2 \\
    \env'[f \mapsto \valr_1, x \mapsto \valr_2], \expr \bigstep \valr, \tr
  }{
    \env, \expr_1 \eapp \expr_2 \bigstep \valr, \trapp{\tr_1}{\tr_2}{f}{x}{\tr}
  }
  %
  \and
  %
  \inferrule{
    \env, \expr_1 \bigstep \valr_1, \tr_1 \\
    \env, \expr_2 \bigstep \valr_2, \tr_2
  }{
    \env, (\expr_1, \expr_2) \bigstep (\valr_1, \valr_2), (\tr_1, \tr_2)
  }
  %
  \and
  %
  \inferrule{
    \env, \expr \bigstep (\valr_1, \valr_2), \tr
  }{
    \env, \eprojl(\expr) \bigstep \valr_1, \trprojl(\tr)
  }
  %
  \and
  %
  \inferrule{
    \env, \expr \bigstep (\valr_1, \valr_2), \tr
  }{
    \env, \eprojr(\expr) \bigstep \valr_2, \trprojr(\tr)
  }
  %
  \and
  %
  \inferrule{
    \env, \expr \bigstep \etrue, \tr \\
    \env, \expr_1 \bigstep \valr, \tr_1
  }{
    \env, \eif(\expr, \expr_1, \expr_2) \bigstep \valr, \trift(\tr, \tr_1)
  }
  %
  \and
  %
  \inferrule{
    \env, \expr \bigstep \efalse, \tr \\
    \env, \expr_2 \bigstep \valr, \tr_2
  }{
    \env, \eif(\expr, \expr_1, \expr_2) \bigstep \valr, \triff(\tr, \tr_2)
  }
  %
  \and
  %
  \inferrule{
    \env, \expr \bigstep \valr, \tr \\
    \eop{}(\valr) = \valr'
  }{
    \env, \eop(\expr) \bigstep \valr', \trop(\tr)
  }
\end{mathpar}
  
  \caption{Big-step semantics with provenance}
  \label{fig:big-step}
\end{figure}


\paragraph{Adaptivity of a trace}
We define the \emph{adaptivity} of a trace $\tr$, $\adap(\tr)$, which
means the maximum number of nested $\eop$s in $\tr$, taking variable
and control dependencies into account. To define this, we need an
auxiliary notion called the \emph{depth of variable $x$} in trace
$\tr$, written $\ddep{x}(\tr)$, which is the maximum number of
$\eop{}$s in any path leading from the root of $\tr$ to an occurence
of $x$ (at a leaf), again taking variable and control dependencies
into account.

If $x$ does not appear free in $\tr$, $\ddep{x}(\tr)$ is $\bot$, a
special symbol. Let $\natb \defeq \nat \cup \{\bot\}$. We extend
addition ($+$) on $\nat$ to $\natb$ by defining $n + \bot = \bot + n =
\bot$. Similarly, we extend the $\max$ operator to $\natb$ by defining
$\max(\bot, n) = \max(n, \bot) = n$. So, $\bot$ behaves exactly like
$-\infty$.

The functions $\adap$ and $\ddep{x}$ are defined by mutual induction
in Figure~\ref{fig:adap}. Note the co-domains of these functions.

\begin{figure}
  \framebox{$\adap: \mbox{Traces} \to \nat$}
  \begin{mathpar}
    \begin{array}{lcl}
      \adap(x) & = & 0 \\
      %
      \adap(\trapp{\tr_1}{\tr_2}{f}{x}{\tr_3}) & = &
      \adap(\tr_1) + \max(\adap(\tr_3), \adap(\tr_2) + \ddep{x}(\tr_3))\\
      %
      \adap(\trfix f(x).\expr) & = & 0 \\
      %
      \adap((\tr_1, \tr_2)) & = & \max(\adap(\tr_1), \adap(\tr_2)) \\
      %
      \adap(\trprojl(\tr)) & = & \adap(\tr) \\
      %
      \adap(\trprojr(\tr)) & = & \adap(\tr) \\
      %
      \adap(\trtrue) & = & 0 \\
      %
      \adap(\trfalse) & = & 0 \\
      %
      \adap(\trift(\tr_b, \tr_t)) & = & \adap(\tr_b) + \adap(\tr_t) \\
      %
      \adap(\triff(\tr_b, \tr_f)) & = & \adap(\tr_b) + \adap(\tr_f) \\
      %
      \adap(\trconst) & = & 0 \\
      %
      \adap(\trop(\tr)) & = & 1 + \adap(\tr)
      \end{array}
  \end{mathpar}
  
  \framebox{$\ddep{x}: \mbox{Traces} \to \natb$}
  \begin{mathpar}
    \begin{array}{lcl}
      \ddep{x}(y) & = &
      \left\lbrace
      \begin{array}{ll}
        0 & \mbox{if } x = y \\
        \bot & \mbox{if } x \neq y
      \end{array}
      \right.\\
      %
      \ddep{x}(\trapp{\tr_1}{\tr_2}{f}{y}{\tr_3}) & = & \adap(\tr_1) + \max(\ddep{x}(\tr_3), \ddep{x}(\tr_2) + \ddep{y}(\tr_3)) \\
      %
      \ddep{x}(\trfix f(y).\expr) & = & \bot \\
      %
      \ddep{x}((\tr_1, \tr_2)) & = & \max(\ddep{x}(\tr_1), \ddep{x}(\tr_2)) \\
      %
      \ddep{x}(\trprojl(\tr)) & = & \ddep{x}(\tr) \\
      %
      \ddep{x}(\trprojr(\tr)) & = & \ddep{x}(\tr) \\
      %
      \ddep{x}(\trtrue) & = & \bot \\
      %
      \ddep{x}(\trfalse) & = & \bot \\
      %
      \ddep{x}(\trift(\tr_b, \tr_t)) & = & \max(\ddep{x}(\tr_b), \adap(\tr_b) + \ddep{x}(\tr_t)) \\
      %
      \ddep{x}(\trift(\tr_b, \tr_f)) & = & \max(\ddep{x}(\tr_b), \adap(\tr_b) + \ddep{x}(\tr_f)) \\
      %
      \ddep{x}(\trconst) & = & \bot \\
      %
      \ddep{x}(\trop(\tr)) & = & 1 + \ddep{x}(\tr)
    \end{array}
  \end{mathpar}
  \caption{Adaptivity of a trace and $\delta$-depth of variable $x$ in a trace}
  \label{fig:adap}
\end{figure}

\paragraph{Explanation of $\adap$}
We explain the interesting cases of the definition of $\adap$. The
case $\trapp{\tr_1}{\tr_2}{f}{x}{\tr_3}$ corresponds to a function
application with $\tr_1$, $\tr_2$, $\tr_3$ being the traces of the
head, the argument and the body, respectively, and $x$ being the
argument. The adaptivity is defined to be $\adap(\tr_1) +
\max(\adap(\tr_3), \adap(\tr_2) + \ddep{x}(\tr_3))$. The term
$\adap(\tr_1)$ occurs additively since the entire computation is
control-dependent on the function the head of the application
evaluates to. The rest of the term $\max(\adap(\tr_3), \adap(\tr_2) +
\ddep{x}(\tr_3))$ is simply the maximum nesting depth in the body,
taking the data dependency on the argument into account. To see this,
consider the following exhaustive cases:
\begin{itemize}
  \item[-] When $x$ appears free in the trace $\tr_3$,
    $\ddep{x}(\tr_3)$ is the maximum $\eop$-nesting depth of $x$ in
    the body. Hence, $\max(\adap(\tr_3), \adap(\tr_2) +
    \ddep{x}(\tr_3))$ represents the maximum number of nested $\eop$s
    in the evaluation of $e[e'/x]$ where $e'$ is the argument
    expression that generates the trace $\tr_2$ and $e$ is the body of
    the function.
  \item[-] When $x$ does not appear free in the trace $\tr_3$ of the
    body (i.e., the body's evaluation does not depend on $x$),
    $\ddep{x}(\tr_3) = \bot$, so $\max(\adap(\tr_3), \adap(\tr_2) +
    \ddep{x}(\tr_3)) = \max(\adap(\tr_3), \adap(\tr_2) + \bot) =
    \max(\adap(\tr_3), \bot) = \adap(\tr_3)$, which is the adaptivity
    of the body in the absence of dependency from $x$.
\end{itemize}

The case $\trift(\tr_b, \tr_t)$ corresponds to the evaluation of
$\eif(\expr_b, \expr_t, \_)$ where $\expr_b$ evaluates to $\etrue$
with trace $\tr_b$ and $\tr_t$ is the trace of $\expr_t$. In this
case, since the entire evaluation of $\expr_t$ is control dependent on
$\expr_b$, the adaptivity is simply $\adap(\tr_b) + \adap(\tr_t)$.

\paragraph{Explanation of $\ddep{x}$}
We explain interesting cases in the definition of $\ddep{x}$.  For the
trace $\trapp{\tr_1}{\tr_2}{f}{x}{\tr_3}$, $\ddep{x}$ is defined as
$\adap(\tr_1) + \max(\ddep{x}(\tr_3), \ddep{x}(\tr_2) +
\ddep{y}(\tr_3))$. Here, $\max(\ddep{x}(\tr_3), \ddep{x}(\tr_2) +
\ddep{y}(\tr_3))$ is the maximum depth of $x$ in the body ($\tr_3$),
taking the dependency on the argument into account. Specifically, when
the argument variable $y$ is not used in the body, $\ddep{y}(\tr_3) =
\bot$, and this term is $\ddep{x}(\tr_3)$.  The term $\adap(\tr_1)$ is
added since the body's entire execution is control-flow dependent on
the function that the head of the application evaluates to.

For the trace $\trift(\tr_b, \tr_t)$, $\ddep{x}$ is defined as
$\max(\ddep{x}(\tr_b), \adap(\tr_b) + \ddep{x}(\tr_t))$. The term
$\ddep{x}(\tr_b)$ is simply the maximum depth of $x$ in $\tr_b$. We
take the $\max$ of this with $\adap(\tr_b) + \ddep{x}(\tr_t)$, the
maximum depth of $x$ in $\tr_t$, taking the control dependency on
$\tr_b$ into account. Note that when $x$ is not used in $\tr_t$, then
$\ddep{x}(\tr_t) = \bot$ and $\ddep{x}(\trift(\tr_b, \tr_t)) =
\ddep{x}(\tr_b)$.

\begin{lem}\label{lem:ddep-leq-adap}
  If we define $\bot \leq n$ for all $n \in \natb$, then for all $\tr$
  and $x$, $\ddep{x}(\tr) \leq \adap(\tr)$.
\end{lem}
%
\begin{proof}
By easy induction on $\tr$, following the definitions of $\ddep{x}$
and $\adap$.
\end{proof}

\paragraph{Remark}
At first glance it may seem that Lemma~\ref{lem:ddep-leq-adap} can be
used to simplify the definition of $\ddep{x}(\trift(\tr_b, \tr_t))$
from $\max(\ddep{x}(\tr_b), \adap(\tr_b) + \ddep{x}(\tr_t))$ to
$\adap(\tr_b) + \ddep{x}(\tr_t)$ since $\ddep{x}(\tr_b) \leq
\adap(\tr_b)$. However, this simplification is not correct, since
$\ddep{x}(\tr_t)$ may be $\bot$. In that case, $\max(\ddep{x}(\tr_b),
\adap(\tr_b) + \ddep{x}(\tr_t))$ equals $\ddep{x}(\tr_b)$ while
$\adap(\tr_b) + \ddep{x}(\tr_t)$ equals $\bot$.

More generally, since $\bot$ behaves like $-\infty$, we do not have
the implication $a \leq b \Rightarrow a \leq b + c$ as $c$ may be
$-\infty$ or $\bot$. As a result, $a \leq b$ does not imply $\max(a, b
+ c) = b + c$.


%%%%%%%%%%%%%%%%%%%%%

\if 0

\paragraph{Types}
We partition types into \emph{labeled} (denoted $\typel$) and
\emph{raw} (denoted $\typeu$). A raw type is like a standard type. A
labeled type pairs a raw type with a \emph{label} $\labelA$, which is
drawn from $\nat \cup \{\infty\}$. The labeled type
$\typeu^\labelA$ ascribes values of (raw) type $\typeu$ that were
computed with adaptivity $\labelA$.

The actual syntax of labeled and raw types is mutually recursive to
allow labeling nested subtypes, e.g., in pairs. The syntax is shown
below. $\tbool$ is the boolean type. $\tbase$ denotes any other base
type. $\labelA$, also called a \emph{label}, is an element of
%
\[\begin{array}{llll}
\mbox{Unlabeled Type} & \typeu & ::= & \tbase ~|~ \tbool ~|~ \typel_1
\to \typel_2 ~|~ \typel_1 \times \typel_2 \\
%
\mbox{Raw Type} & \typel & ::= & \typeu^{\labelA}
\end{array}\]


\paragraph{Type system}
The type system uses the judgment $\Gamma \vdash \expr: \typel$, where
$\Gamma$ maps variables to \emph{labeled} types. For $\typel =
\typeu^\labelA$, the meaning of this judgment is that, under a
substitution for $\Gamma$ of the right types, if $e$ terminates, then
the result is a labeled value $\valr^{\labelA'}$ where $\valr$ is
morally of type $\typeu$\footnote{This is not a precise statement
  because we don't syntactically define when a raw value is of an
  unlabeled type.} and $\labelA' \leq \labelA$. The typing rules are
shown in Figure~\ref{fig:type-rules}.

\begin{figure}
  \begin{mathpar}

  \end{mathpar}

  \caption{Typing rules}
  \label{fig:type-rules}
\end{figure}

\fi

\bibliographystyle{plain}
\bibliography{adaptivity.bib}

\end{document}



