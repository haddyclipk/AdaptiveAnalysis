\documentclass[a4paper,11pt]{article}

\usepackage{mathpartir}
\usepackage{amsmath,amsthm,amsfonts}
\usepackage{color}
\usepackage{algorithm}
\usepackage{algorithmic}

%%% Attempt 1: Linear 1



\newcommand{\diam}{{\color{red}\diamond}}
\newcommand{\dagg}{{\color{blue}\dagger}}
\let\oldstar\star
\renewcommand{\star}{\oldstar}

\newcommand{\im}[1]{\ensuremath{#1}}

\newcommand{\kw}[1]{\im{\mathtt{#1}}}


\newcommand{\set}[1]{\im{\{{#1}\}}}

\newcommand{\mmax}{\ensuremath{\mathsf{max}}}

%%%%%%%%%%%%%%%%%%%%%%%%%%%%%%%%%%%%%%%%%%%%%%%%%%%%%%%%
% Comments
\newcommand{\omitthis}[1]{}

% Misc.
\newcommand{\etal}{\textit{et al.}}
\newcommand{\bump}{\hspace{3.5pt}}

% Text fonts
\newcommand{\tbf}[1]{\textbf{#1}}
%\newcommand{\trm}[1]{\textrm{#1}}

% Math fonts
\newcommand{\mbb}[1]{\mathbb{#1}}
\newcommand{\mbf}[1]{\mathbf{#1}}
\newcommand{\mrm}[1]{\mathrm{#1}}
\newcommand{\mtt}[1]{\mathtt{#1}}
\newcommand{\mcal}[1]{\mathcal{#1}}
\newcommand{\mfrak}[1]{\mathfrak{#1}}
\newcommand{\msf}[1]{\mathsf{#1}}
\newcommand{\mscr}[1]{\mathscr{#1}}









\newcommand{\defeq}{\mathrel{\doteq}}
\newcommand{\conj}{\mathrel{\wedge}}
\newcommand{\disj}{\mathrel{\vee}}

\newcommand{\lzero}{0}


% context
\newcommand{\tctx}{\Gamma}
\newcommand{\ictx}{ }


% expression
\newcommand{\expr}{e}
\newcommand{\aexpr}{a}
\newcommand{\bexpr}{b}
\newcommand{\sexpr}{\textrm{e} }
\newcommand{\qexpr}{\psi}
\newcommand{\qval}{\alpha}
\newcommand{\query}{{\tt query}}
\newcommand{\saexpr}{\textrm{a} }
\newcommand{\sbexpr}{\textrm{b} }
\newcommand{\vall}{w}
\newcommand{\valr}{v}
\newcommand{\eif}{\kw{if}}
\newcommand{\eapp}{\;}
\newcommand{\eprojl}{\kw{fst}}
\newcommand{\eprojr}{\kw{snd}}
\newcommand{\eifvar}{\kw{ifvar}}
%expression and commands for WHILE language
\newcommand{\ewhile}{\kw{while}}
\newcommand{\bop}{*}
\newcommand{\uop}{\circ}
\newcommand{\eskip}{\kw{skip}}

\newcommand{\eloop}{\kw{loop}}
\newcommand{\edo}{\kw{do}}
\newcommand{\qdom}{\mathcal{QD}}

%configuration
\newcommand{\config}[1]{\langle #1 \rangle}
\newcommand{\ematch}{\kw{match}}
\newcommand{\clabel}[1]{\left[ #1 \right]}


%\newcommand{\eprov}[1]{\eta_{#1}}
\newcommand{\etrue}{\kw{true}}
\newcommand{\efalse}{\kw{false}}
\newcommand{\econst}{c}
\newcommand{\eop}{\delta}
\newcommand{\efix}{\mathop{\kw{fix}}}
\newcommand{\elet}{\mathop{\kw{let}}}
\newcommand{\ein}{\mathop{ \kw{in}} }
\newcommand{\eas}{\mathop{ \kw{as}} }
\newcommand{\enil}{\kw{nil}}
\newcommand{\econs}{\mathop{\kw{cons}}}
%\newcommand{\labelA}{\ell}
%monad expressions / terms
\newcommand{\term}{t}
\newcommand{\return}{\kw{return}}
\newcommand{\bernoulli}{\kw{bernoulli}}
\newcommand{\uniform}{\kw{uniform}}
 \newcommand{\epack}{\mbox{pack\;}}
\newcommand{\eunpack}{\mbox{unpack\;}}
\newcommand{\eilam}{\Lambda.}

\newcommand{\evec}{\kw{dict}}
\newcommand{\eget}{\kw{get}}

% trace
\newcommand{\triapp}[2]{\kw{IApp}(#1,#2)}
\newcommand{\trow}{\text{row}}
\newcommand{\tr}{T}
\newcommand{\trift}{\eif^{\kw{t}}}
\newcommand{\triff}{\eif^{\kw{f}}}
\newcommand{\trprojl}{\eprojl}
\newcommand{\trprojr}{\eprojr}
\newcommand{\trtrue}{\etrue}
\newcommand{\trfalse}{\efalse}
\newcommand{\trconst}{\econst}
\newcommand{\trop}{\eop}
\newcommand{\trfix}{\efix}
\newcommand{\trapp}[5]{#1 \; #2 \mathrel{\triangleright} {\efix
#3(#4).#5}}
\newcommand{\trnil}{\enil}
\newcommand{\trcons}{\econs}
\newcommand{\trlet}{\elet}
%types for monad
\newcommand{\treal}{\kw{real}}
\newcommand{\tint}{\kw{int}}
\newcommand{\tmonad}{\kw{M}}
\newcommand{\tunit}{\kw{unit}}
\newcommand{\tdb}{\kw{tdb}}

% adaptivity
\newcommand{\adap}{\kw{adap}}
\newcommand{\ddep}[1]{\kw{depth}_{#1}}
\newcommand{\nat}{\mathbb{N}}
\newcommand{\natb}{\nat_{\bot}}
\newcommand{\natbi}{\natb^\infty}
\newcommand{\nnatA}{Z}
\newcommand{\nnatB}{m}
\newcommand{\nnatbA}{s}
\newcommand{\nnatbB}{t}
\newcommand{\nnatbiA}{q}
\newcommand{\nnatbiB}{r}

%type
\newcommand{\type}{\tau}
\newcommand{\tbase}{\kw{b}}
\newcommand{\tbool}{\kw{bool}}
\newcommand{\tbox}[1]{ \kw{\square} \, (#1) }
\newcommand{\tarr}[5]{#1; #3 \xrightarrow{#4; \, #5} #2}
\newcommand{\tlist}[1]{\kw{list} \, #1 }
\newcommand{\env}{\theta}
\newcommand{\tforall}[3]{\forall#3 \overset{#1, #2}{::} S.\, }
\newcommand{\texists}[1]{\exists#1 {::} S.\, }
\newcommand{\lto}{\multimap}
\newcommand{\bang}[1]{ !_{#1}}
\newcommand{\whynot}[1]{ ?_{#1} }
\newcommand{\ltype}{A}
\newcommand{\adapt}{R}
% index
\newcommand{\idx}{I }
\newcommand{\smax}[2]{\kw{max}(#1,#2)}
\newcommand{\ienv}{\sigma}

%evaluation
\newcommand{\bigstep}[1]{\mathrel{\to^{#1}}}

\newcommand{\dmap}{\rho}
\newcommand{\dmapb}{\bot_\dmap}
\newcommand{\supp}{\kw{supp}}
\newcommand{\dom}{\kw{dom}}
\newcommand{\codom}{\kw{codom}}

\newcommand{\tvdash}[1]{\vdash_{#1}}

\newcommand{\lrv}[1]{[\![ #1 ]\!]_{\text{V}}}
\newcommand{\lre}[3]{[\![ #3 ]\!]_{\text{E}}^{#1, #2}}
\newcommand{\stepiA}{k}
\newcommand{\stepiB}{j}
\newcommand{\size}[1]{|#1|}

%logic relations
\newcommand{\lr}[1]{[\![ #1 ]\!]}
\newcommand{\lrt}[1]{\mathcal{T}[\![ #1 ]\!]}


\newcommand{\wf}[1]{\vdash #1 \, \kw{wf} }
\newcommand{\sub}[2]{ #1 \, <: \, #2 }
\newcommand{\eqv}[3]{ #1 \, \equiv \, #2 \Rightarrow \textcolor{red}
{#3}  }
\newcommand{\eqvt}[3]{ #1 \, \sqsubseteq \, #2 \Rightarrow \textcolor{red}
{#3}  }
\newcommand{\eqvc}[2]{ #1 \, \equiv^c \, #2   }


%core calculus
\newcommand{\ctyping}[3]{ \tvdash{ #1} {#2} :^c #3 }
\newcommand{\cbox}{\mathsf{box}}
\newcommand{\cder}{\mathsf{der}}
\newcommand{\elab}[4]{ \vdash_{ #1} #2 \rightsquigarrow #3 : #4}
\newcommand{\coerce}[2]{\mathsf{coerce}_{#1, #2}}

%algorithmic typing rules
\newcommand{\infr}[4]{{#1} ~ {\textcolor{red}\uparrow} ~ {\color{red} #2} \Rightarrow
{ } {\color{red} #3} }
\newcommand{\chec}[3]{{#1} ~ {\downarrow} ~ {#2} \Rightarrow {\color{red} #3} }
% \newcommand{\restriction}{\Phi}
\newcommand{\fresh}{ \mathsf{fresh}}
\newcommand{\red}[1]{ \textcolor{red} {#1} }
\newcommand{\fiv}[1]{ \mathsf{FIV} (#1)   }
\newcommand{\fv}[1]{ \mathsf{FV} (#1)   }

\newcommand{\todo}[1]{{\small \color{red}\textbf{[[ #1 ]]}}}
\newcommand{\todomath}[1]{{\scriptstyle \color{red}\mathbf{[[ #1 ]]}}}

\newcommand{\caseL}[1]{\item \textbf{#1}\newline}

\newcommand{\attr}{\mathsf{attr}}
\newcommand{\weight}{\mathsf{W}}
\newcommand{\num}{\mathsf{n}}

\usepackage{enumitem}
\setenumerate{listparindent=\parindent}

\newlist{enumih}{enumerate}{3}
\setlist[enumih]{label=\alph*),before=\raggedright, topsep=1ex, parsep=0pt,  itemsep=1pt }

\newlist{enumconc}{enumerate}{3}
\setlist[enumconc]{leftmargin=0.5cm, label*= \arabic*.  , topsep=1ex, parsep=0pt,  itemsep=3pt }


\newlist{enumsub}{enumerate}{3}
\setlist[enumsub]{ leftmargin=0.7cm, label*= \textbf{subcase} \bf \arabic*: }

\newlist{enumsubsub}{enumerate}{3}
\setlist[enumsubsub]{ leftmargin=0.5cm, label*= \textbf{subsubcase} \bf \arabic*: }

\newlist{mainitem}{itemize}{3}
\setlist[mainitem]{ leftmargin=0cm , label= {\bf Case} }

%%%%COLORS
\definecolor{periwinkle}{rgb}{0.8, 0.8, 1.0}
\definecolor{powderblue}{rgb}{0.69, 0.88, 0.9}
\definecolor{sandstorm}{rgb}{0.93, 0.84, 0.25}
\definecolor{trueblue}{rgb}{0.0, 0.45, 0.81}


\usepackage{array}

\newlength\Origarrayrulewidth

% horizontal rule equivalent to \cline but with 2pt width
\newcommand{\Cline}[1]{%
 \noalign{\global\setlength\Origarrayrulewidth{\arrayrulewidth}}%
 \noalign{\global\setlength\arrayrulewidth{2pt}}\cline{#1}%
 \noalign{\global\setlength\arrayrulewidth{\Origarrayrulewidth}}%
}

% draw a vertical rule of width 2pt on both sides of a cell
\newcommand\Thickvrule[1]{%
  \multicolumn{1}{!{\vrule width 2pt}c!{\vrule width 2pt}}{#1}%
}

% draw a vertical rule of width 2pt on the left side of a cell
\newcommand\Thickvrulel[1]{%
  \multicolumn{1}{!{\vrule width 2pt}c|}{#1}%
}

% draw a vertical rule of width 2pt on the right side of a cell
\newcommand\Thickvruler[1]{%
  \multicolumn{1}{|c!{\vrule width 2pt}}{#1}%
}

\newcommand{\command}{c}
\newcommand{\green}[1]{{ \color{green} #1 } }

\newcommand{\func}[2]{\mathsf{AD}(#1) \to (#2)}
\newcommand{\varEst}{\bf{VetxEst}}
\newcommand{\graphGen}{\bf{GraphGen}}

\newcommand{\ag}[2]{\mathsf{VetxEst}{(#1)}\to {(#2)}}
\newcommand{\ad}[2]{\mathsf{GraphGen}{(#1)}\to {(#2)}}
\newcommand{\rb}{\mathsf{RechBound}}
\newcommand{\pathsearch}{\mathsf{AdaptPathSearch}}


\theoremstyle{definition}
\newtheorem{thm}{Theorem}
\newtheorem{lem}[thm]{Lemma}
\newtheorem{cor}[thm]{Corollary}
\newtheorem{prop}[thm]{Proposition}
\newtheorem{defn}[thm]{Definition}

\title{Adaptivity analysis}

\author{}

\date{}

\begin{document}

\maketitle

\paragraph{Abstract}
An adaptive data analysis is based on multiple queries over a data set, in which some queries rely on the results of some other queries. The error of each query is usually controllable and bound independently, but the error can propagate through the chain of different queries and bring to high generalization error. To address this issue, data analysts are adopting different mechanisms in their algorithms, such as Gaussian mechanism, etc. To utilize these mechanisms in the best way one needs to understand the depth of chain of queries that one can generate in a data analysis. In this work, we define a programming language which can provide, through its type system, an upper bound on the adaptivity  depth (the length of the longest chain of queries) of a program implementing an adaptive data analysis. We show how this language can help to analyze the generalization error of two data analyses with different adaptivity structures.

\paragraph{Adaptivity}
Adaptivity is a measure of the nesting depth of a mechanism. To
represent this depth, we use extended natural numbers. Define $\natb =
\nat \cup \{\bot\}$, where $\bot$ is a special symbol and $\natbi =
\natb \cup \{\infty\}$. We use $\nnatA, \nnatB$ to range over $\nat$,
$\nnatbA, \nnatbB$ to range over $\natb$, and $\nnatbiA, \nnatbiB$ to
range over $\natbi$.

The functions $\max$ and $+$, and the order $\leq$ on natural numbers
extend to $\natbi$ in the natural way:
\[\begin{array}{lcl}
\max(\bot, \nnatbiA) & = & \nnatbiA \\
\max(\nnatbiA, \bot) & = & \nnatbiA \\
\max(\infty, \nnatbiA) & = & \infty \\
\max(\nnatbiA, \infty) & = & \infty \\
\\
%
\bot + \nnatbiA & = & \bot \\
\nnatbiA + \bot & = & \bot \\
\infty + \nnatbiA & = & \infty ~~~~ \mbox{if } \nnatbiA \neq \bot \\
\nnatbiA + \infty & = & \infty ~~~~ \mbox{if } \nnatbiA \neq \bot \\
\\
%
\bot \leq \nnatbiA \\
\nnatbiA \leq \infty
\end{array}
\]
One can think of $\bot$ as $-\infty$, with the special proviso that,
here, $-\infty + \infty$ is specifically defined to be $-\infty$.

\paragraph{Language}
Expressions are shown below. $\econst$ denotes constants (of some base
type $\tbase$, which may, for example, be reals or rational
numbers). $\eop$ represents a primitive operation (such as a
mechanism), which determines adaptivity. For simplicity, we assume
that $\eop$ can only have type $\tbase \to \tbool$. We make
environments explicit in closures. This is needed for the tracing
semantics later.
\[\begin{array}{llll}
\mbox{Expr.} & \expr & ::= & x ~|~ \expr_1 \eapp \expr_2 ~|~ \efix f(x).\expr
 ~|~ (\expr_1, \expr_2) ~|~ \eprojl(\expr) ~|~ \eprojr(\expr) ~| \\
%
& & & \etrue ~|~ \efalse ~|~ \eif(\expr_1, \expr_2, \expr_3) ~|~
\econst ~|~ \eop(\expr) \\
& & & ~|~ \elet  = \expr_1 \ein \expr_2 ~|~ \enil ~|~  \econs (
      \expr_1, \expr_2) \\
%
\mbox{Value} & \valr & ::= & \etrue ~|~ \efalse ~|~ \econst ~|~
(\efix f(x).\expr, \env) ~|~ (\valr_1, \valr_2) 
    ~|~ \enil ~|~ \econs (\valr_1, \valr_2) \\
%
\mbox{Environment} & \env & ::= & x_1 \mapsto \valr_1, \ldots, x_n \mapsto \valr_n
\end{array}\]


%%%%%%%%%%%%%%%%%%%%%%%%%%%%%%%%%%%%%%%%%%%%%%%%%%%%%

%%%%%%%%%%%%%%%%%%%%%%%%%%%%%%%%%%%%%%%%%%%%%%%%%%%%%


\section{Tracing operational semantics and adaptivity}

\paragraph{Traces}
A trace $\tr$ is a representation of the big-step derivation of an
expression's evaluation. Our big-step semantics output a trace. We use
traces to define the adaptivity of a run. Our notion of traces and the
tracing semantics is taken from~\cite[Section 4]{perera:dep}, but we
omit their ``holes'' for which we have no need. The construct
$\trapp{\tr_1}{\tr_2}{f}{x}{\tr_3}$ records a trace of function
application. $\tr_1$ is the trace of the head, $\tr_2$ the trace of
the argument and $\tr_3$ is the trace of the function body. $f$ and
$x$ are bound in $\tr_3$.
%
\[\begin{array}{llll}
\mbox{Trace} & \tr & ::= & x ~|~ \trapp{\tr_1}{\tr_2}{f}{x}{\tr_3} ~|~
\trfix f(x).e ~|~ (\tr_1, \tr_2) ~|~ \trprojl(\tr) ~|\\ 
%
& & & \trprojr(\tr) ~|~ \trtrue ~|~ \trfalse ~|~ \trift(\tr_b, \tr_t)
~|~ \triff(\tr_b, \tr_f) ~|~ \trconst ~|~ \trop(\tr) \\
%
& & & \trnil ~|~ \trcons (\tr_1, \tr_2) ~|~ \boxed{\color{red} \triapp{\tr_1}{\tr_2} ~|~ \eilam \expr}
\end{array}\]


\paragraph{Big-step tracing semantics}
The big-step, tracing semantics $\env, \expr \bigstep \valr, \tr$
computes a value $\valr$ and a trace $\tr$ from an expression $\expr$
and an enviroment $\env$ which maps the free variables of $\expr$ to
\emph{closed} values. The rules, taken from~\cite{perera:dep}, are
shown in Figure~\ref{fig:big-step}. Some salient points:
\begin{itemize}
\item[-] Erasing the traces from the semantics yields a standard
  big-step semantics.
\item[-] The trace of a primitive application $\eop(\expr)$
  records that $\eop$ was applied to the trace of
  $\expr$. This enables us to define adaptivity from a trace later.
\item[-] The trace of a variable $x$ is $x$. This way traces record
  where substitutions occur and, hence, variable dependencies. This is
  also needed for defining adaptivity.
\end{itemize}

\begin{figure}
\begin{mathpar}
  \inferrule{ }{\env, x \bigstep \env(x), x}
  %
  \and
  %
  \inferrule{ }{\env, \econst \bigstep \econst, \trconst}
  %
  \and
  %
  \inferrule{ }{\env, \etrue \bigstep \etrue, \trtrue}
  %
  \and
  %
  \inferrule{ }{\env, \efalse \bigstep \efalse, \trfalse}
  %
  \and
  %
  \inferrule{
  }{
    \env, \efix f(x). \expr \bigstep (\efix f(x).\expr, \env), \trfix f(x).\expr
  }
  %
  \and
  %
  \inferrule{
    \env, \expr_1 \bigstep \valr_1, \tr_1 \\
    \valr_1 = (\efix f(x).\expr, \env') \\\\
    \env, \expr_2 \bigstep \valr_2, \tr_2 \\
    \env'[f \mapsto \valr_1, x \mapsto \valr_2], \expr \bigstep \valr, \tr
  }{
    \env, \expr_1 \eapp \expr_2 \bigstep \valr, \trapp{\tr_1}{\tr_2}{f}{x}{\tr}
  }
  %
  \and
  %
  \inferrule{
    \env, \expr_1 \bigstep \valr_1, \tr_1 \\
    \env, \expr_2 \bigstep \valr_2, \tr_2
  }{
    \env, (\expr_1, \expr_2) \bigstep (\valr_1, \valr_2), (\tr_1, \tr_2)
  }
  %
  \and
  %
  \inferrule{
    \env, \expr \bigstep (\valr_1, \valr_2), \tr
  }{
    \env, \eprojl(\expr) \bigstep \valr_1, \trprojl(\tr)
  }
  %
  \and
  %
  \inferrule{
    \env, \expr \bigstep (\valr_1, \valr_2), \tr
  }{
    \env, \eprojr(\expr) \bigstep \valr_2, \trprojr(\tr)
  }
  %
  \and
  %
  \inferrule{
    \env, \expr \bigstep \etrue, \tr \\
    \env, \expr_1 \bigstep \valr, \tr_1
  }{
    \env, \eif(\expr, \expr_1, \expr_2) \bigstep \valr, \trift(\tr, \tr_1)
  }
  %
  \and
  %
  \inferrule{
    \env, \expr \bigstep \efalse, \tr \\
    \env, \expr_2 \bigstep \valr, \tr_2
  }{
    \env, \eif(\expr, \expr_1, \expr_2) \bigstep \valr, \triff(\tr, \tr_2)
  }
  %
  \and
  %
  \inferrule{
    \env, \expr \bigstep \valr, \tr \\
    \eop{}(\valr) = \valr'
  }{
    \env, \eop(\expr) \bigstep \valr', \trop(\tr)
  }
%
\and
%
  \inferrule{
}
{ \env, \enil \bigstep \enil, \trnil }
%
\and
%
\inferrule{
\env, \expr_1 \bigstep \valr_1, \tr_1 \\
\env, \expr_2 \bigstep \valr_2, \tr_2
}
{ \env, \econs (\expr_1, \expr_2)  \bigstep \econs (\valr_1, \valr_2),
  \trcons(\tr_1, \tr_2)
}
%
\and
%
\inferrule{
  \env, \expr_1 \bigstep \valr_1, \tr_1 \\
  \env[x \mapsto \valr_1] , \expr_2 \bigstep \valr, \tr_2
}
{\env, \elet x;q = \expr_1 \ein \expr_2 \bigstep \valr, \trlet (x,
  \tr_1, \tr_2) }
%
\\\\
%
\boxed{\color{red}
\inferrule
{
  \empty
}
{
  \env, \eilam \expr \bigstep (\eilam \expr, \env), \eilam \expr 
}
}
%
\and
%
\boxed{\color{red}
\inferrule{
  \env, \expr \bigstep (\eilam \expr', \env'), \tr_1 \\
  \env, \expr' \bigstep \valr, \tr_2
}
{\env, \expr [] \bigstep \valr, \triapp{\tr_1}{\tr_2} }

}
\end{mathpar}


  
  \caption{Big-step semantics with provenance}
  \label{fig:big-step}
\end{figure}


\paragraph{Adaptivity of a trace}
We define the \emph{adaptivity} of a trace $\tr$, $\adap(\tr)$, which
means the maximum number of nested $\eop$s in $\tr$, taking variable
and control dependencies into account. To define this, we need an
auxiliary notion called the \emph{depth of variable $x$} in trace
$\tr$, written $\ddep{x}(\tr)$, which is the maximum number of
$\eop{}$s in any path leading from the root of $\tr$ to an occurence
of $x$ (at a leaf), again taking variable and control dependencies
into account. Technically, $\adap: \mbox{Traces} \to \nat$ and
$\ddep{x}: \mbox{Traces} \to \natb$. If $x$ does not appear free in
$\tr$, $\ddep{x}(\tr)$ is $\bot$.

The functions $\adap$ and $\ddep{x}$ are defined by mutual induction
in Figure~\ref{fig:adap}. 

\begin{figure}
  \framebox{$\adap: \mbox{Traces} \to \nat$}
  \begin{mathpar}
    \begin{array}{lcl}
      \adap(x) & = & 0 \\
      %
      \adap(\trapp{\tr_1}{\tr_2}{f}{x}{\tr_3}) & = &
      \adap(\tr_1) + \max(\adap(\tr_3), \adap(\tr_2) + \ddep{x}(\tr_3))\\
      %
      \adap(\trfix f(x).\expr) & = & 0 \\
      %
      \adap((\tr_1, \tr_2)) & = & \max(\adap(\tr_1), \adap(\tr_2)) \\
      %
      \adap(\trprojl(\tr)) & = & \adap(\tr) \\
      %
      \adap(\trprojr(\tr)) & = & \adap(\tr) \\
      %
      \adap(\trtrue) & = & 0 \\
      %
      \adap(\trfalse) & = & 0 \\
      %
      \adap(\trift(\tr_b, \tr_t)) & = & \adap(\tr_b) + \adap(\tr_t) \\
      %
      \adap(\triff(\tr_b, \tr_f)) & = & \adap(\tr_b) + \adap(\tr_f) \\
      %
      \adap(\trconst) & = & 0 \\
      %
      \adap(\trop(\tr)) & = & 1 + \adap(\tr) \\
      %
     \adap(\trnil) & = & 0 \\
     %
     \adap(\trcons(\tr_1,\tr_2) ) & = &  \max(\adap(\tr_1),
                                        \adap(\tr_2)) \\
     %
    \adap( \trlet (x, \tr_1,\tr_2) ) & = & \max (\adap(\tr_2),
                                           \adap(\tr_1)+\ddep{x}(\tr_2)  )
                                           \\
    \boxed{\color{red} \adap(\triapp{\tr_1}{\tr_2})} & = & \adap(\tr_1) + \adap(\tr_2)\\
    %
    \boxed{\color{red} \adap(\eilam \expr)} & = & 0
      \end{array}
  \end{mathpar}
  %
  \framebox{$\ddep{x}: \mbox{Traces} \to \natb$}
  \begin{mathpar}
    \begin{array}{lcl}
      \ddep{x}(y) & = &
      \left\lbrace
      \begin{array}{ll}
        0 & \mbox{if } x = y \\
        \bot & \mbox{if } x \neq y
      \end{array}
      \right.\\
      %
      \ddep{x}(\trapp{\tr_1}{\tr_2}{f}{y}{\tr_3}) & = & \max(\ddep{x}(\tr_1), \\
      & & ~~~~~~~\adap(\tr_1) + \max(\ddep{x}(\tr_3), \ddep{x}(\tr_2) + \ddep{y}(\tr_3))) \\
      %
      \ddep{x}(\trfix f(y).\expr) & = & \bot \\
      %
      \ddep{x}((\tr_1, \tr_2)) & = & \max(\ddep{x}(\tr_1), \ddep{x}(\tr_2)) \\
      %
      \ddep{x}(\trprojl(\tr)) & = & \ddep{x}(\tr) \\
      %
      \ddep{x}(\trprojr(\tr)) & = & \ddep{x}(\tr) \\
      %
      \ddep{x}(\trtrue) & = & \bot \\
      %
      \ddep{x}(\trfalse) & = & \bot \\
      %
      \ddep{x}(\trift(\tr_b, \tr_t)) & = & \max(\ddep{x}(\tr_b), \adap(\tr_b) + \ddep{x}(\tr_t)) \\
      %
      \ddep{x}(\trift(\tr_b, \tr_f)) & = & \max(\ddep{x}(\tr_b), \adap(\tr_b) + \ddep{x}(\tr_f)) \\
      %
      \ddep{x}(\trconst) & = & \bot \\
      %
      \ddep{x}(\trop(\tr)) & = & 1 + \ddep{x}(\tr) \\
       %
      \ddep{x}(\trnil) & = & \bot \\
      %
      \ddep{x}(\trcons(\tr_1,\tr_2) ) & = & \max(\ddep{x}(\tr_1),
                                            \ddep{x}(\tr_2)) \\
      %
      \ddep{x}( \trlet(y, \tr_1, \tr_2) ) & = & \max( \ddep{x}(\tr_2),
                                                \ddep{x}(\tr_1)+\ddep{y}(\tr_2)  )\\
      \boxed{\color{red} \ddep{x}(\triapp{\tr_1}{\tr_2})}  & = & 
                                                    \max(\ddep{x}(\tr_1), \adap(\tr_1) + \ddep{x}(\tr_2))\\
    %
    \boxed{\color{red} \ddep{x}(\eilam \expr)} & = & \bot
    \end{array}
  \end{mathpar}
  \caption{Adaptivity of a trace and depth of variable $x$ in a trace}
  \label{fig:adap}
\end{figure}

\paragraph{Explanation of $\adap$}
We explain the interesting cases of the definition of $\adap$. The
case $\trapp{\tr_1}{\tr_2}{f}{x}{\tr_3}$ corresponds to a function
application with $\tr_1$, $\tr_2$, $\tr_3$ being the traces of the
head, the argument and the body, respectively, and $x$ being the
argument. The adaptivity is defined to be $\adap(\tr_1) +
\max(\adap(\tr_3), \adap(\tr_2) + \ddep{x}(\tr_3))$. The term
$\adap(\tr_1)$ occurs additively since the entire computation is
control-dependent on the function the head of the application
evaluates to. The rest of the term $\max(\adap(\tr_3), \adap(\tr_2) +
\ddep{x}(\tr_3))$ is simply the maximum nesting depth in the body,
taking the data dependency on the argument into account. To see this,
consider the following exhaustive cases:
\begin{itemize}
  \item[-] When $x$ appears free in the trace $\tr_3$,
    $\ddep{x}(\tr_3)$ is the maximum $\eop$-nesting depth of $x$ in
    the body. Hence, $\max(\adap(\tr_3), \adap(\tr_2) +
    \ddep{x}(\tr_3))$ represents the maximum number of nested $\eop$s
    in the evaluation of $e[e'/x]$ where $e'$ is the argument
    expression that generates the trace $\tr_2$ and $e$ is the body of
    the function.
  \item[-] When $x$ does not appear free in the trace $\tr_3$ of the
    body (i.e., the body's evaluation does not depend on $x$),
    $\ddep{x}(\tr_3) = \bot$, so $\max(\adap(\tr_3), \adap(\tr_2) +
    \ddep{x}(\tr_3)) = \max(\adap(\tr_3), \adap(\tr_2) + \bot) =
    \max(\adap(\tr_3), \bot) = \adap(\tr_3)$, which is the adaptivity
    of the body in the absence of dependency from $x$.
\end{itemize}

The case $\trift(\tr_b, \tr_t)$ corresponds to the evaluation of
$\eif(\expr_b, \expr_t, \_)$ where $\expr_b$ evaluates to $\etrue$
with trace $\tr_b$ and $\tr_t$ is the trace of $\expr_t$. In this
case, since the entire evaluation of $\expr_t$ is control dependent on
$\expr_b$, the adaptivity is simply $\adap(\tr_b) + \adap(\tr_t)$.

\paragraph{Explanation of $\ddep{x}$}
We explain interesting cases in the definition of $\ddep{x}$.  For the
trace $\trapp{\tr_1}{\tr_2}{f}{y}{\tr_3}$, $\ddep{x}$ is defined as
$\max(\ddep{x}(\tr_1), \adap(\tr_1) + \max(\ddep{x}(\tr_3),
\ddep{x}(\tr_2) + \ddep{y}(\tr_3)))$. Here, $\max(\ddep{x}(\tr_3),
\ddep{x}(\tr_2) + \ddep{y}(\tr_3))$ is the maximum depth of $x$ in the
body ($\tr_3$), taking the dependency on the argument into
account. Specifically, when the argument variable $y$ is not used in
the body, $\ddep{y}(\tr_3) = \bot$, and this term is
$\ddep{x}(\tr_3)$.  The term $\adap(\tr_1)$ is added since the body's
entire execution is control-flow dependent on the function that the
head of the application evaluates to.

For the trace $\trift(\tr_b, \tr_t)$, $\ddep{x}$ is defined as
$\max(\ddep{x}(\tr_b), \adap(\tr_b) + \ddep{x}(\tr_t))$. The term
$\ddep{x}(\tr_b)$ is simply the maximum depth of $x$ in $\tr_b$. We
take the $\max$ of this with $\adap(\tr_b) + \ddep{x}(\tr_t)$, the
maximum depth of $x$ in $\tr_t$, taking the control dependency on
$\tr_b$ into account. Note that when $x$ is not used in $\tr_t$, then
$\ddep{x}(\tr_t) = \bot$ and $\ddep{x}(\trift(\tr_b, \tr_t)) =
\ddep{x}(\tr_b)$.

\begin{lem}\label{lem:ddep-leq-adap}
For all $\tr$ and $x$, $\ddep{x}(\tr) \leq \adap(\tr)$ in $\natb$.
\end{lem}
%
\begin{proof}
By easy induction on $\tr$, following the definitions of $\ddep{x}$
and $\adap$.
\end{proof}

\paragraph{Remark}
At first glance it may seem that Lemma~\ref{lem:ddep-leq-adap} can be
used to simplify the definition of $\ddep{x}(\trift(\tr_b, \tr_t))$
from $\max(\ddep{x}(\tr_b), \adap(\tr_b) + \ddep{x}(\tr_t))$ to
$\adap(\tr_b) + \ddep{x}(\tr_t)$ since $\ddep{x}(\tr_b) \leq
\adap(\tr_b)$. However, this simplification is not correct, since
$\ddep{x}(\tr_t)$ may be $\bot$. In that case, $\max(\ddep{x}(\tr_b),
\adap(\tr_b) + \ddep{x}(\tr_t))$ equals $\ddep{x}(\tr_b)$ while
$\adap(\tr_b) + \ddep{x}(\tr_t)$ equals $\bot$.

More generally, since $\bot$ behaves like $-\infty$, we do not have
the implication $a \leq b \Rightarrow a \leq b + c$ as $c$ may be
$-\infty$ ($\bot$). As a result, $a \leq b$ does not imply $\max(a, b
+ c) = b + c$.


%%%%%%%%%%%%%%%%%%%%%%%%%%%%%%%%%%%%%%%%%%%%%%%%%%%%%

%%%%%%%%%%%%%%%%%%%%%%%%%%%%%%%%%%%%%%%%%%%%%%%%%%%%%


\section{Type system}

We build a type system that statically approximates both $\adap$ and
$\ddep{x}$ for every $x$ in the context.

A \emph{depth map}, denoted $\dmap$, is a map from variables to
$\natbi$.
%
%% We define the support of a depth map, $\supp(\dmap) \defeq \{x ~|~
%% \dmap(x) \neq \bot\}$. We are only interested in depth maps with
%% finite support. Such a depth map can be written $x_1: \nnatbiA_1,
%% \ldots, x_k: \nnatbiA_k$.
%
Our typing judgment takes the form $\Gamma; \dmap \tvdash{\nnatA}
\expr: \type$, where $\Gamma$ is a typing context, $\dmap$ is a depth
map, and $\nnatA \in \nat$. The idea is that $\dmap$ gives an upper
bound on the depth of each free variable of $\expr$. Obviously, we
only care about the values of $\dmap$ on the domain of $\Gamma$ (at
the remaining points, $\dmap$ can be $\bot$; such a $\dmap$ can always
be finitely represented as $x_1: \nnatbiA_1, \ldots, x_k: \nnatbiA_k$,
where $x_1,\ldots,x_k$ are bound by $\Gamma$). $\nnatA$ is an upper
bound on the adaptivity of $\expr$.

\paragraph{Types}
Types $\type$ are simple, except that the function type is annotated
with the adaptivity $\nnatA$ of the function body, a depth-map $\dmap$
which gives the depths of all free variables of the body (i.e.,
variables bound in outer scopes), and a depth $\nnatbiA \in \natbi$ of
the argument variable in the body. The function name, although free in
the body, always has depth $\infty$, so we don't write its depth
explicitly.\footnote{The function name is never substituted by a value
  with a nontrivial trace, so we can choose any depth for it. $\infty$
  is the most permissive depth, as will become clear from the typing
  rules shortly.}

\[
\begin{array}{llll}
  \mbox{Type} & \type & ::= & \tbase ~|~ \tbool ~|~ \type_1 \times
  \type_2 ~|~ \tarr{\type_1}{\type_2}{\nnatbiA}{\dmap}{\nnatA} ~|~
                              \tlist{\type} ~|~ \tbox{\type}
\end{array}
\]

%% Note that the type $\tarr{\type_1}{\type_2}{\nnatbiA}{\dmap}{\nnatA}$
%% is well-formed only in a context $\Gamma$ that binds all variables in
%% the domain of $\dmap$. 

\paragraph{Typing rules}
The typing rules are shown in Figure~\ref{fig:type-rules}. In these
rules, we write $\nnatbiA + \dmap$ or $\dmap + \nnatbiA$ for the depth
map $\lambda x.\eapp  (\nnatbiA + \dmap(x))$ defined on the same domain as
$\dmap$. We also lift $\max$ and $+$ pointwise to depth maps.
%% and define the constant map $\dmapb$ that maps everything to
%% $\bot$.
These are also formally defined in Figure~\ref{fig:type-rules}.

\begin{figure}
  \begin{mathpar}
    \inferrule{
      \Gamma(x) = \type \\ 0 \leq \dmap(x) \mbox{ or equiv.\ } \dmap(x) \neq \bot
    }{
      \Delta; \Gamma; \dmap \tvdash{\nnatA} x: \type
    }
    %
    \and
    %
    \inferrule{
      \Delta; \Gamma; \dmap_1 \tvdash{\nnatA_1} \expr_1: (\tarr{\type_1}{\type_2}{\nnatbiA}{\dmap}{\nnatA}) \\
      \Delta; \Gamma; \dmap_2 \tvdash{\nnatA_2} \expr_2: \type_1 \\\\
      \nnatA' = \nnatA_1 + \max(\nnatA, \nnatA_2 + \nnatbiA) \\
      \dmap' = \max(\dmap_1, \nnatA_1 + \max(\dmap, \dmap_2 + \nnatbiA))
    }{
      \Delta; \Gamma; \dmap' \tvdash{\nnatA'} \expr_1 \eapp \expr_2 : \type_2
    }
    %
    \and
    %
    \inferrule{
      \Delta; \Gamma, f: (\tarr{\type_1}{\type_2}{\nnatbiA}{\dmap}{\nnatA}), x: \type_1;
      \dmap[f: \infty, x: \nnatbiA]
      \tvdash{\nnatA}
      \expr: \type_2
    }{
      \Delta; \Gamma; \dmap' \tvdash{\nnatA'} \efix f(x).\expr: (\tarr{\type_1}{\type_2}{\nnatbiA}{\dmap}{\nnatA})
    }
    %
    \and
    %
    \inferrule{
      \Delta; \Gamma; \dmap_1 \tvdash{\nnatA_1} \expr_1: \type_1 \\
      \Delta; \Gamma; \dmap_2 \tvdash{\nnatA_2} \expr_2: \type_2 \\\\
      \dmap' = \max(\dmap_1,\dmap_2) \\
      \nnatA' = \max(\nnatA_1,\nnatA_2)
    }{
      \Delta; \Gamma; \dmap' \tvdash{\nnatA'} (\expr_1, \expr_2): \type_1 \times \type_2
    }
    %
    \and
    %
    \inferrule{
      \Delta; \Gamma; \dmap \tvdash{\nnatA} \expr: \type_1 \times \type_2
    }{
      \Delta; \Gamma; \dmap \tvdash{\nnatA} \eprojl(\expr): \type_1
    }
    %
    \and
    %
    \inferrule{
      \Delta; \Gamma; \dmap \tvdash{\nnatA} \expr: \type_1 \times \type_2
    }{
      \Delta; \Gamma; \dmap \tvdash{\nnatA} \eprojr(\expr): \type_2
    }
    %
    \and
    %
    \inferrule{
    }{
      \Delta; \Gamma; \dmap \tvdash{\nnatA} \etrue: \tbool
    }
    %
    \and
    %
    \inferrule{
    }{
       \Delta; \Gamma; \dmap \tvdash{\nnatA} \efalse: \tbool
    }
    %
    \and
    %
    \inferrule{
       \Delta;\Gamma; \dmap_1 \tvdash{\nnatA_1} \expr_1: \tbool \\
       \Delta;\Gamma; \dmap \tvdash{\nnatA} \expr_2: \type \\
       \Delta;\Gamma; \dmap \tvdash{\nnatA} \expr_3: \type \\\\
      \nnatA' = \nnatA_1 + \nnatA \\
      \dmap' = \max(\dmap_1, \nnatA_1 + \dmap)
    }{
       \Delta;\Gamma; \dmap' \tvdash{\nnatA'} \eif(\expr_1, \expr_2, \expr_3):  \type
    }
    %
    \and
    %
    \inferrule{
    }{
       \Delta;\Gamma; \dmap \tvdash{\nnatA} \econst: \tbase
    }
    %
    \and
    %
    \inferrule{
       \Delta;\Gamma; \dmap \tvdash{\nnatA} \expr: \tbase \\
      \nnatA' = 1 + \nnatA \\
      \dmap' = 1 + \dmap
    }{
       \Delta;\Gamma; \dmap' \tvdash{\nnatA'} \eop(\expr): \tbool
    }

 \and
    %
    \inferrule{
      \Delta; \Gamma; \dmap \tvdash{\nnatA} \expr:  \tbox{  (\tarr{ \tau_1
        }{ \tau_2 }{0}{\dmap''}{0})     }   \\
      \forall x \in \dmap''. \dmap''(x) = 0 \\
      \nnatA' = 1 + \nnatA \\
      \dmap' = 1 + \dmap
    }{
       \Delta; \Gamma; \dmap' \tvdash{\nnatA'} \eop(\expr): \treal
    }
  \end{mathpar}
  \caption{Typing rules}
  \label{fig:type-rules}
\end{figure}

\clearpage
\begin{figure}
\[
\begin{array}{llll}
\mbox{Index Term} & \idx, \nnatA & ::= &     i ~|~ n ~|~ \idx_1 + \idx_2 ~|~  \idx_1
                                 - \idx_2 ~|~ \smax{\idx_1}{\idx_2}\\
  \mbox{Sort} & S & ::= & \nat \\
\mbox{Expr.} & \expr & ::= & x ~|~ \expr_1 \eapp \expr_2 ~|~ \efix f(x).\expr
            ~|~ (\expr_1, \expr_2) ~|~ \eprojl(\expr) ~|~ \eprojr(\expr) ~| \\
%
& & &           \etrue ~|~ \efalse ~|~ \eif(\expr_1, \expr_2, \expr_3) 
            ~|~ \econst ~|~ \eop(\expr) \\
& & &           ~|~ \elet  = \expr_1 \ein \expr_2 ~|~ \enil ~|~  \econs (
                \expr_1, \expr_2)  \\
& & &  \eilam \expr  ~|~  \expr \eapp []  ~|~
                            \epack \expr ~|~ \eunpack \expr \eas x
                            \ein \expr \\
& & & ~|~ \bernoulli \eapp \valr ~|~ \uniform \eapp \valr_1 \eapp \valr_2\\      
%
  \mbox{Type} & \type & ::= & \tbool ~|~ \type_1 \times
  \type_2 ~|~ \tarr{\type_1}{\type_2}{\nnatbiA}{\dmap}{\nnatA} ~|~
                              \tlist{\type} ~|~ \tbox{\type}\\
& & &                 ~|~  \treal ~|~  \tint ~|~   \tint[I]  ~|~      \tforall{\dmap}{\nnatA}{i} \type  ~|~ \texists{i} \type
%
\end{array}
\]

\framebox{
  \begin{mathpar}
     {\color{blue}
     \inferrule{
           \Delta; \Gamma; \dmap \tvdash{\nnatA} \expr: \type \\
           \forall x \in \dom(\Gamma), \sub{\Gamma(x)}{\tbox{\Gamma(x)}}
           \\
           \delta \not\in \expr
         }{
           \Delta; \Gamma, \Gamma'; \dmap \tvdash{\nnatA} \expr: \tbox{\type}
         }
         }
    %
    \\
    %
    \boxed{\color{red}
     \inferrule{
      \Delta; \Gamma; \dmap \tvdash{\nnatA} \expr: \type \\
      \delta \not\in \expr
    }{
      \Delta; \Gamma, \Gamma'; \dmap \tvdash{\nnatA} \expr: \tbox{\type}
    }    
    }
    %
    \and
    %
    \inferrule{
      \dmap \wf{\type} \\
    }{
      \Delta; \Gamma; \dmap \tvdash{\nnatA} \enil: \tlist{\type}
    }
    % 
    \and
    %
   \inferrule{
   \Delta; \Gamma; \dmap_1 \tvdash{\nnatA_1} \expr_1 : \type \\
   \Delta; \Gamma; \dmap_2 \tvdash{\nnatA_2} \expr_2 : \tlist{\type}\\
   \dmap' = \max(\dmap_1, \dmap_2) \\
   \nnatA' = \max ( \nnatA_1, \nnatA_2 )
   }
   { 
   \Delta; \Gamma; \dmap' \tvdash{\nnatA'} \econs(\expr_1, \expr_2) :
     \tlist{\type}  }
   %
   \and
   %
   \inferrule{
     \Delta; \Gamma; \dmap_1 \tvdash{\nnatA_1} \expr_1 : \type_1 \\
     \Delta; \Gamma, x:\type_1 ; \dmap_2[x:q] \tvdash{\nnatA_2} \expr_2 :
     \type_2 \\
     \dmap' = \max( \dmap_2, \dmap_1 + q ) \\
     \nnatA' = \max ( \nnatA_2, \nnatA_1 + q )
   }
   {  \Delta; \Gamma; \dmap' \tvdash{\nnatA'}  \elet x = \expr_1 \ein \expr_2 : \type}
   %
   \and
   %
  {\color{red}
  \inferrule{
      \Delta, i; \Gamma ;\dmap \tvdash{\nnatA} \expr: \type
    }{
     \Delta;  \Gamma; \dmap \tvdash{\nnatA}    \eilam \expr    :  \tforall{\dmap}{\nnatA}{i} \type 
    }
    %
    \and
    %
  \inferrule{
      \Delta; \Gamma ;\dmap \tvdash{\nnatA} \expr: \tforall{\dmap_1}{\nnatA_1}{i} \type
      \and
       \Delta \tvdash{}  I ::  S
       \\
       \dmap' = \max(\dmap, \nnatA + \dmap_1)
       \\
       \nnatA' = \nnatA_1[I/i] + \nnatA
    }{
     \Delta;  \Gamma; \dmap' \tvdash{\nnatA'}    \expr \eapp []   :
     \type[I/i]
    }
	    }
    \end{mathpar}
}          

\framebox{
    \begin{mathpar}
  \inferrule
  {
       \Delta (i) = S
  }
  {
     \Delta \tvdash{}  i :: S
  }
  %
  \and
  %
  \inferrule
  {
    \empty
  }
  {
    \Delta \tvdash{}  n :: \nat
  }
  %
  \and
  %
  \inferrule
  {
    \Delta \tvdash{}  \idx_1 :: S
    \and
    \Delta \tvdash{}  \idx_2 :: S
    \and
    \Delta \tvdash{}  \diamond \in \{ +, -, \max \}
  }
  {
    \Delta \tvdash{}  \idx_1 \diamond \idx_2 :: S
  }
  \end{mathpar}
}  

  \framebox{
  \begin{mathpar}
    \mbox{\textbf{where: }} \\
%    \dmapb \defeq \lambda x.\eapp  \bot \\
    \nnatbiA + \dmap \defeq \dmap + \nnatbiA \defeq \lambda x.\eapp  (\nnatbiA + \dmap(x)) \\
    \dmap_1 + \dmap_2 \defeq \lambda x.\eapp  (\dmap_1(x) + \dmap_2(x)) \\
    \max(\dmap_1, \dmap_2) \defeq \lambda x.\eapp  \max(\dmap_1(x), \dmap_2(x))
  \end{mathpar}}

  \caption{New Typing rules and Index Term}
  \label{fig:new-type-rules}
\end{figure}


\begin{figure}
   \begin{mathpar}
     \inferrule{
    }{
      \dmap \models \sub{\tbase}{\tbox{\tbase} }
    }
    %
    \and
    %
     \inferrule{
    }{
      \dmap \models \sub{\tbool}{\tbox{\tbool} }
    }
    %
    \and
    %
    \inferrule{
       \dmap \models \sub{\type_1 }{ \type_1'  } \\
        \dmap \models \sub{\type_2 }{ \type_2'  }
    }{
      \dmap \models \sub{\type_1 \times \type_2 }{ \type_1' \times \type_2'  }
    }
    %
    \and
    %
     \inferrule{
       \forall x \in \dmap'. \dmap'(x) =0
    }{
      \dmap \models \sub{\tbox{(
          \tarr{\type_1}{\type_2}{\nnatbiA}{\dmap'}{\nnatA} )} 
      }{\tarr{(\tbox{\type_1}) }{(\tbox{\type_2})}{0}{\dmap'}{0} }
    }
    %
    \and
    %
     \inferrule{
    }{
      \dmap \models \sub{\tarr{\type_1}{\type_2}{0}{\dmap'}{0} }{  \tbox{(
          \tarr{\type_1}{\type_2}{\nnatbiA}{\dmap'}{\nnatA} )  }
    } }
    %
    \and
    %
     \inferrule{
    }
    {
      \dmap \models \sub{ \tint[I] }{ \tint }
    }
    %
    \and
    %
     \inferrule
    {
      \dmap \models \sub{ \type_1' }{ \type_1 }
      \and
      \dmap \models \sub{ \type_2 }{ \type_2' }
      \and
      \dmap \leq \dmap'
      \and
      \nnatA \leq \nnatA'
    }
    {
      \dmap \models \sub{ \tarr{\type_1}{\type_2}{ \nnatbiA }{\dmap}{\nnatA} }
      { \tarr{\type_1'}{\type_2'}{\nnatbiA'}{\dmap}{\nnatA'} }
    }
    \end{mathpar}
 \caption{Subtyping rules}
  \label{fig:sub-type-rules}
\end{figure}


The rules follow \emph{exactly} the definitions of $\adap()$ and
$\ddep{x}()$ to compute $\nnatA$ and $\dmap$ in the conclusion of the
typing rule for every construct. For instance, the typing rule for
$\expr_1 \eapp \expr_2$ does exactly what $\adap()$ and $\ddep{x}()$
do for the trace $\trapp{\tr_1}{\tr_2}{f}{x}{\tr_3}$.

\paragraph{Remark}
The astute reader might note that our type system looks very much like
a coeffect+effect system where the coeffect $\dmap$ estimates the
depth of each free variable and the effect $\nnatA$ estimates the
adaptivity. While this is true at a high-level, note there are two
essential differences between our type system and a coeffect+effect
system.
\begin{itemize}
\item[-] In some of our rules (those for $\expr_1 \eapp \expr_2$ and
  $\eif(\expr_1, \expr_2, \expr_3)$), the conclusion's coeffect
  depends on the effects of the premises. For instance, in the rule
  for $\expr_1 \eapp \expr_2$, the final coeffect $\dmap'$ depends on
  the effect $\nnatA_1$. This does not happen in standard
  coeffect+effect systems.
\item[-] Our type for functions internalizes the effect. In a standard
  coeffect + effect system, the effect is always on the typing
  judgment, and at the point of function construction ($\efix$), one
  anticipatively adds the effects of future function applications to
  the effect in the conclusion.
\end{itemize}
It may be possible to do away with both these differences by tracking
more information in the coeffects (e.g., what is the maximum
adaptivity of a function's argument?), but the details need to be
worked out and verified.

\begin{figure}
\begin{mathpar}
\inferrule
{
	\Gamma(x) = \type
}
{
	\Delta; \Gamma; \dmap \tvdash{\nnatA} x \inferring \type \restriction \dmap(x) \geq 0
}
%
\and
%
\inferrule
{
	\empty
}
{
	\Delta; \Gamma; \dmap \tvdash{\nnatA} c \inferring \tbase \restriction \bot
}
\end{mathpar}
 \caption{Algorithmic rules}
  \label{fig:sub-type-rules}
\end{figure}

%%%%%%%%%%%%%%%%%%%%%%%%%%%%%%%%%%%%%%%%%%%%%%%%%%%%

%%%%%%%%%%%%%%%%%%%%%%%%%%%%%%%%%%%%%%%%%%%%%%%%%%%%

\section{Denotational Semantics}

\begin{figure}
  \begin{mathpar}
    \begin{array}{lll}
      \lr{\tbool} & = & \mathbb{B}\\
      \lr{\type_1 \times \type_2} & = & \lrv{\type_1}\times\lrv{\type_2}\\
      %
      \lr{\tarr{\type_1}{\type_2}{\nnatbiA}{\dmap}{\nnatA}} & = &
      \lr{\type_1} \to \lrv{\type_2}\\
      %
      \lr{\tmonad \type} & = & \mathcal{M}\lrv{\type}\\
      \\
       \\
      \lr{\etrue}_{\env} &=& t\in \mathbb{B}\\
      \lr{\efalse}_{\env}&=& f \in \mathbb{B}\\
      \lr{(\expr_1,\expr_2)}_{\env}&=& (\lr{\expr_1}_{\env}, \lr{\expr_2}_{\env})\\
      \lr{x}_{\env} &=& \env(x)\\
      \lr{\lambda x:\type.\expr}_{\env} &=&  \hat{\lambda} v\in\lr{\type}. \lr{\expr}_{\env[x=v]}\\
      \lr{\expr_1\expr_2}_{\env} &=&  \lr{\expr_1}_{\env}\lr{\expr_2}_{\env}\\
      \lr{\return(\expr)}_{\env} &=& \hat{\delta}\lr{\expr}_{\env}\\
      \lr{\elet x=\expr_1 \ein \expr_2 }_{\env} &=& {\tt mlet}\ v=\lr{\expr_1}_{\env}
                                             \ {\tt in}\ \lr{\expr_2}_{\env[x=\valr]} \\
    \end{array}
  \end{mathpar}
  \caption{Denotational Semantics without Traces}
  \label{fig:denSem}
\end{figure}

where $\mathcal{M} A$ is the set of discrete distributions over $A$ defined as 
$\{d: A \to [0,1] | \sum_{a\in A} d(a)=1 \}$ and $\hat{\delta}$ is the
Dirac distribution.

\begin{figure}
  \begin{mathpar}
    \begin{array}{lll}
      \lrt{\type} & = & \lr{\type}\times\mathcal{\tr} \\
      % \lr{\type_1 \times \type_2} & = & \lrv{\type_1}\times\lrv{\type_2}\\
      % %
      % \lr{\tarr{\type_1}{\type_2}{\nnatbiA}{\dmap}{\nnatA}} & = &
      % \lr{\type_1} \to \lrv{\type_2}\\
      % %
      % \lr{M\type} & = & \mathcal{M}\lrv{\type}\\
      \\
       \\
      \lrt{\etrue}_{\env} &=& (t,\etrue) \\
      \lrt{\efalse}_{\env}&=& (f,\efalse)\\
      \lrt{(\expr_1,\expr_2)}_{\env}&=& \elet (\valr_1,\tr_1)=\lrt{\expr_1}_{\env} \ein 
                                               \elet (\valr_2, \tr_2)=\lrt{\expr_2}_{\env}) \ein 
                                               ((\valr_1,\valr_2),(\tr_1,\tr_2))\\
      \lrt{x}_{\env} &=& (\env(x),x)\\
      \lrt{\lambda x:\type.\expr}_{\env} &=&  (\hat{\lambda} \valr\in\lr{\type}. \lr{\expr}_{\env[x=\valr]}, \lambda x:\type.\expr)\\
      \lrt{\expr_1\expr_2}_{\env} &=&  
      % (\lr{\expr_1}_{\env}\lr{\expr_2}_{\env},\\ 
        % & & \elet (\valr_1, \tr_1) = \lrt{\expr_1}_{\env} \ein\\
        % & & \elet (\valr_2, \tr_2) = \lrt{\expr_2}_{\env} \ein\\
        % & & \elet \valr_1 = \lr{\lambda x:\type. \expr}_{\env} \ein \\
        % & & \elet (\valr, \tr) = \lrt{\expr}_{\env[x \to \valr_2]} \ein\\
        % & & (\tr_1, \tr_2, \mathrel{\triangleright} \lambda x:\type. \tr))\\
\\
\\
      \lr{\return(\expr)}_{\env} &=& \hat{\delta}\lr{\expr}_{\env}\\
      \lr{\elet x=\expr_1 \ein \expr_2 }_{\env} &=& ({\tt mlet}\ \valr=\lr{\expr_1}_{\env}
                                                   
                                                  \ {\tt in}\ \lr{\expr_2}_{\env[x=\valr]},\\
      & & \elet (\valr_1, \tr_1) = \lrt{\expr_1}_{\theta} \ein \\
      & & \elet (\valr_2, \tr_2) = \lrt{\expr_2}_{\theta[x \to \valr_1]} \ein \\
      & & (x, \tr_1, \tr_2)
      ) \\
    \end{array}
  \end{mathpar}
  \caption{Denotational Semantics with Traces}
  \label{fig:denSem_trace}
\end{figure}

\section{Logical relation and soundness}

Our type system is sound in the following sense.

\begin{thm}[Soundness]\label{thm:soundness}
If $\tvdash{\nnatA} \expr: \type$ and $\cdot, \expr \bigstep \valr,
\tr$ then $\adap(\tr) \leq \nnatA$.
\end{thm}

To prove this theorem, we build a logical relation on types. As usual,
the relation consists of a value relation and an expression
relation. We first show a non-step-indexed version of the relation in
Figure~\ref{fig:lr:non-step}. This relation is well-founded but
cannot, as usual, be used directly to prove the soundness of the
$\efix$ typing rule. We show this relation here purely for exposition
without the clutter of step-indices. We will step-index the relation
shortly.

\begin{figure}
  \begin{mathpar}
    \begin{array}{lll}
      \lrv{\tbool} & = & \{\etrue, \efalse\} \\
      %
      \lrv{\tbase} & = & \{\econst ~|~ \econst: \tbase \} \\
      %
      \lrv{\type_1 \times \type_2} & = & \{(\valr_1, \valr_2) ~|~ \valr_1 \in \lrv{\type_1} \conj \valr_2 \in \lrv{\type_2} \}\\
      %
      \lrv{\tarr{\type_1}{\type_2}{\nnatbiA}{\dmap}{\nnatA}} & = &
      \{(\efix f(x).\expr, \env) ~|~ \forall \valr \in \lrv{\type_1}.\\
      & & 
      ~~~~~~(\env[x \mapsto \valr, f \mapsto (\efix f(x).\expr, \env)], \expr) \in \lre{\dmap[x: \nnatbiA, f: \infty]}{\nnatA}{\type_2}\} \\
      %
      \\
      %
      \lre{\dmap}{\nnatA}{\type} & = & \{ (\env, \expr) ~|~ \forall \valr\eapp  \tr.\eapp  (\env, \expr \bigstep \valr, \tr) \\
      & & ~~~~~~~~~~~~~~~~~\Rightarrow (\adap(\tr) \leq \nnatA \conj \\
      & & ~~~~~~~~~~~~~~~~~~~~~~~\forall x \in \mbox{Vars}.\eapp  \ddep{x}(\tr) \leq \dmap(x) \conj \\
      & & ~~~~~~~~~~~~~~~~~~~~~~~\valr \in \lrv{\type})
      \}
    \end{array}
  \end{mathpar}
  \caption{Logical relation without step-indexing}
  \label{fig:lr:non-step}
\end{figure}

The value relation $\lrv{\type}$, as usual, defines which
\emph{closed} values are semantically in the type $\type$. The only
interesting case of this relation is the case for the arrow type. The
expression relation $\lre{\dmap}{\nnatA}{\type}$ is more
interesting. It is indexed by a depth map and an
adaptivity. Importantly, this relation contains sets of configurations
$(\env, \expr)$ where the evaluation of $(\env, \expr)$ produces a
value in the value relation at $\type$, the adaptivity of the trace is
no more than $\nnatA$ and for any variable $x$, $\ddep{x}$ of the
trace is no more than $\dmap(x)$.

The value relation can be lifted to contexts in the usual way. Say
that $\env \in \lrv{\Gamma}$ when $\dom(\env) \supseteq \dom(\Gamma)$
and for every $x \in \dom(\Gamma)$, $\env(x) \in \lrv{\Gamma(x)}$.

The fundamental theorem we would \emph{like} to prove is the
following. Of course, this cannot be proven until we step-index the
relation, but this should give an idea of the connection between the
type system and the logical relation.

\begin{prop}[Fundamental theorem]
\label{prop:fund}
  If $\Gamma; \dmap \tvdash{\nnatA} \expr: \type$ and $ \env \in
  \lrv{\Gamma}$, then $(\env, \expr) \in \lre{\dmap}{\nnatA}{\type}$.
\end{prop}


If this proposition were provable, then Theorem~\ref{thm:soundness}
would follow from it immediately.

\paragraph{Step-indexed logical relation}
To step-index the relation we need to have a notion of the number of
reduction steps. One way to do this is to change the operational
semantics to count the number of reduction steps. However, this is
unnecessary since the number of reduction steps in a derivation is
exactly the size of the derivation's trace. So, we just ``index'' the
relation on the size of the derivation's trace. The same technique has
been used previously~\cite{cicek15}.

\begin{defn}[Trace size]
  The size of a trace is defined as the number of
  $\trapp{\tr_1}{\tr_2}{f}{x}{\tr_3}$, $\trift(\tr_b, \tr_t)$,
  $\triff(\tr_b, \tr_f)$ and $\trop(\tr)$ constructors in
  it. Basically, we count all elimination constructors, but not
  introduction constructors.\footnote{Deepak's note: The purpose of
    counting this way is to make sure that the trace of the evaluation
    of any \emph{value} has size $0$. This may be required in the
    proof of the fundamental theorem, but I am not sure.}

\end{defn}

\begin{defn}[Trace size]
  We use $\size{\tr}$ to denote the size of the trace $\tr$.
\end{defn}
\begin{figure}
  \framebox{$\size{\tr}: \mbox{Traces} \to \nat$}
  \begin{mathpar}
    \begin{array}{lcl}
      \size{x} & = & 0 \\
      %
      \size{\trapp{\tr_1}{\tr_2}{f}{x}{\tr_3}} & = &
      \size{\tr_1} + \size{\tr_2} + \size{\tr_3} + 1 \\
      %
      \size{\trfix f(x).e} & = & 0 \\
      %
      \size{(\tr_1, \tr_2)} & = & \size{\tr_1} + \size{\tr_2} \\
      %
      \size{\trprojl(\tr)} & = & \size{\tr} +1 \\
      %
      \size{\trprojr(\tr)} & = & \size{\tr} + 1 \\
      %
      \size{\trtrue} & = & 0 \\
      %
      \size{\trfalse} & = & 0 \\
      %
      \size{\trift(\tr_b, \tr_t)} & = & \size{\tr_b} + \size{\tr_t} + 1 \\
      %
      \size{\triff(\tr_b, \tr_f)} & = & \size{\tr_b} + \size{\tr_f} + 1 \\
      %
      \size{\trconst} & = & 0 \\
      %
      \size{\trop(\tr)} & = & \size{\tr} + 1 \\
      %
      \size{\trnil} & = & 0 \\
      %
      \size{\trcons(\tr_1,\tr_2)} & = & \size{\tr_1} + \size{\tr_2} \\
      % 
      \size{ \trlet(x, \tr_1, \tr_2) } & = & \size{\tr_1} + \size{\tr_2}\\
      %
      \boxed{\color{red} \size{\triapp{\tr_1}{\tr_2}}} & = & \size{\tr_1} + \size{\tr_2}\\
      %
      \boxed{\color{red} \size{\eilam \expr}} & = & 0\\
      \end{array}
  \end{mathpar}
  \caption{Size of a trace}
  \label{fig:size}
\end{figure}



The step-indexed relation is shown in Figure~\ref{fig:lr:step}. Now,
the value relation $\lrv{\type}$ contains pairs of step-indices
$\stepiA$ and closed values. The expression relation contains pairs of
step-indices and configurations $(\env, \expr)$. The relation
basically mirrors the non-step-indexed relation of
Figure~\ref{fig:lr:non-step}, with step indexes added in the
completely standard way.


\begin{figure}
  \begin{mathpar}
    \begin{array}{lll}
      \lrv{\tbool} & = & \{(\stepiA, \etrue) ~|~ \stepiA \in \nat\} \cup
      \{ (\stepiA, \efalse) ~|~ \stepiA \in \nat\} \\
      %
      \lrv{\tbase} & = & \{(\stepiA, \econst) ~|~ \stepiA \in \nat \conj \econst: \tbase \} \\
      %
      \lrv{\type_1 \times \type_2} & = & \{(\stepiA, (\valr_1, \valr_2)) ~|~ (\stepiA, \valr_1) \in \lrv{\type_1} \conj (\stepiA, \valr_2) \in \lrv{\type_2} \}\\
      %
      \lrv{\tarr{\type_1}{\type_2}{\nnatbiA}{\dmap}{\nnatA}} & = &
      \{(\stepiA, (\efix f(x).\expr, \env)) ~|~ \forall \stepiB < \stepiA.\eapp  \forall (\stepiB, \valr) \in \lrv{\type_1}.\\
      & & 
      ~~~~~~(\stepiB, (\env[x \mapsto \valr, f \mapsto (\efix f(x).\expr, \env)], \expr)) \in \lre{\dmap[x: \nnatbiA, f: \infty]}{\nnatA}{\type_2}\} \\
      %
     \boxed{ \lrv{\tlist{\type}}  } & = & \{  (\stepiA, \enil) ~|~ \stepiA \in
                                \nat \} \cup \{  (\stepiA,
                                \econs(\valr_1,\valr_2) ) ~|~
                                (\stepiA, \valr_1) \in \lrv{\type}
                                \land (\stepiA, \valr_2) \in \lrv{\tlist{\type}} \}
      \\
      %
      \\ 
      %
      \lre{\dmap}{\nnatA}{\type} & = & \{ (\stepiA, (\env, \expr)) ~|~ \forall \valr\eapp  \tr\eapp  \stepiB.\eapp  (\env, \expr \bigstep \valr, \tr) \conj (\size{\tr} = \stepiB) \conj (\stepiB \leq \stepiA) \\
      & & ~~~~~~~~~~~~~~~~~~~~~~~~~\Rightarrow (\adap(\tr) \leq \nnatA \conj \\
      & & ~~~~~~~~~~~~~~~~~~~~~~~~~~~~~~~\forall x \in \mbox{Vars}.\eapp  \ddep{x}(\tr) \leq \dmap(x) \conj \\
      & & ~~~~~~~~~~~~~~~~~~~~~~~~~~~~~~~((\stepiA - \stepiB,  \valr) \in \lrv{\type})
      \}\\
      %	
      \boxed{\color{red} \lrv{\tint}} & = & \{(\stepiA, i) ~|~ \stepiA \in \nat \conj i : \tint \}\\
      %
      \boxed{\color{red} \lrv{\treal}} & = & \{(\stepiA, r) ~|~ \stepiA \in \nat \conj r : \treal \}\\
      %  
      \boxed{\color{red} \lrv{\tbox{\type}}} & = & \{(\stepiA, \valr) ~|~ \stepiA \in \nat \conj \valr \in \lrv{\type} \conj \eop \notin \valr \}\\
      %
      \boxed{\color{red} \lrv{\tforall{\dmap}{\nnatA}{i} \type} } & = & \{(\stepiA, (\eilam \expr, \env)) ~|~ \stepiA \in \nat \conj \forall I. ~  \tvdash{} I ::S, (\stepiA, \expr) \in \lre{\dmap}{\nnatA[I/i]}{\type[I/i]} \}\\
      %
      \boxed{\color{red} \lrv{ \tint[I] } } & = & \{(\stepiA, n) ~|~ \stepiA \in \nat \conj n = I \}\\
  \end{array}
  \end{mathpar}
  \caption{Logical relation with step-indexing}
  \label{fig:lr:step}
\end{figure}

We say that $(\stepiA, \env) \in \lrv{\Gamma}$ when $\dom(\env)
\supseteq \dom(\Gamma)$ and for every $x \in \dom(\Gamma)$, $(\stepiA,
\env(x)) \in \lrv{\Gamma(x)}$.

{\color{red} We say that $\ienv \in \lrv{\Delta} $ when $\dom(\ienv) \supseteq \dom(\Delta) $ and for every $x \in \dom(\Delta) $, $\ienv(x) :: \Delta(x)$ }

\clearpage
\begin{lem}\label{lem:downward}
1. If $(\stepiA, \valr) \in \lrv{\type}$ and $\stepiA' \leq \stepiA$,
then $(\stepiA', \valr) \in \lrv{\type}$.\\
2. 1. If $(\stepiA, \expr) \in \lre{\nnatA}{\dmap}{\type}$ and $\stepiA' \leq \stepiA$,
then $(\stepiA', \expr) \in \lre{\nnatA}{\dmap}{\type}$\\
3.If $(\stepiA, \env) \in \lrv{\Gamma}$ and $\stepiA' \leq \stepiA$, then $(\stepiA', \env) \in \lrv{\Gamma}$.
\end{lem}
%
\begin{proof}
(1,2) simultaneously proved by induction on $\type$, 3 followed by (1, 2).
\end{proof}

\begin{thm}[Fundamental theorem]
  If $\Gamma; \dmap \tvdash{\nnatA} \expr: \type$ and $(\stepiA, \env)
  \in \lrv{\Gamma}$, then $(\stepiA, (\env, \expr)) \in
  \lre{\dmap}{\nnatA}{\type}$.
\end{thm}
{\color{red}
\begin{thm}[Fundamental theorem]
  If $\Delta; \Gamma; \dmap \tvdash{\nnatA} \expr: \type$ and $ \ienv \in \lrv{\Delta}$ and $(\stepiA, \env)
  \in \lrv{\ienv \Gamma}$, then $(\stepiA, (\env, \ienv \expr)) \in
  \lre{\dmap}{\ienv \nnatA}{\ienv \type}$.
\end{thm}%
}
\begin{proof}
By induction on the given typing derivation. For the case of
$\efix$, we subinduct on the step index.\\
\begin{mainitem} 
\caseL{
$
    \boxed{
     \inferrule{
      \Delta; \Gamma; \dmap \tvdash{\nnatA} \expr: \type \\
      \delta \not\in \expr~(\diamond)
    }{
      \Delta; \Gamma, \Gamma'; \dmap \tvdash{\nnatA} \expr: \tbox{\type}
    }    
    }
$
}
Assume $(\stepiA, \env) \in \lrv{\Gamma, \Gamma'}$, $\ienv \in \lrv{\Delta}$, to show: $(\stepiA, (\env, \ienv \expr)) \in \lre{\dmap}{\ienv \nnatA}{\ienv \tbox{\type}}$.\\
%
By inversion, s.t.s. :$ \forall \valr, \tr, j, 
$ s.t. $ (\env, \ienv \expr) \bigstep (\valr, \tr) \land j = \size{\tr} \land j \leq \stepiA $.  \\
(1) $\adap(\tr) \leq \ienv \nnatA $, (2) $\forall x, \ddep{x}(\tr) \leq \dmap(x)$, (3) $(k-j, \valr) \in \lrv{ \ienv \tbox{\type} } $. \\
%
By induction hypothesis, we get: $(k, (\env, \ienv \expr)) \in \lre{\dmap}{\ienv \nnatA}{\ienv \type}$.\\
%
By inversion, we get $\forall \valr', \tr', j'$ s.t. $(\env, \ienv \expr) \bigstep (\valr', \tr') \land \size(\tr') = j' \land j' \leq \stepiA$,\\
(1)' $\adap(\tr') \leq \ienv \nnatA$, (2)' $\forall x, \ddep{x}(\tr') \leq \dmap(x)$, (3)' $(\stepiA - j', \valr') \in \lrv{\ienv \type}$.\\
%
Pick $\valr = \valr', \tr = \tr' , j = j'$, (1) (2) is proved by (1)', (2)'.\\
%
s.t.s. (3) $(\stepiA - j, \valr ) \in \lrv{\ienv \tbox{\type}}$.\\
%
By inversion on (3), s.t.s: (a) $(\stepiA - j, \valr) \in \lrv{\ienv \type}$, (b) $(\delta \notin \valr)$.\\
(a) is proved by (3)', (b) is proved by $(\diamond)$.
\\

\caseL{
$
\boxed{
  \inferrule{
      \Delta, i; \Gamma ;\dmap \tvdash{\nnatA} \expr: \type
    }{
     \Delta;  \Gamma; \dmap \tvdash{\nnatA}    \eilam \expr    :  \tforall{\dmap}{\nnatA}{i} \type 
    }
    }
$
}
Assume $(\stepiA, \env) \in \lrv{\Gamma } ~ (\diamond)$, $(\ienv \in \lrv{\Delta}~(\star)$, TS $(\stepiA, (\env, \eilam \ienv \expr)) \in \lre{\dmap}{\ienv \nnatA}{\tforall{\dmap}{\nnatA}{i} \type} $.\\
%
Since $(\env, \eilam \ienv \expr)$ is value, STS (a) $(\stepiA, (\env, \eilam \ienv \expr)) \in \lrv{\tforall{\dmap}{\nnatA}{i} \type}$.\\
%
Unfolding (a), pick any $I$ s.t. $\tvdash{} I :: S$, STS: (b) $(\stepiA, (\env, \ienv \expr)) \in \lre{\dmap}{(\ienv \nnatA)[I/i]}{\ienv \type [I/i]}$.\\
%
By $(\star)$, we know $\ienv[i \to I] \in \lrv{\Delta, i::S} ~ (\triangle)$.\\
%
(b) is proved by ih on premise and $(\diamond)$ and $(\triangle)$.\\


\caseL{
$
\boxed{
  \inferrule{
      \Delta; \Gamma ;\dmap \tvdash{\nnatA} \expr: \tforall{\dmap_1}{\nnatA_1}{i} \type ~ (\star)
      \and
       \Delta \tvdash{}  I ::  S ~ (\diamond)
       \\
       \dmap' = \max(\dmap, \nnatA + \dmap_1)
       \\
       \nnatA' = \nnatA_1[I/i] + \nnatA
    }{
     \Delta;  \Gamma; \dmap' \tvdash{\nnatA'}    \expr \eapp []   :
     \type[I/i]
    }
    }
$
}

Assume $(\stepiA, \env) \in \lrv{\Gamma}$, $\ienv \in \lrv{\Delta}$, TS: $(\stepiA, (\env, \ienv (\expr []))) \in \lre{\dmap}{\ienv(\nnatA[I/i] + \nnatA_1)}{\ienv(\type[I/i])}$.\\
%
By inversion, STS: $\forall \valr, \tr, j$ s.t. $(\env, \expr []) \bigstep (\valr, \tr) \land \size{\tr} = j \land j \leq \tr $.\\
(1) $ \adap(\tr) \leq \ienv (\nnatA[I/i] + \nnatA_1)$.\\
(2) $ \forall x, \ddep{x}(\tr) \leq \dmap'(x) $.\\
(3) $ (\stepiA - j, \valr) \in \lrv{\ienv(\type[I/i])}$\\
By induction hypothesis on $(\star)$, we get: $(k, (\env, \ienv \expr) \in \lre{\dmap}{\ienv \nnatA}{\ienv \tforall{\dmap_1}{\nnatA_1}{i} \type}$.\\
%
By inversion, assume $(a) (\env, \ienv \expr) \bigstep ((\eilam \expr_1, \env') \tr_1) \land \size(\tr_1) = j_1 \land j_1 \leq \stepiA$. We know:\\
(1)' $\adap(\tr_1) \leq \ienv \nnatA$.\\
(2)' $\forall x, \ddep{x}(\tr_1) \leq \dmap(x)$.\\
(3)' $(\stepiA - j_1, (\eilam \expr, \env')) \in \lrv{\ienv \tforall{\dmap_1}{\nnatA_1}{i} \type} $.\\
Unfold (3)', assume $\tvdash{} \ienv I ::S$ from $(\diamond)$, we get:\\
(4) $(k - j_1, (\env', \expr_1) \in \lre{\dmap_1}{\ienv \nnatA[I/i]}{\ienv (\type[I/i])}$\\
%
Unfold (4), assume $(b) (\env', \expr ) \bigstep (\valr_2, \tr_2) \land \size(\tr_2) = j_2 \land j_2 \leq \stepiA$,  we get:\\
(1)'' $\adap(\tr_2) \leq \ienv \nnatA_1[I/i]$.\\
(2)'' $\forall x, \ddep{x}(\tr_2) \leq \dmap(x)$.\\
(3)'' $(\stepiA - j_2, \valr_2) \in \lrv{\ienv \type[I/i]}$.\\
Pick $\valr = \valr_2$, $\tr = \triapp{\tr_1}{\tr_2}$, $j = j_1 + j_2$,
By operational semantics of IApp, we know:
$(\env, \ienv \expr []) \bigstep (\valr, \tr) \land \size(\tr) = j \land j \leq \stepiA$
(1) (2) (3) are proved by (1)' (2)' (3)' and (1)'' (2)'' (3)''.\\


\caseL{
$
  \boxed{
    \inferrule*[right = PRIMITIVE]{
      \Delta; \Gamma; \dmap \tvdash{\nnatA} \expr:  \tbox{  (\tarr{ \tau_1
        }{ \tau_2 }{0}{\dmap''}{0})     }   \\
      \forall x \in \dmap''. \dmap''(x) = 0 \\
      \nnatA' = 1 + \nnatA \\
      \dmap' = 1 + \dmap
    }{
      \Delta; \Gamma; \dmap' \tvdash{\nnatA'} \eop(\expr): \treal
    }
    }
$
}

Assume $(\stepiA, \env) \in \lrv{\Gamma}$, $ \ienv \in \lrv{\Delta}$ TS: $(\stepiA, \ienv (\eop(\expr), \env)) \in \lre{\dmap'}{\ienv \nnatA'}{\treal}$.\\
%
Unfold, $\forall \valr, \tr, j. (\env, \ienv \eop(\expr) \bigstep \valr, \tr) \land (\size{\tr} = j) \land (j \leq \stepiA)$.\\
%
STS: $(\adap(\tr) \leq \ienv \nnatA') \land (\forall x \in \mbox{Vars}. \ddep{x}(\tr) \leq \dmap'(x)) \land ((\stepiA - j, \valr) \in \lrv{\treal})$.\\
%
By induction hypothesis on $\star$, we get: $(\stepiA-1, (\expr, \ienv \env)) \in \lre{\dmap}{\ienv \nnatA}{\tbox{  (\tarr{ \tau_1}{ \tau_2 }{0}{\dmap''}{0})}} ~ (1)$.\\
%
Inversion on $(1)$, pick $\forall \valr', \tr', j'$. s.t. $ (\env, \expr \bigstep \valr', \tr') \land (\size{\tr'} = j') \land (j' \leq \stepiA-1)$.\\
%
{\color{red}We know: $(\adap(\tr') \leq \ienv \nnatA) ~ (a)
\land (\forall x \in \ddep{x}(\tr') \leq \dmap(x)) ~ (b)
\land ((\stepiA -1- j', \valr') \in \lrv{\tbox{  (\tarr{ \tau_1}{ \tau_2 }{0}{\dmap''}{0})     }}) ~ (c)$.}\\
%
Pick $\valr = \eop(\valr')$, then by E-PRIMITIVE, we have $\env, \ienv \eop(\expr) = (\env, \eop( \ienv \expr) )  \bigstep \valr, \eop(\tr')$.\\
%
Pick $\tr = \eop(\tr'), j = j' + 1$ s.t.  $j \leq k$, STS:\\
%
1. $\adap(\tr) = \adap(\eop(\tr')) = 1 + \adap(\tr') \leq 1 + \ienv \nnatA = \ienv \nnatA'$ proved by $(a)$\\
%
2. $\forall x \in \mbox{Vars}. \ddep{x}(\tr) = \ddep{x}(\eop(\tr')) = 1 + \ddep{x}(\tr') \leq 1 + \dmap(x) = \dmap'(x)$ proved by $(b)$\\
%
{\color{red}
3. $(\stepiA - j, \valr) = (\stepiA - j' - 1, \eop(\valr')) \in \lrv{\treal} $.\\
%
From (c), we know: $\tvdash{} \valr' : \tbox{  (\tarr{ \tau_1}{ \tau_2 }{0}{\dmap''}{0}) }$ . By typing rule PRIMITIVE, we have $\eop(\valr'): \treal$. So, we have $(\stepiA - j' - 1, \eop(\valr')) \in \lrv{\treal}$.\\
%
}

\caseL{
$
   \inferrule*[right = PAIR]{
      \Delta;\Gamma; \dmap_1 \tvdash{\nnatA_1} \expr_1: \type_1 ~ (1) 
   		\\
      \Delta;\Gamma; \dmap_2 \tvdash{\nnatA_2} \expr_2: \type_2 ~ (2)
      	\\\\
      \dmap' = \max(\dmap_1,\dmap_2) 
      	\\
      \nnatA' = \max(\nnatA_1,\nnatA_2)
    }{
      \Delta;\Gamma; \dmap' \tvdash{\nnatA'} (\expr_1, \expr_2): \type_1 \times \type_2
    }
$
}
Assume $(\stepiA, \env) \in \lrv{\Gamma}$, $ \ienv \in \lrv{\Delta}$, 
to show: $(\stepiA, (\env, \ienv (\expr_1, \expr_2))) \in \lre{\dmap'}{\ienv \nnatA'}{\ienv (\type_1 \times \type_2)}$.\\
%
By inversion, STS: $\forall \eapp  \valr, \tr, j.$ 
$(\env, \ienv (\expr_1, \expr_2) \bigstep \valr, \eapp  \tr) \land (\size{\tr} = j) \land (j \leq \stepiA)$\\
%
1. $ (\adap(\tr) < \ienv \nnatA')$\\
2. $ (\forall x \in \mbox{Vars}. \ddep{x}(\tr) \leq \dmap'(x)) $\\
3. $ ((\stepiA - j, \valr) \in \lrv{\ienv (\type_1 \times \type_2)})$.\\
%
By induction hypothesis on $(1)$, we have $(\stepiA, \ienv \expr_1) \in \lre{\dmap_1}{\ienv \nnatA_1}{\ienv  \type_1}$.\\
%
By inversion, pick any $\valr_1,  \tr_1, j_1$ s.t.,
$(\env, \ienv \expr_1 \bigstep \valr_1, \tr_1) ~ (3) 
\land \size{\tr_1} = j_1
\land j_1 < \stepiA $,\\
%
we have: \\
(4). $(\adap(\tr_1) \leq \ienv  \nnatA_1)$;\\
% 
(5). $\forall x, \ddep{x}(\tr_1) \leq \dmap_1(x)$;\\
% 
(6). $ (\stepiA - j_1, \valr_1) \in \lrv{\ienv  \type_1} $.\\
%
By induction hypothesis on $(2)$ and the assumption $(\stepiA - j_1,
\env) \in \lrv{\Gamma}$ from Lemma~\ref{lem:downward}, we have $(\stepiA - j_1, \ienv  \expr_2) \in \lre{\dmap_2}{\ienv  \nnatA_2}{\ienv  \type_2}$.\\
%
By inversion, pick $\valr_2, \tr_2, j_2 $ s.t., $ (\env, \ienv \expr_2 \bigstep \valr_2, \tr_2) ~ (7) \land (\size{\tr_2} = j_2) \land j_2 < \stepiA-j_1 $,\\
%
we have:\\
(8). $\adap(\tr_2) \leq \ienv  \nnatA_2$;\\
%
(9). $\forall x\in \mbox{Vars}. \ddep{x}(\tr_2) \leq \dmap_2(x)$;\\
%
(10). $(\stepiA- j_1- j_2, \valr_2) \in \lrv{\ienv  \type_2}$.\\
%
By E-Pair rule, we have:\\
\[
\inferrule*[right = E-Pair]
{\env, \expr_1 \bigstep \valr_1, \tr_1 ~ (3)  \and \env, \expr_2
\bigstep \valr_2, \tr_2 ~ (7) }
{\env, (\expr_1, \expr_2) \bigstep (\valr_1, \valr_2), (\tr_1, \tr_2)}
\]
Pick $\valr = (\valr_1, \valr_2), \tr = (\tr_1, \tr_2), j = |(\tr_1, \tr_2)| = j_1 + j_2$, s.t. $\env, \ienv (\expr_1, \expr_2) \bigstep \valr, \tr \land \size{\tr} = j \land j \leq \stepiA$.\\
%
1. 2. 3. are proved by: \\
1. $\adap(\tr) = \adap(\tr_1, \tr_2) < \ienv \nnatA' = \max(\ienv \nnatA_1, \ienv \nnatA_2)$, which is proved by $(4), (8)$.\\
%
2. $\forall x \in \mbox{Vars}. \ddep{x}(\tr_1, \tr_2) = \max(\ddep{x}(\tr_1), \ddep{x}(\tr_2)) \leq \dmap'(x) = \max(\dmap_1(x), \dmap_2(x))$, which is proved by $(5),(9)$.\\
%
3. $(\stepiA - (j_1 + j_2), (\valr_1, \valr_2)) \in \lrv{\ienv (\type_1  \times \type_2)}$, 
  which is proved by $(10)
$ and applying Lemma~\ref{lem:downward} on $(6)$.\\



\caseL{
$
    \inferrule*[right = FIX]
    {
      \Delta; \Gamma, f: (\tarr{\type_1}{\type_2}{\nnatbiA}{\dmap}{\nnatA}), x: \type_1;
      \dmap[f: \infty, x: \nnatbiA]
      \tvdash{\nnatA}
      \expr: \type_2
       ~ (1)
    }
    {
      \Delta; \Gamma; \dmap' \tvdash{\nnatA'} \efix f(x).\expr: (\tarr{\type_1}{\type_2}{\nnatbiA}{\dmap}{\nnatA})
    }
$
}
Assume: $(\stepiA, \env) \in \lrv{\tr}$, $ \ienv \in \lrv{\Delta}$.\\
%
TS: $(\stepiA, (\env, \ienv \efix f(x). \expr)) \in \lre{\dmap'}{\ienv \nnatA'}{\ienv (\tarr{\type_1}{\type_2}{\nnatbiA}{\dmap}{\nnatA}) }$.\\
%
By inversion, 
STS: $\forall \valr, \tr, j$ 
$(\env, \ienv \efix f(x). \expr \bigstep \valr, \tr) \land (\size{\tr} = j) \land (j \leq \stepiA)$\\
%
1. $(\adap(\tr) \leq \ienv \nnatA') $;\\
%
2. $ (\forall x \in \mbox{Vars}. \ddep{x}(\tr) \leq \dmap'(x)) $;\\
%
3. $ ((\stepiA - j), \valr) \in \lrv{\ienv  (\tarr{\type_1}{\type_2}{\nnatbiA}{\dmap}{\nnatA})}$.\\
%
By E-FIX, let $\valr = (\ienv \efix f(x). \expr, \env)$, we know:\\
%
(2). $(\env, \ienv \efix f(x). \expr) \bigstep ((\ienv \efix f(x). \expr, \env), \ienv \efix f(x). \expr) $;\\
%
(3). $|\ienv \efix f(x). \expr|= j \land j < \stepiA$.\\
%
Suppose $(2), (3)$, STS:\\
%
1. $ \adap(\ienv \efix f(x). \expr) = 0 \leq \ienv \nnatA'$;\\
%
2. $ \forall x \in \mbox{Vars}. \ddep{x}(\ienv \efix f(x). \expr) = \bot \leq \dmap'(x) $;\\
%
3. $((\stepiA -j), (\ienv \efix f(x). \expr, \env)) \in \lrv{\ienv (\tarr{\type_1}{\type_2}{\nnatbiA}{\dmap}{\nnatA})}$.\\
%
1. and 2. are proved by definition.
3., is proved by general theorem: \\
%
Set $\stepiA - j = \stepiA'$, $\forall m \leq \stepiA', (m, (\ienv \efix f(x). \expr, \env)) \in \lrv{\ienv (\tarr{\type_1}{\type_2}{\nnatbiA}{\dmap}{\nnatA})}$.\\
%
Induction on $m$:

{\bf Subcase 1:} $m = 0$,\\
	%
	$~~~~$ TS: $\forall j' < 0$, $(j', (\env[x \mapsto \valr_m, f \mapsto (\ienv \efix f(x).\expr, \ienv \env)], \expr)) \in \lre{\dmap[x: \nnatbiA, f : \infty]}{\nnatA}{\ienv (\type_2)}$,\\
	%
	$~~~~$ it is obviously true because $j' < 0 \notin \mathbb{N}$.\\
	%
{\bf Subcase 2:} $m = m' + 1 \leq \stepiA'$,\\
	%
	$~~~~$ TS: $ (m, (\ienv \efix f(x). \expr, \env)) \in \lrv{\ienv (\tarr{\type_1}{\type_2}{\nnatbiA}{\dmap}{\nnatA})}$.\\
	%
	$~~~~$ Pick $\forall j' < m' + 1$, $\forall (j', \valr_m) \in \lrv{\type_1}$,\\
	%
	$~~~~$ STS: (4). $(j', (\env[x \mapsto \valr_m, f \mapsto (\ienv \efix f(x).\expr, \env)], \ienv \expr)) \in \lre{\dmap[x: \nnatbiA, f: \infty]}{\nnatA}{\ienv \type_2} $.\\
	%
	$~~~~$ By sub ih, we have:
	(5). $(m', \ienv( \efix f(x). \expr, \env)) \in \lrv{\ienv (\tarr{\type_1}{\type_2}{\nnatbiA}{\dmap}{\nnatA})}$.\\
	%
	$~~~~$ Pick $\env' = \env[x \mapsto \valr_m, f \mapsto \ienv ( \efix f(x).\expr, \env)]$,\\
	%
	$~~~~$ we know: (6). $(j', \env') \in \lrv{\Gamma,
          f:\ienv (\tarr{\type_1}{\type_2}{\nnatbiA}{\dmap}{\nnatA}),  x:
          \ienv  \type_1}$ proved by:
    \begin{enumih}	
	\item $ (\stepiA, \env) \in \lrv{\Gamma}$, applying Lemma~\ref{lem:downward} on assumption, we get: $(j', \env) \in \lrv{\Gamma}$.
	
	\item $(j', \valr_m) \in \lrv{\ienv \type_1}$, from the assumption.
	
	\item $(j', (\efix f(x). \expr, \env)) \in \lrv{\ienv (\tarr{\type_1}{\type_2}{\nnatbiA}{\dmap}{\nnatA})}$ from $(5)$.\\
	\end{enumih}
From above, $(4)$ is proved by induction hypothesis on $(1)$ and $(6)$.\\
%


\caseL
{
$
    \inferrule*[right = APP]
    {
      \Delta; \Gamma; \dmap_1 \tvdash{\nnatA_1} \expr_1: (\tarr{\type_1}{\type_2}{\nnatbiA}{\dmap}{\nnatA})~(\star) \\
      \Delta; \Gamma; \dmap_2 \tvdash{\nnatA_2} \expr_2: \type_1 ~(\diamond)\\\\
      \nnatA' = \nnatA_1 + \max(\nnatA, \nnatA_2 + \nnatbiA) \\
      \dmap' = \max(\dmap_1, \nnatA_1 + \max(\dmap, \dmap_2 + \nnatbiA))
    }{
      \Delta; \Gamma; \dmap' \tvdash{\nnatA'} \expr_1 \eapp \expr_2 : \type_2
    }
$
}
%
Assume $(\stepiA, \env) \in \lrv{\Gamma} ~ (\triangle)$, $ \ienv \in \lrv{\Delta}$. TS: $(\stepiA, (\env, \ienv  (\expr_1 \eapp \expr_2))) \in \lre{\dmap'}{\ienv \nnatA'}{\ienv \type_2}$.\\
%
By inversion, pick any $\valr, \tr, j$
%
s.t. $((\env, \expr_1 \eapp \expr_2) \bigstep (\valr, \tr)) 
\land (\size{\tr} = j) 
\land (j \leq \stepiA) $, \\
%
STS:\\
1. $(\adap(\tr) \leq \ienv \nnatA') $;\\
%
2. $ (\forall x \in \mbox{Vars}. \ddep{x}(\tr) \leq \dmap'(x))$;\\
%
3. $(((\stepiA - j), \valr) \in \lrv{\ienv \type_2})$.\\
%
By ih on $\star$ and $\triangle$, we get: (1) $(\stepiA, (\env, \ienv  \expr_1)) \in \lre{\dmap_1}{\ienv  \nnatA_1}{ \ienv (\tarr{\type_1}{\type_2}{\nnatbiA}{\dmap}{\nnatA}) }$.\\
%
Inversion on $(1)$, we get:\\
%
Pick any $\valr_1, \tr_1, j_1$, s.t. $((\env, \ienv \expr_1) \bigstep (\valr_1, \tr_1)) ~(a) \land (\size{\tr_1} = j_1) \land (j_1 < \stepiA)$, \\
%
we know: \\
(2) $(\adap(\tr_1) \leq \ienv \nnatA_1)$;\\
%
(3) $ (\forall x \in \mbox{Vars}. \ddep{x}(\tr_1) \leq \dmap_1(x))$;\\
%
(4) $((\stepiA - j_1), \valr_1) \in \lrv{ \ienv (\tarr{\type_1}{\type_2}{\nnatbiA}{\dmap}{\nnatA}) }$.\\
%
$\valr_1$ is a function by definition.\\
%
Let $\valr_1 = (\efix f(x). \expr, \env') ~(b)$ s.t. $\tr_1= \efix f(x). \expr$,\\
%
By inversion on $(4)$, we know $\forall j' < (\stepiA - j_1) \land (j', \valr') \in \lrv{\ienv \type_1}$,\\
%
(5). $(j', (\env'[x \mapsto \valr', f \mapsto (\efix f(x).\expr, \env')], \expr)) \in \lre{\dmap[x : \nnatbiA, f: \infty]}{\ienv  \nnatA}{\ienv \type_2}$.\\ 
%
By ih on $\diamond$ and $\square$, we get:
%
(6). $ (\stepiA, (\env, \ienv \expr_2)) \in \lre{\dmap_2}{\ienv \nnatA_2}{\ienv \type_1} $.\\
%
Inversion on (6), we get:\\
%
Pick any $\valr_2, \tr_2, j_2$, s.t. 
%
$((\env, \ienv \expr_2) \bigstep (\valr_2, \tr_2))~(c) 
\land (\size{\tr_2} = j_2) 
\land (j_2 \leq \stepiA)$,\\
%
we know: \\
%
(7). $(\adap(\tr_2) \leq \ienv \nnatA_2)$;\\
%
(8). $(\forall x \in \mbox{Vars}. \ddep{x}(\tr_2) \leq \dmap_2(x))$; \\
%
(9). $((\stepiA - j_2), \valr_2) \in \lrv{\ienv \type_1}$\\
%
Apply Lemma~\ref{lem:downward} on $(9)$, we have $((\stepiA - j_2 - j_1-1), \valr_2) \in \lrv{\ienv  \type_1}$\\
%
Pick $j' = \stepiA - j_1 - j_2 - 1, \valr' = \valr_2$, we have: $(\stepiA - j_1 - j_2-1, (\env'[x \mapsto \valr_2, f \mapsto \valr_1], \expr)) \in \lre{\dmap[x : \nnatbiA, f: \infty]}{\ienv \nnatA}{\ienv \type_2} ~ (10)$ from $(5)$.\\
%
Inversion on $(10)$, let $\env'' = \env'[x \mapsto \valr_2, f \mapsto \valr_1]$, pick any $\valr'', \tr'', j''$, s.t. $((\env'', \expr) \bigstep (\valr'', \tr''))~(d) \land (\size{\tr''} = j'') \land (j'' \leq \stepiA - j_1 - j_2-1)$, we have:\\
%
(11). $\adap(\tr'') \leq \ienv  \nnatA$;\\
%
(12). $ (\forall x \in \ddep{x}(\tr'')\leq
\dmap[x : \nnatbiA, f: \infty](x))$;\\
%
(13). $(\stepiA - j_1 - j_2 - j''-1, \valr'') \in \lrv{\ienv \type_2}$.\\
%
Apply E-APP rule on $(a) (b) (c) (d)$ we have:
\[
  \inferrule{
    \env, \ienv  \expr_1 \bigstep \valr_1, \tr_1 ~ (a) \\
    \valr_1 = (\efix f(x).\expr, \env')~(b) \\\\
    \env, \ienv  \expr_2 \bigstep \valr_2, \tr_2 ~ (c) \\
    \env'[f \mapsto \valr_1, x \mapsto \valr_2], \expr \bigstep \valr'', \tr'' ~(d)
  }{
    \env,\ienv ( \expr_1 \eapp \expr_2) \bigstep \valr'', \trapp{\tr_1}{\tr_2}{f}{x}{\tr''}
  }
\]
Pick $\valr = \valr'', j = j_1 + j_2 + j'' + 1, \tr =
\trapp{\tr_1}{\tr_2}{f}{x}{\tr''}$ s.t. $ \env, \expr_1 \eapp \expr_2
\bigstep \valr, \tr  \land |\tr| = j \land j \leq k $.  \\
Suffice to show the follwing three:\\
%
1. $\adap(\tr) = \adap(\trapp{\tr_1}{\tr_2}{f}{x}{\tr''}) =
\adap(\tr_1) + \max(\adap(\tr''), \adap(\tr_2) + \ddep{x}(\tr'')) \leq
%
\ienv \nnatA_1 + \max(\ienv \nnatA, \ienv \nnatA_2 + \nnatbiA) = \ienv \nnatA'$ proved by $(2),(7),(11),(12)$.\\
%
2. $\forall y \in \mbox{Vars}. \ddep{y}(\tr) = \max(\ddep{y}(\tr_1), \adap(\tr_1) +
\max(\ddep{y}(\tr''), \ddep{y}(\tr_2) + \ddep{x}(\tr''))) \leq
\max(\dmap_1, \nnatA_1 + \max(\dmap, \dmap_2 + \nnatbiA)) = \dmap'(y)$
proved by $(2),(3),(8),(12)$. \\
%
3. $(\stepiA - j, \valr) = (\stepiA - j_1 - j_2 - j''-1, \valr'') \in
\lrv{\ienv \type_2}$ proved by $(13)$.\\
%


\caseL
{
$
    \inferrule*[right = IF]{
      \Delta; \Gamma; \dmap_1 \tvdash{\nnatA_1} \expr_1: \tbool ~ (\star) \\
      \Delta; \Gamma; \dmap \tvdash{\nnatA} \expr_2: \type ~(\diamond) \\
      \Delta; \Gamma; \dmap \tvdash{\nnatA} \expr_3: \type ~(\triangle) \\\\
      \nnatA' = \nnatA_1 + \nnatA \\
      \dmap' = \max(\dmap_1, \nnatA_1 + \dmap)
    }{
      \Delta; \Gamma; \dmap' \tvdash{\nnatA'} \eif(\expr_1, \expr_2, \expr_3):  \type
    }
$
}
Assume $(\stepiA, \env) \in \lrv{\Gamma}$, $ \ienv \in \lrv{\Delta}$. TS: $(\stepiA, (\env, \ienv \eif(\expr_1, \expr_2, \expr_3))) \in \lre{\dmap'}{\ienv \nnatA'}{\ienv \type}$.\\
%
Inversion on it, $\forall \valr, \tr, j$ $(\env, \ienv  \eif(\expr_1, \expr_2, \expr_3) \bigstep \valr, \tr) \land (\size{\tr} = j) \land (j \leq \stepiA)$\\
%
STS: $(\adap(\tr) \leq \ienv  \nnatA') \land (\forall x \in \mbox{Vars}. \ddep{x}(\tr) \leq \dmap'(x)) \land ((\stepiA - j, \valr) \in \lrv{\ienv \type})$.\\
%
By induction hypothesis on $\star$ and $(\stepiA-1, \env) \in
\lrv{\Gamma}$ by  Lemma~\ref{lem:downward}, we get: $(\stepiA-1, (\env, \expr_1)) \in \lre{\dmap_1}{\nnatA_1}{\tbool} ~(1)$.\\
%
Inversion on $(1)$, pick $ \valr_1, \tr_1, j_1. (\env, \ienv \expr_1 \bigstep \valr_1, \tr_1) ~ (a) \land (\size{\tr_1} = j_1) \land (j_1 \leq \stepiA-1)$,\\
%
we know: $(\adap(\tr_1) \leq \ienv \nnatA_1)~(4) \land (\forall x. \in
\ddep{x}(\tr_1) \leq \dmap_1(x)) ~(5) \land ((\stepiA -1- j_1, \valr_1) \in \lrv{\tbool})$\\
%
By induction hypothesis on $\diamond$
%
and $(\stepiA-1-j_1, \env) \in \lrv{\Gamma}$
%
by Lemma~\ref{lem:downward} on the assumption, 
%
we get: $(\stepiA-1-j_1, (\env, \ienv \expr_2)) \in \lre{\dmap}{\ienv \nnatA}{\ienv \type} ~(2)$\\
%
By induction hypothesis on $\triangle$ 
%
and $(\stepiA-1-j_1, \env) \in \lrv{\Gamma}$,
%
by Lemma~\ref{lem:downward} on the assumption, 
%
we get: $(\stepiA-1-j_1, (\env, \ienv \expr_3)) \in \lre{\dmap}{\ienv \nnatA}{\ienv \type} ~(3)$\\
%
Induction on $\valr_1$, we have two following subcases:

{\bf Subcase1:} $\valr_1 = \etrue$, \\
%
inversion on $(2)$, pick any $\valr_2, \tr_2, j_2$, s.t., $ (\env, \ienv \expr_2 \bigstep \valr_2, \tr_2) ~ (b) \land (\size{\tr_2} = j_2) \land (j_2 \leq \stepiA-1-j_1)$,\\
%
we know: $(\adap(\tr_2) \leq \ienv \nnatA)~(6) \land (\forall x \in \mbox{Vars}. \ddep{x}(\tr_2) \leq \dmap(x))~(7) \land ((\stepiA-1 - j_1-j_2, \valr_2) \in \lrv{\ienv \type})~(8)$.\\
%
Apply E-IFT on $(a) (b)$, we get:
\[
  \inferrule{
    \env,\ienv  \expr_1 \bigstep \etrue, \tr_1 ~ (a) \\
    \env, \ienv \expr_2 \bigstep \valr_2, \tr_2 ~ (b)
  }{
    \env, \ienv \eif(\expr_1, \expr_2, \expr_3) \bigstep \valr_2, \trift(\tr_1, \tr_2)
  }
\]
Pick $\valr = \valr_2, \tr = \trift(\tr_1, \tr_2), j =j_1+ j_2+1$ s.t.,\\
%
$\env, \ienv \eif(\expr_1, \expr_2, \expr_3) \bigstep \valr, \tr \land j =
\size{\tr} \land j = j_1 + j_2+1 \leq \stepiA$.\\
The following 3 goals are proved:\\
%
1. $\adap(\tr) = \adap(\trift(\tr_1, \tr_2)) = \adap(\tr_1) +
\adap(\tr_2)) \leq \ienv \nnatA_1 + \ienv \nnatA = \ienv \nnatA'$ is proved by $(4),(6)$.\\
%
2. $\forall x \in \mbox{Vars}$ s.t., $ \ddep{x}(\tr) = \max(\ddep{x}(\tr_1), \adap(\tr_1) +
\ddep{x}(\tr_2)) \leq \max(\dmap_1(x), \ienv \nnatA_1 + \dmap(x)) =
\dmap'(x)  $ is proved by $(5), (7)$.\\
%
3. $(\stepiA - j, \valr) = (\stepiA-1 -j_1- j_2, \valr_2) \in
\lrv{\ienv \type}$ is proved by $(8)$.

{\bf Subcase2:} $\valr_1 = \efalse$\\
%
inversion on $(3)$, pick any $\valr_3, \tr_3, j_3$ s.t., $(\env, \ienv \expr_3 \bigstep \valr_3, \tr_3) ~ (b) \land (\size{\tr_3} = j_3) \land (j_3 \leq \stepiA-1-j_1)$,\\
%
we know: $(\adap(\tr_3) \leq \ienv \nnatA) ~(9) \land (\forall x \in
\mbox{Vars}. \ddep{x}(\tr_3) \leq \dmap(x))~(10) \land ((\stepiA -1-j_1- j_3, \valr_3) \in \lrv{\ienv \type})~(11)$.\\
%
Apply E-IFT on $(a) (b)$, we get:
\[
  \inferrule{
    \env, \ienv \expr_1 \bigstep \etrue, \tr_1 ~ (a) \\
    \env, \ienv \expr_3 \bigstep \valr_3, \tr_3 ~ (b)
  }{
    \env, \ienv \eif(\expr_1, \expr_2, \expr_3) \bigstep \valr_3, \trift(\tr_1, \tr_3)
  }
\]
Pick $\valr = \valr_3, \tr = \trift(\tr_1, \tr_3), j =j_1+ j_3+1$,  s.t. $\\
%
\env, \ienv \eif(\expr_1, \expr_2, \expr_3) \bigstep \valr, \tr \land j =
\size{\tr} \land j = j_1 + j_3+1 \leq \stepiA$.\\
%
The following 3 goals are proved:\\
%
1. $\adap(\tr) = \adap(\trift(\tr_1, \tr_3)) = \adap(\tr_1) +
\adap(\tr_3)) \leq \ienv \nnatA_1 + \ienv \nnatA = \ienv \nnatA'$  proved by $(4),(9)$.\\
%
2. $\forall x \in \mbox{Vars}. \ddep{x}(\tr) = \max(\ddep{x}(\tr_1), \adap(\tr_1) +
\ddep{x}(\tr_3)) \leq \max(\dmap_1(x), \ienv \nnatA_1 + \dmap(x)) =
\dmap'(x) $ proved by $(5),(10)$.\\
%
3. $(\stepiA - j, \valr) = (\stepiA -1 - j_1-j_3, \valr_3) \in
\lrv{\type} $ proved by $(11)$.\\



\caseL
{
$
    \inferrule*[right = FST]{
      \Delta; \Gamma; \dmap \tvdash{\nnatA} \expr: \type_1 \times \type_2
    }{
      \Delta; \Gamma; \dmap \tvdash{\nnatA} \eprojl(\expr): \type_1
    }
$
}
Assume $(\stepiA, \env) \in \lrv{\Gamma}$, $ \ienv \in \lrv{\Delta}$.
%
TS: $( \stepiA, (\env, \ienv \eprojl(\expr)) ) \in \lre{\dmap}{\ienv \nnatA}{\ienv \type_1} $.\\
%
Unfold, pick any $ \valr, \tr, j. (\env, \ienv \eprojl(\expr)   \bigstep \valr, \tr) \land (\size{\tr} = j) \land (j \leq \stepiA) $,\\
%
STS: 
1. $ (\adap(\tr) \leq \ienv \nnatA) $;\\
%
2. $ (\forall x \in \mbox{Vars}. \ddep{x}(\tr) \leq \dmap(x)) $;\\
%
3. $ (\stepiA - j, \valr) \in \lrv{\ienv \type_1} $.\\
%
By induction hypothesis, we get:
%
(1). $(\stepiA-1, (\env, \ienv \expr)) \in \lre{\dmap}{\ienv \nnatA}{\ienv (\type_1 \times \type_2)}$.\\
%
Inversion on (1), $ \forall \valr', \tr', j'. (\env, \ienv \expr \bigstep \valr', \tr') \land (\size{\tr'} = j') \land (j' \leq \stepiA-1) $,\\
%
We know: $(\adap(\tr') \leq \ienv \nnatA) ~ (a)
% 
\land (\forall x \in \mbox{Vars}. \ddep{x}(\tr') \leq \dmap(x)) ~ (b)
%
\land ((\stepiA -1 - j', \valr') \in \lrv{\ienv (\type_1 \times \type_2)}) ~ (c)$.\\
%
Pick any $\valr_1, \valr_2$, $\valr' = (\valr_1, \valr_2)$,
%
inversion on $(c)$, we have:\\
%
$(\stepiA - j', \valr_1) \in \lrv{\ienv \type_1}~(d) \land (\stepiA - j', \valr_2) \in \lrv{\ienv \type_2}$.\\
%
By E-FST, pick $\valr = \valr_1, j = j'+1, \tr = \trprojl(\tr')$, the 3 goals are proved:\\
%
1. $\adap(\tr) = \adap(\trprojl(\tr')) = \adap(\tr') \leq \ienv \nnatA$ by $(a)$.\\
%
2. $\forall x .\ddep{x}(\tr) \implies \forall x . \ddep{x}(\tr') \leq \dmap(x)$ by $(b)$.\\
%
3. $(\stepiA - j, \valr) = (\stepiA -1- j', \valr_2) \in
\lrv{\ienv \type_1}$ by $(d)$.
\\




\caseL
{
$
    \inferrule*[right = SND]{
      \Delta; \Gamma; \dmap \tvdash{\nnatA} \expr: \type_1 \times \type_2
    }{
      \Delta; \Gamma; \dmap \tvdash{\nnatA} \eprojr(\expr): \type_2
    }
$
}
Assume $(\stepiA, \env) \in \lrv{\Gamma}$, $ \ienv \in \lrv{\Delta}$. TS: $(\stepiA, (\env, \ienv \eprojr(\expr))) \in \lre{\dmap}{\ienv \nnatA}{\ienv \type_2} $.\\
%
Unfold, pick any $ \valr, \tr, j. (\env, \ienv \eprojr(\expr) \bigstep \valr, \tr) \land (\size{\tr} = j) \land (j \leq \stepiA) $,\\
%
STS: 
1. $ (\adap(\tr) \leq \ienv \nnatA)$;\\
%
2. $(\forall x \in \mbox{Vars}. \ddep{x}(\tr) \leq \dmap(x))$;\\
%
3. $(\stepiA - j, \valr) \in \lrv{\ienv \type_1} $.\\
%
By induction hypothesis, we get: $(\stepiA-1, (\env, \ienv \expr)) \in \lre{\dmap}{\ienv \nnatA}{\ienv (\type_1 \times \type_2)} ~(1)$.\\
%
Inversion on $(1)$, $\forall \valr', \tr', j'. (\env, \ienv \expr \bigstep \valr', \tr') \land (\size{\tr'} = j') \land (j' \leq \stepiA-1) $,\\
%
We know: $(\adap(\tr') \leq \ienv \nnatA) ~ (a) 
%
\land (\forall x \in \ddep{x}(\tr') \leq \dmap(x)) ~ (b)
%
\land ((\stepiA -1- j', \valr') \in \lrv{\ienv (\type_1 \times \type_2)}) ~ (c)$.\\
%
Pick any $\valr_1, \valr_2$, $\valr' = (\valr_1, \valr_2)$,
% Let $\expr = (\expr_1, \expr_2)$, by E-PAIR, we have:\\
% %
% $\tr' = (\tr_1, \tr_2), \valr' = (\valr_1, \valr_2), j' = \size{\tr'} = \size{\tr_1} + \size{\tr_2} = j_1 + j_2$.\\
%
inversion on $(c)$, we have:\\
%
$(\stepiA - j', \valr_1) \in \lrv{\ienv \type_1}~(d) \land (\stepiA - j', \valr_2) \in \lrv{\ienv \type_2}~(e)$.\\
%
By E-SND, pick $\valr = \valr_2, j = j'+1, \tr = \trprojr(\tr')$, following 3 goals are proved:\\
%
1. $\adap(\tr) = \adap(\trprojr(\tr')) = \adap(\tr') \leq \ienv \nnatA$ by $(a)$\\
%
2. $\forall x \in \mbox{Vars} .\ddep{x}(\tr) \implies \forall x . \ddep{x}(\tr') \leq
\dmap(x)$ proved by $(b)$.\\
%
3. $(\stepiA - j, \valr) = (\stepiA -1 - j', \valr_2) \in \lrv{\ienv \type_2}$
proved by $(e)$.\\


\caseL
{
$
    \inferrule*[right = TRUE]{
    }{
      \Delta; \Gamma; \dmap \tvdash{\nnatA} \etrue: \tbool
    }
$
}
Assume $ (\stepiA, \env) \in \lrv{\tbool} $, $ \ienv \in \lrv{\Delta}$. TS: $(\stepiA, (\env, \etrue)) \in \lre{\dmap}{\ienv \nnatA}{\tbool}$.\\
%
By inversion, STS: $\forall \valr, \tr, j. (\env, \etrue \bigstep \valr, \tr) \land (\size{\tr} = j) \land (j \leq \stepiA) $\\
%
1. $ (\adap(\tr) \leq \ienv \nnatA)$
%
2. $ (\forall x \in \mbox{Vars}. \ddep{x}(\tr) \leq \dmap(x))$
%
3. $ (\stepiA - j, \valr) \in \lrv{\tbool}$\\
%
By E-TRUE, let $\valr = \etrue$, $\tr = \etrue$ and $j = \size{\etrue} = 0$.\\
%
Following 3 goals are proved:\\
%
1. $\adap(\tr) = 0 \leq \ienv \nnatA$.\\
2. $\forall x \in \mbox{Vars}. \ddep{x}(\etrue) = 0 \leq \dmap(x)$\\
3. $(\stepiA - j, \valr) = (\stepiA, \etrue) \in \lrv{\tbool}$.\\



\caseL
{
$
    \inferrule*[right = FALSE]{
    }{
      \Delta; \Gamma; \dmap \tvdash{\nnatA} \efalse: \tbool
    }
$
}
Assume $ (\stepiA, \env) \in \lrv{\tbool} $, $ \ienv \in \lrv{\Delta}$. TS: $(\stepiA, (\env, \efalse)) \in \lre{\dmap}{\ienv \nnatA}{\tbool}$.\\
%
By inversion, STS: $\forall \valr, \tr, j. (\env, \efalse \bigstep \valr, \tr) \land (\size{\tr} = j) \land (j \leq \stepiA) $\\
%
1. $ (\adap(\tr) \leq \nnatA) $
%
2. $ (\forall x \in  \mbox{Vars}. \ddep{x}(\tr) \leq \dmap(x))$
%
3. $(\stepiA - j, \valr) \in \lrv{\tbool}$\\
%
By E-TRUE, let $\valr = \efalse$, $\tr = \efalse$ and $j = \size{\efalse} = 0$.\\
%
Following 3 goals are proved:\\
%
1. $\adap(\tr) = 0 \leq \ienv \nnatA$.\\
2. $\forall x \in \mbox{Vars}. \ddep{x}(\efalse) = 0 \leq \dmap(x)$\\
3. $(\stepiA - j, \valr) = (\stepiA, \efalse) \in \lrv{\tbool}$.\\


\caseL{
$
    \inferrule*[right = PRIMITIVE]{
      \Delta; \Gamma; \dmap \tvdash{\nnatA} \expr: \tbase ~ (\star) \\
      \nnatA' = 1 + \nnatA \\
      \dmap' = 1 + \dmap
    }{
      \Delta; \Gamma; \dmap' \tvdash{\nnatA'} \eop(\expr): \tbool
    }
$
}
Assume $(\stepiA, \env) \in \lrv{\Gamma}$, $ \ienv \in \lrv{\Delta}$. TS: $(\stepiA, (\env, \ienv \eop(\expr))) \in \lre{\dmap'}{\ienv \nnatA'}{\tbool}$.\\
%
By inversion, STS: $\forall \valr, \tr, j. (\env, \ienv \eop(\expr) \bigstep \valr, \tr) \land (\size{\tr} = j) \land (j \leq \stepiA)$\\
%
1. $(\adap(\tr) \leq \ienv \nnatA')$
%
2. $(\forall x \in \mbox{Vars}. \ddep{x}(\tr) \leq \dmap'(x))$
%
3. $((\stepiA - j, \valr) \in \lrv{\tbool})$.\\
%
By induction hypothesis on $\star$, we get: $(\stepiA-1, (\ienv \expr, \env)) \in \lre{\dmap}{\ienv \nnatA}{\tbase} ~ (1)$.\\
%
Inversion on $(1)$, pick any $ \valr', \tr', j'$. s.t. $ (\env, \ienv \expr \bigstep \valr', \tr') \land (\size{\tr'} = j') \land (j' \leq \stepiA-1)$.\\
%
We know: $(\adap(\tr') \leq \ienv \nnatA) ~ (a)
\land (\forall x \in \ddep{x}(\tr') \leq \dmap(x)) ~ (b)
\land ((\stepiA -1- j', \valr') \in \lrv{\tbase}) ~ (c)$.\\
%
Pick $\valr = \eop(\valr')$, then by E-PRIMITIVE, we have $\env, \ienv \eop(\expr) \bigstep \valr, \eop(\tr')$.\\
%
Pick $\tr = \eop(\tr'), j = j' + 1$ s.t.  $j \leq k$, following 3 goals are proved:\\
%
1. $\adap(\tr) = \adap(\eop(\tr')) = 1 + \adap(\tr') \leq 1 + \ienv \nnatA = \ienv \nnatA'$ proved by $(a)$\\
%
2. $\forall x \in \mbox{Vars}. \ddep{x}(\tr) = \ddep{x}(\eop(\tr')) = 1 + \ddep{x}(\tr') \leq 1 + \dmap(x) = \dmap'(x)$ proved by $(b)$\\
%
3. $(\stepiA - j, \valr) = (\stepiA - j' - 1, \eop(\valr')) \in \lrv{\tbool} $.\\
%
By typing rule PRIMITIVE, we have $\eop(\valr'): \tbool$. $\eop(\valr')$ is either $\etrue$ or $\efalse$. So, we have $(\stepiA - j' - 1, \eop(\valr')) \in \lrv{\tbool}$.\\
%



\caseL
{
$   
	\inferrule*[right = VAR]{
      \Delta; \Gamma(x) = \type \\ 0 \leq \dmap(x) \mbox{ or equiv.\ } \dmap(x) \neq \bot
    }{
      \Delta; \Gamma; \dmap \tvdash{\nnatA} x: \type
    }
$
}
Assume $(\stepiA, \env) \in \lrv{\Gamma}$, $ \ienv \in \lrv{\Delta}$. TS: $(\stepiA, (\env, \ienv x)) \in \lre{\dmap}{\ienv \nnatA}{\ienv \type}$.\\
%
By inversion, STS: $\forall \valr, \tr, j. (\env, \ienv x \bigstep \valr, \tr) \land (\size{\tr} = j) \land (j \leq \stepiA) $\\
%
1. $ (\adap(\tr) \leq \ienv \nnatA)$;
%
2. $(\forall x \in \mbox{Vars}. \ddep{x}(\tr) \leq \dmap(x))$;
%
3. $(\stepiA - j, \valr) \in \lrv{\type}$\\
%
By E-VAR, pick $\valr = \env(\ienv x)$, $\tr = \ienv x$, $j = \size{\tr} = 0$, following 3 goals are proved:\\
%
1. $(\adap(\tr) = \adap(\ienv x) = 0 \leq \ienv \nnatA)$\\
%
2. $\forall y \in \ddep{y}(\ienv x)$, \\
$~~ y = \ienv x ~~ \ddep{y}(\ienv x) = 0 \leq \dmap(\ienv x)$\\
$~~ y \neq \ienv x ~~ \ddep{y}(\ienv x) = \bot \leq \dmap(y)$\\
%
3. $(\stepiA - j, \valr) = (\stepiA, \env(\ienv x)) \in \lrv{\ienv \type}$.\\
$~~$By definition of $(\stepiA, \env) \in \lrv{\Gamma}$, we have: $(\stepiA, \env(\ienv x))\in \lrv{\Gamma(\ienv x)}$, and $\Gamma(\ienv x) = \type$. 


\caseL
{
$    
 \inferrule{
     \dmap \wf{\type} \\
    }{
      \Delta; \Gamma; \dmap \tvdash{\nnatA} \enil: \tlist{\type}
    }
$
}
Assume $(\stepiA, \env) \in \lrv{\Gamma}$, $ \ienv \in \lrv{\Delta}$. TS: $(\stepiA, (\env, \enil)) \in \lre{\dmap}{\ienv \nnatA}{\ienv \tlist{\type}}$.\\
%
By inversion, STS: $\forall \valr, \tr, j. (\env, \enil \bigstep \valr, \tr) \land (\size{\tr} = j) \land (j \leq \stepiA) $\\
%
1. $ (\adap(\tr) \leq \ienv \nnatA)$;
%
2. $(\forall x \in \mbox{Vars}. \ddep{x}(\tr) \leq \dmap(x))$;
%
3. $(\stepiA - j, \valr) \in \lrv{\ienv \type}$\\
%
By E-NIL, we know : $ v = \enil $ and $ \tr = \trnil$ and $
\size{\trnil} = 0 \leq \stepiA$. \\
The following 3 goals are proved:\\ 
%
1. $(\adap(\tr) = \adap(\enil) = 0 \leq \ienv \nnatA)$, 
because $\ienv \nnatA$ is not negative.\\
%
2. $\forall x \in \mbox{Vars} \ddep{x}(\enil)$, 
$  \ddep{x}(\enil) = \bot \leq \dmap(x)$ 
proved from the definition of $\bot$. \\
%
3. $(\stepiA - 0, \enil) \in \lrv{ \ienv \tlist{ \type} }$, 
proved by inversion of the interpretation of $\lrv{ \ienv \tlist{\type} } $.\\

\caseL
{
$
  \inferrule{
   \Delta; \Gamma; \dmap_1 \tvdash{\nnatA_1} \expr_1 : \type ~(\star) \\
   \Delta; \Gamma; \dmap_2 \tvdash{\nnatA_2} \expr_2 : \tlist{\type} ~(\diamond)\\
   \dmap' = \max(\dmap_1, \dmap_2) \\
   \nnatA' = \max ( \nnatA_1, \nnatA_2 )
   }
   { 
   \Delta; \Gamma; \dmap' \tvdash{\nnatA'} \econs(\expr_1, \expr_2) :
     \tlist{\type}  
    }
$
}
Assume $(\stepiA, \env) \in \lrv{\Gamma}$, $ \ienv \in \lrv{\Delta}$. TS: $(\stepiA, (\env,
\ienv \econs(\expr_1 , \expr_2 ) )) \in \lre{\dmap'}{\ienv \nnatA'}{\ienv \tlist{\type}}$.\\
%
By inversion, STS: $\forall \valr, \tr, j. (\env, \ienv \econs(e_1,e_2) \bigstep \valr, \tr) \land (\size{\tr} = j) \land (j \leq \stepiA) $\\
% 
1. $ (\adap(\tr) \leq \ienv \nnatA')$;
%
2. $(\forall x \in \mbox{Vars}. \ddep{x}(\tr) \leq \dmap'(x))$;
%
3. $(\stepiA - j, \valr) \in \lrv{\ienv \tlist{\type}} $
%

  By ih on $(\star)$, we get: $(\stepiA, (\env,
\ienv \expr_1  )) \in \lre{\dmap_1}{\ienv \nnatA_1}{\type}~(1)$.\\
%
By inversion, we pick $v_1$, $\tr_1$ and $j_1$ 
%
s.t. $\env, \ienv \expr_1 \bigstep \valr_1, \tr_1$ 
%
and $\size{\tr_1} = j_1$
%
and $j_1 \leq \stepiA$.\\
%
We know : 
%
$ (\adap(\tr_1) \leq \ienv \nnatA_1) ~ (2) 
%
\land (\forall x \in \mbox{Vars}. \ddep{x}(\tr_1) \leq \dmap_1(x)) ~ (3)
%
\land (\stepiA - j_1, \valr_1) \in \lrv{\ienv \type} ~ (4)$.\\
%
By ih on $(\diamond)$
%
and $(\stepiA-j_1, \env) \in \lrv{\Gamma}$ by Lemma~\ref{lem:downward},
%
we get: $(\stepiA-j_1, (\env, \ienv \expr_2  )) \in \lre{\dmap_2}{\ienv \nnatA_2}{ \ienv \tlist{ \type} }~(5)$.\\
%
By inversion, we pick $v_2$, $\tr_2$ and $j_2$,  s.t. $\env, \ienv \expr_2
\bigstep \valr_2, \tr_2$ and $\size{\tr_2} = j_2$ and $j_2 \leq
\stepiA-j_1$.\\
%
We know : $ (\adap(\tr_2) \leq \ienv \nnatA_2) ~ (6) 
\land (\forall x \in \mbox{Vars}. \ddep{x}(\tr_2) \leq \dmap_2(x)) ~ (7)
\land (\stepiA - j_2, \valr_2) \in \lrv{\ienv \tlist{\type}} ~ (8)$.
%
\[
\inferrule{
\env, \ienv \expr_1 \bigstep \valr_1, \tr_1 \\
\env, \ienv \expr_2 \bigstep \valr_2, \tr_2
}
{ \env, \ienv \econs (\expr_1, \expr_2)  \bigstep \econs (\valr_1, \valr_2),
  \trcons(\tr_1, \tr_2)
}
\]
%
By E-Cons, we set : $\valr = \econs(\valr_1, \valr_2)$ and $\tr =
\trcons(\tr_1, \tr_2)$ s.t. $j = \size{\tr}= j_1 + j_2 \leq \stepiA$.\\
3 following goals are proved:\\ 
%
1. $(\adap(\tr) = \adap(\trcons(\tr_1, \tr_2)) \leq
\ienv \nnatA')$, by $(2)$ and $(6)$.\\
2. $\forall x \in \mbox{Vars}. \ddep{x}(\trcons(\tr_1, \tr_2))
= \max ( \ddep{x}(\tr_1), \ddep{x}(\tr_2) ) \leq
\dmap'(x)$ by $(3)$ and $(7)$. \\
3. $(\stepiA - j_1 -j_2, \econs(\valr_1,\valr_2)  )  \in \lrv{ \ienv \tlist{\type} }$.
%
By inversion of the definition of $\lrv{\ienv \tlist{\type}}$,
%
suffice to show:
%
$(\stepiA-j_1-j_2, \valr_1) \in \lrv{\ienv \type}$,
%
which is proved by Lemma~\ref{lem:downward} on $(4)$,
%
$(\stepiA-j_1-j_2, \valr_2) \in \lrv{ \ienv \tlist{\type}}$ 
%
is proved by $(8)$.\\


\caseL{
$
\inferrule
{
    \Delta; \Gamma; \dmap_1 \tvdash{\nnatA_1} \expr_1 : \type_1~(\star) \\
    \Delta; \Gamma, x:\type_1 ; \dmap_2[x:q] \tvdash{\nnatA_2} \expr_2 :
    \type_2 ~(\diamond)\\
    \dmap' = \max( \dmap_2, \dmap_1 + q ) \\
    \nnatA' = \max ( \nnatA_2, \nnatA_1 + q )
}
{
	\Delta; \Gamma; \dmap' \tvdash{\nnatA'}  \elet x = \expr_1 \ein \expr_2 : \type
}
$
}
%
Assume $(\stepiA, \env) \in \lrv{\Gamma} ~ (\triangle)$, $ \ienv \in \lrv{\Delta}$.\\
%
TS: $(\stepiA, (\env, \ienv (\elet x = \expr_1 \ein \expr_2) )) 
	\in \lre{\dmap'}{\ienv \nnatA'}{\ienv \type}$.\\
%
By inversion, STS: 
%
$\forall \valr, \tr, j$, \\
s.t.,
%
$(\env, \ienv (\elet x = \expr_1 \ein \expr_2) \bigstep \valr, \tr)$
%
$\land (\size{\tr} = j)$
%
$\land (j \leq \stepiA) $, \\
% 
1. $ (\adap(\tr) \leq \ienv \nnatA')$;\\
%
2. $(\forall x \in \mbox{Vars}. \ddep{x}(\tr) \leq \dmap'(x))$;\\
%
3. $(\stepiA - j, \valr) \in \lrv{\ienv \type} $
%
By ih on $(\star)$, 
%
we get: $(\stepiA, (\env, \ienv \expr_1  )) \in \lre{\dmap_1}{\ienv \nnatA_1}{\ienv \type_1}~(1)$.\\
%
By inversion, pick any $v_1$, $\tr_1$, $j_1$,
%
s.t. $(\env, \ienv \expr_1) \bigstep (\valr_1, \tr_1)$ 
%
$\land \size{\tr_1} = j_1$
%
$\land j_1 \leq \stepiA$.\\
%
We know:
%
$ \adap(\tr_1) \leq \ienv \nnatA_1 ~ (2)$
%
$\land \forall x \in \mbox{Vars}. \ddep{x}(\tr_1) \leq \dmap_1(x) ~ (3)$
%
$\land (\stepiA - j_1, \valr_1) \in \lrv{\ienv \type_1} ~ (4)$.\\
%
By Lemma~\ref{lem:downward} on $(\triangle)$,
%
we get : $(\stepiA-j_1, \env) \in \lrv{\Gamma}~(5)$. \\
%
Set $\env' = \env[x \mapsto \valr_1]$,
%
we know:
%
 $(\stepiA-j_1, \env') \in \lrv{\Gamma,x: \ienv \type_1} ~(6) $
 %
by $(4)$ and $(5)$.
%
By ih on $(\diamond)$ and $(6)$, 
%
we know:
%
$(\stepiA-j_1, (\env', \ienv \expr_2  )) \in \lre{\dmap_2[x:q]}{\ienv \nnatA_2}{\ienv \type} ~(7) $.\\
%
By inversion $(7)$, pick any $v_2$, $\tr_2$, $j_2$,\\
%
s.t., $(\env', \ienv \expr_2) \bigstep (\valr_2, \tr_2)$ 
%
$\land \size{\tr_2} = j_2$
%
$\land j_2 \leq \stepiA - j_1$.\\
%
We know:
%
$ (\adap(\tr_2) \leq \ienv \nnatA_2) ~ (8) 
%
\land (\forall x \in \mbox{Vars}. \ddep{x}(\tr_2) \leq \dmap_2[x:q](x)) ~ (9)
%
\land (\stepiA - j_1 - j_2, \valr_2) \in \lrv{\ienv \type} ~ (10)$.
%
\[
\inferrule{
  \env, \ienv \expr_1 \bigstep \valr_1, \tr_1 \\
  \env[x \mapsto \valr_1] , \ienv \expr_2 \bigstep \valr_2, \tr_2
}
{\env, \ienv (\elet x;q = \expr_1 \ein \expr_2) \bigstep \valr_2, \trlet (x,
  \tr_1, \tr_2) }
\]
By E-LET, pick $\valr = \valr_2$, 
%
$\tr = \ienv \trlet (x, \tr_1, \tr_2)$,
%
s.t., $j = j_1 +j_2 \leq \stepiA$,\\
%
the following 3 goals are proved: \\
%
1. $(\adap(\tr) = \adap(\ienv \trlet (x, \tr_1, \tr_2) ) \leq \ienv \nnatA')$, 
%
proved by $(2)$ ,$(8)$and $(9)$.\\
%
2. $\forall y \in \mbox{Vars}$,
%
$\ddep{y} (\ienv \trlet (x, \tr_1, \tr_2)) \leq \dmap'(y)$
%
proved by $(3)$ and $(9)$. \\
%
3. $(\stepiA - j_1 -j_2, \valr_2 ) \in \lrv{\ienv \type }$,
%
proved by $(10)$.\\

\end{mainitem}
\end{proof}

%%%%%%%%%%%%%%%%%%%%%%%%%%%%%%%%%%%%%%%%%%%%%%%%%%%%%%%%%%%%%%%%%%%%%%%%%%%%%%%%%%%%%%%%%%%%%%%%%%%%%%%%%%
%Proof of old Fundamental Theorem 
%%%%%%%%%%%%%%%%%%%%%%%%%%%%%%%%%%%%%%%%%%%%%%%%%%%%%%%%%%%%%%%%%%%%%%%%%%%%%%%%%%%%%%%%%%%%%%%%%%%%%%%%%%
% \noindent {\bf case}

% \[
%    \inferrule*[right = PAIR]{
%       \Gamma; \dmap_1 \tvdash{\nnatA_1} \expr_1: \type_1 ~ (1) 
%    		\\
%       \Gamma; \dmap_2 \tvdash{\nnatA_2} \expr_2: \type_2 ~ (2)
%       	\\\\
%       \dmap' = \max(\dmap_1,\dmap_2) 
%       	\\
%       \nnatA' = \max(\nnatA_1,\nnatA_2)
%     }{
%       \Gamma; \dmap' \tvdash{\nnatA'} (\expr_1, \expr_2): \type_1 \times \type_2
%     }
% \]
% Assume $(\stepiA, \env) \in \lrv{\Gamma}$, 
% to show: $(\stepiA, (\env, (\expr_1, \expr_2))) \in \lre{\dmap'}{\nnatA'}{\type_1 \times \type_2}$.\\
% %
% By inversion, STS: $\forall \eapp  \valr, \tr, j.$ 
% $(\env, (\expr_1, \expr_2) \bigstep \valr, \eapp  \tr) \land (\size{\tr} = j) \land (j \leq \stepiA)$\\
% %
% $\Rightarrow (\adap(\tr) < \nnatA') \land (\forall x \in \mbox{Vars}. \ddep{x}(\tr) \leq \dmap'(x)) \land ((\stepiA - j, \valr) \in \lrv{\type_1 \times \type_2})$.\\
% %
% By induction hypothesis on $(1)$, we have $(\stepiA, \expr_1) \in \lre{\dmap_1}{\nnatA_1}{\type_1}$.\\
% %
% By inversion, pick any $\valr_1,  \tr_1, j_1$ s.t.,
% $(\env, \expr_1 \bigstep \valr_1, \tr_1) ~ (3) 
% \land (\size{\tr_1} = j_1) 
% \land (j_1 < \stepiA) $,\\
% %
% we have: \\
% $(\adap(\tr_1) \leq \nnatA_1) ~ (4)$ 
% $\land \forall x, \ddep{x}(\tr_1) \leq \dmap_1(x) ~ (5)$ 
% $\land (\stepiA - j_1, \valr_1) \in \lrv{\type_1} ~ (6) $.\\
% %
% By induction hypothesis on $(2)$ and the assumption $(\stepiA - j_1,
% \env) \in \lrv{\Gamma}$ from Lemma~\ref{lem:downward}, we have $(\stepiA - j_1, \expr_2) \in \lre{\dmap_2}{\nnatA_2}{\type_2}$.\\
% %
% By inversion, pick $\valr_2,  \tr_2,j_2 $ s.t., $ (\env, \expr_2
% \bigstep \valr_2, \tr_2) ~ (7) \land (\size{\tr_2} = j_2) \land j_2 < \stepiA-j_1 $,\\
% %
% we have:\\
% $\adap(\tr_2) \leq \nnatA_2 ~ (8)$
% $\land \forall x\in \mbox{Vars}. \ddep{x}(\tr_2) \leq \dmap_2(x) ~ (9)$
% $\land (\stepiA- j_1- j_2, \valr_2) \in \lrv{\type_2} ~ (10)$.\\
% %
% By E-Pair rule, we have:\\
% \[
% \inferrule*[right = E-Pair]
% {\env, \expr_1 \bigstep \valr_1, \tr_1 ~ (3)  \and \env, \expr_2
% \bigstep \valr_2, \tr_2 ~ (7) }
% {\env, (\expr_1, \expr_2) \bigstep (\valr_1, \valr_2), (\tr_1, \tr_2)}
% \]
% Pick $\valr = (\valr_1, \valr_2), \tr = (\tr_1, \tr_2), j = |(\tr_1, \tr_2)| = j_1 + j_2$, s.t. $\env, (\expr_1, \expr_2) \bigstep \valr, \tr \land \size{\tr} = j \land j \leq \stepiA$.\\
% %
% STS:\\
% 1. $\adap(\tr) = \adap(\tr_1, \tr_2) < \nnatA' = \max(\nnatA_1, \nnatA_2)$, which is proved by $(4), (8)$.\\
% %
% 2. $\forall x \in \mbox{Vars}. \ddep{x}(\tr_1, \tr_2) = \max(\ddep{x}(\tr_1), \ddep{x}(\tr_2)) \leq \dmap'(x) = \max(\dmap_1(x), \dmap_2(x))$, which is proved by $(5),(9)$.\\
% %
% 3. $(\stepiA - (j_1 + j_2), (\valr_1, \valr_2)) \in \lrv{\type_1
%   \times \type_2}$, which is proved by $(10)
% $ and applying Lemma~\ref{lem:downward} on $(6)$.\\


% \noindent {\bf case}
% \[
%     \inferrule*[right = FIX]{
%       \Gamma, f: (\tarr{\type_1}{\type_2}{\nnatbiA}{\dmap}{\nnatA}), x: \type_1;
%       \dmap[f: \infty, x: \nnatbiA]
%       \tvdash{\nnatA}
%       \expr: \type_2
%        ~ (\triangle)
%     }{
%       \Gamma; \dmap' \tvdash{\nnatA'} \efix f(x).\expr: (\tarr{\type_1}{\type_2}{\nnatbiA}{\dmap}{\nnatA})
%     }
% \]
% Assume: $(\stepiA, \env) \in \lrv{\tr}$. TS: $(\stepiA, (\env, \efix f(x). \expr)) \in \lre{\dmap'}{\nnatA}{\type}$.\\
% %
% By inversion, STS: $\forall \valr, \tr, j.$
% $(\env, \efix f(x). \expr \bigstep \valr, \tr) \land (\size{\tr} = j) \land (j \leq \stepiA)$
% $ \Rightarrow (\adap(\tr) \leq \nnatA') \land (\forall x \in \mbox{Vars}. \ddep{x}(\tr) \leq \dmap'(x)) \land ((\stepiA - j), \valr) \in \lrv{(\tarr{\type_1}{\type_2}{\nnatbiA}{\dmap}{\nnatA})}$.\\
% %
% By E-FIX, let $\valr = (\efix f(x). \expr, \env)$, we know:\\
% $(\env, \efix f(x). \expr) \bigstep ((\efix f(x). \expr, \env), \efix f(x). \expr)~(1) $;\\
% $|\efix f(x). \expr|= j \land j < \stepiA ~(2).$\\
% %
% Suppose $(1), (2)$, STS: $ (\adap(\efix f(x). \expr) = 0 \leq \nnatA')
% \land (\forall x \in \mbox{Vars}. \ddep{x}(\efix f(x). \expr) = \bot \leq \dmap'(x)) \land ((\stepiA -j), (\efix f(x). \expr, \env)) \in \lrv{(\tarr{\type_1}{\type_2}{\nnatbiA}{\dmap}{\nnatA})}$.\\
% %
% The first two goals are proved , we prove the third proposition
% $((\stepiA -j), (\efix f(x). \expr, \env)) \in
% \lrv{(\tarr{\type_1}{\type_2}{\nnatbiA}{\dmap}{\nnatA})}$. To prove
% this,  we first prove on general theorem: \\
% set $\stepiA - j = \stepiA'$, $\forall m \leq \stepiA', (m, (\efix f(x). \expr, \env)) \in \lrv{(\tarr{\type_1}{\type_2}{\nnatbiA}{\dmap}{\nnatA})}$.\\
% %
% Induction on $m$:

% {\bf Subcase 1:} $m = 0$,

% 	TS: $\forall j' < 0$, $(j', \env[\dots] \expr) \in \lre{\nnatA}{\dmap[\dots]}{(\tarr{\type_1}{\type_2}{\nnatbiA}{\dmap}{\nnatA})}$,
	
% 	it is obviously true because $j' < 0 \notin \mathbb{N}$.

% {\bf Subcase 2:} $m = m' + 1 \leq \stepiA'$, 

% 	TS: $ (m, (\efix f(x). \expr, \env)) \in \lrv{(\tarr{\type_1}{\type_2}{\nnatbiA}{\dmap}{\nnatA})}$.

% 	Pick $\forall j' < m' + 1$, $\forall (j', \valr_m) \in \lrv{\type_1}$,
	
% 	STS: $(j', (\env[x \mapsto \valr_m, f \mapsto (\efix f(x).\expr, \env)], \expr)) \in \lre{\dmap[x: \nnatbiA, f: \infty]}{\nnatA}{\type_2} ~ (\star)$.
	
% 	By sub ih, we have:
% 	$(m', (\efix f(x). \expr, \env)) \in \lrv{(\tarr{\type_1}{\type_2}{\nnatbiA}{\dmap}{\nnatA})} ~ (a)$.

% 	Pick $\env' = \env[x \mapsto \valr_m, f \mapsto (\efix f(x).\expr, \env)]$,
	
% 	we know: $(j', \env') \in \lrv{\Gamma,
%           f:(\tarr{\type_1}{\type_2}{\nnatbiA}{\dmap}{\nnatA}),  x:
%           \type_1} ~ (\diamond)$ proved from the following premises:\\ 
	
% 	1. $ (\stepiA, \env) \in \lrv{\Gamma}$, applying Lemma~\ref{lem:downward} on assumption, we get: $(j', \env) \in \lrv{\Gamma}$.
	
% 	2. $(j', \valr_m) \in \lrv{\type}$, from the assumption.
	
% 	3. $(j', (\efix f(x). \expr, \env)) \in \lrv{(\tarr{\type_1}{\type_2}{\nnatbiA}{\dmap}{\nnatA})}$ from $(a)$.\\
% %
% From above, $\star$ is proved by induction hypothesis on $\diamond$ and $\triangle$.\\
% %


% \noindent {\bf case}
% \[
%     \inferrule*[right = APP]
%     {
%       \Gamma; \dmap_1 \tvdash{\nnatA_1} \expr_1: (\tarr{\type_1}{\type_2}{\nnatbiA}{\dmap}{\nnatA})~(\star) \\
%       \Gamma; \dmap_2 \tvdash{\nnatA_2} \expr_2: \type_1 ~(\diamond)\\\\
%       \nnatA' = \nnatA_1 + \max(\nnatA, \nnatA_2 + \nnatbiA) \\
%       \dmap' = \max(\dmap_1, \nnatA_1 + \max(\dmap, \dmap_2 + \nnatbiA))
%     }{
%       \Gamma; \dmap' \tvdash{\nnatA'} \expr_1 \eapp \expr_2 : \type_2
%     }
% \]
% %
% Assume $(\stepiA, \env) \in \lrv{\Gamma} ~ (\square)$, TS: $(\stepiA, (\env, \expr_1 \eapp \expr_2)) \in \lre{\nnatA'}{\dmap'}{\type_2}$.\\
% %
% By inversion, pick any $\valr, \tr, j$ s.t. $((\env, \expr_1 \eapp \expr_2) \bigstep (\valr, \tr)) \land (\size{\tr} = j) \land (j \leq \stepiA) $.\\
% %
% STS: $(\adap(\tr) \leq \nnatA') \land (\forall x \in \mbox{Vars}. \ddep{x}(\tr) \leq \dmap'(x)) \land (((\stepiA - j), \valr) \in \lrv{\type_2})$.\\
% %
% By induction hypothesis on $\star$ and $\square$, we get: $(\stepiA, (\env, \expr_1)) \in \lre{\dmap_1}{\nnatA_1}{\tarr{\type_1}{\type_2}{\nnatbiA}{\dmap}{\nnatA}} ~(1) $.\\
% %
% Inversion on $(1)$, we get:\\
% %
% Pick any $\valr_1, \tr_1, j_1$, s.t. $((\env, \expr_1) \bigstep (\valr_1, \tr_1)) ~(a) \land (\size{\tr_1} = j_1) \land (j_1 < \stepiA)$, \\
% %
% we know: $(\adap(\tr_1) \leq \nnatA_1)~(2) \land (\forall x \in \mbox{Vars}. \ddep{x}(\tr_1) \leq \dmap_1(x))~(3) \land ((\stepiA - j_1), \valr_1) \in \lrv{\tarr{\type_1}{\type_2}{\nnatbiA}{\dmap}{\nnatA}} ~ (4)$\\
% %
% we have: $\valr_1$ is a function.
% %
% Let $\valr_1 = (\efix f(x). \expr, \env') ~(b)$ s.t. $\tr_1= \efix f(x). \expr$\\
% By inversion on $(4)$, we have $\forall j' < (\stepiA - j_1). \forall (j', \valr') \in \lrv{\type_1}. (j', (\env'[x \mapsto \valr', f \mapsto (\efix f(x).\expr, \env')], \expr)) \in \lre{\dmap[x : \nnatbiA, f: \infty]}{\nnatA}{\type_2} ~ (5)$.\\ 
% %
% By induction hypothesis on $\diamond$ and $\square$, we get:
% %
% $(\stepiA, (\env, \expr_2)) \in \lre{\dmap_2}{\nnatA_2}{\type_1} ~(6)$.\\
% %
% Inversion on $(6)$, we get:\\
% %
% Pick any $\valr_2, \tr_2, j_2$, s.t. $((\env, \expr_2) \bigstep (\valr_2, \tr_2))~(c) \land (\size{\tr_2} = j_2) \land (j_2 \leq \stepiA)$,\\
% %
% we know: $(\adap(\tr_2) \leq \nnatA_2)~(7) \land (\forall x \in \mbox{Vars}. \ddep{x}(\tr_2) \leq \dmap_2(x))~(8) \land ((\stepiA - j_2), \valr_2) \in \lrv{\type_1} ~ (9)$\\
% %
% Apply Lemma~\ref{lem:downward} on $(9)$, we have $((\stepiA - j_2 - j_1-1), \valr_2) \in \lrv{\type_1}$\\
% %
% Pick $j' = \stepiA - j_1 - j_2 - 1, \valr' = \valr_2$, we have: $(\stepiA - j_1 - j_2-1, (\env'[x \mapsto \valr_2, f \mapsto \valr_1], \expr)) \in \lre{\dmap[x : \nnatbiA, f: \infty]}{\nnatA}{\type_2} ~ (10)$ from $(5)$.\\
% %
% Inversion on $(10)$, let $\env'' = \env'[x \mapsto \valr_2, f \mapsto \valr_1]$, pick any $\valr'', \tr'', j''$, s.t. $((\env'', \expr) \bigstep (\valr'', \tr''))~(d) \land (\size{\tr''} = j'') \land (j'' \leq \stepiA - j_1 - j_2-1)$, we have:\\
% %
% $\adap(\tr'') \leq n ~(11)  \land (\forall x \in \ddep{x}(\tr'')\leq
% \dmap[x : \nnatbiA, f: \infty](x)) ~(12) \land (\stepiA - j_1 - j_2 - j''-1, \valr'') \in \lrv{\type_2}~(13)$.\\
% %
% Apply E-APP rule on $(a) (b) (c) (d)$ we have:
% \[
%   \inferrule{
%     \env, \expr_1 \bigstep \valr_1, \tr_1 ~ (a) \\
%     \valr_1 = (\efix f(x).\expr, \env')~(b) \\\\
%     \env, \expr_2 \bigstep \valr_2, \tr_2 ~ (c) \\
%     \env'[f \mapsto \valr_1, x \mapsto \valr_2], \expr \bigstep \valr'', \tr'' ~(d)
%   }{
%     \env, \expr_1 \eapp \expr_2 \bigstep \valr'', \trapp{\tr_1}{\tr_2}{f}{x}{\tr''}
%   }
% \]
% Pick $\valr = \valr'', j = j_1 + j_2 + j'' + 1, \tr =
% \trapp{\tr_1}{\tr_2}{f}{x}{\tr''}$ s.t. $ \env, \expr_1 \eapp \expr_2
% \bigstep \valr, \tr  \land |tr| = j \land j \leq k $.  \\
% Suffice to show the follwing three:\\
% %
% 1. $\adap(\tr) = \adap(\trapp{\tr_1}{\tr_2}{f}{x}{\tr''}) =
% \adap(\tr_1) + \max(\adap(\tr''), \adap(\tr_2) + \ddep{x}(\tr'')) \leq
% \nnatA_1 + \max(\nnatA, \nnatA_2 + \nnatbiA) = \nnatA'$ proved by $(2),(7),(11),(12)$.\\
% %
% 2. $\forall y \in \mbox{Vars}. \ddep{y}(\tr) = \max(\ddep{y}(\tr_1), \adap(\tr_1) +
% \max(\ddep{y}(\tr''), \ddep{y}(\tr_2) + \ddep{x}(\tr''))) \leq
% \max(\dmap_1, \nnatA_1 + \max(\dmap, \dmap_2 + \nnatbiA)) = \dmap'(y)$
% proved by $(2),(3),(8),(12)$. \\
% %
% 3. $(\stepiA - j, \valr) = (\stepiA - j_1 - j_2 - j''-1, \valr'') \in
% \lrv{\type_2}$ proved by $(13)$.\\
% %


% \noindent {\bf case}
% \[
%     \inferrule*[right = IF]{
%       \Gamma; \dmap_1 \tvdash{\nnatA_1} \expr_1: \tbool ~ (\star) \\
%       \Gamma; \dmap \tvdash{\nnatA} \expr_2: \type ~(\diamond) \\
%       \Gamma; \dmap \tvdash{\nnatA} \expr_3: \type ~(\triangle) \\\\
%       \nnatA' = \nnatA_1 + \nnatA \\
%       \dmap' = \max(\dmap_1, \nnatA_1 + \dmap)
%     }{
%       \Gamma; \dmap' \tvdash{\nnatA'} \eif(\expr_1, \expr_2, \expr_3):  \type
%     }
% \]
% Assume $(\stepiA, \env) \in \lrv{\Gamma}$, TS: $(\stepiA, (\env, \eif(\expr_1, \expr_2, \expr_3))) \in \lre{\dmap'}{\nnatA'}{\type}$.\\
% %
% Inversion on it, $\forall \valr, \tr, j$ $(\env, \eif(\expr_1, \expr_2, \expr_3) \bigstep \valr, \tr) \land (\size{\tr} = j) \land (j \leq \stepiA)$\\
% %
% STS: $(\adap(\tr) \leq \nnatA') \land (\forall x \in \mbox{Vars}. \ddep{x}(\tr) \leq \dmap'(x)) \land ((\stepiA - j, \valr) \in \lrv{\type})$.\\
% %
% By induction hypothesis on $\star$ and $(\stepiA-1, \env) \in
% \lrv{\Gamma}$ by  Lemma~\ref{lem:downward}, we get: $(\stepiA-1, (\env, \expr_1)) \in \lre{\dmap_1}{\nnatA_1}{\tbool} ~(1)$.\\
% %
% Inversion on $(1)$, pick $ \valr_1, \tr_1, j_1. (\env, \expr_1 \bigstep \valr_1, \tr_1) ~ (a) \land (\size{\tr_1} = j_1) \land (j_1 \leq \stepiA-1)$,\\
% %
% we know: $(\adap(\tr_1) \leq \nnatA_1)~(4) \land (\forall x. \in
% \ddep{x}(\tr_1) \leq \dmap_1(x)) ~(5) \land ((\stepiA -1- j_1, \valr_1) \in \lrv{\tbool})$\\
% %
% By induction hypothesis on $\diamond$ and $(\stepiA-1-j_1, \env) \in
% \lrv{\Gamma}$ by Lemma~\ref{lem:downward} on the assumption, we get: $(\stepiA-1-j_1, (\env, \expr_2)) \in \lre{\dmap}{\nnatA}{\type} ~(2)$\\
% %
% By induction hypothesis on $\triangle$ and $(\stepiA-1-j_1, \env) \in
% \lrv{\Gamma}$ , we get: $(\stepiA-1-j_1, (\env, \expr_3)) \in \lre{\dmap}{\nnatA}{\type} ~(3)$\\
% %
% Induction on $\valr_1$, we have two following subcases:

% {\bf Subcase1:} $\valr_1 = \etrue$, \\
% %
% inversion on $(2)$, pick any $\valr_2, \tr_2, j_2. (\env, \expr_2 \bigstep \valr_2, \tr_2) ~ (b) \land (\size{\tr_2} = j_2) \land (j_2 \leq \stepiA-1-j_1)$,\\
% %
% we know: $(\adap(\tr_2) \leq \nnatA)~(6) \land (\forall x \in \mbox{Vars}. \ddep{x}(\tr_2) \leq \dmap(x))~(7) \land ((\stepiA-1 - j_1-j_2, \valr_2) \in \lrv{\type})~(8)$.\\
% %
% Apply E-IFT on $(a) (b)$, we get:
% \[
%   \inferrule{
%     \env, \expr_1 \bigstep \etrue, \tr_1 ~ (a) \\
%     \env, \expr_2 \bigstep \valr_2, \tr_2 ~ (b)
%   }{
%     \env, \eif(\expr_1, \expr_2, \expr_3) \bigstep \valr_2, \trift(\tr_1, \tr_2)
%   }
% \]
% Pick $\valr = \valr_2, \tr = \trift(\tr_1, \tr_2), j =j_1+ j_2+1$ s.t. $
% \env, \eif(\expr_1, \expr_2, \expr_3) \bigstep \valr, \tr \land j =
% |tr| \land j = j_1 + j_2+1 \leq k$.\\
%  STS the following 3 goals:\\
% %
% 1. $\adap(\tr) = \adap(\trift(\tr_1, \tr_2)) = \adap(\tr_1) +
% \adap(\tr_2)) \leq \nnatA_1 + \nnatA = \nnatA'$. proved by $(4),(6)$.\\
% %
% 2. $\forall x \in \mbox{Vars} . \ddep{x}(\tr) = \max(\ddep{x}(\tr_1), \adap(\tr_1) +
% \ddep{x}(\tr_2)) \leq \max(\dmap_1(x), \nnatA_1(x) + \dmap(x)) =
% \dmap'(x)  $ proved by $(5), (7)$.\\
% %
% 3. $(\stepiA - j, \valr) = (\stepiA-1 -j_1- j_2, \valr_2) \in
% \lrv{\type}$ proved by $(8)$.

% {\bf Subcase2:} $\valr_1 = \efalse$\\
% %
% inversion on $(3)$, pick any $\valr_3, \tr_3, j_3. (\env, \expr_3 \bigstep \valr_3, \tr_3) ~ (b) \land (\size{\tr_3} = j_3) \land (j_3 \leq \stepiA-1-j_1)$,\\
% %
% we know: $(\adap(\tr_3) \leq \nnatA) ~(9) \land (\forall x \in
% \mbox{Vars}. \ddep{x}(\tr_3) \leq \dmap(x))~(10) \land ((\stepiA -1-j_1- j_3, \valr_3) \in \lrv{\type})~(11)$.\\
% %
% Apply E-IFT on $(a) (b)$, we get:
% \[
%   \inferrule{
%     \env, \expr_1 \bigstep \etrue, \tr_1 ~ (a) \\
%     \env, \expr_3 \bigstep \valr_3, \tr_3 ~ (b)
%   }{
%     \env, \eif(\expr_1, \expr_2, \expr_3) \bigstep \valr_3, \trift(\tr_1, \tr_3)
%   }
% \]
% Pick $\valr = \valr_3, \tr = \trift(\tr_1, \tr_3), j =j_1+ j_3+1$,  s.t. $
% \env, \eif(\expr_1, \expr_2, \expr_3) \bigstep \valr, \tr \land j =
% |tr| \land j = j_1 + j_3+1 \leq k$.\\
%  STS the following 3 goals:\\
% %
% 1. $\adap(\tr) = \adap(\trift(\tr_1, \tr_3)) = \adap(\tr_1) +
% \adap(\tr_3)) \leq \nnatA_1 + \nnatA = \nnatA'$  proved by $(4),(9)$.\\
% %
% 2. $\forall x \in \mbox{Vars}. \ddep{x}(\tr) = \max(\ddep{x}(\tr_1), \adap(\tr_1) +
% \ddep{x}(\tr_3)) \leq \max(\dmap_1(x), \nnatA_1(x) + \dmap(x)) =
% \dmap'(x) $ proved by $(5),(10)$.\\
% %
% 3. $(\stepiA - j, \valr) = (\stepiA -1 - j_1-j_3, \valr_3) \in
% \lrv{\type} $ proved by $(11)$.\\



% \noindent {\bf case}
% \[
%     \inferrule*[right = FST]{
%       \Gamma; \dmap \tvdash{\nnatA} \expr: \type_1 \times \type_2
%     }{
%       \Gamma; \dmap \tvdash{\nnatA} \eprojl(\expr): \type_1
%     }
% \]
% Assume $(\stepiA, \env) \in \lrv{\Gamma}$, TS: $(\stepiA, (\env, \eprojl(\expr))) \in \lre{\dmap}{\nnatA}{\type_1} $.\\
% %
% Unfold, pick any $ \valr, \tr, j. (\env, \eprojl(\expr)   \bigstep \valr, \tr) \land (\size{\tr} = j) \land (j \leq \stepiA) $,\\
% %
% STS: $ (\adap(\tr) \leq \nnatA) \land (\forall x \in \mbox{Vars}. \ddep{x}(\tr) \leq \dmap(x)) \land (\stepiA - j, \valr) \in \lrv{\type_1} $.\\
% %
% By induction hypothesis, we get: $(\stepiA-1, (\env, \expr)) \in \lre{\dmap}{\nnatA}{\type_1 \times \type_2} ~(1)$.\\
% %
% Inversion on $(1)$, $\forall \valr', \tr', j'. (\env, \expr \bigstep \valr', \tr') \land (\size{\tr'} = j') \land (j' \leq \stepiA-1) $,\\
% %
% We know: $(\adap(\tr') \leq \nnatA) ~ (a) 
% \land (\forall x \in \mbox{Vars}. \ddep{x}(\tr') \leq \dmap(x)) ~ (b)
% \land ((\stepiA -1 - j', \valr') \in \lrv{\type_1 \times \type_2}) ~ (c)$.\\
% %
% Pick any $\valr_1, \valr_2$, $\valr' = (\valr_1, \valr_2)$,
% % Let $\expr = (\expr_1, \expr_2)$, by E-PAIR, we have:\\
% % %
% % $\tr' = (\tr_1, \tr_2), \valr' = (\valr_1, \valr_2), j' = \size{\tr'} = \size{\tr_1} + \size{\tr_2} = j_1 + j_2$.\\
% %
% inversion on $(c)$, we have:\\
% %
% $(\stepiA - j', \valr_1) \in \lrv{\type_1}~(d) \land (\stepiA - j', \valr_2) \in \lrv{\type_2}$.\\
% %
% By E-FST, pick $\valr = \valr_1, j = j'+1, \tr = \trprojl(\tr')$, STS:\\
% %
% 1. $\adap(\tr) = \adap(\trprojl(\tr')) = \adap(\tr') \leq \nnatA$ by $(a)$.\\
% %
% 2. $\forall x .\ddep{x}(\tr) \implies \forall x . \ddep{x}(\tr') \leq \dmap(x)$ by $(b)$.\\
% %
% 3. $(\stepiA - j, \valr) = (\stepiA -1- j', \valr_2) \in
% \lrv{\type_1}$ by $(d)$.
% \\




% \noindent {\bf case}
% \[
%     \inferrule*[right = SND]{
%       \Gamma; \dmap \tvdash{\nnatA} \expr: \type_1 \times \type_2
%     }{
%       \Gamma; \dmap \tvdash{\nnatA} \eprojr(\expr): \type_2
%     }
% \]
% Assume $(\stepiA, \env) \in \lrv{\Gamma}$, TS: $(\stepiA, (\env, \eprojr(\expr))) \in \lre{\dmap}{\nnatA}{\type_2} $.\\
% %
% Unfold, pick any $ \valr, \tr, j. (\env, \eprojr(\expr) \bigstep \valr, \tr) \land (\size{\tr} = j) \land (j \leq \stepiA) $,\\
% %
% STS: $ (\adap(\tr) \leq \nnatA) \land (\forall x \in \mbox{Vars}. \ddep{x}(\tr) \leq \dmap(x)) \land (\stepiA - j, \valr) \in \lrv{\type_1} $.\\
% %
% By induction hypothesis, we get: $(\stepiA-1, (\env, \expr)) \in \lre{\dmap}{\nnatA}{\type_1 \times \type_2} ~(1)$.\\
% %
% Inversion on $(1)$, $\forall \valr', \tr', j'. (\env, \expr \bigstep \valr', \tr') \land (\size{\tr'} = j') \land (j' \leq \stepiA-1) $,\\
% %
% We know: $(\adap(\tr') \leq \nnatA) ~ (a) 
% \land (\forall x \in \ddep{x}(\tr') \leq \dmap(x)) ~ (b)
% \land ((\stepiA -1- j', \valr') \in \lrv{\type_1 \times \type_2}) ~ (c)$.\\
% %
% Pick any $\valr_1, \valr_2$, $\valr' = (\valr_1, \valr_2)$,
% % Let $\expr = (\expr_1, \expr_2)$, by E-PAIR, we have:\\
% % %
% % $\tr' = (\tr_1, \tr_2), \valr' = (\valr_1, \valr_2), j' = \size{\tr'} = \size{\tr_1} + \size{\tr_2} = j_1 + j_2$.\\
% %
% inversion on $(c)$, we have:\\
% %
% $(\stepiA - j', \valr_1) \in \lrv{\type_1}~(d) \land (\stepiA - j', \valr_2) \in \lrv{\type_2}~(e)$.\\
% %
% By E-SND, pick $\valr = \valr_2, j = j'+1, \tr = \trprojr(\tr')$, STS:\\
% %
% 1. $\adap(\tr) = \adap(\trprojr(\tr')) = \adap(\tr') \leq \nnatA$ by $(a)$\\
% %
% 2. $\forall x \in \mbox{Vars} .\ddep{x}(\tr) \implies \forall x . \ddep{x}(\tr') \leq
% \dmap(x)$ proved by $(b)$.\\
% %
% 3. $(\stepiA - j, \valr) = (\stepiA -1 - j', \valr_2) \in \lrv{\type_2}$
% proved by $(e)$.\\


% \noindent {\bf case}
% \[
%     \inferrule*[right = TRUE]{
%     }{
%       \Gamma; \dmap \tvdash{\nnatA} \etrue: \tbool
%     }
% \]
% Assume $ (\stepiA, \env) \in \lrv{\tbool} $, TS: $(\stepiA, (\env, \etrue)) \in \lre{\dmap}{\nnatA}{\tbool}$.\\
% %
% By inversion, STS: $\forall \valr, \tr, j. (\env, \etrue \bigstep \valr, \tr) \land (\size{\tr} = j) \land (j \leq \stepiA) $\\
% $\Rightarrow$ 
% $ (\adap(\tr) \leq \nnatA) ~ (a) 
% \land (\forall x \in \mbox{Vars}. \ddep{x}(\tr) \leq \dmap(x)) ~ (b)
% \land (\stepiA - j, \valr) \in \lrv{\tbool} ~ (c)$\\
% %
% By E-TRUE, let $\valr = \etrue$, $\tr = \etrue$ and $j = \size{\etrue} = 0$.\\
% %
% The following 3 items are proved:\\
% %
% 1. $\adap(\tr) = 0 \leq \nnatA$.\\
% 2. $\forall x \in \mbox{Vars}. \ddep{x}(\etrue) = 0 \leq \dmap(x)$\\
% 3. $(\stepiA - j, \valr) = (\stepiA, \etrue) \in \lrv{\tbool}$.\\



% \noindent {\bf case}
% \[
%     \inferrule*[right = FALSE]{
%     }{
%       \Gamma; \dmap \tvdash{\nnatA} \efalse: \tbool
%     }
% \]
% Assume $ (\stepiA, \env) \in \lrv{\tbool} $, TS: $(\stepiA, (\env, \efalse)) \in \lre{\dmap}{\nnatA}{\tbool}$.\\
% %
% By inversion, STS: $\forall \valr, \tr, j. (\env, \efalse \bigstep \valr, \tr) \land (\size{\tr} = j) \land (j \leq \stepiA) $\\
% $\Rightarrow$ 
% $ (\adap(\tr) \leq \nnatA) ~ (a) 
% \land (\forall x \in  \mbox{Vars}. \ddep{x}(\tr) \leq \dmap(x)) ~ (b)
% \land (\stepiA - j, \valr) \in \lrv{\tbool} ~ (c)$\\
% %
% By E-TRUE, let $\valr = \efalse$, $\tr = \efalse$ and $j = \size{\efalse} = 0$.\\
% %
% The following 3 items are proved:\\
% %
% 1. $\adap(\tr) = 0 \leq \nnatA$.\\
% 2. $\forall x \in \mbox{Vars}. \ddep{x}(\efalse) = 0 \leq \dmap(x)$\\
% 3. $(\stepiA - j, \valr) = (\stepiA, \efalse) \in \lrv{\tbool}$.\\


% \noindent {\bf case}
% \[
%     \inferrule*[right = PRIMITIVE]{
%       \Gamma; \dmap \tvdash{\nnatA} \expr: \tbase ~ (\star) \\
%       \nnatA' = 1 + \nnatA \\
%       \dmap' = 1 + \dmap
%     }{
%       \Gamma; \dmap' \tvdash{\nnatA'} \eop(\expr): \tbool
%     }
% \]
% Assume $(\stepiA, \env) \in \lrv{\Gamma}$, TS: $(\stepiA, (\eop(\expr), \env)) \in \lre{\dmap'}{\nnatA'}{\tbool}$.\\
% %
% Unfold, $\forall \valr, \tr, j. (\env, \eop(\expr) \bigstep \valr, \tr) \land (\size{\tr} = j) \land (j \leq \stepiA)$.\\
% %
% STS: $(\adap(\tr) \leq \nnatA') \land (\forall x \in \mbox{Vars}. \ddep{x}(\tr) \leq \dmap'(x)) \land ((\stepiA - j, \valr) \in \lrv{\tbool})$.\\
% %
% By induction hypothesis on $\star$, we get: $(\stepiA-1, (\expr, \env)) \in \lre{\dmap}{\nnatA}{\tbase} ~ (1)$.\\
% %
% Inversion on $(1)$, pick $\forall \valr', \tr', j'$. s.t. $ (\env, \expr \bigstep \valr', \tr') \land (\size{\tr'} = j') \land (j' \leq \stepiA-1)$.\\
% %
% We know: $(\adap(\tr') \leq \nnatA) ~ (a)
% \land (\forall x \in \ddep{x}(\tr') \leq \dmap(x)) ~ (b)
% \land ((\stepiA -1- j', \valr') \in \lrv{\tbase}) ~ (c)$.\\
% %
% Pick $\valr = \eop(\valr')$, then by E-PRIMITIVE, we have $\env, \eop(\expr) \bigstep \valr, \eop(\tr')$.\\
% %
% Pick $\tr = \eop(\tr'), j = j' + 1$ s.t.  $j \leq k$, STS:\\
% %
% 1. $\adap(\tr) = \adap(\eop(\tr')) = 1 + \adap(\tr') \leq 1 + \nnatA = \nnatA'$ proved by $(a)$\\
% %
% 2. $\forall x \in \mbox{Vars}. \ddep{x}(\tr) = \ddep{x}(\eop(\tr')) = 1 + \ddep{x}(\tr') \leq 1 + \dmap(x) = \dmap'(x)$ proved by $(b)$\\
% %
% 3. $(\stepiA - j, \valr) = (\stepiA - j' - 1, \eop(\valr')) \in \lrv{\tbool} $.\\
% %
% By typing rule PRIMITIVE, we have $\eop(\valr'): \tbool$. $\eop(\valr')$ is either $\etrue$ or $\efalse$. So, we have $(\stepiA - j' - 1, \eop(\valr')) \in \lrv{\tbool}$.\\
% %



% \noindent {\bf case}
% \[    
% 	\inferrule*[right = VAR]{
%       \Gamma(x) = \type \\ 0 \leq \dmap(x) \mbox{ or equiv.\ } \dmap(x) \neq \bot
%     }{
%       \Gamma; \dmap \tvdash{\nnatA} x: \type
%     }
% \]
% Assume $(\stepiA, \env) \in \lrv{\Gamma}$, TS: $(\stepiA, (\env, x)) \in \lre{\dmap}{\nnatA}{\type}$.\\
% %
% By inversion, STS: $\forall \valr, \tr, j. (\env, x \bigstep \valr, \tr) \land (\size{\tr} = j) \land (j \leq \stepiA) $\\
% $\Rightarrow$ 
% $ (\adap(\tr) \leq \nnatA) ~ (a) 
% \land (\forall x \in \mbox{Vars}. \ddep{x}(\tr) \leq \dmap(x)) ~ (b)
% \land (\stepiA - j, \valr) \in \lrv{\type} ~ (c)$\\
% %
% By E-VAR, pick $\valr = \env(x)$, $\tr = x$, $j = \size{x} = 0$, following 3 items are proved:\\
% %
% 1. $(\adap(\tr) = \adap(x) = 0 \leq \nnatA)$\\
% 2. $\forall y \in \ddep{y}(x)$, \\
% $~~ y = x ~~ \ddep{y}(x) = 0 \leq \dmap(x)$\\
% $~~ y \neq x ~~ \ddep{y}(x) = \bot \leq \dmap(y)$\\
% 3. $(\stepiA - j, \valr) = (\stepiA, \env(x)) \in \lrv{\type}$.\\
% $~~$By definition of $(\stepiA, \env) \in \lrv{\Gamma}$, we have: $(\stepiA, \env(x))\in \lrv{\Gamma(x)}$, and $\Gamma(x) = \type$. 


% \noindent {\bf case}
% \[    
%  \inferrule{
%      \dmap \wf{\type} \\
%     }{
%       \Gamma; \dmap \tvdash{\nnatA} \enil: \tlist{\type}
%     }
% \]
% Assume $(\stepiA, \env) \in \lrv{\Gamma}$, TS: $(\stepiA, (\env, \enil)) \in \lre{\dmap}{\nnatA}{\tlist{\type}}$.\\
% %
% By inversion, STS: $\forall \valr, \tr, j. (\env, \enil \bigstep \valr, \tr) \land (\size{\tr} = j) \land (j \leq \stepiA) $\\
% $\Rightarrow$ 
% $ (\adap(\tr) \leq \nnatA) ~ (a) 
% \land (\forall x \in \mbox{Vars}. \ddep{x}(\tr) \leq \dmap(x)) ~ (b)
% \land (\stepiA - j, \valr) \in \lrv{\type} ~ (c)$\\
% %
% By E-NIL, we know : $ v = \enil $ and $ \tr = \trnil$ and $
% \size{\enil} = 0 \leq k$. \\
% STS: 
% %
% 1. $(\adap(\tr) = \adap(\enil) = 0 \leq \nnatA)$, because $\nnatA$ is
% not negative.\\
% 2. $\forall x \in \mbox{Vars}. \ddep{x}(\enil)$, $  \ddep{x}(\enil) = \bot \leq
% \dmap(x)$ proved from the definition of $\bot$. \\
% 3. $(\stepiA - 0, \enil) =  \in \lrv{ \tlist{ \type} }$, proved by
% inversion of the interpretation of $\lrv{ \tlist{ \type} } $.\\

% \noindent{\bf case}
% \[
%   \inferrule{
%    \Gamma; \dmap_1 \tvdash{\nnatA_1} \expr_1 : \type ~(\star) \\
%    \Gamma; \dmap_2 \tvdash{\nnatA_2} \expr_2 : \tlist{\type} ~(\diamond)\\
%    \dmap' = \max(\dmap_1, \dmap_2) \\
%    \nnatA' = \max ( \nnatA_1, \nnatA_2 )
%    }
%    { \Gamma; \dmap' \tvdash{\nnatA'} \econs(\expr_1, \expr_2) :
%      \tlist{\type}  }
% \]

% Assume $(\stepiA, \env) \in \lrv{\Gamma}$, TS: $(\stepiA, (\env,
% \econs(\expr_1 , \expr_2 ) )) \in \lre{\dmap'}{\nnatA'}{\tlist{\type}}$.\\
% %
% By inversion, STS: $\forall \valr, \tr, j. (\env, \econs(e_1,e_2) \bigstep \valr, \tr) \land (\size{\tr} = j) \land (j \leq \stepiA) $\\
% $\Rightarrow$ 
% $ (\adap(\tr) \leq \nnatA')  
% \land (\forall x \in \mbox{Vars}. \ddep{x}(\tr) \leq \dmap'(x)) 
% \land (\stepiA - j, \valr) \in \lrv{\tlist{\type}} $
% %

%   By ih on $(\star)$, we get: $(\stepiA, (\env,
% \expr_1  )) \in \lre{\dmap_1}{\nnatA_1}{\type}~(1)$.\\
% %
% By inversion, we pick $v_1$, $\tr_1$ and $j_1$ s.t. $\env, \expr_1
% \bigstep \valr_1, \tr_1$ and $\size{\tr_1} = j_1$ and $j_1 \leq \stepiA$.\\
% %
% We know : $ (\adap(\tr_1) \leq \nnatA_1) ~ (2) 
% \land (\forall x \in \mbox{Vars}. \ddep{x}(\tr_1) \leq \dmap_1(x)) ~ (3)
% \land (\stepiA - j_1, \valr_1) \in \lrv{\type} ~ (4)$.

%  By ih on $(\diamond)$ and $(\stepiA-j_1, \env) \in \lrv{\Gamma}$ by Lemma~\ref{lem:downward}, we get: $(\stepiA-j_1, (\env,
% \expr_2  )) \in \lre{\dmap_2}{\nnatA_2}{ \tlist{ \type} }~(5)$.\\
% %
% By inversion, we pick $v_2$, $\tr_2$ and $j_2$,  s.t. $\env, \expr_2
% \bigstep \valr_2, \tr_2$ and $\size{\tr_2} = j_2$ and $j_2 \leq
% \stepiA-j_1$.\\
% %
% We know : $ (\adap(\tr_2) \leq \nnatA_2) ~ (6) 
% \land (\forall x \in \mbox{Vars}. \ddep{x}(\tr_2) \leq \dmap_2(x)) ~ (7)
% \land (\stepiA - j_2, \valr_2) \in \lrv{\tlist{\type}} ~ (8)$.
% %
% \[
% \inferrule{
% \env, \expr_1 \bigstep \valr_1, \tr_1 \\
% \env, \expr_2 \bigstep \valr_2, \tr_2
% }
% { \env, \econs (\expr_1, \expr_2)  \bigstep \econs (\valr_1, \valr_2),
%   \trcons(\tr_1, \tr_2)
% }
% \]
% %
% By E-Cons, we set : $\valr = \econs(\valr_1, \valr_2)$ and $\tr =
% \trcons(\tr_1, \tr_2)$ s.t. $j = \size{\tr}= j_1 + j_2 \leq k$.\\
% STS: 
% %
% 1. $(\adap(\tr) = \adap(\trcons(\tr_1, \tr_2)) \leq
% \nnatA')$, by $(2)$ and $(6)$.\\
% 2. $\forall x \in \mbox{Vars}. \ddep{x}(\trcons(\tr_1, \tr_2))
% = \max ( \ddep{x}(\tr_1), \ddep{x}(\tr_2) ) \leq
% \dmap'(x)$ proved by $(3)$ and $(7)$. \\
% 3. $(\stepiA - j_1 -j_2, \econs(\valr_1,\valr_2)  )  \in \lrv{ \tlist{
%     \type} }$, by inversion of the definition of
% $\lrv{\tlist{\type}}$, suffice to show: $(\stepiA-j_1-j_2, \valr_1) \in
%  \lrv{\type}$ proved by Lemma~\ref{lem:downward} on $(4)$, $(\stepiA-j_1-j_2, \valr_2) \in
%  \lrv{ \tlist{\type}} $ proved by $(8)$.\\


% \noindent{\bf case}
% \[
%  \inferrule{
%      \Gamma; \dmap_1 \tvdash{\nnatA_1} \expr_1 : \type_1~(\star) \\
%      \Gamma, x:\type_1 ; \dmap_2[x:q] \tvdash{\nnatA_2} \expr_2 :
%      \type_2 ~(\diamond)\\
%      \dmap' = \max( \dmap_2, \dmap_1 + q ) \\
%      \nnatA' = \max ( \nnatA_2, \nnatA_1 + q )
%    }
%    {  \Gamma; \dmap' \tvdash{\nnatA'}  \elet x = \expr_1 \ein \expr_2 : \type}
% \]
% Assume $(\stepiA, \env) \in \lrv{\Gamma}$, TS: $(\stepiA, (\env,
% \elet x = \expr_1 \ein \expr_2 )) \in \lre{\dmap'}{\nnatA'}{\type}$.\\
% %
% By inversion, STS: $\forall \valr, \tr, j. (\env, \elet x = \expr_1 \ein \expr_2 \bigstep \valr, \tr) \land (\size{\tr} = j) \land (j \leq \stepiA) $\\
% $\Rightarrow$ 
% $ (\adap(\tr) \leq \nnatA')  
% \land (\forall x \in \mbox{Vars}. \ddep{x}(\tr) \leq \dmap'(x)) 
% \land (\stepiA - j, \valr) \in \lrv{\type} $
% %

% By ih on $(\star)$, we get: $(\stepiA, (\env,
% \expr_1  )) \in \lre{\dmap_1}{\nnatA_1}{\type_1}~(1)$.\\
% %
% By inversion, we pick $v_1$, $\tr_1$ and $j_1$ s.t. $\env, \expr_1
% \bigstep \valr_1, \tr_1$ and $\size{\tr_1} = j_1$ and $j_1 \leq \stepiA$.\\
% %
% We know : $ (\adap(\tr_1) \leq \nnatA_1) ~ (2) 
% \land (\forall x \in \mbox{Vars}. \ddep{x}(\tr_1) \leq \dmap_1(x)) ~ (3)
% \land (\stepiA - j_1, \valr_1) \in \lrv{\type_1} ~ (4)$.
% %
% From the assumption  $(\stepiA, \env) \in \lrv{\Gamma}$, by
% Lemma~\ref{lem:downward}, we get : $(\stepiA-j_1, \env) \in
% \lrv{\Gamma}~(5)$. \\
% Set $\env' = \env[x \mapsto \valr_1]$, we know $(\stepiA-j_1, \env')
% \in \lrv{\Gamma,x:\type_1} ~(6) $ by $(4)$ and $(5)$.
% %

% By ih on $(\diamond)$ and $(6)$, we know: $(\stepiA-j_1, (\env',
% \expr_2  )) \in \lre{\dmap_2[x:q]}{\nnatA_2}{\type} ~(7) $.\\
% %
% By inversion $(7)$, we pick $v_2$, $\tr_2$ and $j_2$ s.t. $\env', \expr_2
% \bigstep \valr_2, \tr_2$ and $\size{\tr_2} = j_2$ and $j_2 \leq
% \stepiA - j_1$.\\
% %
% We know:  $ (\adap(\tr_2) \leq \nnatA_2) ~ (8) 
% \land (\forall x \in \mbox{Vars}. \ddep{x}(\tr_2) \leq \dmap_2[x:q](x)) ~ (9)
% \land (\stepiA - j_1 - j_2, \valr_2) \in \lrv{\type} ~ (10)$.
% %
% \[
% \inferrule{
%   \env, \expr_1 \bigstep \valr_1, \tr_1 \\
%   \env[x \mapsto \valr_1] , \expr_2 \bigstep \valr_2, \tr_2
% }
% {\env, \elet x;q = \expr_1 \ein \expr_2 \bigstep \valr_2, \trlet (x,
%   \tr_1, \tr_2) }
% \]
% By E-LET, we know : $\valr = c/valr_2$ and $\tr = \trlet (x,
%   \tr_1, \tr_2)  $. s.t. $j = j_1 +j_2 \leq k$.\\
% %
% STS: 
% %
% 1. $(\adap(\tr) = \adap(\trlet (x,
%   \tr_1, \tr_2) ) \leq
% \nnatA')$, proved by $(2)$ ,$(8)$and $(9)$.\\
% 2. $\forall y \in \mbox{Vars}. \ddep{y}(\trlet (x,
%   \tr_1, \tr_2)  ) \leq
% \dmap'(y)$ proved by $(3)$ and $(9)$. \\
% 3. $(\stepiA - j_1 -j_2, \valr_2 )  \in \lrv{ 
%     \type }$, proved by $(10)$.\\

% \end{proof}

%%%%%%%%%%%%%%%%%%%%%%%%%%%%%%%%%%%%%%%%%%%%%%%%%%%%%%%%%%%%%%%%%%%%%%%%%%%%%%%%%%%%%%%%%%%%%%%%%%%%%%%%%%
%END the Proof of old Fundamental Theorem 
%%%%%%%%%%%%%%%%%%%%%%%%%%%%%%%%%%%%%%%%%%%%%%%%%%%%%%%%%%%%%%%%%%%%%%%%%%%%%%%%%%%%%%%%%%%%%%%%%%%%%%%%%%


\clearpage

\section{Examples}

\begin{algorithm}
\caption{A two-round analyst strategy for random data (Algorithm 4 in ...)}
\label{alg:BitGOF}
\begin{algorithmic}
\REQUIRE Mechanism $\mathcal{M}$ with a hidden state $X\in \{-1,+1\}^{n\times (k+1)}$.
\STATE  {\bf for}\ $j\in [k]$\ {\bf do}.  
\STATE \qquad {\bf define} $q_j(x)=x(j)\cdot x(k)$ where $x\in \{-1,+1\}^{k+1}$.
\STATE \qquad {\bf let} $a_j=\mathcal{M}(q_j)$ 
\STATE \qquad \COMMENT{In the line above, $\mathcal{M}$ computes approx. the exp. value  of $q_j$ over $X$. So, $a_j\in [-1,+1]$.}
\STATE {\bf define} $q_{k+1}(x)=\mathrm{sign}\big (\sum_{i\in [k]} x(i)\times\ln\frac{1+a_i}{1-a_i} \big )$ where $x\in \{-1,+1\}^{k+1}$.
\STATE\COMMENT{In the line above,  $\mathrm{sign}(y)=\left \{ \begin{array}{lr} +1 & \mathrm{if}\ y\geq 0\\ -1 &\mathrm{otherwise} \end{array} \right . $.}
\STATE {\bf let} $a_{k+1}=\mathcal{M}(q_{k+1})$
\STATE\COMMENT{In the line above,  $\mathcal{M}$ computes approx. the exp. value  of $q_{k+1}$ over $X$. So, $a_{k+1}\in [-1,+1]$.}
\RETURN $a_{k+1}$.
\ENSURE $a_{k+1}\in [-1,+1]$
\end{algorithmic}
\end{algorithm}





\newpage

% \[\begin{array}{llll}
% \mbox{Term} & \term & ::= & \return ~ \expr ~|~ \bernoulli ~ \expr ~|~ \uniform ~ \expr
% \end{array}\]

%  \[
% \begin{array}{llll}
%   \mbox{Type} & \type & ::= & \treal ~|~ \tmonad ~ \type
% \end{array}
% \]

% \begin{figure}
%   \begin{mathpar}
%     \inferrule{
%       \Gamma \tvdash{\nnatA} \expr: \type
%     }{
%       \Gamma \tvdash{\nnatA} \return ~ \expr: \tmonad(\type)
%     }
%     %
%     \and
%     %
%     \inferrule{
%       \Gamma \tvdash{\nnatA} \expr: \type
%     }{
%       \Gamma \tvdash{\nnatA} \bernoulli ~ \expr: \tmonad(\type)
%     }
%     %
%     \and
%     %
%     \inferrule{
%       \Gamma \tvdash{\nnatA} \expr: \type
%     }{
%       \Gamma \tvdash{\nnatA} \uniform ~ \expr: \tmonad(\type)
%     }
%     % %
%     % \and
%     % %
%     % \inferrule{
%     % }{
%     %   \Gamma; \dmap_1 \tvdash{\nnatA} \expr_1: \tmonad(\type_1)
%     % }
%     % %
%     % \and
%     % %
%     % \inferrule{
%     % }{
%     %   \Gamma, x: \type_1; \dmap_2 \tvdash{\nnatA} \expr_2: \tmonad(\type_2)
%     % }
%   \end{mathpar}
%   \caption{typing rules - monad}
%   \label{fig:type-rules-monad}
% \end{figure}

% \begin{figure}
% \begin{mathpar}
%   \inferrule{\expr \bigstep \valr}{\return ~ \expr \bigstep^{1} \valr}
%   %
%   \and
%   %
%   \inferrule{\expr \bigstep \valr}{\bernoulli ~ \expr \bigstep^{\frac{1}{2}} \etrue}
%   %
%   \and
%   %
%   \inferrule{\expr \bigstep \valr}{\uniform ~ \expr \bigstep^{\frac{1}{2}} \valr}
% \end{mathpar}
%   \caption{Big-step semantics - monad}
%   \label{fig:big-step-monad}
% \end{figure}

% \begin{figure}
%   \begin{mathpar}
%     \begin{array}{lll}
%       \lr{\type_1 \times \type_2} & = & \lr{\type_1}  \times \lr{\type_2} \\
%       %
%       \lr{\tmonad ~ \type} & = & \tmonad \lr{\type} =  \{ d : \lr{\type} \rightarrow [0,1] \} \\
%       %
%       \lr{\tlist{\type}} & = & \{ \amalg_{\nnatA \in \nat} \lr{\type} \times \dots \times \lr{\type} ~|~ \sum_{x \in \lr{\type}} d(x) = 1 \}\\
%       %
%       \lr{\tarr{\type_1}{\type_2}{\nnatbiA}{\dmap}{\nnatA}} & = &
%       \{(f, \nnatbiA, \dmap, \nnatA ) ~|~ f: \type_1 \rightarrow \type_2, ~ \nnatbiA \in \natbi, \nnatA \in \natbi, \dmap ~ given\} \\
%       %
%       \lr{\Gamma, \dmap \tvdash{\nnatA} \expr: \type} & = & \{ \lr{\Gamma \tvdash{\nnatA} \expr: \type}, \dmap, \nnatA
%       \}
%     \end{array}
%   \end{mathpar}
%   \caption{Logical relation - monad}
%   \label{fig:lr:monad}
% \end{figure}
%%%%%%%%%%%%%%%%%%%%%%%%%%%%%%%%%%%%%%%%%%%%%%%%%%%%

%%%%%%EXAMPLES---TWO ROUNDS%%%%%%%%%%%%%%%%%%%%%%%%%%%%%%%%%%%%%%%%%%%%%%
\newpage
\begin{figure}

Two-rounds:

\[
\begin{array}{l}
 \elet \eapp  g = \efix \eapp  f(j). \lambda k.  \\
 \hspace{.2cm} \eif \big (  (j < k)  ,  \\
  \hspace{.8cm} \elet \eapp  a = \eop \eapp  \boxed{( \lambda x. (x \eapp j)\cdot (x \eapp k) )}  \eapp  \ein \\
  \hspace{1.2cm} (a, j) :: (f  \eapp (j+1) \eapp  k) \\
 \hspace{1.2cm} , \eapp  [] \big) \\
  \hspace{.2cm} \ein \\
  \elet \eapp  l = g \eapp  0\eapp  K \eapp  \ein \\
  \elet \eapp   q =  \lambda x. \mathsf{sign} \eapp  (\mathsf{foldl} \eapp  (\lambda acc. \lambda (a,i). \big(acc\eapp + (x \eapp  i) *lg(\frac{1+a}{1-a})  \big) \eapp  0 \eapp  l )) \eapp  \ein \\
  \eop ( q )
\end{array}
\]
\end{figure}


\begin{tabbing}
    $ x: \trow \equiv \tarr{\tbase}{\treal}{0}{[]}{0}$ \\
    $ \cdot: \tbase * \tbase \to \tbase $   \\
    $ g: \tarr{ \tbase }{ \tarr{\tbase}{\tlist{\treal * \tbase}}{1}{[j:1]}{1} }{0}{[]}{0}  $\\
    $ q: \tbox{  (\tarr{ \trow }{ \treal }{0}{[l:0]}{0})     } $\\
    $ \eop : \tbox{  (\tarr{ \trow }{ \treal }{0}{\dmap''}{0})     } \to \treal $\\
    $\mathsf{foldL} : (\treal \to \tlist{(\treal * \tbase)} \to \treal) \to \treal \to \tlist{\tbase * \treal} \to \treal = FL$
\end{tabbing}

\begin{figure}
Type derivation:\\
    $A = \tarr{ \tbase }{ \tarr{\tbase}{\tlist{\treal * \tbase}}{0}{[j:0]}{1} }{\bot}{[]}{0} $, \\
    $FL1 = \treal \to \tlist{\tbase * \treal} \to \treal$, $FL2 = \treal \to \tlist{\treal * \tbase} \to \treal$\\
    $Q_2 = \tbox{  (\tarr{ \trow }{ \treal }{0}{[l:0]}{0})} $,\\
    $Q_1 = \tbox{  (\tarr{ \trow }{ \treal }{0}{[j:0, k:0]}{0})}$\\
    $\Gamma = f: A,  j: \tbase, k: \tbase$; $\Gamma' = \Gamma, a: \treal$; $\Gamma'' = \Gamma, x: \trow$,\\
    $\Delta = g: A, l: \tlist{\treal * \tbase}$; $\Delta' = \Delta, q:Q_2$; $\Delta'' = \Delta, x: \trow$\\

\[
  \inferrule*[ right = LET-B ]
   {
     \inferrule
     {
        \Pi_L \vartriangleright
     }
     {
        \tvdash{1} \efix \eapp  f(j). \cdots : A
      }
     \and
     \inferrule
     {
        \Pi_R \vartriangleright
     }
     {
      g: A, [g : 1] \tvdash{2} \elet l = \cdots \ein \elet q = \cdots \ein \eop ( q ) : \treal
    }
     \\
     n' = \max(n_1 + 1, n_2 ) = 2
     \and
     [] = \max([] + 1, [])
   }
   { \tvdash{2} \elet g = \cdots \ein \elet l = \cdots \ein \cdots :  \treal }
\]

Derivation $\Pi_L$ and $\Pi_R$ are shown as follows:\\
$\Pi_L$:
\begin{mathpar}
      \inferrule*[right = FIX]
      {
        \inferrule*[right = FIX]
        {
          \inferrule*[right = IF]
          {
            \inferrule*[right = BOOL]
            {
              \empty
            }
            {
              \Gamma; \dmap_1' = [j:0, k:0] \tvdash{0} {j<k : \tbool}
            }
            \\
            \inferrule*[right = LET]
            {
              \dots
            }
            {
              \Gamma; \dmap_2' \tvdash{1} \elet a = \eop ( \cdots ) \ein (a, j) :: \cdots : \tlist{\treal * \tbase}
            }
            \\
            \inferrule*[right = NIL]
            {
              \empty
            }
            {
              \Gamma; \dmap_2'=[f: \infty, j: 0, k: 0] \tvdash{1} {[] : \tlist{\treal * \tbase}}
            }
            \\
            \nnatA' = \nnatA_0 + \nnatA_1 = 0 + 1 = 1
            \and
            \dmap_1' = \max(\dmap_2' + 0 , \dmap_1') = [f: \infty, j: 0, k: 0]
          }
          {
            \Gamma ;  [f: \infty, j: 0, k: 0] 
            \tvdash{1} \eif \cdots : \tlist{\treal * \tbase}
          }
        }
        {
          f: A, j : \tbase; [f: \infty, j: \bot ] \tvdash{1} {\lambda k. \eif \cdots : \tarr{\tbase}{\tlist{\treal * \tbase}}{0}{[j:0]}{1} }
        }
      }
      {
        \tvdash{1} \efix \eapp  f(j). \cdots : A 
      }    
    
    \inferrule*[right = LET]
    {
    \inferrule
        {
         \inferrule
            {
                \Pi_{L1} \vartriangleright
            }
            {
                \Gamma; \dmap_4' \tvdash{0} \lambda x. (x \eapp j)\cdot (x \eapp k) : Q_1
            \\\\
            \dmap_4' = [j: \bot, k : \bot]
            }
        }
        {
            \Gamma; \dmap_3' \tvdash{1} \eop ( \lambda x. (x \eapp j)\cdot (x \eapp k)) : \treal
        }
    \and 
    \inferrule
        {
        \inferrule
            {
              \inferrule*[right = VAR]
              {
                \empty
              }
              {
                \Gamma'; [a:0] \tvdash{0} a: \treal
              }
              \and
              \inferrule*[right = VAR]
              {
                \empty
              }
              {
                \Gamma'; [j:0] \tvdash{0} j : \tbase
              }
            }
            {
                \Gamma'; [a:0, j:0] \tvdash{0} (a, j) : \treal * \tbase
            }
            \\
        \inferrule*[right = APP]
            {
                \Pi_{L2} \vartriangleright
            }
            {
                \Gamma'; \dmap_5' \tvdash{1} f \eapp  j+1 \eapp  k : \tlist{\treal * \tbase}
            }
            \\
            \nnatA' = \max(\nnatA_1, \nnatA_2) = 1
            \and
            \dmap_5' = [f: \infty, a: 0, j : 0, k : 0]
        }
        {
            \Gamma'; \dmap_2'[a: 0] \tvdash{1} 
            (a, j) ::(\cdots) : \tlist{\treal * \tbase}
        }
      \\
      \dmap_3' = [j: \bot, k: \bot]
      \and
      \nnatA' = \max(\nnatA_2, \nnatA_1 + q) = 1
      \and
      \dmap_2' = \max(\dmap_3' + 1, \dmap_2') = [j:0, k:0, f:\infty]
    }
    {
      \Gamma; \dmap_2'  \tvdash{1} 
      \elet a = \eop (\lambda x. (x \eapp j)\cdot (x \eapp k)) \ein (a, j) ::(f \eapp  j+1 \eapp  k): \tlist{\treal * \tbase}
    }
    \end{mathpar}
\end{figure}


\begin{figure}
$\Pi_{L2}$:
\begin{mathpar}

    \inferrule*[right = APP]
    {
      \inferrule*[right = APP]
      {
      \inferrule*[right = VAR]
      {
      \empty
      }
      {
      \Gamma'; [f:\infty] \tvdash{0} f: A
      }
      \and
       \inferrule*[right = VAR]
      {
      \empty
      }
      {
      \Gamma'; [j:0] \tvdash{0} j + 1: \tbase
      }
      \\
      \nnatA' = \max(\nnatA_2 + q, \nnatA) + \nnatA_1 = 0 
      }
      {
      \Gamma'; \dmap_6' \tvdash{0} f \eapp j+1 : \tarr{\tbase}{\tlist{\treal * \tbase}}{0}{\dmap = [j:0]}{1}
      }
      \and
      \inferrule*[right = VAR]
      {
      \empty
      }
      {
      \Gamma'; \dmap_7' \tvdash{0} k : \tbase
      }
      \\
      \dmap_6' = [j:0, f: \infty, a : 0]
      \and
      \dmap_5' = \max(\dmap_6', 0 + \max(\dmap, \dmap_7' + 0))
      \and
      \dmap_7' = [k:0]
      \\\\
      \nnatA' = \nnatA_1 + \max(\nnatA, \nnatA_2 + q) = 1
    }
    {
    \Gamma'; \dmap_5' = [f: \infty, a: 0, j : 0, k : 0] \tvdash{1} f \eapp  j+1 \eapp  k : \tlist{\treal * \tbase}
    }
    \end{mathpar}
\end{figure}



\begin{figure}
$\Pi_{L1}$:
\begin{mathpar}
\inferrule
{
  \inferrule
  {
    \inferrule
    {
      \inferrule
      {
        \inferrule
        {
        \empty
        }
        {
          \Gamma''; [x:0] \tvdash{0} x: \trow
        }
        \and
        \inferrule
        {
          \empty
        }
        {
          \Gamma''; [j:0] \tvdash{0} j : \tbase
        }
      }
      {
        \Gamma''; [x: 0, j:0] \tvdash{0} (x \eapp j) : \treal
      }
      \and
       \inferrule
      {
        \inferrule
        {
        \empty
        }
        {
          \Gamma''; [x:0] \tvdash{0} x: \trow
        }
        \and
        \inferrule
        {
          \empty
        }
        {
          \Gamma''; [k:0] \tvdash{0} k : \tbase
        }
      }
      {
        \Gamma''; [x : 0, k : 0] \tvdash{0} (x \eapp k) : \treal
      }
    }
    {    
      \Gamma''; [x:0, j : 0, k: 0] \tvdash{0} (x \eapp j) \cdot (x \eapp k) : \treal
    }  
  }
  {
    \Gamma; \dmap_4' \tvdash{0} \lambda x. (x \eapp j)\cdot (x \eapp k) : \tarr{\trow}{\treal}{0}{[j:0,k:0]}{0}
  }
}
{
  \Gamma; \dmap_4' \tvdash{0} \lambda x. (x \eapp j)\cdot (x \eapp k) : Q_1
}
\end{mathpar}
\end{figure}


\begin{figure}
$\Pi_R$:
\begin{mathpar}
\inferrule*[right = LET]
    {
    \inferrule
    {
      \inferrule
      {
        \inferrule
        {
          \empty
        }
        {
            g: A ; [g: 0] \tvdash{0} {g : A }
        }
        \and
        \inferrule
        {
          \empty
        }
        {
          g: A; \tvdash{0} 0 : \tbase
        }
      }
      {
         g: A; [g: 0] \tvdash{0} {g \eapp  0  : \tarr{\tbase}{ \tlist{\treal * \tbase}}{1}{[j:1]}{1} }
      }
      \and
      \inferrule
      {
      \empty
      }
      {
        g: A; \tvdash{0} K : \tbase
      }
    }
    {
        g: A; [g:0] \tvdash{1} { g \eapp  0 \eapp  K : \tlist{\treal * \tbase}}
    }
    \and
    \cdots
    \\
    \nnatA' = \max(\nnatA_1 + q, \nnatA_2) = 2
    \and
    [g:1] = \max(([g:0] + 1), [g:1])
    }
    {
    g: A; [g:1] \tvdash{2} \elet l = \cdots \ein \elet q = \cdots \ein \eop ( q ) : \treal   
    }

    \inferrule*[right = LET]
        {
        \inferrule
            {
            \inferrule
            {
              \inferrule
                  {
                    \cdots
                  }
                  {
                      \Delta''; [x: 0,l: 0] \tvdash{0} {\mathsf{sign}(\dots) :  \treal}
                  }          
            }
            {
                \Delta; \dmap_3''\tvdash{0} {\lambda x. \dots : \tarr{\trow}{\treal}{0}{[l:0]}{0} }
            }
            }
            {
                \Delta; \dmap_3'' = [l:\bot]\tvdash{0} {\lambda x. \dots : Q_1}
            }
        \and
        \inferrule
          {
          \inferrule
              {
                  \empty
              }
              {
                  \Delta'; [g:0, l:0, q: 0] \tvdash{0} { q : Q_2}
              }
          }
          {
                  \Delta'; \dmap_2'' = \dmap_1''[q:1] \tvdash{1} {\eop (q) : \treal}
          }
          \\
          \nnatA' = \max(\nnatA_1 + q, \nnatA_2) = 1
          \\
          \dmap_1'' = \max([l:\bot] + 1, \dmap_2'')
        }
        {
            \Delta; \dmap_1'' = [g:1, l:1]  \tvdash{1} \elet q = \cdots \ein \cdots : \treal
        }

\inferrule
{
  \inferrule
  {
    \empty
  }
  {
    \Delta, x: \trow;  \tvdash{0} \mathsf{sign} :  \treal \to \treal
  }
  \and
  \inferrule
  {
    \inferrule
    {
    \cdots
    }
    {
      \Delta''; [x:0] \tvdash{0} \mathsf{foldl}(\cdots) \eapp 0 : \tlist{\tbase * \treal} \to \treal
    }
    \\
    \inferrule
    {
      \empty
    }
    {
      \Delta''; [l:0] \tvdash{0} l : \tlist{\tbase * \treal}
    }
  }
  {
    \Delta''; \dmap_4'' \tvdash{0} \mathsf{foldl}(\cdots) \eapp 0 \eapp l : \treal 
  }
}
{
  \Delta''; \dmap_4'' = [x: 0, l:0] \tvdash{0} \mathsf{sign}(\dots) :  \treal
}

\inferrule
{
  \inferrule
  {
    \inferrule
    {
      \empty
    }
    {
      \Delta'';  \tvdash{0} \mathsf{foldl} : FL
    }
    \and
    \inferrule
    {
      \cdots
    }
    {
      \Delta''; [x : 0] \tvdash{0} \lambda acc. \cdots : FL2
    }
  }
  {
    \Delta''; [x : 0] \tvdash{0} \mathsf{foldl}(\cdots) :FL1 
  }
  \and
  \inferrule
  {
    \empty
  }
  {
    \Delta''; \tvdash{0} 0 : \tbase
  }
}
{
  \Delta''; [x : 0] \tvdash{0} \mathsf{foldl}(\cdots) \eapp 0 : \tlist{\tbase * \treal} \to \treal
}

\inferrule
{
  \inferrule
  {
    \inferrule
    {
      \inferrule
      {\empty
      }
      {
        \Delta_2''; \dmap_6'' \tvdash{0} acc: \treal
        \\\\
        \dmap_6'' = [acc:0]
      }
      \and
      \inferrule
      {
        \inferrule
        {
          \inferrule
          {
          \empty
          }
          {
            \Delta_2''; [x:0] \tvdash{0} x : \trow
          }
          \\
          \inferrule
          {
            \empty
          }
          {
            \Delta_2''; [i:0] \tvdash{0} i : \tbase
          }
        }
        {
          \Delta_2''; [x:0, i:0] \tvdash{0} x \eapp i: \treal
        }
        \and
        \inferrule
        {
          \empty
        }
        {
          \Delta_2''; [a:0] \tvdash{0} \mathsf{lg}(\cdot): \treal
        }
      }
      {
        \Delta_2''; [x:0, (a, i):0] \tvdash{0} (x\eapp i) * \cdots: \treal
      }
    }
    {
      \Delta_2''; \dmap_5''[(a,i):0] \tvdash{0} acc + \cdots : \treal
      \\
      \Delta_2'' = \Delta_1'', (a,i): \tlist{\treal * \tbase}
    }
  }
  {
    \Delta_1''; \dmap_5'' = [x : 0, acc : 0] \tvdash{0} \lambda (a,i). \cdots :  \to \treal
    \and
    \Delta_1'' = \Delta'', acc: \treal
  }
}
{
  \Delta''; [x : 0] \tvdash{0} \lambda acc. \cdots : FL2
}
\end{mathpar}
\vspace{-0.5cm}
\caption{Type Derivation of Two Round Example}
\end{figure}


\clearpage

\begin{algorithm}
\footnotesize
\caption{A multi-round analyst strategy for random data (Algorithm 5 in ...)}
\label{alg:multiRound}
\begin{algorithmic}
\REQUIRE Mechanism $\mathcal{M}$ with a hidden state $X\in [N]^{n}$ sampled u.a.r., control set size $c$
\STATE Define control dataset $C = \{0,1, \cdots, c - 1\}$
\STATE Initialize $Nscore(i) = 0$ for $i \in [N]$, $I = \emptyset$ and $Cscore(C(i)) = 0$ for $i \in [c]$
\STATE  {\bf for}\ $j\in [k]$\ {\bf do} 
\STATE \qquad {\bf let} $p=\uniform(0,1)$ 
\STATE \qquad {\bf define} $q (x) = \bernoulli ( p )$ .
\STATE \qquad {\bf define} $qc (x) = \bernoulli ( p )$ .
\STATE \qquad {\bf let} $a = \mathcal{M}(q)$ 
\STATE \qquad {\bf for}\ $i \in [N]$\ {\bf do}
\STATE \qquad \qquad $Nscore(i) = Nscore(i) + (a - p)*(q (i) - p)$ if $i \notin I$
\STATE \qquad {\bf for}\ $i \in [c]$\ {\bf do}
\STATE \qquad \qquad $Cscore(C(i)) = Cscore(C(i)) + (a - p)*(qc (i) - p)$
\STATE \qquad {\bf let} $I = \{i | i\in [N] \land Nscore(i) > \max(Cscore)\}$
\STATE \qquad {\bf let} $X = X \setminus I$ 
\RETURN $X$.
% \ENSURE 
\end{algorithmic}
\end{algorithm}


\clearpage

\begin{figure}


Multi-rounds:
\[
\begin{array}{l}
 \efix \eapp  \mathsf{multiRound}(\_). \Lambda k. \Lambda j. \lambda k. \lambda j. \lambda sc. \lambda scc. \lambda
  I.   \lambda N. \lambda C. \lambda D.\\
 \hspace{.2cm} \eif   \big (   (j < k)  ,  \\
  \hspace{.2cm} \elet \eapp p = \uniform \eapp 0 \eapp 1 \ein \\
  \hspace{0.4cm} \elet \eapp q = \lambda x. \bernoulli \eapp p \ein \\
 \hspace{0.4cm} \elet \eapp qc = \lambda c. \bernoulli \eapp p \ein \\
 \hspace{0.4cm} \elet \eapp a = \eop (q)  \ein \\
 \hspace{0.8cm} \elet \eapp sc' =  \mathsf{updtSC} \eapp () \eapp sc  \eapp a \eapp p
 \eapp q \eapp I \eapp  \eapp 0 \eapp  N
  \eapp  \ein \\
\hspace{0.8cm} \elet \eapp scc' =  \mathsf{updtSCC} \eapp () \eapp scc \eapp a \eapp p
 \eapp qc \eapp  \eapp 0 \eapp  C \ein \\
\hspace{0.8cm} \elet \eapp maxScc =  \mathsf{foldl} \eapp (\lambda acc. \lambda a. \eif ( acc < a, a, acc)) \eapp 0 \eapp scc' \ein \\
\hspace{0.8cm} \elet \eapp I' =  \mathsf{updtI}  \eapp () \eapp maxScc \eapp sc
  \eapp 0 \eapp N  \ein \\
  \hspace{0.8cm} \elet \eapp D' =  D \setminus I' \ein \\
  \hspace{1.2cm} \mathsf{multiRound} () [k] [j+1] \eapp  k \eapp (j+1)  \eapp sc' \eapp scc' \eapp I'
  \eapp N \eapp C \eapp D'\\ 
\hspace{0.2cm}   ,     D  \big ) \\
 
\end{array}
\]

UpdtSC
\[
\begin{array}{l}
 \mathsf{updtSC} = \efix \eapp  \mathsf{f}(\_). \lambda sc. \lambda a. \lambda
  p. \lambda q.  \lambda I. \lambda i. \lambda N. \\
 \hspace{.2cm} \eif   \Big (   (i < N)  ,  \\
 \hspace{0.4cm}  \eif \big( ( i \eapp \mathsf{in} I  ) ,       \\
 \hspace{0.8cm} \elet \eapp x =( \mathsf{nth} \eapp sc \eapp i) + (a-p)*(q
  \eapp i - p)  \ein \\
 \hspace{0.8cm} \elet \eapp sc' =  \mathsf{updt} \eapp sc \eapp i
  \eapp x \ein \\
  \hspace{1.2cm} \mathsf{f}  \eapp () \eapp sc' \eapp a \eapp p
 \eapp q \eapp I \eapp  \eapp (i+1) \eapp  N  \\ 
\hspace{0.4cm}  ,  \mathsf{f}  \eapp () \eapp sc \eapp a \eapp p
 \eapp q \eapp I \eapp  \eapp (i+1) \eapp  N \big )  \\ 
\hspace{0.2cm}   ,  sc  \Big ) \\
 
\end{array}
\]
UpdtSCC
\[
\begin{array}{l}
 \mathsf{updtSCC} = \efix \eapp  \mathsf{f}(\_). \lambda scc. \lambda a. \lambda
  p. \lambda qc.  \lambda i. \lambda C. \\
 \hspace{.2cm} \eif   \Big (   (i < C)  ,  \\
 \hspace{0.8cm} \elet \eapp x =( \mathsf{nth} \eapp scc \eapp i) + (a-p)*(qc
  \eapp i - p)  \ein \\
 \hspace{0.8cm} \elet \eapp scc' =  \mathsf{updt} \eapp scc \eapp i
  \eapp x \ein \\
  \hspace{1.2cm} \mathsf{f}  \eapp () \eapp scc' \eapp a \eapp p
 \eapp qc   \eapp (i+1) \eapp  C  \\ 
\hspace{0.2cm}   ,  scc  \Big ) \\
\end{array}
\]
UpdtI
\[
\begin{array}{l}
 \mathsf{updtI} = \efix \eapp  \mathsf{f}(\_). \lambda maxScc. \lambda sc. \lambda
  i. \lambda N. \\
 \hspace{.2cm} \eif   \Big (   (i < N)  ,  \\
\hspace{0.4cm}  \eif \big( ( ( \mathsf{nth} \eapp scc \eapp i)  >  maxScc  ) ,       \\
 \hspace{0.8cm}   i :: ( \mathsf{f}  \eapp () \eapp maxScc \eapp sc
  \eapp (i+1) \eapp N  )\\
 \hspace{0.8cm} , \mathsf{f}  \eapp () \eapp maxScc \eapp sc
  \eapp (i+1) \eapp N  \big )  \\
\hspace{0.2cm}   ,  [] \Big ) \\
\end{array}
\]
\end{figure}




\begin{figure}
UpdtSC:
\begin{mathpar}
      \inferrule*[right = FIX]
      {
        \inferrule*[right = FIX...]
        {
          \inferrule*[right = IF]
          {
            \inferrule*[right = BOOL]
            {
              \empty
            }
            {
             i:\tint , I : \tlist{\tint} , \Gamma \tvdash{0} {i
               <  N : \tbool}
            }
            \\
            \inferrule*[right = IF]
            {
              \dots
            }
            {
              f:., \Gamma; \tvdash{0}  \eif (i \in I)\cdots : \tlist{\treal }
            }
            \and
            \inferrule*[right = VAR]
            {
            }
            {
              f:., sc:. ,\Gamma \tvdash{0} {sc : \tlist{\treal}}
            }
            \\
            \and
          }
          {
          f: ., sc: \tlist{\treal}, a:\tint, i:\tint \dots \Gamma
            \tvdash{0} \eif (i < N)  \cdots : \tlist{\treal}
          }
        }
        {
          f: \tunit \rightarrow \dots
          \tlist{\treal}, \Gamma \tvdash{0} {\lambda
            sc. \dots \lambda N.
            \eif \cdots :  \tlist{\treal} \rightarrow \dots \tlist{\treal}   }
        }
      }
      {
       \Gamma \tvdash{0} \efix \eapp  f( \_ ). \lambda sc. \lambda
        a. \dots \lambda N. \eif \Big ( (i <N), \dots, sc \Big ) : \tunit
        \rightarrow \tlist{\treal} \rightarrow \dots \rightarrow \tlist{\treal} 
      }

   \inferrule*[ right = IF ]
   {
     \inferrule
     {
     }
     {
       \Gamma \tvdash{0}. f \eapp () \eapp sc' \dots \eapp N : \tlist{\treal}
      }
     \and
     \inferrule
     {
     \dots
     }
     {
      \Gamma\tvdash{0}  \elet x = \dots \ein \elet sc' = \dots
      \ein f \eapp () \eapp sc' \dots : \tlist{\treal}
    }
     \\
     i :\tint , I : \tlist{\tint},\Gamma \tvdash{0} i \in I : \tbool
   }
   { \Gamma \tvdash{0}  \eif \big(  (i \in I), \elet x = \dots  ,  f \eapp ()
     \eapp sc \dots N \big) : \tlist{\treal }    }

    
    \inferrule*[right = LET]
    {
    \inferrule*[right = LET]
        {
        \inferrule*[right = APP]
            {
            }
            {
                \Gamma \tvdash{0} \mathsf{updt} \eapp sc \eapp
                i \eapp x : \tlist{\treal}
            }
            \and
        \inferrule*[right = APP]
            {
                \dots
            }
            {
                \Gamma \tvdash{0} f () \eapp sc \dots \eapp  (i+1) \eapp  N : \tlist{\treal }
            }
            \\
            {  }
            \and
            {}
        }
        {
            x: \treal ,\Gamma \tvdash{0} 
             \elet sc' = \mathsf{updt} \eapp sc \eapp i \eapp x \ein \dots : \tlist{\treal}
        }
      \\
      \inferrule
        {
          \dots
        }
        {
            \Gamma \tvdash{0} ( \mathsf{nth} \eapp sc \eapp i  )+ (a-p)*(q \eapp i -
      p) : \treal
        }
    }
    {
      \tvdash{0} 
      \elet x =( \mathsf{nth} \eapp sc \eapp i  )+ (a-p)*(q \eapp i -
      p) \ein \dots: \tlist{\treal }
    }
    \end{mathpar}
\end{figure}


Type derivation of Multi-rounds example:
\[
\begin{array}{ll} 
  \mathsf{multiRound} :& \tunit \to \forall  J,K.
                          \tint[K] \to \tint[J] \to \tlist{\treal} \to \tlist{\treal} \to \tlist{\tint} \\
                        & \to \tint \to \tint \to \tarr{\tlist{\tint}}{\tlist{\tint}}{0}{\dmap}{K - J} \\
  \mathsf{updtSC} : & \tunit \to \tlist{\treal} \to \treal \to \treal \to \tbox{ \tint \to \tint } \\
                    & \to \tlist{\tint} \to \tint[J] \to \tarr{\tint}{\tlist{\treal}}{0}{[\_ : 0]}{0}\\
  \mathsf{updtSCC} : & \tunit \to \tlist{\treal} \to \treal \to \treal \to \tbox{ \tint \to \tint } \\
                    & \to \tint[J] \to \tarr{\tint}{\tlist{\treal}}{0}{[\_ : 0]}{0}\\
  \mathsf{updtI} : & \tunit \to \treal \to \tlist{\treal} \to \tint[J] \to \tarr{\tint}{\tlist{\tint}}{0}{[\_ : 0]}{0}\\
  \mathsf{foldl} : & (\treal \to \treal \to \treal) \to \treal \to \tlist{\treal}\\
  q :              & \tbox{\tint \to \tint}\\
  qc :             & \tbox{\tint \to \tint}\\
\end{array}
\]
$\to$ without annotations is equivalent to $\tarr{}{}{\bot}{\dmap_{\bot}}{0}$ in types of $\mathsf{multiRound}, \mathsf{updtSC}, \mathsf{updtSCC}$ and $ \mathsf{updtI} $ functions, where $\forall x, \dmap_{\bot}(x) = \bot$.

\begin{figure}
Multi-rounds:
\begin{mathpar}
\inferrule*[right = FIX]
{
  \inferrule
  {
    \inferrule*[right = FIX]
    {
      \inferrule*[right = IF]
      {
        \inferrule
        {
          \empty
        }
        {
          \Delta; \Gamma; \dmap_1' = [j : 0, k : 0] \tvdash{0} j<k : \tbool
        }
        \\
        \inferrule
        {
          \cdots
        }
        {
          \Delta; \Gamma; \dmap_1 \tvdash{K - J} \elet p = \cdots \ein \cdots : \tlist{\tint}
        }
        \and
        \inferrule
        {
          \empty
        }
        {
          \Delta; \Gamma; \dmap_1 \tvdash{0} D : \tlist{\tint}
        }
      }
      {
        \Delta; \Gamma_0', sc: \tlist{\treal}, \cdots; \dmap_0 
        \tvdash{K - J} \eif(j<k, \elet p = \cdots \ein \cdots, D) : \tlist{\tint}   
      }
    }
    {
    \Delta; \Gamma_0'; [\mathsf{multiRound}: \infty, () : \bot] \tvdash{0} \lambda k. \cdots : \tint[K] \cdots  
    }
  }
  {
  \Delta; \Gamma_0 , \mathsf{multiRound} : \tunit \to \cdots; [\mathsf{multiRound}: \infty, () : \bot]
  \tvdash{0} \Lambda. \Lambda. \cdots : \forall. \forall. \cdots  
  }
}
{
  \Delta; \Gamma_0; [] \tvdash{0} \efix \, \mathsf{multiRound ()}. \cdots : \tunit \to \cdots
}

\inferrule*[right = LET]
{
  \inferrule
  {
    \inferrule
    {
    \empty
    }
    {
    \tvdash{0} 0: \treal    
    }
    \and
    \inferrule
    {
      \empty
    }
    {
      \tvdash{0} 1: \treal
    }
  }
  {  
    \Gamma; [] \tvdash{0} \uniform \eapp 0 \eapp 1 : \treal
  }  
  \and
  \inferrule
  {
    \cdots
  }
  {
    \Delta; \Gamma, p : \treal; \dmap_2 \tvdash{K - J} \elet q = \lambda x. \cdots \ein \cdots : \tlist{\tint}
  }
}
{
  \Delta; \Gamma; \dmap_1 \tvdash{K - J} \elet p = \cdots \ein \cdots : \tlist{\tint}
}

\inferrule*[right = LET]
{
  \inferrule
  {
    \inferrule
    {
      \inferrule
      {
        \inferrule
        {
          \empty
        }
        {
          \tvdash{0} p : \treal        
        }
      }
      {
        \Gamma_{1}, x: \tint; [x: 0] \tvdash{0} \bernoulli \eapp p : \tint
      }
    }
    { 
      \Gamma_{1}; [] \tvdash{0} \lambda x. \bernoulli \eapp p : \tint \to \tint
    }
  }
  {
    \Gamma_{1}; [] \tvdash{0} \lambda x. \bernoulli \eapp p : \tbox{\cdots}
  }
  \and
  \inferrule
  {
    \inferrule
    {
      \inferrule
      {
        \inferrule
        {
          \inferrule
          {
            \empty
          }
          {
            \Gamma_{3}; [q:0]  \tvdash{0} q : \tbox{\tint \to \tint}
          }
        }
        { 
          \Gamma_{3}; [q:1] \tvdash{1} \eop (q): \treal
        }
        \and
        \Pi_1  \vartriangleright
      }
      {
        \Delta; \Gamma_{3}; \dmap_4 \tvdash{K - J} \elet a = \eop ( q ) \ein \cdots : \tlist{\tint}
      }
    }
    {
      \cdots
    }
  }
  {
    \Delta; \Gamma_1, q: \tbox{\cdots}; \dmap_3 \tvdash{K - J} \elet qc = \lambda x. \cdots : \tlist{\tint}
  }
}
{
  \Delta; \Gamma_{1}; \dmap_2 \tvdash{K - J} \elet q = \lambda x. \bernoulli \eapp p \ein \cdots : \tlist{\tint}
}
\end{mathpar}

$\Pi_1:$
\begin{mathpar}
\inferrule*[right = LET]
{ 
  \inferrule
  {
    \inferrule
    {
      \inferrule
      {
        \empty
      }
      {
        \Gamma_4;\dmap_6 \tvdash{0} \mathsf{updtSC} : \cdots
      }
      \\
      \inferrule
      {
        \empty
      }
      {
        \Gamma_4; \dmap_6 \tvdash{0} () : \tunit
      }
    }
    {
      \cdots
    }
  }
  {
    \Gamma_4; \dmap_6 \tvdash{0} \mathsf{updtSC} \eapp () \eapp sc \cdots : \tlist{\treal}
  }
  \and
  \inferrule
  {
    \inferrule
    {
      \Pi_2 \vartriangleright
    }
    {
      \Delta; \Gamma_7; \dmap_7 \tvdash{K - J - 1} \elet D' = \cdots \ein \cdots : \tlist{\tint}
    }
  }
  {
    \cdots
  }
}
{
  \Delta; \Gamma_3, a: \treal; \dmap_5 \tvdash{K - J - 1} \elet sc' = \mathsf{updtSC} \cdots \ein \cdots : \tlist{\tint}
}
\end{mathpar}


$\Pi_2$:
\begin{mathpar}
\inferrule
  {
    \inferrule
    {
      \inferrule
      {
        \empty
      }
      {
        \Gamma_7 \tvdash{0} D: \tlist{\tint}
      }
      \and
      \inferrule
      {
        \empty
      }
      {
        \Gamma_7 \tvdash{0} I': \tlist{\tint}
      }
    }
    {
      \Gamma_7; \dmap_8 \tvdash{0} D \setminus I' : \tlist{\tint}
    }
    \and
    \inferrule
    {
      \cdots
    }
    {
      \Delta; \Gamma_8; \dmap_8 \tvdash{K - J - 1} \mathsf{multiRound} \cdots : \tlist{\tint}
      \\\\
      \Gamma_8 = \Gamma_7, D' : \tlist{\tint}
    }
  }
  {
    \Delta; \Gamma_7; \dmap_7 \tvdash{K - J - 1} \elet D' = \cdots \ein \cdots : \tlist{\tint}
  }
\end{mathpar}
\end{figure}

\begin{figure}
\begin{mathpar}
    \inferrule
{
  \inferrule
  {  
    \inferrule
    {
      \inferrule
      {
        \Pi_3 \vartriangleright
      }
      {
        \Delta; \Gamma_8; \dmap_8 
        \tvdash{0} \mathsf{multiRound} \eapp () \eapp [ ] [ ] : \tint[k] \to \tint[J + 1] \to \cdots
      }
    }
    {
      \cdots
    }
  }
  {
    \Delta; \Gamma_8; \dmap_8 
    \tvdash{0} \mathsf{multiRound} \eapp () \eapp [ ] [] \cdot : \tarr{\cdot}{\cdot}{0}{\dmap}{K - (J + 1)}
  }
  \and
  \inferrule
  {
  \empty
  }
  {
    \Delta; \Gamma_8; [D':0] \tvdash{0} D' : \tlist{\tint}
  }
}
{
  \Delta; \Gamma_7, D' : \tlist{\tint}; \dmap_8 
  \tvdash{K - J - 1} \mathsf{multiRound} \eapp () \cdots : \tlist{\tint}
}
\end{mathpar}

$\Pi_3$:
\begin{mathpar}
\inferrule
{
  \inferrule
  {
    \inferrule
    {
          \inferrule
      {
        \empty
      }
      {
        \Delta; \Gamma_8; \dmap_8 \tvdash{0} \mathsf{multiRound} : \cdots
      }
      \and
      \inferrule
      {
        \empty
      }
      {
        \Delta; \Gamma_8; \dmap_8 \tvdash{0} () : \tunit
      }
    }
    {
      \Delta; \Gamma_8; \dmap_8 
      \tvdash{0} \mathsf{multiRound} \eapp (): \eilam  \eilam  \tint[K] \to \tarr{\cdots}{\cdots}{0}{\dmap}{K - J}
    }
    \and
    \inferrule
    {
      \Delta(K) = S
    }
    {
      \Delta \tvdash{} K :: S
    }
  }
  {
    \Delta; \Gamma_8; \dmap_8 
    \tvdash{0} \mathsf{multiRound} \eapp () \eapp [] : \eilam \tint[K] \to \tint[J] \to \tarr{\cdots}{\cdots}{0}{\dmap}{K - J}
  }
  \and
  \inferrule
  {
    \Delta(J) = S
  }
  {
    \Delta \tvdash{} J :: S
  }
}
{
  \Delta; \Gamma_8; \dmap_8 
  \tvdash{0} \mathsf{multiRound} \eapp () \eapp [ ] [ ] : \tint[K] \to \tint[J + 1] \to \cdots \tarr{\cdots}{\cdots}{0}{\dmap}{K - (J + 1)}
}

\end{mathpar}

\end{figure}


The \emph{depth map} in the type derivation of Multi-rounds example:
\[
\begin{array}{lll}
\dmap     & = & [sc : 0, scc : 0, I : 0, j : 0, k : 0, N : 0, C: 0, D : 0] \\
\dmap_0   & = & [\mathsf{multiRound}: \infty, () : \bot, sc : 0, scc : 0, I : 0, j : 0, k : 0, N : 0, C: 0, D : 0] \\
\dmap_1   & = & [\mathsf{multiRound}: \infty, () : \bot, sc : 0, scc : 0, I : 0, j : 0, k : 0, N : 0, C: 0, D : 0] \\
\dmap_2   & = & \dmap_1[p : K - J]\\
\dmap_3   & = & \dmap_2[q : K - J]\\
\dmap_4   & = & \dmap_3[qc : 0]\\
\dmap_5   & = & \dmap_4[a : K - J - 1]\\
\dmap_6   & = & \dmap_5\\
\dmap_7   & = & \dmap_5[sc' : 0, scc' : 0, I' ; 0]\\
\dmap_8   & = & \dmap_7[D' : 0]\\
\end{array}
\]

$\Delta = [J : S, K : S]$


\newpage
\bibliographystyle{plain}
\bibliography{adaptivity.bib}

\end{document}



