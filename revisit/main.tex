\documentclass[a4paper,11pt]{article}

\usepackage{mathpartir}
\usepackage{amsmath,amsthm,amsfonts}
\usepackage{ amssymb }
\usepackage{color}
\usepackage{algorithm}
\usepackage{algorithmic}
\usepackage{microtype}


%%% Attempt 1: Linear 1



\newcommand{\diam}{{\color{red}\diamond}}
\newcommand{\dagg}{{\color{blue}\dagger}}
\let\oldstar\star
\renewcommand{\star}{\oldstar}

\newcommand{\im}[1]{\ensuremath{#1}}

\newcommand{\kw}[1]{\im{\mathtt{#1}}}


\newcommand{\set}[1]{\im{\{{#1}\}}}

\newcommand{\mmax}{\ensuremath{\mathsf{max}}}

%%%%%%%%%%%%%%%%%%%%%%%%%%%%%%%%%%%%%%%%%%%%%%%%%%%%%%%%
% Comments
\newcommand{\omitthis}[1]{}

% Misc.
\newcommand{\etal}{\textit{et al.}}
\newcommand{\bump}{\hspace{3.5pt}}

% Text fonts
\newcommand{\tbf}[1]{\textbf{#1}}
%\newcommand{\trm}[1]{\textrm{#1}}

% Math fonts
\newcommand{\mbb}[1]{\mathbb{#1}}
\newcommand{\mbf}[1]{\mathbf{#1}}
\newcommand{\mrm}[1]{\mathrm{#1}}
\newcommand{\mtt}[1]{\mathtt{#1}}
\newcommand{\mcal}[1]{\mathcal{#1}}
\newcommand{\mfrak}[1]{\mathfrak{#1}}
\newcommand{\msf}[1]{\mathsf{#1}}
\newcommand{\mscr}[1]{\mathscr{#1}}









\newcommand{\defeq}{\mathrel{\doteq}}
\newcommand{\conj}{\mathrel{\wedge}}
\newcommand{\disj}{\mathrel{\vee}}

\newcommand{\lzero}{0}


% context
\newcommand{\tctx}{\Gamma}
\newcommand{\ictx}{ }


% expression
\newcommand{\expr}{e}
\newcommand{\aexpr}{a}
\newcommand{\bexpr}{b}
\newcommand{\sexpr}{\textrm{e} }
\newcommand{\qexpr}{\psi}
\newcommand{\qval}{\alpha}
\newcommand{\query}{{\tt query}}
\newcommand{\saexpr}{\textrm{a} }
\newcommand{\sbexpr}{\textrm{b} }
\newcommand{\vall}{w}
\newcommand{\valr}{v}
\newcommand{\eif}{\kw{if}}
\newcommand{\eapp}{\;}
\newcommand{\eprojl}{\kw{fst}}
\newcommand{\eprojr}{\kw{snd}}
\newcommand{\eifvar}{\kw{ifvar}}
%expression and commands for WHILE language
\newcommand{\ewhile}{\kw{while}}
\newcommand{\bop}{*}
\newcommand{\uop}{\circ}
\newcommand{\eskip}{\kw{skip}}

\newcommand{\eloop}{\kw{loop}}
\newcommand{\edo}{\kw{do}}
\newcommand{\qdom}{\mathcal{QD}}

%configuration
\newcommand{\config}[1]{\langle #1 \rangle}
\newcommand{\ematch}{\kw{match}}
\newcommand{\clabel}[1]{\left[ #1 \right]}


%\newcommand{\eprov}[1]{\eta_{#1}}
\newcommand{\etrue}{\kw{true}}
\newcommand{\efalse}{\kw{false}}
\newcommand{\econst}{c}
\newcommand{\eop}{\delta}
\newcommand{\efix}{\mathop{\kw{fix}}}
\newcommand{\elet}{\mathop{\kw{let}}}
\newcommand{\ein}{\mathop{ \kw{in}} }
\newcommand{\eas}{\mathop{ \kw{as}} }
\newcommand{\enil}{\kw{nil}}
\newcommand{\econs}{\mathop{\kw{cons}}}
%\newcommand{\labelA}{\ell}
%monad expressions / terms
\newcommand{\term}{t}
\newcommand{\return}{\kw{return}}
\newcommand{\bernoulli}{\kw{bernoulli}}
\newcommand{\uniform}{\kw{uniform}}
 \newcommand{\epack}{\mbox{pack\;}}
\newcommand{\eunpack}{\mbox{unpack\;}}
\newcommand{\eilam}{\Lambda.}

\newcommand{\evec}{\kw{dict}}
\newcommand{\eget}{\kw{get}}

% trace
\newcommand{\triapp}[2]{\kw{IApp}(#1,#2)}
\newcommand{\trow}{\text{row}}
\newcommand{\tr}{T}
\newcommand{\trift}{\eif^{\kw{t}}}
\newcommand{\triff}{\eif^{\kw{f}}}
\newcommand{\trprojl}{\eprojl}
\newcommand{\trprojr}{\eprojr}
\newcommand{\trtrue}{\etrue}
\newcommand{\trfalse}{\efalse}
\newcommand{\trconst}{\econst}
\newcommand{\trop}{\eop}
\newcommand{\trfix}{\efix}
\newcommand{\trapp}[5]{#1 \; #2 \mathrel{\triangleright} {\efix
#3(#4).#5}}
\newcommand{\trnil}{\enil}
\newcommand{\trcons}{\econs}
\newcommand{\trlet}{\elet}
%types for monad
\newcommand{\treal}{\kw{real}}
\newcommand{\tint}{\kw{int}}
\newcommand{\tmonad}{\kw{M}}
\newcommand{\tunit}{\kw{unit}}
\newcommand{\tdb}{\kw{tdb}}

% adaptivity
\newcommand{\adap}{\kw{adap}}
\newcommand{\ddep}[1]{\kw{depth}_{#1}}
\newcommand{\nat}{\mathbb{N}}
\newcommand{\natb}{\nat_{\bot}}
\newcommand{\natbi}{\natb^\infty}
\newcommand{\nnatA}{Z}
\newcommand{\nnatB}{m}
\newcommand{\nnatbA}{s}
\newcommand{\nnatbB}{t}
\newcommand{\nnatbiA}{q}
\newcommand{\nnatbiB}{r}

%type
\newcommand{\type}{\tau}
\newcommand{\tbase}{\kw{b}}
\newcommand{\tbool}{\kw{bool}}
\newcommand{\tbox}[1]{ \kw{\square} \, (#1) }
\newcommand{\tarr}[5]{#1; #3 \xrightarrow{#4; \, #5} #2}
\newcommand{\tlist}[1]{\kw{list} \, #1 }
\newcommand{\env}{\theta}
\newcommand{\tforall}[3]{\forall#3 \overset{#1, #2}{::} S.\, }
\newcommand{\texists}[1]{\exists#1 {::} S.\, }
\newcommand{\lto}{\multimap}
\newcommand{\bang}[1]{ !_{#1}}
\newcommand{\whynot}[1]{ ?_{#1} }
\newcommand{\ltype}{A}
\newcommand{\adapt}{R}
% index
\newcommand{\idx}{I }
\newcommand{\smax}[2]{\kw{max}(#1,#2)}
\newcommand{\ienv}{\sigma}

%evaluation
\newcommand{\bigstep}[1]{\mathrel{\to^{#1}}}

\newcommand{\dmap}{\rho}
\newcommand{\dmapb}{\bot_\dmap}
\newcommand{\supp}{\kw{supp}}
\newcommand{\dom}{\kw{dom}}
\newcommand{\codom}{\kw{codom}}

\newcommand{\tvdash}[1]{\vdash_{#1}}

\newcommand{\lrv}[1]{[\![ #1 ]\!]_{\text{V}}}
\newcommand{\lre}[3]{[\![ #3 ]\!]_{\text{E}}^{#1, #2}}
\newcommand{\stepiA}{k}
\newcommand{\stepiB}{j}
\newcommand{\size}[1]{|#1|}

%logic relations
\newcommand{\lr}[1]{[\![ #1 ]\!]}
\newcommand{\lrt}[1]{\mathcal{T}[\![ #1 ]\!]}


\newcommand{\wf}[1]{\vdash #1 \, \kw{wf} }
\newcommand{\sub}[2]{ #1 \, <: \, #2 }
\newcommand{\eqv}[3]{ #1 \, \equiv \, #2 \Rightarrow \textcolor{red}
{#3}  }
\newcommand{\eqvt}[3]{ #1 \, \sqsubseteq \, #2 \Rightarrow \textcolor{red}
{#3}  }
\newcommand{\eqvc}[2]{ #1 \, \equiv^c \, #2   }


%core calculus
\newcommand{\ctyping}[3]{ \tvdash{ #1} {#2} :^c #3 }
\newcommand{\cbox}{\mathsf{box}}
\newcommand{\cder}{\mathsf{der}}
\newcommand{\elab}[4]{ \vdash_{ #1} #2 \rightsquigarrow #3 : #4}
\newcommand{\coerce}[2]{\mathsf{coerce}_{#1, #2}}

%algorithmic typing rules
\newcommand{\infr}[4]{{#1} ~ {\textcolor{red}\uparrow} ~ {\color{red} #2} \Rightarrow
{ } {\color{red} #3} }
\newcommand{\chec}[3]{{#1} ~ {\downarrow} ~ {#2} \Rightarrow {\color{red} #3} }
% \newcommand{\restriction}{\Phi}
\newcommand{\fresh}{ \mathsf{fresh}}
\newcommand{\red}[1]{ \textcolor{red} {#1} }
\newcommand{\fiv}[1]{ \mathsf{FIV} (#1)   }
\newcommand{\fv}[1]{ \mathsf{FV} (#1)   }

\newcommand{\todo}[1]{{\small \color{red}\textbf{[[ #1 ]]}}}
\newcommand{\todomath}[1]{{\scriptstyle \color{red}\mathbf{[[ #1 ]]}}}

\newcommand{\caseL}[1]{\item \textbf{#1}\newline}

\newcommand{\attr}{\mathsf{attr}}
\newcommand{\weight}{\mathsf{W}}
\newcommand{\num}{\mathsf{n}}

\usepackage{enumitem}
\setenumerate{listparindent=\parindent}

\newlist{enumih}{enumerate}{3}
\setlist[enumih]{label=\alph*),before=\raggedright, topsep=1ex, parsep=0pt,  itemsep=1pt }

\newlist{enumconc}{enumerate}{3}
\setlist[enumconc]{leftmargin=0.5cm, label*= \arabic*.  , topsep=1ex, parsep=0pt,  itemsep=3pt }


\newlist{enumsub}{enumerate}{3}
\setlist[enumsub]{ leftmargin=0.7cm, label*= \textbf{subcase} \bf \arabic*: }

\newlist{enumsubsub}{enumerate}{3}
\setlist[enumsubsub]{ leftmargin=0.5cm, label*= \textbf{subsubcase} \bf \arabic*: }

\newlist{mainitem}{itemize}{3}
\setlist[mainitem]{ leftmargin=0cm , label= {\bf Case} }

%%%%COLORS
\definecolor{periwinkle}{rgb}{0.8, 0.8, 1.0}
\definecolor{powderblue}{rgb}{0.69, 0.88, 0.9}
\definecolor{sandstorm}{rgb}{0.93, 0.84, 0.25}
\definecolor{trueblue}{rgb}{0.0, 0.45, 0.81}


\usepackage{array}

\newlength\Origarrayrulewidth

% horizontal rule equivalent to \cline but with 2pt width
\newcommand{\Cline}[1]{%
 \noalign{\global\setlength\Origarrayrulewidth{\arrayrulewidth}}%
 \noalign{\global\setlength\arrayrulewidth{2pt}}\cline{#1}%
 \noalign{\global\setlength\arrayrulewidth{\Origarrayrulewidth}}%
}

% draw a vertical rule of width 2pt on both sides of a cell
\newcommand\Thickvrule[1]{%
  \multicolumn{1}{!{\vrule width 2pt}c!{\vrule width 2pt}}{#1}%
}

% draw a vertical rule of width 2pt on the left side of a cell
\newcommand\Thickvrulel[1]{%
  \multicolumn{1}{!{\vrule width 2pt}c|}{#1}%
}

% draw a vertical rule of width 2pt on the right side of a cell
\newcommand\Thickvruler[1]{%
  \multicolumn{1}{|c!{\vrule width 2pt}}{#1}%
}

\newcommand{\command}{c}
\newcommand{\green}[1]{{ \color{green} #1 } }

\newcommand{\func}[2]{\mathsf{AD}(#1) \to (#2)}
\newcommand{\varEst}{\bf{VetxEst}}
\newcommand{\graphGen}{\bf{GraphGen}}

\newcommand{\ag}[2]{\mathsf{VetxEst}{(#1)}\to {(#2)}}
\newcommand{\ad}[2]{\mathsf{GraphGen}{(#1)}\to {(#2)}}
\newcommand{\rb}{\mathsf{RechBound}}
\newcommand{\pathsearch}{\mathsf{AdaptPathSearch}}

%Packages
\usepackage[T1]{fontenc}
\usepackage{fourier} 
\usepackage[english]{babel} 
\usepackage{amsmath,amsfonts,amsthm} 
\usepackage{lscape}
\usepackage{geometry}
\usepackage{amsmath}
\usepackage{algorithm}
\usepackage{algorithmic}
\usepackage{amssymb}
\usepackage{amsfonts}
\usepackage{times}
\usepackage{bm}
\usepackage{ stmaryrd }
\usepackage{ amssymb }
\usepackage{ textcomp }
\usepackage[normalem]{ulem}
% For derivation rules
\usepackage{mathpartir}
\usepackage{color}
\usepackage{a4wide}


\newcommand{\sctag}[1]{\tag{\textsc{#1}}\label{eq:#1}}
\newcommand{\ands}{~\wedge~}

%Variations
\newcommand{\astable}{\mathbb{S}}
\newcommand{\achange}{U}

%Index Terms
\newcommand{\scond}[3]{(\eif\im{#1}\ethen\im{#2}\eelse\im{#3})}
\newcommand{\keps}[2]{\epsilon(#1,#2)}
\newcommand{\spower}[2]{#1^{#2}}
\newcommand{\ssum}[4]{\sum\limits_{#1=#2}^{#3}#4}
\newcommand{\smin}[2]{\kw{min}({#1},{#2})}
\newcommand{\smax}[2]{\kw{max}(#1,#2)}
\newcommand{\smaxx}[3]{\kw{max}(#1,#2,#3)}
\newcommand{\smaxxx}[4]{\kw{max}(#1,#2,#3,#4)}

%Sizes
\newcommand{\szero}{0}
\newcommand{\sone}{1}
\newcommand{\splus}[2]{#1 + #2}
\newcommand{\ssucc}[1]{#1 {+} \sone}
\newcommand{\sminus}[2]{#1 - #2}
\newcommand{\sdiv}[2]{\frac{#1}{#2}}
\newcommand{\smult}[2]{#1\cdot#2}
\newcommand{\splusone}[1]{#1+1}
\newcommand{\sceil}[1]{\ceil*{#1}}
\newcommand{\sfloor}[1]{\floor*{#1}}
\newcommand{\size}[1]{|#1|}
\newcommand{\slog}[1]{\kw{log}_2(#1)}
\newcommand{\sinf}{\infty}

%Sorts
\newcommand{\ssize}{\mathbb{N}}
\newcommand{\svar}{\mathbb{V}}
\newcommand{\scost}{\mathbb{R}}
\newcommand{\sfun}[2]{#1\mbox{\ra} #2}
\newcommand{\sfunmon}[2]{#1\xrightarrow{\mbox{mon}} #2}
% \newcommand{\sort}{\varsigma}
\newcommand{\sort}{S}
\newcommand{\sorted}[1]{#1 \mathrel{::} \sort}
\newcommand{\sized}[1]{#1 \mathrel{::} \ssize}

% Types
\newcommand{\grt}{A}
\newcommand{\lbound}{\mathop{\uparrow}}
\newcommand{\tbool}{\mbox{bool}}
\newcommand{\trbool}{\mbox{bool}_r}
\newcommand{\tubool}{\mbox{bool}_u}
\newcommand{\trint}{\mbox{int}_r}
\newcommand{\tint}{\mbox{int}}
\newcommand{\tquery}{\mbox{query}}
\newcommand{\tunit}{\mbox{unit}}
\newcommand{\trunit}{\mbox{unit}_r}
\newcommand{\tlist}[3]{\mbox{list}[#1]^{#2}\,#3}
\newcommand{\tlists}[1]{ #1 \, \mbox{list} }
\newcommand{\ulist}[2]{\mbox{list}[#1]\,#2}
\newcommand{\tslist}[1]{\mbox{list}\,#1}
\newcommand{\ttree}[3]{\mbox{tree}[#1]^{#2}\,#3}
\newcommand{\utree}[2]{\mbox{tree}[#1]\,#2}
\newcommand{\tbase}{b}
\newcommand{\uarr}[2]{\mathrel{\xrightarrow[]{\wexec(#1,#2)}}} 
\newcommand{\uarrs}[1]{\mathrel{\xrightarrow[]{\mu(#1)}}} 
\newcommand{\uarrd}{\mathrel{\xrightarrow{\wdead}}}
\newcommand{\tarrd}[1]{\mathrel{\xrightarrow{\wdiff(#1)}}}



\newcommand{\tforall}[3]{\forall#1\overset{\wexec(#2,#3)}{::}S.\,}
\newcommand{\tforalld}[2]{\forall#1\overset{\wdiff(#2)}{::}S.\,}
\newcommand{\uforalls}[2]{\forall#1\overset{\mu(#2)}{::}S.\,}
\newcommand{\tforallS}[1]{\forall#1.\,}
\newcommand{\tsforall}[1]{\forall#1{::}S.\,}
\newcommand{\tforallN}[1]{\forall#1{::}\ssize.\,}
\newcommand{\texists}[1]{\exists#1{::}S.\,}
\newcommand{\texistsN}[1]{\exists#1{::}\ssize.\,}
\newcommand{\tcimpl}[2]{#1 \mathrel{\supset} #2}
\newcommand{\tcprod}[2]{#1 \mathrel{\&} #2}

\newcommand{\ttimes}{\mathrel{\times}}
\newcommand{\tsum}{\mathrel{+}}
\newcommand{\tinter}{\mathrel{\wedge}}
\newcommand{\tst}[1]{(#1)^{\astable}}
\newcommand{\tch}[2]{U\,(#1,#2)} 
\newcommand{\tchs}[1]{U\,(#1,#1)} 
\newcommand{\tcho}[1]{U\,#1} 
\newcommand{\tmu}[1]{(#1)^{\mu}}
\newcommand{\tno}[1]{(#1)^{\_}}
\newcommand{\trm}[2]{|#1|_{#2}}
\newcommand{\trmo}[1]{|#1|}
\newcommand{\tmup}[1]{(#1)^{\mu'}}
\newcommand{\tdual}[1]{d({#1})}
\newcommand{\tbox}[1]{\square\,#1}
\newcommand{\tdmu}[1]{#1^{\shortdownarrow {\mu}}}
\newcommand{\tmon}[1]{{\color{red}m(#1)}}
\newcommand{\tforce}[1]{#1^{\shortdownarrow \achange }}
\newcommand{\tlift}[2]{(#1,#2)^{\uparrow}}
\newcommand{\tpull}[1]{#1^{\nearrow}}
\newcommand{\tpushd}[1]{(#1)^{\downarrow\square}}

% Terms
\newcommand{\vbase}{r}
\newcommand{\vtrue}{\mbox{tt}}
\newcommand{\vfalse}{\mbox{ff}}

\newcommand{\la}{\langle} 
\newcommand{\ra}{\rangle}
\newcommand{\eapp}{\;} 
\newcommand{\eleft}{\pi_1}
\newcommand{\eright}{\pi_2} 
\newcommand{\econst}{\kw{n}}
\newcommand{\etrue}{\mbox{true}}
 \newcommand{\efalse}{\mbox{false}}
\newcommand{\eif}{\mbox{if\;}} 
\newcommand{\ethen}{\mbox{\;then\;}}
\newcommand{\eelse}{\mbox{\;else\;}} 
\newcommand{\einl}{\mbox{inl\;}}
\newcommand{\einr}{\mbox{inr\;}} 
\newcommand{\elet}{\mbox{let\;}}
\newcommand{\clet}{\mbox{clet}\;}
\newcommand{\ecimp}{\mbox{.}_c\;}
\newcommand{\eelimU}{\mbox{elim}_U\;} 
\newcommand{\ein}{\mbox{\;in\;}}
\newcommand{\ecase}{\mbox{\;case\;}} 
\newcommand{\eof}{\mbox{\;of\;}}
\newcommand{\eas}{\mbox{\;as\;}} 
\newcommand{\ecelim}{\mbox{celim\;}}
\newcommand{\enil}{\mbox{nil}} 
\newcommand{\epack}{\mbox{pack\;}}
\newcommand{\eunpack}{\mbox{unpack\;}}
\newcommand{\efix}{\mbox{fix\;}} 
\newcommand{\efixNC}{\mbox{fix$_{NC}$\;}} 
\newcommand{\eLam}{ \Lambda}
\newcommand{\elam}{ \lambda} 
\newcommand{\eApp}{ [\,]\,}
\newcommand{\eleaf}{\mbox{leaf}} 
\newcommand{\ewith}{\;\mbox{with}\;} 
\newcommand{\enode}{\mbox{node}}
\newcommand{\econs}{\mbox{cons}} 
\newcommand{\econsC}{\mbox{cons$_C$}} 
\newcommand{\econsNC}{\mbox{cons$_{NC}$}} 
\newcommand{\eunit}{()}
\newcommand{\eswitch}{\kw{switch}\;}
\newcommand{\enoch}{\kw{NC}\;}
\newcommand{\eder}{\kw{der}\;}
\newcommand{\esplit}{\kw{split}\;}
\newcommand{\ecoerce}[2]{\kw{coerce}_{#1,#2}\;}
\newcommand{\econtra}{\kw{contra}\;}

\newcommand{\ealloc}[2]{ \mathrel{ \mathsf{alloc}\, {#1} \, {#2} } }
\newcommand{\eallocB}[2]{ \mathrel{ \mathsf{alloc_b}\, {#1} \, {#2} } }
\newcommand{\eupdt}[3]{ \mathrel{ \mathsf{update} \ {#1} \ {#2} \ {#3} }  }
\newcommand{\ereadx}[2] { \mathrel{ \mathsf{read} \ {#1} \ {#2} }  }
\newcommand{\eupdtB}[3]{ \mathrel{ \mathsf{update_b} \ {#1} \ {#2} \ {#3} }  }
\newcommand{\ereadxB}[2] { \mathrel{ \mathsf{read_b} \ {#1} \ {#2} }  }
\newcommand{\eret}[1] {\mathrel{ \mathsf{return} \, {#1} }}
\newcommand{\eletx}[3]{  \mathrel{ \mathsf{let_m} \{ {#1} \} = {#2} \ \mathsf{in} \ {#3}  } }



\newcommand{\caseof}[1]{\mbox{case}~#1~\mbox{of}}
\newcommand{\tcaseof}[1]{\mathsf{case}~#1~\mathsf{of}}
\newcommand{\ofnil}[1]{~~\mbox{nil}~\to#1}
\newcommand{\ofzero}[1]{~~\kw{0}~\to#1}
\newcommand{\ofcons}[3]{|~#1::#2~\to~#3}

% Diff Rel
\newcommand{\udiff}{\gtrapprox}
\newcommand{\rdiff}{\ominus}
\newcommand{\rdiffs}{\lesssim}
\newcommand{\ldiff}{\lesssim}

% Evaluation
\newcommand{\red}[1]{\Downarrow^{#1}}

\newcommand{\wmax}{\mbox{\scriptsize max}}
\newcommand{\wmin}{\mbox{\scriptsize min}}
\newcommand{\wdiff}{\mbox{\scriptsize diff}}
\newcommand{\wexec}{\mbox{\scriptsize exec}}
\newcommand{\wdead}{\mbox{\scriptsize dead}}
%Logical relation
\newcommand{\step}{\text{Step index}}
\newcommand{\world}{\text{World}}
\newcommand{\values}{\text{Value}}
\newcommand{\expr}{\text{Expression}}
\newcommand{\ulr}[1]{\llbracket#1\rrbracket_{v}}
\newcommand{\ulrg}[1]{\llbracket#1\rrbracket_{\grt}}
\newcommand{\lr}[1]{\llparenthesis#1\rrparenthesis_{v}}
\newcommand{\lre}[2]{\llparenthesis#1\rrparenthesis_{\varepsilon}^{#2}}
\newcommand{\lrg}[1]{\llparenthesis#1\rrparenthesis_{\grt}}
\newcommand{\ulre}[3]{\llbracket#1\rrbracket_{\varepsilon}^{#2,#3}}
\newcommand{\ulrew}[1]{\llbracket#1\rrbracket_{\varepsilon}^{0,\sinf}}

\newcommand{\relwith}[2]{\{#1~|~#2\}}
\newcommand{\rel}[1]{\{#1\}}
\newcommand{\del}[1]{\mathcal{D}\llbracket#1\rrbracket}
\newcommand{\dd}[1]{\mathcal{D}\llbracket\Delta\rrbracket}
\newcommand{\ugsubst}[1]{\mathcal{G}\llbracket#1\rrbracket}
\newcommand{\gsubst}[1]{\mathcal{G}\llparenthesis#1\rrparenthesis}
\newcommand{\dsubst}[1]{\mathcal{D}\llbracket#1\rrbracket}
\newcommand{\s}{\sigma}
\newcommand{\peq}{\preceq}
\newcommand{\plt}{\prec}
\renewcommand{\d}{\delta}
\newcommand{\g}{\gamma}


% Typing judgments
\newcommand{\jiterm}[2]{\mathrel{\vdash {#1} :: #2}}
\newcommand{\jtype}[4]{\mathrel{\vdash_{#1}^{#2} {#3} : {#4}}}
\newcommand{\jtypeM}[4]{\mathrel{\vdash_{#1}^{#2} {#3} :^c {#4}}}

\newcommand{\jtypes}[3]{\mathrel{\vdash_{#1}^{\mu} {#2} : {#3}}}

\newcommand{\jstype}[3]{\mathrel{\vdash
    {#1} \backsim {#2} : {#3}}}


\newcommand{\jtypediff}[4]{\mathrel{\vdash% _{\wdiff}
    {#2} \ominus {#3} \ldiff #1 : {#4}}}

\newcommand{\jtypediffM}[4]{\mathrel{\vdash
    {#2} \ominus {#3} \ldiff #1 :^c {#4}}}
\newcommand{\jmintypesame}[3]{\mathrel{\vdash
    {#2} \ominus {#2} \ldiff #1 :^c {#3}}}

\newcommand{\jelab}[6]{\mathrel{\vdash
    {#2} \ominus {#3} \rightsquigarrow {#4} \ominus {#5} \ldiff #1 : {#6}}}
\newcommand{\jelabsame}[4]{\mathrel{\vdash
    {#2} \ominus {#3} \rightsquigarrow {#2} \ominus {#3} \ldiff #1 : {#4}}}

\newcommand{\jelabun}[5]{\mathrel{\vdash_{#1}^{#2}
    {#3} \rightsquigarrow {#4} : {#5}}}

\newcommand{\jelabc}[4]{\mathrel{\vdash
    {#2} \ominus {#3} \rightsquigarrow {#2}^* \ominus {#3}^* \ldiff #1 : {#4}}}


\newcommand{\jelabcu}[4]{\mathrel{\vdash_{#1}^{#2}
    {#3} \rightsquigarrow {#3}^* : {#4}}}

\newcommand{\jtypediffsym}[5]{\mathrel{\vdash
    #1 \ldiff {#3} \ominus {#4} \ldiff #2 : {#5}}}
\newcommand{\sty}[2]{\vdash#1 \mathrel{::} #2}


\newcommand{\rname}[1]{\mbox{\small{#1}}}

\newcommand{\vsem}[2]{\llbracket #1 \rrbracket_{V}^{#2}}
\newcommand{\esem}[2]{\llbracket #1 \rrbracket_{E}^{#2}}
\newcommand{\conj}{\mathrel{\wedge}}

\newcommand{\vusem}[1]{\llparenthesis #1 \rrparenthesis_{V}}
\newcommand{\eusem}[1]{\llparenthesis #1 \rrparenthesis_{E}}

\newcommand{\jsubtype}[2]{\sat#1\sqsubseteq#2}
\newcommand{\jasubtype}[2]{\sat^{\mathsf{\grt}}#1\sqsubseteq#2}
\newcommand{\jeqtype}[2]{\sat#1 \equiv#2}
\newcommand{\under}[2]{\sat #1 \trianglelefteq  #2}

\newcommand{\type}{\text{type}}
\newcommand{\rtype}{\text{relational type}}
\newcommand{\Type}{\text{Unary type}}
\newcommand{\Rtype}{\text{Binary type}}



% Cost Constants
\newcommand{\kvar}{c_{var}}
\newcommand{\kconst}{c_{n}}
\newcommand{\kinl}{c_{inl}}
\newcommand{\kinr}{c_{inl}}
\newcommand{\kcase}{c_{case}}
\newcommand{\kfix}{c_{fix}}
\newcommand{\kapp}{c_{app}}
\newcommand{\kLam}{c_{fix}}
\newcommand{\kiApp}{c_{iApp}}
\newcommand{\kpack}{c_{pack}}
\newcommand{\kunpack}{c_{unpack}}
\newcommand{\knil}{c_{nil}}
\newcommand{\kcons}{c_{cons}}
\newcommand{\kcaseL}{c_{caseL}}
\newcommand{\kleaf}{c_{leaf}}
\newcommand{\knode}{c_{node}}
\newcommand{\kcaseT}{c_{caseT}}
\newcommand{\kprod}{c_{prod}}
\newcommand{\kproj}{c_{proj}}
\newcommand{\klet}{c_{let}}


%Constraints
\newcommand{\creal}{\mathbb{R}}
\newcommand{\sat}[1]{\models#1}
\newcommand{\sata}[1]{\models_A#1}
\newcommand{\ceq}[2]{#1\mathrel{\doteq}#2}
\newcommand{\cleq}[2]{#1 \mathop{\leq} #2}
\newcommand{\cleqspec}[2]{#1 \overline{\mathop{\leq}} #2}
\newcommand{\clt}[2]{#1 \mathop{<} #2}
\newcommand{\cgt}[2]{#1 \mathop{>} #2}
\newcommand{\ceqz}[1]{#1 \mathrel{\doteq} 0}
\newcommand{\cneg}[1]{\mathop{\neg}#1}
\newcommand{\cand}[2]{#1 \wedge #2}
\newcommand{\cexists}[3]{\exists#1::#2.#3}
\newcommand{\cexistsK}[3]{\exists#1:#2.#3}
\newcommand{\cexistsS}[2]{\exists#1.#2}
\newcommand{\cexistsC}[2]{\exists#1::\scost.#2}
\newcommand{\cexistsall}[2]{\exists(#1).#2}
\newcommand{\cforall}[3]{\forall#1::#2.#3}
\newcommand{\cforallS}[3]{\forall#1:#2.#3}
\newcommand{\cimpl}[2]{#1\rightarrow#2}
\newcommand{\cor}[2]{#1 \vee #2}
\newcommand{\ctrue}{\top}
\newcommand{\cfalse}{\bottom}
\newcommand{\blank}[2][100]{\hfil\penalty#1\hfilneg }
\newcommand{\ccond}[3]{\im{#1}\mathrel{\mbox{?}}\im{#2}\mathrel{\colon}\im{#3}}


\newcommand{\wfty}[1]{\vdash   #1~\kw{wf}}
\newcommand{\awfty}[1]{\vdash^{\mathsf{\grt}}   #1~\kw{wf}}
\newcommand{\wfcs}[1]{\vdash   #1~\kw{wf}}
\newcommand{\wfctx}[1]{\vdash   #1~\kw{wf}}
\newcommand{\awfctx}[1]{\vdash^{\mathsf{\grt}}   #1~\kw{wf}}



\newcommand{\dc}{ downward closure (\lemref{lem:down-closure}) }
\newcommand{\ctx}{\Delta; \Phi_a; \Gamma}
\newcommand{\nctx}{\Delta; \Phi_a; \tbox{\Gamma}}
\newcommand{\primctx}{\Upsilon}
\newcommand{\octx}{\Delta; \Phi_a; \Omega}
\newcommand{\rctx}[1]{\Delta; \Phi_a; \trm{\Gamma}{#1}}

\newcommand{\shade}[1]{\colorbox{lightgray}{#1}}
\newcommand{\fv}[1]{\text{FV}(#1)}
\newcommand{\fcv}[1]{\text{dom}(#1)}
\newcommand{\fiv}[1]{\text{FIV}(#1)}
\newcommand{\fdv}[1]{\text{dom}(#1)}


\newcommand{\assC}[2]{\text{Assume that $\sat \s \Phi$ and there exists $\Gamma'$  s.t. $\fv{#2} \subseteq \fcv{\Gamma'} $ and $\Gamma' \subseteq \Gamma$ and $(m, \d) \in \ugsubst{\trm{\sigma \Gamma'}{#1}}$}}
\newcommand{\assCU}[1]{\text{Assume that $\sat \s \Phi$ and there exists $\Omega'$  s.t. $\fv{#1} \subseteq \fcv{\Omega'} $ and $\Omega' \subseteq \Omega$ and $(m, \d) \in \ugsubst{\s \Omega'}$}}
\newcommand{\IHassun}[1]{\text{$\fv{#1} \subseteq \fcv{\Omega'} $ and $\Omega' \subseteq \Omega$ and $(m, \d) \in \ugsubst{\s \Omega'}$}}


\newcommand{\IHassU}[2]{\text{$\fv{#2} \subseteq \fcv{\trm{\Gamma'}{#1}} $ and $\trm{\Gamma'}{#1} \subseteq \trm{\Gamma}{#1}$ and $(m, \d) \in \ugsubst{\trm{\sigma \Gamma'}{#1}}$}}


\newcommand{\IHass}[2]{\text{$\fv{#2} \subseteq \fcv{\Gamma'} $ and $\Gamma' \subseteq \Gamma$ and $(m, \d) \in \ugsubst{\trm{\sigma \Gamma'}{#1}}$}}
% Environment
\newcommand{\memory}{\Gamma}%\Delta \ | \ \Phi \ | \ \Gamma  \ | \
                            %\Sigma}
\newcommand{\senv}{\Delta}
\newcommand{\lenv}{\Sigma}
\newcommand{\uenv}{\Omega}
\newcommand{\renv}{\Gamma} 
\newcommand{\cenv}{\Phi} 
\newcommand{\sep}{ \ | \ }
\newcommand{\monad}[4]{\mathrel{ M( \overset{ \mathrel{\mathrm{exec}{#4 }}}{#2}  })}
\newcommand{\monadR}[4]{\mathrel{ \overset{\mathrm{diff}(#4)}{\{ {#1} \} \ {#2} \ \{ {#3} \} }}}
\newcommand{\depProd}[4]{ \mathrel{ \Pi {#1} \stackrel{\mathrm{exec} {#4}}{:}{#2} . \ {#3}}}
\newcommand{\depProdr}[4]{ \mathrel{ \Pi {#1} \stackrel{\mathrm{diff} {#4}}{:}{#2} . \ {#3}}}
\newcommand{\uarrow}[3]{ \mathrel{ \stackrel{\mathrm{exec} {#3}}{{#1} \longrightarrow#2}}}
\newcommand{\uforall}[4]{ \mathrel{ \stackrel{\mathrm{exec} {#4}}{\forall {#1} :#2 . \ #3}}}
\newcommand{\uexist}[3]{\mathrel{{\exists {#1} :: {#2} . \ {#3}}}}
\newcommand{\rarrow}[3]{ \mathrel{ \stackrel{\mathrm{diff}(#3)}{{#1} \longrightarrow {#2}}}}
\newcommand{\rarrowt}[3]{ \mathrel{ {#1} \stackrel{\mathrm{} {#3}}{\longrightarrow} {#2}}}
\newcommand{\rforall}[4]{ \mathrel{ \stackrel{\mathrm{diff}(#4)}{\forall {#1}{::}{#2} . \ {#3}}}}
\newcommand{\rexists}[3]{ \mathrel{ {\exists {#1} {::}{#2} . \ {#3}}}}
\newcommand{\rforallt}[4]{ \mathrel{ \forall {#1} \stackrel{\mathrm{} {#4}}{:}{#2} . \ {#3}}}
\newcommand{\arr}[3]{ \mathrel{ \mathsf{Array}_{#1}[{#2}] \ {#3}} }
\newcommand{\arrR}[3]{ \mathrel{ \mathsf{Array}_{#1}[{#2}] \ {#3}} }
\newcommand{\lst}[2]{ \mathrel{ \mathsf{list}[{#1}] \ {#2}} }
\newcommand{\lstR}[3]{ \mathrel{ \mathsf{list}^{#1}[{#2}] \ {#3}} }
\newcommand{\abs}[2]{\mathrel { \lambda {#1} . {#2} } }
\newcommand{\app}[2]{\mathrel{ {#1} \, {#2} }}
\newcommand{\ret}[1] {\mathrel{ \mathsf{return} \, {#1} }}

\newcommand{\letx}[3]{  \mathrel{ \mathsf{let}\   {#1} = {#2} \ \mathsf{in} \ {#3}  } }
\newcommand{\packx}[1]{  \mathrel{ \mathsf{pack} \, {#1}} }
\newcommand{\unpackx}[3]{  \mathrel{ \mathsf{unpack} \,  {#1} \, \mathsf{as} \, {#2} \, \mathsf{in} \ {#3}  } }
\newcommand{\alloc}[2]{ \mathrel{ \mathsf{alloc}\, {#1} \, {#2} } }
\newcommand{\updt}[3]{ \mathrel{ \mathsf{update} \ {#1} \ {#2} \ {#3} }  }
\newcommand{\readx}[2] { \mathrel{ \mathsf{read} \ {#1} \ {#2} }  }
\newcommand{\tTt}[3]{\mathrel{  {#1} \xrightarrow \ {#2} } }
\newcommand{\force}[1]{\mathrel{\mathsf{force} \ \ {#1}}}
\newcommand{\tfix}{\mathsf{Fix}}
\newcommand{\fix}[1]{\mathsf{fix} \, f(x). {#1}}

%Relational
\newcommand{\monadx}[3]{\mathrel{ \{ {#1} \} \ {#2} \ \{ {#3} \} }}
\newcommand{\monadu}[4]{\mathrel{ \overset{ \mathrel{\mathrm{exec}{#4 }}}{\{ {#1} \} \ #2 \ \{ {#3} \}} }}
\newcommand{\cmp}[4] {\mathrel{   \vdash  {#1} \ominus {#2} \ldiff {#4}  : {#3}  }}
\newcommand{\pair}[1]{\mathrel{ {#1}_{1}{#1}_{2}}}
\newcommand{\imp}[2]{\mathrel{  {#1} \Rightarrow {#2} }}
\newcommand{\eval}[3]{\mathrel{ {#1} \Downarrow^{#3} {#2}   }}
\newcommand{\evalf}[3]{\mathrel{ {#1} \Downarrow^{#3}_{f} {#2}   }}
\newcommand{\evalp}[3]{\mathrel{ {#1} \Downarrow^{#3}_{p} {#2}   }}
\newcommand{\heap}[1]{ ;  {#1}}
\newcommand {\spc} { \  \ }
\newcommand{\monadL}[3]{\mathrel{ \{ {#1} \}  \\  \ {#2} \ \\  \{ {#3} \} }}

\newcommand{\wfa}[1]{\mathrel{\vdash {#1} \quad wf}}
\newcommand{\wf}[1]{\mathrel{\vdash {#1} \quad wf}}
\newcommand{\subtypeA}[2]{\mathrel{ \models {#1} \sqsubseteq {#2} } }
\newcommand{\subtype}[2]{\mathrel{   \models {#1} \sqsubseteq {#2} }   }
\newcommand{\subcost}[3]{\mathrel{   \models {#1} {#3} {#2} }   }

\newcommand{\emptyhp}{\mathsf{empty}}
\newcommand{\llb}[1]{ \llbracket {#1} \rrbracket }
\newcommand{\llu}[2]{ \llb{#1}_{#2}}
\newcommand{\llp}[2]{ \llparenthesis {#1} \rrparenthesis_{#2} }
\newcommand{\llbe}[1]{ \llbracket {#1} \rrbracket^{E} }
\newcommand{\llpe}[2]{ \llparenthesis {#1} \rrparenthesis_{#2}^{E} }

\newcommand{\mg}[1]{\textcolor[rgb]{.90,0.00,0.00}{[MG: #1]}}
\newcommand{\dg}[1]{\textcolor[rgb]{0.00,0.5,0.5}{[DG: #1]}}
\newcommand{\wq}[1]{\textcolor[rgb]{.50,0.0,0.7}{[WQ: #1]}}

% Helpful shortcuts

\newcommand{\freshSize}[1]{#1\in\text{fresh}(\ssize)}
\newcommand{\freshCost}[1]{#1\in\text{fresh}(\scost)}
\newcommand{\freshVar}[1]{#1 \in \text{fresh}(S)}

\newcommand{\m}{M} 
%Bi-directional Typing Judgement
\newcommand{\chdiff}[5]{\vdash{#1}\rdiff{#2}~{\downarrow}~#3,#4 \Rightarrow
{\color{red}#5}}

\newcommand{\chsdiff}[3]{\vdash{#1}\backsim{#2}~{\downarrow}~#3}


\newcommand{\chdiffNC}[5]{\vdash^{\color{blue}NC}{#1}\rdiff{#2}~{\downarrow}~#3,#4 \Rightarrow
{{\color{red}#5}}}

\newcommand{\infdiff}[6]{\vdash{#1}\rdiff{#2}~{\uparrow}~{\color{red}{#3}}\Rightarrow[{\color{red}#4}],{\color{red}#5},{\color{red}#6}}

\newcommand{\infsdiff}[3]{\vdash{#1}\backsim{#2}~{\uparrow}~{\color{red}{#3}}}

\newcommand{\infdiffsimple}[5]{\vdash{#1}\rdiff{#2}~{\uparrow}~{\color{red}{#3}}\Rightarrow{\color{red}#4},{\color{red}#5}}

\newcommand{\chmax}[4]{\vdash^{\wmax}{#1}~{\downarrow}~#2, #3  \Rightarrow
{{\color{red}#4}}}
\newcommand{\chmin}[4]{\vdash^{\wmin}{#1}~{\downarrow}~#2, #3  \Rightarrow
{{\color{red}#4}}}

\newcommand{\chexec}[5]{\vdash{#1}~{\downarrow}~#2, #3, #4  \Rightarrow
{{\color{red}#5}}}

\newcommand{\infmax}[5]{\vdash^{\wmax}{#1}~{\uparrow}~{\color{red}{#2}}\Rightarrow[{\color{red}#3}], {\color{red}#4},{\color{red}#5}}
\newcommand{\infmin}[5]{\vdash^{\wmin}{#1}~{\uparrow}~{\color{red}{#2}}\Rightarrow[{\color{red}#3}], {\color{red}#4},{\color{red}#5}}

\newcommand{\infexec}[6]{\vdash{#1}~{\uparrow}~{\color{red}{#2}}\Rightarrow[{\color{red}#3}], {\color{red}#4},{\color{red}#5},{\color{red}#6}}

\newcommand{\infexecsimple}[5]{\vdash{#1}~{\uparrow}~{\color{red}{#2}}\Rightarrow{\color{red}#3}, {\color{red}#4},{\color{red}#5}}

\newcommand{\emptypsi}{.}

%Existential elimination
\newcommand{\elimExt}[3]{#1 \vdash \kw{elimExt}(#2)~\downarrow #3}
\newcommand{\solveVar}[6]{#1 \vdash \kw{solve}(#2;#3) \downarrow (#4;#5;#6)}



%Shortcuts
\newcommand{\al}{\alpha}
\newcommand{\algwf}[1]{\vdash  #1~\kw{wf}}
\newcommand{\algwfa}[1]{\vdash^{A} #1~\kw{wf}}
\newcommand{\jalgeqtype}[3]{\sat#1\equiv#2\Rightarrow {\color{red}#3}}
\newcommand{\jalgasubtype}[3]{\sat^{\mathsf{\grt}}#1\sqsubseteq#2\Rightarrow {\color{red}#3}}
\newcommand{\jalgsubtype}[3]{\sat#1\sqsubseteq#2\Rightarrow {\color{red}#3}}
\newcommand{\jalgssubtype}[2]{\sat#1\leq#2}


\newcommand{\fvars}[1]{\text{FV}(#1)}
\newcommand{\fivars}[1]{\text{FIV}(#1)}
\newcommand{\filtercost}[1]{\text{filterCost}(#1)}
\newcommand{\uctx}{\Delta; \psi_a; \Phi_a; \Omega}
\newcommand{\bctx}{\Delta; \psi_a; \Phi_a; \Gamma}


\newcommand{\suba}[1]{{#1}[\theta_a]}
\newcommand{\subaex}[2]{{#1}[\theta_a, #2]}
\newcommand{\subt}[1]{{#1}[\theta]}
\newcommand{\subta}[1]{{#1}[\theta\,\theta_a]}
\newcommand{\subsat}[3]{#1~ \rhd~ #2 : #3}

\newcommand{\erty}[1]{|#1|}
\newcommand{\eanno}[4]{(#1:#2,#3,#4)}
\newcommand{\eannobi}[3]{(#1:#2,#3)}
\newcommand{\e}{\overline{e}}
\newcommand{\trans}{\rightsquigarrow}
\newcommand{\tboxp}[1]{\square(#1)}
\newcommand{\tlr}[1]{\tlift{\trm{#1}{i}}}
\newcommand*\bang{!}



%%% Local Variables:
%%% mode: latex
%%% TeX-master: "main"
%%% End:

\newcommand{\nform}{\mathsf{F}}
\newcommand{\mechanism}{\mathsf{M}}
\newcommand{\depth}{\mathsf{depth}}
\newcommand{\query}{\text{Q}}

% \newcommand{\caseof}[2]{\mathsf{case} \ {#1} \ \mathsf{of}\ \{ {#2}\}}
\newtheorem{lemma}{Lemma}
\newtheorem{theorem}{Theorem}[section]
\newtheorem{corollary}{Corollary}[theorem]
\newcommand{\aexpr}{a}
\newcommand{\bexpr}{b}
\newcommand{\cmd}{c}
\newcommand{\node}{N}
\newcommand{\assign}[2]{ \mathrel{ #1  \leftarrow #2 } }
\newcommand{\fassign}[3]{ \mathrel{ #1  \leftarrow^{#3}  \delta^{#3}(
    #2 ) } }
\newcommand{\impif}[3]{\mathrel{\eif \eapp #1\eapp \ethen \eapp #2 \eapp
    \eelse \eapp #3 }}
\newcommand{\impwhile}[2]{\mathrel{ \kw{while} (#1) \eapp #2 } }
\newcommand{\labl}{l}

\let\originalleft\left
\let\originalright\right
\renewcommand{\left}{\mathopen{}\mathclose\bgroup\originalleft}
\renewcommand{\right}{\aftergroup\egroup\originalright}

\theoremstyle{definition}

\title{Revisit of Adaptivity analysis}

\author{}

\date{}

\begin{document}

\maketitle

% \section{Attempt 1: linear-type based}
% 
% \begin{figure}[h]
 $$
 \begin{array}{rcl}
     \text{Types} & \quad & \tau ::= b \sep \tau \multimap \tau' \sep !_n \tau \sep
     \tau \times \tau \sep \tforallN{i}{\tau} \sep \query \\[2mm]

     \text{Term} & \quad & t ::= c \sep \fix{t} \sep \app{t}{t} \sep !t \sep (t_1,t_2) \sep  \letx{!x}{t_1}{t_2} \sep \Lambda.t \sep t[] \sep \abs{x}{t} \sep  M(t) \sep x \sep q \sep\\
     && \quad   \tcaseof{t}\ \{c_i \Rightarrow t_i\}_{c_i \in b}  \sep \letx{(x_1,x_2)}{t_1}{t_2} \\[2mm]
      
     \text{Normal Form} &\quad & v ::=  c \sep \fix{t} \sep !t \sep (v_1, v_2) \sep \Lambda. t \sep \abs{x}{t}  \sep x \sep q \sep \tcaseof{v}\ \{c_i \Rightarrow v_i\}_{c_i \in b_i} \sep\\
     && \enil \sep \econs(v_1,v_2)  \\[2mm]

     \text{Mechanisms} &\quad & M ::=  {\tt gauss} \sep {\tt thdt} \\[2mm]


	\text{Tree} &\quad& T_b :: = c \sep M(T_{query}) \sep \tcaseof{T_b}\ \{ c_i \Rightarrow T_{b_i}\}_{c_i \in b} \\

	\text{} &\quad& T_{query} :: = q \sep \tcaseof{T_b}\ \{ c_i \Rightarrow T_{query_i}\}_{c_i \in b} \\[2mm]
     \text{Depth} &\quad&   \depth(c) = 0 \\
       &\quad& \depth(!t) = \depth(t) \\
           &\quad&      \depth( \app{t_1}{t_2} ) = \max(\depth(t_1), \depth(t_2)) \\
            &\quad&  \depth(M(t)) = 1 + \depth(t) \\
             &\quad&  \depth(\abs{x}{t}) = \depth(t) \\
              &\quad& \depth(x) = 0 \\
              & \quad & \depth(q) = 0 \\
              & \quad & \depth((t_1, t_2)) = \max(\depth(t_1), \depth(t_2))\\
              & \quad & \depth(\letx{(x_1, x_2)}{t}{t'}) = \max(\depth(t), \depth(t'))\\
              & \quad & \depth{}(\letx{!x}{t}{t'}) = \max(\depth(t), \depth(t'))\\
              & \quad & \depth{}(\tcaseof{t}\ \{c_i \Rightarrow t_i\}_{c_i \in b}) = \max(\depth(t), \depth(t_i))\\
            & \quad & \depth(\Lambda.t) = \depth(t)\\
              & \quad & \depth(t\, []) = \depth(t)\\
\end{array}
$$
% \caption{syntax}
% \end{figure}

%
%
\begin{figure}[h]
\boxed{  \Gamma \jtype{n,m}{}{t}{\tau}    }\\
\boxed{  \Gamma :: = \emptyset \sep \Gamma, x : \tau \sep \Gamma, x : [\tau]_p   }

\begin{mathpar}
  \inferrule*[right = const]
   {\empty}
   {\Gamma \jtype{n,m}{}{c}{b}  }
   
   \and
    \inferrule*[right = abs]
   {\Gamma, x: \tau_1 \jtype{n}{}{t}{\tau_2}}
   { \Gamma \jtype{n,m}{}{\abs{x}{t}}{\tau_1 \multimap \tau_2}  }
   
   \and
   \inferrule*[right = pr]
   {[\Gamma] \jtype{n}{}{t}{\tau}}
   {\Delta, p + [\Gamma] \jtype{n+p}{}{!t}{!_p \tau}  }
%   \boxed{
%   \inferrule*[right = pr]
%   {[\Gamma] \jtype{n}{}{t}{\tau}}
%   {p + [\Gamma]\textbf{} \jtype{n}{}{!t}{!_p \tau}  }}
   
   \and
    \inferrule*[ right = var]
   {\empty}
   {\Gamma, x:\tau \jtype{n}{}{x}{\tau}  } 
   
   \and
   \inferrule*[ right = MT ]
   {[\Gamma] \jtype{n}{}{t}{query}}
   {\Delta, 1 + [\Gamma] \jtype{n+1}{}{M(t)}{b}  }
   
   \and
    \inferrule*[right = query]
   {\empty}
   {\Gamma \jtype{n}{}{q}{query}  }
   
   \and
%     \inferrule*[ right = app ]
%   {\Gamma_1 \jtype{n_1}{}{t_1}{\tau_1 \rightarrow \tau_2} \\ \Gamma \jtype{n_2}{}{t_2}{\tau_1}}
%   { \Gamma_2 \jtype{\max(n_1,n_2)}{}{\app{t_1}{t_2}}{\tau_2}  }
  \inferrule*[ right = app ]
   {\Gamma_1 \jtype{n_1}{}{t_1}{\tau_1 \multimap \tau_2} \\ \Gamma_2 \jtype{n_2}{}{t_2}{\tau_1}}
   { \max(\Gamma_1, \Gamma_2) \jtype{\max(n_1,n_2)}{}{\app{t_1}{t_2}}{\tau_2}  }
   
   \and 
   
%   \inferrule*[ right = der ]
%   {\Gamma, x: \tau \jtype{n}{}{t}{ \tau }  }
%   { \Gamma , x: [\tau]_p \jtype{n }{}{t }{\tau }  }
   \boxed{
   \inferrule*[ right = der ]
   {\Gamma, x: \tau \jtype{n}{}{t}{ \tau }  }
   { \Gamma , x: [\tau]_0 \jtype{n }{}{t }{\tau }  }
 }
   
   \and 
   
   \inferrule*[ right = let-b ]
   {\Gamma_1 \jtype{n_1}{}{t}{!_p \tau} \\ \Gamma_2, x: [\tau]_p \jtype{n_2}{}{t'}{\tau'}}
   { \max(\Gamma_1, \Gamma_2)  \jtype{\max(n_1,n_2)}{}{\letx{!x}{t}{t'}}{\tau'}  }
   
   \and 
   \inferrule*[ right = let-p ]
   {\Gamma_1 \jtype{n_1}{}{t}{\tau_1 \times \tau_2 } \\ \Gamma_2, x_1: \tau_1, x_2 : \tau_2 \jtype{n_2}{}{t'}{\tau'}}
   { \max(\Gamma_1, \Gamma_2)  \jtype{\max(n_1,n_2)}{}{ \letx{(x_1,x_2)}{t}{t'} }{\tau'}  }
   
   \and
     \inferrule*[right = pair]
   {\Gamma_1 \jtype{n_1}{}{t_1}{\tau_1} \\ \Gamma_2 \jtype{n_2}{}{t_2}{\tau_2}}
   { \max(\Gamma_1, \Gamma_2)  \jtype{\max(n_1,n_2)}{}{(t_1, t_2)}{\tau_1 \times \tau_2}  }
 
   \and
    \inferrule*[ right = case-const ]
   {\Gamma_1 \jtype{n_1}{}{t}{b} \\ \Gamma_2 \jtype{n_2}{}{t_i}{b} }
   {\max(n_2 + \Gamma_1, \Gamma_2)  \jtype{ (n_1+n_2)}{}{\tcaseof{t}\ \{c_i \Rightarrow t_i\}_{c_i \in b} } {b} }
   
   \and
    \inferrule*[ right = case-query]
   {\Gamma_1 \jtype{n_1}{}{t}{b} \\ \Gamma_2 \jtype{n_2}{}{t_i}{query} }
   {\max(\Gamma_1, \Gamma_2) \jtype{ (n_1+n_2)}{}{\tcaseof{t}\ \{c_i \Rightarrow t_i\}_{c_i \in b} } {query} }
   
   \inferrule*[right = iabs]
  { 
    \inferrule*[]
    {}
    {i::\mathbb{N};\Gamma \jtype{n}{}{t}{ \tau } }
    \and
    \inferrule*[]
    {}
    { i \notin \fiv{\Gamma }  } 
  }
  { \Gamma \jtype{n}{ }{  \Lambda.t  }{ \tforallN{i}{\tau}  } }
  
   \inferrule*[ right =  iapp]
  { 
    \inferrule*[]
    {}
    { \Gamma  \jtype{n}{}{t}{ \tforallN{i}{\tau}   } }
    \and
    \inferrule*[]
    {}
    { \jiterm{I}{ \mathbb{N} } } 
  }
  {\Gamma \jtype{n }{ }{t\, [] }{ \tau \{ I/i \}  } }

  \inferrule*[right = sub]
  { 
   { \Gamma \jtype{n}{}{t}{\tau} } \\
   {  \Gamma' \subseteq \Gamma } \\
   { \vDash n \leq n' } \\
   { \tau \subseteq \tau' }
  }
  { \Gamma' \jtype{n'}{}{t}{\tau'} }
\end{mathpar}
\caption{Typing judgment}
\end{figure}

% \clearpage

% \begin{figure}[h]
%  \begin{mathpar}
  
%   \inferrule*[right= S-ID]
%   { }
%   { \tau <: \tau  }
%   \and
%   \inferrule*[right = S-B]
%   { 
%    {A <: B}
%    \\
%    { q \leq p }
%   }
%   { !_p A <: !_q B  }
%   \and
%   \inferrule*[right =  S-ARROW]
%   { {A' <: A}
%     \\
%     {B <: B'}
%   }
%   { A \multimap B <: A' \multimap B' }
%   \and
%   \inferrule*[right = S-D ]
%   {
%     { A \subseteq B }\\
%     { q \leq p }
%   }
%   { [A]_p \subseteq [B]_q }
  
%   \and
%   \inferrule*[right = S-IDC]
%   { }
%   { \Gamma \subseteq \Gamma }
  
%    \and
%   \inferrule*[right = S-empty]
%   { }
%   { \Gamma \subseteq \emptyset }
  
%   \and
%   \inferrule*[right = S-Ctx]
%   {
%   {A \subseteq B}\\
%   { \Gamma \subseteq \Delta }
%   }
%   { \Gamma, x: A \subseteq \Delta, x: B }
  
%   \and
%   \inferrule*[right = S-xctx1]
%   {
%   { \Delta \subseteq \Gamma }
%   }
%   { x : \tau, \Delta \subseteq \Gamma }
  
%   \and
%   \inferrule*[right = S-xctx2]
%   {
%   { \Delta \subseteq \Gamma }
%   }
%   { x : [\tau]_p, \Delta \subseteq \Gamma }
  
%  \end{mathpar}
%  \caption{sub typing}
% \end{figure}



\begin{figure}[h]
\boxed{\eval{t }{v }{m}}
\begin{mathpar}
%  \inferrule*[ right=E-values]
%   { }
%   { \eval{ F   }{ F  }{0}}   
%   \and
 \inferrule*[ right=E-const]
  { }
  { \eval{ c   }{ c  }{0}}   
  
  \and

 \inferrule*[ right=E-query]
  { }
  { \eval{  q  }{ q  }{0}}   
  
  \and

 \inferrule*[ right=E-ABS]
  { }
  { \eval{ \abs{x}{t}   }{ \abs{x}{t}  }{0}}   
  
  \and
  
  \inferrule*[ right=E-bang]
  { }
  { \eval{ ! t   }{ ! t  }{0}}   
  
  \and
  
   \inferrule*[ right=E-pair]
  {   
    { \eval{ t_1  }{ v_1  }{m_1} }
    \\
    { \eval{ t_2  }{ v_2  }{m_2} } 
  }
  { \eval{  (t_1,t_2)  }{ (v_1,v_2)  }{ \max(m_1, m_2) %m_1+m_2
  } }  
  
  \and

   \inferrule*[ right=E-app]
  {   
    { \eval{ t_1  }{ \abs{x}{t}  }{m_1} }
    \\
    { \eval{ t_2  }{ v  }{m_2} } 
    \\
    { \eval{t[v/x] }{ v'}{m_3 } }
  }
  { \eval{ \app{t_1}{t_2}  }{ v'  }{ \max(m_1, m_2 ) + m_3  } }  
 
 \boxed{ 
   \inferrule*[ right=E-let-bang]
  {   
    { \eval{ t_1  }{ !t_3  }{m_1} } 
    \\
    {\eval{t_3}{v'}{m_2}}
    \\
    { \eval{t_2[v'/x] }{v}{m_3 } }
  }
  { \eval{  \letx{!x}{t_1}{t_2}  }{ v  }{ \max(m_1+ m_2, m_3)  } }  
}

%  \inferrule*[ right=E-let-bang]
%   {   
%     { \eval{ t_1  }{ !t_3  }{m_1} } 
%     \\
%     { \eval{t_2[!t_3 /x] }{ F}{m_3 } }
%   }
%   { \eval{  \letx{!x}{t_1}{t_2}  }{ F  }{ m_1+m_3  } } 
  
  \inferrule*[ right=E-let-p]
  {   
    { \eval{ t  }{ (v_1,v_2)  }{m_1} } 
    \\
    { \eval{t'[v_1/x_1][v_2/x_2] }{v}{m2 } }
  }
  { \eval{  \letx{(x_1,x_2)}{t}{t'}  }{ v  }{ \max(m_1,m_2)  } } 
  
 

  \inferrule*[ right=E-case]
  { 
    \inferrule*[]
    {}
    {\eval{  t  }{ v }{m }  }
    \\
    \inferrule*[]
    {}
    { \eval{ t_i  }{ v_i   }{ m_i }  }
  }
  { \eval{ \tcaseof{t}\ \{c_i \Rightarrow t_i\}_{c_i \in b}  }{ \tcaseof{v}\ \{c_i \Rightarrow v_i\}_{c_i \in b}  }{  m + \max(m_i) } }
  
    \inferrule*[ right=E-fix]
  { 
  }
  { \eval{  \fix{t}  }{ \fix{t}  }{ 0 } }
  
      \inferrule*[ right=E-x]
  { 
  \empty
  }
  { \eval{  x  }{ x  }{ 0 } }
  
      \inferrule*[ right=E-ILAM]
  { 
    \empty
  }
  { \eval{  \Lambda. t  }{  \Lambda. t }{ 0 } }
  
      \inferrule*[ right=E-iapp]
  { 
    \inferrule*[]
    {}
    {\eval{  t  }{ \Lambda. t' }{m }  }
  }
  { \eval{  t[]  }{  t' }{  m } }
  
      \inferrule*[ right=E-mech]
  { 
    \inferrule*[]
    {}
    {\eval{  t  }{ v }{m }
    \and 
    \eval{M(v)}{v'}{1}
    }
  }
  { \eval{  M(t)  }{ v'  }{  m + 1 } }
  
\end{mathpar}
 \caption{Evaluation Rules}
\end{figure}

% \begin{figure}[h]
% \begin{mathpar}

% \inferrule*[right = E-if-true]
% {
%  \eval{b}{true}{0}
%  \and 
%  \eval{t_1}{v_1}{m} 
% }
% {
% \eval{\tif{b}{t_1}{t_2}}{v_1}{m}
% }

% \inferrule*[right = E-if-false]
% {
%  \eval{b}{false}{0}
%  \and 
%  \eval{t_2}{v_2}{m} 
% }
% {
% \eval{\tif{b}{t_1}{t_2}}{v_2}{m}
% }

% \inferrule*[right = E-nil]
% {
%  \empty
% }
% {
% \eval{\enil}{\enil}{0}
% }

% \inferrule*[right = E-cons]
% {
% \eval{t_1}{v_1}{m_1}
% \and 
% \eval{t_2}{v_2}{m_2}
% }
% {
% \eval{\econs(t_1,t_2)}{\econs(v_1,v_2)}{\max(m_1,m_2)}
% }


% \inferrule*[right = E-let]
% {
% \eval{t_2}{v_2}{m_2}
% \and
% \eval{t[v_2/x]}{v}{m}
% }
% {
% \eval{\letx{x}{t_2}{t}}{v}{\max(m_2,m)}
% }

% \end{mathpar}
%  \caption{New Added Evaluation Rules}
% \end{figure}


\begin{figure*}[h]
$$
\begin{array}{rcl}
      \llu{\tau}{\epsilon}  &\quad &  = \{ \, e \, | \, \exists v. e \Downarrow v \land v \in   \llu{\tau}{v} \,  \}  \\[2mm]
      \llu{b}{v} &\quad &  = \{ \,  v  \, | \, v = T_b \}  \\[2mm]
      \llu{query}{v} &\quad &  = \{ \,  v \, |  \, v = T_{query} \}  \\[2mm]
      \llu{\tau_1 \rightarrow \tau_2}{v} & \quad & = \{\, \abs{x}{t} \, | \, \forall v \in\llu{\tau}{v}. t[v/x] \in \llu{\tau_2}{\epsilon} \, \} \\[2mm]
      \llu{ \ !_n \tau}{v} & \quad & = \{\, !t \, | \, t \in \llu{\tau}{\epsilon} \, \} \\[2mm]
      \llu{\tforallN{i}{\tau}}{v}  & \quad & = \{  \Lambda. t \, | \, \forall I. \vdash i :: \mathbb{N}. t[I/i] \in \llu{\tau}{\epsilon}   \}  \\[2mm]
      \llu{\tau_1 * \tau_2}{v}  & \quad & = \{  (v_1, v_2) \, | \, v_1 \in \llu{\tau_1}{v} \land v_2 \in \llu{\tau_2}{v}     \} \\[2mm]
      \llu{\cdot}{} &\quad & = \{ \emptyset \} \\[2mm]
      \llu{\Gamma, x : [\tau]_p}{} & \quad & = \{ \gamma[x \rightarrow v] | v \in \llu{\tau}{v} \land \gamma \in \llu{\Gamma}{}   \}  \\ [2mm]
      \boxed{\llu{\Gamma, x : [\tau]_p}{}}  & \quad & = \{ \gamma[x \rightarrow v] | v \in \llu{!_p \tau}{v}  \land \gamma \in \llu{\Gamma}{}   \}  \\ [2mm]
      \llu{\Gamma, x : \tau}{} & \quad & = \{ \gamma[x \rightarrow v] | v \in \llu{\tau}{v} \land \gamma \in \llu{\Gamma}{}   \}  \\ [2mm]
      \gamma \vDash \Gamma &\quad & \triangleq dom(\gamma) = dom(\Gamma) \land \forall x \in dom(\Gamma). \gamma(x) \in \llu{\Gamma(x)}{v}
\end{array}
$$
\caption{denotations}
\end{figure*}


% \clearpage
% \begin{lem} $ $
% 	\label{lem:1}
%     \begin{enumerate}
% \item If $\jtype{n,m}{}{v}{b} $ then $ \exists T_{b} : v = T_{b}$.\\
% \item If $\jtype{n,m}{}{v}{query} $ then $ \exists T_{query} : v = T_{query}$
% \end{enumerate}
	
	
% \end{lem}

% \begin{lem}[Depth Definition] 
% 	\label{lem:2}
% 	If $\Gamma \jtype{n,m}{}{t}{\tau} $ then $\depth(t) \leq n$\\
% \end{lem}
% %
% %
% %
% \begin{lem}[Depth Weakening1] 
% 	\label{lem:deweaken1}
% 	$\Gamma \jtype{n_1,m}{}{t}{\tau} \land n_1 \leq n_2 \implies \Gamma \jtype{n_2,m}{}{t}{\tau}$\\
% \end{lem}
% \begin{proof}
%   By induction on $\Gamma \jtype{n_1}{}{t}{\tau}  $.
% \end{proof}

% \begin{lem}[Depth Weakening2] 
% 	\label{lem:deweaken2}
% 	$\Gamma, x:[\tau]_{p_1} \jtype{n}{}{t}{\tau} \land p_1 \leq p_2 \implies \exists m. n \leq m$ s.t. $\Gamma, x:[\tau]_{p_2} \jtype{n}{}{t}{\tau} $\\
% % 	$\Gamma, x:[\tau]_{p_1} \jtype{n}{}{e}{\tau} \land p_1 \leq p_2 \land \Gamma \subseteq \Gamma' \land n \leq n' \implies \Gamma', x:[\tau]_{p_2} \jtype{n'}{}{e}{\tau} $\\
% \end{lem}
% %
% %
% %
% \begin{lem}[Context weakening - 1]
%     \label{lem:coweaken1}
%     $\Gamma \jtype{n,m}{}{t}{\tau}  \implies \Gamma, x:\tau \jtype{n,m}{}{t}{\tau} $\\
% \end{lem}

% \begin{lem}[Context weakening - 2]
%     \label{lem:coweaken2}
%     $\Gamma \jtype{n,m}{}{t}{\tau}  \implies \Gamma, x:[\tau]_p \jtype{n,m}{}{t}{\tau} $\\
% \end{lem}



% \begin{lem}[Context exchange]
%     \label{lem:coex}
%     $\Gamma, x : \tau_1, \Delta, y : \tau_2 \jtype{n,m}{}{t}{\tau}  \implies \Gamma, y : \tau_2, \Delta, x : \tau_1 \jtype{n,m}{}{t}{\tau} $\\
% \end{lem}

% \begin{lem}
% 	\label{lem:sub}
% 	If $\Gamma \jtype{n}{}{t}{\tau}$ and $\gamma \vDash \Gamma$, then $ \cdot \jtype{n}{}{\gamma(t)}{\tau} $\\
% \end{lem}

% \begin{lem}
% 	\label{lem:subext}
% 	If $\Gamma \subseteq \Gamma'$,  and $\Gamma' \jtype{n}{}{t}{\tau}$, then $\exists m. n \leq m$ s.t.  $\Gamma  \jtype{m}{}{t}{\tau} $. \\
% \end{lem}
% %
% %
% %
% %%%%%%%%%%%%%%%%%%%%%%%%%%
% \begin{thm}[Type Safety]
% 	If $\cdot \jtype{n,m}{}{t}{\tau} $ then $ \exists F. t \Downarrow F \land \jtype{n,m}{}{F}{\tau}$
% \end{thm}
% %
% \begin{coro}
% \label{cor:typesafety}
% 	If $ \cdot\jtype{n,m}{}{t}{b} $ then $ \exists T_b. t \Downarrow T_b \land \depth(T_b) \leq n$
% \end{coro}
% %

% \begin{thm}[Normalization] 
% 	If $\cdot\jtype{n,m}{}{t}{\tau} $ then $ \exists F: t \Downarrow F $
% \end{thm}
% We prove two theorems instead.
% \begin{thm}
% If $\gamma(t) \in \llu{\tau}{\epsilon} $, then $\exists F.\eval{ \gamma(t)}{ F}{}$.
% \end{thm}
% %
% %
% \begin{thm}
%  If $\Gamma \jtype{n}{}{t}{\tau}$ and $\gamma \vDash{\Gamma}$, then $\gamma(t) \in \llu{\tau}{\epsilon} $.
% \end{thm} 

% \begin{thm}[Preservation]
% 	If $\cdot\jtype{n}{}{t}{\tau} \land t \Downarrow F$ then $ \jtype{n}{}{F}{\tau} $
% \end{thm}
% %
% %
% \begin{thm} [Substitution]
% 	If $\Gamma \jtype{n_1}{}{t_1}{\tau_1}$  and   $\Delta, x: [\tau_1]_p \jtype{n_2}{}{t_2}{\tau_2}$ then $\max( \Gamma,\Delta) \jtype{\max{(p + n_1, n_2)}}{}{t_2[t_1/x]}{\tau_2} $
% \end{thm}
% %
% %
% \begin{thm} [Substitution] $ $
% \label{thm:sub}
% \begin{enumerate}
%     \item If $\Gamma \jtype{n_1}{}{t_1}{\tau_1}$  and   $\Delta, x: \tau_1 \jtype{n_2}{}{t_2}{\tau_2}$ then $max(\Gamma, \Delta) \jtype{\max{(n_1, n_2)}}{}{t_2[t_1/x]}{\tau_2} $
%     \item If $\Gamma \jtype{n_1}{}{ t_1}{!_p \tau_1}$  and   $\Delta, x: [\tau_1]_p \jtype{n_2}{}{t_2}{\tau_2}$ then $max(\Gamma, \Delta) \jtype{\max{(n_1 + p, n_2)}}{}{t_2[t_1/x]}{\tau_2} $
% \end{enumerate}
% \end{thm}

%
%
\fail{this semantics doesn't work in multi-Round case.
\begin{enumerate}
  \item depth in the multi-round case is variable.
  \item unable to count depth if output of $\delta$ doesn't explicitly affect input of next $\delta$. 
  %
  \\
  %
  In the multi-round case, the $\delta$ result will affect some arguments. These arguments will then affect Database $d$ which will be used in next $\delta$ nested in recursion.
\end{enumerate}
}

%%%%%%%%%%%%%%%%%%%%%%%%%%%%%%%%%%%%%%%%%%%%%%%%%%%%%%%%%%%%%%%%%%%%%%%%%%%%%%%%%%%%%%%%%%%%%%%%%%%%%%
%%%%%%%%%%%%%%%%%%%%%%%%%%%%%%%%%%%%%%%%%%%%%%%%%%%%%%%%%%%%%%%%%%%%%%%%%%%%%%%%%%%%%%%%%%%%%%%%%%%%

% \clearpage
% \section{Attempt 2: Trace-based effect system}
% 
\paragraph{Traces}
A trace $\tr$ is a representation of the big-step derivation of an
expression's evaluation.

The \emph{adaptivity} of a trace $\tr$, $\adap(\tr)$, which
means the maximum number of nested $\eop$s in $\tr$.

The \emph{depth of variable $x$} in trace
$\tr$, written $\ddep{x}(\tr)$, which is the maximum number of
$\eop{}$s in any path leading from the root of $\tr$ to an occurence
of $x$ (at a leaf),.% Technically, $\adap: \mbox{Traces} \to \nat$ and
% $\ddep{x}: \mbox{Traces} \to \natb$. If $x$ does not appear free in
% $\tr$, $\ddep{x}(\tr)$ is $\bot$.


\[\begin{array}{llll}
\mbox{Expr.} & \expr & ::= & x ~|~ \expr_1 \eapp \expr_2 ~|~ {\efix f(x:\type).\expr}
 ~|~ (\expr_1, \expr_2) ~|~ \eprojl(\expr) ~|~ \eprojr(\expr) ~| \\
%
& & & \eif(\expr_1, \expr_2, \expr_3) ~|~
\econst ~|~ \eop(\expr)  ~|~  {\elet  x:q = \expr_1 \ein \expr_2 } ~|~ \enil ~|~  \econs (
      \expr_1, \expr_2) \\
%
\mbox{Value} & \valr & ::= & \econst ~|~
(\efix f(x:\type).\expr, \env) ~|~ (\valr_1, \valr_2) 
    ~|~ \enil ~|~ \econs (\valr_1, \valr_2) | \\
%
\mbox{Environment} & \env & ::= & x_1 \mapsto \valr_1, \ldots, x_n \mapsto \valr_n
\end{array}\]


%%%%%%%%%%%%%%%%%%%%%%%%%%%%%%%%%%%%%%%%%%%%%%%%%%%%%

%%%%%%%%%%%%%%%%%%%%%%%%%%%%%%%%%%%%%%%%%%%%%%%%%%%%%



%
\[\begin{array}{llll}
\mbox{Trace} & \tr & ::= & {(x, \env)} ~|~ \trapp{\tr_1}{\tr_2}{f}{x}{\tr_3} ~|~
{ (\trfix f(x:\type).e, \env) } ~|~ (\tr_1, \tr_2) ~|~ \trprojl(\tr) ~|\\ 
%
& & & \trprojr(\tr) ~|~ \trtrue ~|~ \trfalse ~|~ \trift(\tr_b, \tr_t)
~|~ \triff(\tr_b, \tr_f) ~|~ \trconst ~|~ \trop(\tr) \\
\end{array}\]



\begin{figure}[b]
\begin{mathpar}
   { \inferrule{ }{\env, x \bigstepT \env(x), (x, \env ) }  }
  %
  \and
  %
  \inferrule{ }{\env, \econst \bigstepT \econst, \trconst}
  % %
 %  \and
 %  %
 %  \inferrule{ }{\env, \etrue \bigstep \etrue, \trtrue}
 %  %
 %  \and
 %  %
 %  \inferrule{ }{\env, \efalse \bigstep \efalse, \trfalse}
 %  %
 %  \and
 %  { \inferrule{  \env, \expr \bigstep \econst, \tr }{\env, \bernoulli \eapp \expr \bigstep \econst,
 %      \bernoulli (\tr)
 %    } }
 %  \and
 % \inferrule{ \env, \expr_1 \bigstep \econst, \tr_1 \\ \env, \expr_2 \bigstep \econst, \tr_2  }{\env, \uniform \eapp \expr_1 \eapp
 %      \expr_2\bigstep \econst, \uniform(\tr_1,\tr_2)  } 
 %  \and
  %
  { \inferrule{
  }{
    \env, \efix f(x:\type). \expr \bigstepT (\efix f(:\type).\expr, \env),
    (\trfix f(x:\type).\expr, \env)
  }
}
  %
  \and
  %
  \inferrule{
    \env, \expr_1 \bigstepT \valr_1, \tr_1 \\
    { \valr_1 = (\efix f(x:\type).\expr, \env')} \\\\
    \env, \expr_2 \bigstepT \valr_2, \tr_2 \\
    \env'[f \mapsto \valr_1, x \mapsto \valr_2], \expr \bigstepT \valr, \tr
  }{
    \env, \expr_1 \eapp \expr_2 \bigstepT \valr, \trapp{\tr_1}{\tr_2}{f}{x}{\tr}
  }
  %
  \and
  %
  \inferrule{
    \env, \expr_1 \bigstepT \valr_1, \tr_1 \\
    \env, \expr_2 \bigstepT \valr_2, \tr_2
  }{
    \env, (\expr_1, \expr_2) \bigstepT (\valr_1, \valr_2), (\tr_1, \tr_2)
  }
  %
  \and
  %
  \inferrule{
    \env, \expr \bigstepT (\valr_1, \valr_2), \tr
  }{
    \env, \eprojl(\expr) \bigstepT \valr_1, \trprojl(\tr)
  }
  %
  \and
  %
  \inferrule{
    \env, \expr \bigstepT (\valr_1, \valr_2), \tr
  }{
    \env, \eprojr(\expr) \bigstepT \valr_2, \trprojr(\tr)
  }
  %
  \and
  %
  \inferrule{
    \env, \expr \bigstepT \etrue, \tr \\
    \env, \expr_1 \bigstepT \valr, \tr_1
  }{
    \env, \eif(\expr, \expr_1, \expr_2) \bigstepT \valr, \trift(\tr, \tr_1)
  }
  %
  \and
  %
  \inferrule{
    \env, \expr \bigstepT \efalse, \tr \\
    \env, \expr_2 \bigstepT \valr, \tr_2
  }{
    \env, \eif(\expr, \expr_1, \expr_2) \bigstepT \valr, \triff(\tr, \tr_2)
  }
  %
  \and
  %
  \inferrule{
    \env, \expr \bigstepT \valr, \tr \\
    \eop{}(\valr) = \valr'
  }{
    \env, \eop(\expr) \bigstepT \valr', \trop(\tr)
  }
% %
% \and
% %
%   \inferrule{
% }
% { \env, \enil \bigstep \enil, \trnil }
% %
% \and
% %
% \inferrule{
% \env, \expr_1 \bigstep \valr_1, \tr_1 \\
% \env, \expr_2 \bigstep \valr_2, \tr_2
% }
% { \env, \econs (\expr_1, \expr_2)  \bigstep \econs (\valr_1, \valr_2),
%   \trcons(\tr_1, \tr_2)
% }
% %
% \and
% %
% \inferrule{
%   \env, \expr_1 \bigstep \valr_1, \tr_1 \\
%   \env[x \mapsto \valr_1] , \expr_2 \bigstep \valr, \tr_2
% }
% {\env, \elet x;q = \expr_1 \ein \expr_2 \bigstep \valr, \trlet (x,
%   \tr_1, \tr_2) }
% %
% \\\\
% %
% \boxed{\color{red}
% \inferrule
% {
%   \empty
% }
% {
%   \env, \eilam \expr \bigstep (\eilam \expr, \env), (\eilam \expr , \env)
% }
% }
% %
% \and
% %
% \boxed{\color{red}
% \inferrule{
%   \env, \expr \bigstep (\eilam \expr', \env'), \tr_1 \\
%   \env, \expr' \bigstep \valr, \tr_2
% }
% {\env, \expr [] \bigstep \valr, \triapp{\tr_1}{\tr_2} }

% }
\end{mathpar}
  \caption{Big-step semantics with provenance}
  \label{fig:big-step}
\end{figure}


\begin{figure}
  \framebox{$\adap: \mbox{Traces} \to \nat$}
  \begin{mathpar}
    \begin{array}{lcl}
       { \adap( (x,\env) )} & = & 0 \\
      %
      \adap(\trapp{\tr_1}{\tr_2}{f}{x}{\tr_3}) & = &
      \adap(\tr_1) + \max (\adap(\tr_3), \adap(\tr_2) + \ddep{x}(\tr_3))\\
      %
       {\adap( (\trfix f(x:\type).\expr, \env)  ) } & = & 0 \\
      %
      \adap((\tr_1, \tr_2)) & = & \max(\adap(\tr_1), \adap(\tr_2)) \\
      %
      \adap(\trprojl(\tr)) & = & \adap(\tr) \\
      %
      \adap(\trprojr(\tr)) & = & \adap(\tr) \\
      %
      \adap(\trtrue) & = & 0 \\
      %
      \adap(\trfalse) & = & 0 \\
      %
      \adap(\trift(\tr_b, \tr_t)) & = & \adap(\tr_b) + \adap(\tr_t) \\
      %
      \adap(\triff(\tr_b, \tr_f)) & = & \adap(\tr_b) + \adap(\tr_f) \\
      %
      \adap(\trconst) & = & 0 \\
      %
      \adap(\trop(\tr)) & = & { 1 + \adap(\tr) } \\
           & &       {  +  \textsf{MAX}_{\valr \in \type} \Big(
                              \max \big(\adap(\tr_3 (\valr) ),
                              \ddep{x}(\tr_3(\valr)) \big) \Big) } \\
      &\mathsf{where}&  { \valr_1 = (\efix f(x: \type). \expr, \env ) =
                       \mathsf{extract}(\tr) } \\
 & & { \conj  \env[f \mapsto
                       \valr_1, x \mapsto \valr], \expr \bigstep
                       \valr', \tr_3(\valr) } \\ 
      \end{array}
  \end{mathpar}
  %
  \framebox{$\ddep{x}: \mbox{Traces} \to \natb$}
  \begin{mathpar}
    \begin{array}{lcl}
       { \ddep{x}( ( y, \env )) } & = &
      \left\lbrace
      \begin{array}{ll}
        0 & \mbox{if } x = y \\
        \bot & \mbox{if } x \neq y
      \end{array}
      \right.\\
      %
      \ddep{x}(\trapp{\tr_1}{\tr_2}{f}{y}{\tr_3}) & = & \max(\ddep{x}(\tr_1), \\
      & & \adap(\tr_1) + \max(\ddep{x}(\tr_3), \ddep{x}(\tr_2) + \ddep{y}(\tr_3))) \\
      %
      { \ddep{x}(  (\trfix f(y:\type).\expr,\env)  )  }& = & \bot \\
      %
      \ddep{x}((\tr_1, \tr_2)) & = & \max(\ddep{x}(\tr_1), \ddep{x}(\tr_2)) \\
      %
      \ddep{x}(\trprojl(\tr)) & = & \ddep{x}(\tr) \\
      %
      \ddep{x}(\trprojr(\tr)) & = & \ddep{x}(\tr) \\
      %
      \ddep{x}(\trtrue) & = & \bot \\
      %
      \ddep{x}(\trfalse) & = & \bot \\
      %
      \ddep{x}(\trift(\tr_b, \tr_t)) & = & \max(\ddep{x}(\tr_b), \adap(\tr_b) + \ddep{x}(\tr_t)) \\
      %
      \ddep{x}(\trift(\tr_b, \tr_f)) & = & \max(\ddep{x}(\tr_b), \adap(\tr_b) + \ddep{x}(\tr_f)) \\
      %
      \ddep{x}(\trconst) & = & \bot \\
      %
      \ddep{x}(\trop(\tr)) & = & 1 +  \max(\ddep{x}(\tr),  \\
      & &  \adap(\tr) + \textsf{MAX}_{\valr \in \type} \Big(
          \max(\ddep{x}(\tr_3(\valr)), \bot )   \Big ) ) \\  
 &\mathsf{where}&  { \valr_1 = (\efix f(x: \type). \expr, \env ) =
                       \mathsf{extract}(\tr) } \\
 & & { \conj  \env[f \mapsto
                       \valr_1, x \mapsto \valr], \expr \bigstep
                       \valr', \tr_3(\valr) } \\ 
    \end{array}
  \end{mathpar}
  \caption{Adaptivity of a trace and depth of variable $x$ in a trace}
  \label{fig:adap}
\end{figure}

\subsection{Challenge (Couterexample) in this setting}

\begin{mainitem}
\item[1] adaptivity is not precise. The definition of the max number
  of nested $\delta$s is not very accurate, especially the way it handles the
  application. Another way of understanding adaptivity is not only the
  occurence of $\delta$, but the times the program accesses the
  database ($\delta$).  For instance, $\lambda
  x. ( \eif \, (x < 10 ) \ethen \, x \eelse x  ) \, \delta(v) $ and $
   ( \eif \, ( \delta(v) < 10 ) \ethen \, \delta(v) \eelse \delta(v)  )
   $. should distinguish the two definition of adaptivity. \\
\item[2] This operational semantics has trouble to give the reasonable trace to
  nested lambda. Consider $\lambda x. \lambda y. (x y)$.  and its corresponding 
  application.  consider $ (  \lambda x. \lambda y. (x y) ) \;  \delta(v)  \; \delta(v) $ , its trace equals to 0 \\
\item[3] The typing system is not consistent for $\alpha$ renaming.
  \end{mainitem}


% \clearpage
% \section{Attempt 3: Linear type based 2}
% 
\[\begin{array}{llll}
\mbox{Expr.} & \expr & ::= & x ~|~ \expr_1 \eapp \expr_2 
 ~|~ \lambda x. \expr 
    \\
             & & &  \etrue ~|~ \efalse ~|~
  \eif  \expr \ethen \expr_2 \eelse \expr_3 ~|~
\econst ~|~ \eop(\expr)  \\
\mbox{Environment} & \env & ::= & x_1 \mapsto (\valr_1, \adapt_1), \ldots, x_n \mapsto (\valr_n,\adapt_n)
\end{array}\]

\[
\begin{array}{llll}
  \mbox{Index Term} & \idx, \nnatA & ::= &     i ~|~ n \\
 %                                  - \idx_2 ~|~ \smax{\idx_1}{\idx_2}\\
%                                  \mbox{Sort} & S & ::= & \nat \\
  \mbox{Linear type} & \type &::=  &  \ltype \lto{\nnatA} \type ~|~
                                     \tbase ~|~ \tbool \\
  \mbox{Nonlinear Type} & \ltype & ::= & \bang{\idx} \type   \\
  \mbox{Typing context } & \Gamma & ::= & x_1 : \ltype_1, \ldots,
                                          x_n : \ltype_n
\end{array}
\]


\begin{figure}[h]
  \begin{mathpar}
    \inferrule{
    }{
     \valr, \env \bigstep{0} \valr, \env} ~\textsf{val}
   %  \and
   % \inferrule{  \mathsf{fetch} (\env,x)  =  (\valr, \adapt)  }{x, \env
   %   \bigstep{\adapt} \valr, \env }~\textsf{var}
   \and
   %
     \inferrule{  \env(x)  =  (\valr, \env_1,  \adapt)  }{x,
       \env  \bigstep{\adapt} \valr, \env_1 }~\textsf{var}
     %
 %
   \and
  %
   \inferrule{ }{\econst , \env \bigstep{0} \econst, \env}~\textsf{const}
   %
   \and
   %
 \inferrule{
  }{
    \lambda x. \expr, \env
    \bigstep{0} \lambda x.\expr, \env
  }~\textsf{lambda}
  %
  \and
  %
  \inferrule{
    \expr_1, \env_1 \bigstep{\adapt_1} \lambda x.\expr , \env_1' \\
    %\forall x_i \in \dom(\env_1 \cap \env_2).  \fresh \eapp x_i' \\
     \expr_2, \env_2 \bigstep{\adapt_2} \valr_2 , \env_2' \\
    \fresh \eapp x' \\
    \expr[x'/x], \env_1'[ x'  \to (\valr_2, \env_2', \adapt_2  ) ] 
    \bigstep{\adapt_3} \valr, \env_3
  }{
     \expr_1 \eapp \expr_2 , (\env_1 \uplus \env_2)\bigstep{\adapt_1+\adapt_3} \valr, \env_3
  }~\textsf{app}
 %
  \and
  % %
  \and
  %
  \inferrule{
    \expr , \env \bigstep{\adapt} \valr , \env_1 \\
    \eop{}(\valr\env) = \valr' \\
    FV(\valr')=\emptyset
  }{
    \eop(\expr), \env \bigstep{\adapt +1} \valr,  \env_1
  }~\textsf{delta}
% %
\and
 %
\wq{
 \inferrule{
   \expr, \env \bigstep{\adapt} \efalse, \env'
   \\
   \expr_2 , \env \bigstep{\adapt_2} \valr_2, \env_2
  }{
    \eif \expr \ethen \expr_1 \eelse \expr_2 , \env \bigstep{\adapt +
      \adapt_2 } \valr_2, \env_2
  }~\textsf{if-f}
}
\and
\wq{
 \inferrule{
   \expr , \env \bigstep{\adapt} \etrue, \env'
   \\
    \expr_1, \env \bigstep{\adapt_1} \valr_1, \env_1
  }{
    \eif \expr \ethen \expr_1 \eelse \expr_2 , \env \bigstep{\adapt +
      \adapt_1 } \valr_1, \env_1
  }~\textsf{if-t}
}
% %
%   }
  \\\\
  \begin{array}{llll}
    \env_1 \uplus \emptyset & \triangleq & \env_1 &\\
     \emptyset \uplus \env_2 & \triangleq & \env_2 &\\
  
  \end{array}
\end{mathpar}
  \caption{Big-step semantics}
  \label{fig:semantics1}
\end{figure}

 \begin{figure}[h]
  \begin{mathpar}
    \inferrule{
     \env ( x ) = (\valr, \env', \adapt)
      \\
      \tvdash{\nnatA} ( \valr, \env') : \type
          }{
     \tvdash{\adapt + \nnatA}   ( x, \env):  \type
    }~\textbf{C-Ax}
    %
    \and
    %
    \inferrule{
    }{
      \tvdash{0} (  c, \env) : \tbase
    }~\textbf{C-const}
   
    \and
    %
    \inferrule{
      \tvdash{\nnatA' } ( \valr', \theta') : \type_1
      \\
      \fresh\eapp  x' ~~ \forall \adapt'
      \\
      \tvdash{ S+ \idx \times (\adapt' + \nnatA' ) +\nnatA }
     ( \expr[x'/x], \env[x' \to (\valr', \theta', R')]      ) :
     \type_2
    }{
     \tvdash{S} (  \lambda x. \expr, \env )  : \bang{\idx} \type_1
      \lto{\nnatA} \type_2
    }~\textbf{C-lambda}
    \and
    %
    \inferrule{
       \tvdash{\nnatA_1} ( \expr_1, \env_1) :  \bang{\idx} \type_1
      \lto{\nnatA} \type_2      \\
      \tvdash{\nnatA_2} ( \expr_2, \env_2 ): \type_1
    }{
       \tvdash{    \nnatA_1 +
        \idx \times \nnatA_2 + \nnatA    } (  \expr_1 \eapp \expr_2, \env_1 \uplus \env_2   ) : \type_2
    }~\textbf{C-app}
    %
    \and
    %
    \inferrule{
      \tvdash{\nnatA} (\expr, \env) :  \tbase
   }{  \tvdash{1+\nnatA} (\delta(\expr) , \env ) : \tbase
    }~\textbf{C-delta}
    \\\\
    \begin{array}{lll}
       \theta  & \triangleq (x_i \to (\valr_i, \env_i, R_i)) & i \in 
       \mathbb{N}\\
      (x_i : \bang{ \idx }\type_i), \Gamma \vDash (x_i \to (\valr_i, \env_i, R_i))
      \uplus \theta & \triangleq ~~~\tvdash { \_ } (\valr_i, \env_i)
                                          :  \type_i  &\conj
                                   \Gamma \vDash \theta
      \end{array}
  \end{mathpar}
  \caption{Typing rules, configure}
  \label{fig:configure-rules}
\end{figure}

\begin{figure}[h]
  \begin{mathpar}
    \begin{array}{lll}
      \lrv{\tbase} & = & \{  ( \econst, \env,  \nnatA)  \} \\
      %
      % \lrv{\type_1 \times \type_2} & = & \{(\valr_1, \valr_2) ~|~ \valr_1 \in \lrv{\type_1} \conj \valr_2 \in \lrv{\type_2} \}\\
      %
      \lrv{\bang{k} \type } & = & \{  ( \valr, \env,   \nnatA) |  (\valr, \env,
                                   \nnatA ) \in \lrv{\type}  \} \\
      %
      \lrv{ \bang{k} \type_1 \lto{\nnatA} \type_2    } & = &
      \{( \lambda x.\expr, \env,  \nnatA_1) ~|~ \forall \valr', \env',
      \nnatA'. ( \valr',\env',  \nnatA') \in
      \lrv{ \bang{k} \type_1 }.\\
      & & 
          \implies   \fresh \eapp x' \land \\
      & & \forall \adapt. ( \expr[x'/x], \env[x' \mapsto (\valr', \env', \adapt )] ) \in
          \lre{    }{ \nnatA_1+\nnatA+ \idx \times (\adapt + \nnatA') }{\type_2}     \} \\
      %
      \\
      %
      \lre{}{\nnatA}{\type} & = & \{  ( \expr, \env) ~|~  ( \expr , \env
                                  \bigstep{\adapt}  \valr, \env' ) \\
      & & ~~~~~~~~~~~~~\implies \adapt \leq \nnatA \conj 
     ( \valr, \env', \nnatA- \adapt) \in \lrv{\type})
      \}
    \end{array}
  \end{mathpar}
  \caption{Logical relation without step-indexing}
  \label{fig:lr:non-step}
\end{figure}

\paragraph{Typable Approach By Weihao}
\[
\begin{array}{ll}
 F(\expr, \phi ) & where \eapp ~~ \phi(x_i) = (\idx_i, \adapt_i, \nnatA_i ) \\
   F(x,\phi) & = \sum_{x_i \in \fv{x}  } \idx_i \times (\adapt_i+ \nnatA_i)  \\
F(\lambda x. \expr ,  \phi  ) & =  \sum_{x_i \in \fv{\lambda x.\expr}  } \idx_i \times (\adapt_i+ \nnatA_i)   \\  %\sum_{x_i \in \fv{\lambda x.\expr} } k_i \times R_i  \\
F(\delta(\expr) , \phi ) & = \sum_{x_i \in \fv{\delta(\expr)} } \idx_i \times (\adapt_i+ \nnatA_i)  \\
F(c, \phi ) & = 0  \\
F(\expr_1 \eapp \expr_2, \phi ) & = F(\expr_1, \phi ) +
                                  F(\expr_2,\phi ) \\
F(\eif \expr \ethen \expr_1 \eelse \expr_2, \phi ) & = F(\expr, \phi) + \max(F(\expr_1, \phi),  F(\expr_2, \phi)   )  
\end{array} 
\]

\begin{defn}[Typable]
  \label{typable}
  A closure $( \expr, [ x_1 \to (\valr_1 ,  \env_1 , \adapt_1 ) , \ldots, x_i \to (\valr_i, \env_i, \adapt_i )] )$ is typable with type $\type$ and adaptivity $J$ if exists $k_i$\\
  \[
     x_1 : \bang{\idx_1} \type_1, \ldots, \bang{\idx_i} \type_i 
     \tvdash{\nnatA}  \expr : \type  \]
   and each closure $(\valr_i, \env_i)$  is also typable with type $\bang{\idx_i} \type_i$ and adaptivity $\nnatA_i$, $ \phi = [x_1
     \to (\idx_1, \adapt_1, \nnatA_1), \ldots,  x_i \to (\idx_i, \adapt_i,
     \nnatA_i)  ] $,  $J = \nnatA + F( \expr, \phi ) $.
 \end{defn}

 \begin{defn}[ClosedClosure]
  \label{closure}
   A closure $(\expr, \env)$ is closed if $\fv{\expr} \subseteq \dom(\env)$. 
 \end{defn}

% \begin{lem}[ClosureTypable ]
%   \label{ct}
%    If a closure $(\expr, \env)$ is closed, then there exists $\type$ and $J$ so that $(\expr, \env)$ is typable with $\type$ and $J$.
%    \end{lem}
 

\begin{lem}[programTypable]
  \label{proglemma}
   If $ \tvdash{\nnatA}   \expr : \type $, then $(
     \expr, \emptyset ) $ is typable with $\type$ and adaptivity $\nnatA$. 
   \end{lem}

   \begin{lem}[TypableMono]
     \label{tmono}
     If a closure is $D$ is typable with $\type$ and $\nnatA$, and $\nnatA \leq \nnatA'$, then
     D is typable with $\type$ and $\nnatA'$.
    \end{lem} 

   
\begin{lem}[TypableSoundness]
  \label{tsound}
  If a closure $D$ is typable with $\type$ and $J$, and $D \bigstep{\adapt} E$, then
    closure $E$ is typable with $\type$ and adaptivity $J - \adapt$. 
   \end{lem}
\paragraph{Typable Approach By Marco}
\begin{defn}[Typable Closures]
  \label{def:typable}
Let $\env=[ x_1 \to (\valr_1 ,  \env_1 , R_1 ) ,
  \ldots, x_n \to (\valr_n, \env_n, R_n )]$. 
  The closure $( \expr, \env)$ is typable with
  type $\type$ and adaptivity $J$ if:
\begin{enumerate}\item  
     $x_1 : \bang{k_1} \type_1, \ldots, \bang{k_i} \type_i 
     \tvdash{Z}  \expr : \type$, for some types $\bang{k_i}
   \type_i$ for $(1\leq i\leq n)$, 
\item each closure $(\valr_i, \env_i)$ for $(1\leq i\leq n)$ is typable with type
  $\bang{k_i} \type_i$ and adaptivity $Z_i$,
\item $J = Z + \sum_{(v_i,\theta_i,S_i)\in\theta} k_i \times (R_i
  +Z_i)$.
\end{enumerate}
 \end{defn}
To justify why we chose $\sum$ in the third clause above it is worth
to consider the following configuration:
$$
[x\mapsto (\lambda u.\lambda w.\delta(u)+\delta(w),[\,],0)
,y:\mapsto (v,[\,],2) ], x\, y\, y
$$
   
\begin{lem}[Soundness]
  \label{tsound}
  If a closure $D$ is typable with type $\type$ and adaptivity $J$, and $D \bigstep{R} E$, then
    the closure $E$ is typable with type $\type$ and adaptivity $I$,
    where $I+R\leq J$. 
   \end{lem}
\fail{Soundness is unable to be proven:
%
\\
%
In the semantics app rule: $R_1 + R_3$ adaptivity is numerically added.
%
\\
% 
In the typing-configuration (typable closure): $J = Z + \sum_{(v_i,\theta_i,S_i)\in\theta} k_i \times (R_i +Z_i)$, adaptivity adding is by $\max$. 
  \\
The adpativity in typing rule cannot bound the adaptivity in semantics. We need to have a better understanding on the adptivity flow
}




\clearpage
\section{Attemp4: Call by need style}

\[\begin{array}{llll}
\mbox{Expr.} & \expr & ::= & \econst ~|~ x ~|~ \expr_1 \eapp \expr_2 
 ~|~ \lambda x. \expr  ~|~ \eletx{x}{e_1}{e_2} ~|~
                             \pair{\expr_1}{\expr_2} ~|~
                             \eprojl{\expr} ~|~ \eprojr{\expr}
    \\
             & & &  \etrue ~|~ \efalse ~|~
                   \eif (\expr, \expr_2 , \expr_3) ~|~ \eop(\expr)   \\
    \mbox{Value.} & v & ::= & \econst ~|~   \lambda x. \expr  ~|~ \pair{\valr_1}{\valr_2}  \\
\mbox{Environment} & \env & ::= & \epsilon ~|~ \env, [ x \mapsto (\valr, \adapt)]
\end{array}\]

% \[
% \begin{array}{llll}
%   \mbox{Index Term} & \idx, \nnatA & ::= &     i ~|~ n \\
%  %                                  - \idx_2 ~|~ \smax{\idx_1}{\idx_2}\\
% %                                  \mbox{Sort} & S & ::= & \nat \\
%   \mbox{Linear type} & \type &::=  &  \ltype \lto{\nnatA} \type ~|~
%                                      \tbase ~|~ \tbool \\
%   \mbox{Nonlinear Type} & \ltype & ::= & \bang{\idx} \type   \\
%   \mbox{Typing context } & \Gamma & ::= & x_1 : \ltype_1, \ldots,
%                                           x_n : \ltype_n
% \end{array}
% \]


\begin{figure}[h]
  \begin{mathpar}
    \inferrule{
    }{
     \env, \valr \bigstep{0} \env, \valr } ~\textsf{val}
   %  \and
   % \inferrule{  \mathsf{fetch} (\env,x)  =  (\valr, \adapt)  }{x, \env
   %   \bigstep{\adapt} \valr, \env }~\textsf{var}
   \and
   %
     \inferrule{   }{
       \env[x -> (\valr,\adapt)] , x   \bigstep{\adapt}  \env[x ->
       (\valr, 0)], \valr }~\textsf{var}
     %
 %
   \and
  %
   \inferrule{ }{ \env,  \econst \bigstep{0} \env, \econst}~\textsf{const}
   %
   \and
   %
 \inferrule{
  }{
    \env, \lambda x. \expr
    \bigstep{0} \env,  \lambda x.\expr
  }~\textsf{lambda}
  %
  \and
  %
  \inferrule{
    \env, \expr_1 \bigstep{\adapt_1} \env_1, \lambda x.\expr  \\
     \env_1, \expr_2 \bigstep{\adapt_2} \env_2, \valr_2  \\
    \fresh \eapp x' \\
   \env_2[ x'  \to (\valr_2, \adapt_2  ) ] , \expr[x'/x], 
    \bigstep{\adapt_3}  \env_3, \valr_3
  }{
     \env, \expr_1 \eapp \expr_2 \bigstep{\adapt_1+\adapt_3}  \env_3, \valr_3
   }~\textsf{app}
    %
  \and
  %
  \inferrule{
    \env, \expr_1 \bigstep{\adapt_1} \env_1, \valr_1  \\
    \fresh \eapp x' \\
   \env_1[ x'  \to (\valr_1, \adapt_1  ) ] , \expr_2[x'/x], 
    \bigstep{\adapt_2}  \env_2, \valr_2
  }{
     \env, \eletx{x}{\expr_1}{\expr_2} \bigstep{\adapt_2}  \env_2, \valr_2
  }~\textsf{let}
 %
  \and
  %
  \inferrule{
    \env, \expr  \bigstep{\adapt} \env_1,  \lambda x. \expr_1  \\
    \eop(\lambda x. \expr' )= \valr_1 
  }{
   \env,  \eop(\expr) \bigstep{\adapt +1}   \env_1, \valr_1
  }~\textsf{delta}
% %
\and
 %
 \inferrule{
  \env, \expr_1 \bigstep{\adapt_1} \env_1, \efalse
   \\
   \env_1, \expr_2  \bigstep{\adapt_2} \env_2, \valr_2
  }{
   \env, \eif (\expr_1 ,\expr_2 , \expr_3) \bigstep{\adapt_1 +
      \adapt_2 } \env_2,  \valr_2
  }~\textsf{if-f}
\and
 \inferrule{
  \env, \expr_1 \bigstep{\adapt_1} \env_1, \etrue
   \\
   \env_1, \expr_3  \bigstep{\adapt_3} \env_3, \valr_3
  }{
   \env, \eif (\expr_1 ,\expr_2 , \expr_3) \bigstep{\adapt_1 +
      \adapt_2 } \env_3,  \valr_3
  }~\textsf{if-t}
  % %
 %
  \and
  %
  \inferrule{
    \env, \expr_1  \bigstep{\adapt_1} \env_1,  \valr_1 \\
     \env, \expr_2  \bigstep{\adapt_2} \env_2,  \valr_2 \\
  }{
   \env, \pair{\expr_1}{\expr_2} \bigstep{ \max(\adapt_1, \adapt_2)}
   \env_1 \uplus \env_2 ,  \pair{ \valr_1}{\valr_2}
 }~\textsf{prod}
 %
  \and
  %
  \inferrule{
    \env, \expr  \bigstep{\adapt} \env_1,  (\valr_1, \valr_2) 
  }{
   \env, \eprojl{(\expr)} \bigstep{ \adapt }
   \env_1,  \valr_1
 }~\textsf{projl}
 %
  \and
  %
  \inferrule{
    \env, \expr  \bigstep{\adapt} \env_1,  (\valr_1, \valr_2) 
  }{
   \env, \eprojr{(\expr)} \bigstep{ \adapt }
   \env_1,  \valr_2
 }~\textsf{projr}
  %
  \and
  %
  \inferrule{
    \env, \expr_1  \bigstep{\adapt_1} \env_1,  \valr_1 \\
     \env, \expr_2  \bigstep{\adapt_2} \env_2,  \valr_2 \\
  }{
   \env, \expr_1 < \expr_2 \bigstep{ \max(\adapt_1, \adapt_2)}
   \env_1 \uplus \env_2 ,   \valr_1 < \valr_2
  }~\textsf{bop}
%   }
  \\\\
  \begin{array}{llll}
    \env_1 \uplus \epsilon & \triangleq & \env_1 &\\
    \epsilon \uplus \env_2 & \triangleq & \env_2 &\\
     \env_1[x \to (\valr, \adapt)] \uplus \env_2[x \to (\valr, 0)]  &
                                                                 \triangleq
                                         & (\env_1 \uplus \env_2) [x\to
                                           (\valr, 0)] &\\
     \env_1[x \to (\valr, 0)] \uplus \env_2[x \to (\valr, \adapt )]  &
                                                                 \triangleq
                                         & (\env_1 \uplus \env_2)[x \to
                                           (\valr, 0)] &\\
    \env_1[x \to (\valr, \adapt)] \uplus \env_2  &
                                                                 \triangleq
                                         & (\env_1 \uplus \env_2) [x\to
                                           (\valr, \adapt)] & x
                                                              \not\in dom(\env_2) \\
     \env_1 \uplus \env_2[x \to (\valr, \adapt )]  &
                                                                 \triangleq
                                         & (\env_1 \uplus \env_2)[x \to
                                           (\valr, \adapt)] & x
                                                              \not\in dom(\env_1)\\
  \end{array}
\end{mathpar}
  \caption{Big-step semantics}
  \label{fig:semantics}
\end{figure}

%  \begin{figure}[h]
%   \begin{mathpar}
%     \inferrule{
%      \env ( x ) = (\valr, \env', \adapt)
%       \\
%       \tvdash{\nnatA} ( \valr, \env') : \type
%           }{
%      \tvdash{\adapt + \nnatA}   ( x, \env):  \type
%     }~\textbf{C-Ax}
%     %
%     \and
%     %
%     \inferrule{
%     }{
%       \tvdash{0} (  c, \env) : \tbase
%     }~\textbf{C-const}
   
%     \and
%     %
%     \inferrule{
%       \tvdash{\nnatA' } ( \valr', \theta') : \type_1
%       \\
%       \fresh\eapp  x' ~~ \forall \adapt'
%       \\
%       \tvdash{ S+ \idx \times (\adapt' + \nnatA' ) +\nnatA }
%      ( \expr[x'/x], \env[x' \to (\valr', \theta', R')]      ) :
%      \type_2
%     }{
%      \tvdash{S} (  \lambda x. \expr, \env )  : \bang{\idx} \type_1
%       \lto{\nnatA} \type_2
%     }~\textbf{C-lambda}
%     \and
%     %
%     \inferrule{
%        \tvdash{\nnatA_1} ( \expr_1, \env_1) :  \bang{\idx} \type_1
%       \lto{\nnatA} \type_2      \\
%       \tvdash{\nnatA_2} ( \expr_2, \env_2 ): \type_1
%     }{
%        \tvdash{    \nnatA_1 +
%         \idx \times \nnatA_2 + \nnatA    } (  \expr_1 \eapp \expr_2, \env_1 \uplus \env_2   ) : \type_2
%     }~\textbf{C-app}
%     %
%     \and
%     %
%     \inferrule{
%       \tvdash{\nnatA} (\expr, \env) :  \tbase
%    }{  \tvdash{1+\nnatA} (\delta(\expr) , \env ) : \tbase
%     }~\textbf{C-delta}
%     \\\\
%     \begin{array}{lll}
%        \theta  & \triangleq (x_i \to (\valr_i, \env_i, R_i)) & i \in 
%        \mathbb{N}\\
%       (x_i : \bang{ \idx }\type_i), \Gamma \vDash (x_i \to (\valr_i, \env_i, R_i))
%       \uplus \theta & \triangleq ~~~\tvdash { \_ } (\valr_i, \env_i)
%                                           :  \type_i  &\conj
%                                    \Gamma \vDash \theta
%       \end{array}
%   \end{mathpar}
%   \caption{Typing rules, configure}
%   \label{fig:configure-rules}
% \end{figure}

% \begin{figure}[h]
%   \begin{mathpar}
%     \begin{array}{lll}
%       \lrv{\tbase} & = & \{  ( \econst, \env,  \nnatA)  \} \\
%       %
%       % \lrv{\type_1 \times \type_2} & = & \{(\valr_1, \valr_2) ~|~ \valr_1 \in \lrv{\type_1} \conj \valr_2 \in \lrv{\type_2} \}\\
%       %
%       \lrv{\bang{k} \type } & = & \{  ( \valr, \env,   \nnatA) |  (\valr, \env,
%                                    \nnatA ) \in \lrv{\type}  \} \\
%       %
%       \lrv{ \bang{k} \type_1 \lto{\nnatA} \type_2    } & = &
%       \{( \lambda x.\expr, \env,  \nnatA_1) ~|~ \forall \valr', \env',
%       \nnatA'. ( \valr',\env',  \nnatA') \in
%       \lrv{ \bang{k} \type_1 }.\\
%       & & 
%           \implies   \fresh \eapp x' \land \\
%       & & \forall \adapt. ( \expr[x'/x], \env[x' \mapsto (\valr', \env', \adapt )] ) \in
%           \lre{    }{ \nnatA_1+\nnatA+ \idx \times (\adapt + \nnatA') }{\type_2}     \} \\
%       %
%       \\
%       %
%       \lre{}{\nnatA}{\type} & = & \{  ( \expr, \env) ~|~  ( \expr , \env
%                                   \bigstep{\adapt}  \valr, \env' ) \\
%       & & ~~~~~~~~~~~~~\implies \adapt \leq \nnatA \conj 
%      ( \valr, \env', \nnatA- \adapt) \in \lrv{\type})
%       \}
%     \end{array}
%   \end{mathpar}
%   \caption{Logical relation without step-indexing}
%   \label{fig:lr:non-step}
% \end{figure}

\subsection{Example}
\begin{enumerate}
\item{ [], $\lambda x. (x, x) \eapp \eop{(\lambda z. \expr )} \bigstep{ 0 + 1 } [x' \to
    (v', 0)], v'  $} where $v' = \eop{( \lambda z. \expr)}$.
\item{ [], $\lambda x. \lambda y. (x, y) \eapp \eop{(\lambda z. \expr
      ) \eapp v} $}
\item{ $[x \to (\valr_x, 1), y \to (\valr_y, 2)], \big( (x , \delta(y) ) ,
    (\delta(x), y) \big) $  }
\item{ $[x \to (\valr_x, 1), y \to (\valr_y, 2)], \eif \big( \delta(x),
    \delta(y) , y  \big) $  }
\item{ $[x \to (\valr_x, 1), y \to (\valr_y, 2)], \eif \big( \delta(y),
    \delta(y) , x  \big) $  }
 \item{ [], $\lambda x. \lambda y. \eif \big(  x < y , x+y ,
     \delta(\lambda m. m+x )  \big) \eapp \eop{(\lambda z_1. \expr_1 ) \eapp \delta( \lambda z_2. \expr_2 ) } $}
\wq{\item{ $[x \to (\valr_x, 5), y \to (\valr_y, 6)],  \eletx{z}{  \lambda m. x+y+m}{\delta(z)} $} }
\end{enumerate} 

% \paragraph{Typable Approach By Weihao}
% \[
% \begin{array}{ll}
%  F(\expr, \phi ) & where \eapp ~~ \phi(x_i) = (\idx_i, \adapt_i, \nnatA_i ) \\
%    F(x,\phi) & = \sum_{x_i \in \fv{x}  } \idx_i \times (\adapt_i+ \nnatA_i)  \\
% F(\lambda x. \expr ,  \phi  ) & =  \sum_{x_i \in \fv{\lambda x.\expr}  } \idx_i \times (\adapt_i+ \nnatA_i)   \\  %\sum_{x_i \in \fv{\lambda x.\expr} } k_i \times R_i  \\
% F(\delta(\expr) , \phi ) & = \sum_{x_i \in \fv{\delta(\expr)} } \idx_i \times (\adapt_i+ \nnatA_i)  \\
% F(c, \phi ) & = 0  \\
% F(\expr_1 \eapp \expr_2, \phi ) & = F(\expr_1, \phi ) +
%                                   F(\expr_2,\phi ) \\
% F(\eif \expr \ethen \expr_1 \eelse \expr_2, \phi ) & = F(\expr, \phi) + \max(F(\expr_1, \phi),  F(\expr_2, \phi)   )  
% \end{array} 
% \]

% \begin{defn}[Typable]
%   \label{typable}
%   A closure $( \expr, [ x_1 \to (\valr_1 ,  \env_1 , \adapt_1 ) , \ldots, x_i \to (\valr_i, \env_i, \adapt_i )] )$ is typable with type $\type$ and adaptivity $J$ if exists $k_i$\\
%   \[
%      x_1 : \bang{\idx_1} \type_1, \ldots, \bang{\idx_i} \type_i 
%      \tvdash{\nnatA}  \expr : \type  \]
%    and each closure $(\valr_i, \env_i)$  is also typable with type $\bang{\idx_i} \type_i$ and adaptivity $\nnatA_i$, $ \phi = [x_1
%      \to (\idx_1, \adapt_1, \nnatA_1), \ldots,  x_i \to (\idx_i, \adapt_i,
%      \nnatA_i)  ] $,  $J = \nnatA + F( \expr, \phi ) $.
%  \end{defn}

%  \begin{defn}[ClosedClosure]
%   \label{closure}
%    A closure $(\expr, \env)$ is closed if $\fv{\expr} \subseteq \dom(\env)$. 
%  \end{defn}

% % \begin{lem}[ClosureTypable ]
% %   \label{ct}
% %    If a closure $(\expr, \env)$ is closed, then there exists $\type$ and $J$ so that $(\expr, \env)$ is typable with $\type$ and $J$.
% %    \end{lem}
 

% \begin{lem}[programTypable]
%   \label{proglemma}
%    If $ \tvdash{\nnatA}   \expr : \type $, then $(
%      \expr, \emptyset ) $ is typable with $\type$ and adaptivity $\nnatA$. 
%    \end{lem}

%    \begin{lem}[TypableMono]
%      \label{tmono}
%      If a closure is $D$ is typable with $\type$ and $\nnatA$, and $\nnatA \leq \nnatA'$, then
%      D is typable with $\type$ and $\nnatA'$.
%     \end{lem} 

   
% \begin{lem}[TypableSoundness]
%   \label{tsound}
%   If a closure $D$ is typable with $\type$ and $J$, and $D \bigstep{\adapt} E$, then
%     closure $E$ is typable with $\type$ and adaptivity $J - \adapt$. 
%    \end{lem}
% \paragraph{Typable Approach By Marco}
% \begin{defn}[Typable Closures]
%   \label{def:typable}
% Let $\env=[ x_1 \to (\valr_1 ,  \env_1 , R_1 ) ,
%   \ldots, x_n \to (\valr_n, \env_n, R_n )]$. 
%   The closure $( \expr, \env)$ is typable with
%   type $\type$ and adaptivity $J$ if:
% \begin{enumerate}\item  
%      $x_1 : \bang{k_1} \type_1, \ldots, \bang{k_i} \type_i 
%      \tvdash{Z}  \expr : \type$, for some types $\bang{k_i}
%    \type_i$ for $(1\leq i\leq n)$, 
% \item each closure $(\valr_i, \env_i)$ for $(1\leq i\leq n)$ is typable with type
%   $\bang{k_i} \type_i$ and adaptivity $Z_i$,
% \item $J = Z + \sum_{(v_i,\theta_i,S_i)\in\theta} k_i \times (R_i
%   +Z_i)$.
% \end{enumerate}
%  \end{defn}
% To justify why we chose $\sum$ in the third clause above it is worth
% to consider the following configuration:
% $$
% [x\mapsto (\lambda u.\lambda w.\delta(u)+\delta(w),[\,],0)
% ,y:\mapsto (v,[\,],2) ], x\, y\, y
% $$
   
% \begin{lem}[Soundness]
%   \label{tsound}
%   If a closure $D$ is typable with type $\type$ and adaptivity $J$, and $D \bigstep{R} E$, then
%     the closure $E$ is typable with type $\type$ and adaptivity $I$,
%     where $I+R\leq J$. 
%    \end{lem}
% \fail{Soundness is unable to be proven:
% %
% \\
% %
% In the semantics app rule: $R_1 + R_3$ adaptivity is numerically added.
% %
% \\
% % 
% In the typing-configuration (typable closure): $J = Z + \sum_{(v_i,\theta_i,S_i)\in\theta} k_i \times (R_i +Z_i)$, adaptivity adding is by $\max$. 
%   \\
% The adpativity in typing rule cannot bound the adaptivity in semantics. We need to have a better understanding on the adptivity flow
% }



%%%%%%%%%%%%%%%%%%%%%%%%%%%%%%%%%%%%%%%%%%%%%%%%%%%%%

%%%%%%%%%%%%%%%%%%%%%%%%%%%%%%%%%%%%%%%%%%%%%%%%%%%%%


% \[
% \begin{array}{llll}
%  % \mbox{Index Term} & \idx, \nnatA & ::= &     i ~|~ n ~|~ \idx_1 + \idx_2 ~|~  \idx_1
%  %                                  - \idx_2 ~|~ \smax{\idx_1}{\idx_2}\\
% %                                  \mbox{Sort} & S & ::= & \nat \\
%   \mbox{Linear type} & \ltype &::=  &  \type \lto \type ~|~ \tbase \\
%   \mbox{Nonlinear Type} & \type & ::= & \bang{\idx} \ltype   \\
% \end{array}
% \]

% \begin{figure}
%   \begin{mathpar}
%     \inferrule{
%     }{
%       \ictx \tctx , x: \bang{\nnatA}\ltype, \Gamma' \tvdash{\nnatA} x: \bang{\nnatA}\ltype
%     }~\textbf{Ax}
%     %
%     \and
%     %
%     \inferrule{
%     }{
%       \ictx \Gamma \tvdash{\nnatA} c : \bang{\nnatA}\tbase 
%     }~\textbf{const}
%     %
%     % \and
%     % %
%     % \inferrule{
%     % }{
%     %   \ictx \Gamma \tvdash{\nnatA} \evec : \bang{\nnatA}\tbase 
%     % }~\textbf{Dict}
%     %
%     \and
%     %
%     \inferrule{
%       \ictx \Gamma, x: \type_1
%       \tvdash{\nnatA }
%       \expr: \type_2
%     }{
%       \ictx k+\Gamma \tvdash{k+\nnatA} \lambda x. \expr : \bang{k}  ( \type_1
%       \lto \type_2)
%     }~\textbf{lambda}
%     \and
%     %
%     \inferrule{
%       \ictx \Gamma_1  \tvdash{\nnatA_1} \expr_1:  \bang{0} ( \type_1
%       \lto \type_2      ) \\
%       \ictx \Gamma_2 \tvdash{\nnatA_2} \expr_2: \type_1 
%     }{
%       \ictx \max (\Gamma_1, \Gamma_2 ) \tvdash{\max( \nnatA_1,\nnatA_2) } \expr_1 \eapp \expr_2 : \type_2
%     }~\textbf{app}
%     %
%     \and
%     %
%     \inferrule{
%       \ictx \Gamma \tvdash{\nnatA} \expr: \bang{k} \ltype 
%     }{
%       \ictx \Gamma' ,1+\Gamma  \tvdash{1+\nnatA} \delta(\expr): \bang{k} \ltype 
%     }~\textbf{delta}
%      %
%     \and
%     %
%     \inferrule{
%       \ictx \Gamma'  \tvdash{\nnatA'} \expr: \type' \\
%       \Gamma' \leqslant \Gamma \\
%       \nnatA' \leq \nnatA\\
%       \sub{\type'}{\type} \\
%       \ictx \Gamma \tvdash{\nnatA} \expr: \bang{k} \ltype 
%     }{
%       \ictx \Gamma  \tvdash{\nnatA} \expr: \type 
%     }~\textbf{subtype}
%       %
%     \and
%     %
%     \inferrule{
%       \ictx \Gamma, y: \type', x: \type ,\Gamma'  \tvdash{\nnatA} \expr: \type 
%     }{
%       \ictx \Gamma, x: \type, y: \type' ,\Gamma'  \tvdash{\nnatA} \expr: \type 
%     }~\textbf{exchange}
%     \\\\
%     \boxed{
%  \inferrule{
%       \ictx \Gamma, x: \type_1
%       \tvdash{\nnatA }
%       \expr: \type_2
%     }{
%       \ictx k+\Gamma \tvdash{k} \lambda x. \expr : \bang{k}  ( \type_1
%       \lto^{\nnatA} \type_2)
%     }~\textbf{lambda}
%     \and
%     %
%     \inferrule{
%       \ictx \Gamma  \tvdash{\nnatA_1} \expr_1:  \bang{0} ( \type_1
%       \lto^{\nnatA} \type_2      ) \\
%       \ictx \Gamma \tvdash{\nnatA_2} \expr_2: \type_1 
%     }{
%       \ictx \Gamma  \tvdash{ \nnatA_1 + \max( \nnatA,\nnatA_2) } \expr_1 \eapp \expr_2 : \type_2
%     }~\textbf{app}
%     }
%     \\\\
% \begin{array}{lll}
%    k+\bang{r} \ltype  &\triangleq  &  \bang{k+r} \ltype  \\
%   k + \emptyset   & \triangleq & \emptyset   \\
%   k + ( [x : \type], \Gamma) & \triangleq &  [x : k+\type], k+\Gamma   
%   \\
%   \max(\bang{k_1} \ltype, \bang{k_2} \ltype) & \triangleq& \bang{ \max(k_1,
%                                                     k_2) } \ltype \\
%   \max(\Gamma, \emptyset) & \triangleq & \Gamma \\
%   \max(\emptyset, \Gamma) & \triangleq & \Gamma \\
%   \max\Big(  ([x : \type ],\Gamma),  ([x: \type'],\Delta)  \Big) & \triangleq
%                             & [x: \max(\type, \type')], \max(\Gamma,
%                               \Delta )\\
%   \sub{\Gamma}{\Delta} & \triangleq &  \dom(\Gamma) = \dom(\Delta)
%                                       \land \forall x \in
%                                       \dom(\Gamma), \sub{\Delta(x)}{\Gamma(x)}  
% \end{array}
%   \end{mathpar}
%   \caption{Typing rules, first version}
%   \label{fig:type-rules1}
% \end{figure}

% \begin{figure}
%   \begin{mathpar}
%     \inferrule{
%       k_1 \leq k \\
%       \sub{\ltype}{\ltype_1}
%     }{
%       \sub{\bang{k} \ltype}{\bang{k_1} \ltype_1}
%     }~\textsf{bang}
%     %
%     \and
%     %
%      \inferrule{
%         \sub{\type_1}{\type}   \\
%       \sub{\type'}{\type_1'}
%     }{
%       \sub{\type \lto \type' }{\type_1 \lto \ltype_1'}
%     }~\textsf{arrow}
%     %
%     \and
%     %
%     \inferrule{
%     }{
%     \sub{\tbase}{\tbase}
%     }~\textsf{base}
%   \end{mathpar}
%   \caption{subtyping}
%  \end{figure}

%  \clearpage

%  \begin{thm}[Weaking]
%   \label{sub}
%   \begin{enumerate} 
%    \item If $ \Gamma,x : \type' \tvdash{ \nnatA} \expr : \type $ and $
%   x \not \in \fv{\expr}  $ ,  then  $ \Gamma \tvdash{ \nnatA} \expr : \type $.
%   \end{enumerate}
% \end{thm}

% \begin{thm}[Value Adaptivity]
%   \label{sub}
%   \begin{enumerate} 
%    \item for all type $\bang{k} \ltype$,  exist value $\valr$, then  $
%      \empty \tvdash{ k} \valr : \bang{k} \ltype $.
%   \end{enumerate}
% \end{thm}

% \begin{thm}[Substitution]
%   \label{sub}
%   \begin{enumerate} 
%    \item If $ \Gamma,x : \type' \tvdash{ \nnatA} \expr : \type $ and $
%   \empty \tvdash{\nnatA'} \valr : \type'  $ , then  $\Gamma
%   \tvdash{\max(\nnatA,\nnatA' )} \expr[\valr/x]  : \type$. 
%   \end{enumerate}
% \end{thm}

% \begin{proof}
%   By induction on the typing derivation.\\
% \caseL{
%   $   \inferrule{
%     }{
%       \ictx \tctx , x: \bang{\nnatA}\ltype \tvdash{\nnatA} x: \bang{\nnatA}\ltype
%     }~\textbf{Ax}  $
%   }
% Assume $\empty \tvdash{\nnatA'} \valr : \bang{\nnatA}\ltype $, TS:  $\Gamma
%   \tvdash{\max(\nnatA,\nnatA' )} x[\valr/x]  : \type$. proved by
%   subtype rule on the assumption.
% \caseL{
%  $   \inferrule{
%     }{
%       \ictx \tctx ,y:\type', x: \bang{\nnatA}\ltype \tvdash{\nnatA} x: \bang{\nnatA}\ltype
%     }~\textbf{Ax2}  $
%   }
%   Assume $\empty \tvdash{\nnatA'} \valr : \bang{\nnatA}\ltype $, TS:
%   $\Gamma,   x: \bang{\nnatA}\ltype
%   \tvdash{\max(\nnatA,\nnatA' )} x[\valr/y]  : \type$. proved by rule
%   AX and then subtype.
%   \caseL{
%    \inferrule{
%       \ictx \Gamma, x: \type_1, y:\type'
%       \tvdash{\nnatA }
%       \expr: \type_2
%     }{
%       \ictx k+\Gamma, y: k + \type' \tvdash{k+\nnatA} \lambda x. \expr : \bang{k}  ( \type_1
%       \lto \type_2)
%     }~\textbf{lambda}
%   }
%    Assume $\empty \tvdash{k+\nnatA'} \valr : k+\type' $, TS:
%   $k+\Gamma
%   \tvdash{\max(k+\nnatA,k+\nnatA' )} (\lambda x. \expr)[\valr/y]  : \type$. From the
%   Lemma~\ref{para-dec} on the assumption, we know: $\empty
%   \tvdash{\nnatA'} \valr : \type' ~(1)$.\\
%   By Induction hypothesis on the premise, we get: $ \Gamma, x:\type_1
%   \tvdash{\max( \nnatA, \nnatA' )}
%       \expr[\valr/y]: \type_2 ~(2)$. By rule lambda, we conclude that
%       $k+\Gamma \tvdash{ k+ ( \max(\nnatA,\nnatA ) }
%       \lambda x.\expr[\valr/y]: \type_2 $.
%       \caseL{
%       \inferrule{
%       \ictx \Gamma_1,x:\type'  \tvdash{\nnatA_1} \expr_1:  \bang{0} ( \type_1
%       \lto \type_2      ) \\
%       \ictx \Gamma_2 ,x: \type'', \tvdash{\nnatA_2} \expr_2: \type_1 
%     }{
%       \ictx \max (\Gamma_1, \Gamma_2 ), x:\max(\type',\type'') \tvdash{\max( \nnatA_1,\nnatA_2) } \expr_1 \eapp \expr_2 : \type_2
%     }~\textbf{app}
%   }
%   Assume $\empty \tvdash{\nnatA'} \valr : \max(\type',\type'')$, TS: $\max (\Gamma_1, \Gamma_2 )
%   \tvdash{\max(\nnatA_1,\nnatA_2, \nnatA' )} (\expr_1 \eapp
%   \expr_2)[\valr/x]  : \type_2$. From the definition of $\max(\type',
%   \type'')$, we know that $\type'$ and $\type''$ have similar
%   form. Let us assume $\type'= \bang{k_1} \ltype$ and $\type'' =
%   \bang{k_2} \ltype$ so that $\max(\type',\type'') = \bang{\max(k_1,k_2)}
%   \ltype$.\\
%   From the Lemma~\ref{para-dec} on the assumption, we have $\empty
%   \tvdash{\nnatA' - (\max(k_1,k_2)-k_1) } \valr : \bang{k_1}
%   \ltype~(1)$ and $\empty
%   \tvdash{\nnatA' - (\max(k_1,k_2)-k_2) } \valr : \bang{k_2}
%   \ltype~(2)$.\\ By induction hypothesis on $(1)$ and $(2)$ respctively,
%   we know that:  $ \Gamma_1  \tvdash{ \max( \nnatA_1, \nnatA' - (\max(k_1,k_2)-k_1) ) } \expr_1[\valr/x]:  \bang{0} ( \type_1
%   \lto \type_2   ) ~(3)$  and $ \Gamma_2  \tvdash{\max(\nnatA_2 ,
%     \nnatA' - (\max(k_1,k_2)-k_2)   )} \expr_2[\valr/x]: \type_1 ~(4)$.  By the
%   rule app and $(3)$, $(4)$, we conclude that $$\max (\Gamma_1, \Gamma_2 )
%   \tvdash{\max(  \max( \nnatA_1, \nnatA' - (\max(k_1,k_2)-k_1) )  , \max(\nnatA_2 ,
%     \nnatA' - (\max(k_1,k_2)-k_2)   )  )} \expr_1[\valr/x] \eapp
%   \expr_2[\valr/x]  : \type_2 ~(5).$$
%   Because $\max(\nnatA' - (\max(k_1,k_2)-k_1) ) , \nnatA' -
%   (\max(k_1,k_2)-k_2)   ) \leq \nnatA' $, by subtype, we raise the
%   adaptivity to  $\max(\nnatA_1,\nnatA_2, \nnatA' ) $ from $(5)$.
%    \caseL{
%       \inferrule{
%       \ictx \Gamma_1,x:\type'  \tvdash{\nnatA_1} \expr_1:  \bang{0} ( \type_1
%       \lto \type_2      ) \\
%       \ictx \Gamma_2  \tvdash{\nnatA_2} \expr_2: \type_1 
%     }{
%       \ictx \max (\Gamma_1, \Gamma_2 ), x:\type' \tvdash{\max( \nnatA_1,\nnatA_2) } \expr_1 \eapp \expr_2 : \type_2
%     }~\textbf{app2}
%   }
%   It is another case for application when x only appear in the first
%   premise. In this case, $\expr_2[\valr/x] = \expr_2$. Another case
%   when variable x only appears in the second premise can be proved in
%   a similar way.\\
%   Assume $\empty \tvdash{\nnatA'} \valr :\type'$. TS:$\max (\Gamma_1, \Gamma_2 )
%   \tvdash{\max(\nnatA_1,\nnatA_2, \nnatA' )} (\expr_1 \eapp
%   \expr_2)[\valr/x]  : \type_2$.  By Induction Hypothesis on the first
%   premise using the assumption, we get: $\Gamma_1
%   \tvdash{\max(\nnatA_1, \nnatA')} \expr_1[\valr/x]:  \bang{0} ( \type_1
%       \lto \type_2  )  ~(1)$. By the rule app using (1) and the second
%       premise, we conclude that $$ \max (\Gamma_1, \Gamma_2 )
%       \tvdash{\max( \max(\nnatA_1,\nnatA'),\nnatA_2) }
%       \expr_1[\valr/x] \eapp \expr_2 : \type_2$$
%       \caseL{
%  \inferrule{
%       \ictx \Gamma, x:\type' \tvdash{\nnatA} \expr: \bang{k} \ltype 
%     }{
%       \ictx \Gamma' ,1+\Gamma, x:1+\type'  \tvdash{1+\nnatA} \delta(\expr): \bang{k} \ltype 
%     }~\textbf{delta}
%   }
%   Assume $\empty \tvdash{\nnatA'+1} \valr : 1+\type' $, TS: $ \Gamma'
%   ,1+\Gamma \tvdash{\max(1+\nnatA, 1+\nnatA')} \delta(\expr)
%   [\valr/x]: \bang{k} \ltype $.
%   By Lemma~\ref{para-dec} on the assumption, we have $\empty
%   \tvdash{\nnatA'} \valr : \type'~(1) $. By IH on the first premise
%   along with (1), we have: $\Gamma \tvdash{\max(\nnatA, \nnatA')}
%   \expr[\valr/x]: \bang{k} \ltype~ (2)$.
%    By the rule delta using (2), we conclude that $\Gamma' ,1+\Gamma  \tvdash{1+(\nnatA,\nnatA')} \delta(\expr[\valr/x]): \bang{k} \ltype$.
% \end{proof}



% \begin{proof}
%   By Induction on the typing derivation.
%   \caseL{
%      $   \inferrule{
%     }{
%       \ictx \tctx , x: \bang{\nnatA}\ltype \tvdash{\nnatA} x: \bang{\nnatA}\ltype
%     }~\textbf{Ax}  $
%   }
%   Assume $\env= \Big( \env_1, [x \to (\valr,\adapt
%   )] , \Big) \vDash (\tctx , x: \bang{\nnatA}\ltype  )$ where $\env_1 \vDash \Gamma$. We know that $
%   \empty \tvdash{\adapt} \valr : \bang{\nnatA}\ltype $.
%   From the evaluation rule var, we know $\env , x \bigstep{\adapt} \valr,
%   \env  $.
%   TS:  $ \adapt + adap(\valr, \env)  \leq  \nnatA +
%   F(\env) \implies \adapt + 0 \leq \nnatA + \max( \adapt, F(\env_1))
%   $.It is trivially true.
% \caseL{
%   $
%     \inferrule{
%       \ictx \Gamma, x: \type_1
%       \tvdash{\nnatA }
%       \expr: \type_2
%     }{
%       \ictx k+\Gamma \tvdash{k+\nnatA} \lambda x. \expr : \bang{k}  ( \type_1
%       \lto \type_2)
%     }~\textbf{lambda}
%   $
% }
% Choose $\env \vDash  (k+\Gamma)$ so that $\forall x_i \in
% (\Gamma). \env(x_1) =(\valr_i, \adapt_i ) \land \empty
% \tvdash{\adapt_i } \valr_i: k+\Gamma(x_i) $.  By the evaluation rule
% we know $\env, \lambda x. \expr \bigstep{0}
%                                        \lambda x.\expr, \env $, TS: $0
%                                        + \adap(\lambda x.\expr, \env)
%                                        \leq  k+\nnatA + F(\env)$, which is trivially
%                                        true because $ \adap(\lambda
%                                        x.\expr, \env) \leq F(\env) $.
                                       
% \caseL{
%     $  \inferrule{
%       \ictx \Gamma_1  \tvdash{\nnatA_1} \expr_1:  \bang{0} ( \type_1
%       \lto \type_2      ) \\
%       \ictx \Gamma_2 \tvdash{\nnatA_2} \expr_2: \type_1 
%     }{
%       \ictx \max (\Gamma_1, \Gamma_2 ) \tvdash{\max( \nnatA_1,\nnatA_2) } \expr_1 \eapp \expr_2 : \type_2
%     }~\textbf{app}  $
%   }
%   Choose $\env = [x_i \to (\valr_i,0)] $ for all $x_i$ in
%   $\dom(\max(\Gamma_1,\Gamma_2))$
%   so that  $\empty \tvdash{\nnatA_i} \valr_i  : (\max(\Gamma_1,
%   \Gamma_2)(x_i) $.
%   From the definition, we know that $\env \vDash \Gamma_1$ and $\env
%   \vDash \Gamma_2$. Because $\expr_1$ has the arrow type and will be
%   evaluated to a function, assume exists $\env_1$ so that $\env,
%   \expr_1 \bigstep{\adapt_1} \lambda x.\expr , \env_1 $.  By induction
%   hypothesis on the first premise, we know that: $\adapt_1 +
%   \adap(\lambda x. \expr, \env_1) \leq \nnatA_1 + F(\env,
%   \Gamma_1)~(1)$.Assume exists $\env_2$ so that $\expr_2$ is evaluated
%   to an arbitrary value $\valr_2$ : $ \env, \expr_2 \bigstep{\adapt_2}
%   \valr_2 , \env_2$, by induction hypothesis, we conclude that :  $\adapt_2 +
%   \adap(\valr , \env_2) \leq \nnatA_2 + F(\env,
%   \Gamma_2)~(2)$.
                            


% \[
% \inferrule{
%     \env, \expr_1 \bigstep{\adapt_1} \lambda x.\expr , \env_1 \\
%     \env, \expr_2 \bigstep{\adapt_2} \valr_2 , \env_2 \\
%     (\env_1 \uplus \env_2)[ x  \to (\valr_2,   \adapt_2  ) ], \expr
%     \bigstep{\adapt_3} \valr, \env_3
%   }{
%     \env, \expr_1 \eapp \expr_2 \bigstep{\adapt_1+\adapt_3} \valr, \env_3
%   }~\textsf{app}
% \]
%  \end{proof} 


\end{document}



