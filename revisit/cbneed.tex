
\[\begin{array}{llll}
\mbox{Expr.} & \expr & ::= & \econst ~|~ x ~|~ \expr_1 \eapp \expr_2 
 ~|~ \lambda x. \expr  ~|~ \eletx{x}{e_1}{e_2} ~|~
                             \pair{\expr_1}{\expr_2} ~|~
                             \eprojl{\expr} ~|~ \eprojr{\expr}
    \\
             & & &  \etrue ~|~ \efalse ~|~
                   \eif (\expr, \expr_2 , \expr_3) ~|~ \eop(\expr)   \\
    \mbox{Value.} & v & ::= & \econst ~|~   \lambda x. \expr  ~|~ \pair{\valr_1}{\valr_2}  \\
\mbox{Environment} & \env & ::= & \epsilon ~|~ \env, [ x \mapsto (\valr, \adapt)]
\end{array}\]

% \[
% \begin{array}{llll}
%   \mbox{Index Term} & \idx, \nnatA & ::= &     i ~|~ n \\
%  %                                  - \idx_2 ~|~ \smax{\idx_1}{\idx_2}\\
% %                                  \mbox{Sort} & S & ::= & \nat \\
%   \mbox{Linear type} & \type &::=  &  \ltype \lto{\nnatA} \type ~|~
%                                      \tbase ~|~ \tbool \\
%   \mbox{Nonlinear Type} & \ltype & ::= & \bang{\idx} \type   \\
%   \mbox{Typing context } & \Gamma & ::= & x_1 : \ltype_1, \ldots,
%                                           x_n : \ltype_n
% \end{array}
% \]


\begin{figure}[h]
  \begin{mathpar}
    \inferrule{
    }{
     \env, \valr \bigstep{0} \env, \valr } ~\textsf{val}
   %  \and
   % \inferrule{  \mathsf{fetch} (\env,x)  =  (\valr, \adapt)  }{x, \env
   %   \bigstep{\adapt} \valr, \env }~\textsf{var}
   \and
   %
     \inferrule{   }{
       \env[x -> (\valr,\adapt)] , x   \bigstep{\adapt}  \env[x ->
       (\valr, 0)], \valr }~\textsf{var}
     %
 %
   \and
  %
   \inferrule{ }{ \env,  \econst \bigstep{0} \env, \econst}~\textsf{const}
   %
   \and
   %
 \inferrule{
  }{
    \env, \lambda x. \expr
    \bigstep{0} \env,  \lambda x.\expr
  }~\textsf{lambda}
  %
  \and
  %
  \inferrule{
    \env, \expr_1 \bigstep{\adapt_1} \env_1, \lambda x.\expr  \\
     \env_1, \expr_2 \bigstep{\adapt_2} \env_2, \valr_2  \\
    \fresh \eapp x' \\
   \env_2[ x'  \to (\valr_2, \adapt_2  ) ] , \expr[x'/x], 
    \bigstep{\adapt_3}  \env_3, \valr_3
  }{
     \env, \expr_1 \eapp \expr_2 \bigstep{\adapt_1+\adapt_3}  \env_3, \valr_3
   }~\textsf{app}
    %
  \and
  %
  \inferrule{
    \env, \expr_1 \bigstep{\adapt_1} \env_1, \valr_1  \\
    \fresh \eapp x' \\
   \env_1[ x'  \to (\valr_1, \adapt_1  ) ] , \expr_2[x'/x], 
    \bigstep{\adapt_2}  \env_2, \valr_2
  }{
     \env, \eletx{x}{\expr_1}{\expr_2} \bigstep{\adapt_2}  \env_2, \valr_2
  }~\textsf{let}
 %
  \and
  %
  \inferrule{
    \env, \expr  \bigstep{\adapt} \env_1,  \lambda x. \expr_1  \\
    \eop(\lambda x. \expr' )= \valr_1 
  }{
   \env,  \eop(\expr) \bigstep{\adapt +1}   \env_1, \valr_1
  }~\textsf{delta}
% %
\and
 %
 \inferrule{
  \env, \expr_1 \bigstep{\adapt_1} \env_1, \efalse
   \\
   \env_1, \expr_2  \bigstep{\adapt_2} \env_2, \valr_2
  }{
   \env, \eif (\expr_1 ,\expr_2 , \expr_3) \bigstep{\adapt_1 +
      \adapt_2 } \env_2,  \valr_2
  }~\textsf{if-f}
\and
 \inferrule{
  \env, \expr_1 \bigstep{\adapt_1} \env_1, \etrue
   \\
   \env_1, \expr_3  \bigstep{\adapt_3} \env_3, \valr_3
  }{
   \env, \eif (\expr_1 ,\expr_2 , \expr_3) \bigstep{\adapt_1 +
      \adapt_2 } \env_3,  \valr_3
  }~\textsf{if-t}
  % %
 %
  \and
  %
  \inferrule{
    \env, \expr_1  \bigstep{\adapt_1} \env_1,  \valr_1 \\
     \env, \expr_2  \bigstep{\adapt_2} \env_2,  \valr_2 \\
  }{
   \env, \pair{\expr_1}{\expr_2} \bigstep{ \max(\adapt_1, \adapt_2)}
   \env_1 \uplus \env_2 ,  \pair{ \valr_1}{\valr_2}
 }~\textsf{prod}
 %
  \and
  %
  \inferrule{
    \env, \expr  \bigstep{\adapt} \env_1,  (\valr_1, \valr_2) 
  }{
   \env, \eprojl{(\expr)} \bigstep{ \adapt }
   \env_1,  \valr_1
 }~\textsf{projl}
 %
  \and
  %
  \inferrule{
    \env, \expr  \bigstep{\adapt} \env_1,  (\valr_1, \valr_2) 
  }{
   \env, \eprojr{(\expr)} \bigstep{ \adapt }
   \env_1,  \valr_2
 }~\textsf{projr}
  %
  \and
  %
  \inferrule{
    \env, \expr_1  \bigstep{\adapt_1} \env_1,  \valr_1 \\
     \env, \expr_2  \bigstep{\adapt_2} \env_2,  \valr_2 \\
  }{
   \env, \expr_1 < \expr_2 \bigstep{ \max(\adapt_1, \adapt_2)}
   \env_1 \uplus \env_2 ,   \valr_1 < \valr_2
  }~\textsf{bop}
%   }
  \\\\
  \begin{array}{llll}
    \env_1 \uplus \epsilon & \triangleq & \env_1 &\\
    \epsilon \uplus \env_2 & \triangleq & \env_2 &\\
     \env_1[x \to (\valr, \adapt)] \uplus \env_2[x \to (\valr, 0)]  &
                                                                 \triangleq
                                         & (\env_1 \uplus \env_2) [x\to
                                           (\valr, 0)] &\\
     \env_1[x \to (\valr, 0)] \uplus \env_2[x \to (\valr, \adapt )]  &
                                                                 \triangleq
                                         & (\env_1 \uplus \env_2)[x \to
                                           (\valr, 0)] &\\
    \env_1[x \to (\valr, \adapt)] \uplus \env_2  &
                                                                 \triangleq
                                         & (\env_1 \uplus \env_2) [x\to
                                           (\valr, \adapt)] & x
                                                              \not\in dom(\env_2) \\
     \env_1 \uplus \env_2[x \to (\valr, \adapt )]  &
                                                                 \triangleq
                                         & (\env_1 \uplus \env_2)[x \to
                                           (\valr, \adapt)] & x
                                                              \not\in dom(\env_1)\\
  \end{array}
\end{mathpar}
  \caption{Big-step semantics}
  \label{fig:semantics}
\end{figure}

%  \begin{figure}[h]
%   \begin{mathpar}
%     \inferrule{
%      \env ( x ) = (\valr, \env', \adapt)
%       \\
%       \tvdash{\nnatA} ( \valr, \env') : \type
%           }{
%      \tvdash{\adapt + \nnatA}   ( x, \env):  \type
%     }~\textbf{C-Ax}
%     %
%     \and
%     %
%     \inferrule{
%     }{
%       \tvdash{0} (  c, \env) : \tbase
%     }~\textbf{C-const}
   
%     \and
%     %
%     \inferrule{
%       \tvdash{\nnatA' } ( \valr', \theta') : \type_1
%       \\
%       \fresh\eapp  x' ~~ \forall \adapt'
%       \\
%       \tvdash{ S+ \idx \times (\adapt' + \nnatA' ) +\nnatA }
%      ( \expr[x'/x], \env[x' \to (\valr', \theta', R')]      ) :
%      \type_2
%     }{
%      \tvdash{S} (  \lambda x. \expr, \env )  : \bang{\idx} \type_1
%       \lto{\nnatA} \type_2
%     }~\textbf{C-lambda}
%     \and
%     %
%     \inferrule{
%        \tvdash{\nnatA_1} ( \expr_1, \env_1) :  \bang{\idx} \type_1
%       \lto{\nnatA} \type_2      \\
%       \tvdash{\nnatA_2} ( \expr_2, \env_2 ): \type_1
%     }{
%        \tvdash{    \nnatA_1 +
%         \idx \times \nnatA_2 + \nnatA    } (  \expr_1 \eapp \expr_2, \env_1 \uplus \env_2   ) : \type_2
%     }~\textbf{C-app}
%     %
%     \and
%     %
%     \inferrule{
%       \tvdash{\nnatA} (\expr, \env) :  \tbase
%    }{  \tvdash{1+\nnatA} (\delta(\expr) , \env ) : \tbase
%     }~\textbf{C-delta}
%     \\\\
%     \begin{array}{lll}
%        \theta  & \triangleq (x_i \to (\valr_i, \env_i, R_i)) & i \in 
%        \mathbb{N}\\
%       (x_i : \bang{ \idx }\type_i), \Gamma \vDash (x_i \to (\valr_i, \env_i, R_i))
%       \uplus \theta & \triangleq ~~~\tvdash { \_ } (\valr_i, \env_i)
%                                           :  \type_i  &\conj
%                                    \Gamma \vDash \theta
%       \end{array}
%   \end{mathpar}
%   \caption{Typing rules, configure}
%   \label{fig:configure-rules}
% \end{figure}

% \begin{figure}[h]
%   \begin{mathpar}
%     \begin{array}{lll}
%       \lrv{\tbase} & = & \{  ( \econst, \env,  \nnatA)  \} \\
%       %
%       % \lrv{\type_1 \times \type_2} & = & \{(\valr_1, \valr_2) ~|~ \valr_1 \in \lrv{\type_1} \conj \valr_2 \in \lrv{\type_2} \}\\
%       %
%       \lrv{\bang{k} \type } & = & \{  ( \valr, \env,   \nnatA) |  (\valr, \env,
%                                    \nnatA ) \in \lrv{\type}  \} \\
%       %
%       \lrv{ \bang{k} \type_1 \lto{\nnatA} \type_2    } & = &
%       \{( \lambda x.\expr, \env,  \nnatA_1) ~|~ \forall \valr', \env',
%       \nnatA'. ( \valr',\env',  \nnatA') \in
%       \lrv{ \bang{k} \type_1 }.\\
%       & & 
%           \implies   \fresh \eapp x' \land \\
%       & & \forall \adapt. ( \expr[x'/x], \env[x' \mapsto (\valr', \env', \adapt )] ) \in
%           \lre{    }{ \nnatA_1+\nnatA+ \idx \times (\adapt + \nnatA') }{\type_2}     \} \\
%       %
%       \\
%       %
%       \lre{}{\nnatA}{\type} & = & \{  ( \expr, \env) ~|~  ( \expr , \env
%                                   \bigstep{\adapt}  \valr, \env' ) \\
%       & & ~~~~~~~~~~~~~\implies \adapt \leq \nnatA \conj 
%      ( \valr, \env', \nnatA- \adapt) \in \lrv{\type})
%       \}
%     \end{array}
%   \end{mathpar}
%   \caption{Logical relation without step-indexing}
%   \label{fig:lr:non-step}
% \end{figure}

\subsection{Example}
\begin{enumerate}
\item{ [], $\lambda x. (x, x) \eapp \eop{(\lambda z. \expr )} \bigstep{ 0 + 1 } [x' \to
    (v', 0)], v'  $} where $v' = \eop{( \lambda z. \expr)}$.
\item{ [], $\lambda x. \lambda y. (x, y) \eapp \eop{(\lambda z. \expr
      ) \eapp v} $}
\item{ $[x \to (\valr_x, 1), y \to (\valr_y, 2)], \big( (x , \delta(y) ) ,
    (\delta(x), y) \big) $  }
\item{ $[x \to (\valr_x, 1), y \to (\valr_y, 2)], \eif \big( \delta(x),
    \delta(y) , y  \big) $  }
\item{ $[x \to (\valr_x, 1), y \to (\valr_y, 2)], \eif \big( \delta(y),
    \delta(y) , x  \big) $  }
 \item{ [], $\lambda x. \lambda y. \eif \big(  x < y , x+y ,
     \delta(\lambda m. m+x )  \big) \eapp \eop{(\lambda z_1. \expr_1 ) \eapp \delta( \lambda z_2. \expr_2 ) } $}
\wq{\item{ $[x \to (\valr_x, 5), y \to (\valr_y, 6)],  \eletx{z}{  \lambda m. x+y+m}{\delta(z)} $} }
\end{enumerate} 

% \paragraph{Typable Approach By Weihao}
% \[
% \begin{array}{ll}
%  F(\expr, \phi ) & where \eapp ~~ \phi(x_i) = (\idx_i, \adapt_i, \nnatA_i ) \\
%    F(x,\phi) & = \sum_{x_i \in \fv{x}  } \idx_i \times (\adapt_i+ \nnatA_i)  \\
% F(\lambda x. \expr ,  \phi  ) & =  \sum_{x_i \in \fv{\lambda x.\expr}  } \idx_i \times (\adapt_i+ \nnatA_i)   \\  %\sum_{x_i \in \fv{\lambda x.\expr} } k_i \times R_i  \\
% F(\delta(\expr) , \phi ) & = \sum_{x_i \in \fv{\delta(\expr)} } \idx_i \times (\adapt_i+ \nnatA_i)  \\
% F(c, \phi ) & = 0  \\
% F(\expr_1 \eapp \expr_2, \phi ) & = F(\expr_1, \phi ) +
%                                   F(\expr_2,\phi ) \\
% F(\eif \expr \ethen \expr_1 \eelse \expr_2, \phi ) & = F(\expr, \phi) + \max(F(\expr_1, \phi),  F(\expr_2, \phi)   )  
% \end{array} 
% \]

% \begin{defn}[Typable]
%   \label{typable}
%   A closure $( \expr, [ x_1 \to (\valr_1 ,  \env_1 , \adapt_1 ) , \ldots, x_i \to (\valr_i, \env_i, \adapt_i )] )$ is typable with type $\type$ and adaptivity $J$ if exists $k_i$\\
%   \[
%      x_1 : \bang{\idx_1} \type_1, \ldots, \bang{\idx_i} \type_i 
%      \tvdash{\nnatA}  \expr : \type  \]
%    and each closure $(\valr_i, \env_i)$  is also typable with type $\bang{\idx_i} \type_i$ and adaptivity $\nnatA_i$, $ \phi = [x_1
%      \to (\idx_1, \adapt_1, \nnatA_1), \ldots,  x_i \to (\idx_i, \adapt_i,
%      \nnatA_i)  ] $,  $J = \nnatA + F( \expr, \phi ) $.
%  \end{defn}

%  \begin{defn}[ClosedClosure]
%   \label{closure}
%    A closure $(\expr, \env)$ is closed if $\fv{\expr} \subseteq \dom(\env)$. 
%  \end{defn}

% % \begin{lem}[ClosureTypable ]
% %   \label{ct}
% %    If a closure $(\expr, \env)$ is closed, then there exists $\type$ and $J$ so that $(\expr, \env)$ is typable with $\type$ and $J$.
% %    \end{lem}
 

% \begin{lem}[programTypable]
%   \label{proglemma}
%    If $ \tvdash{\nnatA}   \expr : \type $, then $(
%      \expr, \emptyset ) $ is typable with $\type$ and adaptivity $\nnatA$. 
%    \end{lem}

%    \begin{lem}[TypableMono]
%      \label{tmono}
%      If a closure is $D$ is typable with $\type$ and $\nnatA$, and $\nnatA \leq \nnatA'$, then
%      D is typable with $\type$ and $\nnatA'$.
%     \end{lem} 

   
% \begin{lem}[TypableSoundness]
%   \label{tsound}
%   If a closure $D$ is typable with $\type$ and $J$, and $D \bigstep{\adapt} E$, then
%     closure $E$ is typable with $\type$ and adaptivity $J - \adapt$. 
%    \end{lem}
% \paragraph{Typable Approach By Marco}
% \begin{defn}[Typable Closures]
%   \label{def:typable}
% Let $\env=[ x_1 \to (\valr_1 ,  \env_1 , R_1 ) ,
%   \ldots, x_n \to (\valr_n, \env_n, R_n )]$. 
%   The closure $( \expr, \env)$ is typable with
%   type $\type$ and adaptivity $J$ if:
% \begin{enumerate}\item  
%      $x_1 : \bang{k_1} \type_1, \ldots, \bang{k_i} \type_i 
%      \tvdash{Z}  \expr : \type$, for some types $\bang{k_i}
%    \type_i$ for $(1\leq i\leq n)$, 
% \item each closure $(\valr_i, \env_i)$ for $(1\leq i\leq n)$ is typable with type
%   $\bang{k_i} \type_i$ and adaptivity $Z_i$,
% \item $J = Z + \sum_{(v_i,\theta_i,S_i)\in\theta} k_i \times (R_i
%   +Z_i)$.
% \end{enumerate}
%  \end{defn}
% To justify why we chose $\sum$ in the third clause above it is worth
% to consider the following configuration:
% $$
% [x\mapsto (\lambda u.\lambda w.\delta(u)+\delta(w),[\,],0)
% ,y:\mapsto (v,[\,],2) ], x\, y\, y
% $$
   
% \begin{lem}[Soundness]
%   \label{tsound}
%   If a closure $D$ is typable with type $\type$ and adaptivity $J$, and $D \bigstep{R} E$, then
%     the closure $E$ is typable with type $\type$ and adaptivity $I$,
%     where $I+R\leq J$. 
%    \end{lem}
% \fail{Soundness is unable to be proven:
% %
% \\
% %
% In the semantics app rule: $R_1 + R_3$ adaptivity is numerically added.
% %
% \\
% % 
% In the typing-configuration (typable closure): $J = Z + \sum_{(v_i,\theta_i,S_i)\in\theta} k_i \times (R_i +Z_i)$, adaptivity adding is by $\max$. 
%   \\
% The adpativity in typing rule cannot bound the adaptivity in semantics. We need to have a better understanding on the adptivity flow
% }
