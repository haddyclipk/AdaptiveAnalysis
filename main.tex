\documentclass{article}
\usepackage[utf8]{inputenc}
%Packages
\usepackage[T1]{fontenc}
\usepackage{fourier} 
\usepackage[english]{babel} 
\usepackage{amsmath,amsfonts,amsthm} 
\usepackage{lscape}
\usepackage{geometry}
\usepackage{amsmath}
\usepackage{algorithm}
\usepackage{algorithmic}
\usepackage{amssymb}
\usepackage{amsfonts}
\usepackage{times}
\usepackage{bm}
\usepackage{ stmaryrd }
\usepackage{ amssymb }
\usepackage{ textcomp }
\usepackage[normalem]{ulem}
% For derivation rules
\usepackage{mathpartir}
\usepackage{color}
\usepackage{a4wide}


\newcommand{\sctag}[1]{\tag{\textsc{#1}}\label{eq:#1}}
\newcommand{\ands}{~\wedge~}

%Variations
\newcommand{\astable}{\mathbb{S}}
\newcommand{\achange}{U}

%Index Terms
\newcommand{\scond}[3]{(\eif\im{#1}\ethen\im{#2}\eelse\im{#3})}
\newcommand{\keps}[2]{\epsilon(#1,#2)}
\newcommand{\spower}[2]{#1^{#2}}
\newcommand{\ssum}[4]{\sum\limits_{#1=#2}^{#3}#4}
\newcommand{\smin}[2]{\kw{min}({#1},{#2})}
\newcommand{\smax}[2]{\kw{max}(#1,#2)}
\newcommand{\smaxx}[3]{\kw{max}(#1,#2,#3)}
\newcommand{\smaxxx}[4]{\kw{max}(#1,#2,#3,#4)}

%Sizes
\newcommand{\szero}{0}
\newcommand{\sone}{1}
\newcommand{\splus}[2]{#1 + #2}
\newcommand{\ssucc}[1]{#1 {+} \sone}
\newcommand{\sminus}[2]{#1 - #2}
\newcommand{\sdiv}[2]{\frac{#1}{#2}}
\newcommand{\smult}[2]{#1\cdot#2}
\newcommand{\splusone}[1]{#1+1}
\newcommand{\sceil}[1]{\ceil*{#1}}
\newcommand{\sfloor}[1]{\floor*{#1}}
\newcommand{\size}[1]{|#1|}
\newcommand{\slog}[1]{\kw{log}_2(#1)}
\newcommand{\sinf}{\infty}

%Sorts
\newcommand{\ssize}{\mathbb{N}}
\newcommand{\svar}{\mathbb{V}}
\newcommand{\scost}{\mathbb{R}}
\newcommand{\sfun}[2]{#1\mbox{\ra} #2}
\newcommand{\sfunmon}[2]{#1\xrightarrow{\mbox{mon}} #2}
% \newcommand{\sort}{\varsigma}
\newcommand{\sort}{S}
\newcommand{\sorted}[1]{#1 \mathrel{::} \sort}
\newcommand{\sized}[1]{#1 \mathrel{::} \ssize}

% Types
\newcommand{\grt}{A}
\newcommand{\lbound}{\mathop{\uparrow}}
\newcommand{\tbool}{\mbox{bool}}
\newcommand{\trbool}{\mbox{bool}_r}
\newcommand{\tubool}{\mbox{bool}_u}
\newcommand{\trint}{\mbox{int}_r}
\newcommand{\tint}{\mbox{int}}
\newcommand{\tquery}{\mbox{query}}
\newcommand{\tunit}{\mbox{unit}}
\newcommand{\trunit}{\mbox{unit}_r}
\newcommand{\tlist}[3]{\mbox{list}[#1]^{#2}\,#3}
\newcommand{\tlists}[1]{ #1 \, \mbox{list} }
\newcommand{\ulist}[2]{\mbox{list}[#1]\,#2}
\newcommand{\tslist}[1]{\mbox{list}\,#1}
\newcommand{\ttree}[3]{\mbox{tree}[#1]^{#2}\,#3}
\newcommand{\utree}[2]{\mbox{tree}[#1]\,#2}
\newcommand{\tbase}{b}
\newcommand{\uarr}[2]{\mathrel{\xrightarrow[]{\wexec(#1,#2)}}} 
\newcommand{\uarrs}[1]{\mathrel{\xrightarrow[]{\mu(#1)}}} 
\newcommand{\uarrd}{\mathrel{\xrightarrow{\wdead}}}
\newcommand{\tarrd}[1]{\mathrel{\xrightarrow{\wdiff(#1)}}}



\newcommand{\tforall}[3]{\forall#1\overset{\wexec(#2,#3)}{::}S.\,}
\newcommand{\tforalld}[2]{\forall#1\overset{\wdiff(#2)}{::}S.\,}
\newcommand{\uforalls}[2]{\forall#1\overset{\mu(#2)}{::}S.\,}
\newcommand{\tforallS}[1]{\forall#1.\,}
\newcommand{\tsforall}[1]{\forall#1{::}S.\,}
\newcommand{\tforallN}[1]{\forall#1{::}\ssize.\,}
\newcommand{\texists}[1]{\exists#1{::}S.\,}
\newcommand{\texistsN}[1]{\exists#1{::}\ssize.\,}
\newcommand{\tcimpl}[2]{#1 \mathrel{\supset} #2}
\newcommand{\tcprod}[2]{#1 \mathrel{\&} #2}

\newcommand{\ttimes}{\mathrel{\times}}
\newcommand{\tsum}{\mathrel{+}}
\newcommand{\tinter}{\mathrel{\wedge}}
\newcommand{\tst}[1]{(#1)^{\astable}}
\newcommand{\tch}[2]{U\,(#1,#2)} 
\newcommand{\tchs}[1]{U\,(#1,#1)} 
\newcommand{\tcho}[1]{U\,#1} 
\newcommand{\tmu}[1]{(#1)^{\mu}}
\newcommand{\tno}[1]{(#1)^{\_}}
\newcommand{\trm}[2]{|#1|_{#2}}
\newcommand{\trmo}[1]{|#1|}
\newcommand{\tmup}[1]{(#1)^{\mu'}}
\newcommand{\tdual}[1]{d({#1})}
\newcommand{\tbox}[1]{\square\,#1}
\newcommand{\tdmu}[1]{#1^{\shortdownarrow {\mu}}}
\newcommand{\tmon}[1]{{\color{red}m(#1)}}
\newcommand{\tforce}[1]{#1^{\shortdownarrow \achange }}
\newcommand{\tlift}[2]{(#1,#2)^{\uparrow}}
\newcommand{\tpull}[1]{#1^{\nearrow}}
\newcommand{\tpushd}[1]{(#1)^{\downarrow\square}}

% Terms
\newcommand{\vbase}{r}
\newcommand{\vtrue}{\mbox{tt}}
\newcommand{\vfalse}{\mbox{ff}}

\newcommand{\la}{\langle} 
\newcommand{\ra}{\rangle}
\newcommand{\eapp}{\;} 
\newcommand{\eleft}{\pi_1}
\newcommand{\eright}{\pi_2} 
\newcommand{\econst}{\kw{n}}
\newcommand{\etrue}{\mbox{true}}
 \newcommand{\efalse}{\mbox{false}}
\newcommand{\eif}{\mbox{if\;}} 
\newcommand{\ethen}{\mbox{\;then\;}}
\newcommand{\eelse}{\mbox{\;else\;}} 
\newcommand{\einl}{\mbox{inl\;}}
\newcommand{\einr}{\mbox{inr\;}} 
\newcommand{\elet}{\mbox{let\;}}
\newcommand{\clet}{\mbox{clet}\;}
\newcommand{\ecimp}{\mbox{.}_c\;}
\newcommand{\eelimU}{\mbox{elim}_U\;} 
\newcommand{\ein}{\mbox{\;in\;}}
\newcommand{\ecase}{\mbox{\;case\;}} 
\newcommand{\eof}{\mbox{\;of\;}}
\newcommand{\eas}{\mbox{\;as\;}} 
\newcommand{\ecelim}{\mbox{celim\;}}
\newcommand{\enil}{\mbox{nil}} 
\newcommand{\epack}{\mbox{pack\;}}
\newcommand{\eunpack}{\mbox{unpack\;}}
\newcommand{\efix}{\mbox{fix\;}} 
\newcommand{\efixNC}{\mbox{fix$_{NC}$\;}} 
\newcommand{\eLam}{ \Lambda}
\newcommand{\elam}{ \lambda} 
\newcommand{\eApp}{ [\,]\,}
\newcommand{\eleaf}{\mbox{leaf}} 
\newcommand{\ewith}{\;\mbox{with}\;} 
\newcommand{\enode}{\mbox{node}}
\newcommand{\econs}{\mbox{cons}} 
\newcommand{\econsC}{\mbox{cons$_C$}} 
\newcommand{\econsNC}{\mbox{cons$_{NC}$}} 
\newcommand{\eunit}{()}
\newcommand{\eswitch}{\kw{switch}\;}
\newcommand{\enoch}{\kw{NC}\;}
\newcommand{\eder}{\kw{der}\;}
\newcommand{\esplit}{\kw{split}\;}
\newcommand{\ecoerce}[2]{\kw{coerce}_{#1,#2}\;}
\newcommand{\econtra}{\kw{contra}\;}

\newcommand{\ealloc}[2]{ \mathrel{ \mathsf{alloc}\, {#1} \, {#2} } }
\newcommand{\eallocB}[2]{ \mathrel{ \mathsf{alloc_b}\, {#1} \, {#2} } }
\newcommand{\eupdt}[3]{ \mathrel{ \mathsf{update} \ {#1} \ {#2} \ {#3} }  }
\newcommand{\ereadx}[2] { \mathrel{ \mathsf{read} \ {#1} \ {#2} }  }
\newcommand{\eupdtB}[3]{ \mathrel{ \mathsf{update_b} \ {#1} \ {#2} \ {#3} }  }
\newcommand{\ereadxB}[2] { \mathrel{ \mathsf{read_b} \ {#1} \ {#2} }  }
\newcommand{\eret}[1] {\mathrel{ \mathsf{return} \, {#1} }}
\newcommand{\eletx}[3]{  \mathrel{ \mathsf{let_m} \{ {#1} \} = {#2} \ \mathsf{in} \ {#3}  } }



\newcommand{\caseof}[1]{\mbox{case}~#1~\mbox{of}}
\newcommand{\tcaseof}[1]{\mathsf{case}~#1~\mathsf{of}}
\newcommand{\ofnil}[1]{~~\mbox{nil}~\to#1}
\newcommand{\ofzero}[1]{~~\kw{0}~\to#1}
\newcommand{\ofcons}[3]{|~#1::#2~\to~#3}

% Diff Rel
\newcommand{\udiff}{\gtrapprox}
\newcommand{\rdiff}{\ominus}
\newcommand{\rdiffs}{\lesssim}
\newcommand{\ldiff}{\lesssim}

% Evaluation
\newcommand{\red}[1]{\Downarrow^{#1}}

\newcommand{\wmax}{\mbox{\scriptsize max}}
\newcommand{\wmin}{\mbox{\scriptsize min}}
\newcommand{\wdiff}{\mbox{\scriptsize diff}}
\newcommand{\wexec}{\mbox{\scriptsize exec}}
\newcommand{\wdead}{\mbox{\scriptsize dead}}
%Logical relation
\newcommand{\step}{\text{Step index}}
\newcommand{\world}{\text{World}}
\newcommand{\values}{\text{Value}}
\newcommand{\expr}{\text{Expression}}
\newcommand{\ulr}[1]{\llbracket#1\rrbracket_{v}}
\newcommand{\ulrg}[1]{\llbracket#1\rrbracket_{\grt}}
\newcommand{\lr}[1]{\llparenthesis#1\rrparenthesis_{v}}
\newcommand{\lre}[2]{\llparenthesis#1\rrparenthesis_{\varepsilon}^{#2}}
\newcommand{\lrg}[1]{\llparenthesis#1\rrparenthesis_{\grt}}
\newcommand{\ulre}[3]{\llbracket#1\rrbracket_{\varepsilon}^{#2,#3}}
\newcommand{\ulrew}[1]{\llbracket#1\rrbracket_{\varepsilon}^{0,\sinf}}

\newcommand{\relwith}[2]{\{#1~|~#2\}}
\newcommand{\rel}[1]{\{#1\}}
\newcommand{\del}[1]{\mathcal{D}\llbracket#1\rrbracket}
\newcommand{\dd}[1]{\mathcal{D}\llbracket\Delta\rrbracket}
\newcommand{\ugsubst}[1]{\mathcal{G}\llbracket#1\rrbracket}
\newcommand{\gsubst}[1]{\mathcal{G}\llparenthesis#1\rrparenthesis}
\newcommand{\dsubst}[1]{\mathcal{D}\llbracket#1\rrbracket}
\newcommand{\s}{\sigma}
\newcommand{\peq}{\preceq}
\newcommand{\plt}{\prec}
\renewcommand{\d}{\delta}
\newcommand{\g}{\gamma}


% Typing judgments
\newcommand{\jiterm}[2]{\mathrel{\vdash {#1} :: #2}}
\newcommand{\jtype}[4]{\mathrel{\vdash_{#1}^{#2} {#3} : {#4}}}
\newcommand{\jtypeM}[4]{\mathrel{\vdash_{#1}^{#2} {#3} :^c {#4}}}

\newcommand{\jtypes}[3]{\mathrel{\vdash_{#1}^{\mu} {#2} : {#3}}}

\newcommand{\jstype}[3]{\mathrel{\vdash
    {#1} \backsim {#2} : {#3}}}


\newcommand{\jtypediff}[4]{\mathrel{\vdash% _{\wdiff}
    {#2} \ominus {#3} \ldiff #1 : {#4}}}

\newcommand{\jtypediffM}[4]{\mathrel{\vdash
    {#2} \ominus {#3} \ldiff #1 :^c {#4}}}
\newcommand{\jmintypesame}[3]{\mathrel{\vdash
    {#2} \ominus {#2} \ldiff #1 :^c {#3}}}

\newcommand{\jelab}[6]{\mathrel{\vdash
    {#2} \ominus {#3} \rightsquigarrow {#4} \ominus {#5} \ldiff #1 : {#6}}}
\newcommand{\jelabsame}[4]{\mathrel{\vdash
    {#2} \ominus {#3} \rightsquigarrow {#2} \ominus {#3} \ldiff #1 : {#4}}}

\newcommand{\jelabun}[5]{\mathrel{\vdash_{#1}^{#2}
    {#3} \rightsquigarrow {#4} : {#5}}}

\newcommand{\jelabc}[4]{\mathrel{\vdash
    {#2} \ominus {#3} \rightsquigarrow {#2}^* \ominus {#3}^* \ldiff #1 : {#4}}}


\newcommand{\jelabcu}[4]{\mathrel{\vdash_{#1}^{#2}
    {#3} \rightsquigarrow {#3}^* : {#4}}}

\newcommand{\jtypediffsym}[5]{\mathrel{\vdash
    #1 \ldiff {#3} \ominus {#4} \ldiff #2 : {#5}}}
\newcommand{\sty}[2]{\vdash#1 \mathrel{::} #2}


\newcommand{\rname}[1]{\mbox{\small{#1}}}

\newcommand{\vsem}[2]{\llbracket #1 \rrbracket_{V}^{#2}}
\newcommand{\esem}[2]{\llbracket #1 \rrbracket_{E}^{#2}}
\newcommand{\conj}{\mathrel{\wedge}}

\newcommand{\vusem}[1]{\llparenthesis #1 \rrparenthesis_{V}}
\newcommand{\eusem}[1]{\llparenthesis #1 \rrparenthesis_{E}}

\newcommand{\jsubtype}[2]{\sat#1\sqsubseteq#2}
\newcommand{\jasubtype}[2]{\sat^{\mathsf{\grt}}#1\sqsubseteq#2}
\newcommand{\jeqtype}[2]{\sat#1 \equiv#2}
\newcommand{\under}[2]{\sat #1 \trianglelefteq  #2}

\newcommand{\type}{\text{type}}
\newcommand{\rtype}{\text{relational type}}
\newcommand{\Type}{\text{Unary type}}
\newcommand{\Rtype}{\text{Binary type}}



% Cost Constants
\newcommand{\kvar}{c_{var}}
\newcommand{\kconst}{c_{n}}
\newcommand{\kinl}{c_{inl}}
\newcommand{\kinr}{c_{inl}}
\newcommand{\kcase}{c_{case}}
\newcommand{\kfix}{c_{fix}}
\newcommand{\kapp}{c_{app}}
\newcommand{\kLam}{c_{fix}}
\newcommand{\kiApp}{c_{iApp}}
\newcommand{\kpack}{c_{pack}}
\newcommand{\kunpack}{c_{unpack}}
\newcommand{\knil}{c_{nil}}
\newcommand{\kcons}{c_{cons}}
\newcommand{\kcaseL}{c_{caseL}}
\newcommand{\kleaf}{c_{leaf}}
\newcommand{\knode}{c_{node}}
\newcommand{\kcaseT}{c_{caseT}}
\newcommand{\kprod}{c_{prod}}
\newcommand{\kproj}{c_{proj}}
\newcommand{\klet}{c_{let}}


%Constraints
\newcommand{\creal}{\mathbb{R}}
\newcommand{\sat}[1]{\models#1}
\newcommand{\sata}[1]{\models_A#1}
\newcommand{\ceq}[2]{#1\mathrel{\doteq}#2}
\newcommand{\cleq}[2]{#1 \mathop{\leq} #2}
\newcommand{\cleqspec}[2]{#1 \overline{\mathop{\leq}} #2}
\newcommand{\clt}[2]{#1 \mathop{<} #2}
\newcommand{\cgt}[2]{#1 \mathop{>} #2}
\newcommand{\ceqz}[1]{#1 \mathrel{\doteq} 0}
\newcommand{\cneg}[1]{\mathop{\neg}#1}
\newcommand{\cand}[2]{#1 \wedge #2}
\newcommand{\cexists}[3]{\exists#1::#2.#3}
\newcommand{\cexistsK}[3]{\exists#1:#2.#3}
\newcommand{\cexistsS}[2]{\exists#1.#2}
\newcommand{\cexistsC}[2]{\exists#1::\scost.#2}
\newcommand{\cexistsall}[2]{\exists(#1).#2}
\newcommand{\cforall}[3]{\forall#1::#2.#3}
\newcommand{\cforallS}[3]{\forall#1:#2.#3}
\newcommand{\cimpl}[2]{#1\rightarrow#2}
\newcommand{\cor}[2]{#1 \vee #2}
\newcommand{\ctrue}{\top}
\newcommand{\cfalse}{\bottom}
\newcommand{\blank}[2][100]{\hfil\penalty#1\hfilneg }
\newcommand{\ccond}[3]{\im{#1}\mathrel{\mbox{?}}\im{#2}\mathrel{\colon}\im{#3}}


\newcommand{\wfty}[1]{\vdash   #1~\kw{wf}}
\newcommand{\awfty}[1]{\vdash^{\mathsf{\grt}}   #1~\kw{wf}}
\newcommand{\wfcs}[1]{\vdash   #1~\kw{wf}}
\newcommand{\wfctx}[1]{\vdash   #1~\kw{wf}}
\newcommand{\awfctx}[1]{\vdash^{\mathsf{\grt}}   #1~\kw{wf}}



\newcommand{\dc}{ downward closure (\lemref{lem:down-closure}) }
\newcommand{\ctx}{\Delta; \Phi_a; \Gamma}
\newcommand{\nctx}{\Delta; \Phi_a; \tbox{\Gamma}}
\newcommand{\primctx}{\Upsilon}
\newcommand{\octx}{\Delta; \Phi_a; \Omega}
\newcommand{\rctx}[1]{\Delta; \Phi_a; \trm{\Gamma}{#1}}

\newcommand{\shade}[1]{\colorbox{lightgray}{#1}}
\newcommand{\fv}[1]{\text{FV}(#1)}
\newcommand{\fcv}[1]{\text{dom}(#1)}
\newcommand{\fiv}[1]{\text{FIV}(#1)}
\newcommand{\fdv}[1]{\text{dom}(#1)}


\newcommand{\assC}[2]{\text{Assume that $\sat \s \Phi$ and there exists $\Gamma'$  s.t. $\fv{#2} \subseteq \fcv{\Gamma'} $ and $\Gamma' \subseteq \Gamma$ and $(m, \d) \in \ugsubst{\trm{\sigma \Gamma'}{#1}}$}}
\newcommand{\assCU}[1]{\text{Assume that $\sat \s \Phi$ and there exists $\Omega'$  s.t. $\fv{#1} \subseteq \fcv{\Omega'} $ and $\Omega' \subseteq \Omega$ and $(m, \d) \in \ugsubst{\s \Omega'}$}}
\newcommand{\IHassun}[1]{\text{$\fv{#1} \subseteq \fcv{\Omega'} $ and $\Omega' \subseteq \Omega$ and $(m, \d) \in \ugsubst{\s \Omega'}$}}


\newcommand{\IHassU}[2]{\text{$\fv{#2} \subseteq \fcv{\trm{\Gamma'}{#1}} $ and $\trm{\Gamma'}{#1} \subseteq \trm{\Gamma}{#1}$ and $(m, \d) \in \ugsubst{\trm{\sigma \Gamma'}{#1}}$}}


\newcommand{\IHass}[2]{\text{$\fv{#2} \subseteq \fcv{\Gamma'} $ and $\Gamma' \subseteq \Gamma$ and $(m, \d) \in \ugsubst{\trm{\sigma \Gamma'}{#1}}$}}
% Environment
\newcommand{\memory}{\Gamma}%\Delta \ | \ \Phi \ | \ \Gamma  \ | \
                            %\Sigma}
\newcommand{\senv}{\Delta}
\newcommand{\lenv}{\Sigma}
\newcommand{\uenv}{\Omega}
\newcommand{\renv}{\Gamma} 
\newcommand{\cenv}{\Phi} 
\newcommand{\sep}{ \ | \ }
\newcommand{\monad}[4]{\mathrel{ M( \overset{ \mathrel{\mathrm{exec}{#4 }}}{#2}  })}
\newcommand{\monadR}[4]{\mathrel{ \overset{\mathrm{diff}(#4)}{\{ {#1} \} \ {#2} \ \{ {#3} \} }}}
\newcommand{\depProd}[4]{ \mathrel{ \Pi {#1} \stackrel{\mathrm{exec} {#4}}{:}{#2} . \ {#3}}}
\newcommand{\depProdr}[4]{ \mathrel{ \Pi {#1} \stackrel{\mathrm{diff} {#4}}{:}{#2} . \ {#3}}}
\newcommand{\uarrow}[3]{ \mathrel{ \stackrel{\mathrm{exec} {#3}}{{#1} \longrightarrow#2}}}
\newcommand{\uforall}[4]{ \mathrel{ \stackrel{\mathrm{exec} {#4}}{\forall {#1} :#2 . \ #3}}}
\newcommand{\uexist}[3]{\mathrel{{\exists {#1} :: {#2} . \ {#3}}}}
\newcommand{\rarrow}[3]{ \mathrel{ \stackrel{\mathrm{diff}(#3)}{{#1} \longrightarrow {#2}}}}
\newcommand{\rarrowt}[3]{ \mathrel{ {#1} \stackrel{\mathrm{} {#3}}{\longrightarrow} {#2}}}
\newcommand{\rforall}[4]{ \mathrel{ \stackrel{\mathrm{diff}(#4)}{\forall {#1}{::}{#2} . \ {#3}}}}
\newcommand{\rexists}[3]{ \mathrel{ {\exists {#1} {::}{#2} . \ {#3}}}}
\newcommand{\rforallt}[4]{ \mathrel{ \forall {#1} \stackrel{\mathrm{} {#4}}{:}{#2} . \ {#3}}}
\newcommand{\arr}[3]{ \mathrel{ \mathsf{Array}_{#1}[{#2}] \ {#3}} }
\newcommand{\arrR}[3]{ \mathrel{ \mathsf{Array}_{#1}[{#2}] \ {#3}} }
\newcommand{\lst}[2]{ \mathrel{ \mathsf{list}[{#1}] \ {#2}} }
\newcommand{\lstR}[3]{ \mathrel{ \mathsf{list}^{#1}[{#2}] \ {#3}} }
\newcommand{\abs}[2]{\mathrel { \lambda {#1} . {#2} } }
\newcommand{\app}[2]{\mathrel{ {#1} \, {#2} }}
\newcommand{\ret}[1] {\mathrel{ \mathsf{return} \, {#1} }}

\newcommand{\letx}[3]{  \mathrel{ \mathsf{let}\   {#1} = {#2} \ \mathsf{in} \ {#3}  } }
\newcommand{\packx}[1]{  \mathrel{ \mathsf{pack} \, {#1}} }
\newcommand{\unpackx}[3]{  \mathrel{ \mathsf{unpack} \,  {#1} \, \mathsf{as} \, {#2} \, \mathsf{in} \ {#3}  } }
\newcommand{\alloc}[2]{ \mathrel{ \mathsf{alloc}\, {#1} \, {#2} } }
\newcommand{\updt}[3]{ \mathrel{ \mathsf{update} \ {#1} \ {#2} \ {#3} }  }
\newcommand{\readx}[2] { \mathrel{ \mathsf{read} \ {#1} \ {#2} }  }
\newcommand{\tTt}[3]{\mathrel{  {#1} \xrightarrow \ {#2} } }
\newcommand{\force}[1]{\mathrel{\mathsf{force} \ \ {#1}}}
\newcommand{\tfix}{\mathsf{Fix}}
\newcommand{\fix}[1]{\mathsf{fix} \, f(x). {#1}}

%Relational
\newcommand{\monadx}[3]{\mathrel{ \{ {#1} \} \ {#2} \ \{ {#3} \} }}
\newcommand{\monadu}[4]{\mathrel{ \overset{ \mathrel{\mathrm{exec}{#4 }}}{\{ {#1} \} \ #2 \ \{ {#3} \}} }}
\newcommand{\cmp}[4] {\mathrel{   \vdash  {#1} \ominus {#2} \ldiff {#4}  : {#3}  }}
\newcommand{\pair}[1]{\mathrel{ {#1}_{1}{#1}_{2}}}
\newcommand{\imp}[2]{\mathrel{  {#1} \Rightarrow {#2} }}
\newcommand{\eval}[3]{\mathrel{ {#1} \Downarrow^{#3} {#2}   }}
\newcommand{\evalf}[3]{\mathrel{ {#1} \Downarrow^{#3}_{f} {#2}   }}
\newcommand{\evalp}[3]{\mathrel{ {#1} \Downarrow^{#3}_{p} {#2}   }}
\newcommand{\heap}[1]{ ;  {#1}}
\newcommand {\spc} { \  \ }
\newcommand{\monadL}[3]{\mathrel{ \{ {#1} \}  \\  \ {#2} \ \\  \{ {#3} \} }}

\newcommand{\wfa}[1]{\mathrel{\vdash {#1} \quad wf}}
\newcommand{\wf}[1]{\mathrel{\vdash {#1} \quad wf}}
\newcommand{\subtypeA}[2]{\mathrel{ \models {#1} \sqsubseteq {#2} } }
\newcommand{\subtype}[2]{\mathrel{   \models {#1} \sqsubseteq {#2} }   }
\newcommand{\subcost}[3]{\mathrel{   \models {#1} {#3} {#2} }   }

\newcommand{\emptyhp}{\mathsf{empty}}
\newcommand{\llb}[1]{ \llbracket {#1} \rrbracket }
\newcommand{\llu}[2]{ \llb{#1}_{#2}}
\newcommand{\llp}[2]{ \llparenthesis {#1} \rrparenthesis_{#2} }
\newcommand{\llbe}[1]{ \llbracket {#1} \rrbracket^{E} }
\newcommand{\llpe}[2]{ \llparenthesis {#1} \rrparenthesis_{#2}^{E} }

\newcommand{\mg}[1]{\textcolor[rgb]{.90,0.00,0.00}{[MG: #1]}}
\newcommand{\dg}[1]{\textcolor[rgb]{0.00,0.5,0.5}{[DG: #1]}}
\newcommand{\wq}[1]{\textcolor[rgb]{.50,0.0,0.7}{[WQ: #1]}}

% Helpful shortcuts

\newcommand{\freshSize}[1]{#1\in\text{fresh}(\ssize)}
\newcommand{\freshCost}[1]{#1\in\text{fresh}(\scost)}
\newcommand{\freshVar}[1]{#1 \in \text{fresh}(S)}

\newcommand{\m}{M} 
%Bi-directional Typing Judgement
\newcommand{\chdiff}[5]{\vdash{#1}\rdiff{#2}~{\downarrow}~#3,#4 \Rightarrow
{\color{red}#5}}

\newcommand{\chsdiff}[3]{\vdash{#1}\backsim{#2}~{\downarrow}~#3}


\newcommand{\chdiffNC}[5]{\vdash^{\color{blue}NC}{#1}\rdiff{#2}~{\downarrow}~#3,#4 \Rightarrow
{{\color{red}#5}}}

\newcommand{\infdiff}[6]{\vdash{#1}\rdiff{#2}~{\uparrow}~{\color{red}{#3}}\Rightarrow[{\color{red}#4}],{\color{red}#5},{\color{red}#6}}

\newcommand{\infsdiff}[3]{\vdash{#1}\backsim{#2}~{\uparrow}~{\color{red}{#3}}}

\newcommand{\infdiffsimple}[5]{\vdash{#1}\rdiff{#2}~{\uparrow}~{\color{red}{#3}}\Rightarrow{\color{red}#4},{\color{red}#5}}

\newcommand{\chmax}[4]{\vdash^{\wmax}{#1}~{\downarrow}~#2, #3  \Rightarrow
{{\color{red}#4}}}
\newcommand{\chmin}[4]{\vdash^{\wmin}{#1}~{\downarrow}~#2, #3  \Rightarrow
{{\color{red}#4}}}

\newcommand{\chexec}[5]{\vdash{#1}~{\downarrow}~#2, #3, #4  \Rightarrow
{{\color{red}#5}}}

\newcommand{\infmax}[5]{\vdash^{\wmax}{#1}~{\uparrow}~{\color{red}{#2}}\Rightarrow[{\color{red}#3}], {\color{red}#4},{\color{red}#5}}
\newcommand{\infmin}[5]{\vdash^{\wmin}{#1}~{\uparrow}~{\color{red}{#2}}\Rightarrow[{\color{red}#3}], {\color{red}#4},{\color{red}#5}}

\newcommand{\infexec}[6]{\vdash{#1}~{\uparrow}~{\color{red}{#2}}\Rightarrow[{\color{red}#3}], {\color{red}#4},{\color{red}#5},{\color{red}#6}}

\newcommand{\infexecsimple}[5]{\vdash{#1}~{\uparrow}~{\color{red}{#2}}\Rightarrow{\color{red}#3}, {\color{red}#4},{\color{red}#5}}

\newcommand{\emptypsi}{.}

%Existential elimination
\newcommand{\elimExt}[3]{#1 \vdash \kw{elimExt}(#2)~\downarrow #3}
\newcommand{\solveVar}[6]{#1 \vdash \kw{solve}(#2;#3) \downarrow (#4;#5;#6)}



%Shortcuts
\newcommand{\al}{\alpha}
\newcommand{\algwf}[1]{\vdash  #1~\kw{wf}}
\newcommand{\algwfa}[1]{\vdash^{A} #1~\kw{wf}}
\newcommand{\jalgeqtype}[3]{\sat#1\equiv#2\Rightarrow {\color{red}#3}}
\newcommand{\jalgasubtype}[3]{\sat^{\mathsf{\grt}}#1\sqsubseteq#2\Rightarrow {\color{red}#3}}
\newcommand{\jalgsubtype}[3]{\sat#1\sqsubseteq#2\Rightarrow {\color{red}#3}}
\newcommand{\jalgssubtype}[2]{\sat#1\leq#2}


\newcommand{\fvars}[1]{\text{FV}(#1)}
\newcommand{\fivars}[1]{\text{FIV}(#1)}
\newcommand{\filtercost}[1]{\text{filterCost}(#1)}
\newcommand{\uctx}{\Delta; \psi_a; \Phi_a; \Omega}
\newcommand{\bctx}{\Delta; \psi_a; \Phi_a; \Gamma}


\newcommand{\suba}[1]{{#1}[\theta_a]}
\newcommand{\subaex}[2]{{#1}[\theta_a, #2]}
\newcommand{\subt}[1]{{#1}[\theta]}
\newcommand{\subta}[1]{{#1}[\theta\,\theta_a]}
\newcommand{\subsat}[3]{#1~ \rhd~ #2 : #3}

\newcommand{\erty}[1]{|#1|}
\newcommand{\eanno}[4]{(#1:#2,#3,#4)}
\newcommand{\eannobi}[3]{(#1:#2,#3)}
\newcommand{\e}{\overline{e}}
\newcommand{\trans}{\rightsquigarrow}
\newcommand{\tboxp}[1]{\square(#1)}
\newcommand{\tlr}[1]{\tlift{\trm{#1}{i}}}
\newcommand*\bang{!}



%%% Local Variables:
%%% mode: latex
%%% TeX-master: "main"
%%% End:

\newcommand{\nform}{\mathsf{F}}
\newcommand{\mechanism}{\mathsf{M}}
\newcommand{\depth}{\mathsf{depth}}
\newcommand{\query}{\text{Q}}

% \newcommand{\caseof}[2]{\mathsf{case} \ {#1} \ \mathsf{of}\ \{ {#2}\}}
\newtheorem{lemma}{Lemma}
\newtheorem{theorem}{Theorem}[section]
\newtheorem{corollary}{Corollary}[theorem]




% \DeclareSymbolFont{extraup}{U}{zavm}{m}{n}
% \DeclareMathSymbol{\vardiamond}{\mathalpha}{extraup}{87}

\newcommand{\diam}{{\color{red}\diamond}}
\newcommand{\dagg}{{\color{blue}\dagger}}
\let\oldstar\star
\renewcommand{\star}{\oldstar}

\newcommand{\im}[1]{\ensuremath{#1}}

\newcommand{\kw}[1]{\im{\mathtt{#1}}}


\newcommand{\set}[1]{\im{\{{#1}\}}}

\newcommand{\mmax}{\ensuremath{\mathsf{max}}}

%%%%%%%%%%%%%%%%%%%%%%%%%%%%%%%%%%%%%%%%%%%%%%%%%%%%%%%%
% Comments
\newcommand{\omitthis}[1]{}

% Misc.
\newcommand{\etal}{\textit{et al.}}
\newcommand{\bump}{\hspace{3.5pt}}

% Text fonts
\newcommand{\tbf}[1]{\textbf{#1}}
%\newcommand{\trm}[1]{\textrm{#1}}

% Math fonts
\newcommand{\mbb}[1]{\mathbb{#1}}
\newcommand{\mbf}[1]{\mathbf{#1}}
\newcommand{\mrm}[1]{\mathrm{#1}}
\newcommand{\mtt}[1]{\mathtt{#1}}
\newcommand{\mcal}[1]{\mathcal{#1}}
\newcommand{\mfrak}[1]{\mathfrak{#1}}
\newcommand{\msf}[1]{\mathsf{#1}}
\newcommand{\mscr}[1]{\mathscr{#1}}

% Text mode
\newenvironment{nop}{}{}

% Math mode
\newenvironment{sdisplaymath}{
\begin{nop}\small\begin{displaymath}}{
\end{displaymath}\end{nop}\ignorespacesafterend}
\newenvironment{fdisplaymath}{
\begin{nop}\footnotesize\begin{displaymath}}{
\end{displaymath}\end{nop}\ignorespacesafterend}
\newenvironment{smathpar}{
\begin{nop}\small\begin{mathpar}}{
\end{mathpar}\end{nop}\ignorespacesafterend}
\newenvironment{fmathpar}{
\begin{nop}\footnotesize\begin{mathpar}}{
\end{mathpar}\end{nop}\ignorespacesafterend}
\newenvironment{alignS}{
\begin{nop}\begin{align}}{
\end{align}\end{nop}\ignorespacesafterend}
\newenvironment{salignS}{
\begin{nop}\small\begin{align}}{
\end{align}\end{nop}\ignorespacesafterend}
\newenvironment{falignS}{
\begin{nop}\footnotesize\begin{align*}}{
\end{align}\end{nop}\ignorespacesafterend}

% Stack formatting
\newenvironment{stackAux}[2]{%
\setlength{\arraycolsep}{0pt}
\begin{array}[#1]{#2}}{
\end{array}}
\newenvironment{stackCC}{
\begin{stackAux}{c}{c}}{\end{stackAux}}
\newenvironment{stackCL}{
\begin{stackAux}{c}{l}}{\end{stackAux}}
\newenvironment{stackTL}{
\begin{stackAux}{t}{l}}{\end{stackAux}}
\newenvironment{stackTR}{
\begin{stackAux}{t}{r}}{\end{stackAux}}
\newenvironment{stackBC}{
\begin{stackAux}{b}{c}}{\end{stackAux}}
\newenvironment{stackBL}{
\begin{stackAux}{b}{l}}{\end{stackAux}}

%APPENDIX
\newcommand{\caseL}[1]{% \setcounter{equation}{0}

\item \textbf{#1}\newline}

%% \makeatletter
%% \newcommand\definitionname{Lemma}
%% \newcommand\listdefinitionname{Proofs of Lemmas and Theorems}
%% \newcommand\listofdefinitions{%
%%   \section*{\listdefinitionname}\@starttoc{def}}
%% \makeatother



\newtheoremstyle{athm}{\topsep}{\topsep}%
      {\upshape}%         Body font
      {}%         Indent amount (empty = no indent, \parindent = para indent)
      {\bfseries}% Thm head font
      {}%        Punctuation after thm head
      {.8em}%     Space after thm head (\newline = linebreak)
      {\thmname{#1}\thmnumber{ #2}\thmnote{~\,(#3)}
% \addcontentsline{Lemma}{Lemma}
%   {\protect\numberline{\thechapter.\thelemma}#1}
      % \ifstrempty{#3}%
      {\addcontentsline{def}{section}{#1~#2~#3}}%
      % {\addcontentsline{def}{subsection}{\theathm~#3}}
\newline}%         Thm head spec

 \theoremstyle{athm}


% \newtheoremstyle{break}
%   {\topsep}{\topsep}%
%   {\itshape}{}%
%   {\bfseries}{}%
%   {\newline}{}%
% \theoremstyle{break}

%There are some problems with llncs documentcalss, so commenting these out until i find a solution
\newtheorem{thm}{Theorem}

%\spnewtheorem{thm1}[theorem]{Theorem}{\bfseries}{\upshape}
%\newenvironment{Theorem}[1][]{\begin{thm1}\iffirstargument[#1]\fi\quad\\}{\end{thm1}}

 \newtheorem{lem}[thm]{Lemma}
 \newtheorem{conjec}{Conjecture}
 \newtheorem{corr}[thm]{Corollary}
 \newtheorem{defn}{Definition}
 \newtheorem{prop}[thm]{Proposition}
 \newtheorem{assm}[thm]{Assumption}

\newtheorem{Eg}[thm]{Example}
\newtheorem{hypothesis}[thm]{Hypothesis}
\newtheorem{motivation}{Motivation}

% BNF symbols
\newcommand{\bnfalt}{{\bf \,\,\mid\,\,}}
\newcommand{\bnfdef}{{\bf ::=~}}

%% Highlighting
\newcommand{\hlm}[1]{\mbox{\hl{$#1$}}}

%% Provenance modes
\newcommand{\modifrcationProvenance}{{\bf MP}}
\newcommand{\updateProvenance}{{\bf UP}}

%Lemmas
\newcommand{\lemref}[1]{Lemma \ref{#1}} %name and number
\newcommand{\thmref}[1]{Theorem \ref{#1}} %name and number

\renewcommand{\labelenumii}{\theenumii}
\renewcommand{\theenumii}{\theenumi.\arabic{enumii}.}

\usepackage{enumitem}
\setenumerate{listparindent=\parindent}

\newlist{enumih}{enumerate}{3}
\setlist[enumih]{label=\alph*),before=\raggedright, topsep=1ex, parsep=0pt,  itemsep=1pt }

\newlist{enumconc}{enumerate}{3}
\setlist[enumconc]{leftmargin=0.5cm, label*= \arabic*.  , topsep=1ex, parsep=0pt,  itemsep=3pt }

\newlist{enumsub}{enumerate}{3}
\setlist[enumsub]{ leftmargin=0.7cm, label*= \textbf{subcase} \bf \arabic*: }

\newlist{enumsubsub}{enumerate}{3}
\setlist[enumsubsub]{ leftmargin=0.5cm, label*= \textbf{subsubcase} \bf \arabic*: }

\newlist{mainitem}{itemize}{3}
\setlist[mainitem]{ leftmargin=0cm , label= {\bf Case} }


\newenvironment{subproof}[1][\proofname]{%
  \renewcommand{\qedsymbol}{$\blacksquare$}%
  \begin{proof}[#1]%
}{%
  \end{proof}%
}


\newenvironment{nstabbing}
  {\setlength{\topsep}{0pt}%
   \setlength{\partopsep}{0pt}%
   \tabbing}
  {\endtabbing} 


%%% Local Variables:
%%% mode: latex
%%% TeX-master: "main"
%%% End:

\begin{document}

\title{Adaptive Analysis}
\maketitle

% \begin{enumerate}
% \item [Type] the premise $\Delta;\psi_a; \Phi_a  \vDash  {\cleq{I' }{I} \land I' \in \beta}$ will generate the constraint $\forall \Delta. \forall \Psi_a. (\Phi \to \cleq{I' }{I} \land I' \in \beta ) $, where $\beta : array \, bool$ and $I' \in \beta$ is the same as $\beta[I'] = true$.  Z3 will first check this constraint from the premise and if the  result is valid, we keep on checking, otherwise, exit immediately. The reason why we can first check the premise is that $I' \leq I$ and $I' \in \beta$ do not rely on the new generated existential variables in $\Psi_1$ and $\Psi_2$. Right now what I am doing is generating this premise constraint using why3 array module. \\

% \item [let] I remove the separation logic in the let because it will bring non-determinism in the algorithmic typing rules of let. The corresponding algorithmic typing rules follows.

% \item [fix extension] using fix-extension rule in the last page, I plan to add one more annotation of fix with two unary type into the core language.


% \end{enumerate}

 \begin{figure}[h]
 $$
 \begin{array}{rcl}
     \text{Types} & \quad & \tau ::= b \sep \tau \multimap \tau' \sep !_n \tau \sep
     \tau \times \tau \sep \tforallN{i}{\tau} \sep \query \\[2mm]

     \text{Term} & \quad & t ::= c \sep \fix{t} \sep \app{t}{t} \sep !t \sep (t_1,t_2) \sep  \letx{!x}{t_1}{t_2} \sep \Lambda.t \sep t[] \sep \abs{x}{t} \sep  M(t) \sep x \sep q \sep\\
     && \quad   \tcaseof{t}\ \{c_i \Rightarrow t_i\}_{c_i \in b}  \sep \letx{(x_1,x_2)}{t_1}{t_2} \\[2mm]
      
     \text{Normal Form} &\quad & F ::=  c \sep \fix{t} \sep !t \sep (F_1, F_2) \sep \Lambda. t \sep \abs{x}{t} \sep M(F) \sep x \sep q \sep \tcaseof{F}\ \{c_i \Rightarrow F_i\}_{c_i \in b_i}  \\[2mm]

     \text{Mechanisms} &\quad & M ::=  {\tt gauss} \sep {\tt thdt} \\[2mm]


	\text{Tree} &\quad& T_b :: = c \sep M(T_{query}) \sep \tcaseof{T_b}\ \{ c_i \Rightarrow T_{b_i}\}_{c_i \in b} \\

	\text{} &\quad& T_{query} :: = q \sep \tcaseof{T_b}\ \{ c_i \Rightarrow T_{query_i}\}_{c_i \in b} \\[2mm]
     \text{Depth} &\quad&   \depth(c) = 0 \\
       &\quad& \depth(!t) = \depth(t) \\
           &\quad&      \depth( \app{t_1}{t_2} ) = \max(\depth(t_1), \depth(t_2)) \\
            &\quad&  \depth(M(t)) = 1 + \depth(t) \\
             &\quad&  \depth(\abs{x}{t}) = \depth(t) \\
              &\quad& \depth(x) = 0 \\
              & \quad & \depth(q) = 0 \\
              & \quad & \depth((t_1, t_2)) = \max(\depth(t_1), \depth(t_2))\\
              & \quad & \depth(\letx{(x_1, x_2)}{t}{t'}) = \max(\depth(t), \depth(t'))\\
              & \quad & \depth(\letx{!x}{t}{t'}) = \max(\depth(t), \depth(t'))\\
              & \quad & \depth(\tcaseof{t}\ \{c_i \Rightarrow t_i\}_{c_i \in b}) = \max(\depth(t), \depth(t_i))\\
            & \quad & \depth(\Lambda.t) = \depth(t)\\
              & \quad & \depth(t\, []) = \depth(t)\\
\end{array}
$$
\caption{syntax}
\end{figure}

\clearpage
\begin{figure}
    Abstract example
$$
{\tt gauss}({\tt {\tt count}(\phi)})
$$
Concrete example
$$
{\tt gauss}({\tt {\tt count}(\lambda r. \pi_1 r \leq 5)})
$$
$$
{\tt gauss}({\tt {\tt count}(\lambda r. \pi_1 r \leq 0.134)}) +
{\tt gauss}({\tt {\tt count}(\lambda r. \pi_2 r = "hiv")})
$$

Depth 2\\
Abstract
  $$  \tcaseof{ {\tt gauss}({\tt {\tt count}(\phi)})  }\ \{c_i \Rightarrow {{\tt gauss}({\tt {\tt count}(\phi_i)})} \}_{c_i \in b} $$
Concrete
 $$ \phi =  \lambda r. \pi_1 r \leq 5 ; c_1 = 0, \phi_1 =  \lambda r. \pi_1 r \leq 5 ;  c_2 = 0.1,  \phi_2 =  \lambda r. \pi_1 r \leq 3; \cdots $$
 
 Depth 3: \\
    \begin{tabbing}
    $\tcaseof{ {\tt gauss}({\tt {\tt count}(\phi)})  }$ \\ $\{c_i \Rightarrow \tcaseof{ {\tt gauss}({\tt {\tt count}(\phi')})  }$ \\ $\{c_i' \Rightarrow {{\tt gauss}({\tt {\tt count}(\phi_i')})} \}_{c_i' \in b} \}_{c_i \in b}$
\end{tabbing}    

    \caption{simple examples}
    \label{fig:my_label}
\end{figure}

\begin{figure}
%%

Two-rounds:

\begin{tabbing}
    $\mathsf{let}\, !g = ! \big( \mathsf{fix} \, f(j). \lambda k.$ \\
    $\hspace{0.4cm} \mathsf{if}\, (j < k) \, \mathsf{then}$ \\
    $\hspace{0.8cm} \mathsf{let} \, a = M \, ( \lambda x. (x \,j)\cdot (x \,k) )  \, \mathsf{in}$\\
    $\hspace{1.2cm} (a, j) :: (f  \,(j+1) \, k) $\\
    $\hspace{0.4cm} \mathsf{else} \, [] \big)  $\\
    $\mathsf{in} $\\
    $  \mathsf{let}\, !l = !g \, 0\, K $\\
    $\mathsf{in}$\\
    $\mathsf{let} \, ! q = ! \lambda x. \mathsf{sign} \, (\mathsf{foldl} \, (\lambda acc. \lambda (a,i). \big(acc\,+ (x \, i) *lg(\frac{1+a}{1-a})  \big) \, 0 \, l )) $ \\
    $\mathsf{in}$\\
    $ M ( q ) $
\end{tabbing}  

\begin{tabbing}
    $ x: \tint \to  \tint   $ \\
    $ \cdot: \tint * \tint \to \tint $   \\
    $ g: \tint \to \tint \to \tlists{b * \tint} $\\
    $ q: \tquery $\\
    $ M: \tquery \to b $\\
\end{tabbing}
Type derivation:\\

Let $A = \big( \tint \to \tint \to \tlists{b * \tint} \big) $, $\Gamma = f: A, M : \tquery \to b, j: \tint, k: \tint\ $, $\Gamma_0 = M : \tquery \to b$, $[\Delta]_i = g: [A]_i, l:[\tlists{b*\tint}]_i, q:[\tquery]_i$.\\
\[
  \inferrule*[ right = LET-B ]
   {\Pi_L {\vartriangleright M : \tquery \to b \jtype{2}{}{! \mathsf{fix} \, f \cdots}{!_1 A}}
   \and
   \Pi_R {\vartriangleright M : \tquery \to b, g: [A]_1 \jtype{2}{}{\letx{!l}{! g \, 0\, K}{\letx{!q}{\cdots}{\cdots}}}{b}}
   }
   { M: \tquery \to b \jtype{2}{}{\letx{ !g}{! \mathsf{fix} \, f \cdots}{ \letx{!l}{\cdots}{\cdots}}}{b}}
\]

Derivation $\Pi_L$ and $\Pi_R$ are shown as follows:\\
$\Pi_L$:
\begin{mathpar}
  \inferrule*[ right = PR ]
    {
    \inferrule*[ right = FIX ]
    {
    \inferrule*[right = ABS]
    {
    \inferrule*[right = IF]
    {
    \inferrule*[right = BOOL]
    {
    \empty
    }
    {
    \Gamma \jtype{0}{}{j<k}{\tbool}
    }\and
    \inferrule*[right = LET]
    {
    \dots
    }
    {
    \Gamma \jtype{1}{}{\letx{a}{M \, ( \cdots )}{(a, j) :: \cdots}}{\tlists{b*\tint}}
    }\and
    \inferrule*[right = NIL]
    {
    \empty
    }
    {
    \Gamma \jtype{0}{}{[]}{\tlists{b*\tint}}
    }
    }
    {
    f: A, M : \tquery \to b, j: \tint, k: \tint\ \jtype{1}{}{\mathsf{if} \cdots}{ \tlists{b*\tint}}
    }
    }
    {
    f: A, M : \tquery \to b\ \jtype{1}{}{\lambda\, j. \lambda\, k. \mathsf{if} \cdots}{\tint \to \tint \to \tlists{b*\tint}}
    }
    }
    {M: \tquery \to b \jtype{1}{}{\mathsf{fix} \, f \cdots}{ A} }}
    {M: \tquery \to b \jtype{2}{}{! \mathsf{fix} \, f \cdots}{!_1 A}}
    
    
    \inferrule*[right = LET]
    {
    \inferrule*[right = MT]
        {
         \inferrule*[right = QUERY]
            {
                \empty
            }
            {
                \Gamma \jtype{0}{}{ \lambda x. (x \,j)\cdot (x \,k)}{\tquery}
            }
        }
        {
            \Gamma \jtype{1}{}{M\, ( \lambda x. (x \,j)\cdot (x \,k))}{b}
        }
    \and 
    \inferrule*[right = CONS]
        {
        \inferrule*[right = VAR]
            {
                \empty
            }
            {
                \Gamma, a: b \jtype{0}{}{(a, j) }{b*\tint}
            }
        \and
        \inferrule*[right = APP]
            {
                \dots
            }
            {
                \Gamma, a: b \jtype{0}{}{(f \, j+1 \, k) }{\tlists{b*\tint}}
            }
        }
        {
            \Gamma, a: b \jtype{0}{}{(a, j) ::(f \, j+1 \, k) }{\tlists{b*\tint}}
        }
    }
    {
    \Gamma \jtype{1}{}{\letx{a}{M\, ( \lambda x. (x \,j)\cdot (x \,k))}{(a, j) ::(f \, j+1 \, k) }}{\tlists{b*\tint}}
    }
    \end{mathpar}

$\Pi_R$:
\begin{mathpar}
\inferrule*[right = LET-B]
    {
    \inferrule*[right = PR]
    {
    \inferrule*[right = DER]
    {
    \inferrule*[right = APP]
    {
    \dots
    }
    {
        g: A \jtype{0}{}{g \, 0\, K}{\tlists{b*\tint}}
    }
    }
    {
        g: [A]_0 \jtype{0}{}{g \, 0\, K}{\tlists{b*\tint}}
    }
    }
    {
        g: [A]_1 \jtype{1}{}{!g \, 0\, K}{!_1 \tlists{b*\tint}}
    }
    \and
    \inferrule*[right = LET-B]
        {
        \inferrule*[right = PR]
            {
            \inferrule*[right = ABS]
                {
                \inferrule*[right = VAR]
                    {
                        \empty
                    }
                    {
                        x:row \jtype{0}{}{sign \dots}{ b}
                    }
                }
                {
                    \jtype{0}{}{\lambda x. \dots}{row \to b}
                }
            }
            {
                \jtype{1}{}{!\lambda x. \dots}{!_1 row \to b}
            }
        \and
        \inferrule*[right = MT]
        {
        \inferrule*[right = QUERY]
            {
                \emtpy
            }
            {
                \Gamma_0, \Delta_0 \jtype{0}{}{q}{\tquery}
            }
        }
            {
                \Gamma_0, [\Delta]_1 \jtype{1}{}{M(q)}{b}
            }
        }
        {
            \Gamma_0, g: [A]_1, l:[\tlists{b*\tint}]_1  \jtype{1}{}{\letx{!q}{\cdots}{\cdots}}{ b}
        }
    }
    {
        \Gamma_0, g: [A]_1 \jtype{1}{}{\letx{! l}{! g \, 0\, K}{\letx{! q}{ \dots }{ \dots }}}{b}
    }
\end{mathpar}

\caption{examples}
\end{figure}



\begin{algorithm}
\caption{A two-round analyst strategy for random data (Algorithm 4 in ...)}
\label{alg:BitGOF}
\begin{algorithmic}
\REQUIRE Mechanism $\mathcal{M}$ with a hidden state $X\in \{-1,+1\}^{n\times (k+1)}$.
\STATE  {\bf for}\ $j\in [k]$\ {\bf do}.  
\STATE \qquad {\bf define} $q_j(x)=x(j)\cdot x(k)$ where $x\in \{-1,+1\}^{k+1}$.
\STATE \qquad {\bf let} $a_j=\mathcal{M}(q_j)$ 
\STATE \qquad \COMMENT{In the line above, $\mathcal{M}$ computes approx. the exp. value  of $q_j$ over $X$. So, $a_j\in [-1,+1]$.}
\STATE {\bf define} $q_{k+1}(x)=\mathrm{sign}\big (\sum_{i\in [k]} x(i)\times\ln\frac{1+a_i}{1-a_i} \big )$ where $x\in \{-1,+1\}^{k+1}$.
\STATE\COMMENT{In the line above,  $\mathrm{sign}(y)=\left \{ \begin{array}{lr} +1 & \mathrm{if}\ y\geq 0\\ -1 &\mathrm{otherwise} \end{array} \right . $.}
\STATE {\bf let} $a_{k+1}=\mathcal{M}(q_{k+1})$
\STATE\COMMENT{In the line above,  $\mathcal{M}$ computes approx. the exp. value  of $q_{k+1}$ over $X$. So, $a_{k+1}\in [-1,+1]$.}
\RETURN $a_{k+1}$.
\ENSURE $a_{k+1}\in [-1,+1]$
\end{algorithmic}
\end{algorithm}

\clearpage
\begin{figure}
   $$
 \begin{array}{rcl}
     \text{Types} & \quad & \tau ::= \tau_1 + \tau_2 \sep \tlists{\tau}   \\[2mm]

     \text{Term} & \quad & t ::= \einl \, t \sep \einr \, t \sep \enil \sep \econs (t_1,t_2) \sep \letx{x}{t_1}{t_2} \sep \ecase (t, x.t_1, y.t_2)
     
 \end{array}
     $$
     
     \boxed{  \Gamma \jtype{n,m}{}{t}{\tau}    }\\
     
     \begin{mathpar}
  \inferrule*[right = fix]
   { \Gamma, f : \tau_1 \multimap \tau_2, x: \tau_1 \jtype{n}{}{t}{\tau_2} }
   {\Gamma \jtype{n}{}{\fix{t} }{\tau_1 \multimap \tau_2}  }
   
  \inferrule*[ right = let ]
   {\Gamma_1 \jtype{n_1}{}{t}{\tau} \\ \Gamma_2, x:\tau \jtype{n_2}{}{t'}{\tau'}}
   { \max(\Gamma_1, \Gamma_2)  \jtype{\max(n_1,n_2)}{}{\letx{x}{t}{t'}}{\tau'}  }
   
    \inferrule*[ right = inl ]
   {\Gamma \jtype{n_1}{}{t}{\tau_1} }
   { \Gamma  \jtype{n_1}{}{ \einl \, t }{ \tau_1 + \tau_2 }  }
   
    \inferrule*[ right = inr ]
   {\Gamma \jtype{n_2}{}{t}{\tau_2} }
   { \Gamma  \jtype{n_2}{}{ \einr \, t }{ \tau_1 + \tau_2 }  }
   
   \inferrule*[ right = case ]
   {\Gamma_1 \jtype{n_1}{}{t}{\tau_1 + \tau_2} \\ \Gamma_2, x:\tau_1 \jtype{n_2}{}{t_1}{\tau}
   \\ \Gamma_3, y :\tau_2 
   \jtype{n_3}{}{t_2}{\tau}}
   { \max(\Gamma_1, \Gamma_2,\Gamma_3)  \jtype{\max(n_1,n_2,n_3)}{}{ \ecase (t, x.t_1, y.t_2)}{\tau}  }
   
   \inferrule*[right = nil]
   {   \wfty{\tau}   }
   {\Gamma \jtype{0}{}{\enil }{\tlists{\tau} }  }
   
     \inferrule*[ right = cons ]
   {\Gamma_1 \jtype{n_1}{}{t_1}{\tau} \\
    \Gamma_2 \jtype{n_2}{}{t_2}{\tlists{\tau} }}
   { \max(\Gamma_1,\Gamma_2)  \jtype{\max(n_1,n_2)}{}{ \econs \, (t_1,t_2) }{ \tlists{\tau} }  }
   
   \end{mathpar}
    \boxed{  \tau \subseteq \tau  }\\
   \begin{mathpar}
     \inferrule*[right= S-pair]
  {  }
  { ( !_i \tau_1 , !_j \tau_2) \subseteq !_{\max(i,j)} (\tau_1,\tau_2)  }
  
     \inferrule*[right= S-list]
  {  }
  {  \tlists{!_i \tau_1}  \subseteq !_i \tlists{\tau_1}  }
   \end{mathpar}
     
    \caption{New added components}
    \label{fig:my_label}
\end{figure}

\clearpage
\begin{figure}
\boxed{  \Gamma \jtype{n,m}{}{t}{\tau}    }\\
\boxed{  \Gamma :: = \emptyset \sep \Gamma, x : \tau \sep \Gamma, x : [\tau]_p   }

\begin{mathpar}
  \inferrule*[right = const]
   {\empty}
   {\Gamma \jtype{n,m}{}{c}{b}  }
   
   \and
    \inferrule*[right = abs]
   {\Gamma, x: \tau_1 \jtype{n}{}{t}{\tau_2}}
   { \Gamma \jtype{n,m}{}{\abs{x}{t}}{\tau_1 \multimap \tau_2}  }
   
   \and
   \inferrule*[right = pr]
   {[\Gamma] \jtype{n}{}{t}{\tau}}
   {\Delta, p + [\Gamma] \jtype{n}{}{!t}{!_p \tau}  }
%   \boxed{
%   \inferrule*[right = pr]
%   {[\Gamma] \jtype{n}{}{t}{\tau}}
%   {p + [\Gamma]\textbf{} \jtype{n}{}{!t}{!_p \tau}  }}
   
   \and
    \inferrule*[ right = var]
   {\empty}
   {\Gamma, x:\tau \jtype{n}{}{x}{\tau}  } 
   
   \and
   \inferrule*[ right = MT ]
   {[\Gamma] \jtype{n}{}{t}{query}}
   {\Delta, 1 + [\Gamma] \jtype{n+1}{}{M(t)}{b}  }
   
   \and
    \inferrule*[right = query]
   {\empty}
   {\Gamma \jtype{n}{}{q}{query}  }
   
   \and
%     \inferrule*[ right = app ]
%   {\Gamma_1 \jtype{n_1}{}{t_1}{\tau_1 \rightarrow \tau_2} \\ \Gamma \jtype{n_2}{}{t_2}{\tau_1}}
%   { \Gamma_2 \jtype{\max(n_1,n_2)}{}{\app{t_1}{t_2}}{\tau_2}  }
  \inferrule*[ right = app ]
   {\Gamma_1 \jtype{n_1}{}{t_1}{\tau_1 \multimap \tau_2} \\ \Gamma_2 \jtype{n_2}{}{t_2}{\tau_1}}
   { \max(\Gamma_1, \Gamma_2) \jtype{\max(n_1,n_2)}{}{\app{t_1}{t_2}}{\tau_2}  }
   
   \and 
   
   \inferrule*[ right = der ]
   {\Gamma, x: \tau \jtype{n}{}{t}{ \tau }  }
   { \Gamma , x: [\tau]_0 \jtype{n }{}{t }{\tau }  }
   \boxed{
   \inferrule*[ right = der ]
   {\Gamma, x: \tau \jtype{n}{}{t}{ \tau }  }
   { \Gamma , x: [\tau]_1 \jtype{n }{}{t }{\tau }  }
 }
   
   \and 
   
   \inferrule*[ right = let-b ]
   {\Gamma_1 \jtype{n_1}{}{t}{!_p \tau} \\ \Gamma_2, x: [\tau]_p \jtype{n_2}{}{t'}{\tau'}}
   { \max(\Gamma_1, \Gamma_2)  \jtype{\max(n_1,n_2)}{}{\letx{!x}{t}{t'}}{\tau'}  }
   
   \and 
   \inferrule*[ right = let-p ]
   {\Gamma_1 \jtype{n_1}{}{t}{\tau_1 \times \tau_2 } \\ \Gamma_2, x_1: \tau_1, x_2 : \tau_2 \jtype{n_2}{}{t'}{\tau'}}
   { \max(\Gamma_1, \Gamma_2)  \jtype{\max(n_1,n_2)}{}{ \letx{(x_1,x_2)}{t}{t'} }{\tau'}  }
   
   \and
     \inferrule*[right = pair]
   {\Gamma_1 \jtype{n_1}{}{t_1}{\tau_1} \\ \Gamma_2 \jtype{n_2}{}{t_2}{\tau_2}}
   { \max(\Gamma_1, \Gamma_2)  \jtype{\max(n_1,n_2)}{}{(t_1, t_2)}{\tau_1 \times \tau_2}  }
 
   \and
    \inferrule*[ right = case-const ]
   {\Gamma_1 \jtype{n_1}{}{t}{b} \\ \Gamma_2 \jtype{n_2}{}{t_i}{b} }
   {\max(\Gamma_1, \Gamma_2) \jtype{\max(n_1,n_2)}{}{\tcaseof{t}\ \{c_i \Rightarrow t_i\}_{c_i \in b} } {b} }
   
   \and
    \inferrule*[ right = case-query]
   {\Gamma_1 \jtype{n_1}{}{t}{b} \\ \Gamma_2 \jtype{n_2}{}{t_i}{query} }
   {\max(\Gamma_1, \Gamma_2) \jtype{\max(n_1,n_2)}{}{\tcaseof{t}\ \{c_i \Rightarrow t_i\}_{c_i \in b} } {query} }
   
   \inferrule*[right = iabs]
  { 
    \inferrule*[]
    {}
    {i::\mathbb{N};\Gamma \jtype{n}{}{t}{ \tau } }
    \and
    \inferrule*[]
    {}
    { i \notin \fiv{\Gamma }  } 
  }
  { \Gamma \jtype{n}{ }{  \Lambda.t  }{ \tforallN{i}{\tau}  } }
  
   \inferrule*[ right =  iapp]
  { 
    \inferrule*[]
    {}
    { \Gamma  \jtype{n}{}{t}{ \tforallN{i}{\tau}   } }
    \and
    \inferrule*[]
    {}
    { \jiterm{I}{ \mathbb{N} } } 
  }
  {\Gamma \jtype{n }{ }{t\, [] }{ \tau \{ I/i \}  } }

  \inferrule*[right = sub]
  { 
   { \Gamma \jtype{n}{}{t}{\tau} } \\
   { \Gamma \subseteq \Gamma' } \\
   { \vDash n \leq n' } \\
   { \tau \subseteq \tau' }
  }
  { \Gamma' \jtype{n'}{}{t}{\tau'} }
\end{mathpar}
\caption{Typing judgment}
\end{figure}

\clearpage

\begin{figure}
 \begin{mathpar}
  
  \inferrule*[right= S-ID]
  { }
  { \tau \subseteq \tau  }
  \and
  \inferrule*[right = S-B]
  { 
   {A \subseteq B}
   \\
   { q \leq p }
  }
  { !_p A \subseteq !_q B  }
  \and
  \inferrule*[right =  S-ARROW]
  { {A' \subseteq A}
    \\
    {B \subseteq B'}
  }
  { A \multimap B \subseteq A' \multimap B' }
  \and
  \inferrule*[right = S-D ]
  {
    { A \subseteq B }\\
    { q \leq p }
  }
  { [A]_p \subseteq [B]_q }
  
  \and
  \inferrule*[right = S-IDC]
  { }
  { \Gamma \subseteq \Gamma }
  
  \and
  \inferrule*[right = S-Ctx]
  {
  {B \subseteq A  }\\
  {\Gamma \subseteq \Delta}
  }
  {\Gamma, x: A \subseteq \Delta, x: B }
  
 \end{mathpar}
 \caption{sub typing}
\end{figure}



\begin{figure}
\boxed{  \Gamma \jtype{n,m}{}{t}{\tau}    }
\boxed{  \Gamma :: = \emptyset \sep \Gamma, [\tau]_p   } \\
\boxed{ \max \big( (x:[\tau]_{p_1},\Gamma_1 ),(x:[\tau]_{p_2},\Gamma_2) \big ) \triangleq x:[\tau]_{\max(p_1,p_2)}, \max(\Gamma_1,\Gamma_2) \qquad
\max(\emptyset, \Gamma_2) \triangleq \Gamma_2  \qquad \max(\Gamma_1, \emptyset) \triangleq \Gamma_1 
}

\begin{mathpar}
  \inferrule*[right = const]
   {\empty}
   {\Gamma \jtype{0}{}{c}{b}  }
   
   \and
    \inferrule*[right = abs]
   {\Gamma, x: [\tau_1]_{0} \jtype{n}{}{t}{\tau_2}}
   { \Gamma \jtype{n}{}{\abs{x}{t}}{\tau_1 \multimap \tau_2}  }
   
   \and
   \inferrule*[right = pr]
   {\Gamma \jtype{n}{}{t}{\tau}}
   {p + \Gamma \jtype{n}{}{!t}{!_p \tau}  }
   
   \and
    \inferrule*[ right = var]
   {\empty}
   {\Gamma, x:[\tau]_p \jtype{0}{}{x}{\tau}  } 
   
   \and
   \inferrule*[ right = MT ]
   {\Gamma \jtype{n}{}{t}{query}}
   {1 + \Gamma \jtype{n+1}{}{M(t)}{b}  }
   
   \and
    \inferrule*[right = query]
   {\empty}
   {\Gamma \jtype{0}{}{q}{query}  }
   
   \and
    \inferrule*[ right = app ]
   {\Gamma_1 \jtype{n_1}{}{t_1}{\tau_1 \rightarrow \tau_2} \\ \Gamma_2 \jtype{n_2}{}{t_2}{\tau_1}}
   { \max(\Gamma_1,\Gamma_2) \jtype{\max(n_1,n_2)}{}{\app{t_1}{t_2}}{\tau_2}  }
   
   \and 
   
%   \inferrule*[ right = der ]
%   {\Gamma, x: \tau \jtype{n}{}{t}{ \tau }  }
%   { \Gamma , x: [\tau]_p \jtype{n }{}{t }{\tau }  }
   
%   \and 
   
   \inferrule*[ right = let ]
   {\Gamma_1 \jtype{n_1}{}{t}{!_p \tau} \\ \Gamma_2, x: [\tau]_p \jtype{n_2}{}{t'}{\tau'}}
   { \max(\Gamma_1,\Gamma_2) \jtype{\max(n_1,n_2)}{}{\letx{!x}{t}{t'}}{\tau'}  }
   
   \and 
   \inferrule*[ right = let-p ]
   {\Gamma_1 \jtype{n_1}{}{t}{\tau_1 \times \tau_2 } \\ \Gamma_2, x_1: \tau_1, x_2 : \tau_2 \jtype{n_2}{}{t'}{\tau'}}
   { \max(\Gamma_1,\Gamma_2)  \jtype{\max(n_1,n_2)}{}{ \letx{(x_1,x_2)}{t}{t'} }{\tau'}  }
   
   \and
     \inferrule*[right = pair]
   {\Gamma_1 \jtype{n_1}{}{t_1}{\tau_1} \\ \Gamma_2 \jtype{n_2}{}{t_2}{\tau_2}}
   { \max(\Gamma_1,\Gamma_2)  \jtype{\max(n_1,n_2)}{}{(t_1, t_2)}{\tau_1 \times \tau_2}  }
 
   \and
    \inferrule*[ right = case-const ]
   {\Gamma_1 \jtype{n_1}{}{t}{b} \\ \Gamma_2 \jtype{n_2}{}{t_i}{b} }
   {\max(\Gamma_1,\Gamma_2) \jtype{\max(n_1,n_2)}{}{\tcaseof{t}\ \{c_i \Rightarrow t_i\}_{c_i \in b} } {b} }
   
   \and
    \inferrule*[ right = case-query]
   {\Gamma_1 \jtype{n_1}{}{t}{b} \\ \Gamma_2 \jtype{n_2}{}{t_i}{query} }
   {\max(\Gamma_1,\Gamma_2) \jtype{\max(n_1,n_2)}{}{\tcaseof{t}\ \{c_i \Rightarrow t_i\}_{c_i \in b} } {query} }
   
   \inferrule*[right = iabs]
  { 
    \inferrule*[]
    {}
    {i::\mathbb{N};\Gamma \jtype{n}{}{t}{ \tau } }
    \and
    \inferrule*[]
    {}
    { i \notin FV(\Gamma )  } 
  }
  { \Gamma \jtype{n}{ }{  \Lambda.t  }{ \tforallN{i}{\tau}  } }
  
   \inferrule*[ right =  iapp]
  { 
    \inferrule*[]
    {}
    { \Gamma  \jtype{n}{}{t}{ \tforallN{i}{\tau}   } }
    \and
    \inferrule*[]
    {}
    { \jiterm{I}{ \mathbb{N} } } 
  }
  {\Gamma \jtype{n }{ }{t\, [] }{ \tau \{ I/i \}  } }
\end{mathpar}
\caption{Typing judgment (with only discharged variables in the context) }
\end{figure}

\clearpage



\begin{figure}
\boxed{\eval{t }{v }{m}}
\begin{mathpar}
 \inferrule*[ right=E-values]
  { }
  { \eval{ F   }{ F  }{0}}   
  \and
 \inferrule*[ right=E-const]
  { }
  { \eval{ c   }{ c  }{0}}   
  
  \and

 \inferrule*[ right=E-query]
  { }
  { \eval{  q  }{ q  }{0}}   
  
  \and

 \inferrule*[ right=E-ABS]
  { }
  { \eval{ \abs{x}{t}   }{ \abs{x}{t}  }{0}}   
  
  \and
  
  \inferrule*[ right=E-bang]
  { }
  { \eval{ ! t   }{ ! t  }{0}}   
  
  \and
  
   \inferrule*[ right=E-pair]
  {   
    { \eval{ t_1  }{ F_1  }{m_1} }
    \\
    { \eval{ t_2  }{ F_2  }{m_2} } 
  }
  { \eval{  (t_1,t_2)  }{ (F_1,F_2)  }{ m_1+m_2  } }  
  
  \and

   \inferrule*[ right=E-app]
  {   
    { \eval{ t_1  }{ \abs{x}{t}  }{m_1} }
    \\
    { \eval{ t_2  }{ F  }{m_2} } 
    \\
    { \eval{t[F/x] }{ F'}{m_3 } }
  }
  { \eval{ \app{t_1}{t_2}  }{ F'  }{ m_1+m_2+m_3  } }  
 
 \boxed{ 
   \inferrule*[ right=E-let-bang]
  {   
    { \eval{ t_1  }{ !t_3  }{m_1} } 
    \\
    {\eval{t_3}{F'}{m_2}}
    \\
    { \eval{t_2[F'/x] }{ F}{m_3 } }
  }
  { \eval{  \letx{!x}{t_1}{t_2}  }{ F  }{ m_1+m_2+m_3  } }  
}

 \inferrule*[ right=E-let-bang]
  {   
    { \eval{ t_1  }{ !t_3  }{m_1} } 
    \\
    { \eval{t_2[!t_3 /x] }{ F}{m_3 } }
  }
  { \eval{  \letx{!x}{t_1}{t_2}  }{ F  }{ m_1+m_2+m_3  } } 
  
  \inferrule*[ right=E-let-p]
  {   
    { \eval{ t  }{ (F_1,F_2)  }{m_1} } 
    \\
    { \eval{t'[F_1/x_1][F_2/x_2] }{ F}{m_3 } }
  }
  { \eval{  \letx{(x_1,x_2)}{t}{t'}  }{ F  }{ m_1+m_2+m_3  } } 
  
 

  \inferrule*[ right=E-case]
  { 
    \inferrule*[]
    {}
    {\eval{  t  }{ F }{m }  }
    \\
    \inferrule*[]
    {}
    { \eval{ t_i  }{ F_i   }{ m_i }  }
  }
  { \eval{ \tcaseof{t}\ \{c_i \Rightarrow t_i\}_{c_i \in b}  }{ \tcaseof{F}\ \{c_i \Rightarrow F_i\}_{c_i \in b}  }{  m + m_i } }
  
    \inferrule*[ right=E-fix]
  { 
  }
  { \eval{  \fix{t}  }{ \fix{t}  }{ m } }
  
      \inferrule*[ right=E-x]
  { 
  \empty
  }
  { \eval{  x  }{ x  }{ 0 } }
  
      \inferrule*[ right=E-ILAM]
  { 
    \empty
  }
  { \eval{  \Lambda. t  }{  \Lambda. t }{ 0 } }
  
      \inferrule*[ right=E-iapp]
  { 
    \inferrule*[]
    {}
    {\eval{  t  }{ \Lambda. t' }{m }  }
  }
  { \eval{  t[]  }{  t' }{  m } }
  
      \inferrule*[ right=E-mech]
  { 
    \inferrule*[]
    {}
    {\eval{  t  }{ F }{m }  }
  }
  { \eval{  M(t)  }{ M(F)  }{  m } }
  
\end{mathpar}
\end{figure}


\begin{figure*}
$$
\begin{array}{rcl}
      \llu{\tau}{\epsilon}  &\quad &  = \{ \, e \, | \, \exists F: e \Downarrow F \land F \in   \llu{\tau}{v} \,  \}  \\[2mm]
      \llu{b}{v} &\quad &  = \{ \,  F  \, | \, F = T_b \}  \\[2mm]
      \llu{query}{v} &\quad &  = \{ \,  F \, |  \, F = T_{query} \}  \\[2mm]
      \llu{\tau_1 \rightarrow \tau_2}{v} & \quad & = \{\, \abs{x}{t} \, | \, \forall v \in\llu{\tau}{v}. t[v/x] \in \llu{\tau_2}{\epsilon} \, \} \\[2mm]
      \llu{ \ !_n \tau}{v} & \quad & = \{\, !t \, | \, t \in \llu{\tau}{\epsilon} \, \} \\[2mm]
      \llu{\tforallN{i}{\tau}}{v}  & \quad & = \{  \Lambda. t \, | \, \forall I. \vdash i :: \mathbb{N}. t[I/i] \in \llu{\tau}{\epsilon}   \}  \\[2mm]
      \llu{\tau_1 * \tau_2}{v}  & \quad & = \{  (F_1, F_2) \, | \, F_1 \in \llu{\tau_1}{v} \land F_2 \in \llu{\tau_2}{v}     \} \\[2mm]
      \llu{\cdot}{} &\quad & = \{ \emptyset \} \\[2mm]
    %   \llu{\Gamma, x : [\tau]_p}{} & \quad & = \{ \gamma[x \rightarrow v] | v \in \llu{\tau}{v} \land \gamma \in \llu{\Gamma}{}   \}  \\ [2mm]
      \llu{\Gamma, x : [\tau]_p}{} & \quad & = \{ \gamma[x \rightarrow v] | v \in \llu{!_p \tau}{v} \land \gamma \in \llu{\Gamma}{}   \}  \\ [2mm]
      \llu{\Gamma, x : \tau}{} & \quad & = \{ \gamma[x \rightarrow v] | v \in \llu{\tau}{v} \land \gamma \in \llu{\Gamma}{}   \}  \\ [2mm]
      \gamma \vDash \Gamma &\quad & \triangleq dom(\gamma) = dom(\Gamma) \land \forall x \in dom(\Gamma). \gamma(x) \in \llu{\Gamma(x)}{v}
\end{array}
$$
\caption{denotations}
\end{figure*}

\clearpage


\begin{lemma} $ $
	\label{lem:1}
    \begin{enumerate}
\item If $\jtype{n,m}{}{F}{b} $ then $ \exists T_{b} : F = T_{b}$.\\
\item If $\jtype{n,m}{}{F}{query} $ then $ \exists T_{query} : F = T_{query}$
\end{enumerate}
	
	
\end{lemma}

\clearpage
\begin{lemma}[Depth Definition] 
	\label{lem:2}
	If $\Gamma \jtype{n,m}{}{t}{\tau} $ then $\depth(t) \leq n$\\
\end{lemma}
\begin{proof}
 It is proved by the induction on the structure of the typing derivation.\\
 \noindent \textbf{Case} 
 \[
 \inferrule*[right = pr]
   {\Gamma \jtype{n}{}{t}{\tau} ~(\star) } 
   {p + \Gamma \jtype{n}{}{!t}{!_p \tau}  }
 \]
 TS: $\depth(!t) \leq n $.\\
 By IH on $(\star)$, we get $\depth(t) \leq n $ \\
 This case is proved because $\depth(!t)= \depth(t)$.
 
 \noindent \textbf{Case} 
 \[
  \inferrule*[right = const]
   {\empty}
   {\Gamma \jtype{0}{}{c}{b}  }
 \]
 TS: $\depth(c) \leq 0$ \\
 It is already proved by the definition of $\depth(c)$.
 
  \noindent \textbf{Case} 
  \[
   \inferrule*[right = abs]
   {\Gamma, x: [\tau_1]_{0} \jtype{n}{}{t}{\tau_2} ~(\star)}
   { \Gamma \jtype{n}{}{\abs{x}{t}}{\tau_1 \multimap \tau_2}  }
  \]
  TS: $\depth(\abs{x}{t}) \leq n $. \\
  By IH on ~($\star$) instantiating the context with $\Gamma, x: [\tau]_0 $, we get : $\depth(t) \leq n $. \\
    This case is proved by the definition of $\depth(\abs{x}{t})$.
  
   \noindent \textbf{Case} 
  \[
   \inferrule*[ right = MT ]
   {\Gamma \jtype{n}{}{t}{query} ~ (\star) }
   {1 + \Gamma \jtype{n+1}{}{M(t)}{b}  }
  \]
   TS: $\depth( M(t) ) \leq n+1 $. \\
   By IH on ($\star$), we get $\depth(t) \leq n $. \\
   It is proved by the definition of $\depth(M(t)) = \depth(t) + 1 $.
   
   \noindent \textbf{Case} 
   \[
    \inferrule*[ right = var]
   {\empty}
   {\Gamma, x:[\tau]_p \jtype{0}{}{x}{\tau}  } 
   \]
   TS: $\depth(x) \leq 0$. \\
   It is proved by the definition of $\depth(x)$. \\
   
    \noindent \textbf{Case} 
    \[
     \inferrule*[ right = app ]
   {\Gamma_1 \jtype{n_1}{}{t_1}{\tau_1 \rightarrow \tau_2} ~ (\star) \\ \Gamma_2 \jtype{n_2}{}{t_2}{\tau_1}~(\diamond)}
   { \max(\Gamma_1,\Gamma_2) \jtype{\max(n_1,n_2)}{}{\app{t_1}{t_2}}{\tau_2}  }
    \]
    TS: $\depth(\app{t_1}{t_2}) \leq \max(n_1,n_2) $. \\
    By IH on ($\star$) and ($\diamond$), we get: $\depth(t_1) \leq n_1~(\star\star) $ and $\depth(t_2) \leq n_2 (\diamond\diamond)$. \\
    Unfold the definition of $\depth(\app{t_1}{t_2}) = \max(\depth(t_1), \depth(t_2))$.\\
    This case is proved by the ($\star\star$) and ($\diamond\diamond$).
    
    \noindent \textbf{Case}
 \[
 \inferrule*[right = query]
   {\empty}
   {\Gamma \jtype{0}{}{q}{query}  }
   \]
TS: $\depth(q) \leq 0$.\\
This is proved by the definition of $\depth(q)$.

 \noindent \textbf{Case} 
   \[
   \inferrule*[ right = let ]
   {\Gamma_1 \jtype{n_1}{}{t}{!_p \tau} ~ (\star) \\ \Gamma_2, x: [\tau]_p \jtype{n_2}{}{t'}{\tau'}  ~ (\diamond)}
   { \max(\Gamma_1,\Gamma_2) \jtype{\max(n_1,n_2)}{}{\letx{!x}{t}{t'}}{\tau'}  }
   \]
   To show: $\depth(\letx{!x}{t}{t'}) \leq \max(n_1,n_2)$.\\
     By induction Hypothesis on ($\star$) and ($\diamond$), we get:
   $\depth(t) \leq n_1~(\star\star)$ and $\depth(t') \leq n_2 ~ (\diamond\diamond)$.\\
   Unfolding the definition of $\depth(\letx{!x}{t}{t'}) = \max(\depth(t), \depth(t'))$.\\
   This case is proved by the ($\star\star$) and ($\diamond\diamond$).
   
 \noindent \textbf{Case} 
   \[   
   \inferrule*[ right = let-p ]
   {\Gamma_1 \jtype{n_1}{}{t}{\tau_1 \times \tau_2 } ~ (\star)  \\ \Gamma_2, x_1: \tau_1, x_2 : \tau_2 \jtype{n_2}{}{t'}{\tau'}  ~ (\diamond)}
   { \max(\Gamma_1,\Gamma_2)  \jtype{\max(n_1,n_2)}{}{ \letx{(x_1,x_2)}{t}{t'} }{\tau'}  }
    \]
     To show: $\depth(\letx{(x_1, x_2)}{t}{t'}) \leq \max(n_1,n_2)$.\\
     By induction Hypothesis on ($\star$) and ($\diamond$), we get:
   $\depth(t) \leq n_1~(\star\star)$ and $\depth(t') \leq n_2 ~ (\diamond\diamond)$.\\
   Unfolding the definition of $\depth(\letx{(x_1, x_2)}{t}{t'}) = \max(\depth(t), \depth(t'))$.\\
   This case is proved by the ($\star\star$) and ($\diamond\diamond$).
 
 \noindent \textbf{Case} 
   \[
     \inferrule*[right = pair]
   {\Gamma_1 \jtype{n_1}{}{t_1}{\tau_1} ~ (\star)  \\ \Gamma_2 \jtype{n_2}{}{t_2}{\tau_2} ~ (\diamond)}
   { \max(\Gamma_1,\Gamma_2)  \jtype{\max(n_1,n_2)}{}{(t_1, t_2)}{\tau_1 \times \tau_2}  }
    \]
    To show: $\depth((t_1, t_2)) \leq \max(n_1,n_2)$.\\
    By induction Hypothesis on ($\star$) and ($\diamond$), we get:
    $\depth(t_1) \leq n_1~(\star\star)$ and $\depth(t_2) \leq n_2 ~ (\diamond\diamond)$.\\
    Unfolding the definition of $\depth((t_1, t_2)) = \max(\depth(t_1), \depth(t_2))$.\\
    This case is proved by the ($\star\star$) and ($\diamond\diamond$).
 
 \noindent \textbf{Case} 
   \[
    \inferrule*[ right = case-const ]
   {\Gamma_1 \jtype{n_1}{}{t}{b} ~ (\star) \\ \Gamma_2 \jtype{n_2}{}{t_i}{b} ~ (\diamond) }
   {\max(\Gamma_1,\Gamma_2) \jtype{\max(n_1,n_2)}{}{\tcaseof{t}\ \{c_i \Rightarrow t_i\}_{c_i \in b} } {b} }
   \]
    To show: $\depth( \tcaseof{t}\ \{c_i \Rightarrow t_i\}_{c_i \in b} ) \leq \max(n_1,n_2)$.\\
    By induction Hypothesis on ($\star$) and ($\diamond$), we get:
    $\depth(t) \leq n_1~(\star\star)$ and $\depth(t_i) \leq n_2 ~ (\diamond\diamond)$.\\
    Unfolding the definition of $\depth(\tcaseof{t}\ \{c_i \Rightarrow t_i\}_{c_i \in b}) = \max(\depth(t), \depth(t_i))$.\\
    This case is proved by the ($\star\star$) and ($\diamond\diamond$).
   
 \noindent \textbf{Case} 
   \[   
    \inferrule*[ right = case-query]
   {\Gamma_1 \jtype{n_1}{}{t}{b} ~ (\star)  \\ \Gamma_2 \jtype{n_2}{}{t_i}{query}~ (\diamond)  }
   {\max(\Gamma_1,\Gamma_2) \jtype{\max(n_1,n_2)}{}{\tcaseof{t}\ \{c_i \Rightarrow t_i\}_{c_i \in b} } {query} }
   \]
    To show: $\depth( \tcaseof{t}\ \{c_i \Rightarrow t_i\}_{c_i \in b} ) \leq \max(n_1,n_2)$.\\
    By induction Hypothesis on ($\star$) and ($\diamond$), we get:
    $\depth(t) \leq n_1~(\star\star)$ and $\depth(t_i) \leq n_2 ~ (\diamond\diamond)$.\\
    Unfolding the definition of $\depth(\tcaseof{t}\ \{c_i \Rightarrow t_i\}_{c_i \in b}) = \max(\depth(t), \depth(t_i))$.\\
    This case is proved by the ($\star\star$) and ($\diamond\diamond$).
  
 \noindent \textbf{Case} 
   \[
   \inferrule*[right = iabs]
  { 
    \inferrule*[]
    {}
    {i::\mathbb{N};\Gamma \jtype{n}{}{t}{ \tau } ~ (\diamond)  }
    \and
    \inferrule*[]
    {}
    { i \notin FV(\Gamma )  } 
  }
  { \Gamma \jtype{n}{ }{  \Lambda.t  }{ \tforallN{i}{\tau}  } }
    \]
    To show: $\depth( \Lambda.t ) \leq n$.\\
    By induction Hypothesis on ($\diamond$), we get:
    $\depth(t) \leq n ~ (\diamond\diamond)$.\\
    Unfolding the definition of $\depth( \Lambda.t ) = \depth(t)$.\\
    This case is proved by the ($\diamond\diamond$).
    
 \noindent \textbf{Case} 
   \[ 
   \inferrule*[ right =  iapp]
  { 
    \inferrule*[]
    {}
    { \Gamma  \jtype{n}{}{t}{ \tforallN{i}{\tau}   } ~ (\diamond) }
    \and
    \inferrule*[]
    {}
    { \jiterm{I}{ \mathbb{N} } } 
  }
  {\Gamma \jtype{n }{ }{t\, [] }{ \tau \{ I/i \}  } }
\]
    To show: $\depth(t\, []) \leq n$.\\
    By induction Hypothesis on ($\diamond$), we get:
    $\depth(t) \leq n ~ (\diamond\diamond)$.\\
    Unfolding the definition of $\depth(t\, []) = \depth(t)$.\\
    This case is proved by the ($\diamond\diamond$).
  
\end{proof}

\begin{lemma}[Depth Weakening1] 
	\label{lem:deweaken1}
	$\Gamma \jtype{n_1,m}{}{e}{\tau} \land n_1 \leq n_2 \implies \Gamma \jtype{n_2,m}{}{e}{\tau}$\\
\end{lemma}

\begin{lemma}[Depth Weakening2] 
	\label{lem:deweaken2}
	$\Gamma, x:[\tau]_{p_1} \jtype{n}{}{e}{\tau} \land p_1 \leq p_2 \implies \Gamma, x:[\tau]_{p_2} \jtype{n}{}{e}{\tau} $\\
	$\Gamma, x:[\tau]_{p_1} \jtype{n}{}{e}{\tau} \land p_1 \leq p_2 \land \Gamma \subseteq \Gamma' \land n \leq n' \implies \Gamma', x:[\tau]_{p_2} \jtype{n'}{}{e}{\tau} $\\
\end{lemma}

\begin{lemma}[Context weakening - 1]
    \label{lem:coweaken1}
    $\Gamma \jtype{n,m}{}{e}{\tau}  \implies \Gamma, x:\tau \jtype{n,m}{}{e}{\tau} $\\
\end{lemma}

\begin{lemma}[Context weakening - 2]
    \label{lem:coweaken2}
    $\Gamma \jtype{n,m}{}{e}{\tau}  \implies \Gamma, x:[\tau]_p \jtype{n,m}{}{e}{\tau} $\\
\end{lemma}

\begin{lemma}[Context exchange]
    \label{lem:coex}
    $\Gamma, x : \tau_1, \Delta, y : \tau_2 \jtype{n,m}{}{e}{\tau}  \implies \Gamma, y : \tau_2, \Delta, x : \tau_1 \jtype{n,m}{}{e}{\tau} $\\
\end{lemma}

\begin{lemma}
	\label{lem:sub}
	If $\Gamma \jtype{n}{}{t}{\tau}$ and $\gamma \vDash \Gamma$, then $ \cdot \jtype{n}{}{\gamma(t)}{\tau} $\\
\end{lemma}

\begin{lemma}
	\label{lem:subext}
	If $\Gamma \subseteq \Gamma'$ and $\gamma \vDash \Gamma$ and $\gamma' \vDash \Gamma'$ and $ \cdot \jtype{n}{}{\gamma(t)}{\tau} $, then $ \cdot \jtype{n}{}{\gamma'(t)}{\tau} $ \\
\end{lemma}

\begin{theorem}[Type Safety]
	If $\cdot \jtype{n,m}{}{t}{\tau} $ then $ \exists F. t \Downarrow F \land \jtype{n,m}{}{F}{\tau}$
\end{theorem}
  \begin{proof}
   We prove this theorem by prove Normalization and Preservation.
  \end{proof}

\begin{corollary}
\label{corollary}
	If $ \cdot\jtype{n,m}{}{t}{b} $ then $ \exists T_b. t \Downarrow T_b \land \depth(T_b) \leq n$
\end{corollary}
  

\clearpage

\begin{theorem}[Normalization] 
	If $\cdot\jtype{n,m}{}{t}{\tau} $ then $ \exists F: t \Downarrow F$
\end{theorem}
We prove two theorems instead.

\begin{theorem}
If $\gamma(t) \in \llu{\tau}{\epsilon} $, then $\exists F.\eval{ \gamma(t)  }{ F}{}$.
\end{theorem}
\begin{proof}
  It is proved by unfolding the definition of $\llu{\tau}{\epsilon}$.
\end{proof}
 
\begin{theorem}
 If $\Gamma \jtype{n}{}{t}{\tau}$ and $\gamma \vDash{\Gamma}$, then $\gamma(t) \in \llu{\tau}{\epsilon} $.
\end{theorem} 



 \begin{proof}  Proof by induction on  $\Gamma \jtype{n}{}{t}{\tau} $\\
 We have $ \gamma \vDash \Gamma ~(\spadesuit) $. \\
   \noindent \textbf{Case} 
 \[
 \inferrule*[right = abs]
   {\Gamma, x: \tau_1 \jtype{n}{}{t_1}{\tau_2}~(\star)}
   { \Gamma \jtype{n,m}{}{\abs{x}{t_1}}{\tau_1 \multimap \tau_2}  }
 \]
 TS : $ \gamma(\abs{x}{t}) \in \llu{\tau_1 \multimap \tau_2}{\epsilon} $ \\
 Because $\gamma(\abs{x}{t_1}) =\abs{x}{\gamma(t_1)} $ is value, unfold the definition of $\llu{\tau_1 \multimap \tau_2}{\epsilon}$, STS: $ \gamma(\abs{x}{t_1}) \in \llu{\tau_1 \multimap \tau_2}{v} $\\
 Unfold the definition of $\llu{\tau_1 \multimap \tau_2}{v}$, STS: $\forall v. v \in \llu{\tau_1}{v}. \gamma(t_1)[v/x] \in \llu{\tau_2}{\epsilon}  $.\\
 Pick $v$ s.t $ v \in \llu{\tau_1}{v}$. STS: $\gamma(t_1)[v/x] \in \llu{\tau_2}{\epsilon} $.\\
 We have $\gamma[x \rightarrow v] \vDash \Gamma, x: \tau_1 \ \ (\diamond) $ because $\gamma \vDash \Gamma$ and $ v \in \llu{\tau_1}{v} $(the assumption). \\
 By IH on $(\star)$ and $(\diamond)$, we have : \\
   $$ \gamma[v/x](t_1) \in \llu{\tau_2}{\epsilon}  $$
 Because $\gamma[v/x] (t_1) =  \gamma(t_1[v/x])$, this case is proved.\\
 
 \noindent \textbf{Case} 
 \[
 \inferrule*[right = pr]
   {[\Gamma] \jtype{n}{}{t_1}{\tau_1}}
   {\Delta, p + [\Gamma] \jtype{n}{}{!t_1}{!_p \tau_1}  }
 \]
 We assume $ \gamma \vDash \Delta, p+[\Gamma] ~(\spadesuit) $. \\
 TS : $ \gamma(!t_1) \in \llu{!_p \tau}{\epsilon} $ \\
 Because $\gamma ( !t)= ! \gamma(t) $ is value, unfold the definition of $\llu{!_p \tau}{\epsilon}$, STS: $ ! \gamma(t) \in \llu{!_p \tau}{v} $\\
 Unfold the definition of $\llu{!_p \tau}{v} $, STS: $\gamma (t) \in \llu{\tau}{\epsilon}$. \\
 We pick $\gamma_1$ s.t. $\gamma_1 \vDash [\Gamma] $, By ih, we get : $\gamma'(t_1) \in \llu{\tau_1}{\epsilon} ~(\star) $.\\
 By Lemma ~\ref{lem:subext} on $(\star)$ and we know $ [\Gamma] \subseteq \Delta,p+[\Gamma] $, This case is proved.\\
 
 
  \noindent \textbf{Case} 
  \[
    \inferrule*[ right = let ]
   {\Gamma_1 \jtype{n_1}{}{t_1}{!_p \tau_1}~(\star) \\ \Gamma_2, x: [\tau_1]_p \jtype{n_2}{}{t_1'}{\tau'}~(\diamond)}
   { \max(\Gamma_1,\Gamma_2) \jtype{\max(n_1,n_2)}{}{\letx{!x}{t_1}{t_1'}}{\tau'}  }
  \]
  We assume $\gamma \vDash \max(\Gamma_1,\Gamma_2 ) ~(\spadesuit) $.
  TS: $ \gamma ( \letx{!x}{t_1}{t_1'} ) \in \llu{ \tau'}{\epsilon} $. \\
  Unfold $ \llu{ \tau'}{\epsilon} $, STS: $\exists F. \eval{\gamma ( \letx{!x}{t}{t'} )}{F}{} \land F \in \llu{ \tau'}{v} $.
  
  Pick $\gamma_1$ s.t. $\gamma_1 \vDash \Gamma_1 $. By ih on $(\star)$ we get : $ \gamma_1(t_1) \in \llu{!_p \tau_1 }{\epsilon}$. \\
  By Lemma~\ref{lem:subext} and $\Gamma_1 \subseteq \max(\Gamma_1,\Gamma_2)$, we know: $ \gamma(t_1) \in \llu{!_p \tau_1 }{\epsilon}~(1) $\\
  Unfold (1), we get: $\exists F. \eval{\gamma(t_1)}{F}{} \land F \in \llu{ !_p \tau}{v}~(2)  $.  
 
  Pick $\gamma_2$ s.t. $\gamma_2 \vDash \Gamma_2 \implies \gamma_2[F/x] \vDash \Gamma_2, x:[\tau_1]_p $. By ih on $(\diamond)$ we get : $ \gamma_2[F/x](t_1') \in \llu{ \tau' }{\epsilon} ~(3)$. \\
  By Lemma~\ref{lem:subext} and $\Gamma_2, x:[\tau_1]_p \subseteq \max(\Gamma_1,\Gamma_2), x:[\tau_1]_p$, we know: $ \gamma[F/x](t_1') \in \llu{ \tau' }{\epsilon}~(4) $\\
  

    It is proved by unfolding (2),(4) and the following evaluation rule \rname{E-LET-BANG}.
   \[
   \inferrule*[ right=E-let-bang]
  {   
    { \eval{ \gamma(t_1) }{ !t_2  }{m_1} } 
    \\
    { \eval{\gamma[!t_2/x](t_1') }{ F'}{m_3 } }
  }
  { \eval{  \letx{!x}{\gamma(t_1)}{\gamma(t_1')}  }{ F'  }{ m_1+m_2+m_3  } }  
   \]
  
  
   \noindent \textbf{Case} 
   \[
      \inferrule*[ right = MT ]
   {[\Gamma] \jtype{n}{}{t}{query}}
   {\Delta, 1 + [\Gamma] \jtype{n+1}{}{M(t)}{b}  }
   \]
   Assume $\gamma \vDash{\Delta,1+[\Gamma]}~(\spadesuit)$.\\
   TS: $\gamma(M(t)) \in \llu{b}{\epsilon}$.\\
   STS: $\exists F. \eval{M(\gamma(t))}{F}{} \land F \in \llu{b}{v} \implies \exists T_{b}. \eval{M(\gamma(t))}{T_{b}}{}$.\\
   We assume $\gamma' \vDash{[\Gamma]} $,
   by ih we get : $ \gamma'(t) \in \llu{query}{\epsilon}~(1)$.\\
   By Lemma ~\ref{lem:subext} and $[\Gamma] \subseteq \Delta, 1+[\Gamma] $, we get $ \gamma(t) \in \llu{query}{\epsilon}~(2)$\\
   Unfold (2), we know: $\exists F. \eval{\gamma(t)}{F}{} \land F \in \llu{query}{v} \implies F = T_{query} ~(3) $. 
   
   It is proved by \rname{E-MECH} and (3), $T_b =M( T_{query})$.
   \[
     \inferrule*[ right=E-mech]
  { 
    \inferrule*[]
    {}
    {\eval{  \gamma(t)  }{ T_{query} }{m }  }
  }
  { \eval{  M(t)  }{ M(T_{query})  }{  m } }
   \]
  
    
     \noindent \textbf{Case} 
     \[
       \inferrule*[ right = var]
   {\empty}
   {\Gamma, x:\tau \jtype{n}{}{x}{\tau}  } 
     \]
    Assume $\gamma \vDash \Gamma, x: \tau $.\\
    TS: $\gamma(x) \in \llu{\tau}{\epsilon} $.\\
    Unfold the definition of $\gamma \vDash{\Gamma, x: \tau}$, we know : $ \gamma(x) \in \llu{\tau}{v} $. \\
    STS: $ F \in \llu{\tau}{\epsilon} $.\\
    
   It is proved by the assumption and the evaluation rule \rname{E-Value}.
 
 \noindent \textbf{Case}
  \[
   \inferrule*[ right = case-const ]
   {\Gamma_1 \jtype{n_1}{}{t}{b} ~(\star) \\ \Gamma_2 \jtype{n_2}{}{t_i}{b}~(\diamond) }
   {\max(\Gamma_1, \Gamma_2) \jtype{\max(n_1,n_2)}{}{\tcaseof{t}\ \{c_i \Rightarrow t_i\}_{c_i \in b} } {b} }
  \]
  We assume $ \gamma \vDash \max(\Gamma_1,\Gamma_2)$.\\
  TS: $\gamma( \tcaseof{t}\ \{c_i \Rightarrow t_i\}_{c_i \in b}  )  \in \llu{b}{\epsilon} \implies \tcaseof{\gamma(t)}\ \{c_i \Rightarrow \gamma(t_i)\}_{c_i \in b} \in \llu{b}{\epsilon}$.\\
  We assume $\gamma_1 \vDash \Gamma_1$ and $\gamma_2 \vDash \Gamma_2 $.\\
  By ih on $(\star)$, we get : $ \gamma_1(t) \in \llu{b}{\epsilon}$.\\
  By ih on $(\diamond)$, we get : $ \gamma_2(t_i) \in \llu{b}{\epsilon}$.\\
  By Lemma ~\ref{lem:subext} and $\Gamma_1 \subseteq \max(\Gamma_1,\Gamma_2) $ and $ \Gamma_2 \subseteq \max(\Gamma_1,\Gamma_2) $, we get: $\gamma(t) \in \llu{b}{\epsilon}~(1) $ and $ \gamma(t_i) \in \llu{b}{\epsilon}~(2)$.\\
  It is proved by unfolding (1) and (2) and using the evaluation rule \rname{E-CASE} and the definition of $T_b$.\\ 
  
  \noindent \textbf{Case}
  \[
   \inferrule*[ right = case-query]
   {\Gamma_1 \jtype{n_1}{}{t}{b}(\star) \\ \Gamma_2 \jtype{n_2}{}{t_i}{query}~(\diamond) }
   {\max(\Gamma_1, \Gamma_2) \jtype{\max(n_1,n_2)}{}{\tcaseof{t}\ \{c_i \Rightarrow t_i\}_{c_i \in b} } {query} }
  \]
  We assume $\gamma \vDash \max(\Gamma_1, \Gamma_2) $.\\
  TS: $\gamma(\tcaseof{t}\ \{c_i \Rightarrow t_i\}_{c_i \in b}) \in \llu{query}{\epsilon} \implies \tcaseof{\gamma(t)}\ \{c_i \Rightarrow \gamma(t_i)\}_{c_i \in b} \in \llu{query}{\epsilon} $.\\
  We assume $\gamma_1 \vDash \Gamma_1$ and $\gamma_2 \vDash \Gamma_2 $.\\
  By ih on $(\star)$, we get : $ \gamma_1(t) \in \llu{b}{\epsilon}$.\\
  By ih on $(\diamond)$, we get : $ \gamma_2(t_i) \in \llu{query}{\epsilon}$.\\
  By Lemma ~\ref{lem:subext} and $\Gamma_1 \subseteq \max(\Gamma_1,\Gamma_2) $ and $ \Gamma_2 \subseteq \max(\Gamma_1,\Gamma_2) $, we get: $\gamma(t) \in \llu{b}{\epsilon}~(1) $ and $ \gamma(t_i) \in \llu{query}{\epsilon}~(2)$.\\
   It is proved by unfolding (1) and (2) and using the evaluation rule \rname{E-CASE} and the definition of $T_{query}$.\\ 
  
  
 
 \end{proof}

\clearpage

\begin{theorem}[Preservation]
	If $\cdot\jtype{n}{}{t}{\tau} \land t \Downarrow F$ then $ \jtype{n}{}{F}{\tau} $
\end{theorem}
\begin{proof}
  By induction on typing derivation of $\cdot \jtype{n}{}{t}{\tau} $.\\
  
  \noindent \textbf{Case}
  \[
   \inferrule*[right = const]
   {\empty}
   {\cdot \jtype{n}{}{c}{b}  }
  \]
  $t$ is $c$, From \rname{E-CONST}, we know $F$ is $c$, It is proved.\\
 
 For the cases \rname{ABS},\rname{QUERY},\rname{ILAM},\rname{VAR},\rname{PR}, the proof are similar as the one for \rname{CONST} because $t$ in these cases are values.\\
 
 \noindent \textbf{Case}
 \[
  \inferrule*[ right = MT ]
   {\cdot \jtype{n_1}{}{t_1}{query}~(\star)}
   {\cdot \jtype{n_1+1}{}{M(t_1)}{b}  }
 \]
  $t$ is $M(t_1)$, from the rule \rname{E-MECH}, we get: 
  \[
      \inferrule*[ right=E-mech]
  { 
    \inferrule*[]
    {}
    {\eval{  t_1  }{ F }{m }~(\diamond)  }
  }
  { \eval{  M(t_1)  }{ M(F)  }{  m } }
  \]
  
  By ih on ($\star$) and ($\diamond$), we get : 
  $ \cdot \jtype{n_1}{}{F}{query} $. \\
  Using the rule \rname{MT}, we conclude $\cdot \jtype{n_1+1}{}{M(F)}{b} $. This case is proved.\\
  
  \noindent \textbf{Case}
  \[
  \inferrule*[right = pair]
   {\cdot \jtype{n_1}{}{t_1}{\tau_1}~(\star) \\ \cdot \jtype{n_2}{}{t_2}{\tau_2}~(\diamond) }
   { \cdot \jtype{\max(n_1,n_2)}{}{(t_1, t_2)}{\tau_1 \times \tau_2}  }
  \]
  $t$ is $(t_1,t_2)$, from the evaluation rule \rname{E-PAIR}, we know:
  
  \[
   \inferrule*[ right=E-pair]
  {   
    { \eval{ t_1  }{ F_1  }{m_1}~(\star\star) }
    \\
    { \eval{ t_2  }{ F_2  }{m_2}~(\diamond\diamond) } 
  }
  { \eval{  (t_1,t_2)  }{ (F_1,F_2)  }{ m_1+m_2  } } 
  \]
  By ih on ($\star$) and ($\star\star$), we get: $ \cdot \jtype{n_1}{}{F_1}{\tau_1} $.\\
  By ih on ($\diamond$) and ($\diamond\diamond$), we get: $ \cdot \jtype{n_2}{}{F_2}{\tau_2} $.\\
  This case is proved by using the rule \rname{PAIR}.\\
  
   \noindent \textbf{Case}
   \[
   \inferrule*[ right = app ]
   {\cdot \jtype{n_1}{}{t_1}{\tau_1 \multimap \tau_2}~(\star) \\ \cdot \jtype{n_2}{}{t_2}{\tau_1}~(\diamond)}
   { \cdot \jtype{\max(n_1,n_2)}{}{\app{t_1}{t_2}}{\tau_2}  }
   \]
   $t$ is $\app{t_1}{t_2}$, from the evaluation rule \rname{E-APP}, we know:
   \[
    \inferrule*[ right=E-app]
  {   
    { \eval{ t_1  }{ \abs{x}{t'}  }{m_1} ~(\star\star) }
    \\
    { \eval{ t_2  }{ F  }{m_2}~(\diamond\diamond) } 
    \\
    { \eval{t'[F/x] }{ F'}{m_3 }~(\heartsuit) }
  }
  { \eval{ \app{t_1}{t_2}  }{ F'  }{ m_1+m_2+m_3  } }
   \]
   By ih on ($\star$) and ($\star\star$), we get: $ \cdot \jtype{n_1}{}{\abs{x}{t'}}{\tau_1 \multimap \tau_2}~(\spadesuit) $.\\
   By inversion on ($\spadesuit$), we get: $x:\tau_1 \jtype{n_1}{}{t'}{\tau_2}(\spadesuit\spadesuit)$.\\
  By ih on ($\diamond$) and ($\diamond\diamond$), we get: $ \cdot \jtype{n_2}{}{F}{\tau_1} ~(\clubsuit)$.\\
  From Theorem Substitution with ($\spadesuit\spadesuit$) and ($\clubsuit$), we get: $\cdot \jtype{\max(n_1,n_2)}{}{t'[F/x]}{\tau_2}~(\heartsuit\heartsuit) $.\\
  By ih on ($\heartsuit$) and ($\heartsuit\heartsuit$), we conclude: $ \cdot \jtype{\max(n_1,n_2)}{}{F'}{\tau_2}$.\\
  It proves this case.\\
  
   \noindent \textbf{Case}
   \[
   \inferrule*[ right = let ]
   {\cdot \jtype{n_1}{}{t_1}{!_p \tau}~(\star) \\ \cdot, x: [\tau]_p \jtype{n_2}{}{t_2}{\tau'}~(\diamond) }
   { \cdot \jtype{\max(n_1,n_2)}{}{\letx{!x}{t_1}{t_2}}{\tau'}  }
   \]
   In this case, $t$ is $\letx{!x}{t_1}{t_2} $. From the evaluation rule, we know:\\
   \[
   \inferrule*[ right=E-let-bang]
  {   
    { \eval{ t_1  }{ !t_3  }{m_1} ~(\star\star)} 
    \\
    {\eval{t_3}{F'}{m_2}~(\diamond\diamond) }
    \\
    { \eval{t_2[F'/x] }{ F}{m_3 } }
  }
  { \eval{  \letx{!x}{t_1}{t_2}  }{ F  }{ m_1+m_2+m_3  } }  
   \]
    By ih on ($\star$) and ($\star\star$), we get: $ \cdot \jtype{n_1}{}{!t_3 }{!_p \tau}~(\spadesuit) $.\\
     By inversion on ($\spadesuit$), we get: $\cdot \jtype{n_1}{}{t_3}{\tau}(\spadesuit\spadesuit)$.\\
   By ih on ($\spadesuit\spadesuit $) and ($\diamond\diamond$), we get: $ \cdot \jtype{n_2}{}{F'}{\tau} ~(\clubsuit)$.\\

\noindent \textbf{Case}
\[
 \inferrule*[ right = case-const ]
   {\cdot \jtype{n_1}{}{t'}{b}~(\star) \\ \cdot \jtype{n_2}{}{t_i}{b}~(\diamond) }
   {\cdot \jtype{\max(n_1,n_2)}{}{\tcaseof{t'}\ \{c_i \Rightarrow t_i\}_{c_i \in b} } {b} }
\]
In this case, $t$ is $ \tcaseof{t'} \{c_i \Rightarrow t_i\}_{c_i \in b} $.\\
From the evaluation rule, we get:

\[
\inferrule*[ right=E-case]
  { 
    \inferrule*[]
    {}
    {\eval{  t'  }{ F' }{m } ~(\star\star) }
    \\
    \inferrule*[]
    {}
    { \eval{ t_i  }{ F_i   }{ m_i } ~(\diamond\diamond)  }
  }
  { \eval{ \tcaseof{t'}\ \{c_i \Rightarrow t_i\}_{c_i \in b}  }{ \tcaseof{F'}\ \{c_i \Rightarrow F_i\}_{c_i \in b}  }{  m + m_i } }
 \] 
  By ih on ($\star$) and ($\star\star$), we get: $ \cdot \jtype{n_1}{}{F' }{b}~(\spadesuit) $.\\
  By ih on ($\diamond$) and ($\diamond\diamond$), we get: $ \cdot \jtype{n_2}{}{F_i}{b} ~(\clubsuit)$.\\
 By the rule \rname{CASE-CONST}, we conclude :$ \cdot  \jtype{\max(n_1,n_2)}{}{ \tcaseof{F'}\ \{c_i \Rightarrow F_i\}_{c_i \in b} }{b}$.\\
 This case is proved.\\
 
 \noindent \textbf{Case}
 \[
    \inferrule*[ right = case-query]
   {\cdot \jtype{n_1}{}{t'}{b}~(\star) \\ \cdot \jtype{n_2}{}{t_i}{query}~(\diamond) }
   {\cdot \jtype{\max(n_1,n_2)}{}{\tcaseof{t'}\ \{c_i \Rightarrow t_i\}_{c_i \in b} } {query} }
 \]
 In this case, $t$ is $ \tcaseof{t'} \{c_i \Rightarrow t_i\}_{c_i \in b} $.\\
From the evaluation rule, we get:

\[
\inferrule*[ right=E-case]
  { 
    \inferrule*[]
    {}
    {\eval{  t'  }{ F' }{m } ~(\star\star) }
    \\
    \inferrule*[]
    {}
    { \eval{ t_i  }{ F_i   }{ m_i } ~(\diamond\diamond)  }
  }
  { \eval{ \tcaseof{t'}\ \{c_i \Rightarrow t_i\}_{c_i \in b}  }{ \tcaseof{F'}\ \{c_i \Rightarrow F_i\}_{c_i \in b}  }{  m + m_i } }
 \] 
  By ih on ($\star$) and ($\star\star$), we get: $ \cdot \jtype{n_1}{}{F' }{b}~(\spadesuit) $.\\
  By ih on ($\diamond$) and ($\diamond\diamond$), we get: $ \cdot \jtype{n_2}{}{F_i}{query} ~(\clubsuit)$.\\
 By the rule \rname{CASE-QUERY}, we conclude :$ \cdot  \jtype{\max(n_1,n_2)}{}{ \tcaseof{F'}\ \{c_i \Rightarrow F_i\}_{c_i \in b} }{query}$.\\
 This case is proved.\\
 
 \noindent \textbf{Case}
 \[
 \inferrule*[ right = let-p ]
   {\cdot \jtype{n_1}{}{t_1}{\tau_1 \times \tau_2 }~(\star) \\  x_1: \tau_1, x_2 : \tau_2 \jtype{n_2}{}{t_1'}{\tau'}~(\diamond) }
   { \cdot  \jtype{\max(n_1,n_2)}{}{ \letx{(x_1,x_2)}{t_1}{t_1'} }{\tau'}  }
 \]
 In this case, $t$ is $ \letx{(x_1,x_2)}{t_1}{t_1'} $.
 From the evalutation rule, we get:
 \[
 \inferrule*[ right=E-LET-P]
  {   
    { \eval{ t_1  }{ (F_1,F_2)  }{m_1}~(\star\star) } 
    \\
    { \eval{t_1'[F_1/x_1][F_2/x_2] }{ F}{m_3 }~(\heartsuit) }
  }
  { \eval{  \letx{(x_1,x_2)}{t_1}{t_1'}  }{ F  }{ m_1+m_2+m_3  } }
  \]
 By ih on ($\star$) and ($\star\star$), we know: $ \cdot \jtype{n_1}{}{(F_1,F_2)}{\tau_1 \times \tau_2}~(\spadesuit) $. \\
 By inversion on ($\spadesuit$), we get : $\cdot \jtype{n_1'}{}{F_1}{\tau_1} ~(\clubsuit)$ and $\cdot \jtype{n_2'}{}{F_2}{\tau_2} ~(\clubsuit\clubsuit)$ where $\max(n_1',n_2') = n_1$.\\
 By Theorem Substitution twice with ($\clubsuit$) and ($\clubsuit\clubsuit$) on ($\diamond$), we get $\cdot \jtype{\max(n_1',\max(n_2',n_2))}{}{t_1'[F_1/x_1][F_2/x_2]}{\tau'} ~(\heartsuit\heartsuit)$.\\
 It is proved by ih on $(\heartsuit\heartsuit)$ and ($\heartsuit$).\\
 
 \noindent \textbf{Case}
 \[
     \inferrule*[right = pair]
   {\cdot \jtype{n_1}{}{t_1}{\tau_1} \\ \cdot \jtype{n_2}{}{t_2}{\tau_2}}
   { \cdot  \jtype{\max(n_1,n_2)}{}{(t_1, t_2)}{\tau_1 \times \tau_2}  }\]
   In this case, $t$ is $(t_1,t_2)$. From the evaluation rule we get :
   
   \[  
   \inferrule*[ right=E-pair]
  {   
    { \eval{ t_1  }{ F_1  }{m_1} }
    \\
    { \eval{ t_2  }{ F_2  }{m_2} } 
  }
  { \eval{  (t_1,t_2)  }{ (F_1,F_2)  }{ m_1+m_2  } } 
  \]
  By ih, we know : $ \cdot \jtype{n_1}{}{F_1}{\tau_1}$ and $ \cdot \jtype{n_2}{}{F_2}{\tau_2}$. \\
  This case is proved by using the rule \rname{PAIR}.\\
   
 
\end{proof}



\clearpage
\begin{theorem} [Substitution]
	If $\Gamma \jtype{n_1}{}{t_1}{\tau_1}$  and   $\Delta, x: [\tau_1]_p \jtype{n_2}{}{t_2}{\tau_2}$ then $\max( \Gamma,\Delta) \jtype{\max{(p + n_1, n_2)}}{}{t_2[t_1/x]}{\tau_2} $
\end{theorem}
\begin{proof}
  The theorem is proved by induction on the typing derivation of the second premise  $\Gamma, x: [\tau_1]_p \jtype{n_2}{}{t_2}{\tau_2}$. \\
   
  \noindent \textbf{Case}
  \[
   \inferrule*[ right = var]
   {\empty}
   {\Gamma, x:[\tau_1]_p \jtype{0}{}{x}{\tau_1}  } 
  \]
  where $t_2 = x$ and $\tau_2 = \tau_1$.\\
  Assume we know: $\Gamma \jtype{n_1}{}{t_1}{\tau_1} \ \ (\star) $ \\
  TS: $\Gamma \jtype{\max(p+n_1,0 )}{}{t_2[t_1/x]}{\tau_1}$ \\
  STS: $\Gamma \jtype{\max(p+n_1,0)}{}{x[t_1/x]}{\tau_1} \Rightarrow \Gamma \jtype{\max(p+n_1,0)}{}{t_1}{\tau_1}$ \\
  It is proved by using Lemma ~\ref{lem:weaken} on $(\star)$ because $n_1 \leq \max(p+n_1, 0)$.\\
  
  \noindent \textbf{Case}
  \[
   \inferrule*[ right = var]
   {\empty}
   {\Gamma, y:[\tau_2]_{p'} , x:[\tau_1]_p \jtype{0}{}{y}{\tau_2}  }
  \]
 
  Assume we know: $\Gamma, y: [\tau_2]_{p'} \jtype{n_1}{}{t_1}{\tau_1} \ \ (\star) $ \\
  TS: $\Gamma, y: [\tau_2]_{p'} \jtype{\max(p+n_1,0 )}{}{y[t_1/x]}{\tau_2}$ \\
  STS: $\Gamma, y: [\tau_2]_{p'} \jtype{\max(p+n_1,0)}{}{y}{\tau_2} $ \\
  From \rname{VAR}, we get : $ \Gamma, y: [\tau_2]_{p'} \jtype{0}{}{y}{\tau_2} ~(\diamond) $.
  It is proved by using Lemma ~\ref{lem:weaken} on $(\diamond)$.\\
  
    \noindent \textbf{Case}
     \[
   \inferrule*[ right = MT ]
   {\Gamma, x: [\tau_1]_{p} \jtype{n}{}{t}{query} \ \ (\star\star)}
   {1 + \Gamma, x: [\tau_1]_{p+1} \jtype{n+1}{}{M(t)}{b}  }
  \]
  Assume we know: $1+\Gamma \jtype{n_1}{}{t_1}{\tau_1} \ \ (\star) $ \\
   We know that $\Gamma \jtype{n_1}{}{t_1}{\tau_1 } $ by Lemma ~\ref{lem:weaken2} on the assumption $(\star)$.\\
   We also know that $n+1=n_2$.\\
   TS:  $1+\Gamma \jtype{\max(p+n_1+1,n_2)}{}{ (M(t)) [t_1/x]}{b}$ \\
   By IH on $(\star\star)$, we know: $ \Gamma \jtype{\max(n_1+p, n) }{}{t[t_1/x] }{query} \ \ (\diamond) $\\
   By rule MT, we know that : 
     \[
   \inferrule*[ right = MT ]
   {\Gamma \jtype{\max(n_1+p, n) }{}{t[t_1/x] }{query} }
   {1 + \Gamma \jtype{  1+ \max(n_1+p, n) }{}{M(t[t_1/x])}{b}  }
  \]
  
   We obtain: $ 1 + \Gamma \jtype{  \max(n_1+p+1, n+1) }{}{M(t[t_1/x])}{b} \Rightarrow 
     1 + \Gamma \jtype{  \max(n_1+p+1, n_2) }{}{\big ( M(t) \big )[t_1/x]}{b} $\\
     This case is proved.\\
     
   \noindent \textbf{Case}
   
   \[
   \inferrule*[right = pr]
   {\Gamma, x:[\tau_1]_{p} \jtype{n_2}{}{t}{\tau} \ \ (\star\star)}
   {p' + \Gamma, x:[\tau_1]_{p+p'} \jtype{n_2}{}{!t}{!_{p'} \tau}  }
   \]
   We assume $p'+ \Gamma \jtype{n_1}{}{t_1}{\tau_1}$.
   by Lemma \ref{lem:weaken2}, we know $  \Gamma \jtype{n_1}{}{t_1}{\tau_1}$ $(\star)$.\\
   TS: $ p'+ \Gamma \jtype{\max(n_1+p+p', n_2)}{}{!t[t_1/x]}{!_{p'} \tau } $\\
   
   By IH on $(\star\star)$ along with $(\star)$, we know:
   $ \Gamma \jtype{\max(n_1+p,n_2)}{}{t[t_1/x]}{\tau} \ \ (\diamond)$\\
   By rule \rname{PR}, we know: 
     \[
   \inferrule*[right = pr]
   {\Gamma \jtype{\max(n_1+p,n_2)}{}{t[t_1/x]}{\tau} }
   {p' + \Gamma \jtype{ \max(n_1+p,n_2) }{}{!t[t_1/x]}{!_{p'} \tau} \ \ (\diamond \diamond)  }
   \]
   This case is proved by Lemma \ref{lem:weaken} on $(\diamond\diamond)$. \\
  
  \noindent \textbf{Case}
   \[
     \inferrule*[right = query]
   {\empty}
   {\Gamma, x: [\tau_1]_p \jtype{0}{}{q}{query}  }
   \]   
   TS: $\Gamma \jtype{\max(p+n_1, n_2)}{}{q[t_1/x]}{query} \Rightarrow \Gamma \jtype{\max(p+n_1, n_2)}{}{q }{query} $.\\
   Using rule \rname{QUERY}, we get $\Gamma \jtype{0}{}{q}{QUERY}$,
   This case is proved by Lemma \ref{lem:weaken} on it.\\
   
  \noindent \textbf{Case} 
   \[
    \inferrule*[ right = let ]
   {\Gamma_1, y : [\tau_1]_{p} \jtype{n}{}{t}{!_{p_1} \tau} \ \ (\diamond) \\ \Gamma_2, y : [\tau_1]_{p}, x: [\tau]_{p_1} \jtype{n'}{}{t'}{\tau'} \ \ (\diamond \diamond) }
   { \max(\Gamma_1,\Gamma_2 ), y : [\tau_1]_{p} \jtype{\max(n,n')}{}{\letx{!x}{t}{t'}}{\tau'}  }
   \]
   We assume $ \max(\Gamma_1,\Gamma_2) \jtype{n_1}{}{t_1}{\tau}~(\star)$.\\
   
   TS: $ \max(\Gamma_1,\Gamma_2) \jtype{ \max( n_1 + p, \max(n,n') ) }{}{ (\letx{!x}{t}{t'} ) [t_1/y] }{ \tau' } $\\
   By Lemma~\ref{lem:weaken2}, Lemma~\ref{lem:weaken3}, we can extend the context of $(\diamond)$, to $\max(\Gamma_1,\Gamma_2)$:\\
   $\max(\Gamma_1,\Gamma_2), y : [\tau_1]_{p} \jtype{n}{}{t}{!_{p_1} \tau}$$(\diamond')$.\\
   Similarly, from ($\diamond\diamond$)we get: $ \max(\Gamma_1,\Gamma_2), y : [\tau_1]_{p}, x: [\tau]_{p_1} \jtype{n'}{}{t'}{\tau'} ~(\diamond\diamond') $. \\
   By IH on $(\diamond')$, we get:
   $ \max(\Gamma_1,\Gamma_2) \jtype{\max(n_1+p, n)}{}{t[t_1/y]}{!_{p_1} \tau} \ \ (\spadesuit) $\\
    By IH on $(\diamond\diamond')$, we get:
   $ \max(\Gamma_1,\Gamma_2),x:[\tau]_{p_1} \jtype{\max(n_1+p, n')}{}{t'[t_1/y]}{ \tau'} \ \  (\clubsuit ) $\\
   By rule \rname{let}, we get:
    \[
    \inferrule*[ right = let ]
   {\max(\Gamma_1,\Gamma_2) \jtype{\max(n_1+p, n)}{}{t[t_1/y]}{!_{p_1} \tau} \ \ (\spadesuit) \\ \max(\Gamma_1,\Gamma_2),x:[\tau]_{p_1} \jtype{\max(n_1+p, n')}{}{t'[t_1/y]}{ \tau'} \ \  (\clubsuit ) }
   { \Gamma \jtype{\max( \max(n_1+p, n)  ,\max(n_1+p, n'))}{}{\letx{!x}{t[t_1/y]}{t'[t_1/y]}}{\tau'}  }
   \]
   Because $\max( \max(n_1+p, n)  ,\max(n_1+p, n')) = \max( n_1 + p, \max(n,n') ) $, this case is proved.\\
   
    \noindent \textbf{Case} 
   \[
      \inferrule*[right = abs]
   {\Gamma, y:[\tau]_p, x: [\tau_1]_{0} \jtype{n}{}{t}{\tau_2} (\diamond)}
   { \Gamma, y: [\tau]_p \jtype{n}{}{\abs{x}{t}}{\tau_1 \multimap \tau_2}  }
   \]
   We assume $ \Gamma \jtype{n_1}{}{t_1}{\tau}~(\star) $.\\
   By Lemma ~\ref{lem:weaken3}, we get: $ \Gamma, x:[\tau_1]_0  \jtype{n_1}{}{t_1}{\tau_1}$.\\
   TS: $ \Gamma \jtype{\max(n_1+p , n)}{}{\abs{x}{t[t_1/y]}}{\tau_1 \multimap \tau_2 } $.\\
   By IH on ($\diamond$), we get: $ \Gamma, x:[\tau_1]_0 \jtype{\max(n_1+p,n )}{}{t[t_1/y]}{\tau_2}~(\spadesuit) $.\\ 
   It is proved by the rule \rname{ABS} and $(\spadesuit)$.
   
   \noindent \textbf{Case} 
   \[
    \inferrule*[ right = app ]
   {\Gamma_1, x:[\tau]_p \jtype{n_1}{}{t_1}{\tau_1 \rightarrow \tau_2}~(\diamond) \\ \Gamma_2, x:[\tau]_p \jtype{n_2}{}{t_2}{\tau_1}~(\diamond\diamond) }
   { \max(\Gamma_1,\Gamma_2),x : [\tau]_p \jtype{\max(n_1,n_2)}{}{\app{t_1}{t_2}}{\tau_2}  }
   \]
   We assume $ \max(\Gamma_1,\Gamma_2) \jtype{n}{}{t}{\tau} $.\\
   By Lemma \ref{lem:weaken2}, Lemma \ref{lem:weaken3}, we extend the 
   
   
   \noindent \textbf{Case} 
   \[
      \inferrule*[ right = let-p ]
   {\Gamma_1 \jtype{n_1}{}{t}{\tau_1 \times \tau_2 } \\ \Gamma_2, x_1: \tau_1, x_2 : \tau_2 \jtype{n_2}{}{t'}{\tau'}}
   { \max(\Gamma_1,\Gamma_2)  \jtype{\max(n_1,n_2)}{}{ \letx{(x_1,x_2)}{t}{t'} }{\tau'}  }
   \]
   
   
\end{proof}


\clearpage

% \begin{theorem}[Progress] 
% 	If $ \jtype{n}{}{t}{\tau}$, then $\exists F.\eval{ t  }{ F}{}$.
% \end{theorem}
% We prove two theorems instead.
 
% \begin{theorem}
%  If $\Gamma \jtype{n}{}{t}{\tau}$ and $\gamma \vDash{\Gamma}$, then $\gamma(t) \in \llu{\tau}{\epsilon} $.
% \end{theorem} 

% \begin{theorem}
% If $\gamma(t) \in \llu{\tau}{\epsilon} $, then $\exists F.\eval{ \gamma(t)  }{ F}{}$.
% \end{theorem}


% \begin{proof}
% Proof by induction on $t$.\\
% \noindent \textbf{Case} 
%  $$
%  c \sep \fix{t} \sep !t \sep (F_1, F_2) \sep \Lambda. t \sep \abs{x}{t} \sep M(F) \sep x \sep q \sep \tcaseof{F}\ \{c_i \Rightarrow F_i\}_{c_i \in b_i} 
% $$
% these cases are proved directly by applying the Normal form definition.\\ 
% \noindent \textbf{Case} 
% $$
% \app{t_1}{t_2} 
% $$

% We know there is a typing derivation for $\Gamma \jtype{n}{}{t_1 \, t_2}{\tau}$.
% \[
%   \inferrule*[ right = app ]
%   {\Gamma_1 \jtype{n_1}{}{t_1}{\tau_1 \rightarrow \tau_2} ~(\star) \\ \Gamma_2 \jtype{n_2}{}{t_2}{\tau_1} ~(\diamond)}
%   { \max(\Gamma_1,\Gamma_2) \jtype{\max(n_1,n_2)}{}{\app{t_1}{t_2}}{\tau_2}  }
% \]

% By IH on ($\star$),($\diamond$), we get :
%   $\exists F_1. \eval{ t_1  }{ F_1}{} $ and $\exists F_2. \eval{ t_2  }{ F_2}{} $. 
  
%   We have the evaluation rule \rname{E-APP}
%   \[
%   \inferrule*[ right=E-app]
%   {   
%     { \eval{ t_1  }{ \abs{x}{t}  }{m_1} }
%     \\
%     { \eval{ t_2  }{ F  }{m_2} } 
%     \\
%     { \eval{t[F/x] }{ F'}{m_3 } }
%   }
%   { \eval{ \app{t_1}{t_2}  }{ F'  }{ m_1+m_2+m_3  } }
%   \]
 
%  Since $F_1$ is a value with an arrow type according to the preservation lemma, it must be an abstraction. Say $F_1$ = $\abs{x}{t}$.\\
% We know $ \app{ \abs{x}{t}}{F_2} \rightarrow^* t[F_2/x] $ by $\beta$ reduction, and $\eval{t[F/x]}{F'}{} $ by $\delta$ reduction.\\
% By applying the \rname{E-APP} rule, we get: $\eval{\app{t_1}{t_2}}{F'}{}$.\\ This case is proved.




% \noindent \textbf{Case} 
% $$
% (t_1,t_2) 
% $$
% By induction hypothesis, we have: $\eval{t_1}{F_1}{}$ and $\eval{t_2}{F_2}{}$.\\
% By applying the \rname{E-PAIR} rule, we get: $\eval{(t_1,t_2)}{(F_1, F_2)}{}$.\\
% This case is proved.

% \noindent \textbf{Case} 
% $$
% \letx{!x}{t_1}{t_2} 
% $$
% By induction hypothesis, we have: $\eval{t_1}{!t_3}{m_1}$ and $\eval{t_2[t3/x]}{F}{m_2}$.\\
% By applying the \rname{E-LET-BANG} rule, we get: $\eval{\letx{!x}{t_1}{t_2} 
% }{F}{m_1 + m_2}$.\\
% This case is proved.

% \noindent \textbf{Case} 
% $$
% \letx{(x_1,x_2)}{t}{t}
% $$
% \noindent \textbf{Case} 
% $$
% t[]
% $$
% By induction hypothesis, we have $\eval{t}{\Lambda.t'}{m}$.\\
% By applying the \rname{E-IAPP} rule, we get: $\eval{t[]}{t'}{m}$.\\
% This case is proved.


% \noindent \textbf{Case} 
% $$
% M(t)
% $$
% By induction hypothesis, we have $\eval{t}{F}{m}$.\\
% By applying the \rname{E-MECH} rule, we get: $\eval{M(t)}{M(F)}{m}$, this case is proved.

% \noindent \textbf{Case} 
% $$ 
% \tcaseof{t}\ \{c_i \Rightarrow t_i\}_{c_i \in b}  
% $$
% By induction hypothesis, we have $\eval{t}{F}{m}$ and $\eval{t_i}{F_i}{m_i}$.\\
% By applying the \rname{E-CASE} rule, we get:\\
% $\eval{\tcaseof{t}\ \{c_i \Rightarrow t_i\}_{c_i \in b}}{\tcaseof{F}\ \{c_i \Rightarrow F_i\}_{c_i \in b}}{m + m_i}$.\\
% This case is proved.

% \end{proof}
\newpage
\begin{theorem} [Substitution] $ $
\label{thm:sub}
\begin{enumerate}
    \item If $\Gamma \jtype{n_1}{}{t_1}{\tau_1}$  and   $\Delta, x: \tau_1 \jtype{n_2}{}{t_2}{\tau_2}$ then $max(\Gamma, \Delta) \jtype{\max{(n_1, n_2)}}{}{t_2[t_1/x]}{\tau_2} $
    \item If $\Gamma \jtype{n_1}{}{ t_1}{!_p \tau_1}$  and   $\Delta, x: [\tau_1]_p \jtype{n_2}{}{t_2}{\tau_2}$ then $max(\Gamma, \Delta) \jtype{\max{(n_1 + p, n_2)}}{}{t_2[t_1/x]}{\tau_2} $
\end{enumerate}
\end{theorem}
   \[
   \begin{array}{ccc}
     c [t_1/x]    & ::=& c  \\
     M(t) [t_1/x] &   & M(t[t_1/x]) \\
     x [t_1/x] && t_1 \\
     y [t_1/x] && y \\
     (\app{t}{t'}) [t_1/x] && \app{t[t_1/x]}{t'[t_1/x]}\\
     (\abs{y}{t}) [t_1/x] && \abs{y}{t[t_1/y]} \\
     (\letx{y}{t}{t'})[t_1/x] && \letx{y}{t[t_1/x]}{t'[t_1/x]}\\
   \end{array}
\]

\begin{proof} of \ref{thm:sub}.1 \\
  The theorem is proved by induction on the typing derivation of the second premise $\Delta, x: \tau_1 \jtype{n_2}{}{t_2}{\tau_2} $. Assume we know $\Gamma \jtype{n_1}{}{t_1}{\tau_1} ~ (\diamond)$.

\noindent \textbf{Case} 
$$
   \inferrule*[right = const]
   {\empty}
   {\Delta, x: \tau_1 \jtype{n}{}{c}{b} ~(\star) }
$$
where $t_2 = c$, $\tau_2 = b$ and $n_2 = n$.\\
To show: $\max(\Gamma, \Delta) \jtype{\max{(n_1, n)}}{}{c[t_1/x]}{b} $.\\
Because $c[t_1/x] = c$, it suffices to show $\max(\Gamma, \Delta) \jtype{\max{(n_1, n)}}{}{c}{b} $.\\
This case is proved by applying the $\rname{CONST}$ rule. 

\noindent \textbf{Case} 
$$
    \inferrule*[right = query]
   {\empty}
   {\Delta, x: \tau_1 \jtype{n}{}{q}{query} ~ (\star) }
$$
where $t_2 = q$, $\tau_2 = query$ and $n_2 = n$.\\
To show: $\max{\Gamma, \Delta} \jtype{\max{(n_1, n)}}{}{q[t_1/x]}{query} $.\\
Because $q[t_1/x] = q$, it suffices to show: $max(\Gamma, \Delta) \jtype{\max{(n_1, n)}}{}{q}{query} $.\\
This case is proved by applying the $\rname{QUERY}$ rule. 


\noindent \textbf{Case} 

\noindent \textbf{SubCase 1} 
$$
    \inferrule*[ right = var]
   {\empty}
   {\Delta, x:\tau_1 \jtype{n}{}{x}{\tau_1}  } 
$$
where $t_2 = x$, $\tau_2 = \tau_1$ and $n_2 = n$.\\
To show $\max{(\Gamma, \Delta)} \jtype{\max{(n_1, n)}}{}{x[t_1/x]}{\tau_1} $.\\
it suffices to show $\max{(\Gamma, \Delta)} \jtype{\max{(n_1, n_2)}}{}{t_1}{\tau_1}$.\\
This case is proved by applying Lemma \ref{lem:coweaken1} and \ref{lem:coweaken2} on $(\diamond)$.

\noindent \textbf{SubCase 2} 
$$
    \inferrule*[ right = var]
   {\empty}
   {\Delta, x:\tau_1, y : \tau_2 \jtype{n}{}{y}{\tau_2}} 
$$
where $t_2 = y$ and $n_2 = n$.\\ 
To show $\max{(\Gamma, \Delta, y : \tau_2)} \jtype{\max{(n_1, n)}}{}{y[t_1/x]}{\tau_2} $.\\
it suffices to show $\max{(\Gamma, \Delta, y : \tau_2)} \jtype{n}{}{y}{\tau_2}$.\\
This case is proved by applying the rule \rname{VAR}.

\noindent \textbf{Case} 
$$
    \inferrule*[right = abs]
   {\Delta, x: \tau_1, y: \tau_1' \jtype{n}{}{t}{\tau_2'}~(\star)}
   { \Delta, x: \tau_1 \jtype{n}{}{\abs{y}{t}}{\tau_1' \multimap \tau_2'}  }
$$
where $t_2 = \abs{y}{t}$, $\tau_2 = \tau_1' \multimap \tau_2'$ and $n_2 = n$.\\
To show $\max{(\Gamma, \Delta)} \jtype{\max{(n_1, n)}}{}{\abs{y}{t}[t_1/x]}{\tau_1' \multimap \tau_2'}$.\\
By applying Lemma \ref{lem:coex}  and induction hypothesis on $(\star)$, we get:\\ $\max{(\Gamma, \Delta,  y: \tau_1')} \jtype{\max{(n_1, n_2)}}{}{{t}[t_1/x]}{ \tau_2'}$ $(\star \star)$.\\
By applying the $\rname{ABS}$ rule on $(\star \star)$, we get: $\max{(\Gamma, \Delta)} \jtype{\max{(n_1, n_2)}}{}{\abs{y}{({t}[t_1/x])}}{\tau_1' \multimap \tau_2'}$, this case is proved.\\

\noindent \textbf{Case} 

$$
   \inferrule*[ right = MT ]
   {[\Delta_2] \jtype{n}{}{t}{query} ~ (\star)}
   {\Delta_1, 1 + [\Delta_2], x: \tau_1 \jtype{n+1}{}{M(t)}{b}  }
$$
Where $\Delta = \Delta_1, 1 + [\Delta_2] $, $t_2 = M(t)$, $\tau_2 = b$  and $n_2 = n + 1$.\\
To show $\max(\Gamma, \Delta_1, 1 + [\Delta_2]) \jtype{\max{(n_1, n + 1)}}{}{M(t)[t_1/x]}{b} $.\\
We know $x \not \in \fcv{[\Delta_2]} \land [\Delta_2] \jtype{n}{}{t}{query}  \implies x \not \in \fv{t} $. \\
So we know $t = t[t_1/x] \implies [\Delta_2] \jtype{n}{}{t[t_1/x]}{query}(\star\star)$ \\
By applying the $\rname{MT}$ rule on $(\star \star)$ we get:\\
$\max(\Delta_1, 1 + [\Delta_2]) \jtype{1 + n}{}{M(t)[t_1/x]}{b}~(\clubsuit)$\\
This case is proved using Lemma~\ref{lem:coex}, Lemma~\ref{lem:deweaken1} on ($\clubsuit$).\\


\noindent \textbf{Case} 
$$
  \inferrule*[ right = app ]
   {\Gamma_1 \jtype{n_1}{}{t_1}{\tau_1 \rightarrow \tau_2} \\ \Gamma_2 \jtype{n_2}{}{t_2}{\tau_1}}
   { \max(\Gamma_1, \Gamma_2) \jtype{\max(n_1',n_2')}{}{\app{t_1'}{t_2'}}{\tau_2}  }
$$
There are three sub cases depending on whether $x \in \fcv{\Gamma_1}$ and $\fcv{\Gamma_2}$.\\

 \textbf{SubCase 1} 
$$
  \inferrule*[ right = app ]
   {\Delta_1, x: \tau_1 \jtype{n_1'}{}{t_1'}{\tau_1 \rightarrow \tau_2} ~(\star) \\ \Delta_2 \jtype{n_2'}{}{t_2'}{\tau_1} ~ (\clubsuit)}
   { \max(\Delta_1, x: \tau_1, \Delta_2) \jtype{\max(n_1',n_2')}{}{\app{t_1'}{t_2'}}{\tau_2} }
$$
where $x \not \in \fcv{\Delta_2}$. \\
$\Delta = \max(\Delta_1, \Delta_2)$, $t_2 = \app{t_1'}{t_2'}$ and $n_2 = \max(n_1',n_2')$.\\
To show $\max(\Gamma,\max(\Delta_1, \Delta_2))\jtype{\max(n_1, \max(n_1',n_2'))}{}{\app{t_1'}{t_2'}[t_1/x]}{\tau_2} $.\\
it suffices to show $\max(\Gamma,\Delta_1, \Delta_2)\jtype{\max(n_1, n_1',n_2')}{}{\app{t_1'}{t_2'}[t_1/x]}{\tau_2} $.\\
By induction hypothesis on $(\star)$, we get: 
$\max(\Gamma,\Delta_1)\jtype{\max(n_1, n_1')}{}{t_1'[t_1/x]}{\tau_1 \rightarrow \tau_2}~(\star \star) $.\\
Because $x \not \in \fcv{\Delta_2} \implies t_2'[t_1/x] = t_2'$.\\
By applying $\rname{APP}$ on $(\star \star)$ and $(\clubsuit)$, we get:
$\max(\max(\Gamma,\Delta_1), \Delta_2)\jtype{\max(\max(n_1, n_1'),n_2')}{}{(\app{t_1'}{t_2'})[t_1/x]}{\tau_2} $.\\
This case is proved.\\

 \textbf{SubCase 2} 
$$
  \inferrule*[ right = app ]
   {\Delta_1 \jtype{n_1'}{}{t_1'}{\tau_1 \rightarrow \tau_2} ~(\star) \\ \Delta_2, x: \tau_1 \jtype{n_2'}{}{t_2'}{\tau_1} ~ (\clubsuit)}
   { \max(\Delta_1 \Delta_2, x: \tau_1) \jtype{\max(n_1',n_2')}{}{\app{t_1'}{t_2'}}{\tau_2} }
$$
where $x \not \in \fcv{\Delta_1}$ .\\
This case is proved by induction hypothesis on $(\clubsuit)$ and applying $\rname{APP}$ rule.\\

 \textbf{SubCase 3} 
$$
  \inferrule*[ right = app ]
   {\Delta_1, x: \tau_1 \jtype{n_1'}{}{t_1'}{\tau_1 \rightarrow \tau_2}~(\star) \\ \Delta_2, x: \tau_1 \jtype{n_2'}{}{t_2'}{\tau_1}~(\clubsuit)}
   { \max(\Delta_1, \Delta_2, x: \tau_1) \jtype{\max(n_1',n_2')}{}{\app{t_1'}{t_2'}}{\tau_2}  }
$$
TS: $\max(\Gamma, \max( \Delta_1, \Delta_2) ) \jtype{\max(n_1, \max(n_1',n_2'))}{}{\app{t_1'}{t_2'}}{\tau_2} $. \\
This case is proved by induction hypothesis on both $(\clubsuit)$ and $(\star)$ and then applying $\rname{APP}$ rule.\\

\noindent \textbf{Case} 
$$
   \inferrule*[ right = der ]
   {\Delta', x:\tau_1, y: \tau \jtype{n}{}{t_2}{ \tau_2 }~ (\star)}
   { \Delta' ,  x:\tau_1, y: [\tau]_p \jtype{n}{}{t_2}{\tau_2}  }
$$
where $\Delta = \Delta', y: [\tau]_p$, $n_2 = n$.\\
To show $\max(\Gamma, \Delta', y: [\tau]_p) \jtype{\max(n_1, n)}{}{t_2[t_1/x]}{\tau_2}$.\\
By induction hypothesis on $(\star)$, we get:
$\max(\Gamma, \Delta', y: \tau) \jtype{\max(n_1, n)}{}{t_2[t_1/x]}{\tau_2} (\star \star)$.\\
By applying the $\rname{DER}$ rule on $(\star \star)$, we get:
$\max(\Gamma, \Delta', y: [\tau]_p) \jtype{\max(n_1, n)}{}{t_2[t_1/x]}{\tau_2}$.\\
This case is proved.\\


\noindent \textbf{Case} 
\[
   \inferrule*[ right = let ]
   {\Gamma_1 \jtype{n_1}{}{t}{!_p \tau} \\ \Gamma_2, y: [\tau]_p \jtype{n_2}{}{t'}{\tau'}}
   { \max(\Gamma_1, \Gamma_2) \jtype{\max(n_1,n_2)}{}{\letx{!y}{t}{t'}}{\tau'}  }
\]
There are three sub cases.

\textbf{Subcase 1: x $\not \in \fcv{\Delta_2}$ }
\[
   \inferrule*[ right = let ]
   {\Delta_1, x: \tau_1 \jtype{n_1'}{}{t}{!_p \tau}~(\star) \\ \Delta_2, y: [\tau]_p \jtype{n_2'}{}{t'}{\tau'}~(\clubsuit)}
   {  \max(\Delta_1, x: \tau_1, \Delta_2) \jtype{\max(n_1',n_2')}{}{\letx{!y}{t}{t'}}{\tau'}  }
\]
 TS: $ \max(\Gamma, \max(\Delta_1,\Delta_2) ) \jtype{\max(n_1,\max(n_1',n_2') )}{}{  (\letx{!y}{t}{t'})[t_1/x] }{\tau'}  $. \\
 By IH on ($\star$), we get $\max(\Gamma, \Delta_1) \jtype{\max(n_1,n_1') }{}{t[t_1/x]}{!_p \tau}~(\star\star) $. \\
 $x \not \in \fcv{\Delta_2} \implies t'[t_1/x] =  t'$.\\
 It is proved by the rule \rname{LET} using ($\star\star$) and ($\clubsuit$). \\

\textbf{Subcase 2: x $\not \in \fcv{\Delta_1}$ }
\[
   \inferrule*[ right = let ]
   {\Delta_1 \jtype{n_1'}{}{t}{!_p \tau}~(\star) \\ \Delta_2, x: \tau_1, y: [\tau]_p \jtype{n_2'}{}{t'}{\tau'}~(\clubsuit)}
   {  \max(\Delta_1, x: \tau_1, \Delta_2) \jtype{\max(n_1',n_2')}{}{\letx{!y}{t}{t'}}{\tau'}  }
\]
 TS: $ \max(\Gamma, \max(\Delta_1,\Delta_2) ) \jtype{\max(n_1,\max(n_1',n_2') )}{}{  (\letx{!y}{t}{t'})[t_1/x] }{\tau'}  $. \\
By IH on ($\clubsuit$), we get $\max(\Gamma, \Delta_2),y:[\tau]_p \jtype{\max(n_1,n_2') }{}{t'[t_1/x]}{\tau'}~(\clubsuit\clubsuit) $. \\
 $x \not \in \fcv{\Delta_1} \implies t[t_1/x] = t $.\\
It is proved by the rule \rname{LET} using ($\star$) and ($\clubsuit\clubsuit$). \\

\textbf{Subcase 3 }
\[
   \inferrule*[ right = let ]
   {\Delta_1, x:\tau_1 \jtype{n_1'}{}{t}{!_p \tau}~(\star) \\ \Delta_2, x: \tau_1, y: [\tau]_p \jtype{n_2'}{}{t'}{\tau'}~(\clubsuit)}
   {  \max(\Delta_1, x: \tau_1, \Delta_2) \jtype{\max(n_1',n_2')}{}{\letx{!y}{t}{t'}}{\tau'}  }
\]
TS: $ \max(\Gamma, \max(\Delta_1,\Delta_2) ) \jtype{\max(n_1,\max(n_1',n_2') )}{}{  (\letx{!y}{t}{t'})[t_1/x] }{\tau'}  $. \\
By IH on ($\star$), we get $\max(\Gamma, \Delta_1) \jtype{\max(n_1,n_1') }{}{t[t_1/x]}{!_p \tau}~(\star\star) $. \\
By IH on ($\clubsuit$), we get $\max(\Gamma, \Delta_2),y:[\tau]_p \jtype{\max(n_1,n_2') }{}{t'[t_1/x]}{\tau'}~(\clubsuit\clubsuit) $. \\
It is proved by the rule \rname{LET} using ($\star\star$) and ($\clubsuit\clubsuit$).\\

\noindent \textbf{Case} 
\[
   \inferrule*[ right = let-p ]
   {\Gamma_1 \jtype{n_1'}{}{t}{\tau_1' \times \tau_2' } \\ \Gamma_2, x_1: \tau_1', x_2 : \tau_2' \jtype{n_2'}{}{t'}{\tau'}}
   { \max(\Gamma_1, \Gamma_2)  \jtype{\max(n_1',n_2')}{}{ \letx{(x_1,x_2)}{t}{t'} }{\tau'}  }
\]
There are three sub cases.\\
\textbf{Subcase 1: x $\not \in \fcv{\Delta_2}$ }
\[
   \inferrule*[ right = let-p ]
   {\Delta_1, x: \tau_1 \jtype{n_1'}{}{t}{\tau_1' \times \tau_2'}~(\star) \\ \Delta_2, x_1:\tau_1', x_2: \tau_2' \jtype{n_2'}{}{t'}{\tau'}~(\clubsuit)}
   {  \max(\Delta_1, x: \tau_1, \Delta_2) \jtype{\max(n_1',n_2')}{}{\letx{(x_1,x_2)}{t}{t'}}{\tau'}  }
\]
 TS: $ \max(\Gamma, \max(\Delta_1,\Delta_2) ) \jtype{\max(n_1,\max(n_1',n_2') )}{}{  (\letx{(x_1,x_2)}{t}{t'})[t_1/x] }{\tau'}  $. \\
 By IH on ($\star$), we get $\max(\Gamma, \Delta_1) \jtype{\max(n_1,n_1') }{}{t[t_1/x]}{\tau_1' \times \tau_2'}~(\star\star) $. \\
 $x \not \in \fcv{\Delta_2} \implies t'[t_1/x] =  t'$.\\
 It is proved by the rule \rname{LET-P} using ($\star\star$) and ($\clubsuit$). \\

\textbf{Subcase 2: x $\not \in \fcv{\Delta_1}$ }
\[
   \inferrule*[ right = let-p ]
   {\Delta_1 \jtype{n_1'}{}{t}{\tau_1' \times \tau_2'}~(\star) \\ \Delta_2, x_1:\tau_1', x_2: \tau_2' , x: \tau_1 \jtype{n_2'}{}{t'}{\tau'}~(\clubsuit)}
   {  \max(\Delta_1, x: \tau_1, \Delta_2) \jtype{\max(n_1',n_2')}{}{\letx{(x_1,x_2)}{t}{t'}}{\tau'}  }
\]
 TS: $ \max(\Gamma, \max(\Delta_1,\Delta_2) ) \jtype{\max(n_1,\max(n_1',n_2') )}{}{  (\letx{(x_1,x_2)}{t}{t'})[t_1/x] }{\tau'}  $. \\
By IH on ($\clubsuit$), we get $\max(\Gamma, \Delta_2), x_1:\tau_1', x_2: \tau_2'  \jtype{\max(n_1,n_2') }{}{t'[t_1/x]}{\tau'}~(\clubsuit\clubsuit) $. \\
 $x \not \in \fcv{\Delta_1} \implies t[t_1/x] = t $.\\
It is proved by the rule \rname{LET-P} using ($\star$) and ($\clubsuit\clubsuit$). \\

\textbf{Subcase 3 }
\[
    \inferrule*[ right = let-p ]
   {\Delta_1 , x: \tau_1 \jtype{n_1'}{}{t}{\tau_1' \times \tau_2'}~(\star) \\ \Delta_2, x_1:\tau_1', x_2: \tau_2' , x: \tau_1 \jtype{n_2'}{}{t'}{\tau'}~(\clubsuit)}
   {  \max(\Delta_1, x: \tau_1, \Delta_2) \jtype{\max(n_1',n_2')}{}{\letx{(x_1,x_2)}{t}{t'}}{\tau'}  }
\]
TS: $ \max(\Gamma, \max(\Delta_1,\Delta_2) ) \jtype{\max(n_1,\max(n_1',n_2') )}{}{  (\letx{(x_1,x_2)}{t}{t'} )[t_1/x] }{\tau'}  $. \\
By IH on ($\star$), we get $\max(\Gamma, \Delta_1) \jtype{\max(n_1,n_1') }{}{t[t_1/x]}{\tau_1' \times \tau_2'}~(\star\star) $. \\
By IH on ($\clubsuit$), we get $\max(\Gamma, \Delta_2),x_1: \tau_1', x_2:\tau_2' \jtype{\max(n_1,n_2') }{}{t'[t_1/x]}{\tau'}~(\clubsuit\clubsuit) $. \\
It is proved by the rule \rname{LET-P} using ($\star\star$) and ($\clubsuit\clubsuit$).\\


\noindent \textbf{Case} 
\[
     \inferrule*[right = pair]
   {\Gamma_1 \jtype{n_1'}{}{t_1}{\tau_1} \\ \Gamma_2 \jtype{n_2'}{}{t_2}{\tau_2}}
   { \max(\Gamma_1, \Gamma_2)  \jtype{\max(n_1',n_2')}{}{(t_1, t_2)}{\tau_1 \times \tau_2}  }
\]
There are three sub cases,$x$ only in $\Gamma_1$,  $x$ only in $\Gamma_2$ and $x$  appears in both $\Gamma_1$ and $\Gamma_2$. When $x$ only in $\Gamma_1$, it is proved by ih on the first premise and then using the rule \rname{PAIR}. When $x$ only in $\Gamma_2$, it is proved by ih on the second premise and then using the rule \rname{PAIR}. When $x$  appears in both $\Gamma_1$ and $\Gamma_2$, it is proved by ih on both premises and then using rule \rname{PAIR}.\\

\noindent \textbf{Case} 
\[
    \inferrule*[ right = case-const ]
   {\Gamma_1 \jtype{n_1'}{}{t}{b} \\ \Gamma_2 \jtype{n_2'}{}{t_i}{b} }
   {\max(\Gamma_1, \Gamma_2) \jtype{\max(n_1',n_2')}{}{\tcaseof{t}\ \{c_i \Rightarrow t_i\}_{c_i \in b} } {b} }
\]
There are three sub cases.\\

\textbf{Subcase 1: x $\not \in \fcv{\Delta_2}$ }
\[
   \inferrule*[ right = case-const ]
   {\Delta_1, x: \tau_1 \jtype{n_1'}{}{t}{b}~(\star) \\ \Delta_2 \jtype{n_2'}{}{t_i}{b}~(\clubsuit) }
   {\max(\Delta_1, \Delta_2) \jtype{\max(n_1',n_2')}{}{\tcaseof{t}\ \{c_i \Rightarrow t_i\}_{c_i \in b} } {b} }
\]
 TS: $ \max(\Gamma, \max(\Delta_1,\Delta_2) ) \jtype{\max(n_1,\max(n_1',n_2') )}{}{  (\tcaseof{t}\ \{c_i \Rightarrow t_i\}_{c_i \in b})[t_1/x] }{b}  $. \\
 By IH on ($\star$), we get $\max(\Gamma, \Delta_1) \jtype{\max(n_1,n_1') }{}{t[t_1/x]}{b}~(\star\star) $. \\
 $x \not \in \fcv{\Delta_2} \implies t_i[t_1/x] =  t_i$.\\
 It is proved by the rule \rname{CASE-CONST} using ($\star\star$) and ($\clubsuit$). \\

\textbf{Subcase 2: x $\not \in \fcv{\Delta_1}$ }
\[
     \inferrule*[ right = case-const ]
   {\Delta_1 \jtype{n_1'}{}{t}{b}~(\star) \\ \Delta_2, x: \tau_1 \jtype{n_2'}{}{t_i}{b}~(\clubsuit) }
   {\max(\Delta_1, \Delta_2) \jtype{\max(n_1',n_2')}{}{\tcaseof{t}\ \{c_i \Rightarrow t_i\}_{c_i \in b} } {b} }
\]
 TS: $ \max(\Gamma, \max(\Delta_1,\Delta_2) ) \jtype{\max(n_1,\max(n_1',n_2') )}{}{  (\tcaseof{t}\ \{c_i \Rightarrow t_i\}_{c_i \in b})[t_1/x] }{b}  $. \\
By IH on ($\clubsuit$), we get $\max(\Gamma, \Delta_2) \jtype{\max(n_1,n_2') }{}{t_i[t_1/x]}{b}~(\clubsuit\clubsuit) $. \\
 $x \not \in \fcv{\Delta_1} \implies t[t_1/x] = t $.\\
It is proved by the rule \rname{CASE-CONST} using ($\star$) and ($\clubsuit\clubsuit$). \\

\textbf{Subcase 3 }
\[
    \inferrule*[ right = case-const ]
   {\Delta_1 , x: \tau_1 \jtype{n_1'}{}{t}{b}~(\star) \\ \Delta_2, x: \tau_1 \jtype{n_2'}{}{t_i}{b}~(\clubsuit) }
   {\max(\Delta_1, \Delta_2) \jtype{\max(n_1',n_2')}{}{\tcaseof{t}\ \{c_i \Rightarrow t_i\}_{c_i \in b} } {b} }
\]
 TS: $ \max(\Gamma, \max(\Delta_1,\Delta_2) ) \jtype{\max(n_1,\max(n_1',n_2') )}{}{  (\tcaseof{t}\ \{c_i \Rightarrow t_i\}_{c_i \in b})[t_1/x] }{b}  $. \\
By IH on ($\star$), we get $\max(\Gamma, \Delta_1) \jtype{\max(n_1,n_1') }{}{t[t_1/x]}{b}~(\star\star) $. \\
By IH on ($\clubsuit$), we get $\max(\Gamma, \Delta_2) \jtype{\max(n_1,n_2') }{}{t_i[t_1/x]}{b}~(\clubsuit\clubsuit) $.. \\
It is proved by the rule \rname{CASE-CONST} using ($\star\star$) and ($\clubsuit\clubsuit$).\\



\noindent \textbf{Case} 
\[
    \inferrule*[ right = case-query]
   {\Gamma_1 \jtype{n_1'}{}{t}{b} \\ \Gamma_2 \jtype{n_2'}{}{t_i}{query} }
   {\max(\Gamma_1, \Gamma_2) \jtype{\max(n_1',n_2')}{}{\tcaseof{t}\ \{c_i \Rightarrow t_i\}_{c_i \in b} } {query} }
\]
There are three sub cases. Thre proof are quite similar as the one of \emph{CASE-CONST}.\\

\noindent \textbf{Case} 
\[
   \inferrule*[right = iabs]
  { 
    \inferrule*[]
    {}
    {i::\mathbb{N};\Delta,x:\tau_1 \jtype{n}{}{t}{ \tau } ~(\star)}
    \and
    \inferrule*[]
    {}
    { i \notin \fiv{\Delta,x:\tau_1}  } 
  }
  { \Delta, x:\tau_1 \jtype{n}{ }{  \Lambda.t  }{ \tforallN{i}{\tau}  } }
 \]
 TS: $\max(\Gamma, \Delta) \jtype{\max(n_1,n)}{}{\Lambda.t[t_1/x] }{\tforallN{i}{\tau}} $.\\
 By ih on ($\star$), we get : $ i :: \mathbb{N};\max(\Gamma, \Delta) \jtype{\max(n_1,n)}{}{t[t_1/x]}{\tau}(\star\star)$.\\
 There are two cases.\\
 \textbf{Sub case 1: $i \not \in \fiv{\Gamma}$}
  It is proved by using rule \rname{IABS} with $(\star\star)$.\\
  \textbf{Sub case 2: $i \in \fiv{\Gamma}$}
  We choose $j \not \in \fiv{\max(\Gamma,\Delta)}$. We rename all the $i$ to $j$ in $(\star\star)$ and use rule \rname{IABS} to get:
  $ \max(\Gamma, \Delta) \jtype{\max(n_1,n)}{}{\Lambda.t[t_1/x] }{\tforallN{j}{\tau}}$. It is just a renaming version of the goal.
 

\noindent \textbf{Case} 
\[ 
   \inferrule*[ right =  iapp]
  { 
    \inferrule*[]
    {}
    { \Delta, x:\tau_1  \jtype{n}{}{t}{ \tforallN{i}{\tau}   } (\star) }
    \and
    \inferrule*[]
    {}
    { \jiterm{I}{ \mathbb{N} } } 
  }
  {\Delta, x:\tau_1 \jtype{n }{ }{t\, [] }{ \tau \{ I/i \}  } }
\]
TS: $ \max(\Gamma, \Delta) \jtype{\max(n_1,n)}{}{t[t_1/x] \, []}{ \tau \{ I/i \} } $.\\
By ih on $(\star)$, we get: $ \max(\Gamma,\Delta)  \jtype{\max(n_1,n)}{}{t}{ \tforallN{i}{\tau} } ~ (\star\star) $.\\
It is proved by the rule \rname{IAPP} with $(\star\star)$.\\

\noindent \textbf{Case} 
$$
  \inferrule*[right = sub]
  { 
   { \Delta, x:\tau_1 \jtype{n}{}{t}{\tau} ~(\star) } \\
   { \Delta, x:\tau_1 \subseteq \Delta',x:\tau_1 ~ (\clubsuit) }  \\
   { \vDash n \leq n' } ~(\spadesuit) \\
   { \tau \subseteq \tau' }
  }
  { \Delta',x:\tau_1 \jtype{n'}{}{t}{\tau'} }
$$
TS: $ \max(\Gamma, \Delta') \jtype{\max(n',n_1)}{}{t}{\tau'}$.\\
By ih on ($\star$), we get : $\max(\Gamma, \Delta) \jtype{\max(n,n_1)}{}{t}{\tau} $.\\
$(\clubsuit) \implies \max(\Gamma,\Delta) \subseteq \max(\Gamma,\Delta')~(\clubsuit\clubsuit) $.\\
$ (\spadesuit) \implies  \vDash \max(n_1,n) \leq \max(n_1,n') $.\\
It is proved by the rule \rname{SUB}.\\


\noindent \textbf{Case}
\[
\inferrule*[right = pr]
   {[\Delta_2] \jtype{n}{}{t}{\tau}}
   {\Delta_1, p + [\Delta_2], x:\tau_1 \jtype{n}{}{!t}{!_p \tau}  }
\]
TS: $\max(\Gamma,(\Delta_1, p+[\Delta_2]) ) \jtype{\max(n_1, n)}{}{!t[t_1/x]}{!_p \tau}$.\\
$x \not \in [\Delta_2] \implies [\Delta_2] \jtype{n}{}{t[t_1/x]}{\tau}(\star)$.\\
By rule \rname{PR} with $(\star)$, we get: $\Delta_1, p+[\Delta_2] \jtype{n}{}{!t[t_1/x]}{!_p \tau} $.
It is proved by the Lemma \ref{lem:coex} and Lemma~\ref{lem:deweaken1} from the conclusion.

\end{proof}



\newpage
\begin{proof} of \ref{thm:sub}.2 \\
  The theorem is proved by induction on the typing derivation of the second premise $\Delta, x: [\tau_1]_p \jtype{n_2}{}{t_2}{\tau_2}$. Assume we know $\Gamma \jtype{n_1}{}{t_1}{!_p \tau_1}~(\diamond)$.


$$
   \inferrule*[right = const]
   {\empty}
   {\Delta, x: [\tau_1]_p \jtype{n}{}{c}{b} ~(\star) }
$$
where $t_2 = c$, $\tau_2 = b$ and $n_2 = n$.\\
To show: $\max(\Gamma, \Delta) \jtype{\max{(n_1 + p, n)}}{}{c[t_1/x]}{b} $.\\
Because $c[t_1/x] = c$, it suffices to show $\max(\Gamma, \Delta) \jtype{\max{(n_1 + p, n)}}{}{c}{b} $.\\
This case is proved by applying the $\rname{CONST}$ rule. 


\noindent \textbf{Case} 
$$
    \inferrule*[right = query]
   {\empty}
   {\Delta, x: [\tau_1]_p  \jtype{n}{}{q}{query} ~ (\star) }
$$
where $t_2 = q$, $\tau_2 = query$ and $n_2 = n$.\\
To show: $\max{\Gamma, \Delta} \jtype{\max{(n_1 + p, n)}}{}{q[t_1/x]}{query} $.\\
Because $q[t_1/x] = q$, it suffices to show: $max(\Gamma, \Delta) \jtype{\max{(n_1, n)}}{}{q}{query} $.\\
This case is proved by applying the $\rname{QUERY}$ rule. 


\noindent \textbf{Case} 

$$
    \inferrule*[ right = var]
   {\empty}
   {\Delta, x:[\tau_1]_p, y : \tau_2 \jtype{n}{}{y}{\tau_2}} 
$$
where $t_2 = y$ and $n_2 = n$.\\ 
To show $\max{(\Gamma, \Delta, y : \tau_2)} \jtype{\max{(n_1 + p, n)}}{}{y[t_1/x]}{\tau_2} $.\\
it suffices to show $\max{(\Gamma, \Delta, y : \tau_2)} \jtype{n}{}{y}{\tau_2}$.\\
This case is proved by applying the rule \rname{VAR}.

\noindent \textbf{Case} 
$$
    \inferrule*[right = abs]
   {\Delta, x: [\tau_1]_p, y: \tau_1' \jtype{n}{}{t}{\tau_2'}~(\star)}
   { \Delta, x: [\tau_1]_p \jtype{n}{}{\abs{y}{t}}{\tau_1' \multimap \tau_2'}  }
$$
where $t_2 = \abs{y}{t}$, $\tau_2 = \tau_1' \multimap \tau_2'$ and $n_2 = n$.\\
To show $\max{(\Gamma, \Delta)} \jtype{\max{(n_1 + p, n)}}{}{\abs{y}{t}[t_1/x]}{\tau_1' \multimap \tau_2'}$.\\
By applying Lemma \ref{lem:coex}  and induction hypothesis on $(\star)$, we get:\\ $\max{(\Gamma, \Delta,  y: \tau_1')} \jtype{\max{(n_1 + p, n_2)}}{}{{t}[t_1/x]}{ \tau_2'}$ $(\star \star)$.\\
By applying the $\rname{ABS}$ rule on $(\star \star)$, we get: $\max{(\Gamma, \Delta)} \jtype{\max{(n_1 + p, n_2)}}{}{\abs{y}{({t}[t_1/x])}}{\tau_1' \multimap \tau_2'}$, this case is proved.\\


\noindent \textbf{Case} 
$$
   \inferrule*[ right = MT ]
   {[\Delta_2], x: [\tau_1]_{p-1} \jtype{n}{}{t}{query} ~ (\star)}
   {\Delta_1, 1 + [\Delta_2], x: [\tau_1]_{p} \jtype{n+1}{}{M(t)}{b}  }
$$
Where $\Delta = \Delta_1, 1 + [\Delta_2] $, $t_2 = M(t)$, $\tau_2 = b$ and $n_2 = n + 1$.\\
To show $max(\Gamma, \Delta_1, 1 + [\Delta_2]) \jtype{\max{(n_1 + p, n + 1)}}{}{M(t)[t_1/x]}{b} $.\\
By induction hypothesis on $ (\star)$, we get:
$max(\Gamma, [\Delta_2]) \jtype{\max{(n_1 + p - 1, n)}}{}{t[t_1/x]}{query}~ (\star \star)$.\\
By applying the $\rname{MT}$ rule on $~ (\star \star)$, we get: 
$\Delta_1, max(\Gamma, 1 + [\Delta_2]) \jtype{1 + \max{(n_1 + p - 1, n)}}{}{M(t[t_1/x])}{b}$.\\
Because $\Delta_1$ and $\Delta_2$ are disjoint and $1 + \max{(n_1 + p - 1, n)} = \max{(n_1 + p, n + 1)}$, we get: \\
$\max(\Gamma,(\Delta_1, 1 + [\Delta_2])) \jtype{\max{(n_1 + p, n + 1)}}{}{M(t)[t_1/x]}{b}$.\\
This case is proved.


\noindent \textbf{Case} 
$$
  \inferrule*[ right = app ]
   {\Gamma_1 \jtype{n_1}{}{t_1}{\tau_1 \rightarrow \tau_2} \\ \Gamma_2 \jtype{n_2}{}{t_2}{\tau_1}}
   { \max(\Gamma_1, \Gamma_2) \jtype{\max(n_1,n_2)}{}{\app{t_1}{t_2}}{\tau_2}  }
$$
There are three sub cases depending on whether $x \in \fcv{\Gamma_1}$ and $\fcv{\Gamma_2}$.\\

 \textbf{Subcase 1 x $\not \in \fcv{\Delta_2}$ } 
$$
  \inferrule*[ right = app ]
   {\Delta_1, x: [\tau_1]_p \jtype{n_1'}{}{t_1'}{\tau_1 \rightarrow \tau_2} ~(\star) \\ \Delta_2 \jtype{n_2'}{}{t_2'}{\tau_1} ~ (\clubsuit)}
   { \max(\Delta_1, x: [\tau_1]_p, \Delta_2) \jtype{\max(n_1',n_2')}{}{\app{t_1'}{t_2'}}{\tau_2} }
$$
where $\Delta = \max(\Delta_1, \Delta_2)$, $t_2 = \app{t_1'}{t_2'}$ and $n_2 = \max(n_1',n_2')$.\\
To show $\max(\Gamma,\max(\Delta_1, \Delta_2))\jtype{\max(n_1 + p, \max(n_1',n_2'))}{}{\app{t_1'}{t_2'}[t_1/x]}{\tau_2} $.\\
it suffices to show $\max(\Gamma,\Delta_1, \Delta_2)\jtype{\max(n_1 + p, n_1',n_2')}{}{\app{t_1'}{t_2'}[t_1/x]}{\tau_2} $.\\
By induction hypothesis on $(\star)$, we get: 
$\max(\Gamma,\Delta_1)\jtype{\max(n_1 + p, n_1')}{}{t_1'[t_1/x]}{\tau_1 \rightarrow \tau_2} ~ (\star \star)$.\\
Because $x \not \in \fcv{\Delta_2} \implies t_2'[t_1/x] = t_2'$.\\
By applying $\rname{APP}$ on $(\star \star)$ and $(\clubsuit)$, we get:
$\max(\max(\Gamma,\Delta_1), \Delta_2)\jtype{\max(\max(n_1 + p, n_1'),n_2')}{}{(\app{t_1'}{t_2'})[t_1/x]}{\tau_2} $.\\
This case is proved.\\

\textbf{Subcase 2 x $\not \in \fcv{\Delta_1}$ } 
$$
  \inferrule*[ right = app ]
   {\Delta_1 \jtype{n_1'}{}{t_1'}{\tau_1 \rightarrow \tau_2} ~(\star) \\ \Delta_2, x: [\tau_1]_p \jtype{n_2'}{}{t_2'}{\tau_1} ~ (\clubsuit)}
   { \max(\Delta_1 \Delta_2, x: [\tau_1]_p) \jtype{\max(n_1',n_2')}{}{\app{t_1'}{t_2'}}{\tau_2} }
$$
This case is proved by induction hypothesis on $(\clubsuit)$ and applying $\rname{APP}$ rule.\\

 \textbf{Subcase 3} 
$$
  \inferrule*[ right = app ]
   {\Delta_1, x: [\tau_1]_p \jtype{n_1'}{}{t_1'}{\tau_1 \rightarrow \tau_2}~(\star) \\ \Delta_2, x: [\tau_1]_p \jtype{n_2'}{}{t_2'}{\tau_1}~(\clubsuit)}
   { \max(\Delta_1, \Delta_2, x: [\tau_1]_p) \jtype{\max(n_1',n_2')}{}{\app{t_1'}{t_2'}}{\tau_2}  }
$$
To show: $\max(\Gamma, \max( \Delta_1, \Delta_2) ) \jtype{\max(n_1 + p, \max(n_1',n_2'))}{}{\app{t_1'}{t_2'}}{\tau_2} $. \\
This case is proved by induction hypothesis on both $(\clubsuit)$ and $(\star)$ and then applying $\rname{APP}$ rule.\\


\noindent \textbf{Case} 
$$
   \inferrule*[ right = der ]
   {\Delta, x: \tau \jtype{n}{}{t_2}{ \tau_2 }  }
   { \Delta , x: [\tau]_p \jtype{n }{}{t_2 }{\tau_2 }  }
$$

\textbf{Subcase 1} 
$$
   \inferrule*[ right = der ]
   {\Delta, x: \tau \jtype{n}{}{t_2}{ \tau_2 }  }
   { \Delta , x: [\tau]_p \jtype{n}{}{t_2}{ \tau_2 }  }
$$
where  $n_2 = n$.\\
To show $\max(\Gamma, \Delta) \jtype{\max(n_1 + p, n)}{}{t_2[t_1/x]}{\tau_2}$\\
By induction hypothesis on $(\star)$, this case is proved.

\textbf{Subcase 2} 
$$
   \inferrule*[ right = der ]
   {\Delta', y: \tau, x : [\tau]_p \jtype{n}{}{t_2}{ \tau_2 } ~ (\star) }
   {\Delta' , y: [\tau]_{p'}, x : [\tau]_p \jtype{n }{}{t_2 }{\tau_2 } }
$$
where $\Delta =\Delta', y: [\tau]_{p'} $ and $n_2 = n$.\\
To show $\max(\Gamma, \Delta', y: [\tau]_{p'}) \jtype{\max(n_1 + p, n_2)}{}{t_2[t_1/x]}{\tau_2}$.\\
By induction hypothesis on $(\star)$, we get: 
$ \max(\Gamma, \Delta', y: \tau) \jtype{\max(n_1 + p, n)}{}{t_2[t_1/x]}{\tau_2} (\star \star)$.\\
By applying $\rname{DER}$ rule on $(\star \star)$, we get:
$ \max(\Gamma, \Delta', y: [\tau]_p') \jtype{\max(n_1 + p, n)}{}{t_2[t_1/x]}{\tau_2}$.\\
This case is proved.


\noindent \textbf{Case} 
$$
   \inferrule*[ right = let ]
   {\Gamma_1 \jtype{n_1}{}{t}{!_p \tau} \\ \Gamma_2, x: [\tau]_p \jtype{n_2}{}{t'}{\tau'}}
   { \max(\Gamma_1, \Gamma_2)  \jtype{\max(n_1,n_2)}{}{\letx{!x}{t}{t'}}{\tau'}  }
$$
There are three sub cases.

\textbf{Subcase 1: x $\not \in \fcv{\Delta_2}$ }
\[
   \inferrule*[ right = let ]
   {\Delta_1, x: [\tau_1]_p \jtype{n_1'}{}{t}{!_p \tau}~(\star) \\ \Delta_2, y: [\tau]_p \jtype{n_2'}{}{t'}{\tau'}~(\clubsuit)}
   {  \max(\Delta_1, x: [\tau_1]_p, \Delta_2) \jtype{\max(n_1',n_2')}{}{\letx{!y}{t}{t'}}{\tau'}  }
\]
where $\Delta = \max(\Delta_1, \Delta_2)$, $t_2 =\letx{!y}{t}{t'}$, $\tau_2 = \tau'$ and $n_2 = \max(n_1', n_2')$.\\
To show $ \max(\Gamma, \max(\Delta_1,\Delta_2) ) \jtype{\max(n_1 + p,\max(n_1',n_2') )}{}{(\letx{!y}{t}{t'})[t_1/x] }{\tau'}  $. \\
 By induction hypothesis on ($\star$), we get $\max(\Gamma, \Delta_1) \jtype{\max(n_1 + p,n_1') }{}{t[t_1/x]}{!_p \tau}~(\star\star) $. \\
 $x \not \in \fcv{\Delta_2} \implies t'[t_1/x] =  t'$.\\
 It is proved by the rule \rname{LET} using ($\star\star$) and ($\clubsuit$). \\

\textbf{Subcase 2: x $\not \in \fcv{\Delta_1}$ }
\[
   \inferrule*[ right = let ]
   {\Delta_1 \jtype{n_1'}{}{t}{!_p \tau}~(\star) \\ \Delta_2, x: [\tau_1]_p, y: [\tau]_p \jtype{n_2'}{}{t'}{\tau'}~(\clubsuit)}
   {  \max(\Delta_1, x: [\tau_1]_p, \Delta_2) \jtype{\max(n_1',n_2')}{}{\letx{!y}{t}{t'}}{\tau'}  }
\]
where $\Delta = \max(\Delta_1, \Delta_2)$, $t_2 =\letx{!y}{t}{t'}$, $\tau_2 = \tau'$ and $n_2 = \max(n_1', n_2')$.\\
To show $ \max(\Gamma, \max(\Delta_1,\Delta_2) ) \jtype{\max(n_1 + p,\max(n_1',n_2') )}{}{  (\letx{!y}{t}{t'})[t_1/x] }{\tau'}$. \\
By induction hypothesis on ($\clubsuit$), we get $\max(\Gamma, \Delta_2),y:[\tau]_p \jtype{\max(n_1 + p,n_2') }{}{t'[t_1/x]}{\tau'}~(\clubsuit\clubsuit) $. \\
 $x \not \in \fcv{\Delta_1} \implies t[t_1/x] = t $.\\
It is proved by the rule \rname{LET} using ($\star$) and ($\clubsuit\clubsuit$). \\

\textbf{Subcase 3 }
\[
   \inferrule*[ right = let ]
   {\Delta_1, x:[\tau_1]_p \jtype{n_1'}{}{t}{!_p \tau}~(\star) \\ \Delta_2, x: [\tau_1], y: [\tau]_p \jtype{n_2'}{}{t'}{\tau'}~(\clubsuit)}
   {  \max(\Delta_1, x: [\tau_1]_p, \Delta_2) \jtype{\max(n_1',n_2')}{}{\letx{!y}{t}{t'}}{\tau'}  }
\]
where $\Delta = \max(\Delta_1, \Delta_2)$, $t_2 =\letx{!y}{t}{t'}$, $\tau_2 = \tau'$ and $n_2 = \max(n_1', n_2')$.\\
To show $ \max(\Gamma, \max(\Delta_1,\Delta_2) ) \jtype{\max(n_1 + p,\max(n_1',n_2') )}{}{  (\letx{!y}{t}{t'})[t_1/x] }{\tau'}  $. \\
By induction hypothesis on ($\star$), we get $\max(\Gamma, \Delta_1) \jtype{\max(n_1 + p,n_1') }{}{t[t_1/x]}{!_p \tau}~(\star\star) $. \\
By induction hypothesis on ($\clubsuit$), we get $\max(\Gamma, \Delta_2),y:[\tau]_p \jtype{\max(n_1 + p,n_2') }{}{t'[t_1/x]}{\tau'}~(\clubsuit\clubsuit) $. \\
It is proved by the rule \rname{LET} using ($\star\star$) and ($\clubsuit\clubsuit$).\\

\noindent \textbf{Case} 
$$
   \inferrule*[ right = let-p ]
   {\Gamma_1 \jtype{n_1}{}{t}{\tau_1 \times \tau_2 } \\ \Gamma_2, x_1: \tau_1, x_2 : \tau_2 \jtype{n_2}{}{t'}{\tau'}}
   { \max(\Gamma_1, \Gamma_2)  \jtype{\max(n_1,n_2)}{}{ \letx{(x_1,x_2)}{t}{t'} }{\tau'}  }
$$

\textbf{Subcase 1: x $\not \in \fcv{\Delta_2}$ }
\[
   \inferrule*[ right = let-p ]
   {\Delta_1, x: [\tau_1]_p \jtype{n_1'}{}{t}{\tau_1' \times \tau_2'}~(\star) \\ \Delta_2, x_1:\tau_1', x_2: \tau_2' \jtype{n_2'}{}{t'}{\tau'}~(\clubsuit)}
   {  \max(\Delta_1, x: [\tau_1]_p, \Delta_2) \jtype{\max(n_1',n_2')}{}{\letx{(x_1,x_2)}{t}{t'}}{\tau'}  }
\]
where $\Delta = \max(\Delta_1,\Delta_2)$, $t_2 = \letx{(x_1,x_2)}{t}{t'}$, $\tau_2 = \tau'$ and $n_2 =\max(n_1',n_2')$. \\
To show $ \max(\Gamma, \max(\Delta_1,\Delta_2) ) \jtype{\max(n_1 + p,\max(n_1',n_2') )}{}{  (\letx{(x_1,x_2)}{t}{t'})[t_1/x] }{\tau'}  $. \\
By induction hypothesis on ($\star$), we get $\max(\Gamma, \Delta_1) \jtype{\max(n_1 + p, n_1') }{}{t[t_1/x]}{\tau_1' \times \tau_2'}~(\star\star) $. \\
 $x \not \in \fcv{\Delta_2} \implies t'[t_1/x] =  t'$.\\
 It is proved by the rule \rname{LET-P} using ($\star\star$) and ($\clubsuit$).\\

\textbf{Subcase 2: x $\not \in \fcv{\Delta_1}$ }
\[
   \inferrule*[ right = let-p ]
   {\Delta_1 \jtype{n_1'}{}{t}{\tau_1' \times \tau_2'}~(\star) \\ \Delta_2, x_1:\tau_1', x_2: \tau_2' , x: [\tau_1]_p \jtype{n_2'}{}{t'}{\tau'}~(\clubsuit)}
   {  \max(\Delta_1, x: [\tau_1]_p, \Delta_2) \jtype{\max(n_1',n_2')}{}{\letx{(x_1,x_2)}{t}{t'}}{\tau'}  }
\]
where $\Delta = \max(\Delta_1,\Delta_2)$, $t_2 = \letx{(x_1,x_2)}{t}{t'}$, $\tau_2 = \tau'$ and $n_2 =\max(n_1',n_2')$. \\
To show $ \max(\Gamma, \max(\Delta_1,\Delta_2) ) \jtype{\max(n_1 + p,\max(n_1',n_2') )}{}{  (\letx{(x_1,x_2)}{t}{t'})[t_1/x] }{\tau'}  $. \\
By induction hypothesis on ($\clubsuit$), we get $\max(\Gamma, \Delta_2), x_1:\tau_1', x_2: \tau_2'  \jtype{\max(n_1 + p,n_2') }{}{t'[t_1/x]}{\tau'}~(\clubsuit\clubsuit) $. \\
 $x \not \in \fcv{\Delta_1} \implies t[t_1/x] = t $.\\
It is proved by the rule \rname{LET-P} using ($\star$) and ($\clubsuit\clubsuit$). \\

\textbf{Subcase 3 }
\[
    \inferrule*[ right = let-p ]
   {\Delta_1 , x: [\tau_1]_p \jtype{n_1'}{}{t}{\tau_1' \times \tau_2'}~(\star) \\ \Delta_2, x_1:\tau_1', x_2: \tau_2' , x: [\tau_1]_p \jtype{n_2'}{}{t'}{\tau'}~(\clubsuit)}
   {  \max(\Delta_1, x: [\tau_1]_p, \Delta_2) \jtype{\max(n_1',n_2')}{}{\letx{(x_1,x_2)}{t}{t'}}{\tau'}  }
\]
where $\Delta = \max(\Delta_1,\Delta_2)$, $t_2 = \letx{(x_1,x_2)}{t}{t'}$, $\tau_2 = \tau'$ and $n_2 =\max(n_1',n_2')$. \\
To show $ \max(\Gamma, \max(\Delta_1,\Delta_2) ) \jtype{\max(n_1 + p,\max(n_1',n_2') )}{}{  (\letx{(x_1,x_2)}{t}{t'} )[t_1/x] }{\tau'}  $. \\
By induction hypothesis on ($\star$), we get $\max(\Gamma, \Delta_1) \jtype{\max(n_1 + p, n_1') }{}{t[t_1/x]}{\tau_1' \times \tau_2'}~(\star\star) $. \\
By induction hypothesis on ($\clubsuit$), we get $\max(\Gamma, \Delta_2),x_1: \tau_1', x_2:\tau_2' \jtype{\max(n_1 + p,n_2') }{}{t'[t_1/x]}{\tau'}~(\clubsuit\clubsuit) $. \\
It is proved by the rule \rname{LET-P} using ($\star\star$) and ($\clubsuit\clubsuit$).\\

\noindent \textbf{Case} 
$$
     \inferrule*[right = pair]
   {\Gamma_1 \jtype{n_1}{}{t_1}{\tau_1} \\ \Gamma_2 \jtype{n_2}{}{t_2}{\tau_2}}
   { \max(\Gamma_1, \Gamma_2)  \jtype{\max(n_1,n_2)}{}{(t_1, t_2)}{\tau_1 \times \tau_2}  }
$$
There are three sub cases,$x$ only in $\Gamma_1$,  $x$ only in $\Gamma_2$ and $x$ appears in both $\Gamma_1$ and $\Gamma_2$. When $x$ only in $\Gamma_1$, it is proved by induction hypothesis on the first premise and then using the rule \rname{PAIR}. When $x$ only in $\Gamma_2$, it is proved by induction hypothesis on the second premise and then using the rule \rname{PAIR}. When $x$  appears in both $\Gamma_1$ and $\Gamma_2$, it is proved by induction hypothesis on both premises and then using rule \rname{PAIR}.\\

\noindent \textbf{Case} 
$$
    \inferrule*[ right = case-const ]
   {\Gamma_1 \jtype{n_1}{}{t}{b} \\ \Gamma_2 \jtype{n_2}{}{t_i}{b} }
   {\max(\Gamma_1, \Gamma_2) \jtype{\max(n_1,n_2)}{}{\tcaseof{t}\ \{c_i \Rightarrow t_i\}_{c_i \in b} } {b} }
$$

There are three sub cases.\\

\textbf{Subcase 1: x $\not \in \fcv{\Delta_2}$ }
\[
   \inferrule*[ right = case-const ]
   {\Delta_1, x: [\tau_1]_p \jtype{n_1'}{}{t}{b}~(\star) \\ \Delta_2 \jtype{n_2'}{}{t_i}{b}~(\clubsuit) }
   {\max(\Delta_1, x: [\tau_1]_p, \Delta_2) \jtype{\max(n_1',n_2')}{}{\tcaseof{t}\ \{c_i \Rightarrow t_i\}_{c_i \in b} } {b} }
\]
where $\Delta = \max(\Delta_1, \Delta_2)$, $t_2 =\tcaseof{t}\ \{c_i \Rightarrow t_i\}_{c_i \in b}$, $\tau_2 = b$ and $n_2 = \max(n_1', n_2')$.\\ 
To show $ \max(\Gamma, \max(\Delta_1,\Delta_2) ) \jtype{\max(n_1 + p,\max(n_1',n_2') )}{}{  (\tcaseof{t}\ \{c_i \Rightarrow t_i\}_{c_i \in b})[t_1/x] }{b}  $. \\
 By induction hypothesis on ($\star$), we get $\max(\Gamma, \Delta_1) \jtype{\max(n_1 + p, n_1') }{}{t[t_1/x]}{b}~(\star\star) $. \\
 $x \not \in \fcv{\Delta_2} \implies t_i[t_1/x] =  t_i$.\\
 It is proved by the rule \rname{CASE-CONST} using ($\star\star$) and ($\clubsuit$). \\

\textbf{Subcase 2: x $\not \in \fcv{\Delta_1}$ }
\[
     \inferrule*[ right = case-const ]
   {\Delta_1 \jtype{n_1'}{}{t}{b}~(\star) \\ \Delta_2, x: [\tau_1]_p \jtype{n_2'}{}{t_i}{b}~(\clubsuit) }
   {\max(\Delta_1, x: [\tau_1]_p, \Delta_2) \jtype{\max(n_1',n_2')}{}{\tcaseof{t}\ \{c_i \Rightarrow t_i\}_{c_i \in b} } {b} }
\]
where $\Delta = \max(\Delta_1, \Delta_2)$, $t_2 =\tcaseof{t}\ \{c_i \Rightarrow t_i\}_{c_i \in b}$, $\tau_2 = b$ and $n_2 = \max(n_1', n_2')$.\\ 
To show $ \max(\Gamma, \max(\Delta_1,\Delta_2) ) \jtype{\max(n_1 + p,\max(n_1',n_2') )}{}{  (\tcaseof{t}\ \{c_i \Rightarrow t_i\}_{c_i \in b})[t_1/x] }{b}  $. \\
By induction hypothesis on ($\clubsuit$), we get $\max(\Gamma, \Delta_2) \jtype{\max(n_1 + p, n_2') }{}{t_i[t_1/x]}{b}~(\clubsuit\clubsuit) $. \\
 $x \not \in \fcv{\Delta_1} \implies t[t_1/x] = t $.\\
It is proved by the rule \rname{CASE-CONST} using ($\star$) and ($\clubsuit\clubsuit$). \\

\textbf{Subcase 3 }
\[
    \inferrule*[ right = case-const ]
   {\Delta_1, x: [\tau_1]_p \jtype{n_1'}{}{t}{b}~(\star) \\ \Delta_2, x: [\tau_1]_p \jtype{n_2'}{}{t_i}{b}~(\clubsuit) }
   {\max(\Delta_1, x: [\tau_1]_p, \Delta_2) \jtype{\max(n_1',n_2')}{}{\tcaseof{t}\ \{c_i \Rightarrow t_i\}_{c_i \in b} } {b} }
\]
where $\Delta = \max(\Delta_1, \Delta_2)$, $t_2 =\tcaseof{t}\ \{c_i \Rightarrow t_i\}_{c_i \in b}$, $\tau_2 = b$ and $n_2 = \max(n_1', n_2')$.\\ 
To show $ \max(\Gamma, \max(\Delta_1,\Delta_2) ) \jtype{\max(n_1 + p,\max(n_1',n_2') )}{}{  (\tcaseof{t}\ \{c_i \Rightarrow t_i\}_{c_i \in b})[t_1/x] }{b}  $. \\
By induction hypothesis on ($\star$), we get $\max(\Gamma, \Delta_1) \jtype{\max(n_1 + p, n_1') }{}{t[t_1/x]}{b}~(\star\star) $. \\
By induction hypothesis on ($\clubsuit$), we get $\max(\Gamma, \Delta_2) \jtype{\max(n_1 + p, n_2') }{}{t_i[t_1/x]}{b}~(\clubsuit\clubsuit) $.. \\
It is proved by the rule \rname{CASE-CONST} using ($\star\star$) and ($\clubsuit\clubsuit$).\\



\noindent \textbf{Case} 
\[
    \inferrule*[ right = case-query]
   {\Gamma_1 \jtype{n_1'}{}{t}{b} \\ \Gamma_2 \jtype{n_2'}{}{t_i}{query} }
   {\max(\Gamma_1, \Gamma_2) \jtype{\max(n_1',n_2')}{}{\tcaseof{t}\ \{c_i \Rightarrow t_i\}_{c_i \in b} } {query} }
\]
There are three sub cases. Thre proof are quite similar as the one of $\rname{CASE-CONST}$.\\

\noindent \textbf{Case} 
\[
   \inferrule*[right = iabs]
  { 
    \inferrule*[]
    {}
    {i::\mathbb{N};\Delta,x:[\tau_1]_p \jtype{n}{}{t}{ \tau } ~(\star)}
    \and
    \inferrule*[]
    {}
    { i \notin \fiv{\Delta,x:\tau_1}  } 
  }
  { \Delta, x:[\tau_1]_p \jtype{n}{ }{  \Lambda.t  }{ \tforallN{i}{\tau}  } }
 \]
To show $\max(\Gamma, \Delta) \jtype{\max(n_1 + p,n)}{}{\Lambda.t[t_1/x] }{\tforallN{i}{\tau}} $.\\
 By induction hypothesis on ($\star$), we get : $ i :: \mathbb{N};\max(\Gamma, \Delta) \jtype{\max(n_1,n)}{}{t[t_1/x]}{\tau}(\star\star)$.\\
 There are two cases.\\
 \textbf{Sub case 1: $i \not \in \fiv{\Gamma}$}
  It is proved by using rule \rname{IABS} with $(\star\star)$.\\
  \textbf{Sub case 2: $i \in \fiv{\Gamma}$}
  We choose $j \not \in \fiv{\max(\Gamma,\Delta)}$. We rename all the $i$ to $j$ in $(\star\star)$ and use rule \rname{IABS} to get:
  $ \max(\Gamma, \Delta) \jtype{\max(n_1,n)}{}{\Lambda.t[t_1/x] }{\tforallN{j}{\tau}}$. It is just a renaming version of the goal.
 

\noindent \textbf{Case} 
\[ 
   \inferrule*[ right =  iapp]
  { 
    \inferrule*[]
    {}
    { \Delta, x:[\tau_1]_p \jtype{n}{}{t}{ \tforallN{i}{\tau}   } (\star) }
    \and
    \inferrule*[]
    {}
    { \jiterm{I}{ \mathbb{N} } } 
  }
  {\Delta, x:[\tau_1]_p \jtype{n }{ }{t\, [] }{ \tau \{ I/i \}  } }
\]
To show $ \max(\Gamma, \Delta) \jtype{\max(n_1 + p, n)}{}{t[t_1/x] \, []}{ \tau \{ I/i \} } $.\\
By ih on $(\star)$, we get: $ \max(\Gamma,\Delta)  \jtype{\max(n_1 + p, n)}{}{t}{ \tforallN{i}{\tau} } ~ (\star\star) $.\\
It is proved by the rule \rname{IAPP} with $(\star\star)$.\\

\noindent \textbf{Case} 
\[
  \inferrule*[right = sub]
  { 
   { \Delta, x:[\tau_1]_p \jtype{n}{}{t}{\tau} ~(\star) } \\
   { \Delta, x:[\tau_1]_p\subseteq \Delta',x:\tau_1 ~ (\clubsuit) }  \\
   { \vDash n \leq n' } ~(\spadesuit) \\
   { \tau \subseteq \tau' }
  }
  { \Delta',x:[\tau_1]_p \jtype{n'}{}{t}{\tau'} }
\]
To show $ \max(\Gamma, \Delta') \jtype{\max(n_1 + p, n')}{}{t}{\tau'}$.\\
By induction hypothesis on ($\star$), we get : $\max(\Gamma, \Delta) \jtype{\max(n_1 + p, n)}{}{t}{\tau} $.\\
$(\clubsuit) \implies \max(\Gamma,\Delta) \subseteq \max(\Gamma,\Delta')~(\clubsuit\clubsuit) $.\\
$ (\spadesuit) \implies  \vDash \max(n_1 + p, n) \leq \max(n_1 + p,n') $.\\
It is proved by the rule \rname{SUB}.\\


\noindent \textbf{Case}
\[
\inferrule*[right = pr]
   {[\Delta_2] \jtype{n}{}{t}{\tau}}
   {\Delta_1, p' + [\Delta_2], x:[\tau_1]_p\jtype{n}{}{!t}{!_p' \tau}  }
\]
TS: $\max(\Gamma,(\Delta_1, p+[\Delta_2]) ) \jtype{\max(n_1 + p, n)}{}{!t[t_1/x]}{!_p \tau}$.\\
$x \not \in [\Delta_2] \implies [\Delta_2] \jtype{n}{}{t[t_1/x]}{\tau}(\star)$.\\
By rule \rname{PR} with $(\star)$, we get: $\Delta_1, p'+[\Delta_2] \jtype{n}{}{!t[t_1/x]}{!_{p'} \tau} $.
It is proved by the Lemma \ref{lem:coex} and Lemma~\ref{lem:deweaken1} from the conclusion.


\end{proof}


\end{document}




































