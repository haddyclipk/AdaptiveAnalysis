
\documentclass[a4paper,11pt]{article}
\usepackage[table]{xcolor}

%%% Attempt 1: Linear 1



\newcommand{\diam}{{\color{red}\diamond}}
\newcommand{\dagg}{{\color{blue}\dagger}}
\let\oldstar\star
\renewcommand{\star}{\oldstar}

\newcommand{\im}[1]{\ensuremath{#1}}

\newcommand{\kw}[1]{\im{\mathtt{#1}}}


\newcommand{\set}[1]{\im{\{{#1}\}}}

\newcommand{\mmax}{\ensuremath{\mathsf{max}}}

%%%%%%%%%%%%%%%%%%%%%%%%%%%%%%%%%%%%%%%%%%%%%%%%%%%%%%%%
% Comments
\newcommand{\omitthis}[1]{}

% Misc.
\newcommand{\etal}{\textit{et al.}}
\newcommand{\bump}{\hspace{3.5pt}}

% Text fonts
\newcommand{\tbf}[1]{\textbf{#1}}
%\newcommand{\trm}[1]{\textrm{#1}}

% Math fonts
\newcommand{\mbb}[1]{\mathbb{#1}}
\newcommand{\mbf}[1]{\mathbf{#1}}
\newcommand{\mrm}[1]{\mathrm{#1}}
\newcommand{\mtt}[1]{\mathtt{#1}}
\newcommand{\mcal}[1]{\mathcal{#1}}
\newcommand{\mfrak}[1]{\mathfrak{#1}}
\newcommand{\msf}[1]{\mathsf{#1}}
\newcommand{\mscr}[1]{\mathscr{#1}}









\newcommand{\defeq}{\mathrel{\doteq}}
\newcommand{\conj}{\mathrel{\wedge}}
\newcommand{\disj}{\mathrel{\vee}}

\newcommand{\lzero}{0}


% context
\newcommand{\tctx}{\Gamma}
\newcommand{\ictx}{ }


% expression
\newcommand{\expr}{e}
\newcommand{\aexpr}{a}
\newcommand{\bexpr}{b}
\newcommand{\sexpr}{\textrm{e} }
\newcommand{\qexpr}{\psi}
\newcommand{\qval}{\alpha}
\newcommand{\query}{{\tt query}}
\newcommand{\saexpr}{\textrm{a} }
\newcommand{\sbexpr}{\textrm{b} }
\newcommand{\vall}{w}
\newcommand{\valr}{v}
\newcommand{\eif}{\kw{if}}
\newcommand{\eapp}{\;}
\newcommand{\eprojl}{\kw{fst}}
\newcommand{\eprojr}{\kw{snd}}
\newcommand{\eifvar}{\kw{ifvar}}
%expression and commands for WHILE language
\newcommand{\ewhile}{\kw{while}}
\newcommand{\bop}{*}
\newcommand{\uop}{\circ}
\newcommand{\eskip}{\kw{skip}}

\newcommand{\eloop}{\kw{loop}}
\newcommand{\edo}{\kw{do}}
\newcommand{\qdom}{\mathcal{QD}}

%configuration
\newcommand{\config}[1]{\langle #1 \rangle}
\newcommand{\ematch}{\kw{match}}
\newcommand{\clabel}[1]{\left[ #1 \right]}


%\newcommand{\eprov}[1]{\eta_{#1}}
\newcommand{\etrue}{\kw{true}}
\newcommand{\efalse}{\kw{false}}
\newcommand{\econst}{c}
\newcommand{\eop}{\delta}
\newcommand{\efix}{\mathop{\kw{fix}}}
\newcommand{\elet}{\mathop{\kw{let}}}
\newcommand{\ein}{\mathop{ \kw{in}} }
\newcommand{\eas}{\mathop{ \kw{as}} }
\newcommand{\enil}{\kw{nil}}
\newcommand{\econs}{\mathop{\kw{cons}}}
%\newcommand{\labelA}{\ell}
%monad expressions / terms
\newcommand{\term}{t}
\newcommand{\return}{\kw{return}}
\newcommand{\bernoulli}{\kw{bernoulli}}
\newcommand{\uniform}{\kw{uniform}}
 \newcommand{\epack}{\mbox{pack\;}}
\newcommand{\eunpack}{\mbox{unpack\;}}
\newcommand{\eilam}{\Lambda.}

\newcommand{\evec}{\kw{dict}}
\newcommand{\eget}{\kw{get}}

% trace
\newcommand{\triapp}[2]{\kw{IApp}(#1,#2)}
\newcommand{\trow}{\text{row}}
\newcommand{\tr}{T}
\newcommand{\trift}{\eif^{\kw{t}}}
\newcommand{\triff}{\eif^{\kw{f}}}
\newcommand{\trprojl}{\eprojl}
\newcommand{\trprojr}{\eprojr}
\newcommand{\trtrue}{\etrue}
\newcommand{\trfalse}{\efalse}
\newcommand{\trconst}{\econst}
\newcommand{\trop}{\eop}
\newcommand{\trfix}{\efix}
\newcommand{\trapp}[5]{#1 \; #2 \mathrel{\triangleright} {\efix
#3(#4).#5}}
\newcommand{\trnil}{\enil}
\newcommand{\trcons}{\econs}
\newcommand{\trlet}{\elet}
%types for monad
\newcommand{\treal}{\kw{real}}
\newcommand{\tint}{\kw{int}}
\newcommand{\tmonad}{\kw{M}}
\newcommand{\tunit}{\kw{unit}}
\newcommand{\tdb}{\kw{tdb}}

% adaptivity
\newcommand{\adap}{\kw{adap}}
\newcommand{\ddep}[1]{\kw{depth}_{#1}}
\newcommand{\nat}{\mathbb{N}}
\newcommand{\natb}{\nat_{\bot}}
\newcommand{\natbi}{\natb^\infty}
\newcommand{\nnatA}{Z}
\newcommand{\nnatB}{m}
\newcommand{\nnatbA}{s}
\newcommand{\nnatbB}{t}
\newcommand{\nnatbiA}{q}
\newcommand{\nnatbiB}{r}

%type
\newcommand{\type}{\tau}
\newcommand{\tbase}{\kw{b}}
\newcommand{\tbool}{\kw{bool}}
\newcommand{\tbox}[1]{ \kw{\square} \, (#1) }
\newcommand{\tarr}[5]{#1; #3 \xrightarrow{#4; \, #5} #2}
\newcommand{\tlist}[1]{\kw{list} \, #1 }
\newcommand{\env}{\theta}
\newcommand{\tforall}[3]{\forall#3 \overset{#1, #2}{::} S.\, }
\newcommand{\texists}[1]{\exists#1 {::} S.\, }
\newcommand{\lto}{\multimap}
\newcommand{\bang}[1]{ !_{#1}}
\newcommand{\whynot}[1]{ ?_{#1} }
\newcommand{\ltype}{A}
\newcommand{\adapt}{R}
% index
\newcommand{\idx}{I }
\newcommand{\smax}[2]{\kw{max}(#1,#2)}
\newcommand{\ienv}{\sigma}

%evaluation
\newcommand{\bigstep}[1]{\mathrel{\to^{#1}}}

\newcommand{\dmap}{\rho}
\newcommand{\dmapb}{\bot_\dmap}
\newcommand{\supp}{\kw{supp}}
\newcommand{\dom}{\kw{dom}}
\newcommand{\codom}{\kw{codom}}

\newcommand{\tvdash}[1]{\vdash_{#1}}

\newcommand{\lrv}[1]{[\![ #1 ]\!]_{\text{V}}}
\newcommand{\lre}[3]{[\![ #3 ]\!]_{\text{E}}^{#1, #2}}
\newcommand{\stepiA}{k}
\newcommand{\stepiB}{j}
\newcommand{\size}[1]{|#1|}

%logic relations
\newcommand{\lr}[1]{[\![ #1 ]\!]}
\newcommand{\lrt}[1]{\mathcal{T}[\![ #1 ]\!]}


\newcommand{\wf}[1]{\vdash #1 \, \kw{wf} }
\newcommand{\sub}[2]{ #1 \, <: \, #2 }
\newcommand{\eqv}[3]{ #1 \, \equiv \, #2 \Rightarrow \textcolor{red}
{#3}  }
\newcommand{\eqvt}[3]{ #1 \, \sqsubseteq \, #2 \Rightarrow \textcolor{red}
{#3}  }
\newcommand{\eqvc}[2]{ #1 \, \equiv^c \, #2   }


%core calculus
\newcommand{\ctyping}[3]{ \tvdash{ #1} {#2} :^c #3 }
\newcommand{\cbox}{\mathsf{box}}
\newcommand{\cder}{\mathsf{der}}
\newcommand{\elab}[4]{ \vdash_{ #1} #2 \rightsquigarrow #3 : #4}
\newcommand{\coerce}[2]{\mathsf{coerce}_{#1, #2}}

%algorithmic typing rules
\newcommand{\infr}[4]{{#1} ~ {\textcolor{red}\uparrow} ~ {\color{red} #2} \Rightarrow
{ } {\color{red} #3} }
\newcommand{\chec}[3]{{#1} ~ {\downarrow} ~ {#2} \Rightarrow {\color{red} #3} }
% \newcommand{\restriction}{\Phi}
\newcommand{\fresh}{ \mathsf{fresh}}
\newcommand{\red}[1]{ \textcolor{red} {#1} }
\newcommand{\fiv}[1]{ \mathsf{FIV} (#1)   }
\newcommand{\fv}[1]{ \mathsf{FV} (#1)   }

\newcommand{\todo}[1]{{\small \color{red}\textbf{[[ #1 ]]}}}
\newcommand{\todomath}[1]{{\scriptstyle \color{red}\mathbf{[[ #1 ]]}}}

\newcommand{\caseL}[1]{\item \textbf{#1}\newline}

\newcommand{\attr}{\mathsf{attr}}
\newcommand{\weight}{\mathsf{W}}
\newcommand{\num}{\mathsf{n}}

\usepackage{enumitem}
\setenumerate{listparindent=\parindent}

\newlist{enumih}{enumerate}{3}
\setlist[enumih]{label=\alph*),before=\raggedright, topsep=1ex, parsep=0pt,  itemsep=1pt }

\newlist{enumconc}{enumerate}{3}
\setlist[enumconc]{leftmargin=0.5cm, label*= \arabic*.  , topsep=1ex, parsep=0pt,  itemsep=3pt }


\newlist{enumsub}{enumerate}{3}
\setlist[enumsub]{ leftmargin=0.7cm, label*= \textbf{subcase} \bf \arabic*: }

\newlist{enumsubsub}{enumerate}{3}
\setlist[enumsubsub]{ leftmargin=0.5cm, label*= \textbf{subsubcase} \bf \arabic*: }

\newlist{mainitem}{itemize}{3}
\setlist[mainitem]{ leftmargin=0cm , label= {\bf Case} }

%%%%COLORS
\definecolor{periwinkle}{rgb}{0.8, 0.8, 1.0}
\definecolor{powderblue}{rgb}{0.69, 0.88, 0.9}
\definecolor{sandstorm}{rgb}{0.93, 0.84, 0.25}
\definecolor{trueblue}{rgb}{0.0, 0.45, 0.81}


\usepackage{array}

\newlength\Origarrayrulewidth

% horizontal rule equivalent to \cline but with 2pt width
\newcommand{\Cline}[1]{%
 \noalign{\global\setlength\Origarrayrulewidth{\arrayrulewidth}}%
 \noalign{\global\setlength\arrayrulewidth{2pt}}\cline{#1}%
 \noalign{\global\setlength\arrayrulewidth{\Origarrayrulewidth}}%
}

% draw a vertical rule of width 2pt on both sides of a cell
\newcommand\Thickvrule[1]{%
  \multicolumn{1}{!{\vrule width 2pt}c!{\vrule width 2pt}}{#1}%
}

% draw a vertical rule of width 2pt on the left side of a cell
\newcommand\Thickvrulel[1]{%
  \multicolumn{1}{!{\vrule width 2pt}c|}{#1}%
}

% draw a vertical rule of width 2pt on the right side of a cell
\newcommand\Thickvruler[1]{%
  \multicolumn{1}{|c!{\vrule width 2pt}}{#1}%
}

\newcommand{\command}{c}
\newcommand{\green}[1]{{ \color{green} #1 } }

\newcommand{\func}[2]{\mathsf{AD}(#1) \to (#2)}
\newcommand{\varEst}{\bf{VetxEst}}
\newcommand{\graphGen}{\bf{GraphGen}}

\newcommand{\ag}[2]{\mathsf{VetxEst}{(#1)}\to {(#2)}}
\newcommand{\ad}[2]{\mathsf{GraphGen}{(#1)}\to {(#2)}}
\newcommand{\rb}{\mathsf{RechBound}}
\newcommand{\pathsearch}{\mathsf{AdaptPathSearch}}

%Packages
\usepackage[T1]{fontenc}
\usepackage{fourier} 
\usepackage[english]{babel} 
\usepackage{amsmath,amsfonts,amsthm} 
\usepackage{lscape}
\usepackage{geometry}
\usepackage{amsmath}
\usepackage{algorithm}
\usepackage{algorithmic}
\usepackage{amssymb}
\usepackage{amsfonts}
\usepackage{times}
\usepackage{bm}
\usepackage{ stmaryrd }
\usepackage{ amssymb }
\usepackage{ textcomp }
\usepackage[normalem]{ulem}
% For derivation rules
\usepackage{mathpartir}
\usepackage{color}
\usepackage{a4wide}


\newcommand{\sctag}[1]{\tag{\textsc{#1}}\label{eq:#1}}
\newcommand{\ands}{~\wedge~}

%Variations
\newcommand{\astable}{\mathbb{S}}
\newcommand{\achange}{U}

%Index Terms
\newcommand{\scond}[3]{(\eif\im{#1}\ethen\im{#2}\eelse\im{#3})}
\newcommand{\keps}[2]{\epsilon(#1,#2)}
\newcommand{\spower}[2]{#1^{#2}}
\newcommand{\ssum}[4]{\sum\limits_{#1=#2}^{#3}#4}
\newcommand{\smin}[2]{\kw{min}({#1},{#2})}
\newcommand{\smax}[2]{\kw{max}(#1,#2)}
\newcommand{\smaxx}[3]{\kw{max}(#1,#2,#3)}
\newcommand{\smaxxx}[4]{\kw{max}(#1,#2,#3,#4)}

%Sizes
\newcommand{\szero}{0}
\newcommand{\sone}{1}
\newcommand{\splus}[2]{#1 + #2}
\newcommand{\ssucc}[1]{#1 {+} \sone}
\newcommand{\sminus}[2]{#1 - #2}
\newcommand{\sdiv}[2]{\frac{#1}{#2}}
\newcommand{\smult}[2]{#1\cdot#2}
\newcommand{\splusone}[1]{#1+1}
\newcommand{\sceil}[1]{\ceil*{#1}}
\newcommand{\sfloor}[1]{\floor*{#1}}
\newcommand{\size}[1]{|#1|}
\newcommand{\slog}[1]{\kw{log}_2(#1)}
\newcommand{\sinf}{\infty}

%Sorts
\newcommand{\ssize}{\mathbb{N}}
\newcommand{\svar}{\mathbb{V}}
\newcommand{\scost}{\mathbb{R}}
\newcommand{\sfun}[2]{#1\mbox{\ra} #2}
\newcommand{\sfunmon}[2]{#1\xrightarrow{\mbox{mon}} #2}
% \newcommand{\sort}{\varsigma}
\newcommand{\sort}{S}
\newcommand{\sorted}[1]{#1 \mathrel{::} \sort}
\newcommand{\sized}[1]{#1 \mathrel{::} \ssize}

% Types
\newcommand{\grt}{A}
\newcommand{\lbound}{\mathop{\uparrow}}
\newcommand{\tbool}{\mbox{bool}}
\newcommand{\trbool}{\mbox{bool}_r}
\newcommand{\tubool}{\mbox{bool}_u}
\newcommand{\trint}{\mbox{int}_r}
\newcommand{\tint}{\mbox{int}}
\newcommand{\tquery}{\mbox{query}}
\newcommand{\tunit}{\mbox{unit}}
\newcommand{\trunit}{\mbox{unit}_r}
\newcommand{\tlist}[3]{\mbox{list}[#1]^{#2}\,#3}
\newcommand{\tlists}[1]{ #1 \, \mbox{list} }
\newcommand{\ulist}[2]{\mbox{list}[#1]\,#2}
\newcommand{\tslist}[1]{\mbox{list}\,#1}
\newcommand{\ttree}[3]{\mbox{tree}[#1]^{#2}\,#3}
\newcommand{\utree}[2]{\mbox{tree}[#1]\,#2}
\newcommand{\tbase}{b}
\newcommand{\uarr}[2]{\mathrel{\xrightarrow[]{\wexec(#1,#2)}}} 
\newcommand{\uarrs}[1]{\mathrel{\xrightarrow[]{\mu(#1)}}} 
\newcommand{\uarrd}{\mathrel{\xrightarrow{\wdead}}}
\newcommand{\tarrd}[1]{\mathrel{\xrightarrow{\wdiff(#1)}}}



\newcommand{\tforall}[3]{\forall#1\overset{\wexec(#2,#3)}{::}S.\,}
\newcommand{\tforalld}[2]{\forall#1\overset{\wdiff(#2)}{::}S.\,}
\newcommand{\uforalls}[2]{\forall#1\overset{\mu(#2)}{::}S.\,}
\newcommand{\tforallS}[1]{\forall#1.\,}
\newcommand{\tsforall}[1]{\forall#1{::}S.\,}
\newcommand{\tforallN}[1]{\forall#1{::}\ssize.\,}
\newcommand{\texists}[1]{\exists#1{::}S.\,}
\newcommand{\texistsN}[1]{\exists#1{::}\ssize.\,}
\newcommand{\tcimpl}[2]{#1 \mathrel{\supset} #2}
\newcommand{\tcprod}[2]{#1 \mathrel{\&} #2}

\newcommand{\ttimes}{\mathrel{\times}}
\newcommand{\tsum}{\mathrel{+}}
\newcommand{\tinter}{\mathrel{\wedge}}
\newcommand{\tst}[1]{(#1)^{\astable}}
\newcommand{\tch}[2]{U\,(#1,#2)} 
\newcommand{\tchs}[1]{U\,(#1,#1)} 
\newcommand{\tcho}[1]{U\,#1} 
\newcommand{\tmu}[1]{(#1)^{\mu}}
\newcommand{\tno}[1]{(#1)^{\_}}
\newcommand{\trm}[2]{|#1|_{#2}}
\newcommand{\trmo}[1]{|#1|}
\newcommand{\tmup}[1]{(#1)^{\mu'}}
\newcommand{\tdual}[1]{d({#1})}
\newcommand{\tbox}[1]{\square\,#1}
\newcommand{\tdmu}[1]{#1^{\shortdownarrow {\mu}}}
\newcommand{\tmon}[1]{{\color{red}m(#1)}}
\newcommand{\tforce}[1]{#1^{\shortdownarrow \achange }}
\newcommand{\tlift}[2]{(#1,#2)^{\uparrow}}
\newcommand{\tpull}[1]{#1^{\nearrow}}
\newcommand{\tpushd}[1]{(#1)^{\downarrow\square}}

% Terms
\newcommand{\vbase}{r}
\newcommand{\vtrue}{\mbox{tt}}
\newcommand{\vfalse}{\mbox{ff}}

\newcommand{\la}{\langle} 
\newcommand{\ra}{\rangle}
\newcommand{\eapp}{\;} 
\newcommand{\eleft}{\pi_1}
\newcommand{\eright}{\pi_2} 
\newcommand{\econst}{\kw{n}}
\newcommand{\etrue}{\mbox{true}}
 \newcommand{\efalse}{\mbox{false}}
\newcommand{\eif}{\mbox{if\;}} 
\newcommand{\ethen}{\mbox{\;then\;}}
\newcommand{\eelse}{\mbox{\;else\;}} 
\newcommand{\einl}{\mbox{inl\;}}
\newcommand{\einr}{\mbox{inr\;}} 
\newcommand{\elet}{\mbox{let\;}}
\newcommand{\clet}{\mbox{clet}\;}
\newcommand{\ecimp}{\mbox{.}_c\;}
\newcommand{\eelimU}{\mbox{elim}_U\;} 
\newcommand{\ein}{\mbox{\;in\;}}
\newcommand{\ecase}{\mbox{\;case\;}} 
\newcommand{\eof}{\mbox{\;of\;}}
\newcommand{\eas}{\mbox{\;as\;}} 
\newcommand{\ecelim}{\mbox{celim\;}}
\newcommand{\enil}{\mbox{nil}} 
\newcommand{\epack}{\mbox{pack\;}}
\newcommand{\eunpack}{\mbox{unpack\;}}
\newcommand{\efix}{\mbox{fix\;}} 
\newcommand{\efixNC}{\mbox{fix$_{NC}$\;}} 
\newcommand{\eLam}{ \Lambda}
\newcommand{\elam}{ \lambda} 
\newcommand{\eApp}{ [\,]\,}
\newcommand{\eleaf}{\mbox{leaf}} 
\newcommand{\ewith}{\;\mbox{with}\;} 
\newcommand{\enode}{\mbox{node}}
\newcommand{\econs}{\mbox{cons}} 
\newcommand{\econsC}{\mbox{cons$_C$}} 
\newcommand{\econsNC}{\mbox{cons$_{NC}$}} 
\newcommand{\eunit}{()}
\newcommand{\eswitch}{\kw{switch}\;}
\newcommand{\enoch}{\kw{NC}\;}
\newcommand{\eder}{\kw{der}\;}
\newcommand{\esplit}{\kw{split}\;}
\newcommand{\ecoerce}[2]{\kw{coerce}_{#1,#2}\;}
\newcommand{\econtra}{\kw{contra}\;}

\newcommand{\ealloc}[2]{ \mathrel{ \mathsf{alloc}\, {#1} \, {#2} } }
\newcommand{\eallocB}[2]{ \mathrel{ \mathsf{alloc_b}\, {#1} \, {#2} } }
\newcommand{\eupdt}[3]{ \mathrel{ \mathsf{update} \ {#1} \ {#2} \ {#3} }  }
\newcommand{\ereadx}[2] { \mathrel{ \mathsf{read} \ {#1} \ {#2} }  }
\newcommand{\eupdtB}[3]{ \mathrel{ \mathsf{update_b} \ {#1} \ {#2} \ {#3} }  }
\newcommand{\ereadxB}[2] { \mathrel{ \mathsf{read_b} \ {#1} \ {#2} }  }
\newcommand{\eret}[1] {\mathrel{ \mathsf{return} \, {#1} }}
\newcommand{\eletx}[3]{  \mathrel{ \mathsf{let_m} \{ {#1} \} = {#2} \ \mathsf{in} \ {#3}  } }



\newcommand{\caseof}[1]{\mbox{case}~#1~\mbox{of}}
\newcommand{\tcaseof}[1]{\mathsf{case}~#1~\mathsf{of}}
\newcommand{\ofnil}[1]{~~\mbox{nil}~\to#1}
\newcommand{\ofzero}[1]{~~\kw{0}~\to#1}
\newcommand{\ofcons}[3]{|~#1::#2~\to~#3}

% Diff Rel
\newcommand{\udiff}{\gtrapprox}
\newcommand{\rdiff}{\ominus}
\newcommand{\rdiffs}{\lesssim}
\newcommand{\ldiff}{\lesssim}

% Evaluation
\newcommand{\red}[1]{\Downarrow^{#1}}

\newcommand{\wmax}{\mbox{\scriptsize max}}
\newcommand{\wmin}{\mbox{\scriptsize min}}
\newcommand{\wdiff}{\mbox{\scriptsize diff}}
\newcommand{\wexec}{\mbox{\scriptsize exec}}
\newcommand{\wdead}{\mbox{\scriptsize dead}}
%Logical relation
\newcommand{\step}{\text{Step index}}
\newcommand{\world}{\text{World}}
\newcommand{\values}{\text{Value}}
\newcommand{\expr}{\text{Expression}}
\newcommand{\ulr}[1]{\llbracket#1\rrbracket_{v}}
\newcommand{\ulrg}[1]{\llbracket#1\rrbracket_{\grt}}
\newcommand{\lr}[1]{\llparenthesis#1\rrparenthesis_{v}}
\newcommand{\lre}[2]{\llparenthesis#1\rrparenthesis_{\varepsilon}^{#2}}
\newcommand{\lrg}[1]{\llparenthesis#1\rrparenthesis_{\grt}}
\newcommand{\ulre}[3]{\llbracket#1\rrbracket_{\varepsilon}^{#2,#3}}
\newcommand{\ulrew}[1]{\llbracket#1\rrbracket_{\varepsilon}^{0,\sinf}}

\newcommand{\relwith}[2]{\{#1~|~#2\}}
\newcommand{\rel}[1]{\{#1\}}
\newcommand{\del}[1]{\mathcal{D}\llbracket#1\rrbracket}
\newcommand{\dd}[1]{\mathcal{D}\llbracket\Delta\rrbracket}
\newcommand{\ugsubst}[1]{\mathcal{G}\llbracket#1\rrbracket}
\newcommand{\gsubst}[1]{\mathcal{G}\llparenthesis#1\rrparenthesis}
\newcommand{\dsubst}[1]{\mathcal{D}\llbracket#1\rrbracket}
\newcommand{\s}{\sigma}
\newcommand{\peq}{\preceq}
\newcommand{\plt}{\prec}
\renewcommand{\d}{\delta}
\newcommand{\g}{\gamma}


% Typing judgments
\newcommand{\jiterm}[2]{\mathrel{\vdash {#1} :: #2}}
\newcommand{\jtype}[4]{\mathrel{\vdash_{#1}^{#2} {#3} : {#4}}}
\newcommand{\jtypeM}[4]{\mathrel{\vdash_{#1}^{#2} {#3} :^c {#4}}}

\newcommand{\jtypes}[3]{\mathrel{\vdash_{#1}^{\mu} {#2} : {#3}}}

\newcommand{\jstype}[3]{\mathrel{\vdash
    {#1} \backsim {#2} : {#3}}}


\newcommand{\jtypediff}[4]{\mathrel{\vdash% _{\wdiff}
    {#2} \ominus {#3} \ldiff #1 : {#4}}}

\newcommand{\jtypediffM}[4]{\mathrel{\vdash
    {#2} \ominus {#3} \ldiff #1 :^c {#4}}}
\newcommand{\jmintypesame}[3]{\mathrel{\vdash
    {#2} \ominus {#2} \ldiff #1 :^c {#3}}}

\newcommand{\jelab}[6]{\mathrel{\vdash
    {#2} \ominus {#3} \rightsquigarrow {#4} \ominus {#5} \ldiff #1 : {#6}}}
\newcommand{\jelabsame}[4]{\mathrel{\vdash
    {#2} \ominus {#3} \rightsquigarrow {#2} \ominus {#3} \ldiff #1 : {#4}}}

\newcommand{\jelabun}[5]{\mathrel{\vdash_{#1}^{#2}
    {#3} \rightsquigarrow {#4} : {#5}}}

\newcommand{\jelabc}[4]{\mathrel{\vdash
    {#2} \ominus {#3} \rightsquigarrow {#2}^* \ominus {#3}^* \ldiff #1 : {#4}}}


\newcommand{\jelabcu}[4]{\mathrel{\vdash_{#1}^{#2}
    {#3} \rightsquigarrow {#3}^* : {#4}}}

\newcommand{\jtypediffsym}[5]{\mathrel{\vdash
    #1 \ldiff {#3} \ominus {#4} \ldiff #2 : {#5}}}
\newcommand{\sty}[2]{\vdash#1 \mathrel{::} #2}


\newcommand{\rname}[1]{\mbox{\small{#1}}}

\newcommand{\vsem}[2]{\llbracket #1 \rrbracket_{V}^{#2}}
\newcommand{\esem}[2]{\llbracket #1 \rrbracket_{E}^{#2}}
\newcommand{\conj}{\mathrel{\wedge}}

\newcommand{\vusem}[1]{\llparenthesis #1 \rrparenthesis_{V}}
\newcommand{\eusem}[1]{\llparenthesis #1 \rrparenthesis_{E}}

\newcommand{\jsubtype}[2]{\sat#1\sqsubseteq#2}
\newcommand{\jasubtype}[2]{\sat^{\mathsf{\grt}}#1\sqsubseteq#2}
\newcommand{\jeqtype}[2]{\sat#1 \equiv#2}
\newcommand{\under}[2]{\sat #1 \trianglelefteq  #2}

\newcommand{\type}{\text{type}}
\newcommand{\rtype}{\text{relational type}}
\newcommand{\Type}{\text{Unary type}}
\newcommand{\Rtype}{\text{Binary type}}



% Cost Constants
\newcommand{\kvar}{c_{var}}
\newcommand{\kconst}{c_{n}}
\newcommand{\kinl}{c_{inl}}
\newcommand{\kinr}{c_{inl}}
\newcommand{\kcase}{c_{case}}
\newcommand{\kfix}{c_{fix}}
\newcommand{\kapp}{c_{app}}
\newcommand{\kLam}{c_{fix}}
\newcommand{\kiApp}{c_{iApp}}
\newcommand{\kpack}{c_{pack}}
\newcommand{\kunpack}{c_{unpack}}
\newcommand{\knil}{c_{nil}}
\newcommand{\kcons}{c_{cons}}
\newcommand{\kcaseL}{c_{caseL}}
\newcommand{\kleaf}{c_{leaf}}
\newcommand{\knode}{c_{node}}
\newcommand{\kcaseT}{c_{caseT}}
\newcommand{\kprod}{c_{prod}}
\newcommand{\kproj}{c_{proj}}
\newcommand{\klet}{c_{let}}


%Constraints
\newcommand{\creal}{\mathbb{R}}
\newcommand{\sat}[1]{\models#1}
\newcommand{\sata}[1]{\models_A#1}
\newcommand{\ceq}[2]{#1\mathrel{\doteq}#2}
\newcommand{\cleq}[2]{#1 \mathop{\leq} #2}
\newcommand{\cleqspec}[2]{#1 \overline{\mathop{\leq}} #2}
\newcommand{\clt}[2]{#1 \mathop{<} #2}
\newcommand{\cgt}[2]{#1 \mathop{>} #2}
\newcommand{\ceqz}[1]{#1 \mathrel{\doteq} 0}
\newcommand{\cneg}[1]{\mathop{\neg}#1}
\newcommand{\cand}[2]{#1 \wedge #2}
\newcommand{\cexists}[3]{\exists#1::#2.#3}
\newcommand{\cexistsK}[3]{\exists#1:#2.#3}
\newcommand{\cexistsS}[2]{\exists#1.#2}
\newcommand{\cexistsC}[2]{\exists#1::\scost.#2}
\newcommand{\cexistsall}[2]{\exists(#1).#2}
\newcommand{\cforall}[3]{\forall#1::#2.#3}
\newcommand{\cforallS}[3]{\forall#1:#2.#3}
\newcommand{\cimpl}[2]{#1\rightarrow#2}
\newcommand{\cor}[2]{#1 \vee #2}
\newcommand{\ctrue}{\top}
\newcommand{\cfalse}{\bottom}
\newcommand{\blank}[2][100]{\hfil\penalty#1\hfilneg }
\newcommand{\ccond}[3]{\im{#1}\mathrel{\mbox{?}}\im{#2}\mathrel{\colon}\im{#3}}


\newcommand{\wfty}[1]{\vdash   #1~\kw{wf}}
\newcommand{\awfty}[1]{\vdash^{\mathsf{\grt}}   #1~\kw{wf}}
\newcommand{\wfcs}[1]{\vdash   #1~\kw{wf}}
\newcommand{\wfctx}[1]{\vdash   #1~\kw{wf}}
\newcommand{\awfctx}[1]{\vdash^{\mathsf{\grt}}   #1~\kw{wf}}



\newcommand{\dc}{ downward closure (\lemref{lem:down-closure}) }
\newcommand{\ctx}{\Delta; \Phi_a; \Gamma}
\newcommand{\nctx}{\Delta; \Phi_a; \tbox{\Gamma}}
\newcommand{\primctx}{\Upsilon}
\newcommand{\octx}{\Delta; \Phi_a; \Omega}
\newcommand{\rctx}[1]{\Delta; \Phi_a; \trm{\Gamma}{#1}}

\newcommand{\shade}[1]{\colorbox{lightgray}{#1}}
\newcommand{\fv}[1]{\text{FV}(#1)}
\newcommand{\fcv}[1]{\text{dom}(#1)}
\newcommand{\fiv}[1]{\text{FIV}(#1)}
\newcommand{\fdv}[1]{\text{dom}(#1)}


\newcommand{\assC}[2]{\text{Assume that $\sat \s \Phi$ and there exists $\Gamma'$  s.t. $\fv{#2} \subseteq \fcv{\Gamma'} $ and $\Gamma' \subseteq \Gamma$ and $(m, \d) \in \ugsubst{\trm{\sigma \Gamma'}{#1}}$}}
\newcommand{\assCU}[1]{\text{Assume that $\sat \s \Phi$ and there exists $\Omega'$  s.t. $\fv{#1} \subseteq \fcv{\Omega'} $ and $\Omega' \subseteq \Omega$ and $(m, \d) \in \ugsubst{\s \Omega'}$}}
\newcommand{\IHassun}[1]{\text{$\fv{#1} \subseteq \fcv{\Omega'} $ and $\Omega' \subseteq \Omega$ and $(m, \d) \in \ugsubst{\s \Omega'}$}}


\newcommand{\IHassU}[2]{\text{$\fv{#2} \subseteq \fcv{\trm{\Gamma'}{#1}} $ and $\trm{\Gamma'}{#1} \subseteq \trm{\Gamma}{#1}$ and $(m, \d) \in \ugsubst{\trm{\sigma \Gamma'}{#1}}$}}


\newcommand{\IHass}[2]{\text{$\fv{#2} \subseteq \fcv{\Gamma'} $ and $\Gamma' \subseteq \Gamma$ and $(m, \d) \in \ugsubst{\trm{\sigma \Gamma'}{#1}}$}}
% Environment
\newcommand{\memory}{\Gamma}%\Delta \ | \ \Phi \ | \ \Gamma  \ | \
                            %\Sigma}
\newcommand{\senv}{\Delta}
\newcommand{\lenv}{\Sigma}
\newcommand{\uenv}{\Omega}
\newcommand{\renv}{\Gamma} 
\newcommand{\cenv}{\Phi} 
\newcommand{\sep}{ \ | \ }
\newcommand{\monad}[4]{\mathrel{ M( \overset{ \mathrel{\mathrm{exec}{#4 }}}{#2}  })}
\newcommand{\monadR}[4]{\mathrel{ \overset{\mathrm{diff}(#4)}{\{ {#1} \} \ {#2} \ \{ {#3} \} }}}
\newcommand{\depProd}[4]{ \mathrel{ \Pi {#1} \stackrel{\mathrm{exec} {#4}}{:}{#2} . \ {#3}}}
\newcommand{\depProdr}[4]{ \mathrel{ \Pi {#1} \stackrel{\mathrm{diff} {#4}}{:}{#2} . \ {#3}}}
\newcommand{\uarrow}[3]{ \mathrel{ \stackrel{\mathrm{exec} {#3}}{{#1} \longrightarrow#2}}}
\newcommand{\uforall}[4]{ \mathrel{ \stackrel{\mathrm{exec} {#4}}{\forall {#1} :#2 . \ #3}}}
\newcommand{\uexist}[3]{\mathrel{{\exists {#1} :: {#2} . \ {#3}}}}
\newcommand{\rarrow}[3]{ \mathrel{ \stackrel{\mathrm{diff}(#3)}{{#1} \longrightarrow {#2}}}}
\newcommand{\rarrowt}[3]{ \mathrel{ {#1} \stackrel{\mathrm{} {#3}}{\longrightarrow} {#2}}}
\newcommand{\rforall}[4]{ \mathrel{ \stackrel{\mathrm{diff}(#4)}{\forall {#1}{::}{#2} . \ {#3}}}}
\newcommand{\rexists}[3]{ \mathrel{ {\exists {#1} {::}{#2} . \ {#3}}}}
\newcommand{\rforallt}[4]{ \mathrel{ \forall {#1} \stackrel{\mathrm{} {#4}}{:}{#2} . \ {#3}}}
\newcommand{\arr}[3]{ \mathrel{ \mathsf{Array}_{#1}[{#2}] \ {#3}} }
\newcommand{\arrR}[3]{ \mathrel{ \mathsf{Array}_{#1}[{#2}] \ {#3}} }
\newcommand{\lst}[2]{ \mathrel{ \mathsf{list}[{#1}] \ {#2}} }
\newcommand{\lstR}[3]{ \mathrel{ \mathsf{list}^{#1}[{#2}] \ {#3}} }
\newcommand{\abs}[2]{\mathrel { \lambda {#1} . {#2} } }
\newcommand{\app}[2]{\mathrel{ {#1} \, {#2} }}
\newcommand{\ret}[1] {\mathrel{ \mathsf{return} \, {#1} }}

\newcommand{\letx}[3]{  \mathrel{ \mathsf{let}\   {#1} = {#2} \ \mathsf{in} \ {#3}  } }
\newcommand{\packx}[1]{  \mathrel{ \mathsf{pack} \, {#1}} }
\newcommand{\unpackx}[3]{  \mathrel{ \mathsf{unpack} \,  {#1} \, \mathsf{as} \, {#2} \, \mathsf{in} \ {#3}  } }
\newcommand{\alloc}[2]{ \mathrel{ \mathsf{alloc}\, {#1} \, {#2} } }
\newcommand{\updt}[3]{ \mathrel{ \mathsf{update} \ {#1} \ {#2} \ {#3} }  }
\newcommand{\readx}[2] { \mathrel{ \mathsf{read} \ {#1} \ {#2} }  }
\newcommand{\tTt}[3]{\mathrel{  {#1} \xrightarrow \ {#2} } }
\newcommand{\force}[1]{\mathrel{\mathsf{force} \ \ {#1}}}
\newcommand{\tfix}{\mathsf{Fix}}
\newcommand{\fix}[1]{\mathsf{fix} \, f(x). {#1}}

%Relational
\newcommand{\monadx}[3]{\mathrel{ \{ {#1} \} \ {#2} \ \{ {#3} \} }}
\newcommand{\monadu}[4]{\mathrel{ \overset{ \mathrel{\mathrm{exec}{#4 }}}{\{ {#1} \} \ #2 \ \{ {#3} \}} }}
\newcommand{\cmp}[4] {\mathrel{   \vdash  {#1} \ominus {#2} \ldiff {#4}  : {#3}  }}
\newcommand{\pair}[1]{\mathrel{ {#1}_{1}{#1}_{2}}}
\newcommand{\imp}[2]{\mathrel{  {#1} \Rightarrow {#2} }}
\newcommand{\eval}[3]{\mathrel{ {#1} \Downarrow^{#3} {#2}   }}
\newcommand{\evalf}[3]{\mathrel{ {#1} \Downarrow^{#3}_{f} {#2}   }}
\newcommand{\evalp}[3]{\mathrel{ {#1} \Downarrow^{#3}_{p} {#2}   }}
\newcommand{\heap}[1]{ ;  {#1}}
\newcommand {\spc} { \  \ }
\newcommand{\monadL}[3]{\mathrel{ \{ {#1} \}  \\  \ {#2} \ \\  \{ {#3} \} }}

\newcommand{\wfa}[1]{\mathrel{\vdash {#1} \quad wf}}
\newcommand{\wf}[1]{\mathrel{\vdash {#1} \quad wf}}
\newcommand{\subtypeA}[2]{\mathrel{ \models {#1} \sqsubseteq {#2} } }
\newcommand{\subtype}[2]{\mathrel{   \models {#1} \sqsubseteq {#2} }   }
\newcommand{\subcost}[3]{\mathrel{   \models {#1} {#3} {#2} }   }

\newcommand{\emptyhp}{\mathsf{empty}}
\newcommand{\llb}[1]{ \llbracket {#1} \rrbracket }
\newcommand{\llu}[2]{ \llb{#1}_{#2}}
\newcommand{\llp}[2]{ \llparenthesis {#1} \rrparenthesis_{#2} }
\newcommand{\llbe}[1]{ \llbracket {#1} \rrbracket^{E} }
\newcommand{\llpe}[2]{ \llparenthesis {#1} \rrparenthesis_{#2}^{E} }

\newcommand{\mg}[1]{\textcolor[rgb]{.90,0.00,0.00}{[MG: #1]}}
\newcommand{\dg}[1]{\textcolor[rgb]{0.00,0.5,0.5}{[DG: #1]}}
\newcommand{\wq}[1]{\textcolor[rgb]{.50,0.0,0.7}{[WQ: #1]}}

% Helpful shortcuts

\newcommand{\freshSize}[1]{#1\in\text{fresh}(\ssize)}
\newcommand{\freshCost}[1]{#1\in\text{fresh}(\scost)}
\newcommand{\freshVar}[1]{#1 \in \text{fresh}(S)}

\newcommand{\m}{M} 
%Bi-directional Typing Judgement
\newcommand{\chdiff}[5]{\vdash{#1}\rdiff{#2}~{\downarrow}~#3,#4 \Rightarrow
{\color{red}#5}}

\newcommand{\chsdiff}[3]{\vdash{#1}\backsim{#2}~{\downarrow}~#3}


\newcommand{\chdiffNC}[5]{\vdash^{\color{blue}NC}{#1}\rdiff{#2}~{\downarrow}~#3,#4 \Rightarrow
{{\color{red}#5}}}

\newcommand{\infdiff}[6]{\vdash{#1}\rdiff{#2}~{\uparrow}~{\color{red}{#3}}\Rightarrow[{\color{red}#4}],{\color{red}#5},{\color{red}#6}}

\newcommand{\infsdiff}[3]{\vdash{#1}\backsim{#2}~{\uparrow}~{\color{red}{#3}}}

\newcommand{\infdiffsimple}[5]{\vdash{#1}\rdiff{#2}~{\uparrow}~{\color{red}{#3}}\Rightarrow{\color{red}#4},{\color{red}#5}}

\newcommand{\chmax}[4]{\vdash^{\wmax}{#1}~{\downarrow}~#2, #3  \Rightarrow
{{\color{red}#4}}}
\newcommand{\chmin}[4]{\vdash^{\wmin}{#1}~{\downarrow}~#2, #3  \Rightarrow
{{\color{red}#4}}}

\newcommand{\chexec}[5]{\vdash{#1}~{\downarrow}~#2, #3, #4  \Rightarrow
{{\color{red}#5}}}

\newcommand{\infmax}[5]{\vdash^{\wmax}{#1}~{\uparrow}~{\color{red}{#2}}\Rightarrow[{\color{red}#3}], {\color{red}#4},{\color{red}#5}}
\newcommand{\infmin}[5]{\vdash^{\wmin}{#1}~{\uparrow}~{\color{red}{#2}}\Rightarrow[{\color{red}#3}], {\color{red}#4},{\color{red}#5}}

\newcommand{\infexec}[6]{\vdash{#1}~{\uparrow}~{\color{red}{#2}}\Rightarrow[{\color{red}#3}], {\color{red}#4},{\color{red}#5},{\color{red}#6}}

\newcommand{\infexecsimple}[5]{\vdash{#1}~{\uparrow}~{\color{red}{#2}}\Rightarrow{\color{red}#3}, {\color{red}#4},{\color{red}#5}}

\newcommand{\emptypsi}{.}

%Existential elimination
\newcommand{\elimExt}[3]{#1 \vdash \kw{elimExt}(#2)~\downarrow #3}
\newcommand{\solveVar}[6]{#1 \vdash \kw{solve}(#2;#3) \downarrow (#4;#5;#6)}



%Shortcuts
\newcommand{\al}{\alpha}
\newcommand{\algwf}[1]{\vdash  #1~\kw{wf}}
\newcommand{\algwfa}[1]{\vdash^{A} #1~\kw{wf}}
\newcommand{\jalgeqtype}[3]{\sat#1\equiv#2\Rightarrow {\color{red}#3}}
\newcommand{\jalgasubtype}[3]{\sat^{\mathsf{\grt}}#1\sqsubseteq#2\Rightarrow {\color{red}#3}}
\newcommand{\jalgsubtype}[3]{\sat#1\sqsubseteq#2\Rightarrow {\color{red}#3}}
\newcommand{\jalgssubtype}[2]{\sat#1\leq#2}


\newcommand{\fvars}[1]{\text{FV}(#1)}
\newcommand{\fivars}[1]{\text{FIV}(#1)}
\newcommand{\filtercost}[1]{\text{filterCost}(#1)}
\newcommand{\uctx}{\Delta; \psi_a; \Phi_a; \Omega}
\newcommand{\bctx}{\Delta; \psi_a; \Phi_a; \Gamma}


\newcommand{\suba}[1]{{#1}[\theta_a]}
\newcommand{\subaex}[2]{{#1}[\theta_a, #2]}
\newcommand{\subt}[1]{{#1}[\theta]}
\newcommand{\subta}[1]{{#1}[\theta\,\theta_a]}
\newcommand{\subsat}[3]{#1~ \rhd~ #2 : #3}

\newcommand{\erty}[1]{|#1|}
\newcommand{\eanno}[4]{(#1:#2,#3,#4)}
\newcommand{\eannobi}[3]{(#1:#2,#3)}
\newcommand{\e}{\overline{e}}
\newcommand{\trans}{\rightsquigarrow}
\newcommand{\tboxp}[1]{\square(#1)}
\newcommand{\tlr}[1]{\tlift{\trm{#1}{i}}}
\newcommand*\bang{!}



%%% Local Variables:
%%% mode: latex
%%% TeX-master: "main"
%%% End:

\newcommand{\nform}{\mathsf{F}}
\newcommand{\mechanism}{\mathsf{M}}
\newcommand{\depth}{\mathsf{depth}}
\newcommand{\query}{\text{Q}}

% \newcommand{\caseof}[2]{\mathsf{case} \ {#1} \ \mathsf{of}\ \{ {#2}\}}
\newtheorem{lemma}{Lemma}
\newtheorem{theorem}{Theorem}[section]
\newtheorem{corollary}{Corollary}[theorem]
\newcommand{\THESYSTEM}{\textsf{AdaptFun}}

\begin{document}

\title{Program Adaptivity Analysis}

\author{}

\date{}


\maketitle

\tableofcontents
\newpage
\section{Low Level Control Flow Based Language}
We first consider a low level language where the queries are atomic
and the dependency relations are caused only by control flow.
%
\subsection{Syntax and Semantics}
%
\paragraph{Syntax.}
\[
\begin{array}{llll}
 \mbox{Arithmatic Operators} & *_a & ::= & + ~|~ - ~|~ \times 
%
~|~ \div \\  
  \mbox{Boolean Operators} & *_b & & \\
   \mbox{Relational Operators} & *_r & ::= & < ~|~ \leq ~|~ = \\  
 \mbox{Label} & l & & \\ 
 \mbox{While Map} & w & \in & \mbox{Label} \times \mathbb{N} \\
\mbox{AExpr} & \aexpr & ::= & 
  %
  n ~|~ x ~|~ \aexpr *_a \aexpr ~|~ [] ~|~ [\aexpr_0, \dots, \aexpr_i] \\
    %
\mbox{BExpr} & \bexpr & ::= & 
  %
  \etrue ~|~ \efalse  ~|~ \neg \bexpr
   ~|~ \bexpr *_b \bexpr
  %
  ~|~ \aexpr *_r \aexpr \\
\mbox{Expr} & \expr & ::= & \aexpr \sep \bexpr \\ 
\mbox{Command} & c & ::= &   [\assign x \expr]^{l} ~|~  [\assign x q]^{l}
%
~|~ \jl{[\eswitch( \expr, x, (v_i \to  q_i))]^{l} \sep \eloop ~ [\expr_N]^{l} ~ (f) ~ \edo ~ c }  \\ 
  %
& & & ~|~  c ; c ~|~ \eif([\bexpr]^{l}, c_1, c_2) 
  ~|~ \ewhile([\bexpr]^{l}, c) ~|~ [\eskip]^{l} 
  \\
  %
% \mbox{Binary Operation} & \bop & ::= & + ~|~ - ~|~ \times 
% %
% ~|~ \div ~|~ < ~|~ \leq ~|~ = \\
% %
% \mbox{Unary Operation} & \uop & ::= & \ln ~|~ - \\
% %
\mbox{Query} & q& ::= & q_{0} , q_{1} , \ldots , q_{n}\\  
\mbox{Memory} & m & ::= & [] ~|~ m[x^{l} \to v] \\
%
\mbox{Trace} & t & ::= & [] ~|~ [(q ^{(l, w) }, v)] ~|~ t ++ t
\end{array}
\]
%
\paragraph{Operational Semantics.}

% \[\]
% \newline
\begin{mathpar}
\boxed{ \config{m, \aexpr} \aarrow \aexpr' \, : \, Memory \times AExpr \Rightarrow AExpr }
\\
\inferrule{
}{
 \config{m, x} \aarrow m(x)
}
%
\and
%
\inferrule{
  \config{m, \aexpr_1 } \aarrow \aexpr_1'
}{
 \config{m, \aexpr_1 *_a \aexpr_2 } \aarrow \expr_1' *_a \aexpr_2
}
%
\and
%
\inferrule{
  \config{m, \aexpr_2 } \aarrow \aexpr_2'
}{
 \config{m, n_1 *_a \aexpr_2 } \aarrow n_1 *_a \aexpr_2'
}
%
\and
%
\inferrule{
n_3 = n_1 *_a n_2
}{
 \config{m, n_1 *_a n_2 } \aarrow n_3
}
\end{mathpar}
% \vfill \pagebreak
% \newline
%
\begin{mathpar}
\boxed{ \config{m, \bexpr} \barrow \bexpr' \, : \, Memory \times BExpr \Rightarrow BExpr }
\\
\inferrule{
}{
 \config{m, \etrue} \barrow \etrue
}
%
\and
%
\inferrule{
}{
 \config{m, \efalse} \barrow \efalse
}
%
\and
%
\inferrule{
  \config{m, \aexpr_1 } \aarrow \aexpr_1'
}{
 \config{m, \aexpr_1 *_r \aexpr_2 } \barrow \expr_1' *_r \aexpr_2
}
%
\and
%
\inferrule{
  \config{m, \aexpr_2 } \aarrow \aexpr_2'
}{
 \config{m, n_1 *_r \aexpr_2 } \barrow n_1 *_r \aexpr_2'
}
%
\and
%
\inferrule{
b_3 = n_1 *_r n_2
}{
 \config{m, n_1 *_r n_2 } \barrow b_3
}
%
\and
%
\inferrule{
 \config{m, \bexpr_1  } \barrow \bexpr_1' 
}{
 \config{m, \bexpr_1 *_b \bexpr_2 } \barrow \bexpr_1' *_b \bexpr_2
}
%
\and
%
\inferrule{
 \config{m, \bexpr_2  } \barrow \bexpr_2' 
}{
 \config{m, \etrue *_b \bexpr_2 } \barrow \etrue *_b \bexpr_2'
}
%
\and
%
\inferrule{
 \config{m, \bexpr_2  } \barrow \bexpr_2' 
}{
 \config{m, \efalse *_b \bexpr_2 } \barrow \efalse *_b \bexpr_2'
}
%
\and
%
\inferrule{
 \config{m, \bexpr  } \barrow \bexpr' 
}{
 \config{m, \neg \bexpr } \barrow \neg \bexpr'
}
\end{mathpar}
%
\begin{mathpar}
\boxed{ \config{m, c, t} \xrightarrow{} \config{m', c',  t'} \; }
\and
{  Memory \times Com  \times Trace \times WhileMap \Rightarrow^{} Memory \times Com  \times Trace \times WhileMap}
\and
\inferrule
{
q(D) = v 
}
{
\config{m, [\assign{x}{q}]^l, t, w} \xrightarrow{} \config{m[ v/ x],
  \eskip,  t \mathrel{++} [(q^{(l,w )},v) ] , w }
}
~\textbf{query}
%
%
% \inferrule
% {
% q(D) = v \and v \neq v'
% }
% {
% \config{m, \assign x q^*, D} \Rightarrow^{[(q, v')]} \config{m[x \to v'], \eskip, D}
% }
% ~\textbf{query}^*
%
\and
%
\inferrule
{
m, \expr \Rightarrow \expr'
}
{
\config{m, [\assign{x}{ \expr}]^{l}, t,w} \xrightarrow{} \config{m, [\assign{x}{ \expr'}]^{l} , t,w}
}
~\textbf{assn1}
%
~~~
%
\inferrule
{
}
{
\config{m, [\assign x v]^{l},  t,w} \xrightarrow{} \config{m[v/x], [\eskip]^{l}, t,w}
}
~\textbf{assn2}
%
\and
%
\inferrule
{
\config{m, c_1,  t,w} \xrightarrow{} \config{m', c_1',  t',w}
}
{
\config{m, c_1; c_2,  t,w} \xrightarrow{} \config{m', c_1'; c_2, t',w}
}
~\textbf{seq1}
%
\and
%
\inferrule
{
}
{
\config{m, [\eskip]^{l} ; c_2,  t,w} \xrightarrow{} \config{m, c_2,  t,w}
}
~\textbf{seq2}
%
\and
%
\inferrule
{
\config{ m, \bexpr} \barrow \bexpr'
}
{
\config{m, \eif([\bexpr]^l, c_1, c_2),  t,w} 
\xrightarrow{} \config{m,  \eif([\bexpr']^l, c_1, c_2),  t,w}
}
~\textbf{if}
%
\and
%
\inferrule
{
}
{
\config{m, \eif([\etrue]^l, c_1, c_2),t,w} 
\xrightarrow{} \config{m, c_1,  t,w}
}
~\textbf{if-t}
%
~~~~~~~~~~
%
\inferrule
{
}
{
\config{m, \eif([\efalse]^l, c_1, c_2),  t,w} 
\xrightarrow{} \config{m, c_2,  t,w}
}
~\textbf{if-f}
% %
% \and
% %
% {\inferrule
% {
% }
% {
% \config{m, \ewhile([\bexpr]^l, c),  t,w} 
% \xrightarrow{} \config{m,  \eunfold{[\bexpr^{l}] }{\ewhile([\bexpr]^l,   c)} ,  t,w}
% }
% ~\textbf{while} }
% %
% \and
% %
% {\inferrule
% {
% \config{m, \bexpr} \rightarrow \bexpr'
% }
% {
% \config{m, \eunfold{[\bexpr]^l}{ c}, D, t,w} 
% \xrightarrow{} \config{m, \eunfold{[\bexpr']^l}{ c}, D, t,  w  }
% }
% ~\textbf{unfold}}
% %
% \and
% %
% {\inferrule
% {
% }
% {
% \config{m, \eunfold{[\efalse]^l}{c}, D, t,w} 
% \xrightarrow{} \config{m, [\eskip]^{l}, D, t,  (w \setminus l) }
% }
% ~\textbf{unfold-f}}
%
% \and
% %
% {\inferrule
% {
% }
% {
% \config{m, \eunfold{[\etrue]^l}{ c}, D, t,w} 
% \xrightarrow{} \config{m, c, D, t, (w+l) }
% }
% ~\textbf{unfold-t} }
%
\and
%
{
\inferrule
{
  \config{m, \expr } \xrightarrow{} \expr'
}
{
\config{m, [\eswitch(\expr, x, (v_i \to q_i))]^{l},  t,w} 
\xrightarrow{} \config{m, [ \eswitch(\expr',x, (v_i \to q_i))]^{l},  t, w }
}
~\textbf{switch}
}
%
\and
%
{
\inferrule
{
\empty
}
{
\config{m, [ \eswitch(v_k,x, (v_i \to q_i))]^{l},  t,w} 
\xrightarrow{} \config{m,  [\assign x q_k]^{l},  t, w }
}
~\textbf{switch-v}
}
%
\and
%
{\inferrule
{
\config{m, \expr_N \xrightarrow{} \expr_N'  }
}
{
\config{m,  \eloop ~ [\expr_N]^{l} ~ (f) ~ \edo ~ c ,  t, w }
\xrightarrow{} \config{m, [ \eloop ~ [\expr_N]^{l} ~ (f) ~ \edo ~ c]^{l} ,  t, w }
}
~\textbf{loop}
}
%
\and
%
{\inferrule
{
 \valr_N > 0
}
{
\config{m, \eloop ~ [\valr_N]^{l} ~ (f) ~ \edo ~ c ,  t, w }
\xrightarrow{} \config{m, c;  \eloop ~ [(\valr_N-1)]^{l} ~ (f) ~ \edo ~ c ,  t, (w + l) }
}
~\textbf{loop-t}
}
%
\and
%
{
\inferrule
{
 \valr_N = 0
}
{
\config{m,  \eloop ~ [\valr_N]^{l} ~ (f) ~ \edo ~ c ,  t, w }
\xrightarrow{} \config{m, [\eskip]^{l} ,  t, (w \setminus l) }
}
~\textbf{loop-f}
}
%
\end{mathpar}
%
where $w_l$ refers to a map $w$ without the key $l$.
\[
\begin{array}{ccc}
w \setminus l     & = w  & l \not\in Keys(w)   \\
     & = w_l & Otherwise \\
w + l & = w[l \to 0] & l \not \in Keys(w) \\   
     & w_l [l \to w(l)+1] & Otherwise
\end{array}
\]
% \vfill \pagebreak
%
\subsection{Adaptivity of Programs in Low level language}
%
%
\begin{defn}[Label Order]
$<_w and =_w$.\\
\[
  \begin{array}{lll}
     w_1 =_w w_2  &  \triangleq &  Keys(w_1) = Keys(w_2) \land \forall k \in Keys(w_1). w_1(k) = w_2(k) \\
     \emptyset =_w \emptyset & &   \\
  \end{array}
\] 
$mk(w_i) =MinKey(w_i) $ 
\[
\begin{array}{lllr}
     w_1 <_w w_2 & \triangleq & & w_1 = \emptyset \\
     & \triangleq  & mk(w_1) < mk(w_2) & w_1,w_2 != \emptyset  \\
     & \triangleq & w_1(mk(w_1)) < w_2(mk(w_2))   & mk(w_1) = mk(w_2) \\
     & \triangleq & (w_1 \setminus mk(w_1) ) <_w (w_2 \setminus mk(w_2)) & Otherwise
\end{array}
\]
\end{defn}
%
\begin{defn}[Query Direction]
Direction between two queries.
\\
$\forall q_1,q_2, l_1, l_2, w_1, w_2 $.
$(q_1^{l_1, w_1}) \, \mathsf{TO} \, (q_2^{l_2,w_2})$,
denoted as $\mathsf{To}(q_1^{l_1, w_1}, q_2^{l_2,w_2})$ \\ iff $(q_1^{l_1, w_1}) <_q (q_2^{l_2, w_2})  $\\
where \\
$(q_1^{l_1, w_1}) <_q (q_2^{l_2, w_2})$ is defined:\\
\[
\begin{array}{ll}
    l_1 < l_2  & w_1=\emptyset \lor w_2 = \emptyset \lor w_1 =_w w_2   \\
    w_1 <_w w_2  & \mathsf{Otherwise}
\end{array}  
\]
\end{defn}
%
%
\paragraph{Independence between two queries in Low level language}
%
When two queries $q_1,q_2$ are independent in a program $c$, suppose $q_1$ appears before $q_2$ in the program $c$, we think the choice of queries starting from $q_1$, ending with query $q_2$ should be fixed no matter the change of the result of $q_1$.\\
%
\begin{defn}[Query Independence]
Two queries $q_i$ and $q_j$ in a program $c$ are independent,
$\mathsf{IND}(q^{(l_1, w_1)}_i, q^{(l_2, w_2)}_j, c)=$.
$ \forall m, D. \config{m, c,  []} \rightarrow^{*} \config{m_1, c_1,
  t_1} \rightarrow \config{m_2, c_2,
  t_1++[(q^{(l_1, w_1)}_i, v_i)]} \rightarrow^{*} \config{m_3, \eskip,
  t_3} \land x  = FindVar(q^{(l_1,
  w_1)}_i, c) \land $\\
$w_1 = \emptyset \implies  \Big (
$\\
$ 
\left((q^{(l_1, w_1)}_i, v_i) \in t_3 \land (q^{(l_2, w_2)}_j, v_j) \in t_3  \implies \forall v \in codomain(q^{l_1}_i). 
\left( \config{m, c[v/q_i],  []} \rightarrow^* \config{m', \eskip,  t'} \land (q^{l_2}_j, v_j) \in t'
\right)
\right)$ \\
$ \land
\left( (q^{l_1}_i, v_i) \in t_3 \land (q^{l_2}_j, v_j) \notin t_3  \implies \forall v\in codomain(q^{l_1}_i). 
\left( \config{m, c[v/q_i],  []} \rightarrow^{*} \config{m', \eskip,  t'} \land (q^{l_2}_j, v_j) \notin t'
\right)
\right)
\Big ) $ \\
$\begin{array}{l}
\land {w_1} \not = \emptyset \implies  \Big (  \\
 ( (q^{(l_1, w_1)}_i, v_i) \in t_3 \land (q^{(l_2, w_2)}_j, v_j)
  \in t_3  \implies \\
\forall v \in codomain(q^{l_1}_i). 
\left( \config{m_2[v/x], c_2, t_1++[(q^{(l_1, w_1)}_i, v_i)] } \rightarrow^{*} \config{m', \eskip,  t'}   \land (q^{l_2}_j, v_j) \in t'
\right )
 ) \\
 \land
( (q^{l_1}_i, v_i) \in t \land (q^{l_2}_j, v_j) \notin t
  \implies  \\
 \forall v\in codomain(q^{l_1}_i). 
\left( \config{m_2[v/x], c_2, t_1++[(q^{(l_1, w_1)}_i, v_i)] } \rightarrow^{*} \config{m', \eskip,  t'} \land (q^{l_2}_j, v_j) \notin t'
 )
\right )
\Big ).
\end{array}
$
\end{defn}
%
\noindent
In following examples:
\[
c_1 \triangleq
\begin{array}{l}
     \left[x \leftarrow q_1 \right]^1; \\
    \eif \; \left[x \right]^2\; \\
    \mathsf{then} \; \left[y \leftarrow q_2 \right]^3\; \\
    \mathsf{else} \; \left[y \leftarrow 0 \right]^4; \\
    \eif \; \left[x \right]^5\; \\
    \mathsf{then} \; \left[ y \leftarrow 0 \right]^6\; \\
    \mathsf{else} \; \left[y \leftarrow q_2 \right]^7; \\
\end{array}
%
\hspace{20pt}
%
c_2 \triangleq
\begin{array}{l}
     \left[x \leftarrow q_1\right]^1 ; \\
    \eif \; \left[x \right]^2\; \\
    \mathsf{then} \; \left[y \leftarrow q_2\right]^3 \; \\
    \mathsf{else} \; \left[y \leftarrow q_3\right]^4 ; \\
    \left[z \leftarrow q_2 \right]^5; \\
\end{array}
%
\hspace{20pt}
%
c_3 \triangleq
\begin{array}{l}
     \left[x \leftarrow q_1 \right]^1; \\
    \eif \; \left[x \right]^2\; \\
    \mathsf{then} \; \left[y \leftarrow q_3 \right]^3\; \\
    \mathsf{else} \; \left[y \leftarrow q_4 \right]^4; \\
    \left[z \leftarrow q_2 \right]^5; \\
\end{array}
%
\hspace{20pt}
%
c_4 \triangleq
\begin{array}{l}
   \left[ x \leftarrow q_1 \right]^1; \\
   \left[y \leftarrow q_2\right]^2 ; \\
    \eif \;\left[ x + y = 5 \right]^3\; \\
    \mathsf{then} \;\left[ z \leftarrow q_3\right]^4 \; \\
    \mathsf{else} \;\left[ \eskip\right]^5 ; \\
   \left[ w \leftarrow q_4 \right]^6; \\
\end{array}
\]
We have the dependency as:
\\
In program $c_1$:
$\mathsf{IND}(q_2^3, q_2^7, c_1)$, 
\\
In program $c_2$:
$\mathsf{IND}(q_1^1, q_2^5, c_2)$,
$\mathsf{IND}(q_2^3, q_3^4, c_2)$,
\\
In program $c_3$:
$\mathsf{IND}(q_1^1, q_2^5, c_3)$,
$\mathsf{IND}(q_3^3, q_4^4, c_3)$,
\\
In program $c_4$:
$\mathsf{IND}(q_1^1, q_2^2, c_4)$,
$\mathsf{IND}(q_3^3, q_4^4, c_4)$,
$\mathsf{IND}(q_1^1, q_4^4, c_4)$,
$\mathsf{IND}(q_2^2, q_4^4, c_4)$,\\
In following examples containing $\ewhile$ loop:
\[
c_1 \triangleq
\begin{array}{l}
     \left[x \leftarrow q_1 \right]^1 ; \\
    \ewhile \Big( 
    \left[x \leq 100 \right]^2 , \\
    \left[x \leftarrow q_2 + x \right]^3
    \Big) ;
\end{array}
%
\hspace{20pt}
%
c_2 \triangleq
\begin{array}{l}
     \left[x \leftarrow q_1 \right]^1 ; \\
    \ewhile \Big( 
    \left[x \leq 100 \right]^2 , \\
    \left[y \leftarrow q_2 \right]^3; \\
    \left[z \leftarrow q_3 \right]^4; \\
    \left[x \leftarrow y + z + x \right]^5
    \Big) ;
\end{array}
%
\hspace{20pt}
%
c_3 \triangleq
\begin{array}{l}
     \left[x \leftarrow q_1 \right]^1 ; \\
    \ewhile \Big( 
    \left[x \leq 100 \right]^2 , \\
    \left[y \leftarrow q_2 + x \right]^3; \\
    \ewhile \bigg( 
     \left[y \leq 50 \right]^4 , \\
    \left[z \leftarrow q_3 \right]^5; \\
    \left[y \leftarrow y + z \right]^6;
    \bigg)\\
    \left[x \leftarrow y + x \right]^7
    \Big) ;
\end{array}
%
\]
we have the dependency as:
\\
In program $c_1$: No independent queries.
\\
In program $c_2$: $\mathsf{IND}(q_2^{(3, l)}, q_3^{(4, l)}, c_2)$ for all $l$.
\\
In program $c_3$: No independent queries.
%
%
\paragraph{Dependency between multiple queries }
%
\begin{defn}[Dependency Graph]
A dependency graph over a program $P$ is defined as $G(P) = (V, E, \vend )$, where $V$ is set of verticals and $E$ is the set of directed edges:
\\
%
    $V = \{q_1, q_2, \cdots, q_n \}$, where $q_1, q_2, \cdots, q_n$ are reachable queries in the program $P$.
    %
\\
%
E = $\{  (q_{i}, q_{j}) | \exists m. \mathsf{Dep}(q_i, q_{j}, P, m)  \} $ , in the program $P$.
%
\\
%
$\vend = \{q_{e_1}, \cdots, q_{e_l}\}$ be the set of queries related to the return value.
\end{defn}
%
\begin{defn}[Reachable Query].
A query $q$ in program $P$ is reachable iff $q$ may be executed.
\end{defn}
%
%
%
\begin{defn}[Adaptivity]
Given a program $P$ and its dependency graph $G(P) = (V, E, \vend)$, the adaptivity of the program is defined as $A(P)$, s.t.:
for every $q_i \in \vend$, let $p_i$ be the longest path starting from $q_i$ with length $l_i$,
%
$$A(P) = \max\limits_{q_i \in \vend}\{l_i ~|~ l_i = |p_i| \}$$
\end{defn}
%
%
%
\begin{example}
\textbf{Dependency graphs for programs containing 3 atomic queries}
\\
Let $q_1 \triangleq \lambda D. D_1 * D_j$, 
$q_2  \triangleq \lambda D. D_3 * D_4  $ and $q_3 \triangleq \lambda D. D_3 * D_2 $. 
in program $P_1$ and $P_2$ as follows:\\
%
\[
P_1 \triangleq
\begin{array}{l}
      \left[\assign{w}{100} \right]^0 ; \\
      \left[\assign{x}{q_1} \right]^1 ; \\
   \eif( \left[w > 1 \right]^2)\\
   \ethen \left[ \assign{y}{q_2} \right]^3 ; \\
     \eelse \left[\assign{z}{q_3} \right]^4
\end{array}
\hspace{2cm}
P_2 \triangleq
\begin{array}{l}
      \left[\assign{x}{q_1} \right]^1 ; \\
   \eif( \left[x > 1 \right]^2)\\
   \ethen \left[ \assign{y}{q_2} \right]^3 ; \\
     \eelse \left[\assign{z}{10} \right]^4
\end{array}
\]
% \\
\begin{center}
%
\begin{tikzpicture}
\filldraw[black] (10, 2) circle (2pt) node[anchor=south]{$q_2^3$};
\filldraw[black] (12, 2) circle (2pt) node[anchor=south]{$q_1^1$};
\filldraw[black] (12, 0) circle (2pt) node[anchor=north]{$q_3^4$} ;
\end{tikzpicture}
%
\hspace{2cm}
\begin{tikzpicture}
\filldraw[black] (27, 0) circle (2pt);
\filldraw[black] (30, 2) circle (2pt);
\draw[very thick,->] (27, 0)node[anchor=north]{$q_2^3$}  -- (30, 2) node[anchor=south]{$q_1^1$};
\draw[very thick, red, ->, dashed] (27, 0.2) -- (29.8, 2);
\end{tikzpicture}
\end{center}
%
\end{example}
%
%
%

\section{High level Language}
%
\subsection{Syntax and Semantics}
%
\paragraph{Syntax.}
\[
\begin{array}{llll}
 \mbox{Arithmatic Operators} & *_a & ::= & + ~|~ - ~|~ \times 
%
~|~ \div \\  
  \mbox{Boolean Operators} & *_b & & \\
   \mbox{Relational Operators} & *_r & ::= & < ~|~ \leq ~|~ = \\  
 \mbox{Label} & l & & \\ 
 \mbox{While Map} & w & \in & \mbox{Label} \times \mathbb{N} \\
\mbox{AExpr} & \aexpr & ::= & 
  %
  n ~|~ x ~|~ \aexpr *_a \aexpr ~|~ \jl{[] ~|~ [\aexpr_0, \dots, \aexpr_i] ~|~ \uniform ~|~ \bernoulli ~|~ \aexpr \times \aexpr } \\
    %
\mbox{BExpr} & \bexpr & ::= & 
  %
  \etrue ~|~ \efalse  ~|~ \neg \bexpr
   ~|~ \bexpr *_b \bexpr
  %
  ~|~ \aexpr *_r \aexpr \\
\mbox{Command} & c & ::= &   [\assign x \expr]^{l} ~|~  [\assign x q(\expr)]^{l}
%
 \\
  %
& & & ~|~  c ; c ~|~ \eif([\bexpr]^{l}, c_1, c_2) 
  ~|~ \ewhile([\bexpr]^{l}, c) ~|~ [\eskip]^{l} \sep \jl{\eloop ~ [\valr_N]^{l} ~ (f) ~ \edo ~ c }
  \\
\mbox{Memory} & m & ::= & [] ~|~ m[x^{l} \to v] \\
%
\mbox{Trace} & t & ::= & [] ~|~ [(q, v)^{(l, w) }] ~|~ t ++ t
\end{array}
\]
%
%
\begin{example}
\textbf{Dependency graphs for high level programs containing  non-atomic queries}
\\
Let $q_1 = \lambda D. D_i * D_j$, \\ 
Let $q_2 (x_1) = \lambda D. D_i * D_j + x_1  $.\\
Let $q_3 (x_1 - x_2) = \lambda D. D_i * D_j + x_1 - x_2 $, 
$q_4 (x_2) = \lambda D. D_i * D_j + x_2 $, 
and $q_5(x_1) = \lambda D. D_i * D_j + x_1$ .\\
in program $P_1$, $P_2$ and $P_3$ as following:
\[
P_1 \triangleq
\begin{array}{c}
      \left[\assign{x_1}{q_1} \right]^1; \\
   \left[\assign{x_2}{q_2} \right]^2 ; \\
     \left[\assign{x_3}{q_3} \right]^3
\end{array}
\hspace{2cm}
P_2 \triangleq
\begin{array}{c}
      \left[\assign{x_1}{q_1} \right]^1; \\
   \left[\assign{x_2}{q_2} \right]^2 ; \\
     \left[\assign{x_3}{q_4} \right]^3
\end{array}
\hspace{2cm}
P_3 \triangleq
\begin{array}{c}
      \left[\assign{x_1}{q_1} \right]^1; \\
   \left[\assign{x_2}{q_2} \right]^2 ; \\
     \left[\assign{x_3}{q_5} \right]^3
\end{array}
\]
%
\begin{center}
\begin{tikzpicture}
\draw[very thick,->] (8, 0)node[anchor=north]{$q_3^3$} -- (6, 2) node[anchor=south]{$q_2^2$};
\draw[very thick,->] (8, 0)  -- (10, 2) node[anchor=south]{$q_1^1$};
\draw[very thick,->] (6.2, 2) -- (9.8, 2);
%%%%%draw the longest path
\draw[rounded corners=8mm, very thick, red, dashed, ->] (8, 0.2) -- (6.4, 1.8) -- (9.6, 1.8);
\end{tikzpicture}
%
\begin{tikzpicture}
\draw[very thick,->] (18, 0)node[anchor=north]{$q_4^3$} -- (16, 2) node[anchor=south]{$q_2^2$};
\draw[very thick,->] (16.2, 2) -- (19.8, 2)node[anchor=south]{$q_1^1$};
\draw[rounded corners=8mm, very thick, red, dashed, ->] (18, 0.2) -- (16.4, 1.8) -- (19.6, 1.8);
\end{tikzpicture}
%
\begin{tikzpicture}
\draw[very thick,->] (26, 2) node[anchor=south]{$q_2^2$} -- (29.9, 2);
\draw[very thick,->] (28, 0)node[anchor=north]{$q_5^3$}  -- (30, 2) node[anchor=south]{$q_1^1$};
% \draw[very thick, red, ->, dashed] (26.4, 1.8) -- (29.6, 1.8);
\draw[very thick, red, ->, dashed] (28, 0.2) -- (29.6, 1.8);
\end{tikzpicture}
\end{center}
%
\end{example}
%
\subsection{Rewriting from High Level Program into Low Level Program}
%
The transformation $\ts{e^h} = e^l$ transfers the expression $e^h$ in the high level language to an expression $e_l$ in the low language. 
Let us look at the special cases: the query.
\\
In the first transition, if a query in high level language isn't atomic, i.e., 
$q(e)$ depends on $e$ with free variables, then it will be rewrite into a switch command. 
This rewriting will switch on the possible values $v_i$ of $e$ and convert the $q(e)$ into a series of atomic queries $q_i$.
\\
In the second transition, if a query in high level language is atomic, $q()$ only depends on data base $D$ and some constant values,
then it will be rewrite into identity in our low level language.
\\
Another special case is the sampling command in high level language. To exclude the dependency caused by the randomness, we will rewrite the sampling into an assignment command in low level language. This will assign a constant value to the corresponding variable.
\\
The resting commands will be rewrote identically.
%
%
\[
\begin{array}{lll}
\ts{ [\assign x q(e)]^{l}}
        & \Rightarrow &
        \left[
        \begin{array}{l}
             \eswitch ~ \Bigg(\expr, x, 
            \left(\begin{array}{l}
           v_1 \to q_1, \\
            \cdots, \\
            v_m \to q_m
            \end{array}\right) 
            \Bigg) \\
        \end{array}
        \right]^{l}\\
    \ts{q()} & \Rightarrow & q\\
  \ts{[\assign{x}{\uniform} ]^{l}}   &   \Rightarrow & \left[
\assign x c_{u}
\right]^{l} \\
 \ts{[\assign{x}{\expr} ]^{l}}   &   \Rightarrow & \left[ \assign x \expr \right]^{l} \\
 \ts{ c_1 ; c_2 }     & \Rightarrow  & \ts{c_1} ; \ts{c_2} \\
  \ts{\eif([\bexpr]^{l}, c_1, c_2)}  &  \Rightarrow & \eif([\bexpr]^{l}, \ts{c_1}, \ts{c_2}) \\
 \ts{\ewhile([\bexpr]^{l}, c)} & \Rightarrow & \ewhile([\bexpr]^{l},  \ts{c} ) \\
\end{array}
\]
%
%
\begin{example}[Two Round Algorithm]
\[
TR^H(k) \triangleq
\ts{
\begin{array}{l}
    \left[j \leftarrow 1 \right]^1 ; \\
    \left[a \leftarrow [] \right]^2; \\
    \ewhile \Big( 
    \left[j \leq k \right]^3 , \\
     \left[x \leftarrow q_j() \right]^4 ; \\
    \left[a \leftarrow x :: a \right]^5;\\
    \left[j \leftarrow j + 1 \right]^6
    \Big);\\
    \clabel{l \leftarrow q_{k + 1}(a)}^7;\\
\end{array}
}
%
~~~~~~~~ \Rightarrow ~~~~~~~
%
TR^L \triangleq
\begin{array}{l}
    \left[j \leftarrow 1 \right]^1 ; \\
    \left[a \leftarrow [] \right]^2; \\
    \ewhile \Big( 
    \left[j \leq k \right]^3, \\
     \left[x \leftarrow q_j \right]^4; \\
    \left[a \leftarrow x :: a \right]^5;\\
    \left[j \leftarrow j + 1 \right]^6
    \Big);\\
    \clabel{
\eswitch \Bigg(a, x, 
    \left(\begin{array}{l}
    \left[-n, -n, -n, \cdots, -n\right] \to q_{k + 1,1},\\
    \cdots\\
    \left[n, n, n, \cdots, n\right] \to q^{}_{k + 1, n^k}
\end{array} \right) \Bigg)
    }^7
\end{array}
\]
$\config{\emptyset, TR^L, D, [],\emptyset} \rightarrow \config{m, \eskip, D, t, w } $.
$$t = [ (q_1^{(4, \{3 \to 1\} )},v_1 ), (q_2^{(4, \{3 \to 2\} )},v_2 ), \ldots, (q_k^{(4, \{3 \to k\} )},v_k ) , (q_{k+1, i }^{(7, \emptyset )},v_1 )  ] $$
%
A($TR^L$) = 1

\begin{center}
%
\begin{tikzpicture}
%%% The nodes represents the k query in the first round
\filldraw[black] (0, 2) circle (2pt) node [anchor=south]{$q_1^{(4, \{3 \to 1\} )}$};
\filldraw[black] (3, 2) circle (2pt) node [anchor=south]{$q_2^{(4, \{3 \to 2\} )}$};
% \filldraw[black] (6, 2) circle (2pt) node [anchor=south]{$q^4_3$};
\filldraw[black] (8, 2) circle (2pt) node [anchor=south]{$\cdots$};
\filldraw[black] (12, 2) circle (2pt) node [anchor=south]{$q_k^{(4, \{3 \to k\} )}$};
%%%%%% The nodes represents the n^k queries in the second round
\filldraw[black] (0, 0) circle (2pt) node [anchor=north]{$q_{k+1,1}^{(7, \emptyset)}$};
\filldraw[black] (3, 0) circle (2pt) node [anchor=north]{$q_{k+1,2}^{(7, \emptyset)}$};
% \filldraw[black] (6, 0) circle (2pt) node [anchor=north]{$q^{3, 7}_{k+1}$};
\filldraw[black] (8, 0) circle (2pt) node [anchor=north]{$\cdots$};
\filldraw[black] (12, 0) circle (2pt) node [anchor=north]{$q_{k+1,n^k}^{(7, \emptyset)}$};
%%%%%% The edges represents their dependency relations GROUP 1
\draw[ thick,->] (0, 0)  -- (0, 1.9) ;
\draw[ thick,->] (0, 0)  -- (2.9, 2) ;
% \draw[very thick,->] (0, 0)  -- (6, 2) ;
\draw[ thick,->] (0, 0)  -- (7.9, 2) ;
\draw[ thick,->] (0, 0)  -- (11.9, 2) ;
%%%%%% The edges represents their dependency relations GROUP 2
\draw[ thick,->] (3, 0)  -- (0.1, 1.8) ;
\draw[ thick,->] (3, 0)  -- (3, 1.9) ;
% \draw[very thick,->] (0, 0)  -- (6, 2) ;
\draw[ thick,->] (3, 0)  -- (7.95, 1.9) ;
\draw[ thick,->] (3, 0)  -- (11.95, 1.9) ;
%%%%%% The edges represents their dependency relations GROUP 3
\draw[ thick,->] (8, 0)  -- (0.1, 1.9) ;
\draw[ thick,->] (8, 0)  -- (3.1, 1.9) ;
% \draw[very thick,->] (0, 0)  -- (6, 2) ;
\draw[ thick,->] (8, 0)  -- (8, 1.9) ;
\draw[ thick,->] (8, 0)  -- (12, 1.85) ;
%%%%%% The edges represents their dependency relations GROUP 4
\draw[ thick,->] (12, 0)  -- (0.1, 2) ;
\draw[ thick,->] (12, 0)  -- (3.1, 2) ;
% \draw[very thick,->] (0, 0)  -- (6, 2) ;
\draw[ thick,->] (12, 0)  -- (8.1, 2) ;
\draw[ thick,->] (12, 0)  -- (12, 1.85) ;
%%%% The longest path representing the adaptivity
\draw[ultra thick, red, ->, dashed] (0.1, 0) -- (0.1, 1.9);
\end{tikzpicture}
\end{center}
\end{example}
%
%
\begin{example}[Multi-Round Algorithm]
\[
MR^H \triangleq
\begin{array}{l}
     \left[j \leftarrow 0 \right]^1 ; \\
    \left[I \leftarrow [] \right]^2; \\
    \ewhile \Big( 
    \left[j \leq k \right]^3 , \\
    \left[p \leftarrow \uniform(0,1) \right]^4 ; \\
    \left[a \leftarrow q_j(p, I) \right]^5; \\
    \left[I \leftarrow \eupdt(I, (a, p))  \right]^6;\\
    \left[j \leftarrow j + 1 \right]^7
    \Big) ;
\end{array}
%
~~~~ \Rightarrow ~~~
%
MR^L \triangleq
\begin{array}{l}
     \left[j \leftarrow 0 \right]^1 ; \\
    \left[I \leftarrow [] \right]^2; \\
    \ewhile \Big( 
    \left[j \leq k \right]^3 , \\
    \left[p \leftarrow 0 \right]^4 ; \\
    \left[
    \eswitch \Big( I, x
    \left(\begin{array}{l}
        [ ] \to q_{j,1},\\
        \cdots\\
    \clabel{1, 2, 3, \cdots, n} \to q_{j,n!}
    \end{array}\right)
    \Big)
    \right]^5\\
    \clabel{I \leftarrow \eupdt(I, (a, p))}^6;\\
    \clabel{j \leftarrow j + 1 }^7
    \Big) ;
\end{array}
\]
%
%
Let $k = 4$, given a specific database $D$, we have $\config{\emptyset, MR^L, D, [],\emptyset} \rightarrow \config{m, \eskip, D, t, w } $ and the trace as:
%
$$t = \left[(q^{5, \{3 \to 1\}}_{1, 1}, v_1), 
(q^{5, \{3 \to 2\}}_{2, 3}, v_2),
(q^{5, \{3 \to 3\}}_{3, 2}, v_3)
(q^{5, \{3 \to 4\}}_{4, 3}, v_4)
\right]$$
\\
A($TR^L$) = 3
\begin{center}
%
\begin{tikzpicture}
%%% The nodes represents the k query in the first round
\filldraw[black] (0, 4) circle (2pt) node [anchor=south]{$(q^{5, \{3 \to 1\}}_{1, 1}$};
% \filldraw[black] (8, 4) circle (2pt) node [anchor=south]{$\cdots$};
% \filldraw[black] (12, 4) circle (2pt) node [anchor=south]{$q^{1, (5, k)}_k$};
%%%%%% The nodes represents the n^k queries in the second round
% \filldraw[black] (0, 0) circle (2pt) node [anchor=north]{$q^{n!, (5, 1)}_1$};
\filldraw[black] (3, 0) circle (2pt) node [anchor=north]{$q^{5, \{3 \to 2\}}_{2, 3}$};
% \filldraw[black] (6, 0) circle (2pt) node [anchor=north]{$q^{3, 7}_{k+1}$};
% \filldraw[black] (8, 0) circle (2pt) node [anchor=north]{$\cdots$};
\filldraw[black] (12, 0) circle (2pt) node [anchor=north]{$q^{5, \{3 \to 4\}}_{4, 3}$};
%%% The nodes represents the k query in the first round
% \filldraw[black] (0, 2) circle (2pt) node [anchor=south]{$q^{\cdots, (5, 1)}_1$};
% \filldraw[black] (3, 2) circle (2pt) node [anchor=south]{$q^{\cdots, (5, 2)}_2$};
% \filldraw[black] (6, 2) circle (2pt) node [anchor=south]{$q^4_3$};
\filldraw[black] (8, 2) circle (2pt) node [anchor=south]{$q^{5, (3 \to 3)}_{3, 2}$};
% \filldraw[black] (12, 2) circle (2pt) node [anchor=south]{$q^{\cdots, (5, k)}_k$};
%%%%%% The edges represents their dependency relations GROUP 1
% \draw[very thick,->] (3, 2)  -- (0.1, 2) ;
% \draw[very thick,->] (3, 0)  -- (0.1, 1.9) ;
% \draw[very thick,->] (3, 4)  -- (0.1, 2.1) ;
% %
% \draw[very thick,->] (3, 2)  -- (0.1, 0.1) ;
% \draw[very thick,->] (3, 0)  -- (0.1, 0) ;
% \draw[very thick,->] (3, 4)  -- (0, 0.2) ;
% %
% \draw[very thick,->] (3, 2)  -- (0.1, 3.9) ;
\draw[very thick,->, red] (3, 0)  -- (0, 3.8) ;
% \draw[very thick,->] (3, 4)  -- (0, 4) ;
% \draw[very thick,->] (0, 0)  -- (6, 2) ;
%%%%%% The edges represents their dependency relations GROUP 2
% \draw[very thick,->] (0, 0)  -- (6, 2) ;
% \draw[very thick,->] (8, 2)  -- (3.1, 2) ;
% \draw[very thick,->] (8, 0)  -- (3.1, 1.9) ;
% \draw[very thick,->] (8, 4)  -- (3.1, 2.1) ;
%
\draw[very thick,->, red] (8, 2)  -- (3.1, 0.1) ;
\draw[very thick,->] (8, 2)  -- (0.15, 4) ;
% \draw[very thick,->] (8, 0)  -- (3.1, 0) ;
% \draw[very thick,->] (8, 4)  -- (3, 0.2) ;
% %
% \draw[very thick,->] (8, 2)  -- (3.1, 3.9) ;
% \draw[very thick,->] (8, 0)  -- (3, 3.8) ;
% \draw[very thick,->] (8, 4)  -- (3.1, 4) ;
%%%%%% The edges represents their dependency relations GROUP 4
% \draw[very thick,->] (0, 0)  -- (6, 2) ;
% \draw[very thick,->] (12, 2)  -- (8.1, 2) ;
\draw[very thick,->, red] (12, 0)  -- (8.1, 1.9) ;
\draw[very thick,->] (12, 0)  -- (3.1, 0) ;
\draw[very thick,->] (12, 0)  -- (0.1, 3.9) ;
% \draw[very thick,->] (12, 4)  -- (8.1, 2.1) ;
% %
% \draw[very thick,->] (12, 2)  -- (8.1, 0.1) ;
% \draw[very thick,->] (12, 0)  -- (8.1, 0) ;
% \draw[very thick,->] (12, 4)  -- (8, 0.2) ;
% %
% \draw[very thick,->] (12, 2)  -- (8.1, 3.9) ;
% \draw[very thick,->] (12, 0)  -- (8, 3.8) ;
% \draw[very thick,->] (12, 4)  -- (8.1, 4) ;
%
% \draw[very thick,->] (12, 2)  -- (8.1, 3.9) ;

%%%% The longest path representing the adaptivity
% \draw[ultra thick, red, ->, dashed] (3, 4.1)  -- (0.1, 4.1);
% \draw[ultra thick, red, ->, dashed] (8, 4.1)  -- (3.1, 4.1);
% \draw[ultra thick, red, ->, dashed] (12, 4.1)  -- (8.1, 4.1);
\end{tikzpicture}
\end{center}
%
%
$\forall k. \forall D$, we have $A(TR^L) = (k - 1)$ given all possible execution traces.
\begin{center}
%
\begin{tikzpicture}
%%% The nodes represents the k query in the first round
\filldraw[black] (0, 4) circle (2pt) node [anchor=south]{$q^{1, [(5, 1)]}_1$};
\filldraw[black] (3, 4) circle (2pt) node [anchor=south]{$q^{1, [(5, 2)]}_2$};
% \filldraw[black] (6, 2) circle (2pt) node [anchor=south]{$q^4_3$};
\filldraw[black] (8, 4) circle (2pt) node [anchor=south]{$\cdots$};
\filldraw[black] (12, 4) circle (2pt) node [anchor=south]{$q^{1, (5, k)}_k$};
%%%%%% The nodes represents the n^k queries in the second round
\filldraw[black] (0, 0) circle (2pt) node [anchor=north]{$q^{n!, (5, 1)}_1$};
\filldraw[black] (3, 0) circle (2pt) node [anchor=north]{$q^{n!, (5, 2)}_2$};
% \filldraw[black] (6, 0) circle (2pt) node [anchor=north]{$q^{3, 7}_{k+1}$};
\filldraw[black] (8, 0) circle (2pt) node [anchor=north]{$\cdots$};
\filldraw[black] (12, 0) circle (2pt) node [anchor=north]{$q^{n!, (5, k)}_k$};
%%% The nodes represents the k query in the first round
\filldraw[black] (0, 2) circle (2pt) node [anchor=south]{$q^{\cdots, (5, 1)}_1$};
\filldraw[black] (3, 2) circle (2pt) node [anchor=south]{$q^{\cdots, (5, 2)}_2$};
% \filldraw[black] (6, 2) circle (2pt) node [anchor=south]{$q^4_3$};
\filldraw[black] (8, 2) circle (2pt) node [anchor=south]{$\cdots$};
\filldraw[black] (12, 2) circle (2pt) node [anchor=south]{$q^{\cdots, (5, k)}_k$};
%%%%%% The edges represents their dependency relations GROUP 1
\draw[very thick,->] (3, 2)  -- (0.1, 2) ;
\draw[very thick,->] (3, 0)  -- (0.1, 1.9) ;
\draw[very thick,->] (3, 4)  -- (0.1, 2.1) ;
%
\draw[very thick,->] (3, 2)  -- (0.1, 0.1) ;
\draw[very thick,->] (3, 0)  -- (0.1, 0) ;
\draw[very thick,->] (3, 4)  -- (0, 0.2) ;
%
\draw[very thick,->] (3, 2)  -- (0.1, 3.9) ;
\draw[very thick,->] (3, 0)  -- (0, 3.8) ;
\draw[very thick,->] (3, 4)  -- (0, 4) ;
% \draw[very thick,->] (0, 0)  -- (6, 2) ;
%%%%%% The edges represents their dependency relations GROUP 2
% \draw[very thick,->] (0, 0)  -- (6, 2) ;
\draw[very thick,->] (8, 2)  -- (3.1, 2) ;
\draw[very thick,->] (8, 0)  -- (3.1, 1.9) ;
\draw[very thick,->] (8, 4)  -- (3.1, 2.1) ;
%
\draw[very thick,->] (8, 2)  -- (3.1, 0.1) ;
\draw[very thick,->] (8, 0)  -- (3.1, 0) ;
\draw[very thick,->] (8, 4)  -- (3, 0.2) ;
%
\draw[very thick,->] (8, 2)  -- (3.1, 3.9) ;
\draw[very thick,->] (8, 0)  -- (3, 3.8) ;
\draw[very thick,->] (8, 4)  -- (3.1, 4) ;
%%%%%% The edges represents their dependency relations GROUP 4
% \draw[very thick,->] (0, 0)  -- (6, 2) ;
\draw[very thick,->] (12, 2)  -- (8.1, 2) ;
\draw[very thick,->] (12, 0)  -- (8.1, 1.9) ;
\draw[very thick,->] (12, 4)  -- (8.1, 2.1) ;
%
\draw[very thick,->] (12, 2)  -- (8.1, 0.1) ;
\draw[very thick,->] (12, 0)  -- (8.1, 0) ;
\draw[very thick,->] (12, 4)  -- (8, 0.2) ;
%
\draw[very thick,->] (12, 2)  -- (8.1, 3.9) ;
\draw[very thick,->] (12, 0)  -- (8, 3.8) ;
\draw[very thick,->] (12, 4)  -- (8.1, 4) ;
%
%%%% The longest path representing the adaptivity
\draw[ultra thick, red, ->, dashed] (3, 4.1)  -- (0.1, 4.1);
\draw[ultra thick, red, ->, dashed] (8, 4.1)  -- (3.1, 4.1);
\draw[ultra thick, red, ->, dashed] (12, 4.1)  -- (8.1, 4.1);
\end{tikzpicture}
\end{center}
\end{example}
%
%
\clearpage
\section{Towards Probability}
%
\subsection{Syntax and Semantics}
%
\paragraph{Syntax.}
\[
\begin{array}{llll}
 \mbox{Arithmatic Operators} & *_a & ::= & + ~|~ - ~|~ \times 
%
~|~ \div \\  
  \mbox{Boolean Operators} & *_b & ::= & \lor \sep \land \\
   \mbox{Relational Operators} & *_r & ::= & < ~|~ \leq ~|~ = \\  
 \mbox{Label} & l & & \\ 
 \mbox{While Map} & w & \in & \mbox{Label} \times \mathbb{N} \triangleq w +l \sep w \setminus l \sep w \oplus_\rho w \\
\mbox{AExpr} & \aexpr & ::= & 
  %
  n ~|~ x ~|~ \aexpr *_a \aexpr ~|~ {[] ~|~ [\aexpr_0, \dots, \aexpr_i] \sep \aexpr \times \aexpr } \\
    %
\mbox{BExpr} & \bexpr & ::= & 
  %
  \etrue ~|~ \efalse  ~|~ \neg \bexpr
   ~|~ \bexpr *_b \bexpr
  %
  ~|~ \aexpr *_r \aexpr \\
\mbox{Deterministic Expr} & \expr_d & ::= & \aexpr \sep \bexpr \\
\mbox{Randomized AExpr} & \aexpr_r & ::= & 
  %
   x_r \sep n \sep  \aexpr_r *_a \aexpr_r \\
    %
\mbox{Randomized BExpr} & \bexpr_r & ::= & 
  %
\neg \bexpr_r
   ~|~ \bexpr_r *_b \bexpr_r
  %
  ~|~ \aexpr_r *_r \aexpr_r \\
\mbox{Randomized Expr} & \expr_r & ::=  & \expr_d \sep \aexpr_r \sep \bexpr_r \\
\mbox{Command} & C & ::= &   [\assign x {\expr_d}]^{l} \sep [\assign {x_r} {\expr_r}]^{l} \sep  [\assign x q]^{l} \sep  [\assign {x_r} \uniform ]^{l} \sep   [\assign {x_r} \bernoulli]^{l} 
 %
\\
  %
& & & ~|~  \command ; \command  ~|~ \eif_D ([\bexpr]^{l}, \command_1, \command_2) \sep  \eif_R ([\bexpr_r]^{l}, \command_1, \command_2) 
  ~|~ \ewhile([\bexpr]^{l}, \command) ~|~ [\eskip]^{l} \\
& & & ~|~ {\eunfold{[\bexpr]^{l}}{\command} \sep [\eswitch( \expr, x, (v_i \to  q_i))]^{l}} \sep \jl{\eloop ~ [\valr_N]^{l} ~ (f) ~ \edo ~ \command }
  \\
  %
% \mbox{Binary Operation} & \bop & ::= & + ~|~ - ~|~ \times 
% %
\mbox{Memory} & m & ::= & [] ~|~ m[x^{l} \to v] \\
%
\mbox{Trace} & t & ::= & [] ~|~ [(q, v)^{(l, w) }] ~|~ t ++ t
\end{array}
\]
%
\[
\begin{array}{ccl}
w \setminus l     & = &\left \{  
    \begin{array}{lr} w  & l \not\in Keys(w)   \\
      w_l & Otherwise \\
     \end{array} \right.\\
w + l & = &
 \left \{  
    \begin{array}{lr}
    w[l \to 0] & l \not \in Keys(w) \\   
    w_l [l \to w(l)+1] & Otherwise
          \end{array} \right.\\
\end{array}
\]
%
\paragraph{Semantics}
We have a countable set $\textsf{RV}$ of random variables($x_r \in \textsf{RV}$), a countable set $\textsf{Val}$ of values. For any subset  $\textsf{S} \subseteq \textsf{RV}$, we denote $\textsf{RanM[S]} \triangleq \textsf{S} \to \textsf{Val} $. We let $\textsf{DV}$ as a countable set of deterministic variables ($x$) and the deterministic memory $\textsf{DetM} \triangleq \textsf{DV} \to \textsf{Val}$. 
Distribution over A as $D(A)$. The \emph{distribution unit} unit : $A \to D(A)$. The \emph{distribution bind} bind : 
$ D(A) \to (A \to D(B)) \to D(B)$. 
\\
$\lrr{C}{} : 
(\textsf{DetM} \times D(\textsf{RandM}) \times \left[\textsf{Trace} \right]  
\times \left[\textsf{WhileMap} \right]  \times \textsf{DB} ) 
\to 
(\textsf{DetM} \times D(\textsf{RandM}) \times \left[\textsf{Trace} \right] 
\times \left[\textsf{WhileMap} \right]  \times \textsf{DB} ) $
.
The upper case character $T$ represents the list of traces and 
$W$ represents the list of while map.
The $@$ operator represents the list concatenation, and the
$T.i$ represents the $i^{th}$ element in the list $T$.
%
\[\config{(\sigma, \mu_1 , T_1, W_1, D)} 
\oplus_{\rho} \config{(\sigma, \mu_2 , T_2, W_2, D)} 
\triangleq 
\config{\sigma, \mu_1\oplus_{\rho} \mu_2 , 
T_1 @ T_2, W_1 @ W_2, D}\]
%
\[T  ++ t \triangleq [T.i ++ t] \]
%
\[ W + l \triangleq [W.i + l] \]
%
\[ W \setminus l \triangleq [W.i \setminus l] \]
%
%
\begin{figure}[H]%
    \centering
    \[
    \begin{array}{rll}
        \lrr{ [\eskip]^{l} }{} (\sigma, \mu , T, W, D) & \triangleq & (\sigma, \mu , T, W , D)  \\
        \lrr{ [\assign x {\expr_d}]^{l} }{} (\sigma, \mu , T, W, D)  & \triangleq & (\sigma[x \to \lrr{\expr_d}{}(\sigma)], \mu , T, W, D) \\
        \lrr{ [\assign {x_r} {\expr_r}]^{l}}{} (\sigma, \mu , T, W, D)  & \triangleq & (\sigma, bind(\mu , m \to unit(m[x_r \to \lrr{\expr_r}{}(\sigma, m)]) ) , T, W , D ) \\
        \lrr{ [\assign x q]^{l} }{} (\sigma, \mu , T, W, D)  & \triangleq & (\sigma[x \to v ], \mu , T ++ [(q, v)]^{(l, W)}, W , D) \qquad : v = q(D)\\
        \lrr{ [\assign {x_r} \uniform ]^{l} }{} (\sigma, \mu , T, W, D)  & \triangleq & (\sigma,  bind(\mu , m \to bind(\uniform , u \to m[x_r \to u] ) )  , T, W, D) \\
        \lrr{ [\assign {x_r} \bernoulli ]^{l} }{} (\sigma, \mu , T, W , D)  & \triangleq & (\sigma , bind(\mu , m \to bind(\bernoulli , u \to m[x_r \to u] ) ) , T, W,D) \\
            \lrr{ \command ; \command' }{} (\sigma, \mu , T, W , D)  & \triangleq & \lrr{\command'}{} ( \lrr{\command}{} \sigma , \mu, T, W) \\
         \lrr{ \eif_D ([\bexpr]^{l}, \command_1, \command_2)  }{} (\sigma, \mu , T, W , D)  & \triangleq & \left \{  \begin{array}{l} 
         \lrr{\command_1}{} (\sigma , \mu, T, W, D) \qquad : \lrr{\bexpr}{}(\sigma) = \etrue \\ 
         \lrr{\command_1}{} (\sigma , \mu, T, W,D) \qquad : \lrr{\bexpr}{}(\sigma) = \efalse \end{array} \right . \\    
           \lrr{ \eif_R ([\bexpr_r]^{l}, \command_1, \command_2)  }{} (\sigma, \mu , T, W , D)  & \triangleq & \left \{
           \begin{array}{l}
           \lrr{\command_1}{} (\sigma , \mu | \lrr{\bexpr_r}{} \sigma = \etrue  , T, W, D) \oplus_\rho  \lrr{\command_2}{} (\rho , \mu | \lrr{\bexpr_r}{} \sigma = \efalse  , T, W,D)  \\ 
           \lrr{\command_1}{} (\sigma , \mu | \lrr{\bexpr_r}{} \sigma = \etrue  , T, W, D) \qquad \rho = 1 \\
           \lrr{\command_2}{} (\sigma , \mu | \lrr{\bexpr_r}{} \sigma = \efalse  , T, W,D) \qquad \rho = 0 \\
            \end{array} \right . \\
             & & \textsf{where} \quad \rho = \mu (\lrr{\bexpr_r}{} \sigma = \etrue ) \\
          \lrr{ \ewhile([\bexpr]^{l}, \command) }{} (\sigma, \mu , T, W , D)  & \triangleq & \lrr{\eunfold{[\bexpr]^{l}}{ \ewhile([\bexpr]^{l}, \command) }   }{} (\sigma, \mu , T, W , D) \\
           \lrr{ \eunfold{[\bexpr]^{l}}{\command}  }{} (\sigma, \mu , T, W , D)  & \triangleq & \left \{  \begin{array}{l} \lrr{\command}{} (\sigma , \mu, T, W + l, D) \qquad : \lrr{\bexpr}{}(\sigma) = \etrue \\ \lrr{\eskip}{} (\sigma , \mu, T, W-l, D) \qquad : \lrr{\bexpr}{}(\sigma) = \efalse \end{array} \right . \\  
           \lrr{ {[\eswitch( \expr, x, (v_i \to  q_i))]^{l}} }{} (\sigma, \mu , T, W , D)  & \triangleq & 
          \lrr{ [\assign x q_1]^{l} }{} ( \sigma, \mu , T, W, D ) 
          \qquad : v_1 = \lrr{\expr}{}{(\sigma)} \\ 
           \lrr{ \jl{\eloop ~ [\expr_N]^{l} ~ (f) ~ \edo ~ \command } }{} (\sigma, \mu , T, W , D)  & \triangleq & \left \{  \begin{array}{l} \lrr{f;\command; \eloop ~ [\expr_N-1]^{l} ~ (f) ~ \edo ~ \command }{} (\sigma , \mu, T, W + l, D) \qquad : \lrr{\expr_N}{}(\sigma) >0 \\ \lrr{\eskip}{} (\sigma , \mu, T, W-l, D) \qquad : \lrr{\expr_N}{}(\sigma) = 0 \end{array} \right . \\  
    \end{array}
    \]
    \caption{Semantics of programs}
    \label{fig:my_label}
\end{figure}
%
%
\[
\begin{array}{l}
     \left[j \leftarrow 0 \right]^1 ; \\
    \left[I \leftarrow [] \right]^2; \\
    \ewhile \Big( 
    \left[j \leq k \right]^3 , \\
    \left[p \leftarrow 0 \right]^4 ; \\
    \left[
    \eswitch \Big( I, x
    \left(\begin{array}{l}
        [ ] \to q_{j,1},\\
        \cdots\\
    \clabel{1, 2, 3, \cdots, n} \to q_{j,n!}
    \end{array}\right)
    \Big)
    \right]^5\\
    \clabel{I \leftarrow \eupdt(I, (x, p))}^6;\\
    \clabel{j \leftarrow j + 1 }^7
    \Big) ;
\end{array}
%
~~~~ \Rightarrow ~~~
%
SSA \triangleq
\begin{array}{l}
     \left[j \leftarrow 0 \right]^1 ; \\
    \left[I_1 \leftarrow [] \right]^2; \\
    \ewhile \Big( 
    \left[j \leq k \right]^3 , \phi: {I_2} = f({I_1}, {I_3}) \\
    \left[p \leftarrow 0 \right]^4 ; \\
    \left[
    \eswitch \Big( {I_2}, x
    \left(\begin{array}{l}
        [ ] \to q_{j,1},\\
        \cdots\\
    \clabel{1, 2, 3, \cdots, n} \to q_{j,n!}
    \end{array}\right)
    \Big)
    \right]^5\\
    \clabel{{I_3} \leftarrow \eupdt({I_2}, (x, p))}^6;\\
    \clabel{j \leftarrow j + 1 }^7
    \Big) ;
\end{array}
\]
%
%
\subsection{Extending Adaptivity onto Probabilistic Program}
%
\begin{defn}
Dependency Forest.
\\
Given a program $P$, a database $D$, 
dependency forest $F(P, D) \triangleq \{ G_1, G_2, \cdots, G_m \}$,
$G_k = (V_k, E_k)$ is defined as: 
\\
$V_k =\{q^{l,w} |\forall m, w. \config{\emptyset, P, D,[], \emptyset } \rightarrow \config{m, \eskip, D, T, \emptyset} \land q^{l,w} \in T.k  \}$;
\\
$E_k = \left\{(q_i^{l,w},q_j^{l',w'}) 
~ \left \vert ~ \neg \mathsf{IND}(q_i^{l,w},q_j^{l',w'}, P)
\land \mathsf{To}(q_j^{l',w'}, q_i^{l,w}) \right.\right\}$,
\end{defn}

\begin{defn}
Dependency Graph.
\\
Given a program $P$, a database $D$, 
dependency graph $G(P, D) \triangleq (V, E)$ is defined as: 
\\
$V =\bigcup \{V_k | G_k \in F(P, D)  \}$;
\\
$E_k = \left\{(q_i,q_j) 
~ \left \vert ~ (q_i,q_j) \in E_k , G_k \in F(P, D) \right.\right\}$,
\end{defn}
%
%
\begin{defn}
Adaptivity.
\\
Given a program $P$ and for all the database $D$ in a set of $DBS$ of databases, the total dependency graph G is the combination of all the dependent graphs over every single database $G(P, D) = (V, E)$, the adaptivity of the program is defined as $A(P)$, s.t.:
for every pair $(i,j)$ let $p_{(i,j)}$ be the longest path starting from $q_i^{l, w}$ to $q_j^{l',w'}$,
%
$$A(P) = \max\limits_{q_i^{l,w},q_j^{l',w'} \in V }\{l_i ~|~ l_i = |p_(i,j)| \}$$
\end{defn}
%
%
\section{Analysis of Program Adaptivity}
We have the judgment of the form : $\vdash^{c_1, c_2}_{M,V} ~ P : \Phi \implies \Psi$.  Our grade is a combination of a matrix $M$, used to track the dependency of variables appeared in the statement $S$, and a vector $V$ indicating the variables associated with results from queries $q$. The size of the matrix $M$ is $L \times L$, and vector $V$ of size $N$, where $L$ is the total size of variables needed in the program, which is fixed per program. 
The superscript $c_1,c_2$ is a counter specifying the range of "living" or "active" variables in the matrix and vector. $c_1$ is the starting line (and column) in the matrix where the new generated variables  in program $P$ starts to show up. Likewise, $c_2$ states the ending position of active range by $P$. The new generated variables will be treated carefully when we handle recursive statement, in our case, the loop statement. 

We assume the program $P$ is in the static single assignment form. That is to say, $x \leftarrow e_1; x \leftarrow e_2 $ will be rewritten as $ x_1 \leftarrow e_1; x_2 \leftarrow e_2$. And the if condition $ \eif ~ e_b \ethen x \leftarrow e_1 \eelse x \leftarrow e_2  $ will look like $ \eif ~ e_b \ethen x_1 \leftarrow e_1 \eelse x_2 \leftarrow e_2  $. As we have seen, ssa provides unique variables, and these newly generated variables will be recorded in the matrix $M$. Worth to mention, $c_1,c_2$ can be used to track the exact location of newly generated variables. For example, the assignment statement $x \leftarrow e$ or $x \leftarrow q $ holds $c_2 =c_1+1$, telling us the variable $x$ is at the $c_1$th line(column) of the matrix. As we can notice, the loop increases the matrix by $N \times a$ where $N$ is the number of rounds of the loop and $a$ is the size of the variables generated in the loop body $P$.

We will have a global map, which maps the variable name to the position in the vector. We call it $G: VAR \to \mathbb{N}$,

We give an example of $M$ and $V$ of the program $P$. 
$$
P= \begin{array}{c}
\assign {x_1} {q} ;        \\
\assign {x_2} {x_1+1} ;\\
\assign {x_3} {x_2+2} 
\end{array}~~~~~~~~~~~~
M =  \begin{matrix}
 0 & 0 & 0 \\
1 & 0 & 0 \\
1 & 1 & 0 \\
\end{matrix} ~ , V = \begin{matrix}
  1 \\
 0 \\
 0 \\
\end{matrix}
$$


\paragraph{Analysis Rules.}
%
%
\framebox{$ {\Gamma} \vdash^{c_1, c_2}_{M,V} ~ P : \Phi \implies \Psi$}
%
\begin{mathpar}
\inferrule
{M = L(x) * ( R(\expr) + \Gamma )
}
{\Gamma \vdash_{M, V_{\emptyset}}^{(c, c+1)} \assign x \expr :
\Phi \implies \Psi
}
\and
\inferrule
{M = L(x) * ( R(\emptyset) + \Gamma)
\\
V= L(x)
}
{ \Gamma \vdash^{(c, c+1)}_{M, V} \assign x q : 
\Phi \implies \Psi
}
%
\and 
%
\inferrule
{
\Gamma + R(e_b) \vdash^{(c_1, c_2)}_{M_1, V_1} P_1 : 
\Phi \implies \Psi
\and 
\Gamma + R(e_b) \vdash^{(c_2, c_3)}_{M_2, V_2} P_2 :
\Phi \implies \Psi
}
{
\Gamma \vdash^{(c_1, c_3)}_{M_1 \uplus M_2, V_1 \uplus V_2} 
\eif ~ e_b \ethen P_1 \eelse P_2 :\Phi \implies \Psi
}
%
\and 
%
\inferrule
{
\Gamma \vdash^{(c_1, c_2)}_{M_1, V_1} P_1 : \Phi \implies \Psi' \and 
\Gamma \vdash^{(c_2, c_3)}_{M_2, V_2} P_2 : \Psi' \implies \Psi
}
{
\Gamma \vdash^{(c_1, c_3)}_{M_1 \cdot M_2, V_1 \uplus V_2}
P_1 ; P_2
:\Phi \implies \Psi
}
\and 
\inferrule
{
\forall 0 \leq z < N. \Gamma \vdash^{(c, c+a )}_{M, V} f;P : 
\{ \Phi \land \expr_N = z + 1 \} \implies
\{ \Phi \land \expr_N = z \}
}
{\Gamma \vdash^{(c, c+ N*a)}_{M_{c,a}^N(f), V_{c, a}^N} 
\eloop ~ \expr_N ~ (f) ~ \edo ~ P:
\{\Phi \land \expr_N = N\} \implies \{
\Phi \land \expr_N = 0\}
}
%
\and
\inferrule
{ \Gamma + R(\expr) \vdash^{(c_1, c_1+1)}_{M_i, V_i} \assign{ x_i}{q_i} 
\\
i \in \{1, \dots, N\}     }
{\Gamma \vdash^{(c_1, c_1+N)}_{ \sum_{i=0}^{N} M_i, \sum_{i=0}^{N} V_i} \eswitch(\expr, x,(v_i \rightarrow q_i ) }
\end{mathpar}
%
\[
\begin{array}{lll}
V_1 \uplus V_2 & := & \left\{
\begin{array}{ll}
1 & (V_1[i] = 1 \lor V_2[i] = 1) \land i = 1, \cdots, N \land |V_1| = |V_2|\\
0 & o.w.
\end{array}\right.\\
%
M_1 \uplus M_2 & := & \left\{
\begin{array}{ll}
1 & (M_1[i][j] = 1  \lor M_2[i][j] = 1) \land i, j = 1, \cdots, N \land |M_1| = |M_2|\\
0 & (M_1[i][j] = 0  \land M_2[i, j] = 0) \land i, j = 1, \cdots, N \land |M_1| = |M_2|\\
\bot & o.w.
\end{array}\right.\\
%
V_{(c, a)}^N
& := & \left\{
\begin{array}{ll}
V[c+ i*a, c + (i + 1) * a-1] = V[c, c + a-1] & 
 i = 1, \cdots, N - 1 \\
\bot & o.w.
\end{array}\right.\\
%
M_{(c, a)}^N (f)
& := & \left\{
\begin{array}{ll}
M[c+ i*a, c + (i + 1) * a-1][c+ i*a, c + (i + 1) * a-1] & \\
= M[c, c + a-1][c, c+ a-1] & 
 i = 1, \cdots, N - 1 \\
M[c+ i*a, c + (i + 1) * a-1][0, c + (i) * a-1] = 
0 & 
 i = 1, \cdots, N - 1 \\
M[0, c + i * a-1][c+ i*a, c + (i + 1) * a-1] & \\
=  M[0, c + (i - 1) * a-1][c+ (i - 1)*a, c + (i) * a-1] & 
 i = 1, \cdots, N - 1 \\
M[l][k] = 
1& 
\begin{array}{l}
l \in [c + (i - 1)*a, c + (i) *a-1], \\
k \in [c+ i*a, c + (i + 1) * a-1] \\
\land i = 1, \cdots, N 
\land, l= G(x_l) \\
\land k=G(x_k) \land  x_l = f(\cdots, x_k, \cdots)
\end{array}\\
\bot & o.w.
\end{array}\right.\\
%
\end{array}
\]
%
\begin{center}
\begin{tabular}{p{15pt}|p{15pt}|p{15pt}||p{15pt}|p{15pt}
|p{15pt}||p{15pt}|p{15pt}|
p{15pt}|p{15pt}|p{15pt}|p{15pt}|p{15pt}| } 
 1 & $\cdots$ & c-1 & c & $\cdots$ & \tiny{c+a-1} & {\tiny c+a } 
& $\cdots$ & {\tiny{c+2a-1} }
& $\cdots$ & {\tiny c+N*a-1} & {\tiny c+N*a} & $\cdots$ \\
\hline
$\cdots$  & \cellcolor{green} & \cellcolor{green} & \cellcolor{sandstorm} 0 & \cellcolor{sandstorm} 0 & \cellcolor{sandstorm} 0 & \cellcolor{sandstorm} 0 & \cellcolor{sandstorm} 0 & \cellcolor{sandstorm} 0 &  &  &  & \\[10pt]
\hline
c-1 & \cellcolor{green} & \cellcolor{green} & \cellcolor{sandstorm} 0 & \cellcolor{sandstorm} 0 & \cellcolor{sandstorm} 0 & \cellcolor{sandstorm} 0 &\cellcolor{sandstorm} 0 & \cellcolor{sandstorm} 0 &  &  &  &  \\ [10pt]
\hline
c & \cellcolor{periwinkle} & \cellcolor{periwinkle} & \cellcolor{pink} & \cellcolor{pink} &\cellcolor{pink} & \cellcolor{sandstorm} 0 &
\cellcolor{sandstorm} 0 &
\cellcolor{sandstorm} 0 &&&& \\ [10pt]
\hline
$\cdots$ & \cellcolor{periwinkle} & \cellcolor{periwinkle}
&\cellcolor{pink} &\cellcolor{pink}&\cellcolor{pink} &
\cellcolor{sandstorm} 0 & \cellcolor{sandstorm} 0 &
\cellcolor{sandstorm} 0 &&&& \\ [10pt]
\hline
c+a-1 &\cellcolor{periwinkle} &\cellcolor{periwinkle} & \cellcolor{pink} & \cellcolor{pink} & \cellcolor{pink} 
& \cellcolor{sandstorm} 0 & \cellcolor{sandstorm} 0 
& \cellcolor{sandstorm} 0 &&&& \\ [10pt]
\hline \hline
{\scriptsize c+a }  & \cellcolor{periwinkle} & \cellcolor{periwinkle} & \cellcolor{trueblue} &\cellcolor{trueblue}
& \cellcolor{trueblue}& \cellcolor{pink} 
&\cellcolor{pink} & \cellcolor{pink} & &&& \\ [10pt]
\hline
$\cdots$ &\cellcolor{periwinkle} &\cellcolor{periwinkle} & \cellcolor{trueblue}  & \cellcolor{trueblue} f & \cellcolor{trueblue} 
& \cellcolor{pink} & \cellcolor{pink} &\cellcolor{pink} 
&&&& \\ [10pt]
\hline
{\small c+2a-1 } &\cellcolor{periwinkle} & \cellcolor{periwinkle} & \cellcolor{trueblue} 
& \cellcolor{trueblue}  & \cellcolor{trueblue}
& \cellcolor{pink} & \cellcolor{pink} & \cellcolor{pink} 
&&&& \\ [10pt]
\hline
$\cdots$ & &&&&&&&&&&&  \\ [10pt]
\hline
{\tiny c+N*a-1 } & &&&&&&&&&&& \\ [10pt]
\hline
{\tiny c+N*a} & &&&&&&&&&&&\\ [10pt]
\hline
$\cdots$ & &&&&&&&&&&&\\ [10pt]
\hline
\end{tabular}
\end{center}
%
%
%
\begin{defn}
Dependency Graph.
\\
Given a program $P$, a database $D$, dependency graph $G(P,D) = (V, E)$ is defined as: \\
$V =\{q_i^{l,w} |\forall \sigma, \sigma', \mu,\mu' , w, w',t, t'. \lrr{ P }{} (\sigma, \mu , t, w, D)  \triangleq  (\sigma' , \mu' , t', w' , D)  \land q_i^{l,w} \in {(t'-t)}  \}$.
\\
$E = \left\{(q_i^{l,w},q_j^{l',w'}) 
~ \left \vert ~ \neg \mathsf{IND}(q_i^{l,w},q_j^{l',w'}, P)
\land \mathsf{To}(q_j^{l',w'}, q_i^{l,w}) \right.\right\}$.
\end{defn}
%
%
%
\begin{defn}
Two queries $q_i$ and $q_j$ in a program $c$ are independent, $\mathsf{IND}(q^{l_1}_i, q^{l_2}_j, P)$.

$\forall m, D. \Big( 
\lrr{ P }{} (\sigma, \mu , \jl{t, w} , D)  \triangleq  (\sigma' , \mu' , t', w' , D) 
$\\
$\land 
\left((q^{l_1}_i, v_i) \in t' \land (q^{l_2}_j, v_j) \in t'  \implies \forall v \in \textsf{Codom}(q^{l_1}_i). 
\left( \lrr{ P[v/q_i] }{} (\sigma, \mu , \jl{t, w }, D)  \triangleq  (\sigma' , \mu' , t'', w' , D)  \land (q^{l_2}_j, v_j) \in t''
\right)
\right)$
\\
$
\land 
\left( (q^{l_1}_i, v_i) \in t \land (q^{l_2}_j, v_j) \notin t  \implies \forall v \in \textsf{Codom}(q^{l_1}_i). 
\left( \lrr{ P[v/q_i] }{} (\sigma, \mu , \jl{t,w },  D)  \triangleq  (\sigma' , \mu' , t'', w' , D)  \land (q^{l_2}_j, v_j) \notin t''
\right)
\right)
\Big ) $.
\end{defn}
%
%
\begin{defn}
Adaptivity.
\\
Given a program $P$ and for all the database $D$ in a set of $DBS$ of databases, the total dependency graph G is the combination of all the dependent graphs over every single database $G(P, D) = (V, E)$, the adaptivity of the program is defined as $A(P)$, s.t.:
for every pair $(i,j)$ let $p_{(i,j)}$ be the longest path starting from $q_i^{l, w}$ to $q_j^{l',w'}$,
%
$$A(P) = \max\limits_{q_i^{l,w},q_j^{l',w'} \in V }\{l_i ~|~ l_i = |p_(i,j)| \}$$
\end{defn}


\begin{defn}
[Adapt]
Given a program $P$ s.t. $\cdot \vdash_{M,V}^{c_1, c_2} P: \Phi \implies \Psi$, there exists 1 and only one Graph $G(M, V) = (Nodes, Edges, Weights)$ defined as:
\\
$Nodes = \{i | i \in V\}$
\\
$Edges = \{ (i, j) | M[i][j] \geq 1 \}$
\\
$ Weights = \{ 1 | i \in V \land V[i] = 1\}
        \cup \{ -1 | \in V \land V[i] = 0\}$
\\
Adaptivity of the program defined according to the logic is as:
\[
Adapt(M, V) := \max_{k, l \in Nodes}\{i | V[i] = 1 \land i \in p(k, l) \},
\]
where $p(k, l)$ is the longest weighted path in graph $G(M, V)$ starting from $k$ to $l$.
\end{defn}
%
%
\begin{defn}
[Subgraph]
Given two graphs $G_1 = (V_1, E_1)$, $G_2 = (V_2, E_2)$, $G_1 \subseteq G_2$ iff:
\\
1. $V_1 \subseteq V_2$;
\\
2. $\forall (q_i, q_j) \in E_1$, $\exists $ a path in $G_2$ from $q_i$ to $q_j$.
\end{defn}
%
%
\begin{thm}
[Soundness]
Given a program $P$, $\Gamma$, $\mu$, $c_1, c_2$ and $\sigma$ s.t. $
FreeVar(P) \subseteq dom(\sigma) \cup dom(\mu)  
\land \Gamma \subseteq FreeVar(P) 
\land \Gamma \vdash_{M,V}^{c_1,c_2} P: \Phi \implies \Psi$,
 for all database $D$, $(\sigma, \mu) \vDash t$ s.t. 
$\lrr{ P }{} (\sigma, \mu , t ,w , D)  \triangleq  (\sigma' , \mu' , t', w' , D)$,
then
\[
A(P) \leq Adapt(M, V)
\]
\end{thm}
\begin{proof}
 By induction on the judgment $\Gamma \vdash_{M,V}^{c_1, c_2} P: \Phi \implies \Psi$.
\\ 
\begin{itemize}
    \caseL{Case: $\inferrule
{M = L(x) * ( R(\expr) + \Gamma )
}
{\Gamma \vdash_{M, V_{\emptyset}}^{(c, c+1)} \assign x \expr :
\Phi \implies \Psi
}
$}
%
%
Given $\sigma$, $\mu$,
$t$ and $w$,
for arbitrary database $D$, by induction on $\expr$, we have the following semantic.
\[ 
  \lrr{ [\assign x \expr_d]^{l} }{} (\sigma, \mu , t, w, D)   
  \triangleq  (\sigma[x \to \lrr{\expr_d}{} \sigma ], \mu ,  t, w , D) 
\]
%
%
\[
\lrr{ [\assign {x_r} {\expr_r}]^{l}}{} (\sigma, \mu , t, w, D)  
\triangleq 
(\sigma, bind(\mu , m \to unit(m[x_r \to \lrr{\expr_r}{}(\sigma, m)]) ) , t, w , D ) 
\]
%
In both of the 2 cases, no newly added node in the trace $t$. Then we can derive that $A(P) = 0$.
\\
We also know $V_{\emptyset} = \emptyset$, i.e., $Adapt(M, V_{\emptyset}) = 0$.
\\
Since $0 \leq 0$, this case is proved.
%
%
\caseL{
Case:
$\inferrule
{M = L(x) *  (  R(\emptyset) + \Gamma) 
\\
V= L(x)
}
{ \Gamma \vdash^{(c, c+1)}_{M, V} \assign x q : 
\Phi \implies \Psi
}
$}
Given $\sigma$, $\mu$,  $t$ and $w$, for arbitrary database $D$, we have the following semantic.
\[ 
  \lrr{ [\assign x q]^{l} }{} (\sigma, \mu , t, w, D)   
  \triangleq  (\sigma[x \to v ], \mu ,  t ++ [(q, v)]^{(l, w)}, w , D) \qquad : v = q(D)
\]
There is only one newly added node in the trace for all the possible database $D$. 
Follow the definition of $Adapt(M,V)$, we know that the claim holds.\\
\caseL{
Case: ~
$\inferrule
{
 \Gamma + R(e_b) \vdash^{(c_1, c_2)}_{M_1, V_1} P_1 : 
\Phi \implies \Psi
\and 
 \Gamma + R(e_b) \vdash^{(c_2, c_3)}_{M_2, V_2} P_2 :
\Phi \implies \Psi
}
{
\Gamma \vdash^{(c_1, c_3)}_{M_1 \uplus M_2, V_1 \uplus V_2} 
\eif ~ e_b \ethen P_1 \eelse P_2 :\Phi \implies \Psi
}
$}
The semantics depends on the evaluated value of the conditional.
\[
\lrr{ \eif_D ([\bexpr]^{l}, P_1, P_2)  }{} (\sigma, \mu , t, w , D)   \triangleq  \left \{  \begin{array}{l} 
         \lrr{P_1}{} (\sigma, \mu , t, w , D) 
         \qquad : \lrr{\bexpr}{}(\jl{\sigma}) = \etrue \\ 
         \lrr{P_2}{} (\sigma, \mu , t, w ,D) 
         \qquad : \lrr{\bexpr}{}(\jl{\sigma}) = \efalse 
         \end{array} \right . \\  
\]
By induction hypothesis, we have:
\\
$A(P_1) \leq Adapt(M_1, V_1)$
\\
$A(P_2) \leq Adapt(M_2, V_2)$
\\
By definition of $A(P)$ and $Adapt(M, V)$, we have:
\\
$A(P) \leq \max(A(P_1), A(P_2))
\leq \max(Adapt(M_1, V_1), Adapt(M_2, V_2)) 
\leq Adapt(M_1 \uplus M_2, V_1 \uplus V_2)$.
\\
This case is proved.
%
%
\caseL{
Case: $\inferrule
{
\Gamma \vdash^{(c_1, c_2)}_{M_1, V_1} P_1 : \Phi \implies \Psi' \and 
\Gamma \vdash^{(c_2, c_3)}_{M_2, V_2} P_2 : \Psi' \implies \Psi
}
{
\Gamma \vdash^{(c_1, c_3)}_{M_1 \cdot M_2, V_1 \uplus V_2}
P_1 ; P_2
:\Phi \implies \Psi
}
$}
%
%
Given $\sigma$, $\mu$, $t$ and $w$, for arbitrary database $D$, we have the following semantic.
%
\[
\lrr{ P_1 ; P_2 }{} (\sigma, \mu , t, w , D) 
\triangleq 
\lrr{P_2}{} ( \lrr{P_1}{} (\sigma , \mu, t, w, D))
\]
%
Let $\lrr{P_1}{} (\sigma , \mu, t, w, D) = (\sigma_1 , \mu_1, t_1, w_2, D)$, 
$\lrr{P_2}{} ( \lrr{P_1}{} (\sigma , \mu, t, w, D)) = (\sigma_2 , \mu_2, t_2, w_2, D)$.
%
\\
The goal is to show: $A(P_1; P_2) \leq Adapt(M_1 \cdot M_2, V_1 \uplus V_2)$
%
\\
By induction hypothesis, we have:
$A(P_1) \leq Adapt(M_1, V_1)$ and 
$A(P_2) \leq Adapt(M_2, V_2)$.
\\
By definition of $V$ and $t$, we know the newly added queries in $t_2$ compared to the original trace $t$ must be the same as newly added queries marked in
$V_1 \uplus V_2$, 
i.e., the query nodes in the dependency graph must be contained in the Adapt graph generated by $M_1 \cdot M_2, V_1 \uplus V_2 $.
%
\\
On the other hand, any dependency between newly added queries in $t_2 - t$ is tracked by $M_1 \cdot M_2$. It is shown in $3$ cases: (1) dependency between queries nodes in $P_1$ is recorded in $M_1$. (2) dependency between queries nodes in $P_2$ is recorded in $M_2$.(3) dependency between query nodes from $P_1$ and $P_2$ respectively is tracked by $M_2 \times M_1$. To sum up,   
the dependency relation must be contained in $M_1 \cdot M_2$.
\\
Then we can conclude in this case, the longest path in the dependency graph of $P_1; P_2$ must be contained in the
Adapt graph generated by $M_1 \cdot M_2, V_1 \uplus V_2 $, i.e., $A(P_1; P_2) \leq Adapt(M_1 \cdot M_2, V_1 \uplus V_2)$.
\\
This case is proved.
%
%
\caseL{
Case: $\inferrule
{ \Gamma + R(\expr) \vdash^{(c_1, c_1+1)}_{M_i, V_i} \assign{ x_i}{q_i} 
\\
i \in \{1, \dots, N\}     }
{\Gamma \vdash^{(c_1, c_1+N)}_{ \sum_{i=0}^{N} M_i, \sum_{i=0}^{N} V_i} \eswitch(\expr, x,(v_i \rightarrow q_i ) }$}
%
%
Given $\sigma$, $\mu$, $t$ and $w$, for arbitrary database $D$, we have the following semantic.
%
%
\[
\lrr{ {[\eswitch( \expr, x, (v_i \to  q_i))]^{l}} }{} (\sigma, \mu , t, w , D)  
\triangleq 
\lrr{ [\assign x q_1]^{l} }{} ( \sigma, \mu , t, w, D ) 
\qquad : v_1 = \lrr{\expr}{}{(\sigma)}
\]
%
Let $\lrr{[\assign x q_1]^{l}}{} (\sigma , \mu, t, w, D) = (\sigma[x \to v_1'] , \mu, t++[(v_1, q_1)], w, D)$.
\\
We then have:
$\lrr{ {[\eswitch( \expr, x, (v_i \to  q_i))]^{l}} }{} (\sigma, \mu , t, w , D)  
=(\sigma[x \to v_1'] , \mu, t++[(v_1, q_1)], w, D)
$
\\
The goal is to show:
$A( {[\eswitch( \expr, x, (v_i \to  q_i))]^{l}}) \leq Adapt(\sum_{i = 0}^{N}M_i, \sum_{i = 0}^{N}V_i)$
\\
By induction hypothesis, we have:
%
$A([\assign x q_i]^{l})  \leq Adapt(M_i, V_i)$ for any $v_i = q_i$.
%
\\
Then we have $A( {[\eswitch( \expr, x, (v_i \to  q_i))]^{l}}) \leq Adapt(M_i, V_i)$ for any $v_i = q_i$.
\\
%
Since we also have: $ Adapt(M_i, V_i) \leq Adapt(\sum_{i = 0}^{N}M_i, \sum_{i = 0}^{N}V_i)$ for any $v_i = q_i$, this case is proved.
%
%
\caseL{
Case: $\inferrule
{
\Gamma \vdash^{(c, c+a)}_{M, V} f; P : 
\{ \Phi \land \expr_N = z + 1 \} \implies
\{ \Phi \land \expr_N = z \}
}
{\Gamma \vdash^{(c, c+ N*a)}_{M_{c,a}^N(f), V_{c, a}^N} 
\eloop ~ \expr_N ~ (f) ~ \edo ~ P:
\{\Phi \land \expr_N = N\} \implies \{
\Phi \land \expr_N = 0\}
}
$}
%
Given $\sigma$, $\mu$, $t$ and $w$, for arbitrary database $D$, we have the following semantic.
%
%
\[          
\lrr{ {\eloop ~ [\expr_N]^{l} ~ (f) ~ \edo ~ P } }{} (\sigma, \mu , t, w , D)
\triangleq
\left \{  \begin{array}{l} 
\lrr{f;P; \eloop ~ [\expr_N-1]^{l} ~ (f) ~ \edo ~ P }{} (\sigma , \mu, t, w + l, D) 
\qquad : \lrr{\expr_N}{}(\sigma) >0 \\ 
\lrr{\eskip}{} (\sigma , \mu, t, w-l, D) 
\qquad : \lrr{\expr_N}{}(\sigma) = 0 \end{array} 
\right .
\]
%
By induction on $\lrr{\expr_N}{}(\sigma)$, we have sub-cases of $\lrr{\expr_N}{}(\sigma) = 0$ and $\lrr{\expr_N}{}(\sigma) > 0$.
\\
Sub-case of $\lrr{\expr_N}{}(\sigma) = 0$ is obviously true.
\\
Sub-case of $\lrr{\expr_N}{}(\sigma) > 0$:
\\
The goal is to prove 
$A(f;P; \eloop ~ [\expr_N-1]^{l} ~ (f) ~ \edo ~ P) \leq 
Adapt(M_{c,a}^N(f), V_{c, a})$.
\\
By unfolding the semantics of sequence, we have:
\\
$\lrr{f;P; \eloop ~ [\expr_N-1]^{l} ~ (f) ~ \edo ~ P }{} (\sigma , \mu, t, w + l, D) 
= \lrr{\eloop ~ [\expr_N-1]^{l} ~ (f) ~ \edo ~ P }{ }
(\lrr{f;P}{} (\sigma , \mu, t, w + l, D))$.
\\
By induction hypothesis, we have:
$A(f;P) \leq Adapt(M, V)$.
%
\\
Let $\lrr{f;P}{} (\sigma , \mu, t, w + l, D) 
= (\sigma_1 , \mu_1, t_1, w_1, D)$.  
% by inversion on $A(P) \leq Adapt(M, V)$, 
By Lemma 1(Subgraph),
we know 
\\
(a). the newly added queries in $t_1$ must be contained in the queries nodes in $V$,
\\
(b). the dependency between newly added queries must be contained in $M$.
\\
Let $\lrr{\eloop ~ [\expr_N-1]^{l} ~ (f) ~ \edo ~ P }{} (\sigma_1 , \mu_1, t_1, w_1, D) 
= (\sigma_N , \mu_N, t_N, w_{Nl}, D) $, by definition of $V^N_{(c, a)}$ and $M^N_{(c, a)}(f)$ and (a) and (b), we have:
\\
(1). All newly added queries in $t_N$ must be contained in the queries nodes in $V^N_{(c, a)}$, this is proved jointly by (a).
\\
(2). All the dependency between queries internally in $P$ for all the $N$ rounds are contained in $M^N_{(c, a)}(f)$ according to its definition case 1, this is proved jointly by (b).
\\
(3). For all the dependency between queries in different rounds(for example, one query in the second iteration depending on the result of another query in the first iteration), 
they are recorded in $f$,
which is also contained in $M^N_{(c, a)}(f)$ according to its definition case 3.
\\
(4) For all the dependency between queries in P and outside P, since the newly added queries are all comes from P, 
we don't consider the adaptivity outside the scope of program.
\\
According to (1) - (4), we
can conclude in this case, the longest path in the dependency graph of 
$P; \eloop ~ [\expr_N-1]^{l} ~ (f) ~ \edo ~ P$ must be contained in the
Adapt graph generated by 
$M^N_{(c, a)}(f), V^N_{(c, a)} $, i.e., 
$A(P; \eloop ~ [\expr_N-1]^{l} ~ (f) ~ \edo ~ P) \leq Adapt(M^N_{(c, a)}(f), V^N_{(c, a)} )$.
\\
This case is proved.
%
\end{itemize}
\end{proof}
%
%
\begin{lem}
[Subgraph]
Given a program $P$, $\Gamma$, $\mu$, $c_1, c_2$ and $\sigma$ s.t. $
FreeVar(P) \subseteq dom(\sigma) \cup dom(\mu)  
\land \Gamma \subseteq FreeVar(P) 
\land \Gamma \vdash_{M,V}^{c_1,c_2} P: \Phi \implies \Psi$,
 for all database $D$, $(\sigma, \mu) \vDash t$ s.t. 
$\lrr{ P }{} (\sigma, \mu , t ,w , D)  \triangleq  (\sigma' , \mu' , t', w' , D)$,
then
\[
G(P,D) \subseteq G(M, V)
\]
\end{lem}


\end{document}



